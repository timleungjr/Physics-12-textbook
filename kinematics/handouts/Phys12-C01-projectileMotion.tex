%\documentclass{../../oss-handout}
%\usepackage{enumitem}
%\usepackage{tikz}
%\usepackage{siunitx}
%\usepackage{amsmath}
%\usepackage{newtxtext,newtxmath}
%
%\sisetup{
%  detect-all,
%  per-mode=symbol,
%}
%
%\setlength{\parindent}{0pt}
%\setlength{\parskip}{6pt}
%\setlength{\headheight}{26pt}
%
%\newcommand{\pic}[2]{\includegraphics[width=#1\textwidth]{#2}}
%
%\tikzset{
%  >=latex
%}
%\tikzstyle{axes}=[thick,->]
%\tikzstyle{vectors}=[very thick,->]
%\tikzstyle{every node}=[font=\footnotesize]
%
%
%% Set the page style for the document
%\pagestyle{plain}
%
%% Course & handout information
%\renewcommand{\institution}{Meritus Academy}
%\renewcommand{\coursetitle}{Grade 12 Physics}
%\renewcommand{\term}{Updated: Winter/Spring 2023}
%\title{Unit 1 Handout: Projectile Motion}
%\author{Dr.\ Timothy Leung}
%\date{\today}
%
%\begin{document}
%\thispagestyle{title}
%\gentitle
%
%\begin{center}
%  \textbf{Projectile Motion}
%\end{center}

\section{Projectile Motion}

A \textbf{projectile} is an object that is launched through the air\footnote{Or
  more accurately, in a \emph{vaccum}!} along a parabolic trajectory and
accelerates only due to gravity. When solving projectile motion problems, we
usually define the axes in a way that is consistent with cartesian coordinate
system, as shown in Fig.~\ref{fig:projectile}, where:
\begin{itemize}[nosep]
\item $x$-axis is the \emph{horizontal} direction, with the positive direction
  pointing \emph{forward}
\item $y$-axis is the \emph{vertical} direction, with the positive direction
  pointing \emph{up}
\item the origin of the coordinate system is located at the point where the
  projectile is launched
\end{itemize}
%The initial velocity has both horizontal and vertical components.

\begin{figure}[ht]
  \centering
  \begin{tikzpicture}[scale=1.5]
    \draw[axes] (0,0)--(2,0) node[right]{$x$};
    \draw[axes] (0,0)--(0,2) node[above]{$y$};
    \draw[axes] (.5,0) arc (0:52:.5) node[midway,right]{$\theta$};
    \draw[dotted,domain=0:4.5,thick] plot(\x,{1.2*\x-.2*\x*\x});
    \draw[vectors] (0,0)--(.75,.9) node[above]{$\vec v_i$};
    \draw[vectors,red!80!black] (0,0)--(0,.9) node[midway,left]{$v_{y1}$};
    \draw[vectors,blue!80!black] (0,0)--(.75,0) node[midway,below]{$v_x$};
  \end{tikzpicture}
  \caption{Schematic diagram of a projectile motion.}
  \label{fig:projectile}
\end{figure}
%In this case, the $x$ and $y$ components of velocity are defined as:
%\begin{align*}
%  v_x   &=v_i\cos\theta\\
%  v_{y1}&=v_i\sin\theta
%\end{align*}

In the \textbf{horizontal} ($x$) direction, there is no acceleration (i.e.\
$a_x=0$), therefore the horizontal velocity component is constant. Assuming
that the projectile is launched forward with an initial speed $v_i$ at an angle
of $\theta$ above the horizontal, the kinematic equations reduce to a single
equation:
\begin{equation}
  \Delta x=v_x\Delta t=v_i\cos\theta\Delta t
  \label{horizontal}
\end{equation}

where $\Delta x$ is the final horizontal displacement when motion stops,
$v_i$ is the initial speed (magnitude of the initial velocity),
$v_x=v_i\cos\theta$ is therefore the horizontal component of the
initial velocity (which is constant), and $\Delta t$ is the time of motion.

In the \textbf{vertical} ($y$) direction, there is a constant acceleration due
to gravity alone (i.e.\ $a_y=-g=\SI{-9.81}{\metre\per\second\squared}$).
Acceleration is \emph{negative} because the convention is to point the positive
axis upwards. The kinematic equations now take the form:
\begin{align}
  \Delta y &= v_{y1}\Delta t - \frac12 g\Delta t^2
  = v_i\sin\theta\Delta t - \frac12 g\Delta t^2\label{vertical}\\
  v_y &= v_i\sin\theta -g\Delta t\\
  v_y^2 &= v_i^2\sin^2\theta-2g\Delta y\label{height}
\end{align}
where $v_{1y}=v_i\sin\theta$ is the initial vertical component of velocity
(positive if $\theta$ above horizontal, and negative if $\theta$ is below), and
$\Delta y$ is the final vertical displacement (positive if the object lands
higher, then $\Delta y > 0$, if it lands lower, then $\Delta y<0$.)
%and $v_ In the equations above, we usethe acceleration in the $y$
%direction to be $a_y=-g=\SI{-9.81}{\metre\per\second}$. The acceleration is
%\emph{negative} since we usually define the positive direction to be \emph{up}.

Horizontal and vertical motions are independent of each other, but there are
variables that are shared in both directions (Eqs.\ref{horizontal} and
\ref{vertical}), namely:
\begin{itemize}[nosep]
\item Time interval $\Delta t$
\item Launch angle $\theta$
\item Initial speed $v_i$
\end{itemize}
When solving any projectile motion problems, there will be likely \emph{two}
equations with \emph{two} unknowns that you need to solve for. It is almost
certain that $\Delta t$ will be one of them. For more complicated problems,
where an object lands on a ramp, there will be a third relationship between
the horizontal displacement $\Delta x$ with vertical displacement $\Delta y$.



\subsection{Symmetric Trajectory}

A \textbf{symmetric trajectory} is a special case where an object is launched
at an angle of $\theta$ (between $\ang{0}$ and $\ang{90}$) above
horizontal\footnote{This may be obvious, but any angles \emph{below} the 
horizontal will never have a symmetric trajectory.} with an initial speed
$v_i$, and then lands at the same height. Examples may include hitting a golf
ball towards the hole, or shooting a bullet towards a horizontal
target\footnote{Shooting a bullet towards a horizontal target always require an
upward angle because of gravity}. To derive the equations, we use the $x$-axis
for the horizontal direction and $y$-axis for the vertical.

\textbf{Maximum height} $H$: Apply the kinematic equation in the $y$-direction.
Recognizing that at maximum height $H=\Delta y$, vertical velocity is zero
$v_{y2}=0$. Substituting this into Eq.~\ref{height}:
\begin{equation}
  0 = (v_i\sin\theta)^2-2gH
\end{equation}
Solving for $H$, we get the maximum height equation:
\begin{equation}
  \boxed{H=\frac{v_i^2\sin^2\theta}{2g}}
\end{equation}

\textbf{Total time of flight} $T$: We apply the kinematic equation in the $y$
(vertical) direction. When the object lands at the same height, the final
vertical displacement is zero. We can set $\Delta y=0$ and $T=\Delta t$ in
Eq.~\ref{vertical}:

%velocity is the same in magnitude and opposite in direction as the initial
%velocity, i.e.\  $v_{y2}=-v_{y1}=-v_i\sin\theta$:
\begin{equation*}
  0 = v_i\sin\theta T - \frac12 gT^2
\end{equation*}
Solving for $T$, we have:
\begin{equation}
  \boxed{T=\frac{2v_i\sin\theta}g}
  \label{tmax}
\end{equation}

\textbf{Range} $R$: We substitute the expression for $T$ from
Eq.~\ref{tmax} into the $\Delta t$ in Eq.~\ref{horizontal}, the range
$R=\Delta x$ can be calculated for any given launch angle and initial speed:
\begin{equation*}
  R =v_i\cos\theta\left(\frac{2v_i\sin\theta}g\right)
\end{equation*}
Using the trigonometric identity $\sin(2\theta)=2\sin\theta\cos\theta$, we
simplify the equation to:
\begin{equation}
  \boxed{R=\frac{v_i^2\sin(2\theta)}g}
\end{equation}
It is obvious that for any given initial speed $v_i$, the maximum range
$R_\text{max}$ occurs when $\sin(2\theta)=1$ (i.e.\ $\theta=\ang{45}$), with a
range of:
\begin{equation}
  \boxed{R_\text{max}=\frac{v_i^2}g}
\end{equation}
Also, for a known initial speed $v_i$ and range $R$, the launch angle $\theta$
is given by:
\begin{equation}
  \boxed{
    \theta_1=\frac12\sin^{-1}\left(\frac{gR}{v_i^2}\right)
  }
\end{equation}
This angle is labelled $\theta_1$ because it is \emph{not} the only angle that
can reach this range. Recall that for any angle $\ang{0}<\phi<\ang{90}$, there
is also another angle where the $\sin$ are equal:
\begin{displaymath}
  \sin\phi=\sin(\ang{180}-\phi)
\end{displaymath}
Which means that for any $\theta_1$, there is also another angle $\theta_2$
where $2\theta=\ang{180}-2\theta$, or quite simply:
\begin{equation}
  \boxed{\theta_2=\ang{90}-\theta_1}
\end{equation}
%\newpage
%
%\begin{center}
%  {\Large\textbf{Projectile Motion Additional Practice Problems}}
%\end{center}
%When solving these practice problems, make sure your answers have the correct
%number of significant figures. Write out the kinematic equations in both the
%$x$ and $y$ directions, and identify the unknowns. These questions are not part
%of your homework questions (and therefore will not be reviewed in class), but
%will be marked alongside your homework if you complete them. Attach additional
%sheets for calculations if necessary.
%\begin{enumerate}[leftmargin=15pt,topsep=0pt]
%\item A newspaper delivery boy throws a newspaper towards a porch which is
%  \SI{1.5}{\metre} below the height of his hand and \SI{6.5}{\metre} in front
%  of him when he releases the paper. Given that he throws the paper with a
%  velocity of \magdir{\SI{8.5}{\metre\per\second}}{\ang{30} above horizontal},
%  and neglecting any air resistance, find:
%  \begin{enumerate}[nosep]
%  \item the time it takes for the paper to reach the ground
%  \item the acceleration when the paper is only \SI{1.}{\metre} from the ground
%  \item the horizontal range of the paper. Does it make it to the porch?
%  \item the speed of the newspaper when it lands
%  \end{enumerate}
%  \vspace{\stretch{1}}
%  
%\item A golfer hits the golf ball off the tee, giving it an initial velocity of
%  \SI{32.6}{\metre\per\second} at an angle of \ang{65} with the horizontal. The
%  green where the golf ball lands is \SI{6.30}{\metre} higher than the tee, as
%  shown in the illustration. Find
%  \begin{enumerate}[nosep]
%  \item the time interval when the golf ball was in the air
%  \item the distance to the green
%  \end{enumerate}
%  \pic{.35}{../graphics/golfer}
%  \vspace{\stretch{1}}
%  \newpage
%  
%%\item A diver jumps off a \SI{8.75}{\metre} cliff with an initial velocity of
%%  \SI{.764}{\metre\per\second} at an angle of \ang{25} above the horizontal.
%%  \begin{enumerate}[noitemsep]
%%  \item How long will it take for the diver to hit the water below?
%%  \item Determine the range (horizontal distance) of the diver
%%  \end{enumerate}
%%  (Hint: This time, the initial velocity has both vertical and horizontal
%%  components, but the vertical component is below the horizontal.)
%%  \vspace{\stretch{1}}
%  
%%\item A golf ball is hit on a flat fairway at a launch angle of \ang{33} with a
%%  speed of \SI{12.1}{\metre\per\second}.
%%  \begin{enumerate}[label=(\alph*),noitemsep,leftmargin=-25pt]
%%  \item How long does the golf ball stay in the air?
%%  \item How far does the golf ball travel?
%%  \end{enumerate}
%%  \vspace{2in}
%
%\item You are playing tennis with a friend on tennis courts that are surrounded
%  by a \SI{4.8}{\metre} fence. You opponent hits the ball over the fence and
%  you offer to retrieve it. You find the ball at a distance of
%  \SI{12.4}{\metre} on the other side of the fence. You throw the ball at an
%  angle of \ang{55.} with the horizontal, giving it an initial velocity of
%  \SI{12.1}{\metre\per\second}. The ball is \SI{1.05}{\metre} above the ground
%  when you release it. Did the ball go over the fence, hit the fence, or hit
%  the ground before it reached the fence? (Hint: The initial velocity has both
%  vertical and horizontal components. Once you have resolved the velocities,
%  think about which direction you are able to find $\Delta t$.)
%  \vspace{\stretch{1}}
%  
%\item In a boisterous game of ``Monkey in the Middle'', Kathleen and Shannon
%  are tossing a pencil case back and forth over Kevin's head. The girls were
%  \SI{5.}{\metre} apart, and Kevin was \emph{exactly} in the middle. If Kevin
%  was able to reach a height of \SI{3.2}{\metre} with a jump, calculate how far
%  above his reach Kathleen's throw of \SI{8.7}{\metre\per\second} [\ang{65}
%    above horizontal] would be if it left her hand \SI{1.}{\metre} above the
%  ground. If Shannon, jumping, can reach \SI{3.}{\metre}, would she be able to
%  catch the pencil case?
%  \vspace{\stretch{1}}
%  \newpage
%  
%\item A kangaroo is capable of jumping vertically to a height of
%  \SI{2.62}{\metre}.
%  \begin{enumerate}[noitemsep,topsep=0pt]
%  \item Determine the takeoff speed of the kangaroo.
%  \item If the kangaroo jumps instead at an angle of \ang{45} with the same
%    initial speed, how far can it jump on level ground?
%  \end{enumerate}
%  (Hint: Is the ``trajectory'' of the kangeroo symmetric?)
%  \vspace{\stretch{1}}
%  
%\item A projectile is fired into the air from the edge of a $125$-\si{\metre}
%  high cliff at an angle of \ang{30.2} above the horizontal. The projectile
%  hits a target \SI{455}{\metre} away from the base of the cliff. What is the
%  initial speed of the projectile, $v_i$?
%  
%  \pic{.4}{../graphics/projectile_motion_problems_image1}
%  \vspace{\stretch{1}}
%\end{enumerate}
%\newpage
%For problems with the object lands on an incline, both $\Delta x$ and $\Delta y$
%are unknown. In this case, there is a simple trigonometric relationship between
%the two displacements: $\tan\theta$.
%\begin{enumerate}[leftmargin=15pt,topsep=0pt,resume]
%\item A ski jumper launches from a ski jump that is oriented parallel to a
%  hill. The jump has a vertical drop of \SI{50}{\metre} and the coefficient of
%  kinetic friction $\mu$ between the skier and the ramp is %0.05.
%  negligible. The launch point is \SI{5.}{\metre} above the hill and there is a
%  small lip at the bottom of the jump so that the skier launches horizontally.
%  Assume that the skier started from rest at the top of the jump.
%  \begin{enumerate}[noitemsep,topsep=0pt]
%  \item How long in seconds is the skier in flight?
%  \item What is the horizontal distance that the skier travels?
%  \end{enumerate}
%  \pic{.35}{../graphics/1521-small.png}
%  \vspace{\stretch{1}}
%  
%\item A projectile is launched from point $O$ at an angle of \ang{22} with an
%  initial velocity of \SI{15}{\metre\per\second} up an incline plane that makes
%  an angle of \ang{10} with the horizontal. The projectile hits the incline
%  plane at point $M$.
%  \begin{enumerate}[nosep]
%  \item Find the time it takes for the projectile to hit the incline plane.
%  \item Find the distance OM.
%  \end{enumerate}
%  \pic{.35}{../graphics/incline}
%  \vspace{\stretch{1}}
%  \newpage
