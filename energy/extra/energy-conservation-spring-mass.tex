\documentclass[12pt,compress,aspectratio=169]{beamer}
\input{../../mybeamer}

\usetikzlibrary{decorations.pathmorphing,patterns}

\title{Examples of Conservation of Energy: Spring-Mass Systems}
\date{}
\input{../../mycommands}

\begin{document}

\begin{frame}
  \maketitle
\end{frame}



\begin{frame}[t]{Horizontal spring-mass system}
  \begin{center}
    \begin{tikzpicture}[scale=.85]
      \draw[mass] (5,.5) rectangle (6,1.5);
      \draw[thick,decorate,
        decoration={aspect=.45,segment length=6,amplitude=7,coil}] (0,1)--(5,1);
      \fill[pattern=north east lines] (6.5,.5)--(6.5,.3)--(-.2,.3)
      --(-.2,2)--(0,2)--(0,.5)--cycle;
      \draw[very thick] (0,2)--(0,.5)--(6.5,.5);
      \fill[red] (5.5,1) circle (.08);
      \draw[thick,dash dot] (3,0)--+(0,2);
      \draw[vectors] (3,.2)--+(2,0) node[midway,below]{$\vec x$};
      \draw[vectors,red] (5.5,1)--+(0,-1) node[below]{$m\vec g$};
      \draw[vectors,red] (5.5,1)--+(0,1) node[above]{$\vec N$};
      \draw[vectors,red] (5.5,1)--+(-1,0) node[above]{$\vec F_e$};
    \end{tikzpicture}
  \end{center}
  \uncover<2>{
    \vspace{-.15in}{\color{orange}Assuming no friction, drag or other damping
      forces}, a horizontal spring-mass system is an isolated system:
    \begin{itemize}
    \item The system consists of the spring and the mass
    \item Spring force (conservative force!) is an an internal force
    \item Both gravity ($m\vec g$) and normal force ($\vec N$) are
      \emph{external} forces, but neither do any work on the system because
      both forces are perpendicular to motion of the mass
    \end{itemize}
  }
\end{frame}



\begin{frame}[t]{Horizontal spring-mass system}
  \begin{center}
    \begin{tikzpicture}[scale=.85]
      \draw[mass] (5,.5) rectangle (6,1.5);
      \draw[thick,decorate,
        decoration={aspect=.45,segment length=6,amplitude=7,coil}] (0,1)--(5,1);
      \fill[pattern=north east lines] (6.5,.5)--(6.5,.3)--(-.2,.3)
      --(-.2,2)--(0,2)--(0,.5)--cycle;
      \draw[very thick] (0,2)--(0,.5)--(6.5,.5);
      \fill[red] (5.5,1) circle (.08);
      \draw[thick,dash dot] (3,0)--+(0,2);
      \draw[vectors] (3,.2)--+(2,0) node[midway,below]{$\vec x$};
      \draw[vectors,red] (5.5,1)--+(0,-1) node[below]{$m\vec g$};
      \draw[vectors,red] (5.5,1)--+(0,1) node[above]{$\vec N$};
      \draw[vectors,red] (5.5,1)--+(-1,0) node[above]{$\vec F_e$};
    \end{tikzpicture}
  \end{center}
  \begin{itemize}
  \item\vspace{-.15in}The energies of the system is therefore:
    \begin{itemize}
    \item Kinetic energy ($K$) of the mass
    \item Elastic potential energy ($u_e$) stored in the spring
    \end{itemize}
  \item Work by the spring force (conservative force!)
    \begin{itemize}
    \item Positive work converts $U_e$ into $K$
    \item Negative work converts $K$ into $U_g$
    \end{itemize}
    Energies stay inside the system
  \end{itemize}
\end{frame}



\begin{frame}[t]{Horizontal spring-mass system}
  \begin{center}
    \begin{tikzpicture}[scale=.85]
      \draw[mass] (5,.5) rectangle (6,1.5);
      \draw[thick,decorate,
        decoration={aspect=.45,segment length=6,amplitude=7,coil}] (0,1)--(5,1);
      \fill[pattern=north east lines] (6.5,.5)--(6.5,.3)--(-.2,.3)
      --(-.2,2)--(0,2)--(0,.5)--cycle;
      \draw[very thick] (0,2)--(0,.5)--(6.5,.5);
      \fill[red] (5.5,1) circle (.08);
      \draw[thick,dash dot] (3,0)--+(0,2);
      \draw[vectors] (3,.2)--+(2,0) node[midway,below]{$\vec x$};
      \draw[vectors,red] (5.5,1)--+(0,-1) node[below]{$m\vec g$};
      \draw[vectors,red] (5.5,1)--+(0,1) node[above]{$\vec N$};
      \draw[vectors,red] (5.5,1)--+(-1,0) node[above]{$\vec F_e$};
    \end{tikzpicture}
  \end{center}
  The system energy stays constant because there is no external work:

  \eq{-.1in}{
    \boxed{
      K+U_e=\text{constant}
    }
    \quad\text{\normalsize or}\quad
    \boxed{
      K+U_e=K'+U_e'
    }
  }

  When solving problems, we will substitute $K=\frac12mv^2$ and
  $U_e=\frac12kx^2$ into the above equation
\end{frame}



\begin{frame}[t]{Vertical spring-mass system}
  \begin{columns}[T]
    \column{.2\textwidth}
    \centering
    \begin{tikzpicture}
      \draw[mass] (.5,2.5) rectangle +(1,-1);
      \draw[thick,decorate,
        decoration={aspect=.4,segment length=5,amplitude=7,coil}
      ] (1,5)--(1,2.5); 
      \fill[pattern=north east lines] (0,5) rectangle (2,5.2);
      \draw[very thick] (0,5)--(2,5);
      \draw[vectors,red] (1,2)--+(0,-1.5) node[right]{$m\vec g$};
      \draw[vectors,red] (1,2)--+(0,1.5) node[right]{$\vec F_e$};
      \fill[red] (1,2) circle (.06);      
    \end{tikzpicture}

    \column{.8\textwidth}
    
    {\color{orange}Assuming no friction, drag or other damping forces}, the
    vertical spring-mass system can form an isolated system
    \begin{itemize}
    \item To form an isolated system, it must includes:
      \begin{itemize}
      \item The mass
      \item The spring, and
      \item Earth
      \end{itemize}
    \item The isolated system's energy includes:
      \begin{itemize}
      \item The kinetic energy $K$ of the mass
      \item The elastic potential energy $U_e$ stored in the spring
      \item The gravitational potential energy $U_g$, stored between the mass
        and Earth
      \end{itemize}
    \end{itemize}
  \end{columns}
\end{frame}



\begin{frame}[t]{Vertical spring-mass system}
  \begin{columns}[T]
    \column{.2\textwidth}
    \centering
    \begin{tikzpicture}
      \draw[mass] (.5,2.5) rectangle +(1,-1);
      \draw[thick,decorate,
        decoration={aspect=.4,segment length=5,amplitude=7,coil}
      ] (1,5)--(1,2.5); 
      \fill[pattern=north east lines] (0,5) rectangle (2,5.2);
      \draw[very thick] (0,5)--(2,5);
      \draw[vectors,red] (1,2)--+(0,-1.5) node[right]{$m\vec g$};
      \draw[vectors,red] (1,2)--+(0,1.5) node[right]{$\vec F_e$};
      \fill[red] (1,2) circle (.06);
    \end{tikzpicture}

    \column{.8\textwidth}
    Gravity ($m\vec g$) and spring force ($\vec F_e$) are internal forces
    \begin{itemize}
    \item Earth pulls on the mass while mass pulls back on Earth
    \item The spring pulls on the mass; the mass pulls back on the spring
    \end{itemize}
    Work by gravity
    \begin{itemize}
    \item Positive work transform $U_g$ into $K$
    \item Negative work transform $K$ into $U_g$
    \end{itemize}
    Work by spring force
    \begin{itemize}
    \item Positive work transform $U_e$ into $K$
    \item Negative work transform $K$ into $U_e$
    \end{itemize}
  \end{columns}
\end{frame}



\begin{frame}[t]{Vertical spring-mass system}
  \begin{columns}[T]
    \column{.2\textwidth}
    \centering
    \begin{tikzpicture}
      \draw[mass] (.5,2.5) rectangle +(1,-1);
      \draw[thick,decorate,
        decoration={aspect=.4,segment length=5,amplitude=7,coil}
      ] (1,5)--(1,2.5); 
      \fill[pattern=north east lines] (0,5) rectangle (2,5.2);
      \draw[very thick] (0,5)--(2,5);
      \draw[vectors,red] (1,2)--+(0,-1.5) node[right]{$m\vec g$};
      \draw[vectors,red] (1,2)--+(0,1.5) node[right]{$\vec F_e$};
      \fill[red] (1,2) circle (.06);
    \end{tikzpicture}

    \column{.8\textwidth}
    \begin{itemize}
    \item Since gravity and spring are internal forces, the work done by these
      forces stay inside the system
    \item Therefore the system energy does not change:

      \eq{-.1in}{
        \boxed{ K + U_g + U_e= K' + U_g' + U_e'}
      }
    \end{itemize}
    We can replace $K=\frac12mv^2$, $U_g=mgh$ and $U_e=\frac12kx^2$ into the
    above equation when solving problems
  \end{columns}
\end{frame}



\begin{frame}{What about friction?}
  In both examples, we have assumed that {\color{orange}there is no friction,
    drag or damping forces}. If these forces are present, we would generally
  treat them as external forces, and the work done by these forces are taken
  away from the system's energy:

  \eq{-.1in}{
    K + U_g + U_e + {\color{orange}W_\text{ext}}= K' + U_g' + U_e'
  }

  Friction, drag and damping forces \emph{usually} do negative work to the
  system.
\end{frame}



\begin{frame}[t]{Does the system have to be isolated?}
  \begin{columns}[T]
    \column{.2\textwidth}
    \centering
    \begin{tikzpicture}
      \draw[mass] (.5,2) rectangle (1.5,1);
      \draw[thick,decorate,
        decoration={aspect=.4,segment length=5,amplitude=7,coil}] (1,5)--(1,2); 
      \fill[pattern=north east lines] (0,5) rectangle (2,5.2);
      \draw[very thick] (0,5)--(2,5);
      \draw[vectors,red] (1,1.5)--(1,0) node[right]{$m\vec g$};
      \draw[vectors,red] (1,1.5)--(1,3) node[right]{$\vec F_e$};
      \fill[red] (1,1.5) circle (.06);      
    \end{tikzpicture}

    \column{.8\textwidth}
    Even when there is no friction, drag or other damping forces, can this
    vertical spring-mass system form an \emph{open} system?
    \begin{itemize}
    \item Mass + Spring:
      \begin{itemize}
      \item Gravity is an external force; {\color{red}work done by gravity} is
        external work
      \item Gravitational potential energy is no longer part of the system,
        and conservation of energy reduces to

        \eq{-.2in}{
          K + U_e +{\color{red}W_g} = K' + U_e'
        }

      \item\vspace{-.15in}Gravity alternates between doing positive \& negative
        work, depending on motion, so $E_\text{sys}$ never stays constant
      \end{itemize}
    \end{itemize}
  \end{columns}
\end{frame}



\begin{frame}[t]{Does the system have to be isolated?}
  \begin{columns}[T]
    \column{.2\textwidth}
    \centering
    \begin{tikzpicture}
      \draw[mass] (.5,2) rectangle (1.5,1);
      \draw[thick,decorate,
        decoration={aspect=.4,segment length=5,amplitude=7,coil}] (1,5)--(1,2); 
      \fill[pattern=north east lines] (0,5) rectangle (2,5.2);
      \draw[very thick] (0,5)--(2,5);
      \draw[vectors,red] (1,1.5)--(1,0) node[right]{$m\vec g$};
      \draw[vectors,red] (1,1.5)--(1,3) node[right]{$\vec F_e$};
      \fill[red] (1,1.5) circle (.06);      
    \end{tikzpicture}

    \column{.8\textwidth}
    \begin{itemize}
    \item Mass + Earth
      \begin{itemize}
      \item Spring force is external; {\color{magenta}work done by
        the spring} is external work
      \item Elastic potential energy is not part of the system; conservation of
        energy reduces to

        \eq{-.2in}{
          K + U_e +{\color{magenta}W_e} = K' + U_e'
        }

      \item\vspace{-.15in}Spring force alternates between doing positive \&
        negative work, depending on motion, so $E_\text{sys}$ never stays
        constant
      \end{itemize}
    \item The mass alone
      \begin{itemize}
      \item $m\vec g$ and $F_e$ are both external forces
      \item The only energy in this ``system'' is the kinetic energy $K$ of the
        mass
      \item This reduces to the work-energy theorem:

        \eq{-.2in}{
          W_\text{ext}=W_\text{net}=W_g+W_e=\Delta K
        }
      \end{itemize}
    \end{itemize}
  \end{columns}
\end{frame}

\end{document}
