\chapter{Circular Motion}
\label{chapter:circ-motion}

%The \textbf{circular motion} (Fig.~\ref{fig:circ-motion1}) is the simplest
%form of curvilinear motions, an object of mass $m$ moves in a circular path
%about a fixed centre.
%Like we did in Chapters~\ref{chapter:kinematics} and \ref{chapter:dynamics},
%we will begin studying the circular motion, first be defining the coordinate
%system, and then to the kinematic quantities, and then onto dynamics.

\section{Polar Coordinates}
In the majority of two-dimensional motion, 
%discussed in earlier chapters,
it is usually the most convenient to describe an object's position using the
Cartesian coordinate system, i.e.\ using the $x$ and $y$ coordinates as
functions of time. For example, the position, velocity and acceleration vectors
can be described in two-dimensions as:
\begin{align}
  \bm r(t) &=x(t)\hat{\bm x} + y(t)\hat{\bm y}\\
  \bm v(t) &=v_x(t)\hat{\bm x} + v_y(t)\hat{\bm y}\\
  \bm a(t) &=a_x(t)\hat{\bm x} + a_y(t)\hat{\bm y}
\end{align}
where $\hat{\bm x}$ and $\hat{\bm y}$ are unit vectors specifying the
directions of the $x$ and $y$ axes, respectively. Other vectors such as force
$\bm F$, momentum $\bm p$ can also be written in component form this way.

However, expressing vectors in $x$ and $y$ components are not as easy for a
type of motion called \textbf{curvilinear motions}. Some examples of
curvilinear motion are shown in Fig.~\ref{fig:curvilinear-motions}.
\begin{figure}[ht]
  \centering
  \begin{subfigure}{.4\linewidth}
    \centering
    \begin{tikzpicture}[scale=1.2]
      \draw[axes] (-2,0)--(2,0) node[right]{$x$};
      \draw[axes] (0,-2)--(0,2) node[above]{$y$};
      \draw[function] circle (1.5);
      \end{tikzpicture}
    \caption{Circular motion}
    \label{fig:circ-motion1}
  \end{subfigure}  
  \begin{subfigure}{.4\linewidth}
    \centering
    \begin{tikzpicture}[scale=1.2]
      \draw[axes] (-2,0)--(2,0) node[right]{$x$};
      \draw[axes] (0,-2)--(0,2) node[above]{$y$};
      \draw[function,rotate=30] ellipse (1.8 and 1);
    \end{tikzpicture}
    \caption{Elliptical motion}
  \end{subfigure}
  
  \begin{subfigure}{.4\linewidth}
    \centering
    \begin{tikzpicture}[scale=1.2]
      \draw[axes] (-2,0)--(2,0) node[right]{$x$};
      \draw[axes] (0,-2)--(0,2) node[above]{$y$};
      \draw[function,domain={-1.7:1.7}] plot(\x,{.5*(\x*\x)-.3});
    \end{tikzpicture}
    \caption{Parabolic motion}
  \end{subfigure}
  \begin{subfigure}{.4\linewidth}
    \centering
    \begin{tikzpicture}[scale=1.2]
      \draw[axes] (-2,0)--(2,0) node[right]{$x$};
      \draw[axes] (0,-2)--(0,2) node[above]{$y$};
      \draw[domain=45:1000,samples=500,function]
      plot (\x:{1.75*exp(-0.0025*\x)});
    \end{tikzpicture}
    \caption{Inward/Outward spiral}
  \end{subfigure}
  \caption{Examples of curvilinear motion in two dimensions}
  \label{fig:curvilinear-motions}
\end{figure}
In these cases, it is more advantageous to use the 
\textbf{polar coordinate system} instead. The position of an object is
described by:
\begin{equation}
  \bm r(t)=r(t)\hat{\bm r} + \theta(t)\hat{\bm\theta}
\end{equation}
where $r(t)=|\bm r(t)|$ is the straight-line distance from the origin, and
$\theta(t)$ is the standard angle, measured counter clockwise from the $x$
axis.\footnote{I recognize that not everyone has deep background in vectors,
so may be the matrix form may suit you better:
\begin{displaymath}
  \bm r=\left[
    \begin{matrix}
      r(t)\\
      \theta(t)
    \end{matrix}
    \right]
\end{displaymath}
If you are new to vectors, thick of the position as being described by two
\emph{parameters}: $r$ and $\theta$. Both of which will evolve with time.
}

It is clear from basic geometry that Cartesian and polar coordinates are
related by:
\begin{align*}
  x(t)&=r(t)\cos\left(\theta(t)\right)\\
  y(t)&=r(t)\sin\left(\theta(t)\right)
\end{align*}

Like the Cartesian system, the polar coordinate system is also right-handed;
basis vectors $\hat{\bm r}$, called the \textbf{outward radial direction}, and
$\hat{\bm\theta}$, called the \textbf{angular direction}, point in the
directions shown in Fig.~\ref{fig:basis-vecs},
\begin{figure}[ht]
  \centering
  \begin{tikzpicture}
    \draw[axes] (-3,0)--(3,0) node[pos=1.1]{$x$};
    \draw[axes] (0,-3)--(0,3) node[pos=1.1]{$y$};
    \draw[vector] (0,0)--(1,0) node[below]{$\hat{\bm x}$};
    \draw[vector] (0,0)--(0,1) node[left] {$\hat{\bm y}$};
    \draw[gray] circle (2.5);
    \begin{scope}[rotate=38]
      \draw[vector] (0,0)--(2.45,0) node[midway,above]{$\bm r$};
      \draw[vector] (2.5,0)--(3.5,0) node[pos=1.15]{$\hat{\bm r}$};
      \draw[vector] (2.5,0)--(2.5,1) node[pos=1.15]{$\hat{\bm\theta}$};
      \draw[mass] (2.5,0) circle (.1);
    \end{scope}
    \draw[axes] (1.5,0) arc (0:38:1.5) node[pos=.55,right]{$\theta$};
  \end{tikzpicture}
  \caption{Basis vectors for rectilinear and curvilinear motions}
  \label{fig:basis-vecs}
\end{figure}
and rotate as the object moves.

%\section{Cylindrical Coordinates in 3D}
%
%One way to extend the coordinates coordinate system into 3D is the
%\textbf{cylindrical coordinate system}. Note that the discussions for this
%topic focuses on $xy$ plane. Since the $z$-axis is linearly independent of
%the $xy$ plane, motion along that direction is independent.
%
%\begin{figure}[ht]
%  \centering
%  \begin{tikzpicture}[scale=.75]
%    \draw[axes] (0,0)--(-2.5,-2.5) node[below]{$x$};
%    \draw[axes] (0,0)--(5,0) node[right]{$y$};
%    \draw[axes] (0,0)--(0,5) node[above]{$z$};
%    \draw[axes] (-1,-1) arc (-110:-45:2) node[midway,below]{$\theta$};
%    \draw[dashed,fill=green!40,opacity=.4](0,0)--(3,-1.5)
%    node[pos=.6,below left,opacity=1]{$r$}--(3,2.5)
%    node[midway,right,black,opacity=1]{$z$}--(0,4);
%    \fill (3,2.5) circle(.1) node[right]{$\bm r(r,\theta,z)$};
%  \end{tikzpicture}
%\end{figure}



\section{Kinematics of Circular Motion}

The kinematics of circular motion is similar to that of the one-dimensional
kinematics. There is a positive direction, and a negative direction. From the
context of the $xy$-plane, the positive direction is counterclockwise---as it
is standard in mathematics---and the negative direction is clockwise. The
origin of the coordinate system is where the circular path intersects the
$x$-axis. However, unlike in rectilinear 1D motion, objects moving in one
direction along a circular path will return to the origin.



\subsection{Angular Position}

\textbf{Angular position}\index{Position!Angular} $\theta$ is the location of
an object along the
circular path at a distance $r$ from the origin. The angle is measured in
\emph{radians}. In the context of the Cartesian $xy$-plane, $\theta$ measured
from the $+x$-axis. This angle is positive if it is measured counter-clockwise
from the axis, and negative if measured clockwise. In the example shown in
Fig.~\ref{fig:angular-position},
\begin{figure}[ht]
  \centering
  \begin{tikzpicture}[scale=.7]
    \draw[axes] (-3,0)--(3,0) node[right]{$x$};
    \draw[axes] (0,-3)--(0,3) node[above]{$y$};
    \draw[gray] circle (2.5);
    \begin{scope}[rotate=138]
      %\draw[vector] (0,0)--(2.44,0) node[midway,above]{$\bm r_A$};
      \draw[mass] (2.5,0) circle (.08) node[left]{$A$};
    \end{scope}
    \draw[vector] (2.5,0) arc (0:138:2.5) node[midway,above]{$\theta_A>0$};
    \begin{scope}[rotate=-65]
      %\draw[vector] (0,0)--(2.44,0) node[midway,right]{$\bm r_A$};
      \draw[mass] (2.5,0) circle (.08) node[above]{$B$};
    \end{scope}
    \draw[vector] (2.5,0) arc (0:-65:2.5) node[midway,right]{$\theta_B<0$};
  \end{tikzpicture}
  \caption{Angular position of two objects along the $xy$-plane.}
  \label{fig:angular-position}
\end{figure}
the angular position of object A is positive, while the angular position of
object B is negative. If an object is in circular motion, angular position is a
continuous function of time:
\begin{important-equation}
  \theta=\theta(t)
\end{important-equation}
At this time, we are not concerned with what kind of function this is; the exact
expression will depend only on the forces that act on the object. For now, it
is sufficient to say that the object can be at one position at any given time
(hence it is a \emph{function} of time), and that the object cannot teleport
from one position to another (Hence it is a \emph{continuous} function of time).


For the remainder of the chapter, the angular position of an object is not
particularly important, because the $x$-axis can be defined arbitrarily.
%\begin{itemize}
%\item
%\item There are multiple ways
%\end{itemize}
Instead we will focus on the other related kinematic quantities in circular
motion: angular displacement, angular velocity and angular acceleration.



\subsection{Angular Displacement}

If the object is moving along this circular path, then the change in the
angular position is the object's
\textbf{angular displacement}\index{Displacement!Angular}
$\Delta\theta(t)$. It is defined as the difference between an object's current
(or final) angular position ($\theta(t)$) and its initial angular position
($\theta_i$):
%For constant distance $r$ to the origin, the \textbf{angular position}
%$\theta$ determines an object's position as a continuous function of
%time, i.e.:
%%\footnote{The more mathematically rigorous notation is to express the
%%angular position as a vector along the angular direction:
%%\begin{displaymath}
%%  \bm\theta=\theta(t)\hat{\bm\theta}
%%\end{displaymath}
%%The magnitude is $\theta(t)$ and the direction is $\hat{\bm\theta}$.}:
%\begin{equation}
%  \boxed{
%    \theta=\theta(t)
%  }
%  \label{eq:angular-position}
%\end{equation}
%The unit for angular position is a \emph{radian} (rad): If motion is confined
%to the $xy$-plane, then $\theta$ is the standard angle: $\theta$ is positive
%when measured counterclockwise from the $x$-axis, and negative when it is
%measured clockwise.
%
%When an object moves along this circular path, we can calculate the change in
%the angular position (i.e.\ change in the angle). This is the object's
%\textbf{angular displacement} $\Delta\theta$:
\begin{important-equation}
  \Delta\theta(t)=\theta(t)-\theta_i
  \label{eq:angular-displacement}
\end{important-equation}
Like angular position, $\Delta\theta$ is positive if the object moves
counter clockwise, and negative if it moves clockwise, as shown in
Fig.~\ref{fig:angular-displacement}. As angular position $\theta(t)$ evolves in
time, so does angular displacement, therefore $\Delta\theta(t)$ is also a
continuous function of time. At this point, it should be obvious that the
relationship between angular position and angular displacemnet is analogous to
position and displacement in one-dimensional kinematics presented in
Chapter~\ref{chapter:kinematics}.
\begin{figure}[ht]
  \centering
  \begin{tikzpicture}[scale=.7]
    \draw[axes] (-3,0)--(3,0) node[right]{$x$};
    \draw[axes] (0,-3)--(0,3) node[above]{$y$};
    \draw circle (2.5);

    \draw[vector,rotate=35,red] (2.5,0) arc (0:95:2.5)
    node[midway,above right]{$\Delta\theta>0$};
    \draw[fill=lightgray,rotate=35] (2.5,0) circle (.1);
    \draw[thick,red,fill=pink,rotate=130] (2.5,0) circle (.1);

    \draw[vector,blue,rotate=10] (2.5,0) arc (0:-70:2.5)
    node[midway,right]{$\Delta\theta<0$};
    \draw[fill=lightgray,rotate=10] (2.5,0) circle (.1);
    \draw[thick,blue,fill=blue!40,rotate=-60] (2.5,0) circle (.1);
  \end{tikzpicture}
  \caption{Positive and negative angular displacement}
  \label{fig:angular-displacement}
\end{figure}
Angular displacement is also measured in \emph{radians}.



\subsection{Angular Velocity}

Analogous to the relationship between position and velocity in rectilinear
motions, for circular motion, \textbf{angular velocity}\index{Velocity!Angular}
(or \textbf{angular frequency})\index{Frequency!Angular} $\omega(t)$ is the
rate of change of angular position, i.e.\ how quickly angular position changes
with time. We can define the \textbf{average angular velocity} as the angular
displacement over a finite time interval:
%$\omega_\text{avg}(t)$, is the time derivative of the angular position,

\begin{important-equation}
  \omega_\text{avg}(t)=\frac{\Delta\theta(t)}{\Delta t}
  =\frac{\theta(t)-\theta_i}{t-t_i}
\end{important-equation}
Angular velocity is measured in \emph{radian per second}
(\si{\radian\per\second}). It is also a function of time.\footnote{While it
must be a function of time, it is not necessarily a continuous function.}

When the time interval approaches zero ($\Delta t\rightarrow 0$), we have the
\textbf{instantaneous angular velocity} $\omega(t)$ at time $t$. Both average
and instantaneous velocities are continuous functions of time.

Again, if motion is confined to the $xy$-plane, then $\omega$ is positive when
the object moves in the counterclockwise direction, and negative when it moves
clockwise (Fig.~\ref{fig:omega-plus-minus}).
\begin{figure}[ht]
  \centering
  \begin{tikzpicture}[scale=.75]
    \draw[axes] (-3,0)--(3,0) node[right]{$x$};
    \draw[axes] (0,-3)--(0,3) node[above]{$y$};
    \draw[axes] (1,0) arc (0:38:1) node[midway,right]{$\theta$};
    \draw circle (2.5);
    \begin{scope}[rotate=38]
      \draw[thick] (0,0)--(2.44,0) node[midway,above]{$r$};
      \draw[vector] (2.5,.08)--(2.5,1.5) node[above]{$\bm v$};
      \draw[mass] (2.5,0) circle (.1);
    \end{scope}
  \end{tikzpicture}
  \caption{Sign convention for $\omega$ and direction of velocity vector
    $\bm v$ when circular motion is confined to the $xy$-plane}
  \label{fig:omega-plus-minus}
\end{figure}



\subsection{Velocity and Speed}
As the object moves with angular speed $\omega$, the actual velocity vector
$\bm v$ is tangent to the circle. Obviously $\bm v$ is not constant in time, as
its direction is always changing. However, there is a simple mathematical
relationship between the speed of object $v(t)=|\bm v(t)|$ and the angular
speed $|\omega|$:
\begin{equation}
  \boxed{
    v(t)=r|\omega(t)|
  }
\end{equation}
\begin{remark}
  For your information, the velocity vector $\bm v$ along \emph{any} path is
  \emph{always} tangent to the path. This should be obvious when you consider a
  car being driven on a highway.
\end{remark}

%\begin{itemize}
%\item The direction of $\bm v$ is tangent to circle, along
%  $\hat\theta$, and therefore $\perp$ to $\hat r$
%\item If $\omega>0$, the motion is counter-clockwise
%\item If $\omega<0$, the motion is clockwise
%\end{itemize}
  
%    \begin{tikzpicture}[scale=.75]
%      \draw[axes] (-3,0)--(3,0) node[right]{$x$};
%      \draw[axes] (0,-3)--(0,3) node[above]{$y$};
%      \draw[axes] (1,0) arc (0:38:1) node[midway,right]{$\theta$};
%      \draw circle (2.5);
%      \begin{scope}[rotate=38]
%        \draw[vector] (0,0)--(2.44,0) node[midway,above]{$\bm r$};
%        \draw[vector] (2.5,.08)--(2.5,1.5) node[above]{$\bm v$};
%        \draw[mass] (2.5,0) circle (.1);
%      \end{scope}
%    \end{tikzpicture}

%The velocity of the object in circular motion is more properly related to
%the angular velocity using this vector cross product:
%\begin{equation}
%  \bm v=\bm\omega\times\bm r
%\end{equation}
%$\bm\omega$ is out of the page if motion is counterclockwise, and into the page
%if motion is clockwise. Visualizing $\bm\omega$ takes practice, but this vector
%notation is mathematically rigorous and consistent
%  
%
%
%
%
%%\begin{frame}{Relativity Velocity}
%%  If two points $A$ and $B$ are rotating with the same angular velocity with the
%%  same cent, their relative position is given by:
%%
%%  \begin{equation}
%%    \boxed{
%%      \bm V_B=\bm V_A+ \bm\omega\times\bm r_{BA}
%%    }
%%  }
%%
%%  Where $\bm r_{BA}$ is the position of $B$ relative to $A$.


\subsection{Angular Acceleration and Tangential Acceleration}
Analogous to the relation between velocity $\bm v$ and acceleration
$\bm a$, \textbf{angular acceleration}\index{Acceleration!Angular} $\alpha$ is
the rate of change of angular velocity, i.e.\ how quickly angular velocity
changes with time:
\begin{important-equation}
  \alpha_\text{avg}(t)=\frac{\Delta\omega(t)}{\Delta t}
  =\frac{\omega(t)-\omega_i}{t-t_0}
\end{important-equation}
The unit for angular acceleration is \emph{radian per second squared}
\si{\radian\per\second\squared}.
%The sign convention for $\alpha$ is the same
%as for $\theta$ and $\omega$.
Similar to the relationship between velocity and angular velocity,
\textbf{tangential acceleration}\index{Acceleration!Tangential} $a_t$ along the
direction of motion is related to angular acceleration $\alpha$ by the radius
$r$:
\begin{important-equation}
  |\bm a_t(t)|=r|\alpha|
\end{important-equation}
For \emph{uniform} circular motion, $\omega$ is constant, and therefore
$\bm a_t=0$.

%By the fundamental theorem of calculus, we can of course integrate angular
%acceleration to find the angular velocity (or the \emph{change} in angular
%velocity) as a function of time:
%\begin{equation}
%  \boxed{
%    \omega(t)=\int\alpha(t)\dl t+\omega_0
%  }
%  \quad\quad
%  \boxed{
%    \Delta\omega(t)=\int_{t_0}^t\alpha(t)\dl t
%  }
%\end{equation}
%The relationships are the same as in rectilinear motion.
%
%
%
\subsection{Kinematic Equations for Circular Motion}

For \emph{constant} angular acceleration $\alpha$, the kinematic equations are
the same as in rectilinear motion, but with $\theta$ replacing $x$, $\omega$
replacing $v$, and $\alpha$ replacing $a$:
\begin{align}
  \Delta\theta &= \omega_i t + \frac12\alpha t^2\\
  \Delta\theta &= \omega_f t + \frac12\alpha t^2\\
  \Delta\theta &=\frac{\theta_i+\theta_f}2\Delta t \\
  \omega_f &=\omega_i + \alpha t\\
  \omega_f^2& = \omega_0^2+ 2\alpha\Delta\theta
\end{align}
For non-constant $\alpha$, calculus---or at least methods based on
calculus---will be required.




%  \textbf{Example:} An object moves in a circle with angular acceleration
%  \SI{3.0}{\radian\per\second\squared}. The radius is \SI{2.0}{\metre} and it
%  starts from rest. How long does it take for this object to finish a circle?



\subsection{Centripetal Acceleration}% \& Centripetal Force}
Even when there is no angular acceleration, there must also be another
component of acceleration that points towards the centre of the circular path.
This component of acceleration is called the
\textbf{centripetal acceleration}\index{Acceleration!Centripetal}:
\begin{important-equation}
  \bm a_c=-\frac{v^2}r\hat{\bm r}=-\omega^2r\hat{\bm r}
  \label{eq:centripetal-acceleration}
\end{important-equation}
If you are wondering why the term ``centripetal'' sounds vaguely Latin, that's
because it is a poor attempt to combine two Latin words to mean
``\emph{looking for the centre''}. As much as it infuriates Latin scholars, it
does quite appropriately describes the nature of the acceleration, which is
always towards the centre. Note that the negative sign indicates that the
direction of $\bm a_c$ %and $\bm F_c$ are
is radially \emph{inward}, towards the centre of motion, as $\hat{\bm r}$ is
the outward radial direction.
%\begin{figure}[ht]
%  \centering
%  \begin{tikzpicture}
%    \draw[->](-3,0)--(3,0);
%    \draw[->](0,-3)--(0,3);
%    \draw circle (2.5);
%    \begin{scope}[rotate=30]
%      \draw[->,very thick,blue](2.5,0)--(2.5,1.5) node[above]{$\bm v_i$};
%      \draw[->,very thick,red] (0,0)--(2.5,0)node[pos=.6,below]{$\bm r_i$};
%      \fill (2.5,0) circle(.06);
%    \end{scope}
%    \begin{scope}[rotate=90]
%      \draw[->,very thick,blue] (2.5,0)--(2.5,1.5)node[left]{$\bm v_f$};
%      \draw[->,very thick,red] (0,0)--(2.5,0) node[midway,left]{$\bm r_f$};
%TML%%      \fill (2.5,0) circle(.06);
%TML%%    \end{scope}
%TML%%    \draw(0,1)[<->] arc(90:30:1) node[pos=.6,above]{$\Delta\theta$};
%TML%%  \end{tikzpicture}
%TML%%\end{figure}
%TML%called the \textbf{centripetal acceleration} $\bm a_c$.
%TML%%\begin{equation}
%TML%%  \boxed{
%TML%%    \bm a_c=-\frac{v^2}r\hat{\bm r}=-(\omega^2r)\hat{\bm r}
%TML%%  }
%TML%%\end{equation}




\subsubsection{Derivation of Centripetal Acceleration}

The derivation of the centripetal acceleration is considerably more complicated
than for the tangential acceleration. However, it should still be well within a
student's ability to understand this. Consider an object in uniform circular
motion in the counter clockwise direction with radius $r$ and constant speed
$v$ (i.e.\ constant angular speed $\omega=v/r$), as shown in
Fig.~\ref{fig:v-in-uniform-circ-motion}.

At initial time $t_i$, the position and velocity of the object are given by
$\bm r_i=\bm r(t_i)$ and $\bm v_i=\bm v(t_i)$, respectively. Then, at a later
time $t_f=t_i+\Delta t$, the object has moved through an angular displacement
of $\Delta\theta$, and the final position and velocity are now
$\bm r_f=\bm r(t_f)$ and $\bm v_f=\bm v(t_f)$.
\begin{figure}[ht]
  \centering
  \begin{tikzpicture}[scale=.9]
    \draw[axes] (-3,0)--(3,0);
    \draw[axes] (0,-3)--(0,3);
    \draw circle (2.5);
    \begin{scope}[rotate=30]
      \draw[vector,blue] (2.5,0)--(2.5,1.5) node[above]{$\bm v_i$};
      \draw[vector,red] (0,0)--(2.5,0) node[pos=.6,below]{$\bm r_i$};
      \fill (2.5,0) circle (.06);
    \end{scope}
    \begin{scope}[rotate=90]
      \draw[vector,blue] (2.5,0)--(2.5,1.5) node[left]{$\bm v_f$};
      \draw[vector,red] (0,0)--(2.5,0) node[midway,left]{$\bm r_f$};
      \fill (2.5,0) circle (.06);
    \end{scope}
    \draw[<-,thick] (0,1) arc (90:30:1) node[pos=.6,above]{$\Delta\theta$};
  \end{tikzpicture}
  \caption{An object in counter-clockwise uniform circular motion}
  \label{fig:v-in-uniform-circ-motion}
\end{figure}

From the definition of acceleration, we can find the magnitude of the average
centripetal acceleration during this motion:
\begin{equation}
  \bm a_c=\frac{\Delta\bm v}{\Delta t}=\frac{\bm v_f-\bm v_i}{\Delta t}
  \quad\longrightarrow\quad
  |\bm a_c|=\frac{|\Delta\bm v|}{\Delta t}
\end{equation}
Since $|\bm r_i|=|\bm r_f|=r$ (circular motion), and $|\bm v_i|=|\bm v_f|=v$
(constant speed, uniform circular motion), the triangles formed by the
displacement vector $\Delta\bm r$ and the change in velocity $\Delta\bm v$ are
similar isosceles triangles, as shown in Fig.~\ref{fig:sim-triangles}.
\begin{figure}[ht]
  \centering
  \begin{subfigure}{.4\linewidth}
    \centering
    \begin{tikzpicture}[scale=1.5,vector]
      \draw[rotate=-60,red] (0,0)--(0,2) node[midway,below]{$r_i$};
      \draw[red] (0,0)--(0,2) node[midway,left]{$\bm r_f$};
      \draw (2*sin{60},1)--(0,2) node[midway,above]{$\Delta r$};
      \draw[<-,thick] (0,.8) arc (90:30:.8) node[midway,above]{$\Delta\theta$};
    \end{tikzpicture}
    \caption{Displacement}
  \end{subfigure}
  \begin{subfigure}{.4\linewidth}
    \centering
    \begin{tikzpicture}[scale=1.5,vector]
      \draw[rotate=-60,blue] (0,0)--(-2,0) node[midway,right]{$v_i$};
      \draw[blue](0,0)--(-2,0) node[midway,below]{$v_f$};
      \draw(-1,2*sin{60})--(-2,0) node[midway,left]{$\Delta v$};
    \end{tikzpicture}
    \caption{Change in velocity}
  \end{subfigure}
  \caption{Vector diagrams for change in position and velocity are similar
    isosceles triangles}
  \label{fig:sim-triangles}
\end{figure}


%\begin{tikzpicture}
%  \begin{scope}[very thick,->]
%    \draw[rotate=-60,red](0,0)--(0,2) node[midway,below]{$r$};
%    \draw[red](0,0)--(0,2) node[midway,left]{$r$};
%    \draw (2*sin{60},1)--(0,2)node[midway,above]{$|\Delta\bm r|$};
%  \end{scope}
%\end{tikzpicture}
%\begin{tikzpicture}
%  \begin{scope}[very thick,->]
%    \draw[rotate=-60,blue](0,0)--(-2,0) node[midway,right]{$v$};
%    \draw[blue](0,0)--(-2,0) node[midway,below]{$v$};
%    \draw(-1,2*sin{60})--(-2,0) node[midway,left]{$|\Delta\bm v|$};
%  \end{scope}
%\end{tikzpicture}

That the triangles are similar means that the ratio between $r$ and $\Delta r$
is the same as the ratio between $v$ and $\Delta v$. We can now express the
change in velocity in terms of $v$, $r$ and $\Delta r$:
\begin{equation}
  \frac{|\Delta\bm r|}r=\frac{|\Delta\bm v|}v
  \quad\longrightarrow\quad
  |\Delta\bm v|=\frac vr|\Delta\bm r|
\end{equation}
The magnitude of the centripetal acceleration ($|\bm a_c|$):
\begin{equation}
  |\bm a_c|=\frac{|\Delta\bm v|}{\Delta t}
  =\frac vr\left[\frac{|\Delta\bm r|}{\Delta t}\right]=\frac{v^2}r
\end{equation}
The direction of the centripetal acceleration is easy to show using basic
geometry. When $\Delta t\rightarrow 0$, $\Delta\theta\rightarrow 0$.
Since
\begin{equation}
  2\alpha+\Delta\theta=\ang{180}
\end{equation}
when $\Delta\theta\to 0$, $\alpha\to\ang{90}$.
\begin{figure}[ht]
  \centering
  \begin{tikzpicture}
    \begin{scope}[vector]
      \draw[blue] (0,0)--(0,4) node[midway,right]{$\bm v_i$};
      \draw[blue,rotate=30] (0,0)--(0,4) node[midway,left]{$\bm v_f$};
      \draw(0,4)--(-4*sin{30},4*cos{30}) node[midway,above]{$\Delta\bm v$};
    \end{scope}
    \draw[thick] (0,1) arc (90:90+30:1) node[midway,above]{$\Delta\theta$};
    \draw[thick] (0,3.5) arc (270:195:.5)node[midway,below]{$\alpha$};
  \end{tikzpicture}
\end{figure}
The direction of $\Delta\bm v$ is perpendicular to $\bm v$, Since centripetal
acceleration is in the same direction as $\bm v$, $\bm a_c$ points towards the
centre of the circular path (i.e.\ the inwards radial direction
$-\hat{\bm r}$), giving us the equation for centripetal acceleration.
%, which can
%TML%be expressed using the speed $v$ or the angular speed $\omega=v/r$ of the
%TML%object:




\subsection{Acceleration in General Circular Motion}

To summarize, in general circular motion, there are two
orthogonal\footnote{i.e. \emph{parallel}} components of acceleration, as shown
in Fig.~\ref{fig:circular-motion-accelerations}:
\begin{figure}[ht]
  \centering
  \begin{tikzpicture}[scale=4]
    \draw[dashed] (.866,-.5) arc (-30:30:1);
    \draw[vector,magenta] (1,0)--(.5,0) node[midway,below]{$a_c=\omega^2r$};
    \draw[vector,cyan] (1,0)--(1,.3) node[right]{$a_t=r\alpha$};
    \draw[vector] (1,0)--(.5,.3) node[left]{$\bm a$};
    \fill (1,0) circle (.02);
  \end{tikzpicture}
  \caption{The two component of acceleration in general circular motion.}
  \label{fig:circular-motion-accelerations}
\end{figure}
\begin{itemize}
\item The object's centripetal acceleration ($\bm a_c$) is \emph{always}
  present in \emph{all} circular motions. Its magnitude depends on the radius
  of the circular path $r$ and the angular speed $|\omega|$ (or instantaneous
  speed $v=r|\omega|$) of the object. The direction of centripetal acceleration
  is radially inward (i.e.\ in the $-\hat{\bm r}$ direction), towards the
  centre of the circle.
  
\item The object's tangential acceleration ($\bm a_t$) is present only when the
  angular velocity is changing. Its magnitude depends linearly on the radius of
  the circular path $r$, and angular acceleration $\alpha$. The direction of
  tangential acceleration is in the angular direction ($\hat{\bm\theta}$) along
  the direction of motion, tangent to the circular path. Tangential
  acceleration is zero if the object is in uniform motion.
\end{itemize}
The total acceleration $\bm a$ is therefore the vector sum of both the
centripetal and tangential components:
\begin{important-equation}
  \bm a = \bm a_c+\bm a_t
\end{important-equation}




\subsection{Period \& Frequency}
%    \begin{tikzpicture}[scale=.75]
%      \draw[axes] (-3,0)--(3,0) node[right]{$x$};
%      \draw[axes] (0,-3)--(0,3) node[above]{$y$};
%      \draw[axes] (1,0) arc (0:38:1) node[midway,right]{$\theta$};
%      \draw circle (2.5);
%      \begin{scope}[rotate=38]
%        \draw[vector] (0,0)--(2.44,0) node[midway,above]{$\bm r$};
%        \draw[vector] (2.5,.08)--(2.5,1.5) node[above]{$\bm v$};
%        \draw[mass] (2.5,0) circle (.1);
%      \end{scope}
%    \end{tikzpicture}
For constant angular velocity $\omega$ (uniform circular motion), the motion
is strictly periodic. The \textbf{frequency} $f$, which is how many revolutions
made by the object per second is related to the angular velocity by:
\begin{important-equation}
  f=\frac\omega{2\pi}
\end{important-equation}
The SI unit for frequency in \emph{hertz} (\si\hertz). We can also express
the circular motion using its \textbf{period} ($T$), which is time it takes for
a single revolution along the circular path:
\begin{important-equation}
  T=\frac1f=\frac{2\pi}\omega
\end{important-equation}
The SI unit for period is a \emph{second} (\si\second). Note that frequency and
period are reciprocals of each other.

In uniform circular motion ($\alpha=0$), if the period or frequency are known,
%the speed of the object is:
%\begin{equation}
%  v=\omega r = 2\pi rf = \frac{2\pi r}T
%\end{equation}
we can express centripetal acceleration in terms of $T$ or $f$:
\begin{equation}
  \bm a_c=-(\omega^2r)\hat{\bm r}\quad\longrightarrow\quad
  \boxed{
    \bm a_c=-\frac{4\pi^2r}{T^2}\hat{\bm r}=-4\pi^2rf^2\hat{\bm r}
  }
\end{equation}





\section{Dynamics of Circular Motion}

By the second law of motion ($\bm F_\text{net}=m\bm a$), acceleration must be
caused by a net force along that direction. Along the direction of motion,
tangential acceleration is caused by the \textbf{tangential force}
$\bm F_t$:
\begin{important-equation}
  F_t=ma_t=mr\alpha %\hat{\bm\theta}
\end{important-equation}
while the centripetal acceleration---directed towards the centre of motion---is
caused by the \textbf{centripetal force}:
\begin{important-equation}
  F_c
  =ma_c
  =\frac{mv^2}r%\hat{\bm r}
  =m\omega^2r %\right)$\hat{\bm r}
\end{important-equation}
Just as the total acceleration is the vector sum of tangential and centripetal
accelerations, the net force is also a vector sum of the two components of
force:
\begin{important-equation}
  \bm F_\text{net}=\bm F_t + \bm F_c
\end{important-equation}
The forces that generate the centripetal and tangential forces comes from the
free-body diagram, and may include (but not limited to) the common forces
discussed in Chapter 2:
\begin{itemize}
\item Gravity $\bm F_g$
\item Static friction ($\bm f_s$) or kinetic friction ($\bm f_k$)
\item Normal force ($\bm F_n$)
\item Tension ($\bm F_T$)
\item Spring force ($\bm F_e$)
\end{itemize}



\section{Horizontal Circular Motion}

\subsection{Car Turning on Level Road}

Consider a car with mass $m$ is rounding a curve on a level road. If the radius
of curvature of the road is $r$, and the coefficient of friction between the
tires and the road is $\mu$, what is the maximum speed at which the car can
make the curve without skidding off the road?
\begin{center}
  \pic{.55}{circularMotion/graphics/car-circular2}
\end{center}
%  {
%   \footnotesize
%    (Spot the error: Radius $r$ should be from the centre of the circle to the
%    car, not the edge of the road)\par
%  }


%\begin{example}%\begin{frame}{Example: Horizontal Motion}
%    \pic1{puck-on-table}
%    \textbf{Example:} In the figure on the left, a mass $m_1$ is rolling around
%    a frictionless table with radius $R$ with a speed $v$. What is the mass of
%    $m_2$?
%\end{example}


To simplify the analysis, we will ignore forces in the tangential direction.
\begin{remark}
  Including tangential forces can \emph{greatly} increase the complexity of the
  problem. As a car moves along the track, it will experience a drag force, and
  rolling resistance from the deformation of the tires. Both forces act in
  opposition to motion. To maintain speed, a component of the static friction
  must also point in the direction of motion. For front-wheel drive car, static
  friction force points forward, along the direction of motion\ldots while in
  the rear wheels, static friction points backward, opposite to motion!
  Assessing the magnitudes of the actual tangential forces is not a simple
  process, so our analysis will definitely be a simplication.
\end{remark}


\begin{example}
  A boy is twirling a \SI{555}{\gram} ball on a \SI{65.0}{\centi\metre}
  string in a \emph{horizontal} circle. The string will break if the tension
  reaches \SI{15.0}\newton.
  \begin{enumerate}[itemsep=3pt]
%  \item On the dot below, draw and label all the forces (not components) that
%    act on the ball. Forces should be drawn as arrows originating at, and
%    pointing away from, the dot. (Hint: Tension force $F_T$ does \emph{not}
%    point towards the centre of the circular motion.  This is similar to the
%    in-class example of an airplane turning.)
%    %      \begin{center}
%    %        \vspace{1in}
%    %      {\tikz\fill circle (.15);}
%    %      \vspace{1in}
%    %    \end{center}        

  \item Find the centripetal force when tension is at maximum?
    %(Hint: It may
    %be easier to solve the problem algebraically first, before substituting
    %numerical values.
      
  \item What is the maximum speed at which the ball can move without breaking
    the string?
    %(Hint: Once the centripetal force is found, you can find the
    %centripetal acceleration.
  \end{enumerate}

  \vspace{.1in}\textbf{Solutions:} This is a standard problem in circular
  motion, but it's also a tricky one: the centripetal force is \emph{not}
  \SI{15.0}\newton, and the radius of the circular motion is \emph{not}
  \SI{65.0}{\centi\metre} either! Since the ball moves in a horizontal path,
  the vertical component of the net force must be zero. The horizontal
  component is the centripetal force.
  %Thankfully, for most students, once you
  %see this, there is no unseeing it!
\end{example}

\subsection{Banked Curves on Highways and Racetracks}

Traditionally, racetracks, highways and indoor cycling tracks (``velodromes'')
are banked at an angle at turns to help provide centripetal force for the cars
and bikes moving along the track. %Even high speed railroad tracks are banked.

\begin{figure}[ht]
  \centering
  \begin{subfigure}{.47\textwidth}
    \centering
    \pic{.9}{circularMotion/graphics/Autoroute-A20}
    \caption{Highway off ramp}
  \end{subfigure}
  \begin{subfigure}{.47\textwidth}
    \centering
    \pic{.9}{circularMotion/graphics/185644-main-track-turn}
    \caption{Racetrack}
    \label{fig:daytona}
  \end{subfigure}
  
  \begin{subfigure}{.47\textwidth}
    \centering
    \pic{.9}{circularMotion/graphics/67587c526761b}
    \caption{Cycling velodrome}
    \label{fig:milton-velodrome}
  \end{subfigure}
  \begin{subfigure}{.47\textwidth}
    \centering
    \pic{.9}{circularMotion/graphics/Wall-of-Death-2024}
    \caption{``Wall of death''}
    \label{fig:wall-of-death}
  \end{subfigure}
  \caption{Increasingly steep banking angle.}
\end{figure}

Racetracks generally have a high banking angle. For example, the Daytona
International Speedway in Florida (Fig.~\ref{fig:daytona}) has a bank angle of
\ang{31} at Turn 4. Velodromes for indoor track cycling have even steeper bank
angles. The Milton Velodrome in Canada (Fig.~\ref{fig:milton-velodrome}) has a
bank angle of \ang{41}, and a turn radius of 21 m at the base of the
track\footnote{To give some context, the entire velodrome is only
\SI{250}{\metre} long. In international competitions,
cyclists routinely exceed \SI{80}{\kilo\metre\per\hour} around the corners.}
And if you think that these angles are not extreme enough, there is also the
Wall of Death, which has a near \ang{90} angle (Fig.~\ref{fig:wall-of-death}).
Riding a motorcycle on this track is best left to trained professionals.

The banked curve is an example where the centripetal force is provided by the
horizontal components of two forces: normal force and static friction, between
the tires and the road. Here, we will derive the algebraic expression for the
maximum speed a car can travel along a banked curve.

In contrast, for highway ramps, where the turn radius is high and speed is
comparatively slow, the banking angle is generally less than \ang{10}.









To analyze the banked turn problem, Fig.~\ref{fig:banked-turn-forces} shows
the forces acting on a car turns along a banked curve. There are three forces
that acts on it: gravitational force $\bm F_g$, normal force $\bm F_n$, and
static friction $\bm f_s$ between the tires and the road. For simplicity, we
will once again ignore forces in the tangential direction.
\begin{figure}[ht]
  \centering
  \begin{tikzpicture}[scale=1.1]
    \begin{scope}[rotate=20,thick]
      \node[rotate=20] at (0,0) {
        \pic{.2}{circularMotion/graphics/bmw-back-small-shadowy}
      };
      \draw (-3,-1)--+(6,0);
      \draw[rotate around={-20:(-3,-.95)}] (-3,-1)--+(6*cos{20},0)
      node[pos=.25,above=-2]{$\theta$};
    \end{scope}
    \fill[red] circle (.1);
    \begin{scope}[vectors,red]
      \draw (0,0)--(0,-1.7) node[right]{$m\bm g$};
      \draw[rotate=20] (0,0)--(0,1.7*cos{20}) node[above]{$\bm F_n$};
      \draw[rotate=20] (0,0)--(-1.5,0) node[left]{$\bm F_s$};
    \end{scope}
  \end{tikzpicture}
  \caption{Forces on a car when it turns on a banked roadway}
  \label{fig:banked-turn-forces}
\end{figure}

There is no vertical motion, therefore no vertical net force:
\begin{equation}
  \sum F_y=F_n\cos\theta-F_s\sin\theta-mg=0
  \label{eq:vertical-no-net-force}
\end{equation}
But the horizontal net force is the  centripetal force:
\begin{equation}
  \sum F_x=F_n\sin\theta +F_s\cos\theta = \frac{mv^2}r
  \label{eq:horizontal-centripetal-force}
\end{equation}
%Friction force $\bm f$ may be static or kinetic.
  





The car turns at maximum speed $v_\text{max}$ when static friction is at its
maximum, i.e.\ when $F_s=\mu F_n$. Eqs (\ref{eq:vertical-no-net-force}) and
(\ref{eq:horizontal-centripetal-force}) become:
\begin{align*}
  F_n\left(\cos\theta-\mu\sin\theta\right) &=mg\\
  F_n\left(\sin\theta+\mu\cos\theta\right) &=\frac{mv_\text{max}^2}r
\end{align*}
Dividing Eq (\ref{eq:horizontal-centripetal-force}) by
(\ref{eq:vertical-no-net-force}) removes both the normal force and mass terms:

\begin{equation}
  \frac{\sin\theta+\mu\cos\theta}{\cos\theta-\mu\sin\theta}
  =\frac{v_\text{max}^2}{rg}
\end{equation}

The \emph{maximum} velocity $v_\text{max}$ at which a car can turn on a banked
curve is:
\begin{equation}
  \boxed{v_\text{max}=
    \sqrt{rg\frac{\sin\theta+\mu\cos\theta}{\cos\theta-\mu\sin\theta}}
  }
\end{equation}
Note that $v_\text{max}$ does not depend on mass of the car.

For a flat roadway with no banking ($\theta=0$), therefore $\sin\theta=0$ and
$\cos\theta=1$, the equation reduces to:
\begin{equation}
  \boxed{
    v_{\text{max}}=\sqrt{\mu rg}
  }
\end{equation}

When there is no friction (i.e.\ $\mu=0$), the speed that the car can travel
reduces to:
\begin{equation}
  \boxed{
    v=\sqrt{rg\tan\theta}
  }
\end{equation}
In the frictionless case, the speed is no longer a ``maximum''. The car can
only travel in \emph{one} speed without deviating from the track. Travelling to
quickly would result in the car sliding off the outside of the road; while
travelling too slowly would result in the car sliding towards the centre of
the path.

\begin{example}[Exit Ramp]
  A car exits a highway on a ramp that is banked at \ang{5} to the horizontal.
  The exit ramp has a radius of curvature of \SI{65}{\metre}. If the conditions
  are extremely icy and the driver cannot depend on any friction to help make
  the turn, at what speed should the driver travel so that the car will not
  skid off the ramp? What if there is friction?
\end{example}


\section{Vertical Circles}

Circular motion with a horizontal path is straightforward. However, for
vertical motion, it is generally difficult to solve by dynamics and kinematics.
Instead, use conservation of energy may be used to solve for $\bm v$.
Afterwards, we can use the equation for centripetal force to find other forces.
%\textbf{Remember:} If it is impossible to get the required centripetal
%force, then it could not continue the circular motion




\subsection{Simple Pendulum}
A simple pendulum\index{Pendulum!circular motion}, shown in
Fig.~\ref{fig:simple-pendulum-again}, is an example of a circular motion
problem. In Section~\ref{sec:simple-pendulum-energy}, we have already
established how energy is conserved in a simple pendulum system.
\begin{figure}[ht]
  \centering
  \begin{tikzpicture}[scale=.75]
    \fill[pattern=north east lines] (-1,0) rectangle (1,.2);
    \draw[very thick] (-1,0)--(1,0);
    \begin{scope}[rotate=20]
      \draw[thick] (0,0)--(0,-5);
      \shade[ball color=red] (0,-5) circle (.2) node[below right]{$m$};
      \begin{scope}[vector,red]
        \draw (0,-5)--(0,-3.3) node[left]{$\bm F_T$};
        \draw[rotate around={-20:(0,-5)}] (0,-5)--(0,-6.5)
        node[below]{$m\bm g$};
      \end{scope}
    \end{scope}
    \draw[dashed] (0,0)--(0,-5);
    \draw[dashed] (3.54,-3.54) arc (315:225:5);
  \end{tikzpicture}
  \caption{A simple pendulum is a vertical circular motion problem}
  \label{fig:simple-pendulum-again}
\end{figure}

In this system, which is defined as the pendulum bob and Earth, there are two
forces acting on the pendulum: weight $m\bm g$, and tension $\bm F_T$. As the
pendulum swings, $\bm F_T$ is always perpendicular to motion, therefore it does
not do any mechanical work. The only work is done by $m\bm g$ as the pendulum
changes height. Since gravity is an internal force, the total energy of the
system is constant, or:
\begin{equation*}
  \Delta U_g + \Delta K = 0
\end{equation*}
However, only using conservation of energy does not immediately allow us to
find the tension force on the pendulum, nor the acceleration of the pendulum
bob. We must therefore use equations for circular motion to find the forces:
%\item Speed of the pendulum at any height is found using conservation
%  of energy
%  \begin{itemize}
%  \item 
%  \item Work is done by gravity (a conservative force) alone
%  \end{itemize}
%\item Tangential and centripetal accelerations are based on the net force
%  along the angular and radial directions
%\end{itemize}

\textbf{At the top of the swing}, when the pendulum is deflected by an angle
$\theta$, shown in Fig.~\ref{fig:top-swing}, velocity $v$ is zero by
definition.
\begin{figure}[ht]
  \centering
  \begin{tikzpicture}[scale=.75]
    \fill[pattern=north east lines] (-1,0) rectangle (1,0.2);
    \draw[very thick] (-1,0)--(1,0);
    \begin{scope}[rotate=45]
      \draw[thick] (0,0)--(0,-5);
      \shade[ball color=red] (0,-5) circle (.2) node[right=2.5]{$m$};
      \begin{scope}[vector,red]
        \draw[dotted] (0,-5)--(-1.1,-5)node[left]{$mg\sin\theta$};
        \draw[dotted] (0,-5)--(0,-6.1)
        node[right,fill=yellow!20]{$mg\cos\theta$};
        \draw (0,-5)--(0,-3.9) node[left,fill=yellow!20]{$\bm F_T$};
        \draw[rotate around={-45:(0,-5)}] (0,-5)--(0,-6.5)
        node[below]{$m\bm g$};
      \end{scope}
    \end{scope}
    \draw[dashed] (0,0)--(0,-5);
    \draw[dashed] (3.54,-3.54) arc (315:225:5);
    \draw[axes] (0,-2) arc (270:315:2) node[midway,below]{$\theta$};
  \end{tikzpicture}
  \caption{Forces at the top of the swing of a simple pendulum}
  \label{fig:top-swing}
\end{figure}
From Eq.~\ref{eq:centripetal-acceleration}, we know that centripetal
acceleration must also be zero:
\begin{equation}
  a_c=\frac{v^2}r=0
\end{equation}
Therefore the net force along the radial direction $\hat{\bm r}$ is zero. The
tension force $F_T$ can be calculated:
\begin{equation}
  F_T=F_g\cos\theta=mg\cos\theta
\end{equation}

%    \begin{tikzpicture}[scale=.75]
%      \fill[pattern=north east lines] (-1,0) rectangle (1,0.2);
%      \draw[very thick] (-1,0)--(1,0);
%      \begin{scope}[rotate=45]
%        \draw[thick] (0,0)--(0,-5);
%        \shade[ball color=red] (0,-5) circle(.2) node[right=2.5]{$m$};
%        \begin{scope}[vector,red]
%          \draw[dotted] (0,-5)--(-1.1,-5)
%          node[left,fill=cyan!10]{$F_g\sin\theta$};
%          \draw[dotted] (0,-5)--(0,-6.1) node[right]{$F_g\cos\theta$};
%          \draw (0,-5)--(0,-3.9) node[left]{$\bm F_T$};
%          \draw[rotate around={-45:(0,-5)}] (0,-5)--(0,-6.5)
%          node[below]{$\bm F_g$};
%        \end{scope}
%      \end{scope}
%      \draw[dashed] (0,0)--(0,-5);
%      \draw[dashed] (3.54,-3.54) arc (315:225:5);
%      \draw[axes] (0,-2) arc (270:315:2) node[midway,below]{$\theta$};
%    \end{tikzpicture}
In the tangential direction $\hat{\bm\theta}$, there is still a component of
gravity: $F_t=mg\sin\theta$, therefore, there is a
tangential acceleration with a magnitude of:
\begin{equation}
  a_t=\frac{F_t}m=g\sin\theta
\end{equation}
This is the same acceleration as an object sliding down a frictionless ramp at
an angle of $\theta$.

\textbf{At the bottom of the swing}, where $\theta=0$, as shown in
Fig.~\ref{fig:bottom-swing}, the velocity is at its maximum value,
\begin{figure}[ht]
  \centering
  \begin{tikzpicture}[scale=.8]
    \fill[pattern=north east lines] (-1,0) rectangle (1,.2);
    \draw[very thick] (-1,0)--(1,0);
    \draw[thick] (0,0)--(0,-5);
    \shade[ball color=red] (0,-5) circle (0.2) node[below right]{$m$};
    \draw[vector,red] (0,-5)--(0,-2.5) node[right]{$\bm F_T$};
    \draw[vector,red] (0,-5)--(0,-6.5) node[below]{$\bm F_g$};
    \draw[dashed] (3.54,-3.54) arc (315:225:5);
  \end{tikzpicture}
  \caption{A simple pendulum at the bottom of its swing}
  \label{fig:bottom-swing}
\end{figure}
therefore centripetal acceleration is at maximum value because:
\begin{equation}
    a_c=\frac{v^2}r
\end{equation}
At the lowest point, tension is the highest:
\begin{equation}
  F_T=F_g+F_c=m\left(g+\frac{v^2}r\right)
\end{equation}
There is no tangential acceleration because there are forces in the angular
direction:
\begin{equation}
  a_t=0
\end{equation}  




\begin{example}[Yo-Yo]
  \label{example:yoyo}
  You are playing with a yo-yo with a mass $M$; the length of the string is
  $R$. You decide to see how slowly you can swing it in a vertical circle
  while keeping the string fully extended, even when the yo-yo is at the top of
  its swing.
  \begin{enumerate}%[label=(\Alpha)]
  \item Calculate the minimum speed at which you can swing the yo-yo while
    keeping it on a circular path.
  \item If the yo-yo is at its minimum speed at the top of its swing, find the
    tension in the string when the yo-yo is at the side and at the bottom.
  \end{enumerate}
 
  \vspace{.1in}\textbf{Solution:} This is a very typical problem for vertical
  motion. To solve this problem, we first draw free-body diagrams for each of
  the positions in the circle: top, side and bottom. Assuming that there is no
  friction, drag or other damping forces, there are only two forces acting on
  the yo-yo: gravity ($m\bm g$) and tension force ($\bm F_T$).
  \begin{center}
    \begin{tikzpicture}[scale=.75]
      \draw[thick] circle (2);
      \begin{scope}[red]
        \fill (0,2) circle (.1);
        \draw[vector] (-.06,2)--+(0,-1) node[left]{$\bm F_T$};
        \draw[vector] (.06,2)--+(0,-1.5) node[right]{$m\bm g$};
      \end{scope}
      \begin{scope}[violet]
        \fill (-2,0) circle (.1);
        \draw[vector] (-2,0)--+(1.5,0) node[below left]{$\bm F_T$};
        \draw[vector] (-2,0)--+(0,-1.5) node[left]{$m\bm g$};
      \end{scope}
      \begin{scope}[orange]
        \fill (0,-2) circle (.1);
        \draw[vector] (0,-2)--+(0,1.7) node[right]{$\bm F_T$};
        \draw[vector] (0,-2)--+(0,-1.5) node[left]{$m\bm g$}; 
      \end{scope}
    \end{tikzpicture}
  \end{center}
%  \vspace{-.2in}Since the circular motion is not uniform (i.e.\ the speed of
%  the yo-yo is not constant), we have to also use conservation of energy to
%  solve it.
  At the top of the circle, centripetal force is provided by \emph{both} gravity
  and string tension. Since $M$, $g$ and $R$ are all constants, minimum velocity
  $v_\text{min}$ on the right side must occur when $F_T=0$ on the left side. We
  are now left with:
  \begin{displaymath}
    F_c = Ma_c\quad\rightarrow\quad
    F_T+Mg = \frac{Mv^2}R\quad\rightarrow\quad
    Mg = \frac{Mv_\text{min}^2}R
  \end{displaymath}
  Cancelling $M$ and solving for $v_\text{min}$, we have:
  \begin{displaymath}
    v^2=gR\quad\rightarrow\quad\boxed{v_\text{min} = \sqrt{gR}}
  \end{displaymath}

  \begin{center}
    \begin{tikzpicture}
      \fill circle (.05);
      \draw[dashed] circle (2);
      \begin{scope}[violet]
        \fill (-2,0) circle (.1);
        \draw[vector] (-2,0)--+(1.5,0) node[pos=1.15]{$\bm F_T$};
        \draw[vector] (-2,0)--+(0,-1.5) node[pos=1.1]{$m\bm g$};
      \end{scope}
    \end{tikzpicture}
  \end{center}
  At the side of the circle, centripetal force is provided only by tension,
  i.e. $F_c=F_T$:
  \begin{displaymath}
    F_c = Ma_c\quad\longrightarrow\quad
    F_T = \frac{Mv_\text{side}^2}R
  \end{displaymath}
  But at this time, we do not know the speed $v_\text{side}$ of the yo-yo at this
  location yet. (Actually, we don't \emph{need} to find $v_\text{side}$; what we
  actually need is its square, $v_\text{min}^2$.) This is a problem that we can
  easily solve using the law of conservation of energy:
  \begin{displaymath}
    K_\text{top} + U_\text{top} = K_\text{side}\quad\longrightarrow\quad
    \frac12Mv_\text{top}^2 + MgR =\frac 12Mv_\text{side}^2
  \end{displaymath}
  Cancelling $M$ term and solving for $v_\text{side}^2$, we have:
  \begin{displaymath}
    v_\text{side}^2 = v_\text{top}^2+2gR
  \end{displaymath}
  Since $v_\text{top}^2=v_\text{min}^2=gR$ that we have just calculated,
  \begin{displaymath}
    v_\text{side}^2 = gR+2gR=3gR
  \end{displaymath}
  Substituting the expression for $v_\text{side}^2$, we now have the final
  expression for the tension force $F_T$, which is 3 times the weight of the
  yo-yo!
  \begin{displaymath}
    F_T = \frac{Mv_\text{side}^2}R = \frac{M(3gR)}R=\boxed{3Mg}
  \end{displaymath}
  Tension is 3 times the weight of the yo-yo.
  \begin{center}
    \begin{tikzpicture}
      \fill circle (.05);
      \draw[dashed] circle (2);
      \begin{scope}[orange]
        \fill (0,-2) circle (.1);
        \draw[vector] (0,-2)--+(0,1.7) node[right]{$\bm F_T$};
        \draw[vector] (0,-2)--+(0,-1.5) node[left]{$\bm F_g$}; 
      \end{scope}
    \end{tikzpicture}
  \end{center}
  Finally, at the bottom of the circle, tension contributes to centripetal
  force, while gravity contributes \emph{against} it:
  \begin{displaymath}
    F_c = Ma_c\quad\rightarrow\quad
    F_T-F_g = \frac{Mv_\text{bottom}^2}R
  \end{displaymath}
  Again, we need to find the speed $v_\text{bottom}$ of the yo-yo at the bottom
  of the swing. Using conservation of energy again:
  \begin{align*}
    K_\text{top} + U_\text{top} &= K_\text{bottom}\\
    \frac12Mv_\text{top}^2 + MgR &=\frac 12Mv_\text{bottom}^2
  \end{align*}
  Cancelling $M$ term and solving for $v_\text{bottom}^2$, we have:
  \begin{displaymath}
    v_\text{bottom}^2 = v_\text{top}^2+4gR
  \end{displaymath}
  Recognizing that $v_\text{top}^2=gR$ like we did before:
  \begin{displaymath}
    v_\text{bottom}^2 = gR+4gR=5gR
  \end{displaymath}
  Now the final expression for tension:
  \begin{align*}
    F_T &= Mg+\frac{Mv_\text{bottom}^2}R =Mg + \frac{M(5gR)}R\\
    &=\boxed{6Mg}
  \end{align*}
  The tension force $F_T$ is 6 times the weight of the yo-yo at the bottom.
\end{example}

\begin{example}
  A cord is tied to a pail of water, and the pail is swung
  in a vertical circle of \SI{1.}\metre. What must be the minimum velocity of
  the pail be at its highest point so that no water spills out?
%  \begin{enumerate}[(A)]
%  \item\SI{3.1}{\metre\per\second}
%  \item\SI{5.6}{\metre\per\second}
%  \item\SI{20.7}{\metre\per\second}
  %  \item\SI{100.5}{\metre\per\second}
  
  \textbf{Solution:}
\end{example}

\begin{example}
  A roller coaster car is on a track that forms a circular
  loop, of radius $R$, in the vertical plane. If the car is to maintain contact
  with the track at the top of the loop (generally considered to be a good
  thing), what is the minimum speed that the car must have at the bottom of the
  loop? Ignore air resistance and rolling friction.

  \textbf{Solution:}
%%  \begin{enumerate}[(A)]
%%  \item $\sqrt{2gR}$
%%  \item $\sqrt{3gR}$
%%  \item $\sqrt{4gR}$
%%  \item $\sqrt{5gR}$
%%  \end{enumerate}
\end{example}

\begin{example}
  A stone of mass $m$ is attached to a light strong string
  and whirled in a \emph{vertical} circle of radius $r$. At the exact bottom of
  the path, the tension of the string is three times the weight of the stone.
  The stone's speed at that point is:
  
  \textbf{Solution:}
%%  \begin{enumerate}[(A)]
%%  \item $2\sqrt{gR}$
%%  \item $\sqrt{2gR}$
%%  \item $\sqrt{3gR}$
%%  \item $4gR$
\end{example}


