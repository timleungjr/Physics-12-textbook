\section*{Problems}

\begin{enumerate}[itemsep=6pt]
  
%  \item A boy pushes a crate of mass $m$ across a level floor with a
%  constant speed $v$. The coefficient of friction between the crate and the
%  floor is $\mu$. What is the rate at which the boy does work on the crate?
%  \begin{choices}
%    \choice $\mu mg$
%    \choice $mgv$
%    \choice $\mu mgv$
%    \choice $\mu mg/v$
%    \choice $\mu v/mg$
%  \end{choices}
%
%  \item A power company charges its customers 15 {\textcent} per
%  kilowatt-hour. A kilowatt-hour is a unit of
%  \begin{choices}
%    \choice power
%    \choice energy
%    \choice electricity
%    \choice current
%    \choice voltage
%  \end{choices}
%
%  \item A block slides down a smooth quarter-circle ramp of radius $r$,
%  then onto a rough flat surface at the bottom of the ramp. The friction on the
%  horizontal surface causes the block to come to rest in a distance $d$. The
%  work done by the frictional force on the horizontal surface is
%  \begin{center}
%    \begin{tikzpicture}[scale=.5]
%      \draw[very thick] (0,0)--(5,0);
%      \draw[|<->|] (0,-.4)--(5,-.4) node[midway,below]{$d$};
%      \draw[mass] (-4,4) rectangle (-3.5,4.5);
%      \draw[very thick] (-4,4) arc (180:270:4);
%      \draw[very thick] (-4,4)--(-4,4.7);
%      \fill (0,4) circle (.1);
%      \draw[vectors] (0,4)--(-2.82,1.17) node[midway,left]{$r$};
%    \end{tikzpicture}
%  \end{center}
%  \begin{choices}
%    \choice\vspace{-1.3in} $-mgr$
%    \choice $-\sqrt{mgr}$
%    \choice $-2mgr\cos\theta$
%    \choice $-2mgr(1-\cos\theta)$
%    \choice $-2mgr(1-\sin\theta)$
%  \end{choices}
%
%  \item The gravitational potential energy of a box sliding down an incline
%  decreases by \SI{25}{\joule} while the kinetic energy increases by
%  \SI{23}{\joule}. Which of the following best describes this scenario?
%  \begin{choices}
%    \choice The law of conservation of energy is violated in the situation.
%    \choice 2 J of energy is transferred to thermal energy.
%    \choice The momentum of the box system must remain the same.
%    \choice The total energy of the system must remain at 48 J.
%    \choice The kinetic energy of the box must be 25 J.
%  \end{choices}
%  
%  \item Spring 1 is stretched and stores \SI2{\joule} of elastic potential
%  energy. An identical spring, Spring 2, is stretched and stores
%  \SI{18}{\joule} of elastic potential energy. What is the ratio of the
%  displacement of Spring 2 to the displacement of Spring 1?
%  \begin{choices}
%    \choice $9:1$
%    \choice $3:1$
%    \choice $1:1$
%    \choice $1:3$
%    \choice $1:9$
%  \end{choices}
%  \newpage
%  
%  \item Which of the following is a non-conservative force?
%  %(Select two answers.)
%  \begin{choices}
%    \choice Spring force
%    \choice Gravitational force
%    \choice Static friction
%    %\choice Air resistance (drag)
%    \choice Electrostatic force
%  \end{choices}
%
%  \item Which of the following scenarios result in \emph{no} work done on
%  the underlined object? %(Choose two answers.)
%  \begin{choices}
%    %\choice The gravitational force from the Earth acts on the\\
%    %  \underline{Moon} throughout its circular orbit.
%    \choice The gravitational force from the Sun exerts on a small
%    \underline{asteroid} as it slingshots across the solar system
%    \choice A compressed spring launches a \underline{rock} into the air.
%    \choice A child pushes a \underline{box} across a rough horizontal surface.
%    \choice A \underline{car} skids to a stop after panic braking.
%    \choice A football player pushes on a stationary \underline{wall} with all
%    his might.
%  \end{choices}
%
%  \item Which of the following statements are true?%Choose two answers.
%  \begin{choices}
%    \choice The total mechanical energy of an object subjected only to
%    conservative forces is constant.
%    \choice The energy transferred by non-conservative forces is path
%    independent.
%    \choice The energy transferred by conservative forces is path dependent.
%    \choice If an object subjected to non-conservative forces follows a closed
%    path, the net amount of work done on the object is zero.
%    \choice A conservative force is affected by the speed of an object.
%  \end{choices}
  
\item A boy does \SI{465}{\joule} of work pulling an empty wagon along level
  ground with a force of \SI{111}{\newton} at \ang{31.0} below horizontal. A
  frictional force of \SI{155}{\newton} opposes the motion and is actually
  slowing the wagon down from an initial high velocity. Find
  \begin{enumerate}[itemsep=3pt]
  \item the distance that the wagon travels, and
  \item the non-conservative work done by friction force
  \end{enumerate}

%  \item A spring hanging from the ceiling of a house has a spring constant
%  of \SI{15.3}{\newton\per\metre} and a maximum extension of
%  \SI{1.2}{\centi\metre}. What is the largest mass that can be placed on the
%  spring without damaging it?

\item The system shown in the figure below is at rest when the lower string is
  cut. Using the conservation of energy, find the speed of the objects when
  they are at the same height. (Answer to 3 significant figures.)
  \begin{center}
    \pic{.2}{energy/graphics/snip}
\end{center}

\item An empty freight elevator car with a total mass of \SI{100}{\kilo\gram}
  is moving downward at \SI{3.00}{\metre\per\second} when the supporting cable
  snaps. The car falls \SI{4.00}{\metre} onto a huge spring with a spring
  constant of \SI{8.00e3}{\newton\per\metre}. By how much will the spring be
  compressed when the car reaches zero velocity? (Answer to 3 significant
  figures.)
  \begin{center}
    \pic{.27}{energy/graphics/freight-elevator}
  \end{center}

\item In the figure below, the blocks are initial at rest. Choose $U=0$ at this
  initial position. Find the speed of the \SI2{\kilo\gram} mass after it has
  fallen from rest a distance of \SI2\metre, assuming no friction. (Answer to 3
  significant figures.)
  \begin{center}
    \begin{tikzpicture}[scale=.3]
      \draw[mass] (1.5,0) rectangle (5,3) node[midway]{\small$m_1$}
      node[midway,above=10]{\small\SI4{kg}};
      
      \draw[dash dot] (8.5,0) rectangle (12,3);
    
      \draw[mass] (15.5,-3) rectangle (17.5,-5.5) node[midway]{\small$m_2$}
      node[above right] {\small\SI2{kg}};
    
      \draw[dash dot] (15.5,-10) rectangle (17.5,-12.5);
      \draw[very thick,gray!70] (16.5,-5.5)--(16.5,-10);
    
      \draw[ultra thick,brown](5,1.5)--(16,1.5);
      \draw[ultra thick,brown](16.5,1)--(16.5,-3);
    
      \draw[thick,fill=gray] (16,1) circle (.6);
      \draw[thick,fill=gray!50] (16,1) circle (.4);
      \fill (16,1) circle (.15);
      \draw[ultra thick] (15,0)--(16,1);
      
      \draw[fill=yellow!30,thick] (0,0)--(15,0)--(15,-15)--(14,-15)--(14,-1)
      --(0,-1)--cycle;
    
      \draw[|<->|] (18.5,-5.5)--(18.5,-12.5)
      node[midway,fill=white]{\small\SI2\metre};
      \draw[vectors] (16.5,-12.5)--(16.5,-14.5) node[right]{\small$v$};
    \end{tikzpicture}
  \end{center}

\item Using the same diagram from the previous question, suppose that the
  coefficient of kinetic friction between the \SI4{\kilo\gram} block and the
  table is $\mu=0.35$. \textbf{Answer to 3 significant figures.}
  \begin{enumerate}[itemsep=3pt]
  \item Find the work done by friction when the \SI2{\kilo\gram} block
    falls a distance of \SI2\metre.
  \item Find the total mechanical energy (i.e.\ kinetic plus potential)
    $E_\text{tot}$ of the system after the 2 kg mass falls a distance of 2 m,
    assuming that initially, $U=0$.
    \label{partb}
  \item Use your results from part (\ref{partb}) to find the speed of either
    block after the \SI2{\kilo\gram} block has fallen \SI2\metre.
  \end{enumerate}

  \begin{center}
    \begin{tikzpicture}[scale=.7]
      \begin{scope}[rotate=-30]
        \fill[gray] rectangle (-.1,1);
        \fill[gray] (-1.5,0) rectangle (-1.6,1);
        \draw[thick,decorate,
          decoration={aspect=.4,segment length=4,amplitude=6,coil}]
        (-.1,.5)--(-1.5,.5) node[pos=.37,above=4,rotate=-30]{$k$};
        \draw[|<->|,thick](-1.6,1.3)--+(-4,0)
        node[midway,fill=white,rotate=-30]{\SI4\metre};
        \draw[mass] (-5.6,0) rectangle (-6.3,.7)
        node[black,midway,rotate=-30]{$m$};
      \end{scope}
      \draw[fill=pink!50] (0,0)--(-6.06,0)--(-6.06,3.5)--cycle;
      \draw[<->,thick] (-2,0) arc (180:150:2) node[midway,left]{\ang{30}};
    \end{tikzpicture}
  \end{center}

\item A block of mass $m=\SI2{\kilo\gram}$ is released \SI4{\metre} from
  a massless spring with a spring constant of $k=\SI{100}{\newton\per\metre}$
  that is fixed along a frictionless plane inclined at \ang{30}, as shown in
  the figure above.
  \begin{enumerate}[itemsep=3pt]
  \item Find the maximum compression of the spring.
  \item If the plane is not frictionless, and the coefficient of kinetic
    friction between it and the block is $\mu=0.2$, find the maximum
    compression of the spring.
  \end{enumerate}
  \textbf{Answer to 3 significant figures.}
\end{enumerate}

