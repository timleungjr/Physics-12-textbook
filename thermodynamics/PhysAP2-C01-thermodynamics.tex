%\documentclass[12pt,aspectratio=169,dvipsnames]{beamer}
%\input{../mybeamer}
%
%\usetikzlibrary{decorations.pathmorphing,patterns}
%
\chapter{Thermodynamics}
%\subtitle{AP Physics 2}
%\input{../me}
%\input{../mycommands}
%
%
%\begin{document}
%
%
%  \maketitle
%\end{frame}
%
%
%  \centering
%  {\LARGE\textbf{WELCOME TO AP PHYSICS 2}}
%\end{frame}
%
%
%
\section{Temperature}

%{Temperature}
%  \textbf{Temperature} is:
%  \begin{itemize}
%  \item a measure of the ``coldness''/''hotness'' of an object
%  \item more accurately, a measure of the average
%    \textbf{internal energy} of an object. (We will be defining that energy
%    in our discussion)
%  \end{itemize}
%
%
%
%
%{Thermometric Properties}
%  A \textbf{thermometric property} of an object are the physical properties
%  that change with temperature. For example:
%  \begin{itemize}
%  \item Electrical resistance of a copper wire
%  \item The length of an iron rod
%  \item The volume taken up by mercury
%  \item Pressure exerted by a gas
%  \end{itemize}
%
%  \vspace{.2in}An object is in a \textbf{thermal equilibrium} when
%  \begin{itemize}
%  \item thermometric properties are no longer changing
%  \item heat is transferred in and out of the object at the same rate
%  %\item has the same temperature as its surroundings
%  \end{itemize}
%
%
%
%
%{Zeroth Law of Thermodynamics}
%  \textbf{Zeroth law of thermodynamics:} If two thermodynamic systems are each
%  in thermal equilibrium with a third, then they are in thermal equilibrium
%  with each other.
%  \begin{itemize}
%  \item Two objects $A$ and $B$ are in thermal equilibrium if the
%    temperatures of the objects are the same
%  \item Equal amounts of heat (energy) are flowing from $A$ to $B$ and from
%    $B$ to $A$
%  \item Establishes temperature as a measurement of heat, and allows us to
%    define temperature scales
%  \end{itemize}
%  \begin{center}
%    \pic{.22}{thezero}
%  \end{center}
%
%
%
%
%{Celsius Temperature Scale}
%  For water, there are two equilibrium conditions that occur at atmospheric
%  pressure consistently:
%  \begin{itemize}
%  \item\textbf{Ice point} (or \textbf{normal freezing/melting point}) where
%    water and ice exist at this temperature in thermal equilibrium
%  \item\textbf{Vapour point} (or \textbf{normal boiling point}) where water and
%    water vapour exist in thermal equilibrium
%  \end{itemize}
%  In the \textbf{Celsius scale}, \SI0{\celsius} is the ice point, and
%  \SI{100}{\celsius} is the vapour point\footnote{Fun fact: When Celsius
%  first proposed the scale, the ice point was defined as \SI{100}{\celsius}
%  while the vapour point was defined as \SI0\celsius. This means that the
%  number gets bigger as it gets colder. But within a year or so, it was changed
%  to the current scale}
%
%
%
%
\section{Thermal Expansion}

%  When temperature $T$ of an object with length $L$ increases, the object
%  \emph{usually} expands. The \emph{thermal strain} ($\Delta L/L$) is
%  proportional to the change in temperature ($\Delta T=T_f-T_i$) by the
%  \textbf{coefficient of linear thermal expansion} ($\alpha$):
%  
%  \begin{equation}
%    \boxed{\frac{\Delta L}L\approx\alpha\Delta T}
%  }
%  \begin{center}
%    \begin{tabular}{l|c|c}
%      \rowcolor{pink}
%      \textbf{Quantity} & \textbf{Symbol} & \textbf{SI Unit} \\ \hline
%      Length & $L$ & \si\metre \\
%      Change in length & $\Delta L$ & \si\metre \\
%      Change in Temperature & $\Delta T$ & \si{\kelvin} or \si{\celsius}\\
%      Coefficient of linear expansion & $\alpha$ & \si{\per\kelvin} or
%      \si{\per\celsius}
%    \end{tabular}
%  \end{center}
%  $\alpha$ is independent of pressure for solids and liquids, but may vary
%  with $T$.
%
%
%
%
%{Thermal Expansion}
%  There is also a similar expression for \textbf{coefficient of volume
%    expansion}:
%  
%  \begin{equation}
%    \boxed{\frac{\Delta V}V\approx\beta\Delta T}
%  }
%
%  $\beta$ is also independent of pressure for solids and liquids, but may vary
%  with $T$.
%  \begin{center}
%    \begin{tabular}{l|c|c}
%      \rowcolor{pink}
%      \textbf{Quantity} & \textbf{Symbol} & \textbf{SI Unit} \\ \hline
%      Volume      & $V$  & \si{\metre\cubed} \\
%      Change in volume & $\Delta V$ & \si{\metre\cubed} \\
%      Change in Temperature & $\Delta T$ & \si{\kelvin} or \si{\celsius}\\
%      Coefficient of volume expansion & $\beta$ & \si{\per\kelvin}  or
%      \si{\per\celsius}
%    \end{tabular}
%  \end{center}
%  Careful application of calculus shows that for isotropic material (where
%  $\alpha$ is the same in all direction):
%
%  \eq{-.3in}{
%    \beta = 3\alpha
%  }
%
%
%
%
\section{Ideal Gas Law}
%
%{Kinetic Theory of Gases}
%  We begin the study of the thermodynamics of ideal gases using these
%  assumptions:
%  \begin{enumerate}
%  \item The gas consists of a large number of identical molecules that are in
%    high-speed random motion
%  \item Molecular motion and interaction obey Newtonian laws of motion (i.e.\
%    $\vec F_\text{net}=m\vec a$)
%  \item Collision between molecules and with the walls of the container are
%    \emph{elastic} (i.e.\ both momentum and kinetic energy are conserved)
%  \item Molecules are separated, on average, by distances that are large
%    compared to their diameters (i.e.\ the space occupied by the molecules are
%    small compared to the space occupied by the gas as a whole)
%  \item The only forces that the molecules experience are \emph{contact} forces
%    (i.e.\ no gravitational, electrostatic forces on each other except when
%    they collide)
%  \end{enumerate}
%
%
%
%
%{Random Motion}
%  What does ``random motion'' mean in this case?
%  \begin{itemize}
%  \item Velocity vectors of all the gas molecules are different: some molecules
%    move faster, some more slowly
%  \item There is no preferred direction to the motion
%  \item The average velocity of the gas molecules\footnote{This kind of average
%  is called an ``ensemble average'', which will be discussed later} is zero
%  \item The average \emph{speed} of the gas molecules is not zero (depends on
%    temperature)
%  \end{itemize}
%
%
%
%{Volume of a Gas}
%  \textbf{Volume} is the amount of space occupied by the gas, with an SI unit of
%  \textbf{cubic metre} (\si{\metre\cubed}).
%  \begin{center}
%    \begin{tikzpicture}[scale=1.3]
%      \draw[very thick] (-.1,-.1) rectangle (2.1,2.1);
%      \foreach \i in {1,...,90} \fill[red] (rand+1,rand+1) circle (.04);
%    \end{tikzpicture}
%  \end{center}
%  The volume of a gas is \emph{not} the sum of the volumes of the individual
%  atoms or molecules. Most of this occupied volume is empty space.
%
%
%
%
%{Density of a Gas}
%  The \textbf{density} is the ratio of the mass of the gas and the volume
%  occupied by the gas:
%
%  \begin{equation}
%    \boxed{
%      \rho=\frac MV
%    }
%  }
%
%  The SI unit of density is \textbf{kilogram per cubic metre}
%  (\si{\kilo\gram\per\metre\cubed}).
%  \begin{center}
%    \begin{tikzpicture}
%      \draw[very thick] (-.1,-.1) rectangle (2.1,2.1);
%      \foreach \i in {1,...,90} \fill[red] (rand+1,rand+1) circle (.04);
%    \end{tikzpicture}
%  \end{center}
%  For an ideal gas, the density is very low. (The reality is that no gas is
%  truly ideal. However, the lower the density, the better the gas is
%  approximated as an ideal gas. This is why helium is a very close to an
%  ideal gas.)
%  %The volume of a gas is \emph{not} the sum of the volumes of the individual
%  %atoms. Most of this occupied volume is empty space.
%
%
%
%
%{Pressure of a Gas}
%  
%    \column{.7\textwidth}
%    When gas molecules collide, they exert a force on each other and on the
%    container. The \textbf{pressure} is that force $F$ that a gas exerts on the
%    container, divided by the surface area of the container $A$ when the
%    molecules collide with it:
%
%    \begin{equation}
%      \boxed{
%        P=\frac FA
%      }
%    }
%
%    The SI unit of pressure is  \textbf{pascal} (\si\pascal) where
%
%    \eq{-.15in}{
%      \SI1\pascal=\SI1{\newton\per\metre\squared}
%    }
%
%    \vspace{-.2in}At thermal equilibrium, pressure is evenly distributed in the
%    gas
%
%    \column{.3\textwidth}
%    \centering
%    \begin{tikzpicture}[scale=1.3]
%      \draw[very thick] (-.1,-.1) rectangle (2.1,2.1);
%      \foreach \i in {1,...,90} \fill[red] (rand+1,rand+1) circle (.04);
%      \foreach \xy in {.2,.4,...,1.8}{
%        \begin{scope}[axes]
%          \draw (.2,\xy)--(-.1,\xy);
%          \draw (\xy,.2)--(\xy,-.1);
%          \draw (1.8,\xy)--(2.1,\xy);
%          \draw (\xy,1.8)--(\xy,2.1);
%        \end{scope}
%      }
%    \end{tikzpicture}
%  
%
%
%
%
%{Absolute Temperature}
%  William Thomson (a.k.a.\ Lord Kelvin) and James Joule discovered that when a
%  gas is heated or cooled at \emph{constant volume}, there is a \emph{linear}
%  relationship between pressure and temperature:
%  \begin{center}
%    \begin{tikzpicture}
%      \begin{scope}[rotate=25,thick,OrangeRed]
%        \draw[dashed] (0,0)--(1,0);
%        \draw (1,0)--(5,0);
%      \end{scope}
%      \begin{scope}[rotate=32,thick,MidnightBlue]
%        \draw[dashed] (0,0)--(1,0);
%        \draw (1,0)--(5.2,0);
%      \end{scope}
%      \draw[axes] (0,0)--(5,0) node[right]{$T_C$};
%      \draw[axes] (2,0)--(2,2.5) node[left]{$P$};
%      \draw (0,0)--(0,-.15) node[below]{\SI{-273.15}\celsius};
%    \end{tikzpicture}
%  \end{center}
%  Regardless of the type of gas, amount of gas, or the volume of gas, pressure
%  is always zero at \SI{-273.15}\celsius. Since pressure cannot be negative;
%  no temperature exists below that value.
%
%
%
%
%{Absolute Temperature}
%  The \textbf{absolute temperature} $T$ (or \textbf{thermodynamic temperature})
%  is obtained by shifting the ``null point'' (where temperature is zero) from
%  the Celsius temperature $T_C$.
%  
%  \begin{equation}
%    \boxed{T = T_C + 273.15}
%  }
%  \begin{itemize}
%  \item A temperature of \SI0{\kelvin} is called
%    \textbf{absolute zero} (Note that absolute zero is impossible to obtain
%    because of quantum-mechanic effects, discussed in Class \#14)
%  \item This scale is consistent with physical and thermodynamic properties
%    of gases
%  \item Note: the temperature \emph{change} of \SI1{\kelvin} is the same as
%    \SI1\celsius, i.e.:
%
%    \begin{equation}
%      \Delta T=\Delta T_C
%    }
%  \end{itemize}
%
% 
%
%
%%{Relationship Between Temperature and Pressure}
%%  More accurately, the relationship between absolute temperature $T$ and
%%  pressure $P$ of a gas at constant volume is defined using its triple point:
%%
%%  \begin{equation}
%%    T=\frac{T_\text{tp}}{P_\text{tp}}P
%%  }
%%
%%  \textbf{Triple point} is a combination of pressure and temperature where the
%%  solid, liquid and vapour phases of a substance may coexist in thermal
%%  equilibrium.
%%  \begin{itemize}
%%  \item The triple point temperature for water is at
%%    $T_\text{tp}=\SI{273.16}\kelvin$, or $\SI{.01}\celsius$
%%  \item Triple point pressure for water is $P_\text{tp}=\SI{611.2}\pascal$
%%  \end{itemize}
%%
%
%
%
%
%{Ideal Gas Law for Low-Density Gases}
%  Robert Boyle (1627-1691) discovered that, when a gas is allowed to expand or
%  compress at \emph{constant temperature}, the product of pressure $P$ and $V$
%  remain constant (\textbf{Boyle's law}):
%
%  \begin{equation}
%    PV=\text{constant}
% }
%
%  \vspace{-.15in}From the previous discussion on the absolute temperature scale,
%  we also know that at \emph{constant volume}, temperature is proportional to
%  pressure. Combining the two discoveries yields this equation:
%
%  \eq{-.2in}{
%    PV=CT
%  }
%
%  \vspace{-.15in}where ``C'' is some constant to be determined.
%
%
%
%
%{Ideal Gas Law for Low-Density Gases}
%  Thought experiment:
%  \begin{itemize}
%  \item Two identical containers with the same volume $V$, same amount of same
%    kind of gas, at same pressure $P$ and same temperature $T$
%  \item When the containers are combined and the molecules are free to move,
%    $P$ and $T$ remain the same, but $V$ and $N$ both increases by factor of 2
%  \end{itemize}
%  Therefore $C$ must scale with the number of molecules $N$, which
%  modifies the equation to the \textbf{ideal gas law}:

\begin{equation}
  \boxed{PV=Nk_BT}
\end{equation}
The constant $k_B=\SI{1.381e-23}{\joule\per\kelvin}$ is called
\textbf{Boltzmann's constant}. It is found experimentally to have the same
value for any kind or amount of gas.
%
%
%
%
%{Ideal Gas Law}
The ideal gas law is often expressed in a different form in chemistry courses:  
\begin{equation}
  \boxed{PV=nRT}
\end{equation}
%
%  \vspace{-.15in}where:
%  \begin{itemize}
%  \item $n=N/N_A$ is the number of moles of the gas
%  \item $R=k_BN_A=\SI{8.314}{\kilo\gram\per\mol.\kelvin}$ is the
%    \textbf{universal gas constant}, and
%  \item $N_A=\num{6.022e23}$ is \textbf{Avogadro's number}, which is the number
%    of molecules in one mole\footnote{One \emph{mole} of substance is the
%    number of atoms in 12 grams of carbon-12 atoms. This number is also
%    related to the definition of the \emph{unified atomic mass unit}, which
%    will be discussed in the last topic of the course, on nuclear physics.} of
%    the substance
%  \end{itemize}
%  The ideal gas law
%  %is an \textbf{equation of state}, because it
%  relates all the quantities that define the \emph{state} of a gas: pressure
%  $P$, volume $V$ and temperature $T$. Both forms of the equations are used in
%  AP Physics 2.
%  \vspace{.2in}
%
%
%
%
\section{Maxwell-Boltzmann Distribution}

As gas molecules collide elastically with each other inside the container,
%\footnote{Recall the discussion on elastic collision from AP
%  Physics 1, or Physics 12, except for gases, collisions are not in one
%dimension (which are very easy) but in three dimensions.}:
some molecules \emph{gain} kinetic energy (therefore moving faster), while
other molecules \emph{lose} kinetic energy (therefore moving slower).
Individual collisions are random occurrences (determined by probabilities), but
the overall behaviour of the molecules at thermal equilibrium is very
predictable.

At thermal equilibrium, the distribution of particle speeds $v$ for an ideal
gas in a container at temperature $T$ is given by the
\textbf{Maxwell-Boltzmann distribution} function:
\begin{equation}
  f(v,T)=4\pi\left[\frac m{2\pi k_BT}\right]^{\frac32}v^2
  \exp\left[-\frac{mv^2}{2k_BT}\right]
\end{equation}
%  Note: $\exp(x)=e^x$
%  \begin{center}
%    \begin{tabular}{l|c|c}
%      \rowcolor{pink}
%      \textbf{Quantity} & \textbf{Symbol} & \textbf{SI Unit} \\ \hline
%      Maxwell-Boltzmann dist.\ function & $f(v)$ &\si{\second\per\metre}\\
%      Molecular mass       & $m$   & \si{\kilo\gram} \\
%      Particle speed       & $v$   & \si{\metre\per\second} \\
%      Absolute temperature & $T$   & \si\kelvin \\
%      Boltzmann's constant & $k_B$ & \si{\joule\per\kelvin}
%    \end{tabular}
%  \end{center}
%  AP Physics 2 is not concerned with the (very difficult) mathematics, only
%  the overall concept, so you don't need to memorize this equation
%
%
%
%
%{Maxwell-Boltzmann Distribution}
%  \vspace{.15in}
%  
%    \column{.45\textwidth}
%    \pic{1.05}{maxwell-boltzmann}
%    
%    \column{.55\textwidth}
%    \begin{itemize}
%    \item This kind of function is called a \textbf{probability density
%      function}
%    \item The area under the distribution curve is 1 (all probabilities sums to
%      \SI{100}\percent)
%    \item The peak of the distribution shifts to higher speeds as $T$ increases
%    \item The peak, which represents the most-probable speed $v_\text{prob}$,
%      is lower at higher temperatures
%    \end{itemize}
%  
%
%
%
%
%{Maxwell-Boltzmann Distribution}
%  \begin{center}
%    \pic{.5}{speed-distribution}
%  \end{center}
%  There are three particles speeds that are important: the most-probable speed
%  $v_\text{prob}$, the mean (average) speed $\langle v \rangle$ and the
%  root-mean-square speed $v_\text{rms}$.
%
%
%
%
\subsection{Most Probable Speed}
%  
%    \column{.45\textwidth}
%    \pic1{speed-distribution}
%    
%    \column{.55\textwidth}
The \textbf{most probable speed} $v_\text{prob}$ is the peak (i.e.\ mode) of the
distribution function. This is the speed you are most likely to find a molecule
travelling at:
%
%    \eq{-.05in}{
%      v_\text{prob}=\sqrt{\frac{2k_BT}m}=\sqrt{\frac{2RT}M}
%    }
  
where $m$ is the molecular mass of the gas, $R$ is the universal gas constant,
and $M$ is the molar mass\footnote{That is, the mass of 1 mole of the
substance.}. $v_\text{prob}$ is obtained through using basic
calculus\footnote{Albeit not necessarily simple. It involves finding the
derivative of $f(v)$ and solving for $v$ where $f'(v)=0$.}. We can see that
the most probable speed is proportional to the square root of absolute
temperature:
\begin{equation}  
  v_\text{prob}\propto\sqrt T
\end{equation}

\subsection{Mean/Average Speed}
%  
%    \column{.45\textwidth}
%    \pic1{speed-distribution}
%    
%    \column{.55\textwidth}
The \textbf{mean speed} $\langle v\rangle$ (or \textbf{average speed}) is
the arithmetic average of $N$ particles:
\begin{align*}
  \langle v\rangle &=\frac{v_1+v_2+\cdots+v_N}N\\
  &=\sqrt{\frac{8k_BT}{\pi m}}=\sqrt{\frac{8RT}{\pi M}}
\end{align*}
The above value for $\langle v\rangle$ is obtained using basic integral
calculus\footnote{The Maxwell-Boltzmann distribution function is continuous
in $v$, so this is done by computing the integral
$\langle v\rangle=\int_0^\infty vf(v)dv$. The Maxwell-Boltzmann distribution
acts as a weighting function}. Like $v_\text{prob}$, $\langle v\rangle$ is
also proportional to square root of absolute temperature:
\begin{equation}
  \langle v\rangle\propto\sqrt T
\end{equation}
%
%
%
%
%{Side Note: Averaging}
%  In AP Physics 1,
%  %(or any physics course that you have previously taken),
%  you have encountered two types of averaging:
%  \begin{itemize}
%  \item\textbf{Time averaging} (when the physical quantity of a
%    \emph{single} object evolves over time). For example, average velocity,
%    average acceleration (kinematics) and average force (momentum and impulse)
%    are:
%
%      \eq{-.13in}{
%        \vec v_\text{avg}=\frac{\Delta\vec x}{\Delta t}\quad\quad
%        \vec a_\text{avg}=\frac{\Delta\vec v}{\Delta t}\quad\quad
%        \vec F_\text{avg}=\frac{\Delta\vec p}{\Delta t}
%      }
%
%    \item\textbf{Spatial averaging} (when the physical quantity of a
%      \emph{single} object evolves over some distance). For example, the work
%      done by the average force:
%      
%      \eq{-.13in}{
%        F_\text{avg}=\frac W{\Delta x}
%      }
%  \end{itemize}
%  Notice that there are \emph{two} definitions of average force. It is
%  important to understand what kind of averaging we are doing
%
%
%
%
%
%{Side Note: Ensemble Average}
%  In this current example, the averaging is the arithmetic average amongst
%  \emph{many} particles. This is known as an \textbf{ensemble average}. If
%  there are $N$ particles, each with a physical property $X$, the ensemble
%  average of $X$ (the notation is $\langle x\rangle$) is defined as:
%
%  \begin{equation}
%    \langle x\rangle = \frac{X_1+X_2+\cdots+X_N}N=\frac{\sum_{i=1}^N X_i}N
%  }
%
%  For a large number of gas molecules, travelling with random motion, we
%  care about the average speed ($\langle v\rangle$) and the average of the
%  speed squared ($\langle v^2\rangle$)
%
%
%
%
%
\subsection{Root-Mean-Square (RMS) Speed}
%  %\begin{center}
%  %  \pic{.4}{speed-distribution}
%  %\end{center}
When studying ideal gases, we are \emph{most} interested in the average
\emph{kinetic energy} $\langle K\rangle$ of the gas molecules, which scales
with $v^2$. The average value of $v^2$ (i.e.\ $\langle v^2\rangle$) is therefore
\begin{align*}
  \langle K\rangle &=\dfrac{K_1+K_2+\cdots+K_N}N\\
  &=\dfrac{\frac12mv_1^2+\frac12mv_2^2+\cdots+\frac12mv_N^2}N
  =\frac12m
  \underbracket[1.3pt]{
    \left[\frac{v_1^2+v_2^2+\cdots+v_N^2}N\right]
  }_{\langle v^2\rangle}
\end{align*}
Using ``basic'' integral calculus\footnote{This is accomplished by the integral
$\langle v^2\rangle=\int_0^\infty v^2f(v)dv$. Again, the Maxwell-Boltzmann
distribution function acts as a weighting function}, we can show that
\begin{equation}
    \langle v^2\rangle=\frac{3k_BT}m
\end{equation}
The speed that gives the average kinetic energy is called the
\textbf{root-mean-square speed} $v_\text{rms}$, which is the square root of
$\langle v^2\rangle$:
\begin{equation}
  \underbracket[1.3pt]{v_\text{rms}=\sqrt{\langle v^2\rangle}}_{
    \text{i.e.\ }v_\text{rms}^2=\langle v^2\rangle}
  =\sqrt{\frac{3k_BT}m}=\sqrt{\frac{3RT}M}
\end{equation}
``Root mean square'' literally means ``the square root of the mean of the
square''. Like $v_\text{prob}$ and $\langle v\rangle$, $v_\text{rms}$ is
also proportional to the square root of absolute temperature:
\begin{equation}
  v_\text{rms}\propto\sqrt T
\end{equation}
%
%
%
%
%{Average Kinetic Energy of an Ideal Gas}
%  From the expression of $v_\text{rms}$, and relating to the average kinetic
%  energy, it is easy to show the average kinetic energy of $N$ ideal-gas
%  molecules is:
%  
%  \begin{equation}
%    \langle K \rangle=\frac12mv_\text{rms}^2=
%    \frac12m\left[\sqrt{\frac{3k_BT}m}\right]^2
%    \quad\rightarrow\quad
%    \boxed{\langle K\rangle=\frac32k_BT}
%  }

The \emph{total} kinetic energy of an ideal gas therefore scales linearly
with temperature:
\begin{equation}
  K_\text{total}=N\langle K\rangle=\frac32Nk_BT=\frac32nRT
\end{equation}




\section{First Law of Thermodynamics}

In the \textbf{law of conservation of energy}, which is arguably the most
important concept in fundamental physics,


\begin{definition}{Law of Conservation of Enegy}
  The change in the total energy of a system $E_\text{sys}$ is equal to the net
  external mechanical work $W_\text{ext}$ done from outside the system:
  \begin{equation}
    \boxed{
      \Delta E_\text{sys}=W_\text{ext}
    }
  \end{equation}
\end{definition}
The \emph{consequence} of the law of conservation of energy is that,
\emph{in an isolated system, energy cannot be created or destroyed; it can only
change in form}.

In general, the total energy is the sum of all the kinetic and potential
energies of the objects in the system in the macroscopic scale, collectively
called the \textbf{mechanical energy} of the system:
\begin{equation}
  E_\text{sys} = \sum K + \sum U
\end{equation}
Therefore, the \emph{change} in total energy is the total change in all the
forms of energy:
\begin{equation}
  \Delta E_\text{sys} = \sum\Delta K + \sum\Delta U
\end{equation}
But have we accounted for \emph{all} the energies in the system?

Let us consider a container of gas with a total mass $m$, oscillating from the
ceiling, at a height $h$ about the surface of Earth, as shown in
Fig.~\ref{fig:big-system}.
\begin{figure}[ht]
  \centering
  \begin{tikzpicture}[scale=.75]
  \begin{scope}[thick]
    \draw (-1,5)--(3,5);
    \draw (-1,-4)--(3,-4) node[right]{\scriptsize$h=0$};
    \draw[dashed] (-1,2.8)--+(4,0) node[right]{\scriptsize Unstretched};
    \draw[decoration={aspect=.4, segment length=5, amplitude=6, coil},
      decorate] (1,5)--(1,2.1) node[midway,right=4]{$k$};
    \draw[vectors] (.5,2.8)--(.5,2.1) node[midway,left]{$\vec x$};
    \draw[fill=gray!10] (-.1,-.1) rectangle (2.1,2.1);
    \draw[vectors] (2.5,2)--(2.5,0) node[midway,right]{$\vec v$};
    \draw[|<-] (-.5,1)--(-.5,-4) node[midway,left]{$h$};
  \end{scope}
  \foreach \i in {1,...,90} \fill[red] (rand+1,rand+1) circle (.05);
\end{tikzpicture}

  \caption{An isolated system with a vertically oscillating container of gas}
  \label{fig:big-system}
\end{figure}
At the macroscopic level, the system has a total mechanical energy which
includes:
\begin{itemize}
\item A \emph{bulk} kinetic energy of
  \begin{equation*}
    K=\dfrac12 mv^2
  \end{equation*}
  while the container is moving.
  
\item A gravitational potential energy of
  \begin{equation*}
    U_g=mgh
  \end{equation*}

\item An elastic potential energy of
  \begin{equation*}
    U_e=\frac12kx^2
  \end{equation*}
\end{itemize}
%
%{Kinetic, Potential and Internal Energies}
%  
%    \column{.3\textwidth}
%    \begin{tikzpicture}[scale=.75]
  \begin{scope}[thick]
    \draw (-1,5)--(3,5);
    \draw (-1,-4)--(3,-4) node[right]{\scriptsize$h=0$};
    \draw[dashed] (-1,2.8)--+(4,0) node[right]{\scriptsize Unstretched};
    \draw[decoration={aspect=.4, segment length=5, amplitude=6, coil},
      decorate] (1,5)--(1,2.1) node[midway,right=4]{$k$};
    \draw[vectors] (.5,2.8)--(.5,2.1) node[midway,left]{$\vec x$};
    \draw[fill=gray!10] (-.1,-.1) rectangle (2.1,2.1);
    \draw[vectors] (2.5,2)--(2.5,0) node[midway,right]{$\vec v$};
    \draw[|<-] (-.5,1)--(-.5,-4) node[midway,left]{$h$};
  \end{scope}
  \foreach \i in {1,...,90} \fill[red] (rand+1,rand+1) circle (.05);
\end{tikzpicture}

%
%    \column{.7\textwidth}
But the random motion of the air molecules also contribute to additional energy
of the system, called the \textbf{internal energy}, or \textbf{thermal energy},
$E_\text{int}$ which is the sum of all their kinetic and potential energies at
the \emph{microscopic} level:
\begin{equation}
  \boxed{
    E_\text{int}=K_\text{micro} + U_\text{micro}
  }
\end{equation}
%Internal energy is a function of the molecules' absolute temperature.
%  
%
%
%
%
%{First Law of Thermodynamics}
%  When applying the law of conservation of energy to a thermodynamic system:
%  \begin{itemize}
%  \item No change in bulk kinetic energy and potential energies (i.e.\ how
%    fast the container moves, or how high it is above Earth, or how much the
%    spring stretches, does not affect the temperature of the gas inside)
%  \item The only energy change to the system is its internal energy
%  \end{itemize}
%  This reduces the equation to:
%  
%  \begin{equation}
%    \xcancel{\sum\Delta K} + \xcancel{\sum\Delta U} + \Delta E_\text{int}=
%    W_\text{ext}
%  }
%
%  \textbf{HOWEVER}, aside from the (macroscopic) mechanical work done, there is
%  also a ``non-mechanical'' transfer of energy (i.e.\ work done at the
%  \emph{microscopic} level), called \textbf{heat} $Q$. The law of conservation
%  of energy becomes the first law of thermodynamics:
%
%  \begin{equation}
%    \Delta E_\text{int}= Q + W_\text{ext}
%  }
%
%
%
%
\begin{definition}
  In the \textbf{first law of thermodynamics}, the change in internal energy of
  a thermodynamic system ($\Delta U$) is the sum of the heat transferred ($Q$)
  and the net external mechanical work ($W$) done \emph{to} the system
  \begin{equation}
    \boxed{
      \Delta U=Q+W
    }
  \end{equation}
\end{definition}

%  \begin{center}
%    \begin{tabular}{l|c|c}
%      \rowcolor{pink}
%      \textbf{Quantity} & \textbf{Symbol} & \textbf{SI Unit} \\ \hline
%      Change in internal energy of a system & $\Delta U$ & \si\joule \\
%      Mechanical work done \emph{to} the system & $W$    & \si\joule \\
%      Heat transfer into the system & $Q$ & \si\joule
%    \end{tabular}
%  \end{center}
%  \vspace{.3in}

\begin{remark}
  Equation is written in a \emph{slightly} different notation from the last
  slide by convention, in that here, we use ``$U$'' for internal energy instead
  of macroscopic potential energy
\end{remark}

\begin{remark}
  \textbf{IMPORTANT NOTE:} The equation is also often written as
  $\Delta U=Q-W$, where $W$ is the work done \emph{by} the system. Since both
  conventions are commonly use, you must be very careful.
\end{remark}



\subsection{Internal Energy}

Internal energy $U$ is the total kinetic and potential energies of the
atoms/molecules at the microscopic level. To calculate the internal energy of a
collection of molecules, we first note the average kinetic energy of an
ideal-gas molecule. A molecule can translate in three dimensions (its velocity
can have the $\hat x$, $\hat y$ and $\hat z$ components), so kinetic energy can
be divided amongst the three ``degrees of freedom'' (DoF):
\begin{align*}
  \langle K \rangle&=\frac12m\langle v^2\rangle
  =\frac12m\langle v_x^2+v_y^2+v_z^2\rangle\\
  &=\frac12m\langle v_x^2\rangle+\frac12m\langle v_y^2\rangle+
  \frac12m\langle v_z^2\rangle
\end{align*}
Since the motion of the molecules is random (i.e.\ no preferred direction of
motion), the component averages must be the same, i.e.
\begin{equation}
  \langle v_x^2\rangle=\langle v_y^2\rangle=\langle v_z^2\rangle
\end{equation}

\begin{definition}
  \textbf{Equipartition theorem:} Since the motion is random, the internal
  energy of a thermodynamic system must be evenly divided between each degrees
  of freedom (DoF). Each DoF has a contribution of
  \begin{equation*}
    \frac12Nk_BT\quad\text{\normalsize or}\quad\frac12nRT
  \end{equation*}
  towards the total internal energy
\end{definition}

\emph{Monatomic} gases have 3 translational DoF, therefore the internal
energy is the total kinetic energy, which were shown earlier:
\begin{equation}
  U=\frac32Nk_BT=\frac32nRT
\end{equation}
%
\emph{Diatomic} gases
%\footnote{In
%  %\footnote{By definition, diatomic gases cannot also
%  %be ideal gases. However, in
%  AP Physics 2, you will often hear of a ``diatomic ideal gas''. In those
%  cases, we assume an equality rather than an approximation}
have 3 translational DoF, and 2 rotational DoF, therefore the total internal
energy is:
\begin{equation}
  U=\frac52Nk_BT=\frac52nRT
\end{equation}

\emph{Solids} typically have 3 translational DoF, plus 3 vibrational DoF in
elastic compression, and therefore the internal energy is:
\begin{equation}
  U\approx3Nk_BT\quad\text{\normalsize or}\quad 3nRT
\end{equation}
It should be noted that the above approximation is more accurate for solids
that have well-defined lattice structures (e.g.\ metals and their alloys). For
other substances, this approximation can be \emph{very} inaccurate.

What is conspicuously missing is the description of liquids. In general, there
is no viable approximation for the internal energy of a liquid because of the
complexity. However, regardless of whether a substance is a gas, liquid or a
solid, the internal energy is always proportional to absolute temperature:
\begin{equation}
  U\propto T
\end{equation}
Therefore we can observe changes in the internal energy by measuring how (and
by how much) temperature is changing.



\subsection{Heat}

\textbf{Heat} {\color{blue}$Q$} is the spontaneous non-mechanical transfer of
energy \emph{into} the system, through conduction, convection and radiation.

%  \eq{-.2in}{
%    \boxed{\Delta U={\color{blue}Q}+W}
%  }
%  \begin{itemize}
%  \item Heat transfer $Q$ is:
%    \begin{itemize}
%    \item $+$ if heat is added to the system
%    \item $-$ if heat leaves the system
%    \end{itemize}
%  \item The net flow of energy is always from the higher temperature to low
%    temperature (second law of thermodynamics)
%  \item Two objects are in thermal equilibrium if the temperatures are the same
%  \end{itemize}
%
%
%
%
\subsection{Mechanical Work}

{\color{orange}$W$} is the mechanical work%\footnote{The full
%    definition in calculus form is $W=\int PdV$ (i.e.\ pressure times change in
%    volume) which is derived from the definition of mechanical work:
%    $W=\int\vec F\cdot d\vec x$ (force times distance)} done \emph{to}
%  the system by the surrounding.
%
%  \begin{equation}
%    \boxed{
%      \Delta U=Q+{\color{orange}W}
%    }
%  }
%  \begin{itemize}
%  \item Work is done if the volume changes
%  \item At constant pressure:
%
%    \eq{-.15in}{
%      W=-P\Delta V
%    }
%    
%  \item\vspace{-.1in}\textbf{POSITIVE} if done \emph{to} the system, e.g.\
%    pushing a piston to compress gas in an engine
%  \item\textbf{NEGATIVE} if done \emph{by} the system, e.g.\ using steam
%    pressure to push a piston or shaft
%  \end{itemize}
%  \vspace{.4in}
%
%
%
%
%%{Example of a Thermal System}
%%  \centering
%%  \begin{tikzpicture}[scale=1.2]
%%    \begin{scope}[thick]
%%      \draw[fill=cyan!10] (5,0)--(1.5,0)--(1.5,1)--(0,1)--(0,2)--(4,2)
%%      --(4,.7)--(5,.7);
%%      \draw[fill=black!2] (.2,1.2) rectangle (1.5,1.8);
%%    \end{scope}
%%    \fill (4.8,0) rectangle (5,.7);
%%
%%    \node at (2.8,1.3) {\normalsize gas};
%%    
%%    \foreach \y in {.05,.25,...,.7} \draw[vectors] (5.5,\y)--(5,\y);
%%    \node[right=4,text width=135,draw=black,fill=gray!10] at (5.5,.35){
%%      {\scriptsize External pressure compressing the gas through the piston
%%        does \emph{positive} work \emph{to} the thermal system\par}
%%      
%%      \vspace{-.38in}{\Large
%%        \begin{displaymath}
%%          +W
%%        \end{displaymath}
%%      }
%%      
%%      \vspace{-.07in}{\scriptsize Work done \emph{by} the system is
%%        negative.\par}
%%    };
%%    \foreach \x in {1.75,2.25,...,4}
%%    \draw[axes,red,decorate,decoration=snake] (\x,2.1)--+(0,1.2);
%%    \node[above=4,text width=100,red,draw=red,fill=red!10] at (2.8,3.3){
%%      {\scriptsize Heat loss through radiation at the walls of the container
%%        \par}
%%
%%      \vspace{-.38in}{\Large
%%        \begin{displaymath}
%%          -Q
%%        \end{displaymath}
%%        \par}
%%    };
%%  \end{tikzpicture}
%%
%
%
%
\section{Heat Capacity}
%
%{Heat Capacity}
In many situations, heat transferred from a thermodynamic system ($Q$) can be
directly related to the change in temperature ($\Delta T$) by the object's
\textbf{heat capacity} ($C$):
\begin{equation}
    Q = C\Delta T
\end{equation}

%  \vspace{-.1in}$C$ has a unit of \textbf{joule per kelvin}
%  (\si{\joule\per\kelvin}) or \text{joule per degree Celsius}
%  (\si{\joule\per\celsius}).
Heat capacity depends on the object's mass as well as its composition.
%%  It can be also expressed using a related quantity called the \textbf{molar
%%    heat capacity}, which is the energy required to heat \SI1{\mol} of a
%%  substance by \SI1\kelvin:
%%
%%  \eq{-.15in}{
%%    \boxed{ Q = nc_m\Delta T }
%%  }
However, more practically, the \textbf{specific heat capacity} (lower case $c$)
is the energy required to raise the temperature of \SI1{\kilo\gram} of a
substance by \SI1{\kelvin} or \SI1\celsius:
\begin{equation}
  \boxed{
    Q = mc\Delta T
  }
\end{equation}
$c$ is a physical property of a substance, with an SI unit of
\textbf{joule per kilogram per kelvin} (\si{\joule\per\kilo\gram.\kelvin})



\subsection{Specific Heat Capacity for Solids}

For solids, when heat is added or removed, usually little to no work is done
\emph{by} the solid (i.e.\ $W\approx 0$). Therefore all the heat goes into
changing the internal energy:
\begin{equation}
  Q\approx\Delta U\approx\underbracket[1pt]{3nR}_{mc}\Delta T
\end{equation}
%%  The molar heat capacity is approximately the same for \emph{most} solids.
%%  Note that the universal gas constant still makes an appearance!
%%
%%  \eq{-.15in}{
%%    c_m\approx\frac Cn=3R
%%  }
And the specific heat capacity can be obtained by:
%$m=nM$ from the heat capacity:
\begin{equation}
  c\approx\frac{3nR}m\quad\rightarrow\quad c\approx\frac{3R}M
\end{equation}
This approximation applies best to substances that have a well-defined lattice
structure (e.g.\ metals), but may be \emph{very} inaccurate non-metallic
substances. In general, $c$ is determined experimentally.




\subsection{Heat Capacities for Gases at Constant Volume}

For gases, the value for heat capacities depends on whether work is also done
when heat is added/removed from the gas. When heat is added/removed at
\emph{constant volume}, no work is done, therefore all the heat goes into
changing the internal energy. For monatomic or ideal gases:
\begin{equation}
  Q=\Delta U={\color{magenta}\frac32nR}\Delta T=
  {\color{magenta}mc_v}\Delta T\quad\rightarrow\quad
  c_v=\frac{3nR}{2m}=\frac{3R}M
\end{equation}
while for diatomic gases:
\begin{equation}
  Q=\Delta U\approx{\color{cyan}\frac52nR}\Delta T
  ={\color{cyan}mc_v}\Delta T=\quad\rightarrow\quad
  c_v\approx\frac{5nR}{2m}=\frac{5R}{2M}
\end{equation}
%
%
%
%
%%{Heat Capacities of Gases at Constant Volume}
%%%  The molar heat capacity of a gas is therefore:
%%%
%%%  \begin{equation}
%%%    c_m=\frac{C_v}n=\frac32R\quad\text{(ideal)}\quad\quad
%%%    c_m=\frac{C_v}n\approx\frac52R\quad\text{(diatomic)}
%%%  }
%%
%%  And the specific heat capacities for ideal/monatomic and diatomic gases are,
%%  respectively:
%%
%%  \begin{equation}
%%    c_v=\frac{C_v}m=\frac32\frac RM\quad\text{(ideal)}\quad\quad
%%    c_v=\frac{C_v}m\approx\frac52\frac RM\quad\text{(diatomic)}
%%  }
%%
%
%
%
%{Heat Capacities of Gases at Constant Pressure}
%  When heat is added/removed at \emph{constant pressure}, volume changes,
%  and work is done by/to the gas. This work can be related to temperature change
%  using the ideal gas law:
%
%  \eq{-.15in}{
%    W=-P\Delta V = -nR\Delta T
%  }
%  
%  \vspace{-.15in}The heat added is therefore
%
%  \eq{-.15in}{
%    Q=\Delta U - W =mc_v\Delta T + nR\Delta T\quad\rightarrow\quad
%    \boxed{c_p=c_v+nR}
%  }
%
%%  which means that heat capacities are:
%%
%%  \eq{-.15in}{
%%    C_p=\frac52nR\quad\text{(ideal)}\quad\quad
%%    C_p\approx\frac72nR\quad\text{(diatomic)}
%%  }
%%
%%
%%
%%
%%{Heat Capacities of Gases at Constant Pressure}
%%  The molar heat capacity of a gas at constant pressure is therefore:
%%
%%  \begin{equation}
%%    c_m=\frac52R\quad\text{(ideal)}\quad\quad
%%    c_m\approx\frac72R\quad\text{(diatomic)}
%%  }
%
%  which means that the specific heat capacity is:
%
%  \begin{equation}
%    c_p=\frac52\frac RM\quad\text{(ideal)}\quad\quad
%    c_p\approx\frac72\frac RM\quad\text{(diatomic)}
%  }
%
%
%
%
%{Specific Heat Capacity}
%  
%    \column{.55\textwidth}
The actual specific heat capacities of different substances are usually
found experimentally
%
\begin{table}[ht]
  \centering
  \begin{tabular}{|l|r|}
    \hline
    \rowcolor{pink}
    \textbf{Substance} & $c$ (\si{\joule\per\kilo\gram.\kelvin}) \\
    \hline
    ethyl alcohol & 2450 \\
    glycerine     & 2410 \\
    mercury       & 139 \\
    water (at \SI{15}\celsius) & 4186 \\
    \hline
    aluminum & 900 \\
    copper   & 387 \\
    glass    & 840 \\
    human body (\SI{37}\celsius) & 3500 \\
    ice (\SI{-15}\celsius)       & 2000 \\
    steel    & 452 \\
    lead     & 128 \\
    silver   & 235 \\
    \hline
  \end{tabular}
  \caption{Specific heat capacities of common substances}
  \label{tabl:specific-heat-capacity}
\end{table} 


\section{Phase Change}

%{Fundamental Phases}
We generally understand that there are four natural fundamental states of
matter: \textbf{solid}, \textbf{liquid}, \textbf{gas} and \textbf{plasma}
%  \begin{center}
%    \pic1{state-transition}
%%    \begin{tikzpicture}[
%%        squarenode/.style={rectangle, very thick, minimum width=15mm},
%%      ]
%%      \node[squarenode,draw=green!70!black,fill=green!10] (s) {Solid};
%%      \node[squarenode,draw=cyan,fill=cyan!10] (l) at (3,1.5){Liquid};
%%      \node[squarenode,draw=yellow,fill=yellow!10] (g) at (0,3) {Gas};
%%      \node[squarenode,draw=red!70!black,,fill=red!10] (p) at (3,4.5){Plasma};
%%      \begin{scope}[->,thick]
%%        \draw (g.south)--(s.north)node[midway,above,sloped]{\tiny Deposition};
%%        \draw (s.north)--(g.south)node[midway,above,sloped]{\tiny Sublimation};
%%      %\draw[->] (maintopic.east) -- (rightsquare.west);
%%        %\draw[->] (rightsquare.south) .. controls +(down:7mm) and +(right:7mm) .. (lowercircle.east);
%%      \end{scope}
%%    \end{tikzpicture}
%  \end{center}
%
%
%
%
%{Phase Transition}
%  
%    \column{.38\textwidth}
%    \centering
%    Phase diagram of water\\
%    \pic{1.1}{10-figure-31}
%
%    \column{.62\textwidth}
%    \begin{itemize}
%    \item Which phase change can occur depends on pressure and temperature.
%    \item For example, water in its liquid form cannot exist below
%      \SI{.6}{\kilo\pascal}, and so adding enough energy to ice will directly
%      change it to water vapour
%    \item At \textbf{triple point} (B), the liquid, gas and solid phases can
%      exist in thermal equilibrium
%    \item At \textbf{critical point} (C), the liquid and gas phases become
%      indistinguisheable
%    \item Beyond (C), the matter is considered to be a \emph{supercritical
%    fluid}
%    \end{itemize}
%  
%
%
%
%
%{During Phase Transition}
During phase change, heat is either added or taken away from a substance, but
the temperature does not change. This is a heating curve for water at
atmospheric pressure:
%  \begin{center}
%    \pic{.6}{heating-curve}
%  \end{center}
%  The mechanics of phase transition is \emph{very} complex, and beyond the
%  scope of this course.
%
%
%
%
\subsection{Latent Heat}

\textbf{Specific latent heat} is the heat required to change the phase of
\SI1{\kilo\gram} of a substance. Here, we are only concerned with
\textbf{specific latent heat of fusion} $L_f$ (for melting and freezing) and
\textbf{specific latent heat of vaporization} $L_v$ (vaporization and
condensation).
\begin{equation}
  \boxed{Q = mL_f} \quad\quad \boxed{Q = mL_v}
\end{equation}
%  \begin{center}
%    \begin{tabular}{l|c|c}
%      \rowcolor{pink}
%      \textbf{Quantity} & \textbf{Symbol} & \textbf{SI Unit} \\ \hline
%      Heat required to change state & $Q$ & \si\joule \\
%      Mass                          & $m$ & \si{\kilo\gram} \\
%      Specific latent heat of fusion & $L_f$ & \si{\joule\per\kilo\gram} \\
%      Specific latent heat of vaporization & $L_v$ & \si{\joule\per\kilo\gram}
%    \end{tabular}
%  \end{center}
During phase change, the temperature will remain constant. Specific latent heat
values for some common substances:

\begin{table}[ht]
  \centering
  \begin{tabular}{|l|r|r|r|r|}
    \hline
    \rowcolor{pink}
    Substance & Ice Point (\si\celsius) & $L_f$ (\si{kJ/kg}) &
    Vapour Point (\si\celsius) & $L_v$ (\si{kJ/kg}) \\ \hline
    water         & \num{   0} & \num{335} & \num{ 100} & \num{ 2272} \\
    aluminum      & \num{ 659} & \num{399} & \num{2327} & \num{10530} \\
    copper        & \num{1083} & \num{207} & \num{2595} & \num{ 4730} \\
    ethyl alcohol & \num{-114} & \num{108} & \num{  78} & \num{  855} \\
    hydrogen      & \num{-259} & \num{ 58} & \num{-253} & \num{  455} \\
    lead          & \num{ 328} & \num{ 23} & \num{1750} & \num{  859} \\
    mercury       & \num{ -39} & \num{ 11} & \num{ 357} & \num{  295} \\
    nitrogen      & \num{-210} & \num{ 26} & \num{-196} & \num{  200} \\
    oxygen        & \num{-219} & \num{ 14} & \num{-183} & \num{  213} \\
    silver        & \num{ 962} & \num{111} & \num{1950} & \num{ 2356} \\
    \hline
  \end{tabular}
\end{table}



\section{Quasi-Static Process}
%
%{Work Done By a Gas}
%  On the $P$-$V$ diagram, mechanical work $W$ is the area under the curve.
%  \begin{itemize}
%  \item $W$ is $-$ (work done \emph{by} the system) if the path moves towards
%    the right
%  \item $W$ is $+$ (work done \emph{to} the system) if the path moves towards
%    the left
%  \end{itemize}
In the example below, the gas is changed from the same states (1 to 2), but
the mechanical work done is different.
\begin{figure}[ht]
  \centering
  \begin{subfigure}{.32\textwidth}
    \centering
    \begin{tikzpicture}[scale=.65]
      \fill[pink!50] (1,0) rectangle (4,3.5);
      \draw[dashed] (1,0)--(1,3.5);
      \draw[dashed] (4,0)--(4,.5);
      \draw[axes] (0,0)--(5,0) node[right]{$V$};
      \draw[axes] (0,0)--(0,4) node[left]{$P$};
      \draw[vectors,red] (1,3.5)--(4,3.5)--(4,.6);
      \fill (1,3.5) circle (.1) node[above]{\tiny$(P_1,V_1,T_1)$};
      \fill (4,.5) circle (.1)  node[right]{\tiny$(P_2,V_2,T_2)$};
    \end{tikzpicture}
  \end{subfigure}
  \hspace{\stretch1}
  \begin{subfigure}{.32\textwidth}
    \centering
    \begin{tikzpicture}[scale=.65]
      \fill[pink!50] (1,0) rectangle (4,.5);
      \draw[dashed] (1,0)--(1,3.5);
      \draw[dashed] (4,0)--(4,.5);
      \draw[axes] (0,0)--(5,0) node[right]{$V$};
      \draw[axes] (0,0)--(0,4) node[left]{$P$};
      \draw[vectors,red] (1,3.5)--(1,.5)--(3.9,.5);
      \fill (1,3.5) circle (.1) node[above]{\tiny$(P_1,V_1,T_1)$};
      \fill (4,.5) circle (.1)  node[right]{\tiny$(P_2,V_2,T_2)$};
    \end{tikzpicture}
  \end{subfigure}
  \hspace{\stretch1}
  \begin{subfigure}{.32\textwidth}
    \centering
    \begin{tikzpicture}[scale=.65]
      \fill[pink!50] (1,0)--(1,3.5)..controls(1.5,2) and (3.2,.7)..(4,.5)--
      (4,0)--cycle;
      \draw[dashed] (1,0)--(1,3.5);
      \draw[dashed] (4,0)--(4,.5);
      \draw[axes] (0,0)--(5,0) node[right]{$V$};
      \draw[axes] (0,0)--(0,4) node[left]{$P$};
      \draw[vectors,red] (1,3.5)..controls(1.5,2) and (3.2,.7)..(3.95,.55);
      \fill (1,3.5) circle (.1) node[above]{\tiny$(P_1,V_1,T_1)$};
      \fill (4,.5) circle (.1)  node[right]{\tiny$(P_2,V_2,T_2)$};
    \end{tikzpicture}
  \end{subfigure}
\end{figure}
The fact that $W$ is \emph{path dependent} means that the force exerted by
the gas $F=P\cdot A$ is non-conservative (it may be obvious to you already).

A \textbf{quasi-static process} is a thermodynamic process that happens slowly
enough for the system to remain in internal equilibrium. We are concerned with
four of these processes:
\begin{itemize}
\item\textbf{Isobaric process}--constant pressure\footnote{If you don't know
  already, a ``bar'' is a measurement of pressure, where
  $\SI1{bar}=\SI{10e5}{\pascal}$}
\item\textbf{Isochoric process}--constant volume
\item\textbf{Isothermal process}--constant temperature
\item\textbf{Adiabatic process}--``isolated'', no heat exchanged with
  surrounding
\end{itemize}

\begin{figure}[ht]
  \centering
  \begin{subfigure}{.4\textwidth}
    \centering
    \begin{tikzpicture}[scale=.6]
      \draw[axes] (0,0)--(5,0) node[right]{$V$} node[midway,below]{Isobaric};
      \draw[axes] (0,0)--(0,4) node[left]{$P$};
      \draw[thick] (1,2)--(3.95,2);
      \fill (1,2) circle (.1);
      \fill (4,2) circle (.1);
      %\draw[thick] (0,2)--(-.2,2) node[left]{\scriptsize$P$};
      \draw[axes,red] (1.5,2.2)--(3.5,2.2) node[midway,above=0]{\tiny$W<0$};
      \draw[axes,red] (3.5,1.8)--(1.5,1.8) node[midway,below=0]{\tiny$W>0$};
    \end{tikzpicture}
  \end{subfigure}
  \begin{subfigure}{.4\textwidth}
    \centering
    \begin{tikzpicture}[scale=.6]
      \draw[axes] (0,0)--(5,0) node[right]{$V$} node[midway,below]{Isochoric};
      \draw[axes] (0,0)--(0,4) node[left]{$P$};
      \draw[thick] (2,1)--(2,3.5);
      \fill (2,1) circle (.1);
      \fill (2,3.5) circle (.1);
      \node at (5,4) {$\Delta U=Q+\cancel W$};
    \end{tikzpicture}
  \end{subfigure}
  
  \begin{subfigure}{.4\textwidth}
    \centering
    \begin{tikzpicture}[scale=.6]
      \draw[axes] (0,0)--(5,0) node[right]{$V$} node[midway,below]{Isothermal};
      \draw[axes] (0,0)--(0,4) node[left]{$P$};
      \draw[smooth,samples=15,domain=1:3.5,thick] plot(\x,{3/\x});
      \fill (1,3) circle (.1);
      \fill (3.5,.85) circle (.1);
    \end{tikzpicture}
  \end{subfigure}
  \begin{subfigure}{.4\textwidth}
    \centering
    \begin{tikzpicture}[scale=.6]
      \draw[axes] (0,0)--(5,0) node[right]{$V$} node[midway,below]{Adiabatic};
      \draw[axes] (0,0)--(0,4) node[left]{$P$};
      \begin{scope}[smooth,samples=15,domain=1:4,thick]
        \draw[lightgray] plot(\x,{4/\x});
        \draw[lightgray] plot(\x,{2.3/\x});
        \draw plot(\x,{4*((\x)^(-1.4))});
      \end{scope}
      \fill(1,4) circle (.1);
      \fill(4,.57) circle (.1);
    \end{tikzpicture}
  \end{subfigure}
\end{figure}

%{A Simple Heat Engine Cycle}
%  
%    \column{.8\textwidth}
%    \begin{center}
%      \pic{.52}{heat-engine}
%    \end{center}
%    \begin{enumerate}
%    \item\vspace{-.15in} Heat is added at constant volume; no work done.
%    \item Heat is added as gas expands at constant pressure; work is done by
%      the gas to lift the weight.
%    \item Heat is extracted at constant volume; no work done.
%    \item Heat is extracted at constant pressure; work is done on the gas to
%      compress it.
%    \end{enumerate}
%
%    \column{.2\textwidth}
%    \begin{tikzpicture}[scale=.5]
%      \fill[pink!50](1,1) rectangle(4,3.5);
%      \draw[axes] (0,0)--(5,0) node[below]{$V$};
%      \draw[axes] (0,0)--(0,4) node[left]{$P$};
%      \draw[axes] (1,1)--(1,3.5)--(4,3.5)--(4,1)--(1,1);
%      \fill (1,1) circle (.08) node[left]{$a$};
%      \fill (1,3.5) circle (.08) node[left]{$b$};
%      \fill (4,3.5) circle (.08) node[right]{$c$};
%      \fill (4,1) circle (.08) node[right]{$d$};
%    \end{tikzpicture}\\
%    {\footnotesize $P$-$V$ diagram of the simple heat engine shown on the left.
%      \par}
%  
%
%
%
%
%{Efficiency of Heat Engine}
%  In a heat engine, the internal energy at the beginning and end of the cycle
%  are the same (same point on the $P$-$V$ diagram), so the work done is just
%  the difference between heat added and taken out:
%  
%  \eq{-.25in}{
%    W = Q_\text{in}-Q_\text{out}
%  }
%  
%  Efficiency is defined as the ratio between work done and heat added:
%
%  \begin{equation}
%    \boxed{\eta =\frac W{Q_\text{in}}=1-\frac{Q_\text{out}}{Q_\text{in}}}
%  }
%
%
%
%
%%{Second Law of Thermodynamics}
%%  \begin{block}{Kelvin-Planck Statement}
%%    It is impossible for heat engine working in a cycle to produce no other
%%    effect than that of extracting heat from a reservoir and performing an
%%    equivalent amount of work.
%%  \end{block}
%%
%
%
%
%{Carnot Engine}
%  The Carnot engine cycle is the most efficient.
%  \begin{center}
%    \pic{.4}{carnot}
%  \end{center}
%  The efficiency of a Carnot engine is:
%  
%  \begin{equation}
%    \boxed{\eta_C =1-\frac{Q_\text{out}}{Q_\text{in}}=1-\frac{T_c}{T_h}}
%  }
