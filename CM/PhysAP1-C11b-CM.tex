%\documentclass[12pt,compress,aspectratio=169]{beamer}
%\input{../mybeamer}
%
\chapter{Centre of Mass}
\label{chapter:cm}

%{Centre of Mass}
%  Finding an object's centre of mass is important, because
%  \begin{itemize}
%  \item The laws of motion are formulated by treating an objects as point
%    masses (for real-life objects, we let the forces apply to the centre of
%    mass)
%  \item Objects can have \emph{rotational} motion in addition to
%    \emph{translational} motion as well (we will examine that a bit more in a
%    very-important topic later)
%  \end{itemize}
%\end{frame}
%
%
%
\section{Centre of Mass: Definition}
The \textbf{centre of mass} (``CM'') is the \emph{weighted average} of the
masses in a system. The ``system'' may be:
\begin{itemize}
\item A collection of individual particles
\item A continuous distribution of mass with constant density. In this case,
  CM is also the geometric centre (\textbf{centroid}) of the object
\item A continuous distribution of mass with varying density
\item If the masses are inside of a \emph{uniform} gravitational
  field\footnote{See discussion on gravitational field in the first part of
  this class}, then the CM is also its \textbf{centre of gravity} (``CG'')
\end{itemize}




\section{Finding the Centre of Mass}

%{Simple Example}
%  Two equal masses $m$ along the $x$-axis, located at $x_1$ and $x_2$. Where is
%  the centre of mass of the system?
%  \begin{center}
%    \begin{tikzpicture}
%      \draw[axes] (-1,0)--(8,0) node[right]{$x$};
%      \draw[mass] (2,0) circle (.2) node{$m$};
%      \draw[mass] (6,0) circle (.2) node{$m$};
%      \draw[thick] (0,.85)--(0,-.45) node[below]{$O$};
%      \fill (4,0) circle (.05) node[above]{cm};
%      \begin{scope}[very thick,->|]
%        \draw (0,.35)--(2,.35) node[pos=0,left]{$x_1$};
%        \draw (0,.75)--(6,.75) node[pos=0,left]{$x_2$};
%        \draw[violet] (0,-.3)--(4,-.3) node[pos=0,left]{$x_\text{cm}$};
%      \end{scope}
%    \end{tikzpicture}
%  \end{center}
%  The centre of mass is at the half-way point between the masses:
%
%  \eq{-.1in}{
%    x_\text{cm}=\frac{x_1+x_2}2
%    \quad\quad\text{or}\quad\quad
%    x_\text{cm}=\frac{mx_1+mx_2}{2m}
%  }

%
%
%
%{Slightly More Challenging}
%  What if one of the masses are increased to $2m$? This is still not a
%  difficult problem; you can still \emph{guess} the right answer without
%  knowing the equation for center of mass. 
%  \begin{center}
%    \begin{tikzpicture}
%      \draw[axes] (-4,0)--(4,0) node[right]{$x$};
%      \draw[mass] (-2.5,0) circle (.3) node{$m$};
%      \draw[mass] (2.5,0) circle (.42) node{$2m$};
%      \fill (5/6,0) circle (.05) node[below]{cm};
%    \end{tikzpicture}
%  \end{center}
%  The answer is still simple. The centre of mass is no longer half way between
%  the two masses, but now $\frac13$ the total distance from the larger masses.
%  We can show using a weighted average:
%  
%  \eq{-.1in}{
%    x_\text{cm}=\frac{mx_1+(2m)x_2}{m+2m}
%  }

%
%
%
%{Many Point Masses}
%  The weighted average concept can now be applied to cases when there are
%  masses in two or more dimensions:
%  \begin{center}
%    \begin{tikzpicture}
%      \draw[axes] (-3,0)--(3,0) node[right]{$x$};
%      \draw[axes] (0,-1.5)--(0,1.5) node[above]{$y$};
%      \draw[mass] (-1.3,1) circle (.4) node{$m_1$};
%      \draw[mass] (-1.5,-.5) circle (.3) node{$m_2$};
%      \draw[mass] (1,.3) circle (.25) node{$m_3$};
%      \draw[mass] (0,.3) circle (.2) node{$m_4$};
%      \draw[mass] (2,-1) circle (.25) node{$m_5$};
%    \end{tikzpicture}
%  \end{center}

%
%
%
%{Centre of Mass Equation}
%  For a discrete number of $N$ masses, the centre of mass is defined as the
%  weighted average of the positions of the masses:
%
%  \eq{-.1in}{
%    \boxed{\vec x_\text{cm}=\frac{\sum_{i=1}^N m_i\vec x_i}{\sum_{i=1}^N m_i}}
%  }
%  \begin{center}
%    \begin{tabular}{l|c|c}
%      \rowcolor{pink}
%      \textbf{Quantity} & \textbf{Symbol} & \textbf{SI Unit} \\ \hline
%      Position of centre of mass (vector) & $\vec x_\text{cm}$ & \si\metre \\
%      Position of point mass $i$ (vector) & $\vec x_i$ & \si\metre \\
%      Point mass $i$ & $m_i$ & \si{\kilo\gram}
%    \end{tabular}
%  \end{center}
%  In components:
%
%  \eq{-.1in}{
%    x_\text{cm}=\frac{\sum m_ix_i}{m_\text{tot}}\quad\quad
%    y_\text{cm}=\frac{\sum m_iy_i}{m_\text{tot}}\quad\quad
%    z_\text{cm}=\frac{\sum m_iz_i}{m_\text{tot}}
%  }

%
%
%
%{An Example}
%  \textbf{Example 1:} Consider the following masses and their coordinates
%  which make up a ``discrete mass'' rigid body''
%  \begin{align*}
%    m_1&=\SI{5.0}{\kg} &\vec x_1&=3\xxx-2\zzz\\
%    m_2&=\SI{10.0}{\kg}&\vec x_2&=-4\xxx+2\yyy+7\zzz\\
%    m_3&=\SI{1.0}{\kg}&\vec x_3&=10\xxx-17\yyy+10\zzz
%  \end{align*}
%  What are the coordinates for the centre of mass of this system?

%
%
%
%{Continuous Mass Distribution}
%  When the number of masses approaches infinity, this becomes a continuous
%  distribution of mass. Taking the limit of masses $N\rightarrow\infty$ gives
%  the \emph{integral form} of our equation:
%
%  \eq{-.1in}{
%    \vec x_\text{cm}=\frac{\int\vec xdm}{\int dm}}
%
%  Calculus is not part of the AP Physics 1 curriculum, so you are welcome to
%  skip this slide. But this calculation is part of AP Physics C, which is
%  calculus-based. 

%
%
%
%{Centroid}
%  For an object with a uniform mass distribution (i.e.\ constant density), the
%  centre of mass is also its geometric centre, called the \textbf{centroid}.
%  \begin{center}
%    \pic{.7}{eng130C9_11}
%  \end{center}
%  The locations of centroids can be found in most physics textbooks.

%
%
%
%{Compound Shapes}
%  For compound shapes, the centre of mass is the weighted average of the centre
%  of mass of each component. For example, for the T-beam below:
%  \begin{center}
%    \begin{tikzpicture}[scale=.7]
%      \draw[axes] (0,0)--(4.5,0) node[right]{$x$};
%      \draw[axes] (0,0)--(0,4.5) node[left]{$y$};
%      \draw[blue!70!black,mass] (0,4)--(4,4)--(4,3)--(2.5,3)
%      --(2.5,0)--(1.5,0)--(1.5,3)--(0,3)--(0,4);
%    \end{tikzpicture}
%    \hspace{.1in}
%    \begin{tikzpicture}[scale=.7]
%      \draw[lightgray,fill=gray!20,thick] (0,4)--(4,4)--(4,3)--(2.5,3)--(2.5,0)
%      --(1.5,0)--(1.5,3)--(0,3)--(0,4);
%      \draw[lightgray,dotted,thick] (1.5,3)--(2.5,3);
%      \draw[axes] (0,0)--(4.5,0) node[right]{$x$};
%      \draw[axes] (0,0)--(0,4.5) node[left]{$y$};
%      \fill[blue!70!black] (2,3.5) circle (.08) node[right]{$\vec x_1$};
%      \fill[blue!70!black] (2,1.5) circle (.08) node[right]{$\vec x_2$};
%    \end{tikzpicture}
%    \hspace{.1in}
%    \begin{tikzpicture}[scale=.7]
%      \draw[lightgray,thick] (0,4)--(4,4)--(4,3)--(2.5,3)
%      --(2.5,0)--(1.5,0)--(1.5,3)--(0,3)--(0,4);
%      \draw[axes] (0,0)--(4.5,0) node[right]{$x$};
%      \draw[axes] (0,0)--(0,4.5) node[left]{$y$};
%      \fill[blue!30] (2,3.5) circle (.08) node[right]{$\vec x_1$};
%      \fill[blue!30] (2,1.5) circle (.08) node[right]{$\vec x_2$};
%      \fill[red!70!black] (2,2.64) circle (.08) node[right]{$\vec x_\text{cm}$};
%    \end{tikzpicture}
%  \end{center}

%
%
%
%{Symmetric Configurations}
%  \begin{itemize}
%  \item Any plane of symmetry, mirror line, axis of rotation, point of inversion
%    \emph{must} contain the centre of mass.
%  \item Caveat: only works if the density distribution is also symmetric
%  \item Again: if density is uniform, CM is also geometric centre (centroid)
%  \end{itemize}

%
%
%
%{``Negative Mass''}{A Mathematical Trick}
%  \begin{itemize}
%  \item Where there is a ``hole'' in the geometry, treat it as having negative
%    mass density $-\sigma$ in that region.
%  \item Negative masses don't exist, so this is really just a trick.
%  \item\textbf{Example:} What is the centre of mass of this shape?
%    \begin{center}
%      \begin{tikzpicture}
%        \draw[thick,fill=lightgray] circle (2);
%        \draw[thick,fill=black!2] (0,1) circle (1);
%        \fill circle (.04);
%        \draw[axes] (0,0)--(-1.41,-1.41)node[midway,below]{$r$};
%        \fill (0,1) circle (.04);
%        \draw[axes] (0,1)--(.707,.293) node[right]{$r/2$};
%      \end{tikzpicture}
%    \end{center}
%  \end{itemize}

%
%
%
%{Negative Mass Example}
%  \begin{itemize}
%  \item This is how we would think of it:
%    \begin{center}
%      \begin{tikzpicture}[scale=.6]
%        \draw[thick,fill=lightgray] circle (2);
%        \draw[thick,fill=black!2] (0,1) circle (1);
%        \fill circle (.04);
%        \draw[axes] (0,0)--(-1.41,-1.41) node[midway,below]{$r$};
%        \fill (0,1) circle (.04);
%        \draw[axes] (0,1)--(.707,.293) node[right]{$r/2$};
%        \draw[thick,fill=lightgray] (6,0) circle (2) node{\large$A$};
%        \draw[thick] (11,1) circle (1) node{\large$B$};
%        \node at (3,0) {\huge=};
%        \node at (9,0) {\huge+};
%      \end{tikzpicture}
%    \end{center}
%  \item Let the origin of the coordinate system to located at the centre of $A$
%  \item Based on symmetry: $x_\text{cm}=0$; only have to find $y$-coordinate.
%  \end{itemize}
%
%  \eq{-.2in}{
%    y_\text{cm}
%    =\frac{\sum y_i m_i}{\sum m_i}
%    =\frac{m_A(0) + m_B (r/2)}{m_A+m_B}
%    =\frac{-\sigma\pi\left(r/2\right)^2(r/2)}
%    {\sigma\pi r^2-\sigma\pi\left(r/2\right)^2}
%    =\frac{-r}6
%  }

%
%
%
\section{Momentum and Centre of Mass}

\subsection{Velocity of the Centre of Mass}
Taking the change in the position of the CM ($\Delta\vec x_\text{cm}$) over a
finite time interval ($\Delta t$) gives the expression for $\vec v_\text{cm}$,
the average velocity of the centre of mass:
\begin{equation*}
  \vec v_\text{cm}
  =\frac{\color{magenta}\Delta\vec x_\text{cm}}{\Delta t}
  =\frac1{\Delta t}
  {\color{magenta}\left[\frac{\sum\Delta(m_i\vec x_i)}{\sum m_i}\right]}
  =\frac1{m_\text{tot}}
  \left[\sum m_i{\color{orange}\frac{\Delta\vec x_i}{\Delta t}}\right]
  =\frac1{m_\text{tot}}\sum m_i{\color{orange}\vec v_i}
\end{equation*}
The velocity of the centre of mass is the weighted sum of the velocities of
the discrete masses:
\begin{equation}
  \vec v_\text{cm} = \frac{\sum m_i\vec v_i}{m_\text{tot}}
\end{equation}
Like the expression for the position of the CM, the weight for the sum is the
individual masses.


\subsection{Velocity and Momentum}
We can rearrange the equation for the velocity of the centre of mass to relate
it to momentum, because the term $\sum m_i\vec v_i=\vec p_\text{net}$ is the net
momentum of \emph{all} the discrete masses:
\begin{equation}
  \vec v_\text{cm} = \frac{\sum m_i\vec v_i}{m_\text{tot}}
  \quad\longrightarrow\quad
  \vec p_\text{net}= \sum m_i\vec v_i = m_\text{tot}\vec v_\text{cm}
\end{equation}
During a collision, there is no change in the net momentum\footnote{Because
there are are no external forces}, the centre of mass will continue to move at
the same velocity before/after the collision, as if the collision never
occurred.

%
%
%
%{Centre of Mass During Collision}
%  \vspace{.1in}
%  During a collision\footnote{As we have studied in conservation of momentum in
%  Class 9}, there are no external forces, therefore the velocity of the CM
%  remains constant. Consider this perfectly inelastic collision in 1D  between
%  two masses:
\begin{figure}[ht]
  \centering
  \begin{tikzpicture}[scale=.8,thick]
    \begin{scope}[violet]
      \draw (1,0) rectangle (2,1) node[midway]{$m_1$};
      \draw (4,0) rectangle (5,1) node[midway]{$m_2$};
      \draw[vectors] (.8,1.3)--(2.5,1.3) node[right]{$v_1$};
      \draw[vectors] (4,1.3)--(5,1.3) node[right]{$v_2$};
      \node[above] at (3,1.5) {Before Collision};
    \end{scope}
    \draw[vectors] (2.7,.5)--(3.7,.5) node[above]{$v_\text{cm}$};
    \begin{scope}[orange]
      \draw (10,0) rectangle (11,1) node[midway]{$m_1$};
      \draw (11,0) rectangle (12,1) node[midway]{$m_2$};
      \draw[vectors] (10.2,1.3)--(11.8,1.3) node[right]{$v'$};
      \node[above] at (11,1.5) {After Collision};
    \end{scope}
    \draw (0,0)--(14,0);
    \draw (2.7,.5) circle (.15);
    \fill (2.7,.5)--(2.85,.5) arc(0:90:.15)--cycle;
    \fill (2.7,.5)--(2.55,.5) arc(180:270:.15)
    node[below=-2] (c){\scriptsize cm\par}--cycle; 
    
    \draw (11,.5) circle (.15);
    \fill (11,.5)--(11.15,.5) arc(0:90:.15)--cycle;
    \fill (11,.5)--(10.85,.5) arc(180:270:.15)
    node[below=-2] {\scriptsize cm\par}--cycle; 
    
    \node[text width=2.07in,draw,violet,below right] (a) at (-.5,-.6)
         {Using the definition of the velocity of the CM, we find
           that \emph{before} the collision, the CM moves at:
           \vspace{-.05in}
           \begin{displaymath}
             v_\text{cm} = \frac{\sum m_iv_i}{\sum m_i}
             =\frac{m_1v_1+m_2v_2}{m_1+m_2}
           \end{displaymath}\par
         };
         \draw[axes,violet] (a)--(c);
         
         \node[text width=2in,draw,orange,below left] at (14.5,-.6)
              {Using conservation of momentum, we find the final velocity
                \emph{after} the collision is the velocity of the CM:
                \vspace{-.05in}
                \begin{displaymath}
                  v'=\frac{m_1v_1+m_2v_2}{m_1+m_2}=v_\text{cm}
                \end{displaymath}
                %\vspace{-.08in}It is the same as $v_\text{cm}$ before the
                %collision!
                \par
              };
  \end{tikzpicture}
\end{figure}


\section{Acceleration of the Centre of Mass}
Finding the rate of change of the net momentum (i.e\ applying the 2nd law of
motion to this collection of masses):
%  
%  \eq{-.1in}{
%    \frac{\Delta\vec p_\text{net}}{\Delta t}=
%    \frac{\Delta(m\vec v_\text{cm})}{\Delta t}
%  }
%
%  If the system mass is constant, then this equation reduces to:
%
%  \eq{-.1in}{
%    \frac{\Delta\vec p_\text{net}}{\Delta t}
%    =m\frac{\Delta\vec v_\text{cm}}{\Delta t}
%    \quad\longrightarrow\quad
%    \boxed{
%      \vec F_\text{net}=m\vec a_\text{cm}
%    }
%  }
%  
%  We can see that when a net force is applied to an object, the object's
%  acceleration is evaluated at the centre of mass.

%
%
%%{What This All Means}
%%  \begin{itemize}
%%  \item Newton was right all along by treating all objects as point masses
%%    located at the CM
%%  \end{itemize}
