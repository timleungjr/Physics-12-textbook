%\section{Potential Energy \& Electric Potential}

\section{Electrical Potential Energy}

If an electrostatic force acts on a charge, and if the charge moves, then we
can calculate the work done by the electrostatic force. Bear in mind that if
the electrostatic forces come from point charges, as this charge moves, then
both the magnitude and direction of $\bm F_q$ changes. Therefore, calculating
the work done generally requires a fair amount of integral calculus. And if you
know calculus, you can easily find that the work done \emph{by} the
electrostatic force $W_q$ when two points charges move from a distance of $r_1$
to $r_2$:
\begin{equation}
  \underbracket[1pt]{{\color{magenta}W_q}=
  \int_{\bm r_1}^{\bm r_2}\bm F_q\cdot \dl{\bm r}}_\text{definition using
      calculus}
    =\cdots = kq_1q_2\int_{r_1}^{r_2}\frac{\dl{r}}{r^2}=\cdots
    ={\color{magenta}-\Delta U_q}
\end{equation}
If you don't know calculus, all you need to know is that the work done by the
electrostatic force results in the change in a potential energy $U_q$, defined
as the \textbf{electric potential energy}. For a two-charge system, the energy
stored is:
\begin{important-equation}
  U_q=\frac{kq_1q_2}r
\end{important-equation}
Crucially, because the work done by the electrostatic force is equal to the
negative change in electric potential energy:
\begin{important-equation}
  W_q=-\Delta U_q
  \label{eq:Uq}
\end{important-equation}
It means that the electrostatic force is, in fact \emph{conservative}.


Similar gravitational potential energy
%(Eq.~\ref{eq:GPE-general}),
Eq.~\ref{eq:Uq} uses  $r=\infty$ as the reference\footnote{The \emph{reference}
is where $U=0$ by definition}. However, unlike
gravitational potential energy which is \emph{always} negative, $U_q$ can be
either positive or negative, because charges can be either positive or
negative. When $U_g>0$ (energy stored between two positive charges or two
negative charges), external work must be done to bring the charges from
$r=\infty$ to $r$. Conversely, if $U_q<0$ (energy stored between one positive
and one negative charge), external work is done to pull two charges apart from
$r$ to $r=\infty$.
\begin{definition}
  \begin{enumerate}[leftmargin=15pt]
  \item When the work done by $\bm F_q$ is \emph{positive}, there is a
    \emph{decreases} in electric potential energy by the same amount
  \item When the work done by $\bm F_q$ is \emph{negative}, there is an
    \emph{increases} in electric potential energy by the same amount
  \item Work done is path independent: $W_q$ depends on the end points $r_1$ and
    $r_2$, but not \emph{how} the charges move from $r_1$ to $r_2$
  \item Only work done by $\bm F_q$ can change $U_q$
  \end{enumerate}
\end{definition}



%
%  To bring two charges together, I have to do work against electrostatic force,
%  therefore, there is a gain/lost in \textbf{electric potential energy},
%  defined as:
%  
%  \eq{-.1in}{
%    \boxed{U_q=\frac{kq_1q_2}r}
%  }
%  \begin{center}
%    \begin{tabular}{l|c|c}
%      \rowcolor{pink}
%      \textbf{Quantity} & \textbf{Symbol} & \textbf{SI Unit} \\ \hline
%      Electric potential energy & $U_q$ & \si\joule \\
%      Coulomb's constant & $k$ &\si{\newton\metre\squared\per\coulomb\squared}\\
%      Electric charges          & $q_1$, $q_2$ & \si\coulomb \\
%      Distance                  & $r$   & \si\metre
%    \end{tabular}
%  \end{center}



\subsection{Electric Potential Energy in More Than Two Charges}

When there are more than two charges, the total potential energy is the total
from all the \emph{systems of two charges}. For example, in the
\emph{three}-charge system in Fig.~\ref{fig:3charge},
\begin{figure}[ht]
  \centering
  \begin{tikzpicture}[scale=1.2]
    \draw (0,0)--(1,-2) node[midway,fill=white]{$r_3$}
    --(-3,-2) node[midway,fill=white]{$r_2$}
    --(0,0) node[midway,fill=white]{$r_1$};
    \shade[poscharge] circle (.2) node[white]{$q_1$};
    \shade[poscharge] (1,-2) circle (.2) node[white]{$q_3$};
    \shade[poscharge] (-3,-2) circle (.2) node[white]{$q_2$};
  \end{tikzpicture}
  \caption{A three-charge system}
  \label{fig:3charge}
\end{figure}
there are three separate two-charge systems ($q_1$ and $q_2$, $q_2$ and $q_3$,
and $q_1$ and $q_3$), therefore the total electric potential energy stored is
given by:
\begin{equation*}
  U_q=\frac{kq_1q_2}{r_1} +\frac{kq_2q_3}{r_2} +\frac{kq_1q_3}{r_3}
\end{equation*}
In this example,
$U_q$ may be positive or negative depending on the signs of $q_1$, $q_2$ and
$q_3$. We can repeat this same exercise for the \emph{gravitational} potential
energy stored between three \emph{masses}. The gravitational potential energy
will always be negative.

\begin{example}
  Four identical point charges $q$ are located at the four corners of a square
  of length $L$. What is the total potential energy of the system?
  \begin{center}
    \vspace{-.2in}
    \begin{tikzpicture}[scale=1.1]
      \draw rectangle (2,2) node[midway,above=31]{$L$};
      \begin{scope}[thick]
        \foreach \x in {0,2}{
          \foreach \y in {0,2}
          \shade[poscharge] (\x,\y) circle (.16) node[white]{$q$};
        }
      \end{scope}
    \end{tikzpicture}
  \end{center}
  \textbf{Solution:}
\end{example}



\section{Electric Potential}
An object at a specific location inside a gravitational field has a
gravitational potential energy proportional to its mass, i.e.\
\begin{equation}
  U_g=V_gm
\end{equation}
This ``constant'' $V_g$ is called the \textbf{gravitational potential}, which
is the \emph{gravitational potential energy per unit mass}. In the trivial case
with a uniform gravitational field:
\begin{equation}
  V_g=\frac{U_g}m=gh
\end{equation}
This also applies to the general case of the gravitational potential energy:
\begin{equation}
  V_g=\frac{U_g}m=-\frac{Gm}r
\end{equation}  
This is also true for a charged particle inside an electric field, and the
constant is called the \textbf{electric potential}, which is ``electric
potential energy per unit charge'':
\begin{important-equation}
  V=\frac{U_q}q
\end{important-equation}
For a point source charge $q_s$, it is defined as
\begin{important-equation}
  V=\frac{kq_s}r
\end{important-equation}

The unit for electric potential is a \textbf{volt}\footnote{Named after
Italian physicist Alessandro Volta} which is \emph{one joule per coulomb}:
\begin{equation*}
  \SI1\volt=\SI1{\joule\per\coulomb}
\end{equation*}

\begin{figure}[ht]
  \centering
  \begin{tikzpicture}
    \shade[poscharge] circle (.3) node[white]{$q$};
    \foreach \theta in {15,30,...,360}{
      \draw[rotate=\theta,axes,lightgray] (.3,0)--(2.7,0);
      \draw[rotate=\theta,thick,lightgray] (2.5,0)--(4,0);
    }
    \foreach \r in {1,2,3}{
      \draw[dashed,thick,red!70!black] circle (\r);
      \node[red!70!black,below] at (0,-\r){$V_\r$};
    }
  \end{tikzpicture}
  \caption{Electric field and equipotential contours from a positive charge}
  \label{fig:pt-charge-contours}
%    \vspace{-.4in}\begin{displaymath}
%      V_1>V_2>V_3
%    \end{displaymath}
\end{figure}

For a point charge $q$, every point at a distance $r$ will have the same
electric potential $V(r)$.
\begin{itemize}
\item The (dotted red) lines have the same electric potential; they are
  called \textbf{equipotential lines}, or \textbf{equipotential contours}
\item Equipotential lines are perpendicular to the electric field lines
\item Electric field lines \emph{always} points from higher $V$ towards
  lower $V$
\end{itemize}

%    \centering
%    \begin{tikzpicture}[scale=.8]
%      \shade[poscharge] circle (.3) node[white]{$q$};
%      \foreach \theta in {30,60,...,360}{
%        \draw[rotate=\theta,axes,gray] (.3,0)--(4,0);
%      }
%      \foreach \r in {1,2,3}{
%        \draw[dashed,very thick,red!70!black] circle (\r);
%        \node[red!70!black] at (0,-\r-.22){$V_\r$};
%      }
%      \shade[negcharge,rotate=-20] (2,0) circle (.23) node[white]{$Q$};
%    \end{tikzpicture}
%    
%    A charge $Q$ that is placed inside this electric field will now have an
%    electric potential energy of:
%
%    \eq{-.1in}{
%      U_q=QV=Q\left[\frac{kq}r\right]
%    }
%
%    in agreement with equation for electric potential energy


%\section{Getting Those Names Right}
%  \begin{itemize}
%  \item Electric potential energy between point charges:
%    
%    \eq{-.1in}{
%      U_q=\frac{kq_1q_2}r
%    }
%  \item Electric potential near a point charge:
%
%    \eq{-.1in}{
%      V=\frac{U_q}q=\frac{kq}r
%    }
%  \item Electric potential difference (voltage):
%
%    \eq{-.1in}{
%      \Delta V=\frac{\Delta U_q}q
%    }
%  \end{itemize}
%
%
%
%
%\section{Relating $U_q$, $\bm F_q$ and $\bm E$}
%  From the work-energy theorem:
%  \begin{itemize}  
%  \item Electrostatic force $\bm F_q$ always points from high to low potential
%    energy
%  \item Similarly, electric field $\bm E$ points from high to low electric
%    potential
%  \item Electric field can also be expressed as the change of electric
%    potential per unit distance, which has the unit
%    
%    \eq{-.1in}{
%      \SI1{\newton\per\coulomb}=\SI1{\volt\per\metre}
%    }
%  \item Electric field is also called ``electric potential gradient''
%  \end{itemize}
%
%
%
%
%\section{Example Problem}
%  \textbf{Example:} A small sphere with a charge of
%  $Q=\SI{-3.0}{\micro\coulomb}$ creates an electric field. Calculate the
%  electric potential at point $A$, located \SI{2.0}{\centi\metre} from the
%  source charge, and at point $B$, located \SI{5.0}{\centi\metre} from the same
%  source charge. Which point is at higher potential?
%  \begin{center}
%    \begin{tikzpicture}[scale=1.3]
%      \shade[poscharge] circle (.15) node[left=3]{$Q$};
%      \fill[red] (2,0) circle (.05) node[right,black]{$A$};
%      \draw[|<->|] (0,.2)--(2,.2)
%      node[midway,fill=white]{\SI{2.0}{\centi\metre}};
%      \begin{scope}[rotate=-28]
%        \fill[red] (5,0) circle (.05) node[right,black]{$B$};
%        \draw[|<->|] (0,-.2)--(5,-.2)
%        node[midway,rotate=-28,fill=white]{\SI{5.0}{\centi\metre}};
%      \end{scope}
%    \end{tikzpicture}
%  \end{center}


\begin{example}
  \label{example:V}
  The diagram below shows three charges, $q_A=\SI{5.0}{\micro\coulomb}$,
  $q_B=\SI{-7.0}{\micro\coulomb}$, and $q_C=\SI{2.0}{\micro\coulomb}$, placed
  at three corners of a rectangle. Point $D$ is the fourth corner. What is the
  electric potential at point $D$?
  \begin{center}
    \vspace{-.25in}
    \begin{tikzpicture}[scale=.9]
      \draw rectangle (4,3);
      \shade[poscharge] (0,3) circle (.15) node[left]{$A$};
      \shade[poscharge] (4,3) circle (.15) node[right]{$B$};
      \shade[poscharge] (4,0) circle (.15) node[right]{$C$};
      \fill circle (.08) node[left]{$D$};
      \draw[|<->|] (0,3.5)--(4,3.5)
      node[midway,fill=ocre!8]{\SI{4.0}{\centi\metre}};
      \draw[|<->|] (-.81,3)--(-.81,0)
      node[midway,fill=ocre!8]{\SI{3.0}{\centi\metre}};
    \end{tikzpicture}
  \end{center}
  \vspace{-.2in}
  \textbf{Solution:} The electric potential at $D$ is the sum of the potentials
  generated by charges $A$, $B$ and $C$. Luckily, this is a \emph{scalar}
  addition:
  \begin{equation*}
    V_\text{at D}=V_A + V_B + V_C =
    \frac{kq_A}{r_A}+\frac{kq_A}{r_B}+\frac{kq_A}{r_C} =
    k\left(\frac{q_A}{r_A}+\frac{q_A}{r_B}+\frac{q_A}{r_C}\right)
  \end{equation*}
  where $r_A=\SI{3.0}{cm}$, $r_B=\sqrt{3.0^2+4.0^2}=\SI{5.0}{cm}$ and
  $r_C=\SI{4.0}{cm}$ are the distances from $D$ to $A$, $B$ and $C$,
  respectively. Substituting numerical values, we have:
  \begin{equation*}
    V_\text{at D} =
    (\num{8.99e9})\times
    \left(\frac{5.0}{3.0}-\frac{7.0}{5.0}+\frac{2.0}{4.0}\right)
    \times\frac{10^{-6}}{10^{-2}}=\boxed{\SI{1.234e5}{\volt}}
  \end{equation*}
\end{example}
But what does the answer to Example~\ref{example:V} actually mean? In the
original configuration, there is already electric potential energy stored
between charges $A$ and $B$, between $B$ and $C$, and between $A$ and $C$.



\section{Electric Potential Difference}

\begin{figure}[ht]
  \centering
  \begin{tikzpicture}
    \shade[poscharge] circle (.3) node[white]{$Q$};
    \foreach \theta in {15,30,...,360}{
      \draw[rotate=\theta,axes,lightgray] (.3,0)--(2.7,0);
      \draw[rotate=\theta,thick,lightgray] (2.5,0)--(4,0);
      %\draw[rotate=\theta,axes,lightgray] (.3,0)--(2,0);
      %\draw[rotate=\theta,thick,lightgray] (1.7,0)--({7-4.5*sin(\theta/2)},0);
    }
    \begin{scope}[dashed,thick,red!70!black]
      \draw[rotate=-70] (2,0) arc (0:100:2) node[above]{$V_2$};
      \draw[rotate=-60] (3.7,0) arc (0:90:3.7) node[above]{$V_1$};
    \end{scope}
    \begin{scope}[rotate=-30]
      \shade[negcharge] (2,0) circle (.2) node[white]{$q$};
      \draw[<->,thick] (.3,0)--(1.8,0) node[midway,fill=white,sloped]{$r_2$};
    \end{scope}
    \begin{scope}[rotate=15]
      \shade[negcharge] (3.7,0) circle (.2) node[white]{$q$};
      \draw[<->,thick] (.3,0)--(3.5,0) node[pos=2/3,fill=white,sloped]{$r_1$};
    \end{scope}
  \end{tikzpicture}
\end{figure}
When a charge is moved from $r_1$ to $r_2$, the change in electric potential
energy is related to the change in electric potential by:
\begin{equation}
  \Delta U_q=U_2-U_1=Q\Delta V
\end{equation}
where $\Delta V$ is called the \textbf{potential difference}.
  


%\section{Electric Potential Difference (Voltage)}
Movement of the charged particle inside the electric field changes the
stored electric potential, i.e.\ \textbf{electric potential difference}. In
circuits, it is referred to as the \textbf{voltage}:
\begin{important-equation}
  \Delta V=\frac{\Delta U_q}q
\end{important-equation}
We can relate $\Delta V$ to an equation that we knew from Grade 11 Physics.
The energy dissipated in a resistor in a circuit $\Delta U_q$ to the voltage
drop $\Delta V$ is given by:
\begin{equation}
  \Delta U_q=Q\Delta V
\end{equation}






%\section{Field Structure}
%
%\section{Field Structure}
%  How should we draw the field lines?
%  \begin{center}
%    \pic{.3}{graphics/plate1}
%    \pic{.3}{graphics/plate2}
%  \end{center}
%
%
%
%
%\section{Field Structure}
%  \centering
%  \pic{.7}{graphics/plate3}
%
%
%
%
\section{Electric Field between Two Parallel Plates}

\begin{center}
  \begin{tikzpicture}[xscale=.8]
    \draw[thick] rectangle (8,.1);
    \draw[thick] (0,1.4) rectangle (8,1.5);
    \foreach \x in {.4,.8,1.2,...,7.6}{
      \draw[axes] (\x,1.4)--(\x,.7);
      \draw[thick] (\x,1.4)--(\x,.1);
    }
    \foreach \x in {.8,1.6,2.4,...,7.2}{
      \node at (\x,1.78) {$+$};
      \node at (\x,-.28) {$-$};
    }
    
    \draw[axes] (0,1.4) ..controls(-.15,1.2).. (-.2,.7);
    \draw[thick] (-.2,.8) ..controls(-.15,.4).. (0,.1);
    
    \draw[axes] (8,1.4)..controls(8.15,1.2)..(8.2,.7);
    \draw[thick] (8.2,.8)..controls(8.15,.4)..(8,.1);
  \end{tikzpicture}
\end{center}
The electric field $\bm E$ between two charged parallel plates is uniform at
all points between the parallel plates, independent of position. The magnitude
is:
\begin{important-equation}
  E=\frac\sigma{\epsilon_0}\quad\quad\text{(parallel plates)}
\end{important-equation}
where $\sigma=Q/A$ is the \textbf{surface charge density}, and
%$\epsilon_0=\SI{8.85e-12}{\coulomb\squared\per\newton.\metre\squared}$ is a
%universal constant called the \textbf{permittivity of free space}.
$\bm E$ outside the plates is nearly zero, except for the fringe effects
at the edges of the plates.

Between parallel plates, there is a simple relationship between electric
field, voltage, and the distance between the plates:
\begin{important-equation}
  E=\frac{\Delta V}d\quad\quad\text{(parallel plates)}
\end{important-equation}
%\begin{center}
%  \begin{tabular}{l|c|c}
%    \rowcolor{pink}
%    \textbf{Quantity} & \textbf{Symbol} & \textbf{SI Unit} \\ \hline
%    Electric field strength       & $E$ & \si{\newton\per\coulomb} \\
%    Electric potential difference & $\Delta V$ & \si\volt \\
%    Distance between parallel plates & $d$     & \si\metre
%  \end{tabular}
%\end{center}




%%\section{Example Problem}
%%  \textbf{Example}: An identical pair of metal plates is mounted parallel on
%%  insulating stands \SI{20}{\centi\metre} apart and equal amounts of opposite
%%  charges are placed on the plates. The electric field intensity at the
%%  midpoint between the plates is \SI{400}{\newton\per\coulomb}.
%%  \begin{enumerate}
%%  \item What is the electric field intensity at a point \SI{5.}{\centi\metre}
%%    from the positive plate?
%%  \item If the same amount of charge was placed on plates that have twice the
%%    area and are \SI{20}{\centi\metre} apart, what would be the electric field
%%    intensity at the point \SI{5.}{\centi\metre} from the positive plate?
%%  \item What would be the electric field intensity of the original plates if
%%    the distance of separation of the plates was doubled?
%%  \end{enumerate}
%%
%
%
%
%%\section{Example Problem}
%%  \textbf{Example 7:} Two parallel plates \SI{5.}{cm} apart are oppositely
%%  charged. The electric potential difference across the plates is \SI{80.}{V}.
%%  \begin{itemize}
%%  \item What is the electric field intensity between the plates?
%%  \item What is the potential difference at point A?
%%  \item What is the potential difference at point B?
%%  \item What is the potential difference between points A and B?
%%  \item What force would be experienced by a small \SI{2.}{\micro C} charge
%%    placed at point A?
%%  \end{itemize}

