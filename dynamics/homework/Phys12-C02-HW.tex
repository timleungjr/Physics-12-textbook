\section{Problem Set}

\begin{enumerate}[itemsep=4pt]
%  \question Which of the following involves a net force?
%  \begin{enumerate}[nosep,label=\Roman*.]
%  \item A ball on the end of a string travels in circular motion.
%  \item A space probe travels with a constant velocity in a straight line
%    between planets.
%  \item An object has a constant horizontal velocity, but a decreasing
%    vertical velocity.
%  \end{enumerate}
%  \begin{choices}
%    \choice I only
%    \choice I and II only
%    \choice II and III only
%    \choice I and III only
%    \choice I, II, and III
% \end{choices}
%
%  \question A small moving block collides with a large block at rest. Which of
%  the following is true of the forces the blocks apply to each other?
%  \begin{choices}
%    \choice The small block exerts twice the force on the large block 
%    compared to the force the large block exerts on the small block.
%    \choice The small block exerts half the force on the large block
%    compared to the force the large block exerts on the small block.
%    \choice The small block exerts exactly the same amount of force on the large
%    block that the large block exerts on the small block.
%    \choice The large block exerts a force on the small block, but the small
%    block does not exert a force on the large block.
%    \choice The small block exerts a force on the large block, but the large
%    block does not exert a force on the small block.
%  \end{choices}
%  
%  \question A force of magnitude $F$ pulls up at an angle $\theta$ to the
%  horizontal on a block of mass $m$. The mass remains in contact with the level
%  floor and the coefficient of friction between the block and the floor is
%  $\mu$. The frictional force between the floor and the block is
%  \begin{center}
%    \begin{tikzpicture}[scale=.7]
%      \fill[pattern=north east lines] rectangle (8,-.3);
%      \draw[very thick] (0,0)--(8,0);
%      \draw[mass] (3,0) rectangle (5,1.5);
%      \fill (4,.75) circle (.1);
%      \draw[dashed] (4,.75)--(7,.75);
%      \draw[vectors] (4,.75)--(6,2.75) node[right]{$F$};
%      \draw[axes] (5.5,.75) arc (0:45:1.5) node[midway,right]{$\theta$};
%    \end{tikzpicture}
%  \end{center}
%  \begin{choices}
%    \choice$\mu mg$
%    \choice$\mu(mg-F\sin\theta)$
%    \choice$\mu(mg+F\sin\theta)$
%    \choice$\mu(mg-F\cos\theta)$
%    \choice$\mu(mg+F\cos\theta)$
%  \end{choices}
%  \newpage
%  
%  \question A 1 kg block is sliding up a \ang{30} incline and is slowing down
%  with an acceleration of \SI{-6}{\metre\per\second\squared}. The mass
%  encounters a frictional force $f$ and a normal force $N$. Which of the
%  following free body diagrams best represents the forces acting on the block?
%  \begin{center}
%    \begin{tikzpicture}[scale=1.25]
%      \begin{scope}[rotate=-30]
%        \draw[thick] (0,0)--(-4,0);
%        \draw[mass] (-1,0) rectangle (-1.7,.7);
%        \draw[vectors] (-1.8,.35)--+(-.8,0) node[above]{$v$};
%      \end{scope}
%      \draw[thick] (0,0)--(-3.464,0)--(-3.464,2);
%      \draw[axes] (-1.2,0) arc (180:150:1.2) node[midway,left]{\ang{30}};
%    \end{tikzpicture}
%  \end{center}
%
%  \vspace{-.15in}A.\begin{tikzpicture}
%    \fill circle (.08);
%    \draw[vectors] (0,0)--(0,1) node[above]{$N$};
%    \draw[vectors] (0,0)--(0,-1) node[below]{$mg$};
%    \draw[vectors,rotate=60] (0,0)--(0,1) node[left]{$f$};
%  \end{tikzpicture}
%  \hspace{.2in}B.\begin{tikzpicture}
%    \fill circle (.08);
%    \draw[vectors,rotate=-30] (0,0)--(0,1) node[above]{$N$};
%    \draw[vectors,rotate=-30] (0,0)--(0,-1) node[below]{$mg$};
%    \draw[vectors,rotate=60] (0,0)--(0,1) node[left]{$f$};
%  \end{tikzpicture}
%  \hspace{.2in}C.\begin{tikzpicture}
%    \fill circle (.08);
%    \draw[vectors,rotate=-30] (0,0)--(0,1) node[above]{$N$};
%    \draw[vectors] (0,0)--(0,-1) node[below]{$mg$};
%    \draw[vectors,rotate=60] (0,0)--(0,-1) node[right]{$f$};
%  \end{tikzpicture}
%  \hspace{.2in}D.\begin{tikzpicture}
%    \fill circle (.08);
%    \draw[vectors] (0,0)--(0,1) node[above]{$N$};
%    \draw[vectors] (0,0)--(0,-1) node[below]{$mg$};
%    \draw[vectors,rotate=60] (0,0)--(0,-1) node[right]{$f$};
%  \end{tikzpicture}
%  \hspace{.2in}E.\begin{tikzpicture}
%    \fill circle (.08);
%    \draw[vectors,rotate=150] (0,0)--(0,1) node[left]{$N$};
%    \draw[vectors] (0,0)--(0,-1) node[below]{$mg$};
%    \draw[vectors,rotate=60] (0,0)--(0,-1) node[right]{$f$};
%  \end{tikzpicture}
%
%  \question In the previous question, the magnitude of the frictional force
%  $f$ between the block and the plane is most nearly
%  \begin{choices}
%    \choice\SI1\newton
%    \choice\SI2\newton
%    \choice\SI3\newton
%    \choice\SI4\newton
%    \choice\SI5\newton
%  \end{choices}
%  
%%  \question Two blocks, 4 kg and 2 kg, are connected by a string. An applied
%%  force $\vec F$ of magnitude 18 N pulls the blocks to the left. The
%%  acceleration of the 4 kg block is
%%  \begin{center}
%%    \begin{tikzpicture}[scale=.95]
%%      \fill[pattern=north east lines] (1.5,0) rectangle (9,-.3);
%%      \draw[very thick] (1.5,0)--(9,0);
%%      \draw[very thick] (7,0) rectangle (8,1) node[midway]{2 kg};
%%      \draw[very thick] (4,0) rectangle (6,1) node[midway]{4 kg};
%%      \draw[very thick] (6,.5)--(7,.5);
%%      \draw[very thick,->] (4,.5)--(2.5,.5) node[left]{$\vec F$};
%%    \end{tikzpicture}
%%  \end{center}
%%  \begin{choices}
%%    \choice\SI{2.0}{\metre\per\second\squared}
%%    \choice\SI{3.0}{\metre\per\second\squared}
%%    \choice\SI{4.0}{\metre\per\second\squared}
%%    \choice\SI{4.5}{\metre\per\second\squared}
%%    \choice\SI{6.0}{\metre\per\second\squared}
%%  \end{choices}
%%  
%%  \question In the previous question, the tension in the string between the
%%  blocks is
%%  \begin{choices}
%%    \choice 4.0 N
%%    \choice 6.0 N
%%    \choice 12 N
%%    \choice 16 N
%%    \choice 18 N
%%  \end{choices}
%
%  \question A weight of magnitude $W$ is suspended in equilibrium by two cords,
%  one horizontal and one slanted at an angle of \ang{60} from the horizontal, as
%  shown. The tension in the horizontal cord is \underline{\hspace{1in}}
%  \begin{center}
%    \begin{tikzpicture}
%      \draw[ultra thick,brown] (0,-1.5)--(1.5,-1.5)--(1.5,-2.5);
%      \draw[ultra thick,brown] (1.5,-1.5)--(4.5,0);
%      \fill[pattern=north east lines] (0,0)--(5,0)--(5,.2)--(-.2,.2)--
%      (-.2,-3)--(0,-3)--cycle;
%      \draw[thick] (0,-3)--(0,0)--(5,0);
%      \draw[thick,dashed] (1.5,0)--(1.5,-1.5)--(5,-1.5);
%      \fill (1.5,-1.5) circle (.07);
%      \draw[mass] (1.2,-2.5) rectangle (1.8,-3.1) node[midway]{$W$};
%      \draw[axes] (3,-1.5) arc (0:27:1.5) node[midway,right]{\ang{60}};
%    \end{tikzpicture}
%  \end{center}
%  \begin{choices}
%    \choice equal to the tension in the slanted cord
%    \choice one-third as much as the tension in the slanted cord
%    \choice one-half as much as the tension in the slanted cord
%    \choice twice as much as the tension in the slanted cord
%    \choice three times as much as the tension in the slanted cord
%  \end{choices}
%  \newpage
%  
%  \question Three blocks of mass 3 kg, 2 kg, and 1 kg are pushed along a
%  horizontal frictionless plane by a force of 24 N to the right, as shown
%  below. The acceleration of the 2 kg block is
%  \begin{center}
%    \begin{tikzpicture}[scale=.9]
%      \fill[pattern=north east lines]  rectangle (8,-.3);
%      \draw[thick] (0,0)--(8,0);
%      \draw[mass] (2,0) rectangle (4,1.5) node[midway]{3 kg};
%      \draw[thick,fill=cyan!20] (4,0) rectangle (5.5,1.3) node[midway]{2 kg};
%      \draw[thick,fill=cyan!10] (5.5,0) rectangle (6.5,1.1) node[midway]{1 kg};
%      \draw[vectors] (0,.75)--(2,.75) node[midway,above]{24 N};
%    \end{tikzpicture}
%  \end{center}
%  \begin{choices}
%  \choice\SI{144}{\metre\per\second\squared}
%  \choice\SI{72}{\metre\per\second\squared}
%  \choice\SI{12}{\metre\per\second\squared}
%  \choice\SI{6}{\metre\per\second\squared}
%  \choice\SI{4}{\metre\per\second\squared}
% \end{choices}
%  
%  \question In the previous question, the force that the 2 kg block exerts on
%  the 1 kg block is
%  \begin{choices}
%    \choice 2 N
%    \choice 4 N
%    \choice 6 N
%    \choice 24 N
%    \choice 144 N
%  \end{choices}
%  
%  %\question What is the acceleration of the system shown below if both blocks
%  %have a mass of \SI{5.0}{\kilo\gram}, and the coefficient of kinetic
%  %friction is $0.11$? What is the tension in the rope?
%  %\begin{enumerate}
%  %  \item Determine the tension in the rope
%  %  \item Determine the acceleration of both blocks
%  %\end{enumerate}

\item A hockey stick exerts an average force of \SI{39}{\newton} on a
  \SI{.20}{\kilo\gram} hockey puck over a distance of \SI{.22}\metre. If the
  hockey puck started from rest, what is the final velocity of the puck? 
  Assume that the friction between the puck and the ice is negligible. 

\item A curling stone with mass \SI{20.0}{\kilo\gram} leaves the curler's hand
  at a speed of \SI{.885}{\metre\per\second}. It slides \SI{31.5}{\metre} down
  the rink before coming to rest. 
  \begin{enumerate}[itemsep=4pt]
  \item Draw a free-body diagram of the curling stone after it leaves the
    curler's hand
  \item Find the average force of friction acting on the stone
  \item Find the coefficient of kinetic friction between the ice and the stone
  \item How far would the curling stone travel if its mass was reduced to
    \SI{15.0}{\kilo\gram}, if the initial velocity is the same?
  \end{enumerate}
  (Hint: To do the least amount of work, solve the problem algebraically first,
  and then substitute numerical values at the end.)

\item A skier coasts down a \ang{3.5} slope at constant speed.
  \begin{enumerate}[itemsep=4pt]
    \item Draw a free-body diagram of the skier. The skier should be
    represented by a dot. All forces (not components) should be drawn as arrow
    that originate at, and pointing away from, the dot.
    \item Find the coefficient of kinetic friction between the skis and the snow
    covering the slope.
  \end{enumerate}
  (Hint: If you solve the problem algebraically first, \emph{most} of the
  variables will cancel out, leaving you with a \emph{very} simple expression.
  That's the time to substitute numerical values for your final answer.)

%%\item A pulley device is used to hurl projectiles from a ramp ($\mu_k=0.26$).
%%  Mass A (\SI{5.}{kg}) is accelerated from rest at the bottom of the
%%  \SI{4.}{m}-long
%%  ramp by mass B, a falling \SI{20.}{kg} mass suspended over a frictionless
%%  pulley.
%%  Just as A reaches the top of the ramp, it detaches from the rope (neglect the
%%  mass of the rope) and becomes projected from the ramp.
%%  \begin{enumerate}[noitemsep,topsep=0pt,leftmargin=18pt]
%%  \item Draw free-body diagrams for both masses. (Draw forces directly on the
%%    diagram.)
%%  \item Determine the acceleration of A along the ramp.
%%  \item Determine the tension in the rope during the acceleration of A along
%%    the ramp.
%%  \item Determine the speed of projection of A from the top of the ramp. 
%%  \item Determine the horizontal range of A from the base of the ramp.
%%  \end{enumerate}
%%  \pic{.45}{diagram2}
%  
%  \question One method to increase the storage space in a very small house is to
%  hang storage bins from the ceiling using ropes. In this example, a
%  \SI{26}{\kilo\gram} bin is hung from the ceiling using two ropes of different
%  tension, as shown in the diagram below. What is the tension in each of the
%  ropes? (Hint: Start with a free-body diagram at the junction between the
%  ropes.
%  \begin{center}
%    \begin{tikzpicture}[scale=1.2]
%      \draw[mass] (-1,.7) rectangle (1,2) node[midway]{\SI{26}{\kilo\gram}};
%      \draw[ultra thick,brown] (0,2)--(0,2.5);
%      \draw[ultra thick,brown] (3,3.5)--(0,2.5)--(-1.5,3.5);
%      \fill[pattern=north east lines] (-3,3.5) rectangle (4,3.8);
%      \draw[very thick] (-3,3.5)--(4,3.5);
%      \fill (0,2.5) circle (.06);
%      \draw[axes] (2,3.5) arc (180:198.4:1) node[midway,left]{\ang{18.4}};
%      \draw[axes] (-0.5,3.5) arc (0:-33.7:1) node[midway,right]{\ang{33.7}};
%    \end{tikzpicture}
%  \end{center}

\item Two blocks of masses $m=\SI{.80}{\kilo\gram}$ and
  $M=\SI{2.0}{\kilo\gram}$ are connected by a massless string over a massless
  and frictionless pulley as shown in the diagram below. The blocks are
  released from rest at $t=0$.
  \begin{center}
    \begin{tikzpicture}
      \draw[thick,brown] (-3,.4)--(.1,.4);
      \draw[thick] (0,0)--(-5.5,0) node[midway,below]{$\mu=0.14$};
      \draw[mass] (-3,0) rectangle +(-1,.8) node[midway]{$m$};
      \begin{scope}[rotate=-30]
        \draw[thick,brown] (1,.4)--(-.05,.4);
        \draw[thick] (0,0)--(4.5,0)
        node[midway,below,rotate=-30] {$\mu=0.14$};
        \draw[thick,mass] (1,0) rectangle +(2,1) node[midway,rotate=-30]{$M$};
      \end{scope}
      \begin{scope}[rotate=-15]
        \draw[thick,fill=gray] (0,.3) circle (.15);
        \draw[thick,fill=gray!50] (0,.3) circle (.1);
        \draw[ultra thick] (0,0)--(0,.3);
        \fill (0,.3) circle (.04);
      \end{scope}
      \draw[thick,gray!70] (0,0)--(0,-1.5);
      \draw[axes] (0,-.5) arc (270:330:.5) node[midway,below]{$\phi=\ang{60}$};
    \end{tikzpicture}
  \end{center}
  \begin{enumerate}[itemsep=4pt]
  \item Draw free-body diagrams of the blocks as they move.
  \item Determine the magnitude of acceleration of the blocks.
  \item Calculate the magnitude of the tension force in the string.
%  %\item If the string broke, for what minimum value of the coefficient of
%  %static friction would the \SI{2.0}{\kilo\gram} block not begin to slide?
  \end{enumerate}
%  (Some hints and FYI: The pulley \emph{must} be frictionless and massless,
%  otherwise, when the blocks are released from rest, the pulley will have
%  rotational kinetic energy, and the tension will not be constant throughout.
%  Both blocks will have the same magnitude of acceleration. To solve the
%  problem, apply the second law of motion to each block, and the two unknowns
%  that you need to solve are acceleration an tension. It does not matter which
%  quantity you solve first.)

\item In the figure below, the blocks do not slide against each other when
  horizontal external force $\vec F$ is applied to $m_3$. Assume that there is
  no friction at the contact between the blocks and the table. (Note: Without
  this external force $\vec F$, $m_2$ would accelerate downwards, while $m_1$
  would accelerate towards the right.)
  %With the help of free-body diagrams on
  %each mass, determine the magnitude of $\vec F$. 
  \begin{center}
    \begin{tikzpicture}[scale=1.3]
      \draw[very thick] (-3,0)--(3,0);
      \fill[pattern=north east lines] (-3,0) rectangle (3,-.3);
      \begin{scope}[thick]
        \draw[mass] (-1.7,0) rectangle (0,1.2) node[midway]{$m_3$};
        \draw[fill=gray!70] (-1.7,1.2) rectangle (-.8,1.9) node[midway]{$m_1$};
        \draw[fill=gray!70] (0,.3) rectangle (.6,.78) node[midway]{$m_2$};
      \end{scope}
      \begin{scope}[ultra thick,brown]
        \draw (-.8,1.55)--(.1,1.55);
        \draw (.35,1.2)--(.35,.78);
      \end{scope}
      \begin{scope}[very thick]
        \draw[fill=lightgray] (.1,1.3) circle (.3);
        \draw[fill=gray] (.1,1.3) circle (.2);
      \end{scope}
      \fill (.1,1.3) circle (.05);
      \draw[vectors] (-2.6,.6)--(-1.7,.6) node[pos=0,left]{$\vec F$};
    \end{tikzpicture}
    
    $m_1=\SI{1.8}{\kilo\gram}$, $m_2=\SI{1.2}{\kilo\gram}$,
    $m_3=\SI{3.0}{\kilo\gram}$
  \end{center}
  \begin{enumerate}[itemsep=4pt]
  \item Draw free-body diagrams for each of the blocks. The block should be
    presented by a dot. Forces should be drawn as arrows originating at, and
    pointing away from, the dot that represent the object.

%    \uplevel{
%      Hints: For parts (b) and (c), you will only need to use the free-body
%      diagrams of two of the 3 blocks. If the blocks don't slide against each
%      other, they will all have the same acceleration vector.)
%    }
  \item Calculate the acceleration of the blocks (both magnitude and direction)
    when the force is applied.
  \item Calculate the magnitude of the force applied.
  \end{enumerate}

\item In a tractor-pull competition, a tractor applies a force of
  \SI{1.3}{\kilo\newton} to the sled, which has mass \SI{1.1e4}{\kilo\gram}. At
  that point, the coefficient of kinetic friction between the sled and the
  ground has increased to 0.80. What is the acceleration of the sled? Explain
  the significance of the sign of the acceleration. 
  
%%  \question A solo Arctic adventurer pulls a string of two toboggans of supplies
%%  across level, snowy ground. The toboggans have masses of \SI{95}{\kilo\gram}
%%  and \SI{55}{\kilo\gram}. Applying a force of \SI{165}{\newton} causes the
%%  toboggans to accelerate at \SI{.61}{\metre\per\second\squared}.
%%  \begin{enumerate}
%%    \item Calculate the frictional force acting on the toboggans. 
%%    \item Find the tension in the rope attached to the second
%%    (\SI{55}{\kilo\gram}) toboggan.
%%  \end{enumerate}

\item A \SI{6.8}{\kilo\gram} bicycle, with a cyclist of mass
  \SI{63.5}{\kilo\gram}, are travelling at \SI{45.4}{\kilo\metre\per\hour}
  during a bike race, when the cyclist sees a crash ahead. To avoid the crash,
  he applies the brakes, locking the wheels.
  \begin{enumerate}[itemsep=4pt]
  \item How far does the bike travel if the coefficient of kinetic friction
    between the tires and the road is 0.60?
  \item How far would the bike travel if the mass of the cyclist is only
    \SI{40.2}{\kilo\gram}?
  \end{enumerate}

%%  \question A \SI{3.0}{\kilo\gram} counterweight is connected to a
%%  \SI{4.5}{\kilo\gram} window that freely slides vertically in its frame. How
%%  much force must you exert to start the window opening with an acceleration of
%%  \SI{.25}{\meter\per\second\squared}.

%%  %\uplevel{
%%  %  \pic{.35}{../graphics/pulley-a-b}
%%  %}
%%  %\question A rope of negligible mass passes over a pulley of negligible
%%  %mass attached to the ceiling, as shown above. One end of the rope is held by
%%  %Student $A$ of mass \SI{70}{\kilo\gram}, who is at rest on the floor. The
%%  %opposite end of the rope is held by Student $B$ of mass \SI{60}{\kilo\gram},
%%  %who is suspended at rest above the floor.
%%  %\begin{enumerate}
%%  %  \item On the dots below that represent the students, draw and label
%%  %  free-body diagrams showing the forces on Student $A$ and on Student $B$.
%%  %
%%  %  \begin{center}
%%  %    \begin{tikzpicture}
%%  %      \fill circle (.08) node[right]{$\;A$};
%%  %      \fill (1,2.5) circle (.08) node[right]{$\;B$};
%%  %    \end{tikzpicture}
%%  %  \end{center}

%%  %  \item Calculate the magnitude of the force exerted by the floor on Student
%%  %  $A$.
%%  %  
%%  %  \uplevel{
%%  %    Student $B$ now climbs up the rope at a constant acceleration of
%%  %    \SI{.25}{\metre\per\second\squared} with respect to the floor.
%%  %  }
%%  %  
%%  %  \item Calculate the tension in the rope while Student B is accelerating.
%%  %
%%  %  \item As Student $B$ is accelerating, is Student $A$ pulled upward off the
%%  %  floor? Justify your answer.
%%  %
%%  %  \item With what minimum acceleration must Student $B$ climb up the rope to
%%  %  lift Student $A$ upward off the floor?  
%%  %\end{enumerate}
\end{enumerate}
