\chapter{Gravity and Planetary Motion}
\label{chapter:gravity}

\section{Law of Universal Gravitation}
In classical/Newtonian physics, \textbf{gravity} is the mutual attraction
between all massive objects, as shown in Fig.~\ref{fig:gravity1}.
\begin{figure}[ht]
  \centering
  \begin{tikzpicture}[scale=.65]
    \draw[vectors,red] (0,0)--(2,0) node[right]{$\bm F_g$};
    \draw[vectors,blue] (8,0)--(6,0) node[left]{$\bm F_g$};
    \shade[balloon1] circle (.4) node[white]{$m_1$};
    \shade[balloon2] (8,0) circle (1) node[white]{$m_2$};
    \draw[dashed] (0,0)--(0,-1.5);
    \draw[dashed] (8,0)--(8,-1.5);
    \draw[<->] (0,-1.3)--(8,-1.3) node[midway,fill=white]{$r$};
  \end{tikzpicture}
  \caption{Two massive objects ($m_1$ and $m_2$) applying a gravitational force
    on each other.}
  \label{fig:gravity1}
\end{figure}

The magnitude of \textbf{gravitational force} ($\bm F_g$)  between two
masses is proportional to their masses ($m_1$, $m_2$), and inversely
proportional to the square of the distance ($r$) between them:
\begin{equation}
  \boxed{
    F_g=\frac{Gm_1m_2}{r^2}
  }
  \label{eq:law-of-gravity}
\end{equation}
where $G=\SI{6.67e-11}{N.m^2/kg^2}$ is the \textbf{universal gravitational
  constant}. This equation is derived based on the second law of motion and
Kepler's law of planetary motion (discussed later in this chapter). 

Like \emph{all} forces, gravity obey the third law of motion: if $m_1$ exerts a
force $\bm F_g$ on $m_2$, then $m_2$ also exerts a force $-\bm F_g$ on
$m_1$. The attractive forces are equal in magnitude and opposite in direction
(i.e.\ third law of motion). The masses $m_1$ and $m_2$ are assumed to be
\emph{point masses} that do not occupy any space. For objects with
\emph{spatial extend} (i.e.\ the mass actually takes up space), we assume that
one object is not inside the other.

This equation does not work for extremely massive objects (e.g.\ black holes
and neutron stars), and the law must be replaced by the theory of general
relativity. It is also unclear as to whether the law applies to
extremely small masses (e.g.\ elementary particles like protons, electrons and
neutons). However, aside from the extreme, using Eq.~\ref{eq:law-of-gravity} is
generally acceptable.
\begin{common-question}
  \textbf{What happens when $r=0$?} From Eq.~\ref{eq:law-of-gravity}, it
  appears that when $r\rightarrow 0$, $F_g\rightarrow\infty$. However, it is
  worth pointing out that point masses do not actually exist (even an electron
  has a finite size). Therefore the situation for $r=0$ never arises; $F_g$ is
  never be infinite. From a more practical point of view, whether a mass can be
  treated as a point mass is a matter of \emph{scale}. For example, at the
  scale of the solar system, the Sun and all the planets, moons, comets can all
  be considered as point masses, but near the surface of an
  irregularly-shaped asteroid, the point mass model would be very inaccurate.
\end{common-question}




\begin{example}
  A \SI{65.0}{\kilo\gram} astronaut is walking on the surface of the moon,
  which has a mean radius of \SI{1.74e3}{\kilo\metre} and a mass of
  \SI{7.35e22}{\kilo\gram}. What is the weight of the astronaut?

  \textbf{Solution:} When the astronaut walks on the moon, the
  distance between his centre of mass (CM) and the Moon's CM is just the
  radius of the Moon.
  \begin{equation*}
    F_g=\frac{Gm_1m_2}{r^2}=
  \end{equation*}
  We have previously known that the gravitational pull on the Moon is
  approximately $1/6$ that of Earth. This calculation confirms it.
\end{example}



\begin{example}
  How far apart would you have to place two
  \SI{7.0}{\kilo\gram} bowling balls so that the force of gravity between them
  would be \SI{1.25e-8}\newton? (Notice the magnitude of gravitational force
  between the two objects. In fact, gravitational force is the weakest of all
  fundamental forces.)

  \vspace{.15in}\textbf{Solution:}
  \begin{equation*}
    F_g=\frac{Gm_1m_2}{r^2}=\frac{Gm^2}{r^2}=\quad\rightarrow\quad
    r=\sqrt{\frac{Gm^2}{F_g}}=\left(\sqrt{\frac G{F_g}}\right)m=
  \end{equation*}
\end{example}



\begin{figure}[ht]
  \centering
  \begin{tikzpicture}[scale=.4]
    \shade[ball color=red] circle (.7) node[white]{$M$};
    
    \draw[vectors,blue,rotate=45] (.7,0)--(2.5,0) node[right]{$\bm F_1$};
    \shade[ball color=blue] (5,5) circle (.8) node[white]{$m_1$};
    
    \draw[vectors,green,rotate=-45] (-.7,0)--(-2,0) node[left]{$\bm F_2$};
    \shade[ball color=green] (-5,5) circle (.6) node[white]{$m_2$};
    
    \draw[vectors,violet,rotate=45](-.7,0)--(-3.5,0) node[left]{$\bm F_3$};
    \shade[ball color=violet] (-5,-5) circle (1.2) node[white]{$m_3$};

    \draw[vectors,magenta,rotate=-45](.7,0)--(2.7,0) node[right]{$\bm F_4$};
    \shade[ball color=magenta] (5,-5) circle (.8) node[white]{$m_4$};
    
    %\draw[vectors] (.7,0)--(3.5,0) node[left]{$\bm F$};
  \end{tikzpicture}
\end{figure}

For a mass $M$ subjected to the influence of multiple discrete point masses
$m_i$, the total gravitational force on $M$ is the vector sum of all the
forces $\bm F_i$:
\begin{equation}
  \boxed{
    \bm F=\sum_i\bm F_i
  }
\end{equation}
\begin{itemize}
\item Even if $m_1,\ldots,m_4$ are all stationary, as $M$ moves, its
  acceleration would not be constant
\item It is generally difficult to describe the motion of $M$ as a function
  of time, even with calculus
\end{itemize}
  



\section{Gravitational Field}
We usually describe gravitational force using the familiar equation:
\begin{equation*}
  \bm F_g=m\bm g
\end{equation*}
To find $g$, we can group together the variables in the law of
universal gravitation. For the gravitational force acting on $m_2$, we have:
\begin{equation} 
  F_g=\underbracket[1pt]{\left[\frac{Gm_1}{r^2}\right]}_{=g}m_2=m_2g
\end{equation}
On or near Earth's surface, we use $m_1=m_\text{Earth}$ and $r=r_\text{Earth}$
to compute $g=\SI{9.81}{\metre\per\second\squared}$, or
$g=\SI{9.81}{\newton\per\kilo\gram}$ (both units are equivalent)




%\section{Gravitational Field}
The \textbf{gravitational field} ($\bm g$) generated by a source point mass
($m_s$) is a measure of how it influences the gravitational forces on other
masses in its vicinity.
%The \textbf{gravitational field} $\bm g$ is a function of a source mass
%$m_s$ and the distance $r$ from it;
Its magnitude\footnote{Also referred to as its \emph{strength} or
\emph{intensity} of the field} is given as a function of two variables:
\begin{equation}
  \boxed{
    g(m_s,r)=\frac{Gm_s}{r^2}}
\end{equation}
The direction of the gravitational field is \emph{towards} the source mass
that created it.
\begin{remark}%sidenote}
  For those with a strong background in vectors, we can directly express
  gravitational field in vector form:
  \begin{equation*}
    \boxed{
      \bm g(m_s,\bm r)=-\frac{Gm_s}{|\bm r|^2}\hat{\bm r}
    }
  \end{equation*}
  where $\hat{\bm r}$ is the \emph{outward radial direction}. Therefore,
  the direction of the field is the inward radial direction $-\hat{\bm r}$,
  i.e.\ the gravitational field vector points towards the source mass that
  created it, as stated above.
\end{remark}%sidenote}
%\begin{center}
%  \begin{tabular}{l|c|c}
%    \rowcolor{pink}
%    \textbf{Quantity} & \textbf{Symbol} & \textbf{SI Unit} \\ \hline
%    Gravitational field strength & $g$ & \si{\newton\per\kilo\gram}\\
%    Universal gravitational constant & $G$ & \si{N.m^2/kg^2} \\
%    Source mass (a point mass) & $m_s$ & \si{\kilo\gram} \\
%    Distance from source mass & $r$ & \si\metre
%  \end{tabular}
%\end{center}



%\section{Relating Gravitational Field \& Gravitational Force}
When a mass $m$ is placed inside a gravitational field $\bm g$, it
experiences a gravitational force given by the familiar equation:
%$\bm g$ itself doesn't do anything unless another mass $m$ is inside this
%field. At which point, the other mass $m$ experiences a gravitational force
%related to $\bm g$ by:
\begin{equation*}
  \bm F_g=m\bm g
\end{equation*}
Regardless of what generated the gravitational field.



%\section
%  \centering
%  \begin{tikzpicture}
%    \draw[blue!70!black,mass] circle (.3) node[midway]{$M$};
%    \foreach \theta in {0,45,...,359}
%    \draw[vectors,blue!70!black,rotate=\theta] (2,0)--+(-1,0);
%    \foreach \theta in {0,30,...,359}
%    \draw[vectors,blue!70!black,rotate=\theta] (3,0)--+(-.4,0);
%    \node[right,blue!70!black] at (2,0) {$\bm g$};
%    \node[
%      text width=140,
%      draw=blue!70!black,
%      fill=blue!5,
%      text=blue!70!black] at (5.7,0) {Mass $M$ generates a gravitational field
%      that extends from the mass itself over the entire space, with a magnitude
%      of
%
%      \vspace{-.07in}
%      \begin{displaymath}
%        g=\frac{GM}{r^2}
%      \end{displaymath}
%      $\bm g$ points \emph{towards} the mass that generated it. This
%      gravitational field does not do anything until there is another mass
%      inside the field.\par
%    };
%    \uncover<2->{
%      \begin{scope}[orange,rotate=20]
%        \draw[vectors] (-3.5,0)--+(1.2,0) node[right=0]{$\bm F_g$};
%        \draw[thick,fill=orange!5] (-3.5,0) circle (.18) node{$m$};
%      \end{scope}
%      \node[
%        text width=220,
%        draw=orange,
%        fill=orange!5,
%        text=orange] at (-2.7,-3.2){Mass $m$ is inside the gravitational field
%        generated by {\color{blue!70!black}$M$}, therefore it experiences a
%        gravitational force of
%        
%        \vspace{-.08in}
%        \begin{displaymath}
%          \bm F_g=m{\color{blue!70!black} \bm g}
%        \end{displaymath}
%        The gravitational force is in the same direction as
%        {\color{blue!70!black}$\bm g$}\par
%      };
%    }
%  \end{tikzpicture}
%
%
%
%
\subsection{Multiple Gravitational Fields}
When there are multiple source masses, the total gravitational field is the
vector sum of all the gravitational fields from each source mass.
\begin{equation}
  \bm g =\bm g_1+\bm g_2+\bm g_3+\cdots=\sum_i \bm g_i
\end{equation}



\subsection{Gravitational Field Lines}

\begin{figure}[ht]
  \centering
  \begin{tikzpicture}[scale=1.5]
    \foreach \x in {0,20,...,359}{
      \begin{scope}[red,very thick,rotate=\x]
        \draw (0,0)--(0,1);
        \draw[<-] (0,0.95)--(0,1.5);
      \end{scope}
        \draw[mass] circle (.17) node{$m$};
    }
  \end{tikzpicture}
\end{figure}
\begin{itemize}
\item The direction of $\bm g$ is towards the centre of the object that
  created it
\item Field lines do not tell the intensity (i.e.\ magnitude) of $\bm g$,
  only the direction
\end{itemize}
%
%
%
%
%
%\section{Gravitational Field Lines}
%  When there are multiple masses, the total gravitational field (dotted line)
%  is the vector sum of all the individual fields.
%  \begin{center}
%    \pic{.4}{graphics/grav-fields}
%  \end{center}
%  The solid lines are called \textbf{equipotential lines}, where the potential
%  energy is constant. Equipotential lines are perpendicular to
%  gravitational field lines.




\section{Gravitational Potential Energy}
The expression for \textbf{gravitational potential energy} can be obtained
through the law of universal gravitation using basic integral calculus, by
calculating the work done by the gravitational force as it moves an object from
$r_1$ to $r_2$. Of course, the calculus is ``basic'' only if you already know
calculus, otherwise it is impossible:
\begin{equation*}
  \underbracket[1pt]{
    W_g = \int_{\bm r_1}^{\bm r_2}\bm F_g\cdot d\bm r
  }_\text{Definition using calculus}
  = \cdots = Gm_1m_2 \int_{r_1}^{r_2}\frac{dr}{r^2}
  =\frac{Gm_1m_2}{r_1} - \frac{Gm_1m_2}{r_2}=-\Delta U_g
\end{equation*}
But we can skip over to the final result: $U_g$ is the gravitational potential
energy stored between masses $m_1$ and $m_2$, defined as:
\begin{equation}
  \boxed{U_g=-\frac{Gm_1m_2}r}
  \label{eq:GPE-general}
\end{equation}
By definition, $U_g$ is the work done by gravity required to move two objects
from $r$ to $\infty$. To generalize the equation, the reference where $U_g=0$
is set at $r=\infty$, and $U_g$ \emph{decrease} as $r$ decreases.

%%  Since $g$ is not a constant, we use an equation consistent with the law of
%%  universal gravity to obtain the general expression for
%%  \textbf{gravitational potential energy} stored between a system of two
%%  masses:
%  
%  \eq{-.05in}{
%    \boxed{U_g=-\frac{Gm_1m_2}r}
%  }
%\begin{center}
%  \begin{tabular}{l|c|c}
%    \rowcolor{pink}
%    \textbf{Quantity} & \textbf{Symbol} & \textbf{SI Unit} \\ \hline
%    Gravitational potential energy & $U_g$ & \si\joule \\
%    Point masses & $m_1$, $m_2$ & \si{\kilo\gram} \\
%    Distance between centres of mass & $r$ & \si\metre \\
%    Universal gravitational constant & $G$ & \si{N.m^2/kg^2}
%  \end{tabular}
%\end{center}


%  %Similar to the simpler expression for gravitational potential energy
%  %($U_g=mgh$),


%  \eq{-.1in}{
%    \boxed{U_g=-\frac{Gm_1m_2}r}\quad\text{\normalsize and }
%    \quad\boxed{  W_g=-\Delta U_g}
%  }

Eq.~\ref{eq:GPE-general} also shows that the gravitational force is
conservative:

\begin{enumerate} %[itemsep=3pt,leftmargin=15pt]
\item When work done by gravtational force is \emph{positive} (i.e.\
  $W_g>0$), there is a \emph{decrease} in gravitational potential energy by
  the same amount ($\Delta U_g>0$), while
\item When work done by graviational force is \emph{negative} (i.e.\
  $W_g<0$), there is an \emph{increase} in gravitational potential energy by
  the same amount ($\Delta U_g>0$)
\item The work by gravity is \emph{path independent}: $W_g$, and therefore
  $\Delta U_g$, depends on the end points $r_1$ and $r_2$, but not \emph{how}
  the objects move from $r_1$ to $r_2$
  \item Only work done by gravity can affect $U_g$
\end{enumerate}

%\fcolorbox{black}{yellow!5}{
%  \begin{minipage}{.95\textwidth}
%    \begin{itemize}
%    \item \emph{Positive} work by $\bm F_g$ \emph{decreases} gravitational
%      potential energy $U_g$, while
%    \item \emph{Negative} work by $\bm F_g$ \emph{increases} gravitational
%      potential energy $U_g$
%    \item $W_g$ depends on $r_1$ and $r_2$, but not \emph{how} the mass moves
%      from $r_1\rightarrow r_2$
%    \item Only work done by $\bm F_g$ can change $U_g$
%    \end{itemize}
%  \end{minipage}
%}




%%\section{Relating Gravitational Potential Energy to Force}
%%  The work-energy theorem tells us that
%%  \begin{itemize}
%%  \item $\bm F_g$ always points in the direction from high to low potential
%%    energy, i.e.
%%  \item A falling object is always decreasing in $U_g$
%%  \item ``Steepest descent'': the direction of $\bm F_g$ is the shortest path
%%    to decrease $U_g$ 
%%  \item Objects travelling perpendicular to $\bm F_g$ has constant $U_g$
%%  \end{itemize}  
%%  In vector calculus, we say that gravitational force ($\bm F_g$) is the
%%  negative gradient of the gravitational potential energy ($U_g$):
%%  
%%  \eq{-.1in}{
%%    \bm F_g=-\nabla U_g=-\frac{dU_g}{dr}\hat{\bm{r}}
%%  }
%%
%
%
%
%%\section{Relating $U_g$, $\bm F_g$ and $\bm g$}
%%  Knowing that $\bm F_g$ and $\bm g$ only differ by a constant, we can
%%  also relate gravitational field to $U_g$ by the gradient operator:
%%
%%  \eq{-.1in}{
%%    \bm g=\frac{\bm F_g}{m}=-\nabla\left(\frac{U_g}{m}\right)=
%%    -\frac{d}{dr}\left(\frac{U_g}{m}\right)\hat{\bm{r}}
%%  }
%%
%%  We already know that the direction of $\bm g$ is the same as $\bm F_g$,
%%  i.e.
%%  \begin{itemize}
%%  \item The direction of $\bm g$ is the shortest path to decrease $U_g$ 
%%  \item Objects travelling perpendicular to $\bm g$ has constant $U_g$
%%  \end{itemize}
%%
%
%
%
%%\section
%%  We will now combine our knowledge of gravitational force and gravitational
%%  energy to understand motion of satellites, planets and stars.
%%
%
%
%
%\section{Kepler's Laws}
%
%%    \pic{1.1}{graphics/kepler}
%%      
%\begin{itemize}
%\item German mathematician, astronomer \& astrologer
%\item Formulated his laws based on the observation of his teacher, Tycho
%  Brahe
%\item Published his first two laws in 1609, third law in 1619
%\item His work was controversial because of another competing theory,
%  but by 1670, most scientists have accepted his findings
%\item Kepler's laws are considered to be \emph{empirical}:
%  \begin{itemize}
%  \item Based purely on observed data
%  \item Kepler had no physical theory
%  \item ``Fitting the curve'' to the data
%  \end{itemize}
%\end{itemize}
%  
%
%
%
%
%\section{Kepler's Laws of Planetary Motion}
%  \textbf{1.\ Law of Ellipses: The orbit of a planet is an ellipse with the Sun
%    at one of the two foci.}
%  \begin{center}
%    \pic{.35}{graphics/23kepler1}
%  \end{center}
%  This is an extraordinary claim, because it is different from those of
%  Nicolaus Copernicus, who claimed that the orbit of a planet is circular.
%
%
%
%
%\section{Kepler's Laws of Planetary Motion}
%  \textbf{2.\ Law of Equal Areas: A line segment joining a planet and the Sun
%    sweeps out equal areas during equal intervals of time.}
%  \begin{center}
%    \pic{.4}{graphics/201532-132212364-3243-planet}
%  \end{center}
%  The planet moves faster when it is closer to the Sun, and slower when it
%  farther.
%
%
%
%
%\section{Kepler's Laws of Planetary Motion}
%  \textbf{3.\ Law of Periods: The square of the orbital period of a planet is
%    proportional to the cube of the semi-major axis of its orbit.}
%
%  \vspace{.1in}For two different planets $A$ and $B$ orbiting the same sun:
%  
%  \eq{-.1in}{
%    \boxed{\frac{T^2}{r^3}=\text{constant}}\quad\text{or}\quad
%    \boxed{\frac{T_A^2}{r_A^3}=\frac{T_B^2}{r_B^3}}
%  }
%
%
%
%
%\section{Orbital Radii and Periods of Different Planets}
%  
%    \centering
%    \begin{tabular}{c|c|r}
%      \rowcolor{pink}
%      \textbf{Planet} &
%      \textbf{$R$ (AU)} &
%      \textbf{$T$ (days)} \\ \hline
%      Mercury & 0.389 & 87.77  \\
%      Venus   & 0.742 & 224.70 \\
%      Earth   & 1.000 & 365.25 \\
%      Mars    & 1.524 & 686.98 \\
%      Jupiter & 5.200 & 4332.62 \\
%      Saturn  & 9.150 & \num{10579.20}
%    \end{tabular}
%    
%    \centering
%    \pic{.8}{graphics/kep8}
%  
%  \vspace{.1in}\textbf{Astronomical unit} (AU) is defined as the average radius
%  of Earth's orbit
%
%
%
%
%\section{Kepler's Law of Planetary Motion}  
%    The elliptical orbits of most of the planets in the solar system have very
%    small eccentricity (i.e.\ their orbits are close to being circular) but
%    comets can have much higher eccentricity
%
%    \begin{tabular}{l|l}
%      \rowcolor{pink}
%      \textbf{Object} & $e$ \\ \hline
%      Mercury	& \num{.206} \\
%      Venus	& \num{.0068} \\
%      Earth	& \num{.0167} \\
%      Mars	& \num{.0934} \\
%      Jupiter	& \num{.0485} \\
%      Saturn	& \num{.0556} \\
%      Uranus	& \num{.0472} \\
%      Neptune	& \num{.0086} \\
%      Pluto	& \num{.25} \\ \hline
%      Halley's Comet   & \num{.9671} \\
%      Comet Hale-Bopp  & \num{.9951} \\
%      Comet Ikeya-Seki & \num{.9999}
%    \end{tabular}
%  
%
%
%
%%\section{Kepler's Laws of Planetary Motion}
%
%%  Also, Brahe's original observations deviated from Kepler's laws slightly,
%%  especially for Jupiter (Kepler knew that something is missing)


\section{Orbital Motion}
\label{sec:orbital-motion}

In \emph{Treatise of the System of the World}, the third book in
\emph{Principia}, Newton presented a thought experiment, shown in
Fig.~\ref{fig:thought-experiment}. The dashed line in the figures represent a
circular path that is concentric with the centre of Earth. A cannonball is
launched parallel to the surface of Earth with some initial velocity
$\bm v$.
\begin{figure}[ht]
  \centering
  \begin{subfigure}{.4\textwidth}
    \centering
    \begin{tikzpicture}
      \node at (0,0) {\pic{.45}{gravity/earth-1200px}};
      \fill[red] (0,1.7) circle (.05);
      \draw[function,->] (0,1.7) arc (90:30:1 and 1.45);
      %\draw[function,->,domain=0:.5] plot(\x,{-.5*\x*\x+1.7});
      \draw[dashed,gray,thick] circle (1.7);
    \end{tikzpicture}
    \caption{If initial velocity is too slow, the object falls back onto the
      planet after travelling a short distance}
  \end{subfigure}
  \hspace{.3in}
  \begin{subfigure}{.4\textwidth}
    \centering
    \begin{tikzpicture}
      \node at (0,0) {\pic{.45}{gravity/earth-1200px}};
      \fill[red] (0,1.7) circle (.05);
      \draw[function,->] (0,1.7) arc (90:-55:1.5 and 1.5);
      \draw[dashed,gray,thick] circle (1.7);
    \end{tikzpicture}
    \caption{With a higher initial velocity, the projectile will travel a longer
      distance but eventually, it still falls.}
  \end{subfigure}

  \vspace{.2in}\begin{subfigure}{.4\textwidth}
    \centering
    \begin{tikzpicture}
      \node at (0,0) {\pic{.45}{gravity/earth-1200px}};
      \fill[red] (0,1.7) circle (.05);
      \draw[function,->] (0,1.7) arc (90:-270:1.7);
    \end{tikzpicture}
    \caption{An object launched with a sufficient initial speed stays on a
      circular path (orbit) around the planet}
    \label{fig:in-orbit}
  \end{subfigure}
  \hspace{.3in}
  \begin{subfigure}{.4\textwidth}
    \centering
    \begin{tikzpicture}
      \node at (0,0) {\pic{.45}{gravity/earth-1200px}};
      \fill[red] (0,1.7) circle (.05);
      \draw[dashed,gray,thick] circle (1.7);
      \draw[function,->,domain=0:2.2] plot(\x,{-.3*\x*\x+1.7});
    \end{tikzpicture}
    \caption{The object launched with very high initial speed escapes the
      gravitational pull of the planet}
    \label{fig:escape-from-orbit}
  \end{subfigure}
  \caption{Newton's thought experiment on orbital motion}
  \label{fig:thought-experiment}
\end{figure}

In this discussion, we want to answer two questions:
\begin{enumerate}[itemsep=3pt]
\item How fast does the cannonball have to travel before it goes around Earth
  without falling, as dipicted in Fig.~\ref{fig:in-orbit}? (i.e.\ what is the
  sufficient speed required goes into orbit?)
\item How fast does the cannonball have to travel before it never comes back,
  as depicted in Fig.~\ref{fig:escape-from-orbit}?
\end{enumerate}  



\subsection{Orbital Velocity}
\label{sec:orbital-velocity}

%\textbf{Orbital velocity} is the speed required for an object to stay in a
%circular path without falling back onto the surface. For example:
%\begin{itemize}
%\item A spy satellite orbiting around the Earth
%\item The moon orbiting around the Earth
%\item Planets of the solar system orbiting around the Sun
%\end{itemize}

The example depicted in Fig.~\ref{fig:in-orbit} (``fast enough that it doesn't
fall down'') is that of an object moving in circular motion around the planet.
If we assume that a small mass $m$ is in a circular orbit around a much larger
mass $M$, we can further assume that $M$ is stationary. The distance $r$
between the masses is also the orbital radius of that circular motion, and the
centre of $M$ is also the centre of the circular motion. The required
centripetal force is supplied by the gravitational force. Since there are no
other forces, the object moves in uniform circular motion with a constant speed
$v_\text{orbit}$, called the \textbf{orbital speed}, and the velocity vector
$\bm v_\text{orbit}$ is called the \textbf{orbital velocity}. This is shown
graphically in Fig.~\ref{fig:orbital-velocity}.
\begin{figure}[ht]
  \centering
  \begin{tikzpicture}
    \node at (0,0) {\pic{.12}{gravity/earth-1200px}};
    \draw[function,->] (0,3) arc (90:-270:3);
    \begin{scope}[rotate=50]
      \draw[thick] (0,0)--+(0,-.8);
      \draw[very thick,<->] (0,-.5)--(3,-.5) node[midway,fill=white]{$r$};
      \fill circle (.05) node[above]{$M$};
      \draw[vector,violet] (3,0)--+(0,-2) node[right]{$\bm v_\text{orbit}$};
      \draw[mass] (3,0) circle (.07) node[above]{$m$};
    \end{scope}
  \end{tikzpicture}
  \caption{Object travelling in a circular orbit at the orbital velocity.}
  \label{fig:orbital-velocity}
\end{figure}

Starting with the second law of motion, $\bm F_c=m\bm a_c$, and
substituting the gravitational force $\bm F_g$ into the expression for
centripetal force $\bm F_c$:
\begin{equation}
  F_g=ma_c\quad\longrightarrow\quad \frac{GMm}{r^2}=\frac{mv^2}r
\end{equation}
cancelling the $r$ and $m$ terms on both sides of the equation, we can solve
for the expression for orbital speed:
%which does not depend on the mass $m$ of the object in orbit:
\begin{equation}
  \boxed{
    v_\text{orbit}=\sqrt{\frac{GM}r}
  }
  \label{eq:orbital-velocity}
\end{equation}
Note that this equation is only applicable for perfectly circular orbits. Like
all uniform circular motion, the direction of $\bm v_\text{orbit}$ is
tangent to the circular path.



%\begin{tikzpicture}[scale=.8]
%  \foreach \x in {0,3} \draw (\x,0)--+(0,-1);
%  \draw[dashed,thick] circle (3);
%  \shade[balloon2] circle (.7) node[white]{$M$};
%  \draw[vectors,red] (3,0)--+(-1.2,0) node[above]{$\bm F_g$};
%  \shade[balloon1] (3,0) circle (.16) node[white]{$m$};
%  \draw[<->] (0,-.8)--(3,-.8) node[midway,fill=white]{$r$};
%\end{tikzpicture}
%    
%To calculate the magnitude of orbital velocity, we make these assumptions:
%\begin{itemize}
%\item A small mass ($m$) orbits a much larger one ($M$), i.e.\ $M\gg m$
%\item The orbit is perfectly circular
%\end{itemize}
%Following these assumptions:
%\begin{itemize}
%\item $M$ is treated as a stationary object; only $m$ moves
%\item Orbital radius ($r$) is the same as the distance between the masses
%\item Centripetal force is provided by gravity
%  %\eq{-.1in}{
%  %  F_c=F_g=\frac{GMm}{r^2}
%  %}
%\item The orbital motion is a uniform circular motion
%\end{itemize}
%  
%The centripetal force for mass $m$ staying in orbit is provided by gravity:
%\begin{equation}
%  \underbracket[1pt]{F_c}_{{\color{orange}=F_g}} =
%  m{\color{magenta}a_c}\quad\rightarrow\quad
%  %F_g = ma_c\quad\rightarrow\quad
%  {\color{orange}\frac{GMm}{r^2}}
%  = \frac{m{\color{magenta}v^2}}{\color{magenta}r}
%\end{equation}
%Cancelling $m$ and $r$ terms on both sides, and solving for
%magnitude of velocity, we find the magnitude of the orbital velocity to be:
%\begin{equation}
%  v=\boxed{v_\text{orbit}=\sqrt{\frac{GM}r}}
%\end{equation}
%  \begin{center}
%    \begin{tabular}{l|c|c}
%      \rowcolor{pink}
%      \textbf{Quantity} & \textbf{Symbol} & \textbf{SI Unit} \\ \hline
%      Orbital velocity  & $v_\text{orbit}$ & \si{\metre\per\second} \\
%      Universal gravitational constant & $G$ & \si{N.m^2/kg^2}\\
%      Distance between centres of mass & $r$ & \si\metre \\
%      Mass of the larger object        & $M$ & \si{\kilo\gram}
%    \end{tabular}
%  \end{center}
Orbital velocity does not depend on $m$, the smaller mass of the object in
orbit. In other words, a \SI{1.5e-13}{\kilo\gram} speck of cosmic dust and the
\SI{4.2e5}{\kilo\gram} International Space Station both have the same
$v_\text{orbit}$ around Earth at the same altitude.

\begin{example}
  What is the orbital speed of a satellite at a height of
  \SI{300}{\kilo\metre} above the surface of Earth? The mass of Earth is
  \SI{5.97e24}{\kilo\gram} and the radius of Earth is \SI{6.37e3}{\kilo\metre}.

  \textbf{Solution:} To calculate the orbital speed, we must first calculate
  the orbital radius $r$. In this case, it is the satellite's altitude plus
  the radius of Earth:
  \begin{equation*}
    r=h+r_E=300+6370=\SI{6670}{\kilo\metre}=\SI{6.67e6}\metre
  \end{equation*}
  Now we can use Eq.~\ref{eq:orbital-velocity} to solve the problem:
  \begin{equation*}
    v_\text{orbit}=\sqrt{\frac{GM}r}
    =\sqrt{\frac{(6.67\times 10^{-11})\times(5.97\times10^{24})}{6.67\times10^6}}
    =\SI{7.7e3}{\metre\per\second}
  \end{equation*}
  The problem itself is straightforward; whether you get the correct answer
  is generally determined by whether you have used the correct value for $r$.
\end{example}

\begin{example}
  At what speed $v$ and altitude $h$ must a satellite orbit above the equator
  in order to be geostationary? ``Geostationary'' means that the position of
  the satellite is over the same position on earth all the time. This means
  that period of the satellite must be the same as Earth's rotation (1 day).
\end{example}


\subsection{Escape Velocity}
An object can leave the surface of Earth at any velocity. But when all the
kinetic energy of that object is converted into gravitational potential
energy, it will return back to the surface of Earth. However, there is a
\emph{minimum} velocity at which the object would not fall back to Earth
because of gravity, called the \textbf{escape velocity}. To derive the escape
velocity, we assume that
\begin{itemize}[itemsep=2pt]
\item The object and Earth form an isolated system
\item All the work is done by gravity
\item There are no other external forces such as drag, or the thrust from
  engines
\item There is no change in the mass of the object
\end{itemize}
Note that this is not the case for launching a \emph{real} rocket, where we must
consider the rapid change in mass when the rocket is burning fuel, and
the (positive) work done by the rocket's thrust, and the extraordinary amount of
drag that the rocket experiences. Nevertheless, we can use these assumptions to
get an approximation of the speed required to escape the gravitational pull of
Earth.

Since gravity is a conservative force, energy is conserved, therefore
\begin{equation}
  \underbracket[1pt]{U_i+K_i}_\text{launch}=
  \underbracket[1pt]{U_\infty+K_\infty}_\text{after escape}
  \label{eq:launch}
\end{equation}
where $U_i$ and $K_i$ are the gravitational potential and kinetic energy at
launch respectively. At launch, $U_i=-\dfrac{GMm}r$. At $r=\infty$, $U_\infty=0$.
At this time, the object is no longer the under the gravitational influence
of the planet, and therefore we can set  $K_\infty=0$. The right-hand-side of
Eq.~\ref{eq:launch} reduces to zero.

In the left-hand side of the the energy conservation equation, we set $K=-U$
and solving for $v$ that satisfies the equation:
\begin{equation}
  \frac12mv^2=\frac{GMm}r
  \quad\longrightarrow\quad
  v=\boxed{v_\text{esc}=\sqrt{\frac{2GM}r}}
\end{equation}
%\begin{center}
%  \begin{tabular}{l|c|c}
%    \rowcolor{pink}
%    \textbf{Quantity} & \textbf{Symbol} & \textbf{SI Unit} \\ \hline
%    Escape velocity & $v_\text{esc}$ & \si{\metre\per\second} \\
%    Universal gravitational constant & $G$ & \si{N.m^2/kg^2}\\
%    Distance between centres of mass & $r$ & \si\metre \\
%    Mass of the larger object        & $M$ & \si{\kilo\gram}
%  \end{tabular}
%\end{center}




%Any object with $v\geq v_\text{esc}$ can break free of a planet's
%gravitational pull.

\begin{example}
  Determine the escape velocity for a
  \SI{1.60e6}{\kilo\gram} rocket leaving the surface of Earth.

  \textbf{Solution:} Although the mass of the rocket is given,
  the escape speed does not depend on it. Given that the Earth's mass and
  radius are, 
  \begin{equation*}
    v_\text{esc}=\sqrt{\frac{2GM}r}
    =\sqrt{\frac{2(6.67\times10^{-11})(5.792\times10^{24})}{6.371\times10^6}}=
    \boxed{\SI{1.12e4}{\metre\per\second}}
  \end{equation*}
  The launch a rocket into space, the rocket must have an initial speed of
  approximately \SI{11.2}{\kilo\metre\per\second}. Consider that commercial
  airplanes generally fly at approximately 13 km above Earth's surface, a
  rocket launched must reach this altitude in about 1 second.
\end{example}
%
%
%
%\subsection{Comparing Orbital Velocity to Escape Velocity}





%\subsection{Non-Circular Orbits}
%  \begin{center}
%    \pic{.6}{graphics/newton-cannon-orbital-types-Seeds}
%  \end{center}


It's important to note that the object can escape from the \emph{surface} or
the planet, or it can escape from \emph{orbit}. In Fig.~\ref{fig:two-escapes},
both objects have the same escape velocity.
\begin{figure}[ht]
  \centering
  \begin{subfigure}{.4\linewidth}
    \centering
    \begin{tikzpicture}[scale=1.8]
      \node at (0,0) {\pic{.15}{gravity/earth-1200px}};
      %\draw[thick] circle (.25) node[below]{$M$};
      \draw[thick,dashed] circle (1);
      \draw[mass] (0,1) circle (.05);
      \draw[mass] (2,0) circle (.05);
      \draw[thick,<->] (0,0)--(0,1) node[midway,right=0]{$r$};
      \draw[vector,red] (0,1) to[out=0,in=135] (2,0);
    \end{tikzpicture}
  \end{subfigure}
  \begin{subfigure}{.4\linewidth}
    \centering
    \begin{tikzpicture}[scale=1.8]
      \node at (0,0) {\pic{.61}{gravity/earth-1200px}};
      %\draw[thick] circle (1) node[below]{$M$};
      \draw[mass] (0,1) circle (.05);
      \draw[mass] (2,0) circle (.05);
      \draw[thick,<->] (0,0)--(0,1) node[midway,right]{$r$};
      \draw[vector,red] (0,1) to[out=0,in=135] (2,0);
    \end{tikzpicture}
  \end{subfigure}
  \caption{Escape velocity only depends on $M$ and $r$}
  \label{fig:two-escapes}
\end{figure}
%but the
%one in orbit (left) already has orbital velocity $v_\text{orbit}$, so escaping
%from orbit only requires an additional speed of
Both are at a distance $r$ from the centre of the planet of mass $M$.
The difference is that the object in orbit (left) already has orbital speed
$v_\text{orbit}$, so escaping from that orbit requires only an additional
speed of
\begin{equation}
  \Delta v=v_\text{esc}-v_\text{orbit}=(\sqrt2-1)v_\text{orbit}
\end{equation}



\subsection{Elliptical Orbits}
Orbital and escape velocities differ by a factor of $\sqrt2$:
\begin{equation*}
  v_\text{orbit}=\sqrt{\frac{GM}r}
  \quad\quad
  v_\text{esc}=\sqrt{\frac{2GM}r}=\sqrt2v_\text{orbit}
\end{equation*}
What happens if $v_\text{orbit}<v<v_\text{esc}$? What kind of an orbit does it
have? What happens if $v<v_\text{orbit}$? Will the object crash into the planet?


As shown in Fig.~\ref{fig:circular-elliptical-parabolic-hyperbolic},
depending on the magnitude of $\bm v$ at $P$:
\begin{figure}[ht]
  \centering
  \begin{tikzpicture}
    \node at (0,0) {\pic{.28}{gravity/earth-1200px}};
    \draw[vector,magenta] (0,4) arc (90:-250:4);
    \draw[vector,violet,domain={0:4},dash dot] plot(\x,-.1*\x*\x+4);
    %node[right]{$v_\text{esc}$ (parabola)};
    \draw[very thick,blue,dotted] (0,.5) ellipse (3.2 and 3.5);
    \draw[very thick,orange,dashed] (0,-.5) ellipse (4.2 and 4.5);
    \draw[mass] (0,4) circle (.08) node[above]{$P$};
    \draw[vector] (0,4)--(1.5,4) node[above]{$\bm v$};
    \fill (0,-3) circle (.08) node[below]{$P'$};
    \fill (0,-5) circle (.08) node[below]{$P''$};
    \node[above,blue] at (0,-3) {\scriptsize Elliptical};
    \node[above,magenta] at (0,-4) {\scriptsize Circular};
    \node[above,orange] at (0,-5) {\scriptsize Elliptical};
    \draw[thick,|<->] (0,0)--(0,4) node[pos=.6,right]{$r$};
  \end{tikzpicture}
  \caption{Different orbital paths}
  \label{fig:circular-elliptical-parabolic-hyperbolic}
\end{figure}
\begin{itemize}[itemsep=6pt]
\item{\color{magenta}If the object moves at orbital velocity, i.e.\
  $v=v_\text{orbit}$, it will orbit around the planet in uniform circular
  motion. The path is perfectly circular, and the speed is constant.}
  
\item{\color{violet}If the object is at escape velocity, $v=v_\text{esc}$, it
  will leave the planet's gravitational pull in a parabolic path.}
  
\item{\color{green!80!black}If the object is faster than escape velocity, i.e.\
    $v>v_\text{esc}$, it will leave the planet's gravitational pull in a
  hyperbolic path.}
  
\item{\color{orange} If the object is moving faster than the orbital speed, but
  not faster enough for escape speed, i.e.\ $v_\text{orb}<v<v_\text{esc}$, the
  object would be too fast to maintain a circular path, and therefore gain
  altitude, arriving at point $P'$ on the other side of the planet, which is
  the furthest away from the centre. the resulting path is elliptical.}
  
\item{\color{blue} If the object is moving slower than the orbital speed, i.e.\
  $v<v_\text{orb}$, the object would lose altitude, arriving at point $P''$ on
  the other side of the planet, which is the closest away from the centre. The
  resulting path is also elliptical. The object does not necessarily crash; that
  will depend on what the actual speed is.}
\end{itemize}



\section{Orbital Energies}

\textbf{Orbital kinetic energy:} We can obtain the orbital kinetic energy in a
perfectly circular orbit by applying the orbital velocity in our expression of
kinetic energy:
\begin{equation}
  K_\text{orbit}=\frac12mv_\text{orbit}^2=\frac12m
  \left(\sqrt{\frac{GM}r}\right)^2=\boxed{\frac{GMm}{2r}}
\end{equation}
This equation uses the expression for orbital velocity $v_\text{orbit}$,
therefore it relies on the same assumption, i.e.\ a perfectly circular orbit



\textbf{Total orbital energy:} Total orbital energy is the sum of kinetic and
gravitational potential energies while in orbit:
\begin{equation}
  E_\text{orbit}=K_\text{orbit}+U_g
\end{equation}
Substituting the expressions for $K$ and $U_g$ derived for objects in orbit:
\begin{equation}
  E_\text{orbit}=
  \underbracket[1pt]{\frac{GMm}{2r}}_{K_\text{orbit}} +
  \underbracket[1pt]{\left(-\frac{GMm}r\right)}_{U_g}=
  \boxed{-\frac{GMm}{2r}}
\end{equation}

%Orbital kinetic energy can be exprssed in terms of orbital velocities, or
%in terms of $M$ and orbital radius $r$:
%
%%  \eq{-.1in}{
%%    K_\text{orbit}=\frac12mv_\text{orbit}^2=\frac{GMm}{2r}
%%  }
%  
%Gravitational potential energy is $-2$ times the orbital kinetic energy:
%  
%%  \eq{-.1in}{
%%    U_g=-\frac{GMm}r=-2K_\text{orbit}
%%  }
%%
%Adding $K_\text{orbit}$ and $U_g$ together, the total orbital
%energy must be the negative of the orbit kinetic energy:
%  
%  \eq{-.15in}{
%    E_\text{orbit}=K_\text{orbit}+U_g=-\frac{GMm}{2r}=-K_\text{orbit}
%  }
%
%
%
\textbf{Binding Energy:} Binding energy $E_\text{binding}$ is the energy
required to remove the mass $m$ from its orbit, and bring it to $r=\infty$,
where total energy $E_\infty=0$. To do so, we need to make up for the deficit
in total energy, therefore:
\begin{equation}
  E_\text{binding}=E_\infty-E_\text{orbit}=0-(-K_\text{orbit})=K_\text{orbit}
\end{equation}
For those with a background in chemistry, the binding energy is analogous to
the \emph{ionization energy} of removing the valence electron from an atom.



%%\begin{example}
%%  On March 6, 2001, the Mir space station was deliberately
%%  crashed into Earth. At the time, its mass was \SI{1.39e3}{\kilo\gram} and its
%%  altitude was \SI{220}{\kilo\metre}.
%%  \begin{itemize}
%%  \item Prior to the crash, what was its binding energy to Earth?
%%  \item How much energy was released in the crash? Assume that its orbit was
%%    circular
%%  \end{itemize}
%%
%
%
%
%%\section{Space Travel}
%%
%%\section{Energy and Momentum in Space}
%%  How does a spacecraft move in space anyway? Well, like this:
%%  \begin{center}
%%    \pic{.8}{graphics/throwing-stuff}
%%  \end{center}
%%
%%
%%
%%
%%\section{Energy and Momentum in Space}
%%  We don't throw out blocks of cargo from the back of the spacecraft,
%%  but we \emph{do} expel burnt fuel from the rockets at very high speeds\ldots
%%  \begin{center}
%%    \pic{.8}{graphics/rocket}
%%  \end{center}
%%
%%
%%
%%
%%\section{Energy and Momentum in Space}
%%  Let's go back to the impulse-momentum theorem from Unit 2:
%%
%%  \eq{-.2in}{ \bm F\Delta t= \Delta\bm{p}=\Delta(m\bm{v}) }
%%  
%%  Therefore
%%
%%  \eq{-.35in}{
%%    \bm F_\text{(on\;gas)}\Delta t= m\Delta\bm{v}_\text{gas}
%%  }
%%  
%%  Or
%%
%%  \eq{-.35in}{
%%    \boxed{\bm F_\text{(on\;gas)}=
%%      \left(\frac{m}{\Delta t}\right)\Delta\bm{v}_\text{gas}}
%%  }
%%
%%
%%
%%
%%\section{The Space Shuttle}{Pinnacle of Technology in the 1970's and 1980's}
%%  
%%    \pic{1.1}{graphics/atlantis}

%%    \begin{itemize}
%%    \item Two solid-fuel booster rockets during take-off
%%      \begin{itemize}
%%      \item Once turned on, they can't be turned off
%%      \item Retrived from the ocean to be re-used again
%%      \end{itemize}
%%    \item One single-use external liquid-fuel tank (``ET'') to supply additional
%%      fuel to the shuttle's main rocket engines (``SSME'')
%%    \end{itemize}
%%  
%%
%%
%%
%%
%%\begin{example}
%%  A rocket engine consumes \SI{50.}{\kilo\gram} of hydrogen
%%  and \SI{400.}{\kilo\gram} of oxygen during a \SI{5.00}{\second} burn.
%%  \begin{itemize}
%%  \item If the exhaust speed of the gas is \SI{3.54}{\kilo\metre\per\second},
%%    determine the thrust of the engine
%%  \item If the rocket has a mass of \SI{1.5e4}{\kilo\gram}, calculate the
%%    acceleration of the rocket
%%  \end{itemize}
%%\end{example}
%%
%%
%%
%%\section{Gravitational Assist or Slingshot}
%%  
%%    The interaction between the objects is a \emph{very} elastic ``collision''
%%    
%%    \begin{center}
%%      \pic{1}{graphics/slingshot}
%%    \end{center}
%%  
%%
%%
%%
%%
%%\section{Gravitational Assist or Slingshot}
%%  The interaction between the objects is \emph{very} elastic\\
%%  \begin{center}
%%      \pic{.9}{graphics/bounce}
%%    \end{center}

