%\documentclass[12pt,compress,aspectratio=169]{beamer}
%\input{../mybeamer}
%
\chapter{Transformers}
%\subtitle{Unit 5: Electricity and Magnetism}
%\input{../term}
%\input{../mycommands}
%
%
%\begin{document}
%
%\begin{frame}
%  \titlepage
%\end{frame}
%
%
%
%
%\begin{frame}{Direct and Alternating Current}
%  In a \textbf{direct current} (\textbf{DC}), the flow of charges is always in
%  the same direction, but the current itself does not have to be constant in
%  time.
%  \begin{center}
%    \begin{tikzpicture}[scale=.6]
%      \draw[functions] (0,3)--(4,3);
%      \draw[axes] (0,0)--(4,0) node[right]{$t$};
%      \draw[axes] (0,0)--(0,4) node[above]{$I$};
%    \end{tikzpicture}
%    \begin{tikzpicture}[scale=.6]
%      \draw[smooth,samples=20,domain=0:3.5,functions]
%      plot(\x,{3*(exp(-.8*\x))});
%      \draw[axes] (0,0)--(4,0) node[right]{$t$};
%      \draw[axes] (0,0)--(0,4) node[above]{$I$};
%    \end{tikzpicture}
%    \begin{tikzpicture}[scale=.6]
%      \draw[smooth,samples=20,domain=0:3.5,functions] plot(\x,{sin(120*\x)+2});
%      \draw[axes] (0,0)--(4,0) node[right]{$t$};
%      \draw[axes] (0,0)--(0,4) node[above]{$I$};
%    \end{tikzpicture}
%  \end{center}
%
%
%
%
%\section{Direct and Alternating Current}
%  In an \textbf{alternating current} (AC), the flow of charges changes
%  direction, usually as a sinusoidal function of time.
%  \begin{center}
%    \begin{tikzpicture}[scale=.6]
%      \draw[axes] (0,0)--(5,0) node[right]{$t$};
%      \draw[axes] (0,-2)--(0,2) node[above]{$I$};
%      \draw[smooth,samples=50,domain=0:4,functions] plot(\x,{1.5*sin(150*\x)});
%    \end{tikzpicture}
%    \begin{tikzpicture}[scale=.6]
%      \draw[axes] (0,0)--(5,0) node[right]{$t$};
%      \draw[axes] (0,-2)--(0,2) node[above]{$V$};
%      \draw[smooth,samples=40,domain=0:4,functions] plot(\x,{1.5*sin(150*\x)});
%    \end{tikzpicture}
%  \end{center}
%  The power outlet in North America are all AC, with a maximum voltage of
%  \SI{120}\volt, and a frequency of \SI{60}\hertz. AC current are important
%  in the power generation.
%
%
%
%
%\section{AC Generator}
%  
%    \pic1{graphics/generator}
%
%    \begin{itemize}
%    \item When the coil is turning, the charge carriers inside the coil move
%      relative to the magnetic field
%    \item The charges experience a magnetic force, creating an electromotive
%      force
%%    \item When the coil is turning, the orientation between the coil and the
%%      magnetic field changes
%%    \item Because the \emph{magnetic flux} is changing, a potential difference
%%      (voltage) is created in the coil itself
%    \item When connected to a resistor, it acts as an AC power source
%    \end{itemize}
%  
%
%
%
%
%\section{Transformers}
%
%\section[t]{How Transformers Work}
%  \begin{center}
%    \pic{.8}{graphics/transformers}
%  \end{center}
%  \vspace{-.2in}\begin{itemize}
%  \item Consists of two coils (\emph{primary} and \emph{secondary})
%  \item Think of each coil as solenoids
%  \item The coils are not electrically connected (i.e.\ no wires connecting the
%    two circuits)
%  \item Both coils may be wrapped around a square soft-iron core (shown above)
%  \end{itemize}
%
%
%
%
%\section[t]{How Transformers Work}
%  \begin{center}
%    \pic{.8}{graphics/transformers}
%  \end{center}
%  \vspace{-.2in}\begin{itemize}
%  \item The primary coil is connected to a AC power source
%  \item When current flows through the primary circuit, a magnetic field is
%    generated in the primary coil
%  \item Because the current is AC, the magnetic field inside the primary coil
%    is constantly changing in strength and direction
%  \end{itemize}
%
%
%
%
%\section[t]{How Transformers Work}
%  \begin{center}
%    \pic{.8}{graphics/transformers}
%  \end{center}
%  \vspace{-.2in}\begin{itemize}
%  \item A magnetic field is also generated inside the entire soft-iron core
%  \item The magnetic field in the iron core oscillates with the same
%    frequency as the primary coil
%  \item A current is induced in the secondary coil because the magnetic field
%    through the coil (i.e.\ the iron core) is changing
%  \end{itemize}
%
%
%
%
%\section[t]{How Transformers Work}
%  \begin{center}
%    \pic{.8}{graphics/transformers}
%  \end{center}
%  \vspace{-.2in}\begin{itemize}
%  \item An alternating current is generated in the secondary coil with the same
%    frequency as the primary coil
%  \item The output voltage and current in the secondary coil depends on the
%    ratio of windings in both coils
%    \begin{itemize}
%    \item \textbf{(a)} is called a \emph{step-down} transformer, while
%    \item \textbf{(b)} is called a \emph{step-up} transformer
%    \end{itemize}
%  \end{itemize}
%
%
%
%
%\section{Transformers in Circuit Diagrams}
%  The circuit diagram for a transformer consists of two coils. The vertical
%  line between the coil indicate whether this transformer has a soft-iron core:
%
%  \vspace{.2in}
%  
%    \centering
%    \begin{tikzpicture}[scale=2]
%      \draw (0,0) node[transformer] (T) {}
%      (T.A2) to [short,o-] ++ (-1,0) coordinate (aux)
%      to [sI] (aux |- T.A1)
%      to [short,-o] (T.A1);
%    \end{tikzpicture}
%
%    Air core
%
%    \centering
%    \begin{tikzpicture}[scale=2]
%      \draw (0,0) node[transformer] (T) {}
%      (T.A2) to [short,o-] ++ (-1,0) coordinate (aux)
%      to [sI] (aux |- T.A1)
%      to [short,-o] (T.A1);
%      \draw[thick] ( .03,.3)--( .03,-.3);
%      \draw[thick] (-.03,.3)--(-.03,-.3);
%    \end{tikzpicture}
%
%    Iron core
%  
%
%
%
%
%\section{Transformers}
%  Assuming that the electric potential energy (and therefore power) in the
%  primary coil is transmitted to the secondary coil, and using the equation for
%  power in a circuit:
%
%  \eq{-.1in}{
%    \Delta E_p=\Delta E_s
%    \quad\rightarrow\quad
%    P_p=P_s
%    \quad\rightarrow\quad
%    \boxed{V_pI_p = V_sI_s}
%  }
%
%
%
%
%\section{Transformers}
%  Voltage, current in the two coils (primary and secondary) are related to the
%  number of windings:
%  
%  \eq{-.1in}{
%    \boxed{\frac{V_p}{V_s}=\frac{I_s}{I_p}=\frac{N_p}{N_s}}
%  }
%  \begin{center}
%    \begin{tabular}{l|c|c}
%      \rowcolor{pink}
%      \textbf{Quantity} & \textbf{Symbol} & \textbf{SI Unit} \\ \hline
%      Number of windings in primary \& secondary coils & $N_p$, $N_s$ & N/A \\
%      Current in primary \& secondary coils & $I_p$, $I_s$ & \si\ampere \\
%      Voltage across primary \& secondary coils & $V_p$, $V_s$ & \si\volt
%    \end{tabular}
%  \end{center}
%  These two equations assume that the transformer is \SI{100}{\percent}
%  efficient. In practice, the efficiency of transformers is
%  $\approx\SI{90}\percent$.
%
%
%
%
%%\section{Transformers}
%%  \begin{itemize}
%%  \item Because only a \emph{changing} magnetic field can induce current, the
%%    transformer can only be used with alternating current
%%  \end{itemize}
%%
%
%
%
%\section{Power Grid}
%  Between where power is generated, and your electrical outlet, AC power
%  usually goes through a number of step-up and step-down transformers. This is
%  usually referred to as a \textbf{power grid}.
%  \begin{center}
%    \pic{.7}{graphics/powerline}
%  \end{center}
