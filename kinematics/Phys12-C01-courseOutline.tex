\documentclass{../oss-handout}
%\usepackage[sfdefault,lf]{carlito}
\usepackage{newtxtext}
\usepackage{enumitem}
\usepackage{titlesec}
\usepackage{xcolor,color,colortbl}

\setlength{\parindent}{0pt}
\setlength{\parskip}{2pt}
\setlength{\headheight}{26pt}

\titleformat*{\section}{\bfseries}%\large}
\titlespacing\section{0pt}{10pt plus 4pt minus 2pt}{4pt plus 12pt minus 2pt}

% Set the page style for the document
\pagestyle{plain}

% Course & handout information
\renewcommand{\institution}{Meritus Academy}
\renewcommand{\coursetitle}{Grade 12 Physics}
\renewcommand{\term}{Summer 2024}
\title{GRADE 12 PHYSICS COURSE OUTLINE}
\author{}
\date{\today}

\begin{document}
\thispagestyle{title}
\gentitle



\section{Course Objectives}
\begin{itemize}[nosep]
\item Develop analytic skills, strategies, and habits of mind required for
  scientific inquiry, including critical thinking and inferring
\item Develop communicative skills, strategies, and habits required for
  scientific inquiry
\item Learn fundamental concepts of introductory high school physics
\item Extend fundamental concepts beyond the mandate of the Ontario curriculum
\item Gain exposure to both mainstream and unconventional applications of
  scientific concepts
\end{itemize}
%\textbf{Pre-requisites:}
%\begin{itemize}[nosep]
%\item\textbf{Grade 11 Physics:} You should be familiar with the concepts
%  covered in Grade 11 Physics.
%\item\textbf{Math 11 (Functions):} Concepts about functions and basic
%  trigonometry are applied in this course.
%\end{itemize}



\section{Teacher Information}
Teacher name \& contact information: \underline{\hspace{4.5in}}



\section{Class Times}
%The Grade 12 Physics course at Meritus Academy runs for 40 hours (16 classes).
There are \underline{seven} sections of the course for the Summer 2024 session:
\begin{center}
  \bgroup
  \def\arraystretch{1.1}
  \begin{tabular}{|c|c|c|c|}
    \hline
    \rowcolor{lightgray}
    \textbf{Section} & \textbf{Day} & \textbf{Time} & \textbf{Format}\\
    \hline\hline
    Physics 12-1 & Mondays \& Thursdays & 1:00 pm -- 3:30 pm & Online \\
    \hline
    Physics 12-2 & Mondays \& Thursdays & 7:00 pm -- 9:30 pm & Online \\
    \hline
    Physics 12-3 & Tuesdays \& Fridays & 10:00 am -- 12:30 pm & Online\\
    \hline
    Physics 12-4 & Tuesdays \& Fridays & 4:00 pm -- 6:30 pm & Online\\
    \hline
    Physics 12-5 & Saturdays \& Sundays & 1:00 pm -- 3:30 pm & In-person\\
    \hline
    Physics 12-6 & Saturdays \& Sundays & 4:00 pm -- 6:30 pm & In-person\\
    \hline
    Physics 12-7 & Mondays, Tuesdays, Thursdays \& Fridays (Aug.)
    & 4:00 pm -- 6:30pm & Online\\
    \hline                
  \end{tabular}
  \egroup
\end{center}
\vspace{-.1in}Please contact \texttt{info@olympiads.ca} to request a Zoom link
for make-up class(es).

%\begin{enumerate}[nosep]
%
%%There are \underline{five} sections of the course for the Summer 2023 session:
%%\item Mondays \& Thurdays 7:00pm--9:30pm (Dr.\ Timothy Leung, online)
%%\item Mondays, Tuesdays, Thursdays \& Fridays 7:00pm--9:30pm
%%  (July 31 to August 25, Mr.\ Behzad Ghadyanloo TBD, online)
%%\item Tuesdays \& Fridays 1:00pm--3:30pm (Dr.\ Michael Horbatsch, online)
%%\item Tuesdays \& Fridays 7:00pm--9:30pm (Mr.\ Neell Young, online)
%%\item Saturdays \& Sundays 4:00pm--6:30pm (Dr.\ Timothy Leung, in-person)
%%\item Saturdays \& Sundays 4:00pm--6:30pm (Mr.\ Ryan Lin, in-person)
%
% %WINTER/SRPING 2024 SCHEDULE
%\item Tuesdays 6:30 pm to 9:00 pm (online with Mr.\ Neell Young)
%\item Saturdays 10:30 am to 1:00 pm (in person with Dr.\ Timothy Leung)
%\item Sundays 1:20 pm to 3:50 pm (online with Mr.\ Hardev Lad)
%
%%  % FALL SCHEDULE
%%\item Saturdays 7:00pm--9:30pm (Dr.\ Timothy Leung, In-Person)
%%\item Sundays 10:30am--1:00pm (Mr.\ Neell Young, Online)
%\end{enumerate}



\section{Course Material}
No textbook is required. Presentation slides, handouts and homework sets are
downloadable from the school website; students are expected to download them
prior to class. Please have a pen/pencil (or tablet for online classes) for
note-taking, and a scientific calculator for working out in-class example
problems.



\section{Homework}
Homework questions are assigned after every class based on the topics covered.
There are usually about 15--20 questions, consisting of multiple-choice,
short-answer and problem-solving questions. Homework is due at the beginning of
each class. Homework questions are reviewed in their entirety during the
homework take-up tutorial on Monday nights (see Section~\ref{tutorial}).


\section{Tests}
There are two tests:
\begin{itemize}[nosep]
\item A \textbf{take-home midterm test} assigned for Class 7 (due at Class 8)
\item An \textbf{in-class final test} on the last day (Class 16)
\end{itemize}



\section{Online Tutorial}
\label{tutorial}
There are two online tutorials. Homework questions from the previous class will
be taken-up during the Monday session, while the Thursday evening drop-in
sessions is for students to ask questions. Zoom links for the tutorials can be
found in the student account.
\begin{center}
  \bgroup
  \def\arraystretch{1.1}
  \begin{tabular}{|p{2.1in}|c|c|c|}
    \rowcolor{lightgray}
    \hline
    & \textbf{Day} & \textbf{Time} & \textbf{Dates} \\
    \hline\hline
    \textbf{Homework take-up tutorial} &
    \hspace{.2in}Monday\hspace{.2in} &
    \hspace{.2in}7:30 -- 8:30 pm\hspace{.2in} &
    \hspace{.2in}February 8 -- May 23\hspace{.2in} \\
    \hline
    \textbf{Drop-in tutorial} &
    Fridays &
    7:30 -- 8:30 pm & 
    February 19 -- May 20 \\
    \hline
  \end{tabular}
  \egroup
\end{center}



\section{Academic Integrity}
Meritus Academy values academic integrity. Students are encouraged to complete
their homework and tests using knowledge gained from class. Handing in work
that are copied from the internet, or generated by AI is not allowed. If a
student is suspected to have copied, cheated or plagiarized, they will be
temporarily assigned a 0\% or incomplete, and the issue will be brought to the
school administration.



\section{Class Schedule}

\bgroup
\def\arraystretch{1.1}
\begin{tabular}{|c|c|p{4.05in}|c|}
  \hline
  \rowcolor{lightgray}
  \textbf{Class} & \textbf{Date} & \textbf{Description} & \textbf{HW Due} \\
  \hline\hline
  1 & Feb.\ 6 -- Feb.\ 11 & Kinematics & --- \\
  \hline
  2 & Feb.\ 13 -- Feb.\ 18 & Dynamics  & HW 1\\
  \hline
  3 & Feb.\ 20 -- Feb.\ 25 & Circular motion & HW 2 \\
  \hline
  4 & Feb.\ 27 -- Mar.\ 3 & Work and energy & HW 3 \\
  \hline
  5 & Mar.\ 5 -- Mar.\ 10 & Momentum, impulse \& collisions & HW 4 \\
  \hline
  6 & Mar.\ 12 -- Mar.\ 17 & Harmonic motion & HW 5 \\
  \hline
  \rowcolor{lightgray!50}
  7 & Mar.\ 19 -- Mar.\ 24 &
  \parbox{3.4in}{\vspace{.07in}Gravity (1) \\
    \textbf{Take-home midterm test}\vspace{.07in}} & HW 6 \\
  \hline
  8 & Mar.\ 26 -- Mar.\ 31 & Gravity (2) and Electricity (1) & Midterm \\
  \hline
  9 & Apr.\ 2 -- Apr.\ 7 & Electricity (2) & HW 8 \\
  \hline
  10 & Apr.\ 9 -- Apr.\ 14 & Magnetism & HW 9 \\
  \hline
  11 & Apr.\ 16 -- Apr.\ 21 & Light waves & HW 10 \\
  \hline
  12 & Apr.\ 23 -- Apr.\ 28 & Light wave interference & HW 11 \\
  \hline
  13 & Apr.\ 30 -- May 5 & Thin-film interference and electromagnetic waves &
  HW 12 \\
  \hline
  14 & May 7 -- May 12 & Introduction to special relativity & HW 13 \\
  \hline
  15 & May 14 -- May 19 & Introduction to quantum mechanics & HW 14 \\
  \hline
  \rowcolor{lightgray!30}
  16 & May 21 -- May 26 & \textbf{In-class final test} & HW 15 \\
  \hline
\end{tabular}
\egroup



%\section*{Course Outline}
%\begin{enumerate}[topsep=0pt,itemsep=.1ex]
%\item\textbf{Fundamentals of Dynamics} We extend our understanding of the
%  concepts in kinematics and dynamics introduced in Physics 11, and apply
%  them to problems of connected bodies and pulleys. We will also learn about
%  circular motion.
%  \begin{itemize}[nosep]
%  \item Kinematics
%  \item Laws of motion
%  \item Connected bodies, pulleys
%  \item Circular motion
%  \end{itemize}
%  
%\item\textbf{Energy and Momentum} We will introduce the concept of linear
%  momentum, which is crucial in understanding collisions. We will also
%  strengthen our understanding of energy, including
%  kinetic, gravitational potential and elastic potential energy, and apply
%  conservation of momentum in collision problems.
%  \begin{itemize}[nosep]
%  \item Mechanical work and law of conservation energy
%  \item Definitions of kinetic and potential energies
%  \item Hooke's law and elastic potential energy
%  \item Momentum and impulse
%  \item Conservation of momentum
%  \item Elastic and inelastic collisions
%  \item Simple harmonic motion
%  \end{itemize}
%  
%\item\textbf{Force Fields} In this unit we study one of the most important
%  concepts in physics: field theory. We will study the three most common force
%  fields that we experience every day: gravitational, electric and magnetic
%  fields.
%  \begin{itemize}[nosep]
%  \item Law of universal gravitation, gravitational field and gravitational
%    potential energy
%  \item Kepler's laws of planetary motion and orbital mechanics
%  \item Coulomb's law for electrostatic force
%  \item Electric field, electric potential energy, electric potential, and
%    electric potential difference
%  \item Magnetic field and magnetic force on charged particles
%  \item Mass spectrometers
%  \end{itemize}
%
%\item\textbf{Wave Nature of Light} In this unit we gain understanding that
%  light is a wave in nature. We will study how light propagates through a
%  medium and through openings. Finally we study the nature of electromagnetic
%  waves, which include light, infrared, ultraviolet and radio waves.
%  \begin{itemize}[nosep]
%  \item Reflection and refraction of light through a medium
%  \item Dispersion of light
%  \item Diffraction and interference of light through an opening
%  \item Optical resolution and the Raleigh diffraction limit
%  \item Thin-film interference of light
%  \item Electromagnetic waves
%  \end{itemize}
%  \newpage
%  
%\item\textbf{Modern Physics} In this unit we study two important discoveries:
%  relativity and quantum mechanics. In special relativity, we revisit
%  fundamental understanding of space and time to describe motion close
%  to the speed of light. We will learn about the relativity of simultaneity,
%  time and space. From there, relativistic mass, momentum, kinetic energy
%  and mass-energy equivalence are introduced. The quantum mechanics section
%  focuses on the seemingly contradictory notion of wave-particle duality,
%  and probabilistic behaviour of electrons to gain understanding in the
%  fundamental relationship between waves and particles.
%  \begin{itemize}[nosep]
%  \item Postulates of special relativity
%  \item Speed of light
%  \item Relativity of simultaneity, time and space 
%  \item Relativistic mass, momentum and kinetic energy
%  \item Black-body radiation \& the quantization of energy
%  \item Photoelectric effect \& the quantization of light (photon)
%  \item Wave behaviour in matter
%  \item Uncertainty principle
%  \item Particle in a box
%  \end{itemize}
%\end{enumerate}
%
%
%
%
%\section*{Classroom Expectations (In-Person)}
%Students are expected to:
%\begin{itemize}[nosep]
%\item Be in your seat and ready to learn and participate during class.
%\item Stay on task without disturbing or distracting others.
%\item Raise your hand if you have any questions or comments and wait to be
%  called. Don't wait too long before you ask a question.
%\item If you need to leave the class early, your parent needs to pick you up at
%  the classroom door, or be brought to the front desk by a secretary.
%\item Be respectful for yourself, others, and the facilities; act in
%  a responsible manner in everything you do.
%\end{itemize}
%
%
%\section*{Classroom Expectations (Online)}
%Students attending this course will be expected to:
%\begin{itemize}[nosep]
%\item Log into the Zoom meeting a few minutes before the start of the class
%\item Have your full name appear on the screen
%\item Have your camera turned on showing your face
%\item Be ready to learn and participate during class
%\item Type in all your in-class questions and comments into the chat window.
%  Don't wait too long before you ask a question; the longer you wait, the less
%  effective it will be to answer your questions
%\item Be \emph{specific} with your questions. As skilled as our teachers are,
%  vague statements and questions like ``I don't understand, can you explain it
%  again'' are impossible to answer
%%\item Please inform your teacher if you have to leave the class early for
%%  whatever reason
%\item Be respectful to yourself, your teacher, and your fellow students. You
%  are expected to act maturely and responsibly
%\end{itemize}
%
%
%\section*{Homework}
%\begin{itemize}[nosep]
%\item Homework sets are assigned after every class based on the topics covered
%  in class. There are usually about 15 questions, consisting of
%  multiple-choice, short-answer and problem-solving questions
%\item Homework sets are distributed at the beginning of each class. (For online
%  classes, they are posted on Classkick. Please log into your account to check
%  for them.)
%\item \emph{Most} homework questions will be reviewed in class before they are
%  due. However, this does \emph{not} mean you don't need to do your homework at
%  home. Always do your best.
%\item Homework questions are due at the end of the next class. (The exceptions
%  are the one-month and weekend classes during the summer session.)
%\item Late homework is accepted. However, the usefulness of late-submissions
%  are often diminished
%\item The homework is marked with ``P'' for pass and ``I'' for incomplete. For
%  a pass, you must have 80 \% of the questions correctly. This should not be
%  difficult as we review questions in class. If you do get an ``I'', your
%  teacher will let you know how to fix your work
%\item Not regularly completing the homework may prompts a phone call to
%  communicate with the parents in order to help you better manage your time and
%  achieve your goal.
%\end{itemize}
%  
%\section*{Homework Standard}
%\begin{itemize}
%\item You do not need to show work for multiple-choice questions. For online
%  classes, please answer them directly on selection box on Classkick by
%  clicking your mouse on your answers (do not use the highlight or pen
%  functions). For in-person classes, simply circle the correct answer.
%\item For problem-solving questions, you must show \emph{all} work by providing
%  complete and organized steps.
%  \begin{itemize}[nosep]
%  \item Always write out the relevant equation(s) \emph{before} substituting
%    numerical values
%  \item If you introduce any new variables, briefly explain what they are
%  \item Draw diagrams whenever necessary, or whenever it helps to explain your
%    work
%  \item Only use standard variable names in you calculations (e.g.\ when
%    calculating speed, always use $v$ instead of ``$x$'')
%  \item Show all algebraic steps and numerical calculations
%  \item Circle or box all your final answers
%  \item Proper math format must be observed (e.g.\ proper use of ``='' sign,
%    units, etc.)
%  \end{itemize}
%  In short, answer the questions as if the reader is learning the concept from
%  you, not as if they already understands it.
%\item If a question requires you to \emph{explain}, please do so using short
%  complete sentences with supporting detail. There is no need to write long
%  paragraphs.
%\end{itemize}
\end{document}


