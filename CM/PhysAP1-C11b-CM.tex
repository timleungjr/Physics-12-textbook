\chapter{Centre of Mass}
\label{chapter:cm}

%{Centre of Mass}
%  Finding an object's centre of mass is important, because
%  \begin{itemize}
%  \item The laws of motion are formulated by treating an objects as point
%    masses (for real-life objects, we let the forces apply to the centre of
%    mass)
%  \item Objects can have \emph{rotational} motion in addition to
%    \emph{translational} motion as well (we will examine that a bit more in a
%    very-important topic later)
%  \end{itemize}



\section{Centre of Mass: Definition}
The \textbf{centre of mass} (``CM'') is the \emph{weighted average} of the
masses in a system. The ``system'' may be:
\begin{itemize}
\item A collection of individual particles
\item A continuous distribution of mass with constant density. In this case,
  CM is also the geometric centre (\textbf{centroid}) of the object
\item A continuous distribution of mass with varying density
\item If the masses are inside of a \emph{uniform} gravitational
  field\footnote{See discussion on gravitational field in the first part of
  this class}, then the CM is also its \textbf{centre of gravity} (``CG'')
\end{itemize}




\section{Finding the Centre of Mass}

Let's begin with the simplest case. Two equal masses $m$ lie along the
$x$-axis, located at $x_1$ and $x_2$. Where is the centre of mass of the system?
\begin{figure}[ht]
  \centering
  \begin{tikzpicture}
    \draw[axes] (-1,0)--(8,0) node[right]{$x$};
    \draw[mass] (2,0) circle (.2) node{$m$};
    \draw[mass] (6,0) circle (.2) node{$m$};
    \draw[thick] (0,.85)--(0,-.45) node[below]{$O$};
    \fill (4,0) circle (.05) node[above]{cm};
    \begin{scope}[very thick,->|]
      \draw (0,.35)--(2,.35) node[pos=0,left]{$x_1$};
      \draw (0,.75)--(6,.75) node[pos=0,left]{$x_2$};
      \draw[violet] (0,-.3)--(4,-.3) node[pos=0,left]{$x_\text{cm}$};
    \end{scope}
  \end{tikzpicture}
\end{figure}
It is obvious that the centre of mass is at the half-way point between the
masses:
\begin{equation*}
  x_\text{cm}=\frac{x_1+x_2}2
  \quad\quad\text{or}\quad\quad
  x_\text{cm}=\frac{mx_1+mx_2}{2m}
\end{equation*}

%
%
%
%{Slightly More Challenging}
But what if one of the masses are increased to $2m$? This is still not a
difficult problem; you can still \emph{guess} the right answer without knowing
the equation for centre of mass. 
%  \begin{center}
%    \begin{tikzpicture}
%      \draw[axes] (-4,0)--(4,0) node[right]{$x$};
%      \draw[mass] (-2.5,0) circle (.3) node{$m$};
%      \draw[mass] (2.5,0) circle (.42) node{$2m$};
%      \fill (5/6,0) circle (.05) node[below]{cm};
%    \end{tikzpicture}
%  \end{center}
The answer is still simple. The centre of mass is no longer half way between
the two masses, but now $\frac13$ the total distance from the larger masses. We
can show using a weighted average:
\begin{equation}
  x_\text{cm}=\frac{mx_1+(2m)x_2}{m+2m}
\end{equation}
The weighted average concept can now be applied to cases when there are
masses in two or more dimensions:
\begin{figure}
  \centering
  \begin{tikzpicture}
    \draw[axes] (-3,0)--(3,0) node[right]{$x$};
    \draw[axes] (0,-1.5)--(0,1.5) node[above]{$y$};
    \draw[mass] (-1.3,1) circle (.4) node{$m_1$};
    \draw[mass] (-1.5,-.5) circle (.3) node{$m_2$};
    \draw[mass] (1,.3) circle (.25) node{$m_3$};
    \draw[mass] (0,.3) circle (.2) node{$m_4$};
    \draw[mass] (2,-1) circle (.25) node{$m_5$};
  \end{tikzpicture}
\end{figure}
For a discrete number of $N$ point masses, the position centre of mass
$\bm x_\text{cm}$ is defined as the weighted average of the positions $\bm x_i$
of the masses:
\begin{important-equation}
  \bm x_\text{cm}=\frac{\sum_{i=1}^N m_i\bm x_i}{\sum_{i=1}^N m_i}
\end{important-equation}
The sum in the denominator ($\sum_{i=1}^N m_i$) is the total mas of the system.
In components form:
\begin{equation}
  x_\text{cm}=\frac{\sum m_ix_i}{m_\text{tot}}\quad\quad
  y_\text{cm}=\frac{\sum m_iy_i}{m_\text{tot}}\quad\quad
  z_\text{cm}=\frac{\sum m_iz_i}{m_\text{tot}}
\end{equation}

\begin{example}
  Consider the following masses and their coordinates which make up a
  ``discrete mass'' rigid body:
  \begin{align*}
    m_1 &=\SI{5.0}{\kilo\gram}  & \bm x_1&=3\xxx-2\zzz\\
    m_2 &=\SI{10.0}{\kilo\gram} & \bm x_2&=-4\xxx+2\yyy+7\zzz\\
    m_3 &=\SI{1.0}{\kilo\gram}  & \bm x_3&=10\xxx-17\yyy+10\zzz
  \end{align*}
  What is the position of the centre of mass of this system?
\end{example}

%{Continuous Mass Distribution}
When the number of masses approaches infinity, this becomes a continuous
distribution of mass. Taking the limit of masses $N\rightarrow\infty$ gives
the \emph{integral form} of our equation:
\begin{equation}
  \bm x_\text{cm}=\frac{\int\bm x\dl m}{\int\dl m}
\end{equation}
%  Calculus is not part of the AP Physics 1 curriculum, so you are welcome to
%  skip this slide. But this calculation is part of AP Physics C, which is
%  calculus-based. 



\section{Centroid}
For an object with a uniform mass distribution (i.e.\ constant density), the
centre of mass is also its geometric centre, called the \textbf{centroid}.
Some 2D examples are shown in the figure.
\begin{figure}[ht]
  \centering
  \begin{subfigure}{.3\textwidth}
    \centering
    \begin{tikzpicture}[scale=.45]
      \draw[mass] rectangle (7,4);
      \draw[<->] (-.3,0)--+(0,4) node[midway,left=-1]{$a$};
      \draw[<->] (0,-.3)--+(7,0) node[midway,below=-1]{$b$};
      
      \draw[<->] (3.5,0)--+(0,2) node[midway,right=-1]{$\frac a2$};
      \draw[<->] (0,2)--+(3.5,0) node[midway,above=-1]{$\frac b2$};

      \fill[red] (3.5,2) circle (.08) node[right]{cm};
    \end{tikzpicture}
    \caption{Rectangular Area}
  \end{subfigure}
  \begin{subfigure}{.3\textwidth}
    \centering  
    \begin{tikzpicture}[scale=.45]
      \draw[mass] (0,0)--(5.5,4)--(8,0)--(0,0);
      \draw (5.5,0)--+(0,4);
      \draw[<->] (-.3,0)--+(0,4) node[midway,fill=black!2]{$h$};
      \draw[<->] (0,-.3)--+(5.5,0) node[midway,below=-1]{$b$};
      \draw[<->] (5.5,-.3)--+(2.5,0) node[midway,below=-1]{$a$};
      
      \draw[<->] (11.5/3,0)--+(0,4/3) node[midway,left=-1]{$\frac h3$};
      \draw[<->] (11.5/3,4/3)--(8,4/3) node[midway,above=-1]{
        \small $\frac{b+2a}3$};
      \draw (8,0)--(8,1.5);
      \draw (-.5,4)--(6,4);
      \fill[red] (11.5/3,4/3) circle (.08) node[above]{cm};
    \end{tikzpicture}
    \caption{Triangular Area}
  \end{subfigure}
  \begin{subfigure}{.3\textwidth}
    \centering  
    \begin{tikzpicture}[scale=.45]
      \draw[mass] (0,0)--(0,4.5)--(7,3)--(7,0)--(0,0);
      \draw[<->] (-.3,0)--+(0,4.5) node[midway,left=-1]{$a$};
      \draw[<->] (7.3,0)--+(0,3) node[midway,right=-1]{$b$};
      \draw[<->] (0,-.3)--+(7,0) node[midway,below=-1]{$L$};
      
      \draw[<->] (3.3,0)--+(0,1.9) node[midway,right=-1]{
        $\frac{a^2+ab+b^2}{3(a+b)}$};
      \draw[<->] (0,1.9)--+(3.3,0) node[midway,above=-1]{
        $\frac{L(a+2b)}{2(a+b)}$};

      \fill[red] (3.3,1.9) circle (.08) node[right]{cm};
    \end{tikzpicture}
    \caption{Trapezoidal Area}
  \end{subfigure}
  \begin{subfigure}{.3\textwidth}
    \centering  
    \begin{tikzpicture}[scale=.5]
      \draw[mass] circle (2.5);
      \draw[->,rotate=-50] (0,0)--(2.5,0) node[midway,above=-1]{$r$};
      \fill[red] circle (.08) node[left]{cm};
    \end{tikzpicture}
    \caption{Circular Area}
  \end{subfigure}
  \begin{subfigure}{.3\textwidth}
    \centering  
    \begin{tikzpicture}[scale=.5]
      \draw[black!2] circle (2.5);
      \draw[mass] (2.5,0) arc (0:180:2.5)--(2.5,0);
      \draw[dash dot] (0,0)--(0,2.7);
      \draw[<->] (0,0)--+(0,1.1) node[midway,right=-1]{$\frac{4r}{3\pi}$};

      \draw[->,rotate=-45] (0,0)--(-2.5,0) node[midway,above=-1]{$r$};
      \fill[red] (0,1.1) circle (.08) node[above]{cm};
    \end{tikzpicture}
    \caption{Semi-Circular Area}
  \end{subfigure}
  \begin{subfigure}{.3\textwidth}
    \centering  
    \begin{tikzpicture}[scale=.5]
      \draw[black!2] circle (2.5);
      \draw[mass] (0,0)--(0,2.5) arc (90:180:2.5)--(0,0);
      \draw (-1.1,-.5)--(-1.1,1.1)--(.5,1.1);
      
      \draw[<->|] (.3,1.1)--(.3,0) node[midway,right=-1]{
        \small$\frac{4r}{3\pi}$};
      \draw[<->|] (-1.1,-.3)--(0,-.3) node[midway,below=-1]{
        \small$\frac{4r}{3\pi}$};
      \draw[->,rotate=-45] (0,0)--(-2.5,0) node[midway,above=-1]{$r$};
      \fill[red] (-1.1,1.1) circle (.08) node[left]{cm};
    \end{tikzpicture}
    \caption{Quarter-Circular Area}
  \end{subfigure}
\end{figure}

%\begin{center}
%  \pic{.7}{CM/eng130C9_11}
%\end{center}
%The locations of centroids can be found in most physics textbooks.


\section{Compound Shapes}
For compound shapes, the centre of mass is the weighted average of the centre
of mass of each component. For example, for the T-beam below:
\begin{figure}[ht]
  \centering
  \begin{subfigure}{.32\textwidth}
    \centering
    \begin{tikzpicture}[scale=.7]
      \draw[axes] (0,0)--(4.5,0) node[right]{$x$};
      \draw[axes] (0,0)--(0,4.5) node[left]{$y$};
      \draw[blue!70!black,mass] (0,4)--(4,4)--(4,3)--(2.5,3)
      --(2.5,0)--(1.5,0)--(1.5,3)--(0,3)--(0,4);
    \end{tikzpicture}
  \end{subfigure}
  \begin{subfigure}{.32\textwidth}
    \centering
    \begin{tikzpicture}[scale=.7]
      \draw[lightgray,fill=gray!20,thick] (0,4)--(4,4)--(4,3)--(2.5,3)--(2.5,0)
      --(1.5,0)--(1.5,3)--(0,3)--(0,4);
      \draw[lightgray,dotted,thick] (1.5,3)--(2.5,3);
      \draw[axes] (0,0)--(4.5,0) node[right]{$x$};
      \draw[axes] (0,0)--(0,4.5) node[left]{$y$};
      \fill[blue!70!black] (2,3.5) circle (.08) node[right]{$\bm x_1$};
      \fill[blue!70!black] (2,1.5) circle (.08) node[right]{$\bm x_2$};
    \end{tikzpicture}
  \end{subfigure}
  \begin{subfigure}{.32\textwidth}
    \centering
    \begin{tikzpicture}[scale=.7]
      \draw[lightgray,thick] (0,4)--(4,4)--(4,3)--(2.5,3)
      --(2.5,0)--(1.5,0)--(1.5,3)--(0,3)--(0,4);
      \draw[axes] (0,0)--(4.5,0) node[right]{$x$};
      \draw[axes] (0,0)--(0,4.5) node[left]{$y$};
      \fill[blue!30] (2,3.5) circle (.08) node[right]{$\bm x_1$};
      \fill[blue!30] (2,1.5) circle (.08) node[right]{$\bm x_2$};
      \fill[red!70!black] (2,2.64) circle (.08) node[right]{$\bm x_\text{cm}$};
    \end{tikzpicture}
  \end{subfigure}
\end{figure}


\section{Symmetric Configurations}
%  \begin{itemize}
Any plane of symmetry, mirror line, axis of rotation, point of inversion
\emph{must} contain the centre of mass. But there is a caveat: only works if
the density distribution is also symmetric Again, if density is uniform, the
centre of mass is also geometric centre (centroid).




\section{``Negative Mass''}%{A Mathematical Trick}

Where there is a ``hole'' in the geometry, treat it as having negative mass
density $-\sigma$ in that region. Negative masses don't exist, so this is
really just a trick.
\begin{example}
  What is the centre of mass of this shape?
  \begin{center}
    \begin{tikzpicture}
      \draw[thick,fill=lightgray] circle (2);
      \draw[thick,fill=black!2] (0,1) circle (1);
      \fill circle (.04);
      \draw[axes] (0,0)--(-1.41,-1.41)node[midway,below]{$r$};
      \fill (0,1) circle (.04);
      \draw[axes] (0,1)--(.707,.293) node[right]{$r/2$};
    \end{tikzpicture}
  \end{center}

  \textbf{Solution:} This is how we would think of it:
  \begin{center}
    \begin{tikzpicture}[scale=.6]
      \draw[thick,fill=lightgray] circle (2);
      \draw[thick,fill=black!2] (0,1) circle (1);
      \fill circle (.04);
      \draw[axes] (0,0)--(-1.41,-1.41) node[midway,below]{$r$};
      \fill (0,1) circle (.04);
      \draw[axes] (0,1)--(.707,.293) node[right]{$r/2$};
      \draw[thick,fill=lightgray] (6,0) circle (2) node{\large$A$};
      \draw[thick] (11,1) circle (1) node{\large$B$};
      \node at (3,0) {\huge=};
      \node at (9,0) {\huge+};
    \end{tikzpicture}
  \end{center}
  Let the origin of the coordinate system to located at the centre of $A$.
  Based on symmetry: $x_\text{cm}=0$; only have to find $y$-coordinate.
  \begin{equation}
    y_\text{cm}
    =\frac{\sum y_i m_i}{\sum m_i}
    =\frac{m_A(0) + m_B (r/2)}{m_A+m_B}
    =\frac{-\sigma\pi\left(r/2\right)^2(r/2)}
    {\sigma\pi r^2-\sigma\pi\left(r/2\right)^2}
    =\frac{-r}6
  \end{equation}
\end{example}



%\section{Momentum and Centre of Mass}

\section{Velocity of the Centre of Mass}
Taking the change in the position of the centre of mass ($\Delta\bm x_\text{cm}$)
over a finite time interval ($\Delta t$) gives the expression for
$\bm v_\text{cm}$, the average velocity of the centre of mass:
\begin{equation*}
  \bm v_\text{cm}
  =\frac{\color{magenta}\Delta\bm x_\text{cm}}{\Delta t}
  =\frac1{\Delta t}
  {\color{magenta}\left[\frac{\sum\Delta(m_i\bm x_i)}{\sum m_i}\right]}
  =\frac1{m_\text{tot}}
  \left[\sum m_i{\color{orange}\frac{\Delta\bm x_i}{\Delta t}}\right]
  =\frac1{m_\text{tot}}\sum m_i{\color{orange}\bm v_i}
\end{equation*}
The velocity of the centre of mass is the weighted sum of the velocities of
the discrete masses:
\begin{equation}
  \bm v_\text{cm} = \frac{\sum m_i\bm v_i}{m_\text{tot}}
\end{equation}
Like the expression for the position of the centre of mass, the weight for the sum is the
individual masses.



\subsection{Velocity and Momentum}

We can rearrange the equation for the velocity of the centre of mass to relate
it to momentum, because the term $\sum m_i\bm v_i=\bm p_\text{net}$ is the net
momentum of \emph{all} the discrete masses:
\begin{equation}
  \bm v_\text{cm} = \frac{\sum m_i\bm v_i}{m_\text{tot}}
  \quad\longrightarrow\quad
  \bm p_\text{net}= \sum m_i\bm v_i = m_\text{tot}\bm v_\text{cm}
\end{equation}
During a collision, there is no change in the net momentum\footnote{Because
there are are no external forces}, the centre of mass will continue to move at
the same velocity before/after the collision, as if the collision never
occurred.


\subsection{Centre of Mass During Collision}

During a collision,\footnote{As we have studied in conservation of momentum in
Chapter~\ref{chapter:momentum}}, there are no external forces, therefore the
velocity of the centre of mass remains constant. Consider this perfectly
inelastic collision in 1D  between two masses.
\begin{figure}[ht]
  \centering
  \begin{subfigure}{.45\textwidth}
    \centering
    \begin{tikzpicture}[thick]
      \draw[mass] (1,0) rectangle (2,1) node[midway]{$m_1$};
      \draw[mass] (4,0) rectangle (5,1) node[midway]{$m_2$};
      \draw[vectors] (.8,1.3)--(2.5,1.3) node[right]{$v_1$};
      \draw[vectors] (4,1.3)--(5,1.3) node[right]{$v_2$};

      \draw (0,0)--(6,0);
      
      \draw (2.7,.5) circle (.15);
      \fill (2.7,.5)--(2.85,.5) arc(0:90:.15)--cycle;
      \fill (2.7,.5)--(2.55,.5) arc(180:270:.15)
      node[below=-2] (c){\scriptsize cm\par}--cycle; 
      \draw[vectors] (2.7,.5)--(3.7,.5) node[above]{$v_\text{cm}$};
      
%      \draw (11,.5) circle (.15);
%      \fill (11,.5)--(11.15,.5) arc(0:90:.15)--cycle;
%      \fill (11,.5)--(10.85,.5) arc(180:270:.15)
%      node[below=-2] {\scriptsize cm\par}--cycle; 
%      
%      \node[text width=2.07in,draw,violet,below right] (a) at (-.5,-.6)
%           {Using the definition of the velocity of the centre of mass, we find
%             that \emph{before} the collision, the centre of mass moves at:
%             \vspace{-.05in}
%             \begin{displaymath}
%               v_\text{cm} = \frac{\sum m_iv_i}{\sum m_i}
%               =\frac{m_1v_1+m_2v_2}{m_1+m_2}
%             \end{displaymath}\par
%           };
%           \draw[axes,violet] (a)--(c);
%           
%           \node[text width=2in,draw,orange,below left] at (14.5,-.6)
%                {Using conservation of momentum, we find the final velocity
%                  \emph{after} the collision is the velocity of the centre of mass:
%                  \vspace{-.05in}
%                  \begin{displaymath}
%                    v'=\frac{m_1v_1+m_2v_2}{m_1+m_2}=v_\text{cm}
%                  \end{displaymath}
%                  %\vspace{-.08in}It is the same as $v_\text{cm}$ before the
%                  %collision!
%                  \par
%                };
    \end{tikzpicture}
    \caption{Before Collision}
  \end{subfigure}
  \begin{subfigure}{.45\textwidth}
    \centering
    \begin{tikzpicture}[thick]
      \draw[mass] (10,0) rectangle (11,1) node[midway]{$m_1$};
      \draw[mass] (11,0) rectangle (12,1) node[midway]{$m_2$};
      \draw[vectors] (10.2,1.3)--(11.8,1.3) node[right]{$v'$};

      \draw (9,0)--(13,0);
      
      \draw (11,.5) circle (.15);
      \fill (11,.5)--(11.15,.5) arc(0:90:.15)--cycle;
      \fill (11,.5)--(10.85,.5) arc(180:270:.15)
      node[below=-2] {\scriptsize cm\par}--cycle; 
      
%      \node[text width=2.07in,draw,violet,below right] (a) at (-.5,-.6)
%           {Using the definition of the velocity of the centre of mass, we find
%             that \emph{before} the collision, the centre of mass moves at:
%             \vspace{-.05in}
%             \begin{displaymath}
%               v_\text{cm} = \frac{\sum m_iv_i}{\sum m_i}
%               =\frac{m_1v_1+m_2v_2}{m_1+m_2}
%             \end{displaymath}\par
%           };
%           \draw[axes,violet] (a)--(c);
%           
%           \node[text width=2in,draw,orange,below left] at (14.5,-.6)
%                {
    \end{tikzpicture}
    \caption{After Collision}
  \end{subfigure}
\end{figure}

Using conservation of momentum that we have found in
Eq.~\ref{eq:momentum-same}, we find the final velocity ($v'$) \emph{after} the
collision is the velocity of the centre of mass $v_\text{cm}$!
\begin{displaymath}
  v'=\frac{m_1v_1+m_2v_2}{m_1+m_2}=v_\text{cm}
\end{displaymath}
In other words, during a collision in isolation, when there is no external
forces, and momentum is conserved, the centre of mass of the system of masses
moves with a constant velocity
%\vspace{-.08in}It is the same as $v_\text{cm}$ before the collision!


\section{Acceleration of the Centre of Mass}
Finding the rate of change of the net momentum (i.e\ applying the 2nd law of
motion to this collection of masses):
\begin{equation}
  \frac{\Delta\bm p_\text{net}}{\Delta t}=
    \frac{\Delta(m\bm v_\text{cm})}{\Delta t}
\end{equation}
If the system mass is constant, then this equation reduces to:
\begin{equation}
  \frac{\Delta\bm p_\text{net}}{\Delta t}
  =m\frac{\Delta\bm v_\text{cm}}{\Delta t}
  \quad\longrightarrow\quad
  \boxed{
    \bm F_\text{net}=m\bm a_\text{cm}
  }
\end{equation}
We can see that when a net force is applied to an object, the object's
acceleration is evaluated at the centre of mass.

%{What This All Means}
%  \begin{itemize}
%  \item Newton was right all along by treating all objects as point masses
%    located at the centre of mass
%  \end{itemize}
