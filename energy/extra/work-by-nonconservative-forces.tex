\documentclass[12pt,compress,aspectratio=169]{beamer}
\input{../../mybeamer}

\usetikzlibrary{decorations.pathmorphing,patterns}

\title{Work by Non-Conservative Forces}
\date{}
%\input{../me}
\input{../../mycommands}

\begin{document}

\begin{frame}
  \maketitle
\end{frame}




\begin{frame}{Non-Conservative Force}
  While the list of conservative forces is very short; the majority of forces
  are \textbf{non-conservative}. Examples of non-conservative forces include:
  \begin{itemize}
  \item Aerodynamic drag (fluid resistance) and lift
  \item Static and kinetic friction
  \item Applied force
  \item Tension force
  \item Normal force
  \end{itemize}
\end{frame}



\begin{frame}{Non-Conservative Force}
  The work done by non-conservative forces differs from conservative forces in
  that:
  \begin{itemize}
  \item There is no related potential energies
  \item Transforms one form of \emph{kinetic} energy to another, or
  \item Transforms kinetic energy from one object to another
  \item Transforms kinetic energy into \emph{internal} energy
  \item Work is path dependent
  \end{itemize}
\end{frame}



\begin{frame}{Work by Static Friction}
  We can learn about the work by non-conservative forces by studying the Work
  by static friction in a multi-body example studied in dynamics. Two blocks
  ($m_1$ and $m_2$), stacked vertically, move to the right without
  slipping\footnote{This means that both blocks travel the same displacement
  $\Delta d$} by external force $\vec F$ :
  \begin{center}
    \begin{tikzpicture}[scale=.75]
      \fill[pattern=north east lines] rectangle (7,-.2);
      \draw[thick] (0,0)--(7,0);
      \draw[thick] (1,0) rectangle (4,1.2) node[midway]{$m_1$};
      \draw[thick] (1.7,1.2) rectangle (3.3,2) node[midway]{$m_2$};
      \draw[vectors] (4,.6)--(6,.6) node[right]{$\vec F$};
      \draw[<-] (.95,.05) to[out=150,in=0] (0,.5) node[left]{frictionless};
      \draw[<-] (3.35,1.25) to[out=20, in=180] (4.5,1.8) node[right]{$\mu$};
    \end{tikzpicture}
  \end{center}
  \uncover<2>{
    The FBDs of the blocks are shown below. (Highlighted forces are
    action-reaction pairs.)
    \begin{center}
      \vspace{-.1in}
      \begin{tikzpicture}[scale=.8,vectors,orange]
        \fill circle (.1);
        \draw (0,0)--(0,-1) node[below]{$m_2\vec g$};
        \draw (0,0)--(0, 1) node[above,fill=pink!10]{$\vec N_{12}$};
        \draw (0,0)--(1.5,0) node[right,fill=yellow!10]{$\vec f_s$};
      \end{tikzpicture}
      \hspace{.4in}
      \begin{tikzpicture}[scale=.8,vectors,violet]
        \fill circle (.1);
        \draw (-.05,0)--(-.05,-1) node[below left]{$m_1\vec g$};
        \draw (.05,0)--(.05,-1) node[below right,fill=pink!10]{$\vec N_{12}$};
        \draw (0,0)--(0, 1) node[above]{$\vec N_1$};
        \draw (0,0)--(-1.5,0) node[left,fill=yellow!10]{$\vec f_s$};
        \draw (0,0)--(2,0) node[right]{$\vec F$};
      \end{tikzpicture}
    \end{center}
  }
\end{frame}



\begin{frame}{Work by Static Friction}
  \begin{columns}
    \column{.25\textwidth}
    \centering
    \begin{tikzpicture}[orange,vectors]
      \fill circle (.08);
      \draw (0,0)--(0,-1) node[below]{$m_2\vec g$};
      \draw (0,0)--(0,1) node[above]{$\vec N_{12}$};
      \draw (0,0)--(1.5,0) node[right,fill=yellow!20]{$\vec f_s$};
    \end{tikzpicture}

    \column{.75\textwidth}
    For the top block $m_2$, when it moves to the right
    \begin{itemize}
    \item Friction between the blocks $\vec f_s$ is the only force doing work
    \item The work done by $\vec f_s$ is positive
    \item Mass $m_2$ gains kinetic energy: $W_f=\Delta K>0$
    \end{itemize}
  \end{columns}
  \uncover<2>{
    \begin{columns}
      \column{.65\textwidth}
      For the bottom block $m_1$, when it moves to the right
      \begin{itemize}
      \item $\vec F$ does positive work on $m_1$, while
      \item $\vec f_s$ does negative work on $m_1$
      \item Therefore $\vec f_s$ decreases the kinetic energy of $m_1$
        \begin{itemize}
        \item Without friction, $m_1$ would move faster
        \end{itemize}
      \end{itemize}
    
      \column{.35\textwidth}
      \centering
      \begin{tikzpicture}[violet,vectors]
        \fill circle (.08);
        \draw (0,0)--(0,-1) node[below]{$m_1\vec g+\vec N_{12}$};
        \draw (0,0)--(0, 1) node[above]{$\vec N_1$};
        \draw (0,0)--(-1.5,0) node[left,fill=yellow!20]{$\vec f_s$};
        \draw (0,0)--(2,0) node[right]{$\vec F$};
      \end{tikzpicture}    
    \end{columns}
  }
\end{frame}



\begin{frame}{Work by Static Friction}
  The work done by static friction is
  \begin{itemize}
  \item positive on one object ($m_2$)
  \item negative on another ($m_1$)
  \end{itemize}
  Therefore, work by friction transforms the kinetic energy of $m_1$ into the
  kinetic energy of $m_2$ by the same amount
\end{frame}



\begin{frame}{Work by Kinetic Friction}
  But what if the blocks \emph{do} slide against each other?
  \begin{center}
    \begin{tikzpicture}[scale=.72]
      \fill[pattern=north east lines] rectangle (7,-.2);
      \draw[thick] (0,0)--(7,0);
      \draw[thick] (1,0) rectangle (4,1.2) node[midway]{$m_1$};
      \draw[thick] (1.7,1.2) rectangle (3.3,2) node[midway]{$m_2$};
      \draw[vectors] (4,.6)--(6,.6) node[right]{$\vec F$};
      \draw[<-] (.95,.05) to[out=150,in=0] (0,.5) node[left]{frictionless};
      \draw[<-] (3.35,1.25) to[out=20, in=180] (4.5,1.8) node[right]{$\mu$};
    \end{tikzpicture}
  \end{center}
  \uncover<2>{
    The FBDs of the blocks are shown below\ldots with slight differences. The
    friction force is still an action-reaction pair, but now the friction is
    \emph{kinetic}.
    \begin{center}
      \vspace{-.1in}
      \begin{tikzpicture}[scale=.8,vectors,orange]
        \fill circle (.1);
        \draw (0,0)--(0,-1) node[below]{$m_2\vec g$};
        \draw (0,0)--(0, 1) node[above,fill=pink!10]{$\vec N_{12}$};
        \draw (0,0)--(1.5,0) node[right,fill=yellow!10]{$\vec f_k$};
      \end{tikzpicture}
      \hspace{.4in}
      \begin{tikzpicture}[scale=.8,vectors,violet]
        \fill circle (.1);
        \draw (-.05,0)--(-.05,-1) node[below left]{$m_1\vec g$};
        \draw (.05,0)--(.05,-1) node[below right,fill=pink!10]{$\vec N_{12}$};
        \draw (0,0)--(0, 1) node[above]{$\vec N_1$};
        \draw (0,0)--(-1.5,0) node[left,fill=yellow!10]{$\vec f_k$};
        \draw (0,0)--(2,0) node[right]{$\vec F$};
      \end{tikzpicture}
    \end{center}
  }
\end{frame}



\begin{frame}{Work by Kinetic Friction}
  \begin{columns}
    \column{.25\textwidth}
    \centering
    \begin{tikzpicture}[orange,vectors]
      \fill circle (.08);
      \draw (0,0)--(0,-1) node[below]{$m_2\vec g$};
      \draw (0,0)--(0,1) node[above]{$\vec N_{12}$};
      \draw (0,0)--(1.5,0) node[right,fill=yellow!20]{$\vec f_k$};
    \end{tikzpicture}

    \column{.75\textwidth}
    For the top block $m_2$, when it moves to the right
    \begin{itemize}
    \item Kinetic friction between the blocks $\vec f_k$ is the only force
      doing (positive) work
    \item Mass $m_2$ gains kinetic energy: $W_f=\Delta K>0$
    \end{itemize}
    This is the same as the static friction case, except that $f_k\leq\max f_s$
  \end{columns}
  \uncover<2>{
    \begin{columns}
      \column{.65\textwidth}
      For the bottom block $m_1$, when it moves to the right
      \begin{itemize}
      \item $\vec F$ does positive work on $m_1$, while
      \item Kinetic friction $\vec f_k$ does negative work on $m_1$
      \item Therefore $\vec f_s$ decreases the kinetic energy of $m_1$
        \begin{itemize}
        \item Without friction, $m_1$ would move faster
        \end{itemize}
      \end{itemize}
    
      \column{.35\textwidth}
      \centering
      \begin{tikzpicture}[violet,vectors]
        \fill circle (.08);
        \draw (0,0)--(0,-1) node[below]{$m_1\vec g+\vec N_{12}$};
        \draw (0,0)--(0, 1) node[above]{$\vec N_1$};
        \draw (0,0)--(-1.5,0) node[left,fill=yellow!20]{$\vec f_k$};
        \draw (0,0)--(2,0) node[right]{$\vec F$};
      \end{tikzpicture}    
    \end{columns}
  }
\end{frame}



\begin{frame}{Work by Kinetic Friction}
  The work done by static friction is
  \begin{itemize}
  \item positive on one object ($m_2$)
  \item negative on another ($m_1$)
  \end{itemize}
  Therefore, work by friction transforms the kinetic energy of $m_1$ into the
  kinetic energy of $m_2$ by the same amount.
\end{frame}



\begin{frame}{Work by Kinetic Friction}
  Where kinetic friction is different from static friction is that:
  \begin{itemize}
  \item The displacement of the blocks are different: $m_1$ moves farther than
    $m_2$
  \item The positive work by $f_k$ on $m_2$ is less than the negative work done
    by $f_k$ on $m_1$
  \item<2->Kinetic friction took more energy away from $m_1$ than what is put
    into $m_2$
  \end{itemize}
  \uncover<3->{\textbf{Where did the energy go?}}
  \begin{itemize}
  \item<4-> It went into heating up the interface between the blocks
  \item<4-> Transformed kinetic energy of $m_1$ into
    \begin{itemize}
    \item Internal energy of $m_1$ and $m_2$
    \item Kinetic energy of $m_2$ 
    \end{itemize}
  \end{itemize}
\end{frame}
\end{document}
