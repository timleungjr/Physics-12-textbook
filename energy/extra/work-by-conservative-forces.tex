\documentclass[12pt,compress,aspectratio=169]{beamer}
\input{../../mybeamer}

\usetikzlibrary{decorations.pathmorphing,patterns}

\title{Work by Conservative Forces}
\date{}
%\subtitle{Unit 2: Energy and Momentum}
%\input{../term}
\input{../../mycommands}

\begin{document}

\begin{frame}
  \maketitle
\end{frame}



\begin{frame}{Gravitational Force \& Gravitational Potential Energy}
  Consider an object free-falling under the force of gravity over a
  displacement $\Delta\vec d$:
  \begin{center}
    \begin{tikzpicture}[scale=.6]
      \draw[thick,fill=gray!30] (7.75,0) arc (75:105:30);
      \draw[mass] (0,4) circle (.2) node[right=2]{$m$};
      \draw[vectors,red] (0,4)--(0,2) node[right]{$m\vec g$};
      \draw[vectors] (-.3,4)--+(0,-2.5) node[above left]{$\Delta\vec d$};
    \end{tikzpicture}
  \end{center}
  \uncover<2>{
    \begin{itemize}
    \item\vspace{-.2in}Assuming that $\Delta\vec d$ is small, $\vec g$ can be
      considered to be constant
    \item The work done by the gravity ($W_g=mg\Delta d$) on the object is
      \emph{positive}
    \item Since gravity is the only force acting on the object,
      $W_g=W_\text{net}$, therefore, there is an increase in kinetic energy.
      The object speeds up.
      
      \eq{-.2in}{
        W_g=W_\text{net}=\Delta K > 0
      }
    \end{itemize}
  }
\end{frame}



\begin{frame}{Gravitational Potential Energy}
  Work done by gravity can also be expressed using the change in height
  ($\Delta d=h_1-h_2$):% Using the ground as the ``reference level'' (i.e.\
  %where $h=0$):
  \begin{center}
    \begin{tikzpicture}[scale=.55]
      \draw[thick,fill=gray!30] (7.75,0) arc (75:105:30);
      \draw[mass] (0,6) circle (.25);
      \draw[vectors,violet] (0,6)--(0,3) node[midway,right]{$\Delta d$};
      \begin{scope}[thick,magenta]
        \draw[->|] (-.4,1)--(-.4,6) node[midway,left]{$h_1$};
        \draw[->|] (0,1)--(0,3) node[midway,right]{$h_2$};
        \draw[dash dot] (-2,1.02)--(4,1.02) node[right]{$h=0$};
      \end{scope}
    \end{tikzpicture}
  \end{center} 

  \eq{-.4in}{
    {\color{blue}W_g} = mg{\color{magenta}(h_1-h_2)} =
    {\color{magenta}-}mg{\color{magenta}(h_2-h_1)}=-(mgh_2-mgh_1) =
    {\color{blue}-\Delta U_g}
  }
\end{frame}



\begin{frame}{Gravitational Potential Energy}
  Defining the \textbf{gravitational potential energy} $U_g$ as:

  \eq{-.1in}{
    \boxed{U_g=mgh}
  }

  The work done by gravity is related to this gravitationalpotential energy by:
  
  \eq{-.1in}{
    \boxed{
      W_g=-\Delta U_g=-mg\Delta h }
  }

  \uncover<2>{
    \fcolorbox{black}{yellow!15}{
      \begin{minipage}{.95\textwidth}
        \begin{itemize}
        \item\emph{Positive} work by gravity \emph{decreases} gravitational
          potential energy, while
        \item\emph{Negative} work by gravity \emph{increases} gravitational
          potential energy
        \item $W_g$ depends on the end points $h_1$ and $h_2$, but not
          \emph{how} it moves from $h_1\rightarrow h_2$
        \item Only work done by gravity can change $U_g$
        \end{itemize}
      \end{minipage}
    }
  }
\end{frame}



\begin{frame}{Work Done by Gravity---Path Independence}
  The change in gravitational potential energy (i.e.\ work done by gravity)
  depends only on the weight ($mg$) of the object, and the end points ($h_1$ and
  $h_2$), but not the path:
  \begin{center}
    \begin{tikzpicture}
      \draw[thick,dashed] (0,0)--(12,0) node[right]{$h_1$};
      \draw[thick,dashed] (0,-2)--+(12,0) node[right]{$h_2$};

      \uncover<2->{
        \fill[red] (1,0) circle (.07);
        \draw[vectors,red] (1,0)--+(0,-2);
        \draw[thick,red] (1,-2) circle (.07);
        \node[below,red] at (1,-2.3){Dropped};
      }
      \uncover<3->{
        \fill[violet] (3.5,0) circle (.07);
        \draw[vectors,violet] (3.5,0)--+(0,1.2) arc(180:0:.05)--+(0,-3.2);
        \draw[thick,violet] (3.6,-2) circle (.07);
        \node[below,violet] at (3.55,-2.3){Thrown straight up};
      }
      \uncover<4->{
        \fill[orange] (5.5,0) circle (.07);
        \draw[vectors,orange] (5.5,0) to[out=50,in=120] +(2,-2);
        \draw[thick,orange] (7.5,-2) circle (.07);
        \node[below,orange] at (6.5,-2.3){Projectile};
      }
      \uncover<5->{
        \fill[magenta] (8.5,0) circle (.07);
        \draw[vectors,magenta] (8.5,0) to[out=-50,in=230] +(2.5,-2);
        \draw[thick,magenta] (11,-2) circle (.07);
        \node[below,magenta] at (9.75,-2.3){Arbitrary ramp/slide };
      }
    \end{tikzpicture}
  \end{center}
  \uncover<6>{
    $W_g$ (this is not necessarily $W_\text{net}$) is the same in all above cases
    because they all have the same $h_1$ and $h_2$.
  }
\end{frame}



\begin{frame}{Work Done by Gravitational Force}
  \begin{center}
    \begin{tikzpicture}[scale=.5]
      \draw[thick,fill=gray!30] (7.75,0) arc (75:105:30);
      \draw[mass] (0,4) circle (.2) node[right=2]{$m$};
      \draw[vectors,red] (0,4)--(0,2) node[right]{$m\vec g$};
    \end{tikzpicture}
  \end{center}
  From the example of the \underline{free-falling} object, we can see
  \emph{how} energy is transformed. \uncover<2->{The positive work done by
    gravity increases kinetic energy}

  \eq{-.1in}{
    \uncover<2->{
      W_g = \Delta K
    }
    \uncover<3>{
      =-\Delta U_g
    }
  }
  
  \uncover<3>{\vspace{-.2in}while simultaneously decreases gravitational
    potential energy by the same amount}
\end{frame}



%\begin{frame}{Work Done by Gravitational Force}
%  \begin{columns}[T]
%    \column{.47\textwidth}
%    \begin{center}
%      \begin{tikzpicture}[scale=.75]
%      \draw[mass] (0,4) circle (.2) node[right=2]{$m$};
%      \draw[vectors,red] (0,4)--+(0,-1.5) node[right]{$m\vec g$};
%      \draw[vectors,white] (-.3,4)--+(0,2.5) node[right]{$\Delta\vec d$};
%      \draw[vectors,blue] (-.3,4)--+(0,-2.5) node[right]{$\Delta\vec d$};
%      \end{tikzpicture}
%    \end{center}   
%
%    {\footnotesize When the object is free-falling downwards, $W_g>0$.
%      $K$ increases while $U_g$ decreases by the same amount.\par
%    }
%    \column{.47\textwidth}
%    \uncover<2>{
%      \begin{center}
%        \begin{tikzpicture}[scale=.75]
%          \draw[mass] (0,4) circle (.2) node[right=2]{$m$};
%          \draw[vectors,red] (0,4)--+(0,-1.5) node[right]{$m\vec g$};
%          \draw[vectors,blue] (-.3,4)--+(0,2.5) node[right]{$\Delta\vec d$};
%          \draw[vectors,white] (-.3,4)--+(0,-2.5) node[right]{$\Delta\vec d$};
%        \end{tikzpicture}
%      \end{center}
%    
%      {\footnotesize When the object is free-falling, but moving upwards,
%        $W_g<0$. $K$ \emph{decreases} while $U_g$
%        \emph{increases} by the same amount.\par
%      }
%    }
%  \end{columns}
%\end{frame}



\begin{frame}{Work Done by Gravitational Force}
  Work done by gravity transforms gravitational potential energy into kinetic
  energy when it does {\color{blue}positive} work (e.g.\ dropping something):
  
  \vspace{-.4in}{\huge
    \begin{displaymath}
      U_g\longrightarrow K
    \end{displaymath}
  }

  \uncover<2>{
    \vspace{.1in}Work done by gravity transforms kinetic energy into
    gravitational potential energy when it does {\color{magenta}negative} work
    (e.g.\ tossing something upwards):
  
    \vspace{-.4in}{\huge
      \begin{displaymath}
        U_g\longleftarrow K
      \end{displaymath}
    }
  }
\end{frame}



\begin{frame}{Spring Force \& Elastic Potential Energy}
  The spring force $\vec F_e$ is the force that a compressed/stretched spring
  exerts on the object connected to it. An \emph{ideal} spring obeys
  \textbf{Hooke's law}:
    
  \eq{-.1in}{
    \boxed{
      \vec F_e=-k\vec x
    }
  }

  \vspace{-.1in}$\vec F_e$ is in the opposite direction to the spring's
  displacement $\vec x$, and is proportional to the amount of
  compression/stretching.
  \begin{center}
    \begin{tikzpicture}[scale=.8]
      \draw[mass] (5,.5) rectangle (6,1.5);
      \draw[thick,
        decoration={aspect=.6,segment length=5mm, amplitude=2.5mm, coil},
        decorate] (0,1)--(5,1);
      \fill[pattern=north east lines] (-.2,0) rectangle (0,2);
      \draw[thick] (0,0)--(0,2);
      \fill[red] (5.5,1) circle (.06);
      \draw[vectors,red] (5.5,1)--(4,1) node[above]{$\vec F_e$};
      \draw[dashed] (3,0)--(3,2) node[above]{\scriptsize Equilibrium position};
      \draw[vectors] (3,.3)--(5,.3) node[midway,below]{$x$};
    \end{tikzpicture}
    \hspace{.2in}
    \begin{tikzpicture}[scale=.8]
      \draw[thick,gray!40,fill=gray!20] (5,.5) rectangle (6,1.5);
      \draw[thick,gray!20,
        decoration={aspect=.6,segment length=5mm, amplitude=2.5mm, coil},
        decorate] (0,1)--(5,1);
      \fill[pattern=north east lines] (-.2,0) rectangle (0,2);
      \draw[thick] (0,0)--(0,2);
      \fill[gray!30] (5.5,1) circle (.06);
      \draw[vectors,gray!30] (5.5,1)--(4,1) node[above]{$\vec F_e$};
      \draw[dashed] (3,0)--(3,2);
      \draw[vectors,gray!30] (3,.3)--(5,.3) node[midway,below]{$x$};
      \draw[mass] (1.5,.5) rectangle (2.5,1.5);
      \draw[thick,
        decoration={aspect=.3,segment length=1.5mm, amplitude=2.5mm, coil},
        decorate] (0,1)--(1.5,1);
      \draw[vectors] (3,.3)--(1.5,.3) node[midway,below]{$x$};
      \fill[red] (2,1) circle (.06);
      \draw[vectors,red] (2,1)--(3,1) node[above]{$\vec F_e$};
    \end{tikzpicture}
  \end{center}
%  The constant $k$ (called the \textbf{spring constant}, \textbf{force
%    constant}, \textbf{Hooke's constant} or \textbf{spring rate}) is the
%  stiffness of the spring. It has a unit of \si{\newton\per\metre}.
\end{frame}



\begin{frame}{Elastic Potential Energy}
  Calculating the work done by the spring force to the mass attached to it
  requires some
  calculus %\footnote{Calculus is necessary here because the force is not
  %constant, but you do not need to know this for Physics 12.}
  that is not shown here, but we can focus on the end result:

  \eq{-.1in}{
    \underbrace{
      {\color{blue}W_e}=\int_{x_1}^{x_2}F_edx}_\text{definition using calculus}
    =\cdots
    =-\left(\frac12kx_2^2-\frac12kx_1^2\right) = {\color{blue}-\Delta U_e}
  }
  
  Defining the \textbf{elastic potential energy} $U_e$:

  \eq{-.1in}{
    U_e=\frac12kx^2
  }
\end{frame}



\begin{frame}{Elastic Potential Energy}
  The work done by the spring force can be related to the elastic potential
  energy by:
  
  \eq{-.1in}{
    \boxed{  W_e=-\Delta U_e }
  }

  \uncover<2->{
    \fcolorbox{black}{yellow!15}{
      \begin{minipage}{.95\textwidth}
        \begin{itemize}
        \item\emph{Positive} work done by the spring \emph{decreases} elastic
          potential energy, while
        \item\emph{Negative} work done by the spring \emph{increases} elastic
          potential energy
        \item $W_e$ depends on the end points $x_1$ and $x_2$, but not
          \emph{how} it moves from $x_1\rightarrow x_2$
        \item Only work done by the spring force can change $U_e$
        \end{itemize}
      \end{minipage}
    }
  
    \vspace{.1in}These properties are identical to that of gravitational force
  }
\end{frame}



\begin{frame}{Work Done by Spring Force}
  As the mass moves towards the equilibrium position, $W_e>0$
  (i.e.\ motion and $F_s$ are in the same direction), \uncover<2->{the work
    done by the spring increases the kinetic energy of the object,}
  \uncover<3->{while simultaneously decreases elastic potential enegy $U_e$ by
    the same mount
  }

  \eq{-.2in}{
    \uncover<2->{
      W_e=\Delta K
    }
    \uncover<3->{
      =-\Delta U_e
    }
  }
  \uncover<4>{
    \begin{center}
      \begin{tikzpicture}[scale=.7]
        \draw[mass] (5,.5) rectangle (6,1.5);
        \draw[thick,
          decoration={aspect=.6,segment length=5mm, amplitude=2.5mm, coil},
          decorate] (0,1)--(5,1);
        \fill[pattern=north east lines] (-.2,0) rectangle (0,2);
        \draw[thick] (0,0)--(0,2);
        \fill[red] (5.5,1) circle (.06);
        \draw[vectors,red] (5.5,1)--(4,1) node[above]{$\vec F_e$};
        \draw[dashed] (3,0)--(3,2) node[above]{\scriptsize Equilibrium position};
        \draw[vectors] (3,.3)--(5,.3) node[midway,below]{$x$};
      \end{tikzpicture}
    \end{center}
    Similar to gravity, work done by spring transforms energy from
    elastic potential energy into kinetic energy, and vice versa.
  }
\end{frame}



\begin{frame}{Work Done by Spring Force}
  Work done by the spring transforms elastic potential energy into kinetic
  energy when it does {\color{blue}positive} work (i.e.\ the mass is moving
  towards equilibrium position):
  
  \vspace{-.4in}{\huge
    \begin{displaymath}
      U_e\longrightarrow K
    \end{displaymath}
  }

  \uncover<2>{
    \vspace{.1in}Work done by the spring transforms kinetic energy into elastic
    potential energy when it does {\color{magenta}negative} work (i.e.\ moving
    away from equilibrium):
  
    \vspace{-.4in}{\huge
      \begin{displaymath}
        U_e\longleftarrow K
      \end{displaymath}
    }
  }
\end{frame}


%%\begin{frame}{Mass-Spring Simulation}
%%  \begin{center}
%%    \textbf{Click for external link:}
%%    \href{https://phet.colorado.edu/sims/html/hookes-law/latest/hookes-law_en.html}
%%         {Hooke's Law}
%%  \end{center}
%%\end{frame}
%
%
%
%\begin{frame}{Example Problem}
%  \textbf{Example:} A typical compound archery bow requires a force of
%  \SI{133}{\newton} to hold an arrow at ``full draw'' (pulled back) of
%  \SI{71}{\centi\metre}. Assuming that the bow obeys Hooke's law, what is its
%  spring constant?
%\end{frame}
%
%
%
%%\begin{frame}{Elastic Potential Energy}
%%  \begin{columns}
%%    \column{.25\textwidth}
%%    \begin{tikzpicture}[scale=1.2]
%%      \draw[->] (0,0)--(2,0)node[right]{$\Delta d$};
%%      \draw[->] (0,0)--(0,4)node[above]{$F$};
%%      \draw[very thick, red] (0,0)--(1.5,3);
%%      \draw[dashed] (1.25,0)--(1.25,2.5);
%%      \draw[dashed] (0,2.5)--(1.25,2.5);
%%      \draw[<->] (0,-0.15)--(1.25,-0.15) node[midway,below]{$x$};
%%      \draw[<->] (-0.15,0)--(-0.15,2.5) node[midway,left]{$kx$};
%%      \fill[red] (1.25,2.5) circle (.06);
%%
%%      \uncover<2->{
%%        \draw[fill=gray,draw=gray] (0,0)--(1.25,0)--(1.25,2.5)--cycle;
%%      }
%%    \end{tikzpicture}
%%    \column{.75\textwidth}
%%    \begin{itemize}
%%    \item Work done to extend/compress a spring is the area under the
%%      force-displacement graph
%%    \item If we use some calculus, we can see that work
%%      done is the potential stored in the spring
%%
%%      \eq{-.2in}{\boxed{W=U_e=\frac12 kx^2}}
%%    \end{itemize}
%%  \end{columns}
%%\end{frame}
%%
%%
%%
%%\begin{frame}{Example Problem}
%%  \textbf{Example:} A spring with spring constant of \SI{75}{\newton\per\metre}
%%  is resting on a table.
%%  \begin{itemize}
%%  \item If the spring is compressed by \SI{28}{\centi\metre}, what is the
%%    increase in its potential energy?
%%  \item What force must be applied to hold the spring in this position?
%%  \end{itemize}
%%\end{frame}
%
%
%
%\section{Conservative vs.\ Non-Conservative Forces}

\begin{frame}{Conservative Forces}
  The highlighted properties only apply to very few forces, called
  \textbf{conservative forces}:
  \begin{itemize}
  \item Gravitational force $\vec F_g$
  \item Spring force $\vec F_e$ (only for ideal springs)
  \item Electromagnetic forces
  \item Nuclear forces
  \end{itemize}
  They are called conservative because they conserve the mechanical energy
  of the object.

  \uncover<2>{
    \begin{itemize}
    \item Work done by a conservative force causes a change in a related
      potential energy
      
      \eq{-.1in}{
        \boxed{ W_c=-\Delta U }
      }
    \item Only conservative force can change the potential energy
    \item Work done by a conservative force is \emph{path independent}
    \end{itemize}
  }
\end{frame}



\begin{frame}{Work by Conservative Forces}
  The work done by a conservative force \emph{always} transforms the related
  potential energy into kinetic energy, and vice versa:
  \begin{itemize}
  \item It transforms potential energy into kinetic energy when it
    does {\color{blue}positive} work:

    \vspace{-.4in}{\huge
      \begin{displaymath}
        U\longrightarrow K
      \end{displaymath}
    }

  \item It transforms kinetic energy into potential energy when it
    does {\color{magenta}negative} work:

    \vspace{-.4in}{\huge
      \begin{displaymath}
        U\longleftarrow K
    \end{displaymath}
    }
  \end{itemize}  
\end{frame}

%%\begin{frame}{Conservation of Mechanical Energy}
%%  Mathematically, the work done by conservative forces can be expressed as:
%%  
%%  \eq{-.1in}{
%%    W_c=-\Delta U = \Delta K
%%  }
%%  
%%  which show that when only conservative forces act on an object, its total
%%  mechanical energy $E_T$ is conserved:
%%
%%  \eq{-.1in}{
%%    \boxed{
%%      \Delta K + \Delta U =0
%%    }\quad\rightarrow\quad
%%    \boxed{
%%      E_T = U_1 + K_1 = U_2 + K_2
%%    }
%%  }
%%\end{frame}
%
%
%
%\begin{frame}{Non-Conservative Force}
%  The majority of forces are \textbf{non-conservative}. Examples include:
%  \begin{itemize}
%  \item Normal force
%  \item Static and kinetic friction
%  \item Applied force
%  \item Tension force
%  \item Aerodynamic forces: Drag (fluid resistance) and lift
%  \end{itemize}
%  The work-energy theorem ($W_\text{net}=\Delta K$) still applies regardless of
%  whether the forces are conservative or non-conservative
%\end{frame}
%
%
%
%\begin{frame}{Work by Non-Conservative Forces}
%  The work done by non-conservative forces differs from conservative forces in
%  that:
%  \begin{itemize}
%  \item There is no related potential energies: the work done by a
%    non-conservative force transform energy from one form of kinetic energy to
%    another
%  \item The work is path dependent
%  \end{itemize}
%\end{frame}
%
%
%
%
%\section{Internal Energy}
%
%\begin{frame}{Internal Energy}
%  \begin{columns}
%    \column{.31\textwidth}
%    \begin{tikzpicture}
%      \draw[thick,fill=gray!5] (-.2,-.2) rectangle (2.2,2.2);
%      \draw[vectors] (2.3,1)--(3.5,1) node[right]{$v$};
%      \draw[very thick,|<-|] (-.5,1)--(-.5,-3) node[midway,fill=white]{$h$};
%      \foreach \i in {1,...,50} \fill (rand+1,rand+1) circle (.055);
%    \end{tikzpicture}
%
%    \column{.69\textwidth}
%    Consider a container of gas of mass $m$ moving at speed $v$ at a height $h$
%    above Earth.
%    \begin{itemize}
%    \item It has a \emph{bulk} kinetic energy of $K=\dfrac12 mv^2$
%    \item It has a gravitational potential energy of $U_g=mgh$
%    \item The random motion of the air molecules have an additional energy,
%      called the \textbf{internal energy} $E_\text{int}$% (or \textbf{thermal
%      %energy})
%    \item Internal energy is the sum of all the kinetic and potential energies
%      at the \emph{microscopic} level:
%
%      \eq{-.2in}{
%        E_\text{int}=K_\text{micro} + U_\text{micro}
%      }
%      
%      \vspace{-.15in}Internal energy depends linearly on temperature.
%    \end{itemize}
%  \end{columns}
%\end{frame}
%
%
%
%%\begin{frame}{Internal Energy}
%%  \textbf{Internal energy}, or \textbf{thermal energy} from the random motion
%%  depends linearly on temperature.
%%  \begin{itemize}
%%  \item For an ideal gas, or a monatomic gas:
%%    
%%    \eq{-.2in}{
%%      E_\text{int}=\dfrac32nRT
%%    }
%%  \item For a diatomic gas:
%%
%%    \eq{-.2in}{
%%      E_\text{int}\approx\dfrac52nRT
%%    }
%%  \item For a solid:
%%
%%    \eq{-.2in}{
%%      E_\text{int}\approx3nRT
%%    }    
%%  \end{itemize}
%%\end{frame}
%
%
%\section{Conservation of Energy}
%
%\begin{frame}{Law of Conservation of Energy}
%  The \textbf{law of conservation of energy} states that \emph{the change in
%  the total energy of a system is equal to the external work done to it}:
%
%  \eq{-.1in}{
%    \boxed{
%      \Delta E_\text{sys}=W_\text{ext}
%%      \Delta K + \Delta U + \Delta E_\text{int}= W_\text{ext}
%    }
%  }
%
%  which can be expanded into:
%
%  \eq{-.1in}{
%    \boxed{
%      \sum\Delta K_i + \sum\Delta U_i + \Delta E_\text{int}= W_\text{ext}
%    }
%  }
%
%  where $K_i$ and $U_i$ are the kinetic and potential energies of the objects
%  in the system, and $E_\text{int}$ is the internal energy of the sytem.
%\end{frame}
%
%
%\begin{frame}{Law of Conservation of Energy}
%  In problem encountered in Physics 12, changes in the internal energy can
%  usually be considered to be outside of the system, therefore the law of
%  conservation of energy can be written as:
%  
%  \eq{-.2in}{
%    \boxed{ \sum\Delta K + \sum\Delta U = W_\text{ext} }\;\;\text{or}\;\;
%    \boxed{ \sum U_1 + \sum K_1 + W_\text{ext} = \sum U_2 + \sum K_2 }
%  }
%
%  The external work $W_\text{ext}$ is
%  \begin{itemize}
%  \item\textbf{Positive} if work is done {\color{red}to} the system
%  \item\textbf{Negative} if work is done {\color{red}by} the system to the
%    surrounding
%  \end{itemize}
%\end{frame}
%
%
%
%\begin{frame}{Conservation of Energy in an Isolated System}
%  \begin{columns}
%    \column{.28\textwidth}
%    \centering
%    \begin{tikzpicture}[scale=.7]
%      \fill[pattern=north east lines] rectangle (5,4);
%      \draw[very thick] rectangle (5,4);
%      \draw[very thick,fill=white] (.2,.2) rectangle (4.8,3.8);
%      \draw[thick,
%        decoration={aspect=0.3,segment length=2mm, amplitude=2.5mm, coil},
%        decorate] (2.5,3.75)--+(0,-1.5) node[midway,right=4]{$k$};
%      \draw[mass] (2,2.25) rectangle +(1,-1) node[midway]{$m$};
%    \end{tikzpicture}
%
%    \column{.72\textwidth}
%    An \textbf{isolated system} is a system of objects that does not interact
%    with its surroundings. (Think of a bunch of objects inside an insulated
%    box.) In such a case, the conservation of energy reduces to:
%
%    \eq{-.1in}{
%      \boxed{
%        \sum\Delta K_i + \sum\Delta U_i + \Delta E_\text{int}= 0
%      }
%    }
%  \end{columns}
%\end{frame}
%
%
%
%
%%\begin{frame}{Isolated Systems and Conservation of Energy}
%%  The system is isolated from the surrounding environment, therefore
%%  \begin{itemize}
%%  \item The environment can't do any work on it
%%  \item Energy inside the system cannot escape either
%%  \end{itemize}
%%  Forces are now \emph{internal} to the system, and that work only converts
%%  kinetic energy into potential energies inside the system, and vice versa.
%%\end{frame}
%
%
%
%\begin{frame}{Example: Gravity}
%  Assuming no friction and drag, a free-falling object forms an isolated system
%  with Earth:
%  \begin{center}
%    \begin{tikzpicture}[scale=.7]
%      \draw[thick,fill=gray!30] (7.75,0) arc (75:105:30);
%      \draw[mass] (0,3) circle (.2);
%      \draw[vectors,red] (0,3)--(0,1.25) node[below]{$m\vec g$};
%    \end{tikzpicture}
%  \end{center}
%  \begin{itemize}
%  \item The only force doing work is gravity (conservative!) on the mass
%  \item The sum of the kinetic energy of the mass ($K$) and the gravitational
%    potential energy ($U_g$) is constant
%    
%    \eq{-.15in}{
%      \boxed{ K+U_g=\text{constant} }
%    }
%  \end{itemize}
%  %The system isolated until the two masses collide (This is a topic for next
%  %class.)
%\end{frame}
%
%
%\begin{frame}{Example: Gravity}
%  An object sliding down an arbitrarily-shaped ramp forms an isolated
%  system with Earth, assuming that there is no friction or drag:
%  \begin{center}
%    \begin{tikzpicture}[scale=.65]
%      \draw[thick](0,4) to[out=-30,in=180] (3,1) to[out=0,in=180] (5,3)
%      to[out=0,in=170] (8,0) to[out=-10,in=180] (10,0);
%      \draw[mass,rotate around={-60.5:(1,2.93)}] (1,2.93) rectangle +(.6,.6);
%      \fill[red] (1.4,2.8) circle (.08);
%      \draw[vectors,red] (1.4,2.8)--+(0,-1.5) node[left]{$m\vec g$};
%      \draw[vectors,red,rotate around={30:(1.4,2.8)}]
%      (1.4,2.8)--+(1.4,0) node[right]{$\vec F_N$};
%    \end{tikzpicture}
%  \end{center}
%  \begin{itemize}
%  \item Normal force and gravity act on the object, but only gravity does work
%    ($\vec F_N$ is perpendicular to motion)
%  \item The sum of the kinetic energy ($K$) and gravitational potential energy
%    ($U_g$) is constant
%    
%    \eq{-.15in}{
%      \boxed{
%        K+U_g=\text{constant}
%      }
%    }
%  %\item The shape of the ramp does not matter, only the initial and final
%  %  height relative to the referene level
%  \end{itemize}
%\end{frame}
%
%
%\begin{frame}{Example Problem}
%  \textbf{Example:} A skier glides with a speed of \SI{2.0}{\metre\per\second}
%  at the top of a ski hill, \SI{40}{\metre} high. She then begins to slide down
%  the icy (i.e.\ frictionless) hill.\footnote{In reality, there will always be
%  \emph{some} friction and drag as she slides down. In that case, we will also
%  need to know the non-conservative work done by friction.}
%  \begin{enumerate}[(a)]
%  \item What is the skier's speed at a height of \SI{25}\metre?
%  \item At what height does the skier have a speed of
%    \SI{10}{\metre\per\second}?
%  \end{enumerate}
%\end{frame}
%
%
%
%
%\begin{frame}{Example: Horizontal Spring-Mass System}
%  Assuming no friction, drag or other damping forces, a horizontal spring-mass
%  system is an isolated system:
%  \begin{center}
%    \vspace{-.18in}
%    \begin{tikzpicture}[scale=.85]
%      \draw[mass] (5,.5) rectangle (6,1.5);
%      \draw[thick,decorate,
%        decoration={aspect=.45,segment length=6,amplitude=7,coil}] (0,1)--(5,1);
%      \fill[pattern=north east lines] (6.5,.5)--(6.5,.3)--(-.2,.3)
%      --(-.2,2)--(0,2)--(0,.5)--cycle;
%      \draw[very thick] (0,2)--(0,.5)--(6.5,.5);
%      \fill[red] (5.5,1) circle (.08);
%      \draw[vectors,red] (5.5,1)--+(0,-1) node[below]{$\vec F_g$};
%      \draw[vectors,red] (5.5,1)--+(0,1) node[above]{$\vec F_N$};
%      \draw[vectors,red] (5.5,1)--+(-1,0) node[above]{$\vec F_e$};
%    \end{tikzpicture}
%  \end{center}
%  \begin{itemize}
%  \item\vspace{-.1in}The only force doing work is the spring force ($\vec F_g$
%    and $\vec F_N$ are perpendicular to motion)
%  \item The sum of the kinetic energy of the mass ($K$) and the elastic
%    potential energy stored in the spring ($U_e$) is constant
%
%    \eq{-.2in}{
%      \boxed{ K+U_e=\text{constant} }
%    }
%  \end{itemize}
%\end{frame}
%
%
%
%\begin{frame}{Example Problem}
%  \textbf{Example:} A toy cart with a mass of \SI{.25}{\kilo\gram} travels
%  along a frictionless horizontal track and collides head on with a spring that
%  has a spring constant of \SI{155}{\newton\per\metre}. If the spring is
%  compressed by \SI{6.0}{\centi\metre}, how fast is the cart initially
%  travelling?
%  \begin{center}
%    \begin{tikzpicture}
%      \fill[blue!30] rectangle (10,-.3);
%      \draw (0,0)--(10,0);
%      \draw[fill=brown] (.5,.2) rectangle (3,.5) node[midway,above=3]{$m$};
%      \draw[fill=gray] (1,.2) circle (.2);
%      \draw[fill=gray] (2.5,.2) circle (.2);
%      \fill (1,.2) circle (.06);
%      \fill (2.5,.2) circle (.06);
%      \draw[vectors] (3,.35)--(4.5,.35) node[above]{$v$};
%      \draw[pattern=north east lines] (9.3,0) rectangle (10,1.7);
%      \draw[ultra thick,decorate,
%        decoration={aspect=.4,segment length=1.5mm, amplitude=2mm, coil}]
%      (6.3,.4)--(9.3,.4) node[midway,above=3]{spring};
%    \end{tikzpicture}
%  \end{center}
%\end{frame}
%
%
%
%
%\begin{frame}{Example: Vertical Spring-Mass System}
%  \begin{columns}
%    \column{.25\textwidth}
%    \centering
%    \begin{tikzpicture}
%      \draw[mass] (.5,2) rectangle (1.5,1);
%      \draw[thick,decorate,
%        decoration={aspect=.4,segment length=5,amplitude=7,coil}] (1,5)--(1,2); 
%      \fill[pattern=north east lines] (0,5) rectangle (2,5.2);
%      \draw[very thick] (0,5)--(2,5);
%      \draw[vectors,red] (1,1.5)--(1,0) node[right]{$\vec F_g$};
%      \draw[vectors,red] (1,1.5)--(1,3) node[right]{$\vec F_e$};
%      \fill[red] (1,1.5) circle (.06);      
%    \end{tikzpicture}
%
%    \column{.75\textwidth}
%    Assuming no friction, drag or other damping forces, the vertical
%    spring-mass system (consists of the mass, the spring and Earth) is an
%    isolated system
%    \begin{itemize}
%    \item Both gravity and spring force are doing work
%    \item The sum of the kinetic energy of the mass ($K$), the gravitational
%      potential energy ($U_g$), and the elastic potential energy in the
%      spring ($U_e$) is constant.
%
%      \eq{-.1in}{
%        \boxed{ K + U_g + U_e=\text{constant} }
%      }
%    \end{itemize}
%  \end{columns}
%\end{frame}
%
%
%\begin{frame}{Example Problem}
%  \begin{columns}
%    \column{.6\textwidth}
%    \textbf{Example:} A freight elevator car with a total mass of
%    \SI{100}{\kilo\gram} is moving downward at \SI{3.00}{\metre\per\second},
%    when the cable snaps. The car falls \SI{4.00}{\metre} onto a huge spring
%    with a spring constant of \SI{8.00e3}{\newton\per\metre}. By how much will
%    the spring be compressed when the elevator car reaches zero velocity?
%
%    \column{.4\textwidth}
%    \pic1{graphics/freight-elevator}
%  \end{columns}
%\end{frame}
%
%
%
%
%\begin{frame}{Example: Simple Pendulum}
%  \begin{columns}
%    \column{.75\textwidth}
%    Assuming no friction, drag or other damping forces, the simple pendulum
%    system (consists of the mass and Earth) is isolated
%    \begin{itemize}
%    \item Gravity ($\vec F_g$) is the only force that does work
%    \item Tension ($\vec F_T$) idoes not do work on the pendulum because it is
%      perpendicular to its motion
%    \item The sum of the kinetic energy of the mass ($K$) and the gravitational
%      potential energy ($U_g$) is constant:
%
%      \eq{-.1in}{
%        \boxed{ K + U_g =\text{constant} }
%      }
%    \end{itemize}
%    
%    \column{.23\textwidth}
%    \centering
%    \begin{tikzpicture}
%      \fill[pattern=north east lines] (-1,0) rectangle (1,0.2);
%      \draw[thick] (-1,0)--(1,0);
%      \begin{scope}[rotate=15]
%        \draw[thick] (0,0)--(0,-5);
%        \shade[ball color=red] (0,-5) circle (.2) node[right=3]{$m$};
%        \draw[vectors,red] (0,-5)--(0,-3.5) node[left]{$\vec F_T$};
%        \draw[vectors,red,rotate around={-15:(0,-5)}] (0,-5)--(0,-6.3)
%        node[below]{$\vec F_g$};
%      \end{scope}
%      \draw[dashed] (0,0)--(0,-5);
%    \end{tikzpicture}
%  \end{columns}
%\end{frame}
%
%
%
%\begin{frame}{What if there is friction?}
%  Energy is always conserved as long as your system is defined properly. In
%  this case, the system consists of a mass, a spring, Earth and all the air
%  molecules inside the box:
%  \input{../common/closed-box}
%  The energies of this system include
%  \begin{itemize}
%  \item Kinetic energy of the mass ($K$)
%  \item Gravitational potential energy ($U_g$) between the mass and Earth
%  \item Elastic potential energy ($U_e$) stored in the spring
%  \item Internal energies ($E_\text{int}$) of the air molecules and the mass
%  \end{itemize}
%\end{frame}
%
%
%
%\begin{frame}{Isolated System with Changing Internal Energy}
%  \input{../common/closed-box}
%  
%  \eq{-.1in}{
%    \boxed{ K + U_g + U_e + E_\text{int}=\text{constant}}
%  }
%  
%  \textbf{Note:} Work done by kinetic friction and drag forces (both
%  non-conservative) always convert the kinetic energy of the mass into the
%  internal energy of the air molecules and the mass, heating them up. It does
%  not go the other way.
%\end{frame}
%
%
%
%\begin{frame}{Isolated vs.\ Open System}
%  Accounting for the change in $E_\text{int}$ is usually impractical,
%  especially when the air molecules are not confined to a box.
%  \begin{center}
%    \begin{tikzpicture}[scale=.5]
%      \fill[pattern=north east lines] (0,4) rectangle (5,4.5);
%      \fill[gray!10] rectangle (5,4);
%      \draw[very thick] (0,4)--(5,4);
%      \draw[thick,
%        decoration={aspect=.3,segment length=2mm, amplitude=2.5mm, coil},
%        decorate] (2.5,4)--(2.5,2.25) node[midway,right=5]{$k$};
%      \draw[mass] (2,2.25) rectangle (3,1.25) node[midway]{$m$};
%    \end{tikzpicture}
%  \end{center}
%  \vspace{-.05in}Solution:
%  \begin{itemize}
%  \item Take the air molecule out of the \emph{system}
%  \item No longer an isolated system
%  \item Account for the work by kinetic friction and drag as
%    \emph{external work} between initial and final states:
%
%    \eq{-.2in}{
%      K + U_g + U_e + W_\text{ext}= K' + U_g' + U_e'
%    }
%  \end{itemize}
%\end{frame}
%
%
%
%\section{Power \& Efficiency}
%
%\begin{frame}{Power}
%  Power is the rate at which work is done, i.e.\ the rate at which energy is
%  being transformed:
%
%  \eq{-.1in}{
%    \boxed{P = \frac W{\Delta t}}\quad\quad
%    \boxed{P = \frac{\Delta E}{\Delta t}}
%  }
%  \begin{center}
%    \begin{tabular}{l|c|c}
%      \rowcolor{pink}
%      \textbf{Quantity}  & \textbf{Symbol} & \textbf{SI Unit} \\ \hline
%      Power              & $P$        & \si\watt \\
%      Energy transformed & $\Delta E$ & \si\joule \\
%      Work done          & $W$        & \si\joule \\
%      Time interval      & $\Delta t$ & \si\second
%    \end{tabular}
%  \end{center}
%  In engineering, power is often more critical than the actual amount of work
%  done.
%\end{frame}
%
%
%
%\begin{frame}{Power}
%  If a constant force is used to push an object at a constant velocity, the
%  power produced by the force is:
%  
%  \eq{-.1in}{
%    P=\frac W{\Delta t}=\frac{F\Delta d}{\Delta t}
%    \quad\longrightarrow\quad \boxed{P=Fv}
%  }
%  
%  Application: aerodynamics
%  \begin{itemize}
%  \item When an object moves through air, the applied force must overcome air
%    resistance (drag), which is proportional with $v^2$
%    \item Therefore ``aerodynamic power'' must scale with $v^3$ (i.e.\ doubling
%      your speed requires $2^3=8$ times more power)
%    \item Important when aerodynamic forces dominate
%  \end{itemize}
%\end{frame}
%
%
%
%\begin{frame}{Efficiency}
%  The ratio of useful energy or work output to the total energy or work input
%
%  \eq{-.1in}{
%    \boxed{\eta = \frac{E_o}{E_i}\times\SI{100}{\percent}}\quad
%    \boxed{\eta = \frac{W_o}{W_i}\times\SI{100}{\percent}}
%  }
%  \begin{center}
%    \begin{tabular}{l|c|c}
%      \rowcolor{pink}
%      \textbf{Quantity} & \textbf{Symbol} & \textbf{SI Unit} \\ \hline
%      Useful output energy & $E_o$ & \si\joule \\
%      Input energy         & $E_i$ & \si\joule \\
%      Useful output work   & $W_o$ & \si\joule \\
%      Input work           & $W_i$ & \si\joule \\
%      Efficiency           & $\eta$ & no units
%    \end{tabular}
%  \end{center}
%  Efficiency is always $0 \leq\eta < 100\%$
%\end{frame}
%
\end{document}
