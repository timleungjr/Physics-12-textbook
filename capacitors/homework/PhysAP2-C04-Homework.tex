\documentclass{../../oss-apphys-exam}

\tikzstyle{charge}=[ball color=red]

\begin{document}
%\genheader

\gentitle{4}{ELECTROSTATICS, PART 2}



\begin{questions}
  \classkickMCinstructions
  
  \question Paper is considered an insulator. How does a positively charged
  piece of tape pick up a neutral paper bit?
  \begin{choices}
    \choice The tape makes the protons flow to the opposite end of the paper,
    causing an attraction between the electrons left behind and the tape.
    \choice The tape polarizes the paper atoms, attracting the electrons to the
    side of the atoms closest to the tape.
    \choice The tape forces electrons at the opposite end of the paper to flow
    through the paper toward the tape.
    \choice The tape polarizes the paper atoms, moving the protons within the
    atoms to the side of the atom farthest from the tape.
    \choice It is not possible for a charged object to attract a neutral object.
  \end{choices}


  
  \question A pith ball is a tiny piece of Styrofoam that is covered with a
  conductive paint. One pith ball initially has a charge of \SI{6.4e-8}\coulomb,
  and it touches an identical, neutral pith ball. After the pith balls are
  separated, what is the charge on the pith ball that had the initial charge?
  \begin{choices}
    \choice\SI{6.4e-8}\coulomb
    \choice\SI{3.2e-8}\coulomb
    \choice\SI{0}\coulomb
    \choice\SI{-3.2e-8}\coulomb
    \choice\SI{-6.4e-8}\coulomb
  \end{choices}

  \question Glass becomes positively charged when it is rubbed with silk. Which
  of the following is the best description of what's happening?
  \begin{choices}
    \choice Electrons are rubbed off the glass onto the silk.
    \choice Electrons are rubbed off the silk onto the glass.
    \choice Protons are rubbed off the glass onto the silk.
    \choice Protons are rubbed off the silk onto the glass.
    \choice Neutrons in the glass have an affinity for positive charge.
  \end{choices}

  \question Consider an isolated, neutral system consisting of wool fabric and a
  rubber rod. If the rubber rod is rubbed with wool to become negatively
  charged, what can be said about the wool fabric?
  \begin{choices}
    \choice It becomes equally negatively charged.
    \choice It becomes equally positively charged.
    \choice It becomes negatively charged but not equally.
    \choice It becomes positively charged but not equally.
    \choice In a neutral system, neither object can become charged.
  \end{choices}

  \question A negatively charged object is placed near, but not touching, a
  neutral conductor. As a result, the two objects are attracted to each other.
  Which of the following is true?
  \begin{choices}
    \choice The neutral object gains positive charges to become positively
    charged.
    \choice The neutral object loses negative charges to become positively
    charged.
    \choice The neutral object loses positive charges to become negatively
    charged.
    \choice The neutral object gains negative charges to become negatively
    charged.
    \choice Negative charges of the neutral object move to the side opposite
    the negatively charged object.
  \end{choices}
  \newpage
  
  \uplevel{
    \centering
    \begin{tikzpicture}[scale=2]
      \draw[dashed] (0,0)--(1,0)--(1,1) node[midway,right]{$a$}--(0,1)--cycle;
      \fill (.5,.5) circle (.03);
      \draw[fill=white] (0,0) circle (.05) node[below=2]{$-q$};
      \draw[fill=white] (1,0) circle (.05) node[below=2]{$+q$};
      \draw[fill=white] (0,1) circle (.05) node[above=2]{$+q$};
      \draw[fill=white] (1,1) circle (.05) node[above=2]{$-q$};
    \end{tikzpicture}
  }
  \question Four charges are arranged at the corners of a square of side $a$ as
  shown above. Which of the following is true of the electric field and the
  electric potential at the centre of the square?

  \begin{tabular}{rcc}
    & \underline{Electric Field} & \underline{Electric Potential}\\
    (A) & zero & zero \\
    (B) & $\dfrac{kQ}{a\sqrt 2}$ & zero \\
    (C) & $\dfrac{kQ^2}{2a^2}$ &  $\dfrac{kQ}{2a}$\\
    (D) & zero &  $\dfrac{kQ}{\sqrt{2a}}$\\
    (E) & $\dfrac{kQ^2}{2a}$ & $\dfrac{kQ}{a\sqrt 2}$
  \end{tabular}
  \newpage  


  
  \classkickFRQinstructions
  
  % THIS IS ALSO A FINAL-EXAM QUESTION FOR PHYSICS 12. THIS IS TAKEN FROM A
  % QUESTION FROM TIPLER
  \question An electron starts at the position shown in the figure below with
  an initial velocity of $V_0=\SI{5.0e6}{\metre\per\second}$ at an angle of
  \ang{30} to the $x$-axis. The electric field is in the positive $y$-direction
  and has a magnitude of \SI{3.5e3}{\newton\per\coulomb}. The electron carries
  the elementary charge of $e=\SI{-1.6e-19}\coulomb$ and has a mass of
  $m_e=\SI{9.1e-31}{\kilo\gram}$.
  \begin{center}
    \begin{tikzpicture}
      \draw[axes] (-3,0)--(-2,0) node[right]{$x$};
      \draw[axes] (-3,0)--(-3,1) node[above]{$y$};
      \draw[line width=1mm] (0,0)--(10,0);
      \draw[line width=1mm] (0,3)--(10,3);
      \begin{scope}[|<->|,thick]
        \draw (0,3.35)--(10,3.35) node[midway,fill=white]{10 cm};
        \draw (10.25,0)--(10.25,3) node[midway,fill=white]{3.0 cm};
      \end{scope}
      \foreach \x in {2,3.3,...,9.5} \draw[vectors,lightgray] (\x,0)--(\x,3);
      \node at (5.2,1.5) {$\vec E$};
      \draw[vectors,rotate=30] (0,0)--(2,0)
      node[midway,above left,black]{$\vec V_0$};
      \draw[axes] (1.25,0) arc (0:28:1.25) node[midway,right]{\ang{30}};
      \shade[charge] circle (.2) node[left=3]{$e^-$};
    \end{tikzpicture}
  \end{center}
  \begin{parts}
    \part Which plate has a positive charge?
    \part What is the electric potential difference across the plates?
    \part What is the acceleration (both magnitude and direction) of the
    electron? Assume that gravitational effects can be neglected.
    \part On which plate and at what location will the electron strike?
    (Hint: think projectile motion, but acceleration is not due to gravity.)
  \end{parts}
  \newpage

  \uplevel{
    \centering
    \pic{.55}{contours1}
  }
  \question The figure above shows the equipotential lines around spherical
  charges 1 and 2. Points A, B, C, and D represent locations on the
  equipotential lines.
  \begin{parts}
    \part Sketch electric field vectors at points A and C. The vectors should be
    drawn so their relative strengths are reflected in the drawing.
    \vspace{.2in}

    \part What are the signs of the two charges, and how do their relative
    magnitudes compare? Explain how the equipotential lines help you determine
    this.
    \vspace{\stretch1}
    
    \part The spheres have masses in the same ratio as the magnitude of their
    charges. Will the equipotential lines for gravity have a similar shape as
    the lines shown? Explain.
    \vspace{\stretch1}
    
    \part A proton is released from point C and moves through an electric
    potential difference of magnitude \SI{40}\volt.
    \begin{subparts}
      \subpart On which equipotential line will the proton end up?
      \vspace{\stretch1}
      
      \subpart The proton will have kinetic energy when it arrives at this new
      equipotential line. Where does this kinetic energy come from? Explain
      your answer in terms of the system that includes the two charges and the
      proton.
      \vspace{\stretch1}
    \end{subparts}
    \newpage
    
    \part An electron at point A is moved to point B. Has the electric potential
    energy of the electron--charges system increased or decreased? Justify
    your answer with an equation.
    \vspace{\stretch1}
    
    \part The distance between points C and D is $d$. Derive a symbolic
    expression for the magnitude of the average electric field between the
    two points. Also, indicate the direction of the electric field.
    \vspace{\stretch1}
    
    \part A particle with positive charge of $Q$ is released from point C and
    gains kinetic energy on its path to point D. Derive a symbolic equation for
    the amount of work done by the electric field and the final kinetic
    energy of the proton.
    \vspace{\stretch5}
    
  \end{parts}
  \newpage

  %\question Electric field vectors around three charges 1, 2, and 3, are shown
  %in the figure below.
  %\uplevel{
  %  \centering
  %  \pic{.4}{efield}
  %}
  %\begin{parts}
  %  \part On the diagram, draw the direction of the force on an electron placed
  %  at point C.
  %  \vspace{.3in}
  %  
  %  \part What are the signs of the three charges? Explain what aspects of
  %  the electric field indicate the sign of the charges.
  %  \vspace{\stretch1}
  %
  %  \part Sketch two equipotential lines---one that passes through point A and
  %  another that passes through point B.
  %  \vspace{.3in}
  %  
  %  \part Which equipotential line has a higher electric potential, the line
  %  that passes through point A or the one that passes through point B?
  %  Justify your answer.
  %  \vspace{\stretch1}
  %\end{parts}
  %\newpage
  
  \uplevel{
    \cpic{.45}{contours}
  }
  \question The dots in the figure above represent two identical spheres,
  $X$ and $Y$, that are fixed in place with their centers in the plane of the
  page. Both spheres are charged, and the charge on sphere $Y$ is positive. The
  lines are equipotential lines, also in the plane of the page, with
  a potential difference of \SI{10}{\volt} between each set of adjacent lines.
  The absolute value of the electric potential of the outermost line is
  \SI{50}\volt.
  \begin{parts}
    \part Indicate the values of the potentials, including the signs, at the
    labeled points A and B.

    \vspace{.1in}
    Potential at point A \underline{\hspace{.5in}}\hspace{.5in}
    Potential at point B \underline{\hspace{.5in}}

    \part How do the magnitudes and the signs of the charges of the spheres
    compare? Explain your answer in terms of the equipotential lines shown.
    \label{partb}
    \vspace{\stretch1}
      
    \part The spheres at points $X$ and $Y$ have masses in the same ratio as
    the magnitudes of their charges. The equipotential lines for gravity for
    the spheres have shapes similar to those of the equipotential lines shown.
    Explain why the two sets of lines have similar shapes.
    \vspace{\stretch1}
    
    \uplevel{
      Let the potentials at the three labeled points be $V_A$, $V_B$, and $V_C$.
      A proton with charge $+q$ and mass $m$ is released from rest at point $B$.
    }
    
    \part Based on your answer to part (\ref{partb}), briefly describe one
    similarity and one difference between the electric and gravitational forces
    exerted on the proton by the system of the two spheres. The similarity and
    difference you describe must not be ones that generally apply to all forces.
    \vspace{\stretch1}
    \newpage
    
    \uplevel{
      At some time after being released from rest at point $B$, the proton
      has moved through a potential difference of magnitude \SI{20}\volt.
    }
    
    \part Determine the change in electric potential energy of the
    proton-spheres system when the proton has moved through the \SI{20}{\volt}
    potential difference. Express your answer symbolically in terms of $q$,
    $V_A$, $V_B$, $V_C$, and physical constants, as appropriate.
    \vspace{\stretch1}
      
    \part As it moved through the \SI{20}{\volt} potential difference, the
    proton was displaced a distance $d$ by the electric force. Determine a
    symbolic expression for the total work done on the proton by the electric
    field in terms of the average magnitude $E_\text{avg}$ of the electric
    field over that distance.
    \vspace{\stretch1}
      
    \part Two students are discussing how and why the kinetic energy of
    the proton would change after it is released.
    \begin{itemize}
    \item Student 1 says that if the system is defined as the proton and the
      spheres, the increase in the proton’s kinetic energy is due to a change
      in the system's potential energy as the proton moves through the
      \SI{20}{\volt} potential difference.
    \item Student 2 says that if the system is defined as only the proton,
      the kinetic energy of the proton increases because positive work is
      done on the proton by the electric field as the proton moves through
      the \SI{20}{\volt} potential difference.
    \end{itemize}
    Discuss each student's claims, explaining why each is correct or incorrect.
    \vspace{\stretch2}
  \end{parts}
  \newpage
  
  \question A parallel plate capacitor with a capacitance of $C$ is shown in the
  figure. The area of one plate is $A$, and the distance between the plates is
  $d$.
  \uplevel{
    \cpic{.3}{parallel-plate-capacitor}
  }
  \begin{parts}
    \part If the area of both capacitor plates as well as the distance between
    them were doubled, what would be the effect on the capacitance of the
    capacitor? Explain.
    \vspace{\stretch1}
    
    \part The capacitor is connected to a battery of potential difference
    $\Delta V$. If the potential difference of the battery is doubled, what
    happens to the charge stored on the plates and the capacitance of the
    capacitor? Justify your answer.
    \vspace{\stretch1}
    
    \part In an experiment, the area $A$ of the capacitor plates is changed to
    investigate the effect on the capacitance $C$ of the capacitor. Sketch
    the graph of the lab data you expect to see from this experiment.
    \uplevel{
      \centering
      \begin{tikzpicture}[axes]
        \draw (0,0)--(5,0) node[right]{$A$};
        \draw (0,0)--(0,4.5) node[above]{$C$};
      \end{tikzpicture}
    }
    \newpage
    
    \part In another experiment, the distance between the plates ($d$) is
    changed to investigate the effect on the capacitance ($C$) of the capacitor.
    Sketch the graph of the lab data you expect to see from this experiment.
    \uplevel{
      \centering
      \begin{tikzpicture}[axes]
        \draw (0,0)--(5,0) node[right]{$d$};
        \draw (0,0)--(0,4.5) node[above]{$C$};
      \end{tikzpicture}
    }
    
    \part You are going to use a capacitor to power a light bulb. You need the
    bulb to shine for a long time. Describe the geometry of the capacitor
    you would choose to power the bulb. Explain your answer.
  \end{parts}
  \newpage
  
  %\question A charged capacitor generates the electric field shown in the
  %figure below.
  %\uplevel{
  %  \cpic{.2}{charged}
  %}
  %\begin{parts}
  %  \part An electric dipole shown below is placed between the plates of the
  %  capacitor. Draw any electric forces experienced by the dipole, and
  %  describe its motion.
  %  \uplevel{
  %    \centering
  %    \begin{tikzpicture}[scale=1.2,thick]
  %      \draw circle (.2) node{$+$};
  %      \draw (0,1) circle (.2) node{$-$};
  %      \draw (0,.2)--(0,.8);
  %    \end{tikzpicture}
  %  }
  %
  %  \part An uncharged metal box is placed between the plates of the capacitor.
  %  \begin{subparts}
  %    \subpart Draw any net charge on the box that results from it being placed
  %    inside the capacitor on the figure on the left.
  %    
  %    \uplevel{
  %      \centering
  %      \tikz{\draw[fill=lightgray] rectangle (1,2);}
  %    }
  %    
  %    \subpart Draw any electric field generated by the charges of the box on
  %    the figure in the middle.
  %
  %    \uplevel{
  %      \centering
  %      \tikz{\draw[fill=lightgray] rectangle(1,2);}
  %    }
  %    
  %    \subpart Draw the net electric field inside the box due to its own charge
  %    distribution and due to the external electric field of the capacitor
  %    on the figure on the right.
  %
  %    \uplevel{
  %      \centering
  %      \tikz{\draw[fill=lightgray] rectangle (1,2);}
  %    } 
  %  \end{subparts}
  %\end{parts}
  
\end{questions}
\end{document}
