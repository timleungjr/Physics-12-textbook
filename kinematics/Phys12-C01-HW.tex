%\documentclass{../../ossphysics}
%
%\begin{document}
%
%\setheader{Physics 12 Class 1 Homework}
%
%\hwtitle{12}{1}{Kinematics}

\newpage
\section*{Problems}

\begin{multicols}{2}
  \begin{enumerate}[leftmargin=12pt]
  \item A ball is thrown towards the north. What are the directions of the
    acceleration and instantaneous velocity, respectively, of the ball at
    maximum height (e.g.\ the peak of its trajectory)?
    \begin{enumerate}[noitemsep]
    \item north, north
    \item up, north
    \item down, north
    \item north, down
    \item down, down
    \end{enumerate}

  \item A cyclist cycles 50 km [N] and then 30 km [E]. The total time taken
    for the trip is 3.0 h. What is its average velocity?
    \begin{enumerate}[noitemsep]
    \item\SI{80}{\kilo\metre\per\hour} [\ang{31} E of N]
    \item\SI{19}{\kilo\metre\per\hour} [\ang{31} E of N]
    \item\SI{27}{\kilo\metre\per\hour} [\ang{31} E of N]
    \item\SI{19}{\kilo\metre\per\hour} [\ang{59} E of N]
    \item\SI{19}{\kilo\metre\per\hour} [NE]
    \end{enumerate}

  \item A baseball player is trying to maximize her throwing distance. She
    must release the ball \underline{\hspace{.7in}}
    \begin{enumerate}[noitemsep]
    \item at an angle that lets the ball reach the highest possible height
    \item horizontally
    \item at an angle of \ang{45}
    \item with the maximum possible speed, regardless of angle
    \item at an angle between \ang{45} and \ang{90}
    \end{enumerate}

  \item A boy throws a ball off of a second floor balcony by throwing it up
    into the air at some angle. It comes back down, landing on the ground.
    Neglecting air resistance, the magnitude of velocity is greatest
    \begin{enumerate}[noitemsep]
    \item just after it leaves the boy's hand
    \item at the peak of the ball's trajectory
    \item just before it hits the ground
    \item It remains the same throughout the motion
    \item Impossible to tell without knowing the angle of projection
    \end{enumerate}
  
  \item At time $t=0$, a red car and a blue car are both located at $x=0$,
    with the red car travelling at a constant speed $v$ along the positive
    $x$-axis and the blue car just beginning to accelerate along a path parallel
    to the red car. The velocity of both cars from $0$ to $2t$ is graphed below.
    At time $t$:
    \begin{center}
      \begin{tikzpicture}[scale=.7]
        \draw[axes] (0,0)--(6.5,0);
        \draw[axes] (0,0)--(0,4.5);
        \draw[very thick,red] (0,2)--(6,2) node[pos=0,left,black]{$v$}
        node[right]{Red car};
        \draw[very thick,dash dot,blue] (0,0)--(3,4)
        node[pos=.8,above,sloped,dash dot,blue]{Blue car} --(6,0);
        \draw (0,4)--(6,4) node[pos=0,left]{$2v$};
        \draw (3,0)--(3,4) node[pos=0,below]{$t$};
        \draw (6,0)--(6,4) node[pos=0,below]{$2t$};
      \end{tikzpicture}
    \end{center}
    \begin{enumerate}[noitemsep]
    \item The blue car has travelled further, and both cars have the
      same instantaneous velocity
    \item Both cars have travelled the same distance, and the blue
      car has a greater instantaneous velocity
    \item The red car has travelled further, and both cars have the same
      instantaneous velocity
    \item Both cars have travelled the same distance, and both cars have the
      same instantaneous velocity
    \item The blue car has travelled further, and the blue car has a greater
      instantaneous velocity
    \end{enumerate}
    
  \item A car is travelling west and approaching a stop sign. As it is
    slowing to a stop, the directions associated with the object's velocity and
    acceleration, respectively, are
    \begin{enumerate}[noitemsep]
    \item West, East
    \item West, West
    \item East, East
    \item East, West
    \item There is not enough information to tell
    \end{enumerate}
    
  %\item The direction equivalent to [\ang{40} W of S] is
  %  \begin{enumerate}[noitemsep]
  %  \item [\ang{40} E of S]
  %  \item [\ang{40} W of N]
  %  \item [\ang{40} E of N]
  %  \item [\ang{50} S of W]
  %  \item [\ang{50} E of N]
  %  \end{enumerate}

  \item If a car travelling at \SI{60}{\kilo\metre\per\hour} [S] stops
    in a time of \SI{3.5}\second, its acceleration is:
    \begin{enumerate}[noitemsep]
    \item \SI{4.77}{\metre\per\second\squared} [S]
    \item \SI{4.77}{\metre\per\second\squared} [N]
    \item \SI{16.7}{\metre\per\second\squared} [S]
    \item \SI{16.7}{\metre\per\second\squared} [N]
    \item \SI{17.1}{\metre\per\second\squared} [S]
    \end{enumerate}

  \item Which of the following objects are in ``free fall''?
    \begin{enumerate}[noitemsep]
    \item A ball that was thrown horizontally
    \item A ball that was thrown at an angle above horizontal
    \item A ball that was thrown at an angle below horizontal
    \item A ball that was dropped
    \item All of the above
  \end{enumerate}
    
  \item An airplane is flying to a city due south from its current location. If
    there is a slight wind blowing to the south-east, the plane must head
    (point) to the:
    \begin{enumerate}[noitemsep]
    \item South
    \item West
    \item South-west
    \item South-east
    \item East
    \end{enumerate}

  \item A ball is thrown up in the air and then caught at the same height.
    The acceleration is \SI{9.81}{\metre\per\second\squared} [down]
    \begin{enumerate}[noitemsep]
    \item on the way up
    \item on the way down
    \item at the peak of its trajectory
    \item two of A, B, and C are correct
    \item all of A, B, and C are correct
  \end{enumerate}

  \item Two velocity vectors $v_1$ and $v_2$ each have the same magnitude.
    Graph 1 shows the velocity $v_1$ at $t=\SI0\second$, and then the same
    object has a velocity $v_2$ at $t=\SI2\second$, shown in Graph 2. Which of
    the following vectors best represents the average acceleration vector that
    causes the object's velocity to change from $v_1$ to $v_2$ ?
    \begin{center}
      \begin{tikzpicture}[scale=.6]
        \draw (-2,0)--(2,0);
        \draw (0,-2)--(0,2);
        \draw[vectors] (0,0)--(1.8,0) node[above]{$v_1$};
      \end{tikzpicture}
      \hspace{.2in}
      \begin{tikzpicture}[scale=.6]
        \draw (-2,0)--(2,0);
        \draw (0,-2)--(0,2);
        \draw[vectors] (0,0)--(0,1.8) node[right]{$v_2$};
      \end{tikzpicture}
    \end{center}
    A. \begin{tikzpicture}[scale=.6]
      \draw (-2,0)--(2,0);
      \draw (0,-2)--(0,2);
      \draw[vectors] (0,0)--(0,1.8);
    \end{tikzpicture}
    \hspace{.2in}
    B. \begin{tikzpicture}[scale=.6]
      \draw (-2,0)--(2,0);
      \draw (0,-2)--(0,2);
      \draw[vectors] (0,0)--(1.8,0);
    \end{tikzpicture}
    \hspace{.2in}
    C. \begin{tikzpicture}[scale=.6]
      \draw (-2,0)--(2,0);
      \draw (0,-2)--(0,2);
      \draw[vectors] (0,0)--(-1.8,0);
    \end{tikzpicture}
    \hspace{.2in}
    D. \begin{tikzpicture}[scale=.6]
      \draw (-2,0)--(2,0);
      \draw (0,-2)--(0,2);
      \draw[vectors,rotate=-45] (0,0)--(0,1.8);
    \end{tikzpicture}
    \hspace{.2in}
    E. \begin{tikzpicture}[scale=.6]
      \draw (-2,0)--(2,0);
      \draw (0,-2)--(0,2);
      \draw[vectors,rotate=45] (0,0)--(0,1.8);
    \end{tikzpicture}
  
  \item A passenger on a train moving horizontally at a constant speed
    relative to the ground drops a ball from his window. A stationary observer
    on the ground sees the ball falling with a speed $v_1$ at an angle to
    the vertical at an instant after it is dropped from the train window, but
    the ball appears to be falling vertically with a speed $v_2$ at the same
    instant as viewed by the train passenger. What is the speed (magnitude of
    velocity) of the train relative to the ground after the ball is dropped?
    Neglect air resistance.
    \begin{enumerate}[noitemsep]
    \item $v_1 + v_2$
    \item $v_1-v_2$
    \item $v_1^2 + v_2^2$
    \item $v_1^2-v_2^2$
    \item $\sqrt{v_1^2-v_2^2}$
    \end{enumerate}
    
  \item A ball is dropped from rest from the top of a cliff \SI{80}{\metre}
    high. At the same time, a rock is thrown horizontally from the top of the
    same cliff. The rock and ball hit the level ground below a distance of
    \SI{40}{\metre} apart. The horizontal velocity of the rock that was thrown
    was most nearly
    \begin{center}
      \begin{tikzpicture}[scale=1.3]
        \draw[thick] (-.8,0)--(0,0)--(0,-2)--(3,-2);
        \draw[thick,<->|](-.5,0)--(-.5,-2) node[midway,fill=white]{\SI{80}\metre};
        \draw[mass] (.2,0) circle (.1);
        \draw[mass] (.4,0) circle (.1);
        \draw[vectors] (.5,0)--(1.2,0) node[right]{$v$};
        \draw[axes,dashed] (.2,0)--(.2,-1.9);
        \draw[axes,dashed] plot[smooth,domain=.4:1.75] (\x, {-(\x-.4)^2});
        \draw[thick,|<->|] (.2,-2.3)--(1.8,-2.3)
        node[midway,fill=white]{\SI{40}\metre};
      \end{tikzpicture}
    \end{center}
    \begin{enumerate}[noitemsep]
    \item\SI5{\metre\per\second}
    \item\SI{10}{\metre\per\second}
    \item\SI{20}{\metre\per\second}
    \item\SI{40}{\metre\per\second}
    \item\SI{80}{\metre\per\second}
    \end{enumerate}
  
  \item A golf ball is hit from level ground and has a horizontal range of
    100 m. The ball leaves the golf club at an angle of \ang{60} to the level
    ground. At what other angle(s) can the ball be struck at the same initial
    velocity and still have a range of 100 m?
    \begin{center}
      \begin{tikzpicture}
        \draw[axes] (0,0)--(4.5,0) node[right]{$x$};
        \draw[axes] (0,0)--(0,2.3) node[right]{$y$};
        \fill circle (.07);
        \draw[vectors,rotate=57] (0,0)--(1.5,0) node[above]{$v$};
        \draw[thick,dashed,->,domain=0:4] plot(\x,{-.4*((\x-2)*(\x-2))+1.6});
        \draw[axes] (.7,0) arc (0:57:.7) node[midway,right]{\ang{60}};
      \end{tikzpicture}
    \end{center}
    \begin{enumerate}[noitemsep]
    \item\ang{30}
    \item\ang{20} and \ang{80}
    \item\ang{10} and \ang{120}
    \item\ang{45} and \ang{135}
    \item There is no other angle other than \ang{60} in which the ball will
      have a range of 100 m.
  \end{enumerate}
  
  \item The motion of an object is represented by the acceleration vs.\ time
    graph below. If the object is initially at rest, which of the
    following statements is true about its motion?
    \begin{center}
      \begin{tikzpicture}[yscale=.45,xscale=.6]
        \draw[axes] (0,0)--(6.3,0) node[right]{$t$};
        \draw[axes] (0,-2.5)--(0,2.8) node[above]{$a$};
        \draw[ultra thick] (0,2)--(2,2);
        \draw[ultra thick] (2,-2)--(4,-2);
        \draw[gray] (2,2)--(2,-2);
        \draw[thick] (2,.2)--(2,-.2) node[below]{2};
        \draw[thick] (4,.2)--(4,-.2) node[below]{4};
        \draw[thick] (.2,2)--(-.2,2) node[left]{$+2$};
        \draw[thick] (.2,-2)--(-.2,-2) node[left]{$-2$};
      \end{tikzpicture}
    \end{center}
    \begin{enumerate}[noitemsep]
    \item The object returns to its original position.
    \item The velocity of the object is zero at a time of \SI2\second.
    \item The velocity of the object is zero at a time of \SI4\second.
    \item The displacement of the object is zero at a time of \SI4\second.
    \item The acceleration of the object is zero at a time of \SI2\second.
    \end{enumerate}
    
  \item Can an object ever be accelerating and experiencing an
    instantaneous velocity of \SI0{\metre\per\second}? Explain. 
    
  \item Large insects such as locusts can jump as far as \SI{75}{cm}
    horizontally on a level surface. An entomologist analyzed a photograph and
    found that the insect's launch angle was \ang{55}. What was the insect's
    initial velocity?
  
  \item Because of an oncoming storm, a boat must cross a river in the
    shortest amount of time possible regardless of where it lands on the
    opposite shore. Given that the river has a current, in what direction
    should the boat point? Explain.
    \vspace{\stretch1}
  
  \item While hiking in the wilderness, you come to the top of a cliff that
    is \SI{60}{\metre} high. You throw a stone from the cliff, giving it an
    initial velocity of \SI{21}{\metre\per\second} at \ang{35} above the
    horizontal. How far from the base of the cliff does the stone land?
    \vspace{\stretch2}
  
  \item You want to shoot a stone with a sling shot and hit a target on the
    ground \SI{14.6}{\metre} away. If you give the stone an initial speed of
    \SI{12.5}{\metre\per\second}, neglecting friction and air resistance, what
    is/are the launch angle(s) in order for the stone to hit the target? What
    would be the maximum height(s) by the stone? What would be its time of
    flight? Assume motion is symmetric.
  
  \item A sharpshooter shoots a bullet horizontally over level ground with
    a velocity of \SI{1.22e3}{\metre\per\second}. At the instant that the bullet
    leaves the barrel, its empty shell casing falls vertically and strikes the
    ground with a vertical velocity of \SI{5.50}{\metre\per\second}.
    \begin{enumerate}
      \item Neglecting air friction, how far does the bullet travel?
      \item What is the vertical component of the bullet's velocity at the
      instant before it hits the ground?
    \end{enumerate}

  \item A projectile is launched from point $O$ at an angle of \ang{22} with
    an initial velocity of $v_0=\SI{15}{\metre\per\second}$ up an incline plane
    that makes an angle of \ang{10} with the horizontal. The projectile lands on
    the incline plane at point $M$.
    \begin{center}
      \begin{tikzpicture}[xscale=1.3,yscale=1.5]
        \draw (0,0)--(4.5,0);
        \fill circle (.05) node[left]{$O$};
        \draw[vectors,rotate=57] (0,0)--(1.5,0) node[left]{$v_0$};
        \draw[thick,domain=0:3.55] plot(\x,{-.4*((\x-2)*(\x-2))+1.6});
        \draw[axes,rotate=10] (.7,0) arc(0:47:.7) node[midway,right]{\ang{22}};
        \draw[very thick,rotate=10] (0,0)--(4.5,0);
        \draw[axes] (2,0) arc (0:10:2) node[midway,right]{\ang{10}};
        \fill (3.55,.63) circle (.05) node[below]{$M$};
      \end{tikzpicture}
    \end{center}
    \begin{enumerate}
      \item Find the time it takes for the projectile to hit the incline plane.
      \item Find the distance $OM$.
    \end{enumerate}
    
  \item A baseball is thrown by an outfielder ($O$) towards the catcher
    ($C$) with an initial speed of \SI{20}{\metre\per\second} at an angle of
    \ang{45} with the horizontal. At the moment that the ball is thrown, $C$ is
    \SI{50}{\metre} from $O$. At what speed and in what direction must $C$ run
    to catch the ball at the same height at which it was released? Assume that
    $C$ catches the ball at the same moment that it arrives. Please answer in
    two significant figures.

  \item A car is travelling north on a city street at
    \SI{12.5}{\metre\per\second}. Just as the car crosses a perpendicularly
    intersecting crossroad, the passenger throws out a can horizontally, towards
    the east. The initial speed of the can relative to the car is
    \SI{10.0}{\metre\per\second}. It is released at a height of 1.75 m above the
    road.
    \begin{enumerate}
      \item What is the initial velocity of the can relative to the road?
      \item Where does the can land relative to the centre of the intersection?
    \end{enumerate}
  
%  %\item An object is thrown upward on a slope and reached a height of
%  %$h=\SI{15}{\metre}$ as shown, the object lands near the base of slope at a
%  %distance of $l=\SI{75}{\metre}$ away. The slope angle $\alpha$ is \ang{35}.
%  %\begin{center}
%  %  \pic{.35}{../graphics/FF55a}
%  %\end{center}
%  %Determine
%  %\begin{parts}
%  %\item the magnitude $v$, and
%  %\item the direction of the initial velocity $\theta$
%  %\end{parts}
  \end{enumerate}
\end{multicols}
