\chapter*{Preface}

This is not a textbook. At least, it wasn't meant to be when I first started.
The material presented in this book is a culmination of many years of
experience teaching Physics courses at Meritus Academy in Toronto, Canada.
When I started, I had a few simple sets of in-class (handwritten) notes, they
became a set of presentation slides, and grew further with homework questions
and additional handouts for students. As my course material became more refined
and comprehensive, it also became more obvious that further refinements would
only be possible by organizing my work into a textbook.

The topics presented in the book not only contain my thoughts on preparing my
students for the Advanced Placement exams, but also my own general insight in
what the topics may mean to students pursuing higher education in physics and
engineering.

Everyone begins learning physics from a different scientific and mathematical
background. Here, I make no assumptions about the readers' skills in
mathematics,  other than basic algebra and trigonometry. Calculus, while very
much necessary for a deeper understanding, will be avoided as much as possible.

Most of the students who come to my introductory physics courses are learning
linear algebra at the same time. So, for anyone with little to no background in
vector arithmetics and linear algebra, it will be helpful to have a quick read
of Appendix C. This can be done before starting your journey with kinematics,
but it is best if you can read them concurrently.

\begin{common-question}
  These are some of the smartest, most interesting, and most profound questions
  that have been asked by my students over the years. These are the questions
  that make me say, ``That's a \emph{really} good question!'' They are included
  here because I think that the answers to these questions are thought
  provoking; they can further our understanding, not only in physics, but also
  in mathematics as well.
\end{common-question}

\begin{remark}
  Remarks are the extra bits of information that you may not have thought
  about, or information that is not immediately relevant to our learning, but
  are nevertheless interesting. If you have a bit more background in
  mathematics, this may be
\end{remark}

I would like to thank everyone who have given me constructive feedback on the
book, and those who had to put up with me in this long process.

\vspace{.5in}
Tim
Toronto, Canada
2025
