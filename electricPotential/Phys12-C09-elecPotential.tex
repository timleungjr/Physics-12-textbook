%\section{Potential Energy \& Electric Potential}

\section{Electrical Potential Energy}
If you know calculus, you can easily find that the work done \emph{by} the
electrostatic force $W_q$ when two charges move from a distance of $r_1$ to
$r_2$:
\begin{equation}
  \underbracket[1pt]{{\color{magenta}W_q}=
  \int_{\mathbf r_1}^{\mathbf r_2}\mathbf F_q\cdot d\mathbf r}_\text{definition using
      calculus}
    =\cdots = kq_1q_2\int_{r_1}^{r_2}\frac{dr}{r^2}=\cdots
    ={\color{magenta}-\Delta U_q}
\end{equation}
If you don't know calculus, all you need to know is the result: $U_q$ is
defined as the \textbf{electric potential energy}:
\begin{equation}
  \boxed{
    U_q=\frac{kq_1q_2}r
  }
\end{equation}
Crucially, the electrostatic force is \emph{conservative} because the work done
by the electrostatic force is equal to the negative change in electric
potential energy:
\begin{equation}
  \boxed{
    W_q=-\Delta U_q
  }
\end{equation}



Like the gravitational potential energy (Eq.~\ref{eq:GPE-general}), the
electric potential energy uses  $r=\infty$ as the reference. However, unlike
gravitational potential energy which is \emph{always} negative, $U_q$ can be
either positive or negative, because charges can be either positive or
negative. When $U_g>0$ (energy stored between two positive charges or two
negative charges), external work must be done to bring the charges from
$r=\infty$ to $r$. Conversely, if $U_q<0$ (energy stored between one positive
and one negative charge), external work is done to pull two charges apart from
$r$ to $r=\infty$.


\begin{itemize}
\item\emph{Positive} work by the electric force \emph{decreases} the
  electric potential energy by the same amount
\item \emph{Negative} work by the electric force \emph{increases} the
  electric potential energy by the same amount
\item Work done is path independent: $W_q$ depends on the end points $r_1$ and
  $r_2$, but not \emph{how} the charges move from $r_1$ to $r_2$
\item Only work done by $\mathbf F_q$ can change $U_q$
\end{itemize}




%%\section{Electric Potential Energy}
%%  To bring two charges together, I have to do work against electrostatic force,
%%  therefore, there is a gain/lost in \textbf{electric potential energy},
%%  defined as:
%%  
%%  \eq{-.1in}{
%%    \boxed{U_q=\frac{kq_1q_2}r}
%%  }
%%  \begin{center}
%%    \begin{tabular}{l|c|c}
%%      \rowcolor{pink}
%%      \textbf{Quantity} & \textbf{Symbol} & \textbf{SI Unit} \\ \hline
%%      Electric potential energy & $U_q$ & \si\joule \\
%%      Coulomb's constant & $k$ &\si{\newton\metre\squared\per\coulomb\squared}\\
%%      Electric charges          & $q_1$, $q_2$ & \si\coulomb \\
%%      Distance                  & $r$   & \si\metre
%%    \end{tabular}
%%  \end{center}


When there are more than two charges, the total potential energy is the total
from all the \emph{systems of two charges}. For example, in the
\emph{three}-charge system in Fig.~\ref{fig:3charge}, the total electric
potential energy stored is given by:
\begin{equation*}
  U_q=\frac{kq_1q_2}{r_1} +\frac{kq_2q_3}{r_2} +\frac{kq_1q_3}{r_3}
\end{equation*}

\begin{figure}[ht]
  \centering
  \begin{tikzpicture}
    \draw (0,0)--(1,-2) node[midway,fill=white]{$r_3$}
    --(-3,-2) node[midway,fill=white]{$r_2$}
    --(0,0) node[midway,fill=white]{$r_1$};
    \shade[poscharge] circle (.2) node[white]{$q_1$};
    \shade[poscharge] (1,-2) circle (.2) node[white]{$q_3$};
    \shade[poscharge] (-3,-2) circle (.2) node[white]{$q_2$};
  \end{tikzpicture}
  \caption{A three-charge system}
  \label{fig:3charge}
\end{figure}
$U_q$ may be positive or negative depending on the signs of $q_1$, $q_2$ and
$q_3$. We can repeat this same exercise for the \emph{gravitational} potential
energy stored between three \emph{masses}. The gravitational potential energy
will always be negative.

\begin{example}
  Four identical point charges $q$ are located at the four corners of a square
  of length $L$. What is the total potential energy of the system?
  \begin{center}
    \vspace{-.2in}
    \begin{tikzpicture}[scale=1.1]
      \draw rectangle (2,2) node[midway,above=31]{$L$};
      \begin{scope}[thick]
        \foreach \x in {0,2}{
          \foreach \y in {0,2}
          \shade[poscharge] (\x,\y) circle (.16) node[white]{$q$};
        }
      \end{scope}
    \end{tikzpicture}
  \end{center}
  \textbf{Solution:}
\end{example}



\section{Electric Potential}
An object at a specific location inside a gravitational field has a
gravitational potential energy proportional to its mass, i.e.\
\begin{equation}
  U_g=V_gm
\end{equation}
This ``constant'' $V_g$ is called the \textbf{gravitational potential}, which
is the \emph{gravitational potential energy per unit mass}. In the trivial case
with a uniform gravitational field:
\begin{equation}
  V_g=\frac{U_g}m=gh
\end{equation}
This also applies to the general case of the gravitational potential energy:
\begin{equation}
  V_g=\frac{U_g}m=-\frac{Gm}r
\end{equation}  
This is also true for a charged particle inside an electric field, and the
constant is called the \textbf{electric potential}, which is ``electric
potential energy per unit charge''. For a point charge, it is defined as
\begin{equation}
  \boxed{
    V=\frac{U_q}q=\frac{kq_s}r
  }
\end{equation}
The unit for electric potential is a \textbf{volt}\footnote{Named after
Italian physicist Alessandro Volta} which is \emph{one joule per coulomb}:
\begin{equation*}
  \SI1\volt=\SI1{\joule\per\coulomb}
\end{equation*}

%\section{Electric Potential}
%  

%    \centering
%    \begin{tikzpicture}[scale=.8]
%      \shade[poscharge] circle (.3) node[white]{$q$};
%      \foreach \theta in {20,40,...,360}{
%        \draw[rotate=\theta,axes,gray] (.3,0)--(3,0);
%        \draw[rotate=\theta,gray,thick] (2.8,0)--(4,0);
%      }
%      \foreach \r in {1,2,3}{
%        \draw[dashed,very thick,red!70!black] circle (\r);
%        \node[red!70!black] at (0,-\r-.22){$V_\r$};
%      }
%    \end{tikzpicture}
% 
%    \vspace{-.4in}\begin{displaymath}
%      V_1>V_2>V_3
%    \end{displaymath}

For a point charge $q$, every point at a distance $r$ will have the same
electric potential $V(r)$.
\begin{itemize}
\item The (dotted red) lines have the same electric potential; they are
  called \textbf{equipotential lines}, or \textbf{equipotential contours}
\item Equipotential lines are perpendicular to the electric field lines
\item Electric field lines \emph{always} points from higher $V$ towards
  lower $V$
\end{itemize}

%    \centering
%    \begin{tikzpicture}[scale=.8]
%      \shade[poscharge] circle (.3) node[white]{$q$};
%      \foreach \theta in {30,60,...,360}{
%        \draw[rotate=\theta,axes,gray] (.3,0)--(4,0);
%      }
%      \foreach \r in {1,2,3}{
%        \draw[dashed,very thick,red!70!black] circle (\r);
%        \node[red!70!black] at (0,-\r-.22){$V_\r$};
%      }
%      \shade[negcharge,rotate=-20] (2,0) circle (.23) node[white]{$Q$};
%    \end{tikzpicture}
%    
%    A charge $Q$ that is placed inside this electric field will now have an
%    electric potential energy of:
%
%    \eq{-.1in}{
%      U_q=QV=Q\left[\frac{kq}r\right]
%    }
%
%    in agreement with equation for electric potential energy



\section{Electric Potential Difference}

\begin{figure}[ht]
  \centering
  \begin{tikzpicture}[scale=.75]
    \shade[poscharge] circle (.3) node[white]{$q$};
    \foreach \theta in {30,60,...,360}{
      \draw[rotate=\theta,axes,gray] (.3,0)--({7-4*sin(\theta/2)},0);
    }
    \draw[dashed,very thick,red!70!black,rotate=-70]
    (2.5,0) arc (0:100:2.5) node[left]{$V_2$};
    \draw[dashed,very thick,red!70!black,rotate=-60]
    (5,0) arc (0:90:5) node[left]{$V_1$};
    
    \begin{scope}[rotate=-30]
      \shade[negcharge] (2.5,0) circle (.23) node[white]{$Q$};
      \draw[<->,thick] (.3,0)--(2.27,0)
      node[midway,fill=white,rotate=-30]{$r_2$};
    \end{scope}
    \begin{scope}[rotate=15]
      \shade[negcharge] (5,0) circle (.23) node[white]{$Q$};
      \draw[<->,thick] (.3,0)--(4.77,0)
      node[pos=2/3,fill=white,rotate=15]{$r_1$};
    \end{scope}
  \end{tikzpicture}
\end{figure}
When a charge is moved from $r_1$ to $r_2$, the change in electric potential
energy is related to the change in electric potential by:
\begin{equation}
  \Delta U_q=U_2-U_1=Q\Delta V
\end{equation}
where $\Delta V$ is called the \textbf{potential difference}.
  


%\section{Electric Potential Difference (Voltage)}
Movement of the charged particle inside the electric field changes the
stored electric potential, i.e.\ \textbf{electric potential difference}. In
circuits, it is referred to as the \textbf{voltage}:
\begin{equation}
  \boxed{\Delta V=\frac{\Delta U_q}q}
\end{equation}
We can relate $\Delta V$ to an equation that we knew from Grade 11 Physics.
The energy dissipated in a resistor in a circuit $\Delta U_q$ to the voltage
drop $\Delta V$ is given by:
\begin{equation}
  \Delta U_q=Q\Delta V
\end{equation}




%\section{Getting Those Names Right}
%  \begin{itemize}
%  \item Electric potential energy between point charges:
%    
%    \eq{-.1in}{
%      U_q=\frac{kq_1q_2}r
%    }
%  \item Electric potential near a point charge:
%
%    \eq{-.1in}{
%      V=\frac{U_q}q=\frac{kq}r
%    }
%  \item Electric potential difference (voltage):
%
%    \eq{-.1in}{
%      \Delta V=\frac{\Delta U_q}q
%    }
%  \end{itemize}
%
%
%
%
%\section{Relating $U_q$, $\vec F_q$ and $\vec E$}
%  From the work-energy theorem:
%  \begin{itemize}  
%  \item Electrostatic force $\vec F_q$ always points from high to low potential
%    energy
%  \item Similarly, electric field $\vec E$ points from high to low electric
%    potential
%  \item Electric field can also be expressed as the change of electric
%    potential per unit distance, which has the unit
%    
%    \eq{-.1in}{
%      \SI1{\newton\per\coulomb}=\SI1{\volt\per\metre}
%    }
%  \item Electric field is also called ``electric potential gradient''
%  \end{itemize}
%
%
%
%
%\section{Example Problem}
%  \textbf{Example:} A small sphere with a charge of
%  $Q=\SI{-3.0}{\micro\coulomb}$ creates an electric field. Calculate the
%  electric potential at point $A$, located \SI{2.0}{\centi\metre} from the
%  source charge, and at point $B$, located \SI{5.0}{\centi\metre} from the same
%  source charge. Which point is at higher potential?
%  \begin{center}
%    \begin{tikzpicture}[scale=1.3]
%      \shade[poscharge] circle (.15) node[left=3]{$Q$};
%      \fill[red] (2,0) circle (.05) node[right,black]{$A$};
%      \draw[|<->|] (0,.2)--(2,.2)
%      node[midway,fill=white]{\SI{2.0}{\centi\metre}};
%      \begin{scope}[rotate=-28]
%        \fill[red] (5,0) circle (.05) node[right,black]{$B$};
%        \draw[|<->|] (0,-.2)--(5,-.2)
%        node[midway,rotate=-28,fill=white]{\SI{5.0}{\centi\metre}};
%      \end{scope}
%    \end{tikzpicture}
%  \end{center}


\begin{example}
  The diagram below shows three charges, $q_A=\SI{5.0}{\micro\coulomb}$,
  $q_B=\SI{-7.0}{\micro\coulomb}$, and $q_C=\SI{2.0}{\micro\coulomb}$, placed
  at three corners of a rectangle. Point $D$ is the fourth corner. What is the
  electric potential at point $D$?
  \begin{center}
    \vspace{-.25in}
    \begin{tikzpicture}[scale=.75]
      \draw rectangle (4,3);
      \shade[poscharge] (0,3) circle (.15) node[left]{$A$};
      \shade[poscharge] (4,3) circle (.15) node[right]{$B$};
      \shade[poscharge] (4,0) circle (.15) node[right]{$C$};
      \fill circle (.08) node[left]{$D$};
      \draw[|<->|] (0,3.5)--(4,3.5)
      node[midway,fill=ocre!8]{\SI{4.0}{\centi\metre}};
      \draw[|<->|] (-.81,3)--(-.81,0)
      node[midway,fill=ocre!8]{\SI{3.0}{\centi\metre}};
    \end{tikzpicture}
  \end{center}
  \vspace{-.2in}
  \textbf{Solution:} The electric potential at $D$ is the sum of the potentials
  generated by charges $A$, $B$ and $C$. Luckily, this is a \emph{scalar}
  addition:
  \begin{equation*}
    V_\text{at D}=V_A + V_B + V_C =
    \frac{kq_A}{r_A}+\frac{kq_A}{r_B}+\frac{kq_A}{r_C} =
    k\left(\frac{q_A}{r_A}+\frac{q_A}{r_B}+\frac{q_A}{r_C}\right)
  \end{equation*}
  where $r_A=\SI{3.0}{cm}$, $r_B=\sqrt{3.0^2+4.0^2}=\SI{5.0}{cm}$ and
  $r_C=\SI{4.0}{cm}$ are the distances from $D$ to $A$, $B$ and $C$,
  respectively. Substituting numerical values, we have:
  \begin{equation*}
    V_\text{at D} =
    (\num{8.99e9})\times
    \left(\frac{5.0}{3.0}-\frac{7.0}{5.0}+\frac{2.0}{4.0}\right)
    \times\frac{10^{-6}}{10^{-2}}=\boxed{\SI{1.234e5}{\volt}}
  \end{equation*}
  But what does this mean? In the original configuration, there is already
  energy stored between charges $A$ and $B$, between $B$ and $C$, and
  between $A$ and $C$.
\end{example}



%\section{Field Structure}
%
%\section{Field Structure}
%  How should we draw the field lines?
%  \begin{center}
%    \pic{.3}{graphics/plate1}
%    \pic{.3}{graphics/plate2}
%  \end{center}
%
%
%
%
%\section{Field Structure}
%  \centering
%  \pic{.7}{graphics/plate3}
%
%
%
%
\section{Electric Field between Two Parallel Plates}

\begin{center}
  \begin{tikzpicture}[xscale=.8]
    \draw[thick] rectangle (8,.1);
    \draw[thick] (0,1.4) rectangle (8,1.5);
    \foreach \x in {.4,.8,1.2,...,7.6}{
      \draw[axes] (\x,1.4)--(\x,.7);
      \draw[thick] (\x,1.4)--(\x,.1);
    }
    \foreach \x in {.8,1.6,2.4,...,7.2}{
      \node at (\x,1.78) {$+$};
      \node at (\x,-.28) {$-$};
    }
    
    \draw[axes] (0,1.4) ..controls(-.15,1.2).. (-.2,.7);
    \draw[thick] (-.2,.8) ..controls(-.15,.4).. (0,.1);
    
    \draw[axes] (8,1.4)..controls(8.15,1.2)..(8.2,.7);
    \draw[thick] (8.2,.8)..controls(8.15,.4)..(8,.1);
  \end{tikzpicture}
\end{center}
The electric field $\vec E$ between two charged parallel plates is uniform at
all points between the parallel plates, independent of position. The magnitude
is:
\begin{equation}
  \boxed{
    E=\frac\sigma{\epsilon_0}
  }
\end{equation}
where $\sigma=Q/A$ is the \textbf{surface charge density}, and
$\epsilon_0=\SI{8.85e-12}{\coulomb\squared\per\newton.\metre\squared}$ is a
universal constant called the \textbf{permittivity of free space}. $\mathbf E$
outside the plates is nearly zero, except for the fringe effects at the edges
of the plates.

Between parallel plates, there is a simple relationship between electric
field, voltage, and the distance between the plates:
\begin{equation}
  \boxed{
    E=\frac{\Delta V}d
  }
\end{equation}
%\begin{center}
%  \begin{tabular}{l|c|c}
%    \rowcolor{pink}
%    \textbf{Quantity} & \textbf{Symbol} & \textbf{SI Unit} \\ \hline
%    Electric field strength       & $E$ & \si{\newton\per\coulomb} \\
%    Electric potential difference & $\Delta V$ & \si\volt \\
%    Distance between parallel plates & $d$     & \si\metre
%  \end{tabular}
%\end{center}




%%\section{Example Problem}
%%  \textbf{Example}: An identical pair of metal plates is mounted parallel on
%%  insulating stands \SI{20}{\centi\metre} apart and equal amounts of opposite
%%  charges are placed on the plates. The electric field intensity at the
%%  midpoint between the plates is \SI{400}{\newton\per\coulomb}.
%%  \begin{enumerate}
%%  \item What is the electric field intensity at a point \SI{5.}{\centi\metre}
%%    from the positive plate?
%%  \item If the same amount of charge was placed on plates that have twice the
%%    area and are \SI{20}{\centi\metre} apart, what would be the electric field
%%    intensity at the point \SI{5.}{\centi\metre} from the positive plate?
%%  \item What would be the electric field intensity of the original plates if
%%    the distance of separation of the plates was doubled?
%%  \end{enumerate}
%%
%
%
%
%%\section{Example Problem}
%%  \textbf{Example 7:} Two parallel plates \SI{5.}{cm} apart are oppositely
%%  charged. The electric potential difference across the plates is \SI{80.}{V}.
%%  \begin{itemize}
%%  \item What is the electric field intensity between the plates?
%%  \item What is the potential difference at point A?
%%  \item What is the potential difference at point B?
%%  \item What is the potential difference between points A and B?
%%  \item What force would be experienced by a small \SI{2.}{\micro C} charge
%%    placed at point A?
%%  \end{itemize}

