%\documentclass[12pt,aspectratio=169]{beamer}
%\input{../mybeamer}

\chapter{Capacitors}
\label{chapter:capacitors}

%\section{Capacitors}
%
%{Electric Field Between Parallel Charged Plates}
%  \begin{center}
%    \begin{tikzpicture}[xscale=.8]
%      \draw[thick,fill=gray!30] rectangle (8,.1);
%      \draw[thick,fill=gray!30] (0,1.5) rectangle (8,1.6);
%      \foreach \x in {.4,.8,1.2,...,7.6}{
%        \draw[->,thick] (\x,1.5)--(\x,.7);
%        \draw[thick] (\x,1.5)--(\x,.1);
%      }
%      \foreach\x in {.8,1.6,2.4,...,7.2}{
%        \node at (\x,1.78) {\scriptsize $+$};
%        \node at (\x,-.28) {\scriptsize $-$};
%      }
%      
%      \draw[->,thick] (0,1.5) ..controls(-.15,1.2).. (-.2,.7);
%      \draw[thick] (-.2,.8) ..controls(-.15,.4).. (0,.1);
%
%      \draw[->,thick] (8,1.5) ..controls(8.15,1.2).. (8.2,.7);
%      \draw[thick] (8.2,.8) ..controls(8.15,.4).. (8,.1);
%    \end{tikzpicture}
%  \end{center}
%  \begin{itemize}
%  \item Two charged plates, each producing an electric field pointing in the
%    same direction
%  \item The total electric field is twice the value of \emph{one} infinite
%    plane, pointing from the positively charged plate toward the negatively
%    charged plate
%
%    \eq{-.1in}{
%      \boxed{E=\frac\sigma{\epsilon_0}}
%    }
%  \item $\vec E$ outside the plates is very low (close to zero), except for
%    fringe effects at the edges of the plates
%  \end{itemize}
%
%
%
%
%{Electric Field and Electric Potential Difference}
%  \begin{center}
%    \begin{tikzpicture}[xscale=.6,yscale=.8]
%      \draw[thick,fill=gray!30] rectangle (8,.1);
%      \draw[thick,fill=gray!30](0,1.5) rectangle (8,1.6);
%      \foreach \x in {.4,.8,1.2,...,7.6}{
%        \draw[->,thick] (\x,1.5)--(\x,.7);
%        \draw[thick] (\x,1.5)--(\x,.1);
%      }
%      \foreach\x in {.8,1.6,2.4,...,7.2}{
%        \node at (\x,1.78) {\scriptsize $+$};
%        \node at (\x,-.28) {\scriptsize $-$};
%      }
%      
%      \draw[->,thick] (0,1.5)..controls(-.15,1.2)..(-.2,.7);
%      \draw[thick] (-.2,.8)..controls(-.15,.4)..(0,.1);
%
%      \draw[->,thick] (8,1.5)..controls(8.15,1.2)..(8.2,.7);
%      \draw[thick] (8.2,.8)..controls(8.15,.4)..(8,.1);
%
%      \draw[thick,|<->|] (-.5,.1)--(-.5,1.5) node[midway,left]{$d$};
%    \end{tikzpicture}
%  \end{center}
%  In the case of two parallel plates, the electric field between the plates is
%  uniform, and the relationship simplifies to:
%
%  \eq{-.1in}{
%    \boxed{E=\frac{\Delta V}d}
%  }
%  \begin{center}
%    \begin{tabular}{l|c|c}
%      \rowcolor{pink}
%      \textbf{Quantity} & \textbf{Symbol} & \textbf{SI Unit} \\ \hline
%      Electric field intensity & $E$ & \si{\newton\per\coulomb}\\
%      Voltage between plates & $\Delta V$ & \si\volt \\
%      Distance between plates & $d$ & \si\metre
%    \end{tabular}
%  \end{center}
%
%
%
%{Capacitors}
\textbf{Capacitors} is a device that stores energy in an electric field. The
simplest form of a capacitor is a set of closely spaced parallel plates:
%  \begin{center}
%    \pic{.45}{cap19}
%  \end{center}

When the plates are connected to a battery, the battery transfer charges to the
plates until the voltage $V$ equals the battery terminals. After that, one
plate has charge $+Q$; the other has $-Q$.


\section{Parallel-Plate Capacitors}
The (uniform) electric field between two parallel plates is proportional to the
surface charge density $\sigma$, which is the charge $Q$ divided by the area of
the plates $A$:

%  \eq{-.1in}{
%    E=\frac{\color{red}\sigma}{\epsilon_0}=
%    \frac{\color{red}Q}{{\color{red}A}\epsilon_0}
%  }
  
Substituting this into the relationship between the plate voltage $V$ and
electric field, we find a relationship between the charges across the plates
and the voltage:

%  \eq{-.1in}{
%    V={\color{blue}E}d=
%    \frac{\textcolor{blue}{Q}d}{\textcolor{blue}{A\epsilon_0}}
%    \quad\longrightarrow\quad
%    \boxed{Q=\left[\frac{A\epsilon_0}d\right]V}
%  }
%

Since area $A$, distance of separation $d$ and the vacuum permittivity
$\epsilon_0$ are all constants, the relationship between charge $Q$ and voltage
$V$ is \emph{linear}. And the constant is called the \textbf{capacitance} $C$,
defined as:

%  \eq{-.1in}{
%    \boxed{C=\frac QV}
%  }
%
For parallel plates:
%
%  \eq{-.1in}{
%    C=\frac{A\epsilon_0}d
%  }
%
The unit for capacitance is a \textbf{farad} (named after Michael Faraday),
where $\SI1\farad=\SI1{\coulomb\per\volt}$.




\subsection{Cylindrical Capacitors}

%%    \pic{1.05}{cylindrical-capacitor}
%%
Not all capacitors are parallel plates. Cylindrical capacitors are also popular.
%%    \begin{itemize}
%%    \item The capacitor has length $\ell$ which is much larger than the radii
%%      of the inner \& outer cylinders ($a$, $b$)
%%    \item Inner cylinder has total charge $Q$
%%    \item Outer cylinder has total charge $-Q$
%%    \item Inside the capacitor, the electric field in the radial direction
%%    \item Outside of the capacitor, there is no electric field
%%    \end{itemize}



%%{Cylindrical Capacitors: Electric Field}
%%  \begin{columns}
%%    \column{.4\textwidth}
%%    \pic{1.05}{cylindrical-capacitor}
%%
%%    \column{.56\textwidth}
%%
%%    Using a bit of calculus, we can also see that, like the parallel plate, the
%%    relationship between voltage and charge is still linear. In this case, the
%%    capacitance is defined as:
%%
%%    \eq{-.1in}{
%%      \boxed{C=\frac QV=\frac{2\pi\ell\epsilon_0}{\ln(b/a)}}
%%    }
%%    
%%    The capacitance is generally expressed in terms of $C/\ell$. Capacitance
%%    depends only on the geometry (i.e.\ the raidii $a$ and $b$) and the
%%    permittivity.
%%  \end{columns}



\section{Practical Capacitors}

%    \column{.3\textwidth}
%    \pic1{Figure_20_05_05a}
%    
%    \column{.7\textwidth}
%    \begin{itemize}
%    \item Parallel-plate capacitors are very common in electric circuits, but
%      the vacuum between the plates is not very effective
%    \item Instead, a non-conducting \textbf{dielectric} material is inserted
%      between the plates
%    \item When the plates are charged, the electric field of the plates
%      polarizes the dielectric
%    \item The polarization  produces an electric field that opposes the field
%      from the plates, therefore reduces the effective voltage, and increasing
%      the capacitance
%    \end{itemize}
%
%
%
%
\section{Dielectric Constant}
If electric field without dielectric is $E_0$, then $E$ in the dielectric is
reduced by $\kappa$, the \textbf{dielectric constant}:
%
%  \eq{-.1in}{
%    \boxed{\kappa=\frac{E_0}E}
%  }
%
%  The capacitance of the plates with the dielectric is now amplified by the
%  same factor $\kappa$:
%
%  \eq{-.1in}{
%    \boxed{C=\kappa C_0}
%  }
%
%  We can also view the dielectric as something that increases the
%  \emph{effective permittivity}:
%  
%  \eq{-.1in}{
%    \boxed{\epsilon=\kappa\epsilon_0}
%  }
%
%
%
%
%{Dielectric Constant}
%  The dielectric constants of commonly used materials are:
%  \begin{center}
%    \begin{tabular}{l|l}
%      \rowcolor{pink}
%      \textbf{Material} & $\kappa$ \\ \hline
%      Air         & \num{1.00059} \\
%      Bakelite    & \num{4.9} \\
%      Pyrex glass & \num{5.6} \\
%      Neoprene    & \num{6.9} \\
%      Plexiglas   & \num{3.4} \\
%      Polystyrene & \num{2.55} \\
%      Water (\SI{20}\celsius) & \num{80} 
%    \end{tabular}
%  \end{center}
%
%
%
%
%{Storage of Electrical Energy}
%  When charging up a capacitor, imagine positive charges moving from the
%  ($-$) plate to the ($+$) plate.
%  \begin{center}c
%    \pic{.4}{slide14}
%  \end{center}
%  Initially neither plates are charged, so moving the first charge takes very
%  little work; as the electric field builds, more work needs to be done.
%  The total work done is the potential energy inside the capacitor:
%
%  \eq{-.1in}{
%    \boxed{U_c=\frac12\frac{Q^2}C=\frac12QV=\frac12CV^2}
%  }
%
%
%
%
%
%{Notes About Storage of Electric Energy}
%  \begin{itemize}
%  \item The presence of a dielectric \emph{increases} the capacitance; therefore
%    the work (and potential energy stored) to move a charge \emph{decreases}
%    with the dielectric constant $\kappa$
%  \item After the capacitor is charged, removing the dielectric material from
%    the capacitor plates will require additional work.
%  \end{itemize}
%
%
%
%
%{Capacitors in Electric Circuits}
%  Capacitors are an important part of an electric circuits because it stores
%  energy in the electric field
%  \begin{itemize}
%  \item Denoted by this symbol (with reference to the parallel-plate capacitor):
%    \begin{center}
%      \vspace{.1in}
%      \begin{tikzpicture}
%        \draw[thick] (3,0) to[C,o-o] (5,0);
%      \end{tikzpicture}
%    \end{center}
%  \item Act like a voltage source
%  \item Unlike a battery, the voltage increases or decreases as the charge
%    across the capacitor plates change.
%  \end{itemize}
