\documentclass{../oss-handout}
\usepackage{enumitem}
\usepackage{titlesec}
\usepackage{newtxtext}
\usepackage{xcolor,color,colortbl}

\setlength{\parindent}{0pt}
\setlength{\parskip}{2pt}
\setlength{\headheight}{26pt}

\titleformat*{\section}{\bfseries}
\titlespacing\section{0pt}{10pt plus 4pt minus 4pt}{4pt plus 12pt minus 5pt}

% Set the page style for the document
\pagestyle{plain}

% Course & handout information
\renewcommand{\institution}{Meritus Academy}
\renewcommand{\coursetitle}{Grade 11 Physics}
\renewcommand{\term}{Fall 2024}
\title{GRADE 11 PHYSICS COURSE OUTLINE}
\author{}
\date{\today}

\begin{document}
\thispagestyle{title}
\gentitle



\section{Course Objectives}
\begin{itemize}[nosep,leftmargin=20pt]
\item Develop analytical skills, strategies, and habits of mind required for
  scientific inquiry, including critical thinking and inferring
\item Develop communicative skills, strategies, and habits required for
  scientific inquiry
\item Learn fundamental concepts of introductory high-school physics
\item Extend fundamental concepts beyond the mandate of the Ontario curriculum
\item Gain exposure to both mainstream and unconventional applications of
  scientific concepts
\end{itemize}



\section{Teacher Information}
Teacher name \& contact information: \underline{\hspace{4.5in}}



\section{Class Times}
There are six sections of the Grade 11 Physics course in the Fall 2024 term.
\begin{center}
  \bgroup
  \def\arraystretch{1.25}
  \begin{tabular}{|c|c|c|c|}
    \hline
    \rowcolor{lightgray}
    \textbf{Section} & \textbf{Day} & \textbf{Time} & \textbf{Format}\\
    \hline\hline

    % FOR FALL 2024 SESSION
    Physics G11-1 & Wednesdays & 6:30 pm -- 9:00 pm & Online \\
    \hline
    Physics G11-2 & Fridays & 7:00 pm -- 9:30 pm & Online \\
    \hline
    Physics G11-3 & Saturdays & 10:30 am -- 1:00 pm & In-Person \\
    \hline
    Physics G11-4 & Sundays  & 1:20 pm -- 3:50 pm & In-Person \\
    \hline
    Physics G11-5 & Sundays  & 7:00 pm-- 9:30 pm & Online \\
    \hline
    %Physics G11-6 & Tuesdays & 6:30 pm-- 9:00 pm & Online \\
    %\hline

    
%    % FOR SUMMER 2024 SESSION
%    Physics G11-1 & Thursdays \& Mondays & 10:00 am -- 12:30 pm & Online \\
%    \hline
%    Physics G11-2 & Thursdays \& Mondays & 7:00 pm -- 9:30 pm & Online \\
%    \hline
%    Physics G11-3 & Tuesdays \& Fridays  & 1:00 pm -- 3:30 pm & Online \\
%    \hline
%    Physics G11-4 & Tuesdays \& Fridays  & 4:00 pm -- 6:30 pm & In-Person \\
%    \hline
%    Physics G11-5 & Saturdays \& Sundays  & 10:00 am-- 12:30 pm & In-Person \\
%    \hline
%    Physics G11-6 & Saturdays \& Sundays  & 1:00 pm-- 3:30 pm & In-Person \\
%    \hline
%    Physics G11-7 & Saturdays \& Sundays  & 7:00 pm -- 9:30 pm & Online \\
%    \hline
%    Physics G11-8 & Tuesdays, Thursdays, Fridays \& Mondays (Aug.) &
%    1:00 -- 3:30 pm & Online \\
%    \hline
    
    % FOR FALL 2024 SESSION
    %Physics G11-1 & Tuesdays  & 6:30 -- 9:00 pm & Online \\
    %\hline
    %\hspace{.3in}Physics G11-2\hspace{.3in} &
    %\hspace{.3in}Saturdays\hspace{.3in} &
    %\hspace{.3in}7:00 -- 9:30 pm\hspace{.3in} &
    %\hspace{.3in}In-person\hspace{.3in} \\
    %\hline
    %Physics G11-3 & Sundays   & 4:10 -- 6:40 pm & In-person \\
    %\hline
  \end{tabular}
  \egroup
\end{center}
Due to on-going constructions, in-person classes (sections 3 and 4) will be
online in September. Additionally, if you are unable to attend your class,
please contact \texttt{info@olympiads.ca} to request a Zoom link for make-up
class(es).


\section{Course Material}
No textbook is required. Presentation slides, handouts and homework sets are
downloadable from the school website; students are expected to download them
prior to class. Please have a pen/pencil (or tablet for online classes) for
note-taking, and a scientific calculator for working out in-class example
problems.



%\section{Classroom Expectations (Online)}
%
%Students attending the online sections are expected to:
%\begin{itemize}[nosep]
%\item Log into the Zoom meeting a few minutes before the start of the class
%\item Have your full name appear on the screen
%\item Have your camera turned on showing your face
%\item Be ready to learn and participate during class
%\item Type in all your in-class questions and comments into the chat window.
%  Don't wait too long before you ask a question; the longer you wait, the less
%  effective it will be to answer your questions
%\item Be \emph{specific} with your questions. As skilled as the teachers are,
%  vague statements and questions like ``I don't understand, can you explain it
%  again'' are impossible to answer
%\item Please inform your teacher if you have to leave the class early for
%  whatever reason
%\item Be respectful to yourself, your teacher, and your fellow students. You
%  are expected to act maturely and responsibly
%\end{itemize}


%\section{Classroom Expectations (In-Person)}
%Students attending the in-person sections are expected to:
%\begin{itemize}[nosep]
%\item Be in your seat and ready to learn and participate during class.
%\item Stay on task without disturbing or distracting others.
%\item Raise your hand if you have any questions or comments and wait to be
%  called. Don't wait too long before you ask a question.
%\item If you need to leave the class early, your parent needs to pick you up at
%  the classroom door, or be brought to the front desk by a secretary.
%\item Be respectful for yourself, others, and the facilities; always act in
%  a responsible manner.
%\end{itemize}


\section{Homework}
Homework questions are assigned after every class based on the topics covered.
There are usually about 15 to 20 questions, consisting of multiple-choice,
short-answer and problem-solving questions. Weekly homework sets are posted on
the school's website, and for online classes, on \texttt{classkick.com}.
Homework is due at the beginning of the following class. Homework question
reviews are done via pre-recorded videos.
%Homework questions are reviewed in their entirety during the
%homework take-up tutorial (see Section~\ref{tutorial}).

% and marked with ``P'' for pass and ``I'' for incomplete.
%\item Homework sets are distributed at the beginning of each class. (For online
%  classes, they are posted on Classkick. Please log into your account to check
%  for them.)
%\item Most of the homework questions will be reviewed in class before they are
%  due. However, this does \emph{not} mean you don't need to do your homework at
%  home. Always do your best.
%\item Homework questions are due at the end of the next class. (The exceptions
%  are the one-month and weekend classes during the summer session.)
%\item Late homework is accepted. However, the usefulness of late-submissions
%  are often diminished
%
%  you do get an ``I'', your teacher will let you know how to fix it so you can
%  re-submit to pass the homework.
%Not regularly completing the homework may prompts a phone call to communicate
%with the parents.
% in order to help you better manage your time and achieve your
%goal.


\section{Tests}
There are two tests for assessment:
\begin{itemize}[nosep]
\item A \textbf{take-home midterm test} assigned for Class 7 (instead of a
  homework set), covering material from the first two units.
\item An \textbf{in-class final test} on Class 16, covering material from the
  \emph{entire} course.
\end{itemize}


\section{Online Tutorial}
\label{tutorial}
For the Fall 2024 session, there is a weekly online drop-in tutorial for Grade
11 Physics students. The tutorial is shared with Grade 12 Physics. Zoom
link for the tutorial can be found in the student account.
%There are two online tutorials. Homework questions from the previous class will
%be taken-up during the Monday session, while the Thursday evening drop-in
%sessions is for students to ask questions.
\begin{center}
  \bgroup
  \def\arraystretch{1.15}
  \begin{tabular}{|l|c|c|c|}
    \rowcolor{lightgray}
    \hline
    & \textbf{Day} & \textbf{Time} & \textbf{Dates} \\
    \hline\hline
%    \textbf{Homework take-up tutorial} &
%    \hspace{.2in}Monday\hspace{.2in} &
%    \hspace{.2in}6:30 -- 7:30 pm\hspace{.2in} &
%    \hspace{.2in}February 8 -- May 23\hspace{.2in} \\
%    \hline
    \textbf{Online drop-in tutorial} &
    Tuesdays & 6:00 pm to 7:00 pm & September 17, 2024 -- January 7, 2025 \\
    \hline
  \end{tabular}
  \egroup
\end{center}


\section{Academic Integrity}
Meritus Academy values academic integrity. Students are encouraged to complete
their homework and tests using knowledge gained from class. Handing in work
that are copied from the internet, or generated by AI is not allowed. If a
student is suspected to have copied, cheated or plagiarized, they will be
temporarily assigned a 0\% or incomplete, and the issue will be brought to the
school administration.


%\section{Homework Standard}
%\begin{itemize}
%\item You do not need to show work for multiple-choice questions. For online
%  classes, please answer them directly on selection box on Classkick. For
%  in-person classes, simply circle the correct letter.
%\item For problem-solving questions, you must show \emph{all} work by providing
%  complete and organized steps.
%  \begin{itemize}[nosep]
%  \item Always write out the relevant equations \emph{before} substituting
%    numerical values
%  \item If you introduce any new variables, explain what they are
%  \item Draw diagrams whenever necessary, or whenever it helps to explain your
%    work
%  \item Only use standard variable names in you calculations (e.g.\ when
%    calculating speed, always use $v$ instead of ``$X$'')
%  \item Show all algebraic steps and numerical calculations
%  \item Circle or box all your final answers
%  \item Proper math format must be observed (e.g.\ proper use of ``='' sign,
%    units, etc.)
%  \end{itemize}
%  In short, answer the questions as if the reader is learning the concept from
%  you, not as if they already understands it.
%\item If a question requires you to \emph{explain}, please do so using short
%  complete sentences with supporting detail. There is no need to write full
%  paragraphs.
%\end{itemize}


\section{Class Schedule}

\bgroup
\def\arraystretch{1.15}
\begin{tabular}{|c|c|p{3.4in}|c|}
  \hline
  \rowcolor{lightgray}
  \textbf{Class} & \textbf{Date} & \textbf{Description} & \textbf{HW Due} \\
  \hline\hline
  1 & September 10 -- 15 & Motion (1): Motion quantities & --- \\
  \hline
  2 & September 17 -- 22 & Motion (2): Kinematic equations \& motion graphs
  & HW 1\\
  \hline
  3 & September 24 -- 29 & Motion (3): Two-dimensional vectors & HW 2 \\
  \hline
  4 & October 1 -- 6 & Motion (4): Projectile motion \& relative motion &
  HW 3 \\
  \hline
  5 & October 8 -- 13 & Forces (1): Laws of motion \& common forces &
  HW 4 \\
  \hline
  6 & October 15 -- 20 & Forces (2): Dynamics & HW 5 \\
  \hline
  \rowcolor{lightgray!50}
  7 & October 22 -- 27 &
  \parbox{3.4in}{\vspace{.07in}Energy (1): Mechanical work \& kinetic energy\\
    \textbf{Take-home midterm test}\vspace{.07in}} & HW 6 \\
  \hline
  8 & October 29 -- November 3 & Energy (2): Conservation of energy &
  Midterm \\
  \hline
  9 & November 5 -- 10 & Energy (3): Thermodynamics & HW 8 \\
  \hline
  10 & November 12 -- 17 & Energy (4): Nuclear physics & HW 9 \\
  \hline
  11 & November 19 -- 24 & Waves (1): Mechanical waves & HW 10 \\
  \hline
  12 & November 26 -- December 1 & Waves (2): Sound waves and music & HW 11 \\
  \hline
  13 & December 3 -- 8 & E\&M (1): Electricity & HW 12 \\
  \hline
  14 & December 10 -- 15 & E\&M (2): Magnetism & HW 13 \\
  \hline
  15 & December 17 -- 22 & E\&M (3): Power generation & HW 14 \\
  \hline
  \rowcolor{lightgray!30}
  16 & January 7 -- 12 & \textbf{In-class final test} & HW 15 \\
  \hline
\end{tabular}
\egroup
%\vspace{.1in}For the summer 2024 session, due dates for weekend classes may be
%adjusted
\end{document}
