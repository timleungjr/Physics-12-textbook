\chapter{Electric Field}

%\section{The Charges Are: Let's Review Some Basics}
%  We already know quite a bit about charge particles:
%  \begin{itemize}
%  \item The \emph{net charge} of an object means an excess of protons or
%    electrons.
%  \item The SI unit for electric charge is a \emph{coulomb} (\si\coulomb),
%    which is the amount of charges that \SI1{\ampere} of current passes through
%    a circuit in \SI1{\second} (i.e.\ $\SI1\coulomb=\SI1{\ampere\second}$)
%  \item Elementary charge $e=\SI{1.602e-19}\coulomb$
%  \end{itemize}
%\end{frame}




\section{Electric Charge}

From the onset, we know that there are two types of charge, which we have
assigned a positive or negative sign to facility mathematical calculations.

The SI unit of charge is a \emph{coulomb}, which is defined as the amount of
charge that one \emph{amp\`{e}re} of current passing through a circuit in one
second:
\begin{equation*}
  \SI1\coulomb=\SI1\ampere\cdot\SI1\second
\end{equation*}
At this time, we have not defined what an amp\`{e}re of current is. It is one
of the seven base SI units, and it is defined based on the magnetic force that
two current-carrying wire exert on each other.

With the development of atomic theory in the beginning of the 20th Century, we
understand that there are two stable elementary particles that carry an
electrical charge. The positive charge is carried by the proton, while the
negative the charge is carried by the electron. The proton is inside the
nucleus of the atom, while the electron orbits around the nucleus.

%    \begin{center}
%      \textbf{Proton}
%    \end{center}
%    \begin{itemize}
%    \item Mass: $m_p=\SI{1.673e-27}{\kilo\gram}$
%    \item Charge: $e^+=+\SI{1.602e-19}\coulomb$
%    \end{itemize}
%
%    \begin{center}
%      \textbf{Electron}
%    \end{center}
%    \begin{itemize}
%    \item Mass: $m_e=\SI{9.109e-31}{\kilo\gram}$
%    \item Charge: $e^-=\SI{-1.602e-19}\coulomb$
%    \end{itemize}


Although the mass of the electron and the proton are different, they both carry
the same amount of charge, called the elementary charge $e$, with a value of
\begin{equation*}
  e=\SI{-1.602e-19}\coulomb
\end{equation*}
The proton carries a charge of $+e$, while the electron carries a charge of
$-e$.

A ``charge particle'' $q$ may be a single electron, a single proton, an ion, or
any object that has an imbalance of protons and electrons. The total charge of
an object is the difference between the number of protons ($N_\text{proton}$) and
electrons ($N_\text{electron}$), multiplied by the elementary charge $e$:
\begin{equation}
  q=e|N_\text{proton}-N_\text{electrons}|
\end{equation}



\section{Coulomb's Law for Electrostatic Force}

We will begin with \textbf{electrostatics}, where charges that are not moving
(or moving slowly) relative to one another

Electric charges exert an \textbf{electrostatic force}\footnote{Also known as
the \textbf{electric force}, or the \textbf{coulomb force}. It is a
fundamental force; and also a conservative force.} $\mathbf F_q$ on each other.
The direction of the force is based on the \emph{sign} of the charges:
\begin{itemize}
\item Like charges \emph{repel} (positive \& positive; negative \& negative)
\item Opposite charges \emph{attract} (positive \& negative)
\end{itemize}
as shown in Fig.~\ref{fig:coulomb-charge}.
\begin{figure}[ht]
  \centering
  \begin{subfigure}{.3\linewidth}
    \centering
    \begin{tikzpicture}[scale=.6]
      \draw[vectors] (0,0)--(2,0) node[above]{$\mathbf F_q$};
      \draw[vectors] (6,0)--(4,0) node[above]{$\mathbf F_q$};
      \draw[poscharge] circle (.4) node[white]{$\bm +$};
      \draw[negcharge] (6,0) circle (.4) node[white]{$\bm -$};
    \end{tikzpicture}
    \caption{Electrostatic force between opposite charges is attractive}
  \end{subfigure}
  \begin{subfigure}{.3\linewidth}
    \centering
    \begin{tikzpicture}[scale=.6]
      \draw[vectors] (0,0)--(-2.5,0) node[above]{$\mathbf F_q$};
      \draw[vectors] (2.5,0)--(5,0) node[above]{$\mathbf F_q$};
      \draw[poscharge] circle (.4) node[white]{$\bm +$};
      \draw[poscharge] (2.5,0) circle (.4) node[white]{$\bm +$};
    \end{tikzpicture}
    \caption{Electrostatic force between two positive charges is repulsive}
  \end{subfigure}
  \begin{subfigure}{.3\linewidth}
    \centering
    \begin{tikzpicture}[scale=.6]
      \draw[vectors] (0,0)--(-2.5,0) node[above]{$\mathbf F_q$};
      \draw[vectors] (2.5,0)--(5,0) node[above]{$\mathbf F_q$};
      \draw[negcharge] circle (.4) node[white]{$\bm -$};
      \draw[negcharge] (2.5,0) circle (.4) node[white]{$\bm -$};
    \end{tikzpicture}
     \caption{Electrostatic force between two negative charges is repulsive}
  \end{subfigure}
  \caption{Electrostatic force between charged particles}
  \label{fig:coulomb-charge}
\end{figure}


%\begin{center}
%  \begin{tikzpicture}[scale=.5]
%    \draw[vectors] (0,0)--(3,0) node[right]{$F_q$};
%    \draw[vectors] (10,0)--(7,0) node[left]{$F_q$};
%    \draw[poscharge] circle (.3) node[white]{$+$} node[above=3]{$q_1$};
%    \draw[negcharge] (10,0) circle (.3) node[white]{$-$} node[above=3]{$q_2$};
%  \end{tikzpicture}
%\end{center}
The magnitude of the electrostatic force between two \emph{point charges} is
given by \textbf{Coulomb's law}:
\begin{important-equation}
  F_q=\frac{k\left|q_1q_2\right|}{r^2}
\end{important-equation}
%\begin{center}
%  \begin{tabular}{l|c|c}
%    \rowcolor{pink}
%    \textbf{Quantity} & \textbf{Symbol} & \textbf{SI Unit} \\ \hline
%    Magnitude of electrostatic force & $F_q$ &  \si{\newton} \\
%    Coulomb's constant (electrostatic constant) & $k$ & \si{N.m^2/C^2} \\
%    Point charges 1 and 2 & $q_1$, $q_2$ & \si\coulomb \\
%    Distance between point charges & $r$ & \si\metre
%  \end{tabular}
%\end{center}
where $k$ is called \textbf{Coulomb's constant},with a value of
\begin{equation*}
  k=\SI{8.99e9}{\newton\metre\squared\per\coulomb\squared}
\end{equation*}
Coulomb's law is defined based on another fundamental constant called the
\textbf{permittivity of free space}, or \textbf{vacuum permittivity}
($\varepsilon_0$):
\begin{important-equation}
  k=\frac1{4\pi\varepsilon_o}
\end{important-equation}
Permittivity of free space is a measure of how well a vacuum can resist the
formation of an electric field, with a value of
\begin{equation*}
  \varepsilon_0=\SI{8.85e-12}{C^2/N.m^2}
\end{equation*}
A vacuum is not particularly good at resisting electric fields from forming,
hence the very small value of $\varepsilon_0$. However, inside dielectric
materials, the effective permittivity would be much higher.

Like the point mass model used in the law of universal gravitation, in
Coulomb's law, charges $q_1$ and $q_2$ are also assumed to be
\emph{point charges} that do not occupy any space. In reality, of course, point
charges do not actually exist, therefore $r$ cannot be zero, and $\mathbf F_q$
cannot be infinitely large. But whether you can treat a charge as a point
charge is a matter of scale. If the charge is the size of an electron or a
proton, then even at the scale of the atom. However, if the charge is a large
piece of metal of irregular shape, then the point-charge model would be very
inaccurate near the charge. Additionally, using the point charge model for
charges, we must also assume that one charge is not inside another.

As is the cae for \emph{all} forces, the elctrostatic force obeys the third law
of motion: if $q_1$ exerts an electrostatic force $\mathbf F_q$ on $q_2$, then
$q_2$ likewise exerts a reaction force of $-\mathbf F_q$ on $q_1$. The two
forces are equal in magnitude and opposite in direction.

%  \begin{center}
%    \begin{tikzpicture}[scale=.65]
%      \draw[vectors,red] (0,0)--(2,0) node[right]{$\mathbf F_q$};
%      \draw[vectors,blue] (8,0)--(6,0) node[left]{$\mathbf F_q$};
%      \shade[ball1] circle (.2) node[above=2]{$q_1$};
%      \shade[ball2] (8,0) circle (.2) node[above=2]{$q_2$};
%      \draw[|<->|,thick] (0,-1)--(8,-1) node[midway,fill=black!2]{$r$};
%    \end{tikzpicture}
%  \end{center}
%  \begin{itemize}
%  \item\vspace{-.1in}
%  \item We assume that one charge is not \emph{inside} the other
%  \end{itemize}


Coulomb's law has a very similar form to the law of universal
%gravitation (Eq.~\ref{eq:law-of-gravity}).
gravitation.
The major difference is that whereas masses are \emph{always} positive, and the
gravitational force is \emph{always} attractive, charges can be either positive
or negative, and therefore the electrostatic force may be attractive or
repulsive. Nevetheless, much of the analysis used




%\section{Comparing $\mathbf F_g$ to $\mathbf F_q$}

%    \centering
%    Electrostatic force (point charge):
%
%    \eq{-.2in}{
%      \boxed{F_q=\frac{k|q_1q_2|}{r^2}}
%    }
%    
%    \centering
%    Gravitational force (point mass):
%
%    \eq{-.2in}{
%      \boxed{F_g=\frac{Gm_1m_2}{r^2}}
%    }
%  
%  \vspace{.15in}Similarities:
%  \begin{itemize}
%  \item Both inversely proportional to $r^2$ (inverse square law)
%  \item Both are scaled by a constant
%  \end{itemize}
%  Differences:
%  \begin{itemize}
%  \item For gravity there are only positive masses (only attraction)
%  \item Electric charge can be either positive or negative (charges can
%    attract or repel)
%  \end{itemize}


\subsection{Electrostatic Force from More Than One Charge}

For a charge $Q$ that is subjected to the influence of multiple discrete
point charges $q_i$, as shown in Fig.~\ref{fig:multiple-Fq}, the total
electrostatic force that $Q$ experiences is the vector sum of all the forces
$\mathbf F_i$:
\begin{equation}
  \mathbf F_\text{tot}
  =\sum_{i=1}^N\mathbf F_i
  =\mathbf F_1 + \mathbf F_2 + \mathbf F_3 + \cdots
\end{equation}
Note that it is common in AP Physics 2 to have questions that require
calculating the vector sum of the forces, but usually it is straightforward
(i.e.\ the geometry is relatively simple).
\begin{figure}[ht]
  \centering
  \begin{tikzpicture}[scale=.4]
    \shade[ball color=red] circle (.7) node[white]{$Q$};

    \draw[vectors,blue] (0,.7)--(0,3) node[right]{$\mathbf F_1$};
    \shade[ball color=blue] (0,8) circle (.6) node[white]{$q_1$};

    \draw[vectors,orange](.7,0)--(4,0) node[right]{$\mathbf F_2$};
    \shade[ball color=orange] (8,0) circle (.6) node[white]{$q_2$};

    \draw[vectors,violet,rotate=45](-.7,0)--(-3.5,0)
    node[below]{$\mathbf F_3$};
    \shade[ball color=violet] (8,8) circle (.6) node[white]{$q_3$};

    \draw[vectors,rotate=-35](.7,0)--(2.7,0) node[right]{$\mathbf F$};
  \end{tikzpicture}
  \caption{A charge $Q$ that is subjected to multiple electrostatic force}
  \label{fig:multiple-Fq}
\end{figure}


There is a net force that acts on $Q$, therefore it will accelerate (second
law of motion) with an acceleration of
\begin{equation*}
  a=\frac{\mathbf F_\text{tot}}{m_Q}
\end{equation*}
where $m_Q$ is the mass of charge $Q$. However, we note that even if
$q_1,\ldots,q_N$ remain stationary, as soon as $Q$ moves, all the forces acting
on it (and therefore its acceleration) will change in both magnitude and
direction. In general, predicting the motion of $Q$ is difficult because of the
changing forces and acceleration, even with calculus.

%    \begin{tikzpicture}[scale=.37]
%      \shade[ball1] circle(.7) node[white]{$Q$};
%
%      \draw[vectors,blue] (0,.7)--(0,3) node[right]{$\mathbf F_1$};
%      \shade[ball2] (0,8) circle(.6) node[white]{$q_1$};
%
%      \draw[vectors,orange](.7,0)--(4,0) node[right]{$\mathbf F_2$};
%      \shade[ball color=orange] (8,0) circle (.6) node[white]{$q_2$};
%
%      \draw[vectors,violet,rotate=45](-.7,0)--(-3.5,0)
%      node[below]{$\mathbf F_3$};
%      \shade[ball color=violet] (8,8) circle (.6) node[white]{$q_3$};
%
%      \draw[vectors,rotate=-35](.7,0)--(2.7,0) node[right]{$\mathbf F$};
%    \end{tikzpicture}



\section{Electric Field}


%
%\begin{frame}{Review of Gravitational Field (from AP Physics 1)}
We begin the formulation of electric field by reviewing how
\emph{gravitational} field is derived for a point source mass. Gravitational
force is based on the law of universal gravitation:
\begin{equation*}
  F_g=mg \quad\longrightarrow\quad 
  F_g = \frac{Gm_1m_2}{r^2} =
  \underbracket[1pt]{\left[\frac{Gm_1}{r^2}\right]}_gm_2
\end{equation*}
From this, we can define the gravitational field, generated from a source point
mass $m_1$:
\begin{equation*}
  \mathbf g(m_s,\mathbf r)=-\frac{Gm_s}{|\mathbf r|^2}\hat{\mathbf r}
\end{equation*}
Gravitational field $\mathbf g$ is function of two variables: source $m_S$ and
distance $r$ from the source. It is a ``vector field'' that shows how $m_S$
affects the gravitational force on other objects. As long as you know what
$\mathbf g$, you can also find the force that acts on a mass:
$\mathbf F_g=m\mathbf g$.



We can define the \textbf{electric field} by repeating the same procedure as
with gravitational field. Grouping the variables in Coulomb's law:
\begin{equation*}
  F_q=\underbracket[1pt]{\left[\frac{k|q_1|}{r^2}\right]}_{E}q_2
\end{equation*}
A source charge $q_s$ creates an ``electric field'' ($\mathbf E$) with a
magnitude\footnote{i.e.\ also known as the \emph{strength}, or \emph{intensity}
of the field} of
\begin{important-equation}
  E(q_s,r)=\frac{k|q_s|}{r^2}
  \label{eq:point-charge-electric-field}
\end{important-equation}
where $r$ is the distance from the point charge.
%\begin{center}
%  \begin{tabular}{l|c|c}
%      \rowcolor{pink}
%      \textbf{Quantity} & \textbf{Symbol} & \textbf{SI Unit} \\ \hline
%      Electric field strength & $E$ & \si{\newton\per\coulomb} \\
%      Coulomb's constant & $k$ & \si{N.m^2/C^2} \\
%      Source charge & $q_s$ & \si\coulomb \\
%      Distance from source charge & $r$ & \si\metre
%  \end{tabular}
%\end{center}
Similar to gravitational field, electric field $\mathbf E$ created by $q_s$ is
a function that shows how it influences other charged particles around it. The
direction of the field is radially outward from a positive point charge and
radially inward towards a negative charge.

For those who have a strong background in vectors, we can directly express
the electric field from a point charge in vector form, just as we did with
gravity:
\begin{important-equation}
  \mathbf E(q_s,\mathbf r)=\frac{kq_s}{|\mathbf r|^2}\hat{\mathbf r}
\end{important-equation}
where $\hat{\mathbf r}$ is the \emph{outward radial direction}. If the source
charge $q_s$ is positive, the direction of the electric field points away from
the charge, (in the $\hat{\mathbf r}$ direction with a magnitude speicified in
Eq.~\ref{eq:point-charge-electric-field}). Conversely, if $q_s$ is negative,
the direction of the electric field is towards the charge.

%  \centering
%  \begin{tikzpicture}
%    \shade[poscharge] circle (.25) node[midway,white]{$Q$};
%    \begin{scope}[vectors,blue]
%      \foreach \theta in {0,45,...,359} \draw[rotate=\theta] (.5,0)--+(1.2,0);
%      \foreach \theta in {0,30,...,359} \draw[rotate=\theta] (2.3,0)--+(.5,0);
%    \end{scope}
%    \node[left,blue] at (2.3,0) {$\mathbf E$};
%    \node[text width=153, draw=blue, fill=cyan!5, text=blue] at (0,3.5) {
%      \scriptsize Positive charge $Q$ generates an electric field with a
%      magnitude of
%
%      \vspace{-.07in}
%      \begin{displaymath}
%        E=\frac{kQ}{r^2}
%      \end{displaymath}
%      The direction of $\mathbf E$ is \emph{away from} the charge.\par
%    };
%
%    \uncover<3->{
%      \begin{scope}[violet,rotate=30]
%        \draw[vectors] (-1.2,0)--+(-1.5,0) node[above=0]{$\mathbf F_q$};
%        \draw[thick,fill=orange!5] (-1.2,0) circle (.18) node{$q$};
%      \end{scope}
%      \node[
%        text width=190,
%        draw=violet,
%        fill=violet!5,
%        text=violet] at (0,-2.7){\scriptsize Positive charge $q$ is inside the
%        electric field generated by {\color{blue}$Q$}, therefore it
%        experiences an electric force:
%        
%        \vspace{-.08in}
%        \begin{displaymath}
%          \mathbf F_q=q{\color{blue} \mathbf E}
%        \end{displaymath}
%        in the same direction as {\color{blue}$\mathbf E$}. The force is
%        repulsive.\par
%      };
%    }
%  \end{tikzpicture}
%  \hspace{.3in}
%  \begin{tikzpicture}
%    \uncover<2->{
%      \shade[negcharge] circle (.3) node[midway,white]{$-Q$};
%      \begin{scope}[vectors,magenta]
%        \foreach \x in {0,45,...,359} \draw[rotate=\x] (1.5,0)--+(-1,0);
%        \foreach \x in {0,30,...,359} \draw[rotate=\x] (2.8,0)--+(-.4,0);
%      \end{scope}
%      \node[right,magenta] at (1.5,0) {$\mathbf E$};
%      \node[text width=144, draw=magenta, fill=magenta!5, text=magenta]
%      at (0,3.5) {\scriptsize Negative charge $-Q$ generates an electric field
%        with a magnitude of
%
%        \vspace{-.07in}
%        \begin{displaymath}
%          E=\frac{k|-Q|}{r^2}=\frac{kQ}{r^2}
%        \end{displaymath}
%        The direction of $\mathbf E$ is \emph{towards} the charge.\par
%      };
%    }
%    \uncover<4->{
%      \begin{scope}[orange,rotate=15]
%        \draw[vectors] (-2.5,0)--+(1.2,0) node[right=0]{$\mathbf F_q$};
%        \draw[thick,fill=orange!5] (-2.5,0) circle (.18) node{$q$};
%      \end{scope}
%      \node[
%        text width=190,
%        draw=orange,
%        fill=orange!5,
%        text=orange] at (0,-2.7){\scriptsize Positive charge $q$ is inside the
%        electric field generated by {\color{magenta}$-Q$}, therefore it
%        experiences an electric force:
%        
%        \vspace{-.08in}
%        \begin{displaymath}
%          \mathbf F_q=q{\color{magenta}\mathbf E}
%        \end{displaymath}
%        in the same direction as {\color{magenta}$\mathbf E$}. The force is
%        attractive.\par
%      };
%    }
%  \end{tikzpicture}

When another charge $q$ enters the electric field $\mathbf E$, the field exerts
an electrostatic force $\mathbf F_q$ on the charge, regardless of what created
the electric field.
\begin{important-equation}
  \mathbf F_q=q\mathbf E
\end{important-equation}
The equation is similar to the law of gravitation ($\mathbf F_g=m\mathbf g$),
but with $\mathbf F_q$ replacing $\mathbf F_g$, $q$ replacing $m$, and $\mathbf
E$ replacing $\mathbf g$. If a positive charge ($q>0$) is inside the field,
$\mathbf F_q$ is in the \emph{same} direction as $\mathbf E$; if it is a
negative charge ($q<0$), $\mathbf F_q$ is in the \emph{opposite} direction as
$\mathbf E$.



\subsection{Electric Field from Multiple Charges}

When there are multiple source charges, then the total electric field at any
location is the vector sum of all the electric field from each source
charge, i.e.:
\begin{equation}
  \mathbf E=\sum_{i=1}^N \mathbf E_i
  =\mathbf E_1+\mathbf E_2+\mathbf E_3+\mathbf E_4+\cdots+\mathbf E_N
\end{equation}
Once the total electric field is known, we can use the equation
$\mathbf F_q=q\mathbf E$ without knowing \emph{what} created $\mathbf E$.





\begin{example}
  Three charges, $q_A=\SI{6.0}{\micro\coulomb}$,
  $q_B=\SI{-5.0}{\micro\coulomb}$, and $q_C=\SI{6.0}{\micro\coulomb}$, are
  located at three corners of a square with sides $L=\SI{5.0}{\centi\metre}$
  long. What is the electric field at $D$?
  %Remember that electric field is a vector.
  \begin{center}
    \vspace{-.2in}
    \begin{tikzpicture}%[scale=1.4]
      \draw[axes] (2,2)--(3,2) node[right]{$x$};
      \draw[axes] (2,2)--(2,3) node[above]{$y$};
      \draw rectangle (2,2) node[midway,above]{$L$};
      \shade[poscharge] circle (.1) node[below left] {$B$};
      \shade[poscharge] (2,0) circle (.1) node[below right]{$C$};
      \shade[poscharge] (0,2) circle (.1) node[above left] {$A$};
      \fill (2,2) circle (.045) node[above right,black]{$D$};
      
    \end{tikzpicture}
  \end{center}
  \textbf{Solution:} If the charges are small, we can assume that they are
  \emph{point} charges.
\end{example}



\section{Electric Field Lines}

\textbf{Electric field lines} can be used to visualize the direction of the
electric field.
%  If you place a positive charge in an electric field, the force on the charge
%  will be in the direction of the electric field.

%\pic{.4}{electricField/graphics/pos_charge}
%\pic{.4}{electricField/graphics/neg_charge}\\
%  \pic{.8}{electricField/graphics/2charges}

\begin{figure}[ht]
  \centering
  \begin{subfigure}{.45\textwidth}
    \centering
    \begin{tikzpicture}[scale=.7]
      \shade[poscharge] circle (.4) node[white]{\tiny $+q$};
      \foreach \theta in {15,30,...,360}{
        \draw[rotate=\theta,->,red!70!black,thick] (.35,0)--(3,0);
        \draw[rotate=\theta,red!70!black,thick] (2.8,0)--(4,0);
      }
    \end{tikzpicture}
    \caption{A positive point charge}
  \end{subfigure}
  \begin{subfigure}{.45\textwidth}
    \centering
    \begin{tikzpicture}[scale=.7]
      \shade[negcharge] circle (.4) node[white]{\tiny $-q$};
      \foreach \theta in {15,30,...,360}{
        \draw[rotate=\theta,red!70!black,thick] (.35,0)--(2.5,0);
        \draw[rotate=\theta,<-,red!70!black,thick] (2.3,0)--(4,0);
        }
    \end{tikzpicture}
    \caption{A negative point charge}
  \end{subfigure}
\end{figure}

\begin{itemize}
\item Electric field lines must begin and/or end at a charge
\item Field lines do not cross
\item Direction of the electric field is tangent to the field lines
\end{itemize}
