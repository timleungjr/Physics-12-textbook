\chapter{Circular Motion}
\label{chapter:circ-motion}

The \textbf{circular motion} (Fig.~\ref{fig:circ-motion1}) is the simplest
form of curvilinear motions, an object of mass $m$ moves in a circular path
about a fixed centre.
Like we did in Chapters~\ref{chapter:kinematics} and \ref{chapter:dynamics},
we will begin studying the circular motion, first be defining the coordinate
system, and then to the kinematic quantities, and then onto dynamics.

\section{Polar Coordinates}
In the majority of two-dimensional motion, 
%discussed in earlier chapters,
it is usually the most convenient to describe an object's position using the
Cartesian coordinate system, i.e.\ using the $x$ and $y$ coordinates as
functions of time:
\begin{equation}
  \bm r(t)=x(t)\hat{\bm x} + y(t)\hat{\bm y}
\end{equation}
However, for circular motion or general \textbf{curvilinear motions}, the
\textbf{polar coordinate system} is preferred. The position of an object is
described by:
\begin{equation}
  \bm r(t)=r(t)\hat{\bm r} + \theta(t)\hat{\bm\theta}
\end{equation}
where $r(t)$ is distance from the origin, and $\theta(t)$ is the standard
angle, measured counterclockwise from the $x$ axis. Some examples of
curvilinear motion are shown in Fig.~\ref{fig:curvilinear-motions}.

It is clear from basic geometry that Cartesian and polar coordinates are
related by:
\begin{align*}
  x(t)&=r(t)\cos\left(\theta(t)\right)\\
  y(t)&=r(t)\sin\left(\theta(t)\right)
\end{align*}

\begin{figure}[ht]
  \centering
  \begin{subfigure}{.4\linewidth}
    \centering
    \begin{tikzpicture}
      \draw[axes] (-2,0)--(2,0) node[right]{$x$};
      \draw[axes] (0,-2)--(0,2) node[above]{$y$};
      \draw[function] circle (1.5);
      \end{tikzpicture}
    \caption{Circular motion}
    \label{fig:circ-motion1}
  \end{subfigure}  
  \begin{subfigure}{.4\linewidth}
    \centering
    \begin{tikzpicture}
      \draw[axes] (-2,0)--(2,0) node[right]{$x$};
      \draw[axes] (0,-2)--(0,2) node[above]{$y$};
      \draw[function,rotate=30] ellipse (1.8 and 1);
    \end{tikzpicture}
    \caption{Elliptical motion}
  \end{subfigure}
  
  \begin{subfigure}{.4\linewidth}
    \centering
    \begin{tikzpicture}
      \draw[axes] (-2,0)--(2,0) node[right]{$x$};
      \draw[axes] (0,-2)--(0,2) node[above]{$y$};
      \draw[function,domain={-1.7:1.7}] plot(\x,{.5*(\x*\x)-.3});
    \end{tikzpicture}
    \caption{Parabolic motion}
  \end{subfigure}
  \begin{subfigure}{.4\linewidth}
    \centering
    \begin{tikzpicture}
      \draw[axes] (-2,0)--(2,0) node[right]{$x$};
      \draw[axes] (0,-2)--(0,2) node[above]{$y$};
      \draw[domain=45:1000,samples=500,function]
      plot (\x:{1.75*exp(-0.0025*\x)});
    \end{tikzpicture}
    \caption{Inward/Outward spiral}
  \end{subfigure}
  \caption{Examples of curvilinear motion in two dimensions}
  \label{fig:curvilinear-motions}
\end{figure}

%Like the Cartesian system, the polar coordinate system is also right-handed;
%TML%basics vectors $\hat{\bm r}$ (``radial direction'') and $\hat{\bm\theta}$
%TML%(``angular direction'') point in the directions shown in
%TML%Fig.~\ref{fig:basis-vecs},
%TML%\begin{figure}[ht]
%TML%  \centering
%TML%  \begin{tikzpicture}[scale=.75]
%TML%    \draw[axes] (-3,0)--(3,0) node[right]{$x$};
%TML%    \draw[axes] (0,-3)--(0,3) node[above]{$y$};
%TML%    \draw[vector] (0,0)--(1,0) node[below]{$\iii$};
%TML%    \draw[vector] (0,0)--(0,1) node[left] {$\jjj$};
%TML%    \draw circle (2.5);
%TML%    \begin{scope}[rotate=38]
%TML%      \draw[vector] (0,0)--(2.45,0) node[midway,above]{$r$};
%TML%      \draw[vector] (2.5,0)--(3.5,0) node[right]{$\hat r$};
%TML%      \draw[vector] (2.5,0)--(2.5,1) node[above]{$\hat\theta$};
%TML%      \draw[mass] (2.5,0) circle (.1);
%TML%    \end{scope}
%TML%    \draw[axes] (1.5,0) arc (0:38:1.5) node[pos=.55,right]{$\theta$};
%TML%  \end{tikzpicture}
%TML%  \caption{Basis vectors for rectilinear and curvilinear motions}
%TML%  \label{fig:basis-vecs}
%TML%\end{figure}
  %TML%and rotate as the object moves.

%TML%%\section{Cylindrical Coordinates in 3D}
%TML%%
%TML%%One way to extend the coordinates coordinate system into 3D is the
%TML%%\textbf{cylindrical coordinate system}. Note that the discussions for this
%TML%%topic focuses on $xy$ plane. Since the $z$-axis is linearly independent of
%TML%%the $xy$ plane, motion along that direction is independent.
%TML%%
%TML%%\begin{figure}[ht]
%TML%%  \centering
%TML%%  \begin{tikzpicture}[scale=.75]
%TML%%    \draw[axes] (0,0)--(-2.5,-2.5) node[below]{$x$};
%TML%%    \draw[axes] (0,0)--(5,0) node[right]{$y$};
%TML%%    \draw[axes] (0,0)--(0,5) node[above]{$z$};
%TML%%    \draw[axes] (-1,-1) arc (-110:-45:2) node[midway,below]{$\theta$};
%TML%%    \draw[dashed,fill=green!40,opacity=.4](0,0)--(3,-1.5)
%TML%%    node[pos=.6,below left,opacity=1]{$r$}--(3,2.5)
%TML%%    node[midway,right,black,opacity=1]{$z$}--(0,4);
%TML%%    \fill (3,2.5) circle(.1) node[right]{$\bm r(r,\theta,z)$};
%TML%%  \end{tikzpicture}
%TML%%\end{figure}
%TML%
%TML%
\section{Kinematics of Circular Motion}
The kinematics of circular motion is similar to that of the one-dimensional
kinematics. There is a positive direction, and a negative direction. From the
context of the $xy$-plane, the positive direction is counterclockwise---as it
is standard in mathematics---and the negative direction is clockwise. The
origin of the coordinate system is where the circular path intersects the
$x$-axis. However, unlike in rectilinear 1D motion, objects moving in one
direction along a circular path will return to the origin.

\subsection{Angular Position}

\textbf{Angular position} $\theta(t)$ is the location of an object along the
circular path at a distance $r$ from the origin. The angle is measured in
\emph{radians}. In the context of the Cartesian $xy$-plane, $\theta$ measured
from the $+x$-axis. This angle is positive if it is measured counter-clockwise
from the axis, and negative if measured clockwise. In the example shown in
Fig.~\ref{fig:angular-position}, the angular position of object A is positive,
while the angular position of object B is negative. If an object is in circular
motion, angular position is a continuous function of time. (At this time, we
are not concerned with what kind of function this is; it only means that the
object can be at one position at any given time.)
\begin{figure}[ht]
  \centering
  \begin{tikzpicture}[scale=.7]
    \draw[axes] (-3,0)--(3,0) node[right]{$x$};
    \draw[axes] (0,-3)--(0,3) node[above]{$y$};
    \draw[gray] circle (2.5);
    \begin{scope}[rotate=138]
      %\draw[vector] (0,0)--(2.44,0) node[midway,above]{$\bm r_A$};
      \draw[mass] (2.5,0) circle (.08) node[above]{$A$};
    \end{scope}
    \draw[vector] (2.5,0) arc (0:138:2.5) node[midway,above]{$\theta_A>0$};
    \begin{scope}[rotate=-65]
      %\draw[vector] (0,0)--(2.44,0) node[midway,right]{$\bm r_A$};
      \draw[mass] (2.5,0) circle (.08) node[above]{$B$};
    \end{scope}
    \draw[vector] (2.5,0) arc (0:-65:2.5) node[midway,right]{$\theta_B<0$};
  \end{tikzpicture}
  \caption{Angular position of two objects along the $xy$-plane.}
  \label{fig:angular-position}
\end{figure}

For the remainder of the chapter, the angular position of an object is not
particular important, because
\begin{itemize}
\item The $x$-axis can be defined arbitrarily
\item There are multiple ways
\end{itemize}
Instead we will focus on the other related kinematic quantities in circular
motion: angular displacement, angular velocity and angular acceleration.


\subsection{Angular Displacement}
If the object is moving along this circular path, then the change in the
angular position is the object's \textbf{angular displacement}
$\Delta\theta(t)$. It is defined as the difference between an object's current
angular position $\theta(t)$ and its initial angular position $\theta_i$:
%For constant distance $r$ to the origin, the \textbf{angular position}
%$\theta$ determines an object's position as a continuous function of
%time, i.e.:
%%\footnote{The more mathematically rigorous notation is to express the
%%angular position as a vector along the angular direction:
%%\begin{displaymath}
%%  \bm\theta=\theta(t)\hat{\bm\theta}
%%\end{displaymath}
%%The magnitude is $\theta(t)$ and the direction is $\hat{\bm\theta}$.}:
%\begin{equation}
%  \boxed{
%    \theta=\theta(t)
%  }
%  \label{eq:angular-position}
%\end{equation}
%The unit for angular position is a \emph{radian} (rad): If motion is confined
%to the $xy$-plane, then $\theta$ is the standard angle: $\theta$ is positive
%when measured counterclockwise from the $x$-axis, and negative when it is
%measured clockwise.
%
%When an object moves along this circular path, we can calculate the change in
%the angular position (i.e.\ change in the angle). This is the object's
%\textbf{angular displacement} $\Delta\theta$:
\begin{equation}
  \boxed{
    \Delta\theta(t)=\theta(t)-\theta_i
  }
  \label{eq:angular-displacement}
\end{equation}
Like angular position, $\Delta\theta$ is positive if the object moves
counterclockwise, and negative if it moves clockwise, as shown in
Fig.~\ref{fig:angular-displacement}. Angular displacement is also measured in
\emph{radians}.

%At this point, it should be obvious that angular position and
%angular displacemnet are analogous to the equations one-dimensional position
%and displacement presented in Chapter~\ref{chapter:kinematics}.
\begin{figure}[ht]
  \centering
  \begin{tikzpicture}[scale=.7]
    \draw[axes] (-3,0)--(3,0) node[right]{$x$};
    \draw[axes] (0,-3)--(0,3) node[above]{$y$};
    \draw circle (2.5);

    \draw[axes,rotate=35] (2.5,0) arc (0:95:2.5)
    node[midway,right=5]{$\Delta\theta>0$};
    \draw[fill=lightgray,rotate=35] (2.5,0) circle (.1);
    \draw[mass,rotate=130] (2.5,0) circle (.1);

    \draw[axes,rotate=10] (2.5,0) arc (0:-70:2.5)
    node[midway,right]{$\Delta\theta<0$};
    \draw[fill=lightgray,rotate=10] (2.5,0) circle (.1);
    \draw[mass,rotate=-60] (2.5,0) circle (.1);
  \end{tikzpicture}
  \caption{Positive and negative angular displacement}
  \label{fig:angular-displacement}
\end{figure}

\subsection{Angular Velocity}
Analogous to the relationship between position and velocity in rectilinear
motions, for circular motion, \textbf{angular velocity} (or
\textbf{angular frequency}) $\omega(t)$ is the rate of change of angular
position, i.e.\ how quickly angular position changes with time. We can define
the \textbf{average angular velocity} as the angular displacement over a
finite time interval:
%$\omega_\text{avg}(t)$, is the time derivative of the angular position,
%measured in \emph{radian per second} (\si{\radian\per\second}). It is also a
%function of time:
\begin{equation}
  \boxed{
    \omega_\text{avg}(t)=\frac{\Delta\theta}{\Delta t}
    =\frac{\theta(t)-\theta_i}{t-t_0}
  }
\end{equation}
When the time interval approaches zero ($\Delta t\rightarrow 0$), we have the
\textbf{instantaneous angular velocity} $\omega(t)$ at time $t$. Both average
and instantaneous velocities are continous functions of time.

Again, if motion is confined to the $xy$-plane, then $\omega$ is positive when
the object moves in the counterclockwise direction, and negative when it moves
clockwise (Fig.~\ref{fig:omega-plus-minus}).
\begin{figure}[ht]
  \centering
  \begin{tikzpicture}[scale=.75]
    \draw[axes] (-3,0)--(3,0) node[right]{$x$};
    \draw[axes] (0,-3)--(0,3) node[above]{$y$};
    \draw[axes] (1,0) arc (0:38:1) node[midway,right]{$\theta$};
    \draw circle (2.5);
    \begin{scope}[rotate=38]
      \draw[thick] (0,0)--(2.44,0) node[midway,above]{$r$};
      \draw[vector] (2.5,.08)--(2.5,1.5) node[above]{$\bm v$};
      \draw[mass] (2.5,0) circle (.1);
    \end{scope}
  \end{tikzpicture}
  \caption{Sign convention for $\omega$ and direction of velocity vector
    $\bm v$ when circular motion is confined to the $xy$-plane}
  \label{fig:omega-plus-minus}
\end{figure}



\subsection{Velocity and Speed}
As the object moves with angular speed $\omega$, the actual velocity vector
$\bm v$ is tangent to the circle. Obviously
$\bm v$ is not constant in time, as its direction is always changing.
However, there is a simple mathematical relationship between the speed of object
$v(t)=|\bm v(t)|$ and the angular speed $|\omega|$:
\begin{equation}
  \boxed{
    v=r|\omega|
  }
\end{equation}
\begin{remark}
  For your information, the velocity vector $\bm v$ along \emph{any} path
  is \emph{always} tangent to the path. This should be obvious when you
  consider a car being driven on a highway.
\end{remark}

%\begin{itemize}
%\item The direction of $\bm v$ is tangent to circle, along
%  $\hat\theta$, and therefore $\perp$ to $\hat r$
%\item If $\omega>0$, the motion is counter-clockwise
%\item If $\omega<0$, the motion is clockwise
%\end{itemize}
  
%    \begin{tikzpicture}[scale=.75]
%      \draw[axes] (-3,0)--(3,0) node[right]{$x$};
%      \draw[axes] (0,-3)--(0,3) node[above]{$y$};
%      \draw[axes] (1,0) arc (0:38:1) node[midway,right]{$\theta$};
%      \draw circle (2.5);
%      \begin{scope}[rotate=38]
%        \draw[vector] (0,0)--(2.44,0) node[midway,above]{$\bm r$};
%        \draw[vector] (2.5,.08)--(2.5,1.5) node[above]{$\bm v$};
%        \draw[mass] (2.5,0) circle (.1);
%      \end{scope}
%    \end{tikzpicture}

%The velocity of the object in circular motion is more properly related to
%the angular velocity using this vector cross product:
%\begin{equation}
%  \bm v=\bm\omega\times\bm r
%\end{equation}
%$\bm\omega$ is out of the page if motion is counterclockwise, and into the page
%if motion is clockwise. Visualizing $\bm\omega$ takes practice, but this vector
%notation is mathematically rigorous and consistent
%  
%
%
%
%
%%\begin{frame}{Relativity Velocity}
%%  If two points $A$ and $B$ are rotating with the same angular velocity with the
%%  same cent, their relative position is given by:
%%
%%  \begin{equation}
%%    \boxed{
%%      \bm V_B=\bm V_A+ \bm\omega\times\bm r_{BA}
%%    }
%%  }
%%
%%  Where $\bm r_{BA}$ is the position of $B$ relative to $A$.


\subsection{Angular Acceleration and Tangential Acceleration}
Analogous to the relation between velocity $\bm v$ and acceleration
$\bm a$, \textbf{angular acceleration} $\alpha$ is the rate of change of
angular velocity, i.e.\ how quickly angular velocity changes with time:
\begin{equation}
  \boxed{
    \alpha_\text{avg}(t)=\frac{\Delta\omega(t)}{\Delta t}
    =\frac{\omega(t)-\omega_i}{t-t_0}
  }
\end{equation}
The unit for angular acceleration is \emph{radian per second squared}
\si{\radian\per\second\squared}.
%The sign convention for $\alpha$ is the same
%as for $\theta$ and $\omega$.
Similar to the relationship between velocity and angular velocity,
\textbf{tangential acceleration} $a_t$ along the direction of motion is related
to angular acceleration $\alpha$ by the radius $r$:
\begin{equation}
  \boxed{
    |\bm a_t(t)|=r|\alpha|
  }
\end{equation}
For \emph{uniform} circular motion, $\omega$ is constant, and therefore
$\bm a_t=0$.

%By the fundamental theorem of calculus, we can of course integrate angular
%acceleration to find the angular velocity (or the \emph{change} in angular
%velocity) as a function of time:
%\begin{equation}
%  \boxed{
%    \omega(t)=\int\alpha(t)\dl t+\omega_0
%  }
%  \quad\quad
%  \boxed{
%    \Delta\omega(t)=\int_{t_0}^t\alpha(t)\dl t
%  }
%\end{equation}
%The relationships are the same as in rectilinear motion.
%
%
%
\subsection{Period \& Frequency}
%%    \begin{tikzpicture}[scale=.75]
%%      \draw[axes] (-3,0)--(3,0) node[right]{$x$};
%%      \draw[axes] (0,-3)--(0,3) node[above]{$y$};
%%      \draw[axes] (1,0) arc (0:38:1) node[midway,right]{$\theta$};
%%      \draw circle (2.5);
%%      \begin{scope}[rotate=38]
%%        \draw[vector] (0,0)--(2.44,0) node[midway,above]{$\bm r$};
%%        \draw[vector] (2.5,.08)--(2.5,1.5) node[above]{$\bm v$};
%%        \draw[mass] (2.5,0) circle (.1);
%%      \end{scope}
%%    \end{tikzpicture}
For constant angular velocity $\omega$ (uniform circular motion), the motion
is strictly periodic. Its \textbf{frequency} and \textbf{period} are given by:
\begin{equation}
  \boxed{ f=\frac\omega{2\pi} }\quad\quad
  \boxed{ T=\frac{2\pi}\omega}\quad\quad
  \boxed{ f=\frac1T}
\end{equation}
Period $T$ is measured in \emph{seconds} (\si\second) and frequency $f$ is
measured in \textbf{hertz} (\si\hertz).
  



\subsection{Kinematic Equations for Circular Motion}
For \emph{constant} angular acceleration $\alpha$, the kinematic equations are
the same as in rectilinear motion, but with $\theta$ replacing $x$, $\omega$
replacing $v$, and $\alpha$ replacing $a$:
\begin{align}
  \Delta\theta &= \omega_i t + \frac12\alpha t^2\\
  \Delta\theta &= \omega_f t + \frac12\alpha t^2\\
  \Delta\theta &=\frac{\theta_i+\theta_f}2 \\
  \omega_f &=\omega_i + \alpha t\\
  \omega_f^2& = \omega_0^2+ 2\alpha\Delta\theta
\end{align}
For non-constant $\alpha$, calculus will be required.




%  \textbf{Example:} An object moves in a circle with angular acceleration
%  \SI{3.0}{\radian\per\second\squared}. The radius is \SI{2.0}{\metre} and it
%  starts from rest. How long does it take for this object to finish a circle?



\subsection{Centripetal Acceleration}% \& Centripetal Force}
Even when there is no angular acceleration, there is also a component of
acceleration towards the centre of the motion,
%\begin{figure}[ht]
%  \centering
%  \begin{tikzpicture}
%    \draw[->](-3,0)--(3,0);
%    \draw[->](0,-3)--(0,3);
%    \draw circle (2.5);
%    \begin{scope}[rotate=30]
%      \draw[->,very thick,blue](2.5,0)--(2.5,1.5) node[above]{$\bm v_i$};
%      \draw[->,very thick,red] (0,0)--(2.5,0)node[pos=.6,below]{$\bm r_i$};
%      \fill (2.5,0) circle(.06);
%    \end{scope}
%    \begin{scope}[rotate=90]
%      \draw[->,very thick,blue] (2.5,0)--(2.5,1.5)node[left]{$\bm v_f$};
%      \draw[->,very thick,red] (0,0)--(2.5,0) node[midway,left]{$\bm r_f$};
%TML%%      \fill (2.5,0) circle(.06);
%TML%%    \end{scope}
%TML%%    \draw(0,1)[<->] arc(90:30:1) node[pos=.6,above]{$\Delta\theta$};
%TML%%  \end{tikzpicture}
%TML%%\end{figure}
%TML%called the \textbf{centripetal acceleration} $\bm a_c$.
%TML%%\begin{equation}
%TML%%  \boxed{
%TML%%    \bm a_c=-\frac{v^2}r\hat{\bm r}=-(\omega^2r)\hat{\bm r}
%TML%%  }
%TML%%\end{equation}
%TML%%The negative sign indicates that the direction of $\bm a_c$ %and $\bm F_c$ are
%TML%%is radially \emph{inward}, towards the centre of motion, as $\hat{\bm r}$ is
%TML%%the outward radial direction. In uniform circular motion ($\alpha=0$), where
%TML%%the period or frequency are known, the speed of the object is:
%TML%%\begin{equation}
%TML%%  v=\omega r = 2\pi rf = \frac{2\pi r}T
%TML%%\end{equation}
%TML%%Centripetal acceleration can therefore be expressed based on $T$ or $f$:
%TML%%\begin{equation}
%TML%%  \bm a_c=-(\omega^2r)\hat{\bm r}\quad\rightarrow\quad
%TML%%  \boxed{
%TML%%    \bm a_c=-\frac{4\pi^2r}{T^2}\hat{\bm r}=-4\pi^2rf^2\hat{\bm r}
%TML%%  }
%TML%%\end{equation}
%TML%
%TML%Consider an object in uniform circular motion in the counterclockwise
%TML%direction with constant radius $r$ and constant speed $v$ (i.e.\ constant
%TML%angular speed $\omega=v/r$), as shown in
%TML%Fig.~\ref{fig:v-in-uniform-circ-motion}. At initial time $t_0$, the position
%TML%and velocity of the object are given by $\bm r_0=\bm r(t_0)$ and
%TML%$\bm v_0=\bm v(t_0)$. Then, at a later time $t_1=t_0+\Delta t$, the object has
%TML%moved through an angular displacement of $\Delta\theta$, and the final
%TML%position and velocity are now $\bm r_1=\bm r(t_1)$ and $\bm v_1=\bm v(t_1)$.
%TML%\begin{figure}[ht]
%TML%  \centering
%TML%  \begin{tikzpicture}
%TML%    \draw[axes] (-3,0)--(3,0);
%TML%    \draw[axes] (0,-3)--(0,3);
%TML%    \draw circle (2.5);
%TML%    \begin{scope}[rotate=30]
%TML%      \draw[vector,blue] (2.5,0)--(2.5,1.5) node[above]{$\bm v_0$};
%TML%      \draw[vector,red] (0,0)--(2.5,0) node[pos=.6,below]{$\bm r_0$};
%TML%      \fill (2.5,0) circle (.06);
%TML%    \end{scope}
%TML%    \begin{scope}[rotate=90]
%TML%      \draw[vector,blue] (2.5,0)--(2.5,1.5) node[left]{$\bm v_1$};
%TML%      \draw[vector,red] (0,0)--(2.5,0) node[midway,left]{$\bm r_1$};
%TML%      \fill (2.5,0) circle (.06);
%TML%    \end{scope}
%TML%    \draw[<-,thick] (0,1) arc (90:30:1) node[pos=.6,above]{$\Delta\theta$};
%TML%  \end{tikzpicture}
%TML%  \caption{An object in counter-clockwise uniform circular motion}
%TML%  \label{fig:v-in-uniform-circ-motion}
%TML%\end{figure}
%TML%
%TML%From the definition of acceleration,
%TML%\begin{equation}
%TML%  \bm a_c=\frac{\Delta\bm v}{\Delta t}=\frac{\bm v_1-\bm v_0}{\Delta t}
%TML%\end{equation}
%TML%And the magnitude of the centripetal acceleration is
%TML%\begin{equation}
%TML%  |\bm a_c|=\frac{|\Delta\bm v|}{\Delta t}
%TML%\end{equation}
%TML%Since $|\bm r_0|=|\bm r_i|=r$ (circular motion), and $|\bm v_0|=|\bm v_1|=v$
%TML%(constant speed, uniform circular motion), the triangles formed by the
%TML%displacement vector $\Delta\bm r$ and the change in velocity $\Delta\bm v$ are
%TML%similar isosceles triangles, as shown in Fig.~\ref{fig:sim-triangles}.
%TML%\begin{figure}[ht]
%TML%  \centering
%TML%  \begin{subfigure}{.4\linewidth}
%TML%    \centering
%TML%    \begin{tikzpicture}[scale=1.5,vector]
%TML%      \draw[rotate=-60,red] (0,0)--(0,2) node[midway,below]{$\bm r_0$};
%TML%      \draw[red] (0,0)--(0,2) node[midway,left]{$\bm r_1$};
%TML%      \draw (2*sin{60},1)--(0,2) node[midway,above]{$\Delta\bm r$};
%TML%      \draw[<-,thick] (0,.8) arc (90:30:.8) node[midway,above]{$\Delta\theta$};
%TML%    \end{tikzpicture}
%TML%    \caption{Displacement}
%TML%  \end{subfigure}
%TML%  \begin{subfigure}{.4\linewidth}
%TML%    \centering
%TML%    \begin{tikzpicture}[scale=1.5,vector]
%TML%      \draw[rotate=-60,blue] (0,0)--(-2,0) node[midway,right]{$\bm v_0$};
%TML%      \draw[blue](0,0)--(-2,0) node[midway,below]{$\bm v_1$};
%TML%      \draw(-1,2*sin{60})--(-2,0) node[midway,left]{$\Delta\bm v$};
%TML%    \end{tikzpicture}
%TML%    \caption{Change in velocity}
%TML%  \end{subfigure}
%TML%  \caption{Vector diagrams for change in position and velocity are similar
%TML%    isosceles triangles}
%TML%  \label{fig:sim-triangles}
%TML%\end{figure}
%TML%
%TML%
%TML%%\begin{tikzpicture}
%TML%%  \begin{scope}[very thick,->]
%TML%%    \draw[rotate=-60,red](0,0)--(0,2) node[midway,below]{$r$};
%TML%%    \draw[red](0,0)--(0,2) node[midway,left]{$r$};
%TML%%    \draw (2*sin{60},1)--(0,2)node[midway,above]{$|\Delta\bm r|$};
%TML%%  \end{scope}
%TML%%\end{tikzpicture}
%TML%%\begin{tikzpicture}
%TML%%  \begin{scope}[very thick,->]
%TML%%    \draw[rotate=-60,blue](0,0)--(-2,0) node[midway,right]{$v$};
%TML%%    \draw[blue](0,0)--(-2,0) node[midway,below]{$v$};
%TML%%    \draw(-1,2*sin{60})--(-2,0) node[midway,left]{$|\Delta\bm v|$};
%TML%%  \end{scope}
%TML%%\end{tikzpicture}
%TML%
%TML%That the triangles are similar means that:
%TML%\begin{equation}
%TML%  \frac{|\Delta\bm r|}r=\frac{|\Delta\bm v|}v
%TML%  \quad\rightarrow\quad
%TML%  |\Delta\bm v|=\frac vr|\Delta\bm r|
%TML%\end{equation}
%TML%The magnitude of the centripetal acceleration ($|\bm a_c|$):
%TML%\begin{equation}
%TML%  |\bm a_c|=\frac{|\Delta\bm v|}{\Delta t}
%TML%  =\frac vr\left[\frac{|\Delta\bm r|}{\Delta t}\right]=\frac{v^2}r
%TML%\end{equation}
%TML%The direction of the centripetal acceleration is easy to show using basic
%TML%geometry. When $\Delta t\rightarrow 0$, $\Delta\theta\rightarrow 0$.
%TML%
%TML%Since
%TML%\begin{equation}
%TML%  2\alpha+\Delta\theta=\ang{180}
%TML%\end{equation}
%TML%when $\Delta\theta\rightarrow 0$, $\alpha\rightarrow\ang{90}$.
%TML%\begin{figure}[ht]
%TML%  \centering
%TML%  \begin{tikzpicture}
%TML%    \begin{scope}[vector]
%TML%      \draw[blue] (0,0)--(0,4) node[midway,right]{$\bm v_i$};
%TML%      \draw[blue,rotate=30] (0,0)--(0,4) node[midway,left]{$\bm v_f$};
%TML%      \draw(0,4)--(-4*sin{30},4*cos{30}) node[midway,above]{$\Delta\bm v$};
%TML%    \end{scope}
%TML%    \draw[thick] (0,1) arc (90:90+30:1) node[midway,above]{$\Delta\theta$};
%TML%    \draw[thick] (0,3.5) arc (270:195:.5)node[midway,below]{$\alpha$};
%TML%  \end{tikzpicture}
%TML%\end{figure}
%TML%The direction of $\Delta\bm v$ is perpendicular to $\bm v$, Since centripetal
%TML%acceleration is in the same direction as $\bm v$, $\bm a_c$ points towards the
%TML%centre of the circular path (i.e.\ the inwards radial direction
%TML%$-\hat{\bm r}$), giving us the equation for centripetal acceleration, which can
%TML%be expressed using the speed $v$ or the angular speed $\omega=v/r$ of the
%TML%object:
\begin{equation}
  \boxed{
    \bm a_c=-\frac{v^2}r\hat{\bm r}=-\omega^2r\hat{\bm r}
  }
  \label{eq:centripetal-acceleration}
\end{equation}



\subsection{Acceleration in General Circular Motion}

To summarize, in general circular motion, there are two components of
acceleration, as shown in Fig.~\ref{fig:circular-motion-accelerations}:
\begin{figure}[ht]
  \centering
  \begin{tikzpicture}[scale=4]
    \draw[dashed] (.866,-.5) arc (-30:30:1);
    \draw[vector,magenta] (1,0)--(.5,0) node[midway,below]{$a_c=\omega^2r$};
    \draw[vector,cyan] (1,0)--(1,.3) node[right]{$a_t=r\alpha$};
    \draw[vector] (1,0)--(.5,.3) node[left]{$\bm a$};
    \fill (1,0) circle (.02);
  \end{tikzpicture}
  \caption{The two component of acceleration in general circular motion.}
  \label{fig:circular-motion-accelerations}
\end{figure}
\begin{itemize}
\item The object's centripetal acceleration $\bm a_c$ depends on radius of
  curvature $r$ and angular speed $\omega$ (or instantaneous speed
  $v=r|\omega|$). The direction of the acceleration is towards the centre of
  the circular path. Centripletal acceleration is \emph{never} zero if the
  object remains in circular motion.
\item The object's tangential acceleration $\bm a_t$ depends on radius $r$
  and angular acceleration $\alpha$. The direction of the acceleration is
  tangent to the circle, along the direction of motion. Tangential acceleration
  is zero if the object is in uniform motion.
\end{itemize}
The total acceleration $\bm a$ is the vector sum of both components:
\begin{equation}
  \bm a = \bm a_c+\bm a_t
\end{equation}



\section{Dynamics of Circular Motion}
By the second law of motion ($\bm F_\text{net}=m\bm a$), acceleration must
be caused by a net force along that direction. Along the direction of motion,
tangential acceleration is caused by the \textbf{tangential force}
$\bm F_t$:
\begin{equation}
  \boxed{
    F_t=ma_t=mr\alpha %\hat{\bm\theta}
  }
\end{equation}
while the centripetal acceleration---directed towards the centre of motion---is
caused by the \textbf{centripetal force}:
\begin{equation}
  \boxed{
    F_c
    =ma_c
    =\frac{mv^2}r%\hat{\bm r}
    =m\omega^2r %\right)$\hat{\bm r}
  }
\end{equation}
Just as the total acceleration is the vector sum of tangential and centripetal
accelerations, the net force is also a vector sum of the two components of
force:
\begin{equation}
  \boxed{
    \bm F_\text{net}=\bm F_t + \bm F_c
  }
\end{equation}
The forces that generate the centripetal and tangential forces comes from the
free-body diagram, and may include (but not limited to) the common forces
discussed in Chapter 2:
\begin{itemize}
\item Gravity $\bm F_g$
\item Static friction ($\bm f_s$) or kinetic friction ($\bm f_k$)
\item Normal force ($\bm F_n$)
\item Tension ($\bm F_T$)
\item Spring force ($\bm F_e$)
\end{itemize}




%TML%%\begin{frame}{Example: Horizontal Motion}
%TML%%  
%TML%%    \column{.4\textwidth}
%TML%%    \pic1{puck-on-table}
%TML%%    
%TML%%    \column{.6\textwidth}
%TML%%    \textbf{Example:} In the figure on the left, a mass $m_1$ is rolling around
%TML%%    a frictionless table with radius $R$ with a speed $v$. What is the mass of
%TML%%    $m_2$?
%TML%%  
%TML%%
%TML%%
%TML%%
%TML%%\begin{frame}{Banked Curves on Highways and Racetracks}
%TML%%  
%TML%%    \column{.35\textwidth}
%TML%%    \centering
%TML%%    \pic{.8}{banked-turn-acceleration}
%TML%%    
%TML%%    \begin{tikzpicture}[vector]
%TML%%      \fill circle (.08);
%TML%%      \draw[rotate=-30] (0,0)--(0,1) node[above]{$\bm N$};
%TML%%      \draw (0,0)--(0,-1) node[below]{$\bm F_g$};
%TML%%      \draw[rotate=60] (0,0)--(0,-1) node[right]{$\bm f$};
%TML%%    \end{tikzpicture}
%TML%%    \begin{tikzpicture}[axes]
%TML%%      \draw (0,0)--(1,0) node[right]{$x$};
%TML%%      \draw (0,0)--(0,1) node[above]{$y$};
%TML%%    \end{tikzpicture}
%TML%%make
%TML%%    \column{.65\textwidth}
%TML%%    No motion in the $y$ direction, i.e.\ no net force:
%TML%%
%TML%%    \begin{equation}
%TML%%      \sum F_y=N\cos\theta-f\sin\theta-F_g=0
%TML%%    }
%TML%%
%TML%%    Net force in the $x$ direction is the centripetal force:
%TML%%
%TML%%    \begin{equation}
%TML%%      \sum F_x=N\sin\theta +f\cos\theta = \frac{mv^2}r
%TML%%    }
%TML%%
%TML%%    Friction force $\bm f$ may be static or kinetic.
%TML%%  
%TML%%
%TML%%
%TML%%
%TML%%
%TML%%\begin{frame}{Banked Curves on Highways and Racetracks}
%TML%%  
%TML%%    \column{.35\textwidth}
%TML%%    \centering
%TML%%    \pic{.8}{banked-turn-acceleration}
%TML%%    
%TML%%    \begin{tikzpicture}[vector]
%TML%%      \fill circle (.08);
%TML%%      \draw[rotate=-30] (0,0)--(0,1)node[above]{$\bm N$};
%TML%%      \draw (0,0)--(0,-1)node[below]{$\bm F_g$};
%TML%%      \draw[rotate=60] (0,0)--(0,-1)node[right]{$\bm f$};
%TML%%    \end{tikzpicture}
%TML%%    \begin{tikzpicture}[axes]
%TML%%      \draw (0,0)--(1,0) node[right]{$x$};
%TML%%      \draw (0,0)--(0,1) node[above]{$y$};
%TML%%    \end{tikzpicture}
%TML%%
%TML%%    \column{.65\textwidth}
%TML%%    For analysis, use the simplified equation for friction $f=\mu N$ (i.e.\
%TML%%    assume either kinetic friction or maximum static friction), and weight
%TML%%    $F_g=mg$, the equations on the previous slides can be arranged as:
%TML%%
%TML%%    \vspace{-.3in}{\large
%TML%%      \begin{align*}
%TML%%        N\left(\cos\theta-\mu\sin\theta\right) &=mg\\
%TML%%        N\left(\sin\theta+\mu\cos\theta\right) &=\frac{mv^2}r
%TML%%      \end{align*}
%TML%%    }
%TML%%  
%TML%%
%TML%%
%TML%%
%TML%%\begin{frame}{Banked Curves on Highways and Racetracks}
%TML%%  
%TML%%    \column{.35\textwidth}
%TML%%    \centering
%TML%%    \pic{.8}{banked-turn-acceleration}\\
%TML%%    \begin{tikzpicture}[vector]
%TML%%      \fill circle(.08);
%TML%%      \draw[rotate=-30] (0,0)--(0,1) node[above]{$\bm N$};
%TML%%      \draw (0,0)--(0,-1) node[below]{$\bm F_g$};
%TML%%      \draw[rotate=60] (0,0)--(0,-1) node[right]{$\bm f$};
%TML%%    \end{tikzpicture}
%TML%%    \begin{tikzpicture}[axes]
%TML%%      \draw (0,0)--(1,0) node[right]{$x$};
%TML%%      \draw (0,0)--(0,1) node[above]{$y$};
%TML%%    \end{tikzpicture}
%TML%%
%TML%%    \column{.65\textwidth}
%TML%%    Dividing the two equations removes both the normal force and mass terms:
%TML%%
%TML%%    \begin{equation}
%TML%%      \frac{\sin\theta+\mu\cos\theta}{\cos\theta-\mu\sin\theta}
%TML%%      =\frac{v^2}{rg}
%TML%%    }
%TML%%
%TML%%    The \emph{maximum} velocity $v_\text{max}$ can be expressed as:
%TML%%
%TML%%    \begin{equation}
%TML%%      \boxed{v_\text{max}=
%TML%%        \sqrt{rg\frac{\sin\theta+\mu\cos\theta}{\cos\theta-\mu\sin\theta}}
%TML%%      }
%TML%%    }
%TML%%
%TML%%    Note that $v_\text{max}$ does not depend on mass.
%TML%%  
%TML%%
%TML%%
%TML%%
%TML%%
%TML%%\begin{frame}{Banked Curves on Highways and Racetracks}
%TML%%  In the limit of $\mu=0$ (frictionless case), the equation reduces to:
%TML%%
%TML%%  \begin{equation}
%TML%%    \boxed{ v_\text{max}=\sqrt{rg\tan\theta} }
%TML%%  }
%TML%%
%TML%%  And in the limit of a flat roadway with no banking ($\theta=0$,
%TML%%  $\sin\theta=0$ and $\cos\theta=1$), the equation reduces to:
%TML%%
%TML%%  \begin{equation}
%TML%%    \boxed{
%TML%%      v_{\text{max}}=\sqrt{\mu rg}
%TML%%    }
%TML%%  }
%TML%%
%TML%%
%TML%%
%TML%%
%TML%%
%TML%%%
%TML%%%
%TML%%%\begin{frame}{Another Example: Exit Ramp}
%TML%%%  \textbf{Example:} A car exits a highway on a ramp that is banked at
%TML%%%  \ang{15} to the horizontal. The exit ramp has a radius of curvature of
%TML%%%  \SI{65}{\metre}. If the conditions are extremely icy and the driver cannot
%TML%%%  depend on any friction to help make the turn, at what speed should the driver
%TML%%  travel so that the car will not skid off the ramp? What if there is friction?



\section{Vertical Circles}

Circular motion with a horizontal path is straightforward. However, for
vertical motion, it is generally difficult to solve by dynamics and kinematics.
Instead, use conservation of energy may be used to solve for $\bm v$.
Afterwards, we can use the equation for centripetal force to find other forces.
%\textbf{Remember:} If it is impossible to get the required centripetal
%force, then it could not continue the circular motion




\subsection{Simple Pendulum}
A simple pendulum, shown in Fig.~\ref{fig:simple-pendulum-again}, is an example
of a circular motion problem. In Section~\ref{sec:simple-pendulum-energy}, we
have already established how energy is conserved in a simple pendulum system.
\begin{figure}[ht]
  \centering
  \begin{tikzpicture}[scale=.75]
    \fill[pattern=north east lines] (-1,0) rectangle (1,.2);
    \draw[very thick] (-1,0)--(1,0);
    \begin{scope}[rotate=20]
      \draw[thick] (0,0)--(0,-5);
      \shade[ball color=red] (0,-5) circle (.2) node[below right]{$m$};
      \begin{scope}[vector,red]
        \draw (0,-5)--(0,-3.3) node[left]{$\bm F_T$};
        \draw[rotate around={-20:(0,-5)}] (0,-5)--(0,-6.5)
        node[below]{$\bm F_g$};
      \end{scope}
    \end{scope}
    \draw[dashed] (0,0)--(0,-5);
    \draw[dashed] (3.54,-3.54) arc (315:225:5);
  \end{tikzpicture}
  \caption{A simple pendulum is a vertical circular motion problem}
  \label{fig:simple-pendulum-again}
\end{figure}

In this system, which is defined as the pendulum bob and Earth, there are two
forces acting on the pendulum: weight $\bm F_g$, and tension $\bm F_T$. As the
pendulum swings, $\bm F_T$ is always perpendicular to motion, therefore it does
not do any mechanical work. The only work is done by $\bm F_g$ as the pendulum
changes height. Since gravity is an internal force, the total energy of the
system is constant, or:
\begin{equation*}
  \Delta U_g + \Delta K = 0
\end{equation*}
However, only using conservation of energy does not immediately allow us to
find the tension force on the pendulum, nor the acceleration of the pendulum
bob. We must therefore use equations for circular motion to find the forces:
%\item Speed of the pendulum at any height is found using conservation
%%  of energy
%%  \begin{itemize}
%%  \item 
%%  \item Work is done by gravity (a conservative force) alone
%%  \end{itemize}
%%\item Tangential and centripetal accelerations are based on the net force
%%  along the angular and radial directions
%%\end{itemize}

\textbf{At the top of the swing}, when the pendulum is deflected by an angle
$\theta$, shown in Fig.~\ref{fig:top-swing}, velocity $v$ is zero by
definition.
\begin{figure}[ht]
  \centering
  \begin{tikzpicture}[scale=.75]
    \fill[pattern=north east lines] (-1,0) rectangle (1,0.2);
    \draw[very thick] (-1,0)--(1,0);
    \begin{scope}[rotate=45]
      \draw[thick] (0,0)--(0,-5);
      \shade[ball color=red] (0,-5) circle (.2) node[right=2.5]{$m$};
      \begin{scope}[vector,red]
        \draw[dotted] (0,-5)--(-1.1,-5)node[left]{$F_g\sin\theta$};
        \draw[dotted] (0,-5)--(0,-6.1)
        node[right,fill=yellow!20]{$F_g\cos\theta$};
        \draw (0,-5)--(0,-3.9) node[left,fill=yellow!20]{$\bm F_T$};
        \draw[rotate around={-45:(0,-5)}] (0,-5)--(0,-6.5)
        node[below]{$\bm F_g$};
      \end{scope}
    \end{scope}
    \draw[dashed] (0,0)--(0,-5);
    \draw[dashed] (3.54,-3.54) arc (315:225:5);
    \draw[axes] (0,-2) arc (270:315:2) node[midway,below]{$\theta$};
  \end{tikzpicture}
  \caption{Forces at the top of the swing of a simple pendulum}
  \label{fig:top-swing}
\end{figure}
From Eq.~\ref{eq:centripetal-acceleration}, we know that centripetal
acceleration must also be zero:
\begin{equation}
  a_c=\frac{v^2}r=0
\end{equation}
Therefore the net force along the radial direction $\hat{\bm r}$ is zero. The
tension force $F_T$ can be calculated:
\begin{equation}
  F_T=F_g\cos\theta=mg\cos\theta
\end{equation}
%    \begin{tikzpicture}[scale=.75]
%      \fill[pattern=north east lines] (-1,0) rectangle (1,0.2);
%      \draw[very thick] (-1,0)--(1,0);
%      \begin{scope}[rotate=45]
%        \draw[thick] (0,0)--(0,-5);
%        \shade[ball color=red] (0,-5) circle(.2) node[right=2.5]{$m$};
%        \begin{scope}[vector,red]
%          \draw[dotted] (0,-5)--(-1.1,-5)
%          node[left,fill=cyan!10]{$F_g\sin\theta$};
%          \draw[dotted] (0,-5)--(0,-6.1) node[right]{$F_g\cos\theta$};
%          \draw (0,-5)--(0,-3.9) node[left]{$\bm F_T$};
%          \draw[rotate around={-45:(0,-5)}] (0,-5)--(0,-6.5)
%          node[below]{$\bm F_g$};
%        \end{scope}
%      \end{scope}
%      \draw[dashed] (0,0)--(0,-5);
%      \draw[dashed] (3.54,-3.54) arc (315:225:5);
%      \draw[axes] (0,-2) arc (270:315:2) node[midway,below]{$\theta$};
%    \end{tikzpicture}
In the tangential direction $\hat{\bm\theta}$, there is still a tangential
component of gravity: $F_t=F_g\sin\theta=mg\sin\theta$, therefore, there is a
tangential acceleration with a magnitude of:
\begin{equation}
  a_t=\frac{F_t}m=g\sin\theta
\end{equation}
This is the same acceleration as an object sliding down a frictionless ramp at
an angle of $\theta$.

\textbf{At the bottom of the swing}, where $\theta=0$, as shown in
Fig.~\ref{fig:bottom-swing}, the velocity is at its maximum value,
\begin{figure}[ht]
  \centering
  \begin{tikzpicture}[scale=.8]
    \fill[pattern=north east lines] (-1,0) rectangle (1,.2);
    \draw[very thick] (-1,0)--(1,0);
    \draw[thick] (0,0)--(0,-5);
    \shade[ball color=red] (0,-5) circle (0.2) node[below right]{$m$};
    \draw[vector,red] (0,-5)--(0,-2.5) node[right]{$\bm F_T$};
    \draw[vector,red] (0,-5)--(0,-6.5) node[below]{$\bm F_g$};
    \draw[dashed] (3.54,-3.54) arc (315:225:5);
  \end{tikzpicture}
  \caption{A simple pendulum at the bottom of its swing}
  \label{fig:bottom-swing}
\end{figure}
therefore centripetal acceleration is at maximum value because:
\begin{equation}
    a_c=\frac{v^2}r
\end{equation}
At the lowest point, tension is the highest:
\begin{equation}
  F_T=F_g+F_c=m\left(g+\frac{v^2}r\right)
\end{equation}
There is no tangential acceleration because there are forces in the angular
direction:
\begin{equation}
  a_t=0
\end{equation}  




\begin{example}
  You are playing with a yo-yo with a mass $M$; the length of the string is
  $R$. You decide to see how slowly you can swing it in a vertical circle
  while keeping the string fully extended, even when the yo-yo is at the top of
  its swing.
  \begin{enumerate}%[a.]
  \item Calculate the minimum speed at which you can swing the yo-yo while
    keeping it on a circular path.
  \item If the yo-yo is at its minimum speed at the top of its swing, find the
    tension in the string when the yo-yo is at the side and at the bottom.
  \end{enumerate}
\end{example}



%TML%%\begin{frame}{Example Problem: Vertical Motion}
%TML%%  %This is a very typical problem for vertical motion.
%TML%%  %To solve this problem, we
%TML%%  First, we draw free-body diagrams for each of the positions in the circle.
%TML%%  There are two forces acting on the yo-yo: gravity ($\bm F_g$) and tension
%TML%%  ($\bm F_T$).
%TML%%  %\footnote{We are, of
%TML%%  %course, ignoring drag and friction, but a this speed, this will not affect
%TML%%  %our answers}
%TML%%  \begin{center}
%TML%%    \begin{tikzpicture}[scale=.75]
%TML%%      \draw[thick] circle (2);
%TML%%      \begin{scope}[red]
%TML%%        \fill (0,2) circle (.1);
%TML%%        \draw[vector] (-.06,2)--+(0,-1) node[left]{$\bm F_T$};
%TML%%        \draw[vector] (.06,2)--+(0,-1.5) node[right]{$\bm F_g$};
%TML%%      \end{scope}
%TML%%      \begin{scope}[violet]
%TML%%        \fill (-2,0) circle (.1);
%TML%%        \draw[vector] (-2,0)--+(1.5,0) node[below left]{$\bm F_T$};
%TML%%        \draw[vector] (-2,0)--+(0,-1.5) node[left]{$\bm F_g$};
%TML%%      \end{scope}
%TML%%      \begin{scope}[orange]
%TML%%        \fill (0,-2) circle (.1);
%TML%%        \draw[vector] (0,-2)--+(0,1.7) node[right]{$\bm F_T$};
%TML%%        \draw[vector] (0,-2)--+(0,-1.5) node[left]{$\bm F_g$}; 
%TML%%      \end{scope}
%TML%%    \end{tikzpicture}
%TML%%  \end{center}
%TML%%  \vspace{-.2in}Since the circular motion is not uniform (i.e.\ the speed of
%TML%%  the yo-yo is not constant), we have to also use conservation of energy to
%TML%%  solve it.
%TML%%
%TML%%
%TML%%
%TML%%
%TML%%
%TML%%\begin{frame}{Example Problem: Vertical Motion}
%TML%%  \centering
%TML%%  \begin{tikzpicture}
%TML%%    \draw[dashed] circle (2);
%TML%%    \begin{scope}[red]
%TML%%      \fill (0,2) circle (.1);
%TML%%      \draw[vector] (-.06,2)--+(0,-1) node[left]{$\bm F_T$};
%TML%%      \draw[vector] (.06,2)--+(0,-1.5) node[right]{$\bm F_g$};
%TML%%      \draw[vector,black] (-.2,2)--+(-1,0) node[left]{$\bm v$};
%TML%%    \end{scope}
%TML%%
%TML%%    \node[text width=7.8cm,fill=red!10] (fc) at (6.6,2.5){
%TML%%      At the top of the circle, centripetal force is provided by both gravity
%TML%%      and string tension, i.e.:
%TML%%      
%TML%%      \vspace{-.1in}\begin{displaymath}
%TML%%        F_c = Ma_c\quad\rightarrow\quad
%TML%%        F_T+F_g = \frac{Mv^2}R
%TML%%      \end{displaymath}\par
%TML%%    };
%TML%%    \draw[axes,red] (fc) to[out=180,in=80] (0,2.2);
%TML%%    \uncover<2->{
%TML%%      \node[text width=7.8cm,fill=yellow!10] (min) at (6.6,0){
%TML%%        Since $M$, $g$ and $R$ are constant, minimum velocity $v_\text{min}$ on
%TML%%        the right side means $F_T=0$ on the left side. We are left with:
%TML%%        
%TML%%        \vspace{-.1in}\begin{displaymath}
%TML%%          Mg = \frac{Mv_\text{min}^2}R
%TML%%        \end{displaymath}\par
%TML%%      };
%TML%%    }
%TML%%    \uncover<3->{
%TML%%      \node[text width=7.8cm,fill=green!12] at (6.6,-2.4){
%TML%%        Cancelling $M$ and solving for $v_\text{min}$, we have:
%TML%%      
%TML%%        \vspace{-.12in}\begin{displaymath}
%TML%%          v^2=gR\quad\rightarrow\quad\boxed{v_\text{min} = \sqrt{gR}}
%TML%%        \end{displaymath}\par
%TML%%      };
%TML%%    }
%TML%%  \end{tikzpicture}
%TML%%
%TML%%
%TML%%
%TML%%
%TML%%
%TML%%\begin{frame}{Example Problem: Vertical Motion}
%TML%%  \centering
%TML%%  \begin{tikzpicture}
%TML%%    \fill circle (.05);
%TML%%    \draw[dashed] circle (2);
%TML%%    \begin{scope}[violet]
%TML%%      \fill (-2,0) circle (.1);
%TML%%      \draw[vector] (-2,0)--+(1.5,0) node[below left]{$\bm F_T$};
%TML%%      \draw[vector] (-2,0)--+(0,-1.5) node[right]{$\bm F_g$};
%TML%%    \end{scope}
%TML%%
%TML%%    \node[text width=5.5cm,fill=red!10] (fc) at (-5.2,2.2){
%TML%%      At the side of the circle, centripetal force is provided only by
%TML%%      tension:
%TML%%
%TML%%      \vspace{-.1in}\begin{displaymath}
%TML%%        F_c = Ma_c\quad\rightarrow\quad
%TML%%        F_T = \frac{Mv^2}R
%TML%%      \end{displaymath}
%TML%%      But we do not know the speed $v$ of the yo-yo at this location yet.
%TML%%    };
%TML%%    
%TML%%    \uncover<2->{
%TML%%      \node[text width=4.5cm,fill=yellow!15] (min) at (4.5,2.6){
%TML%%        Using conservation of energy:
%TML%%        
%TML%%        \vspace{-.2in}\begin{align*}
%TML%%          K_\text{top} + U_\text{top} &= K_\text{side}\\
%TML%%          \frac12Mv_\text{top}^2 + MgR &=\frac 12Mv_\text{side}^2
%TML%%        \end{align*}
%TML%%      };
%TML%%    }
%TML%%    
%TML%%    \uncover<3->{
%TML%%      \node[text width=4.5cm,fill=green!10] at (4.5,.4){
%TML%%        Cancelling $M$ term and solving for $v_\text{side}^2$, we have:
%TML%%
%TML%%        \vspace{-.1in}\begin{displaymath}
%TML%%          v_\text{side}^2 = v_\text{top}^2+2gR
%TML%%        \end{displaymath}\par
%TML%%      };
%TML%%    }
%TML%%
%TML%%
%TML%%    \uncover<4->{
%TML%%      \node[text width=4.5cm,fill=blue!15] at (4.5,-1.6){
%TML%%        Since $v_\text{top}^2=v_\text{min}^2=gR$ that we have just calculated,
%TML%%
%TML%%        \vspace{-.2in}\begin{displaymath}
%TML%%          v_\text{side}^2 = gR+2gR=3gR
%TML%%        \end{displaymath}\par
%TML%%      };
%TML%%    }
%TML%%
%TML%%    \uncover<5->{
%TML%%      \node[text width=5.5cm,fill=violet!15] at (-5.2,-1.1){
%TML%%        Now the final expression for tension:
%TML%%        
%TML%%        \vspace{-.2in}\begin{displaymath}
%TML%%          F_T = \frac{Mv^2}R = \frac{M(3gR)}R=\boxed{3Mg}
%TML%%        \end{displaymath}
%TML%%        Tension is 3 times the weight of the yo-yo!
%TML%%      };
%TML%%    }
%TML%%  \end{tikzpicture}
%TML%%
%TML%%
%TML%%
%TML%%
%TML%%\begin{frame}{Example Problem: Vertical Motion}
%TML%%  \centering
%TML%%  \begin{tikzpicture}
%TML%%    \fill circle (.05);
%TML%%    \draw[dashed] circle (2);
%TML%%    \begin{scope}[orange]
%TML%%      \fill (0,-2) circle (.1);
%TML%%      \draw[vector] (0,-2)--+(0,1.7) node[right]{$\bm F_T$};
%TML%%      \draw[vector] (0,-2)--+(0,-1.5) node[left]{$\bm F_g$}; 
%TML%%    \end{scope}
%TML%%
%TML%%    \node[text width=5.6cm,fill=red!10] (fc) at (-5,1.2){
%TML%%      At the bottom of the circle, tension contributes to centripetal force,
%TML%%      while gravity contributes \emph{against} it:
%TML%%
%TML%%      \vspace{-.22in}\begin{displaymath}
%TML%%        F_c = Ma_c\quad\rightarrow\quad
%TML%%        F_T-F_g = \frac{Mv^2}R
%TML%%      \end{displaymath}
%TML%%      Again, we need to find the speed of the yo-yo at this location.
%TML%%    };
%TML%%    
%TML%%    \uncover<2->{
%TML%%      \node[text width=4.6cm,fill=yellow!10] (min) at (4.5,1.7){
%TML%%        Using conservation of energy again:
%TML%%        
%TML%%        \vspace{-.2in}\begin{align*}
%TML%%          K_\text{top} + U_\text{top} &= K_\text{bottom}\\
%TML%%          \frac12Mv_\text{top}^2 + MgR &=\frac 12Mv_\text{bottom}^2
%TML%%        \end{align*}
%TML%%      };
%TML%%    }
%TML%%    
%TML%%    \uncover<3->{
%TML%%      \node[text width=4.6cm,fill=green!10] at (4.5,-.6){
%TML%%        Cancelling $M$ term and solving for $v_\text{bottom}^2$, we have:
%TML%%
%TML%%        \vspace{-.1in}\begin{displaymath}
%TML%%          v_\text{bottom}^2 = v_\text{top}^2+4gR
%TML%%        \end{displaymath}\par
%TML%%      };
%TML%%    }
%TML%%
%TML%%
%TML%%    \uncover<4->{
%TML%%      \node[text width=4.6cm,fill=blue!15] at (4.5,-2.5){
%TML%%        Recognizing that $v_\text{top}^2=gR$ like we did before:
%TML%%
%TML%%        \vspace{-.2in}\begin{displaymath}
%TML%%          v_\text{bottom}^2 = gR+4gR=5gR
%TML%%        \end{displaymath}\par
%TML%%      };
%TML%%    }
%TML%%
%TML%%    \uncover<5->{
%TML%%      \node[text width=5.6cm,fill=violet!15] at (-5,-2.2){
%TML%%        Now the final expression for tension:
%TML%%        
%TML%%        \vspace{-.2in}\begin{align*}
%TML%%          F_T &= Mg+\frac{Mv^2}R =Mg + \frac{M(5gR)}R\\
%TML%%          &=\boxed{6Mg}
%TML%%        \end{align*}
%TML%%        $F_T$ is 6 times the weight of the yo-yo
%TML%%      };
%TML%%    }
%TML%%  \end{tikzpicture}

\begin{example}
  A cord is tied to a pail of water, and the pail is swung
  in a vertical circle of \SI{1.}\metre. What must be the minimum velocity of
  the pail be at its highest point so that no water spills out?
%  \begin{enumerate}[(A)]
%  \item\SI{3.1}{\metre\per\second}
%  \item\SI{5.6}{\metre\per\second}
%  \item\SI{20.7}{\metre\per\second}
  %  \item\SI{100.5}{\metre\per\second}
  
  \textbf{Solution:}
\end{example}

\begin{example}
  A roller coaster car is on a track that forms a circular
  loop, of radius $R$, in the vertical plane. If the car is to maintain contact
  with the track at the top of the loop (generally considered to be a good
  thing), what is the minimum speed that the car must have at the bottom of the
  loop? Ignore air resistance and rolling friction.

  \textbf{Solution:}
%%  \begin{enumerate}[(A)]
%%  \item $\sqrt{2gR}$
%%  \item $\sqrt{3gR}$
%%  \item $\sqrt{4gR}$
%%  \item $\sqrt{5gR}$
%%  \end{enumerate}
\end{example}

\begin{example}
  A stone of mass $m$ is attached to a light strong string
  and whirled in a \emph{vertical} circle of radius $r$. At the exact bottom of
  the path, the tension of the string is three times the weight of the stone.
  The stone's speed at that point is:
  
  \textbf{Solution:}
%%  \begin{enumerate}[(A)]
%%  \item $2\sqrt{gR}$
%%  \item $\sqrt{2gR}$
%%  \item $\sqrt{3gR}$
%%  \item $4gR$
\end{example}

