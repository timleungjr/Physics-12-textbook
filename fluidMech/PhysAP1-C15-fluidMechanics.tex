\chapter{Fluid Mechanics}
\label{chapter:fluid-mechanics}

\section{What is a Fluid?}
%  \begin{itemize}
%  \item\textbf{Non-scientific definition:} anything that \emph{flows}, which
%    covers most gases and liquids
%  \item\textbf{Scientific definition:} any substance that deforms
%    \emph{continuously} under oblique stress
%  \item Fluid is continuous: it will fill all available space without gaps
%  \end{itemize}

%
%
%
%{Deformation Under Oblique Stress}
%  When an oblique force $F$ is applied to a solid, it deforms until all the
%  internal forces are balanced, and the deformation stops:
%  \begin{center}
%    \begin{tikzpicture}[scale=.6]
%      \fill[yellow!70] rectangle (4,2);
%      \draw[vectors] (1,2.4)--(3,2.4) node[right]{$F$};
%      \draw[vectors] (3,-.4)--(1,-.4) node[left] {$F$};
%      \draw[ultra thick] (-.5,0)--(4.5,0);
%      \draw[ultra thick] (-.5,2)--(4.5,2);
%      \draw[line width=2.5,->] (6,1)--(10,1);
%      \fill[yellow!70] (12,0)--(13,2)--(17,2)--(16,0)--cycle;
%      \draw[vectors] (14,2.4)--(16,2.4) node[right]{$F$};
%      \draw[vectors] (15,-.4)--(13,-.4) node[left]{$F$};
%      \draw[vectors,red] (15.7,1.6)--(13.7,1.6) node[left]{$F$};
%      \draw[vectors,red] (13.3,0.4)--(15.3,0.4) node[right]{$F$};
%      \draw[ultra thick] (11.5,0)--(16.5,0);
%      \draw[ultra thick] (12.5,2)--(17.5,2);
%    \end{tikzpicture}
%  \end{center}
%  However, when a force is applied to a fluid, it continues to deform in shape
%  as long as the force is present
%  \begin{center}
%    \begin{tikzpicture}[scale=.6]
%      \fill[yellow!70] rectangle (4,2);
%      \draw[vectors] (1,2.4)--(3,2.4) node[right]{$F$};
%      \draw[vectors] (3,-.4)--(1,-.4) node[left] {$F$};
%      \draw[ultra thick] (-.5,0)--(4.5,0);
%      \draw[ultra thick] (-.5,2)--(4.5,2);
%      \draw[line width=2.5,->] (6,1)--(10,1);
%      \fill[yellow!70] (12,0)--(13,2)--(17,2)--(16,0)--cycle;
%      \draw[vectors] (14,2.4)--(16,2.4) node[right]{$F$};
%      \draw[vectors] (15,-.4)--(13,-.4) node[left]{$F$};
%      \draw[vectors] (18,2)--(19.5,2) node[right]{$v$};
%      \draw[vectors] (11,0)--(9.5,0) node[left]{$v$};
%      \draw[ultra thick] (11.5,0)--(16.5,0);
%      \draw[ultra thick] (12.5,2)--(17.5,2);
%    \end{tikzpicture}
%  \end{center}

%
%
%
%{Density of a Fluid}
The \textbf{density} $\rho$ of a fluid is defined as the mass of the fluid
$m_\text{fluid}$ per unit volume $V_\text{fluid}$ that it occupies:
\begin{equation}
  \boxed{
    \rho=\frac{m_\text{fluid}}{V_\text{fluid}}
  }
\end{equation}
%  \begin{itemize}
%  \item Unit for density is \textbf{kilograms per meter cubed}
%    (\si{\kilo\gram\per\metre\cubed})
%  \item Below Mach number $M\approx 0.3$, density can be assumed to be constant
%    throughout the fluid
%  \item May be dependent on temperature by basic thermal expansion
%  \end{itemize}

%
%
%
%{Viscosity of a Fluid}
\textbf{Viscosity} $\mu$ measures how ``thick'' a fluid is. It relates the rate
of deformation ($\Delta u/\Delta y$) of the fluid to the shear stress $\tau$
that it experiences:
\begin{equation}
  \boxed{
    \tau = \mu\frac{\Delta u}{\Delta y}
  }
\end{equation}
%  \begin{itemize}
%  \item e.g.\ honey is more viscous than water.
%  \item Shear stress is defined as $\tau=F/A$, with a unit of \textbf{pascal}
%    (\si\pascal), which is the same as for pressure
%  \item In AP Physics 2, we will mostly ignore viscous effects, as important as
%    they are
%  \end{itemize}

%
%
%
%{Hydrostatics}
%  The fluid \textbf{pressure} on its density and depth:
\begin{equation}
  \boxed{
    P=P_0+\rho_\text{fluid}gz
  }
\end{equation} 
%  where $g$ is the acceleration due to gravity, $z$ is the depth below the
%  surface, and $P_0=\SI{1.01e5}\pascal$ is the atmospheric pressure at the
%  surface.
%  \begin{itemize}
%  \item Pressure is the same in all directions
%  \item Pressure is defined as force per unit area, and the unit is
%    \textbf{pascal} (\si\pascal) where
%    $\SI1\pascal=\SI1{\newton\per\metre\squared}$:
%
%    \eq{-.1in}{
%      \boxed{P=\frac FA}\quad\quad\SI1\pascal=\SI1{\newton\per\metre\squared}
%    }
%  \end{itemize}

%
%
%
%{Pascal's Principle}
%  If force is applied somewhere on a container holding fluid, the pressure
%  increases \emph{everywhere} in the fluid, not just where the force is applied.
%
%  i.e.\ the pressure of the force will be transmitted into the fluid.

%
%
%
%{Pressure with Different Fluids}
%  
%    \column{.3\textwidth}
%    \begin{tikzpicture}
%      \draw[axes] (-1,4.5)--(-1,2) node[below]{$z$};
%      \fill[white!10!blue!60] (0,1) rectangle (3,2.5);
%      \fill[white!30!blue!45] (0,2.5) rectangle (3,3);
%      \fill[white!40!blue!30] (0,3) rectangle (3,4.2);
%      \fill[white!50!blue!20] (0,4.2) rectangle (3,5);
%      \node at (-.2,5){\tiny $z=0$};
%      \node at (-.2,4.2){\tiny $z_1$};
%      \node at (-.2,3){\tiny $z_2$};
%      \node at (-.2,2.5){\tiny $z_3$};
%      \node at (-.2,1){\tiny $z_4$};
%
%      \node at (1.5,5.1){\tiny Known pressure $P_0$};
%      \node at (1.5,4.6){\tiny Oil, $\rho_O$};
%      \node at (1.5,3.6){\tiny Water, $\rho_W$};
%      \node at (1.5,2.75){\tiny Glycerine, $\rho_G$};
%      \node at (1.5,1.75){\tiny Mercury, $\rho_M$};
%    \end{tikzpicture}
%    
%    \column{.7\textwidth}
%    For the fluid surface to remain \emph{static}, the fluid pressure on both
%    side of the interface have to be equal. In this example:
%
%      \begin{displaymath}
%        \begin{split}
%          P_1-P_0&=\rho_Ogz_1\\
%          P_2-P_1&=\rho_Wg(z_2-z_1)\\
%          P_3-P_2&=\rho_Gg(z_3-z_2)\\
%          P_4-P_3&=\rho_Mg(z_4-z_3)\\ \hline
%          P_4-P_0&=\sum\Delta P
%        \end{split}
%      \end{displaymath}
%  

%
%
%
\begin{example}
  An aquarium is filled with water. The lateral wall of the aquarium is
  \SI{40}{\centi\metre} long and \SI{30}{\centi\metre} high. Using
  \SI{10}{\metre\per\second\squared} for the acceleration due to gravity, and
  \SI1{\gram\per\centi\metre\squared} for density of water, the force on the
  lateral wall of the aquarium is:
%    \begin{enumerate}[(A)]
%    \item\SI{36}\newton
%    \item\SI{90}\newton
%    \item\SI{180}\newton
%    \item\SI{1500}\newton
%    \end{enumerate}
%
%    \column{.3\textwidth}
%    \pic{1}{home-fish-tank}
\end{example} 

\begin{example}
  Consider the hydraulic jack in the diagram. A person stands on a piston that
  pushes down on a thin cylinder full of water. The cylinder is connected via
  pipes to a wide platform on top of which rests a 1-ton
  (\SI{1000}{\kilo\gram}) car. The area of the platform under the car is
  \SI{25}{\metre\squared}; the person stands on a \SI{.3}{\metre\squared}
  piston. What is the lightest weight of a person who could successfully lift
  the car?
%  \begin{center}
%    \pic{.35}{jack}
%    
%    \vspace{-.2in}{\tiny Believe it or not, there \emph{is} someone who draws
%      worse diagrams than me!}
%  \end{center}
\end{example}

\begin{example}%{A ``Manometer'' Example}
  Pressure gauge $B$ is to measure the pressure at point $A$ in a water flow,
  as shown in the figure on the right. If the pressure at $B$ is
  \SI{87}{\kilo\pascal}, estimate the pressure at $A$, in \si{\kilo\pascal}.
  Assume all fluids are at \SI{20}\celsius. The densities of water, mercury and
  SAE 30 oil are, respectively?
  \begin{align*}
    \rho_\text{water}&=\SI{1000}{\kilo\gram\per\metre\cubed}\\
    \rho_\text{Hg}&=\SI{13600}{\kilo\gram\per\metre\cubed}\\
    \rho_\text{oil}&=\SI{890}{\kilo\gram\per\metre\cubed}
  \end{align*}
   
%    \pic1{mano}
\end{example}  

%
%
%
%%
%%  \frametitle{Hydrostatic Example: Forces on a Hinge}
%%  
%%
%%    The gate in the figure on the left is \SI{5}{m} wide, is hinged at point
%%    $B$, and rests against a smooth wall at point $A$. Compute
%%    \begin{enumerate}[(a)]
%%    \item the force on the gate due to seawater pressure, and
%%    \item the horizontal force $P$ exerted by the wall at point $A$, and
%%    \item the reaction at the hinge $B$
%%    \end{enumerate}
%%  
%
%
%
\section{Buoyant Force}
%
%{Buoyancy: Everything Floats a Little}
%  When an object is submerged inside a fluid
%  \begin{itemize}
%  \item The fluid exerts a pressure at the surface of the object
%  \item By hydrostatics, the pressure is higher at the bottom than at the top
%  \end{itemize}
%  \begin{center}
%    \pic{.3}{rock_fbvectors}
%  \end{center}

%
%
%
%{Buoyancy}
Using some basic calculus (well, depends on who you ask), we can find the
pressure over the entire surface to find the total buoyant force $\vec F_B$ the
fluid exerts on the object. The expression is surprisingly simple:  
\begin{equation}
  \boxed{
    \vec F_B=\rho_\text{fluid}gV\hat z=m_\text{fluid}g\hat z
  }
\end{equation}
where $\rho_\text{fluid}$ is the density of the displaced fluid, and $V$ is
the volume displaced. The direction of the force is upward ($\hat z$). This
equation is known as \textbf{Archimedes' principle}.

%  \textbf{Buoyant force has a magnitude that equals to the
%    weight of the fluid displaced by the submerged object, pointing upward.}

%
%
%
%{Buoyancy}
%
%  \eq{0in}{
%    \boxed{\vec B = \rho_\text{fluid}gV\hat z=
%      m_\text{fluid}g\hat z}
%\end{equation}
%
%  Buoyancy does not depend on:
%  \begin{itemize}
%  \item the mass of the immersed object, or
%  \item the density of the immersed object
%  \end{itemize}
%
%  \vspace{.15in}Objects immersed in a fluid have an ``apparent weight''
%  $\vec w'$ that is reduced by the buoyant force:
%
\begin{equation}
  \vec w'=\vec w-\vec B=\rho'\vec gV
\end{equation}
%  
%  where $\rho'=\rho_\text{obj}-\rho_\text{fluid}$ is the relative density

%
%
%
%{How Submarines Work}
%  Like this:
%  \begin{center}
%    \pic{.7}{EbHMOXk}
%  \end{center}

%
%
%
%{How Submarines Work}
%  Like all ships, a submarine does not naturally sink due to buoyancy. When a
%  submarine submerges, water is pumped into the  ``ballast tanks'' in the hull
%  hull to make the submarine heavier.
%  \begin{center}
%    \pic{1}{risinglemur}
%  \end{center}

%
%
%
\begin{example}
  An apple is held completely submerged just below the surface of a container
  of water. The apple is then moved to a deeper point in the water. Compared
  with the force needed to hold the water just below the surface, what is the
  force needed to hold it at a deeper point?
%    \begin{enumerate}[(A)]
%    \item Larger
%    \item The same
%    \item Smaller
%    \item Impossible to determine
%    \end{enumerate}
%    \pic1{apple}
\end{example} 

%
%
%
\begin{example}  
%    \pic1{hpa_b}
  A salvage ship tries to raise a sunken miniature submarine from the bottom of
  Lake Superior. The submarine and its contents have a mass of
  \SI{72000}{\kilo\gram} and a volume of \SI{18.9}{\metre\cubed}. What upward
  force must be applied to raise the submarine? The density of water is
  \SI{1000}{\kilo\gram\per\metre\cubed}.
%    \begin{enumerate}[(A)]
%    \item\SI{1.8e5}\newton
%    \item\SI{2.0e5}\newton
%    \item\SI{4.8e5}\newton
%    \item\SI{5.2e5}\newton
%    \end{enumerate}
\end{example} 


\section{Fluid Flow}

%%{Fluid Flow}
%%  \begin{center}
%%    As important as it is to understand hydrostatics,\\
%%    it's way more interesting when the fluid is moving!
%%  \end{center}
%
%%
%%
%%
%%{Control Volume and Control Surfaces}
%%  A control volume ``CV'' is a fixed volume in which fluid is able to flow in
%%  and out of it. The surfaces of the control volume is called the control
%%  surface ``CS''.
%%  \begin{center}
%%    \pic{.5}{CV-CS}
%%  \end{center}
%
%
%
%
\subsection{Navier-Stokes Equations}
The governing equations in fluid mechanics is called
\textbf{Navier-Stokes equations}, which is written in complicated vector
and calculus symbols:
\begin{align}
  \frac{\partial\rho}{\partial t} + \nabla\cdotp(\rho\vec v)&=0\\
  \frac{\partial(\rho\vec v)}{\partial t} +
  \nabla(\rho\vec v\otimes\vec v) &= -\nabla P+\nabla\cdot\vec\tau
  +\rho\vec g\\
  \rho\frac{\partial e}{\partial t} + P(\nabla\cdotp\vec v) &=
  \nabla\cdotp(k\nabla T)+\Phi
\end{align}
Even for university students experienced with calculus, solving these equations
is still an intimating task. \textbf{But what do they actually mean?}

%{Continuity Equation}
%  The first equation is called the \textbf{continuity equation}, which is the
%  conservation of mass:
%
%\begin{equation}
%    \frac{\partial\rho}{\partial t} + \nabla\cdot(\rho\vec v)=0
%\end{equation}
%
%  \textbf{Meaning:} The decrease in fluid mass in a fixed volume containing a
%  fluid is the amount of mass that flows out minus the volume that flows into
%  the volume.

%
%
%
%{Momentum Equation}
%  The second equation is the \textbf{momentum equation}, which is the 
%  momentum-impulse theorem applied to fluids in a volume:
%
%\begin{equation}
%    \frac{\partial(\rho\vec v)}{\partial t} +
%    \nabla(\rho\vec v\otimes\vec v) = -\nabla P+\nabla\cdot\vec\tau
%    +\rho\vec g
%\end{equation}
%
%  \textbf{Meaning:} The decrease in the total momentum of a fluid in a fixed
%  volume is the sum of all the external forces (pressure, shear \& body forces)
%  applied to the fluids, plus the change in momentum through the flow of
%  particles in and out of the volume.

%
%
%
%{Energy Equation}
%  The third and final equation is the \textbf{energy equation}, which is the
%  the conservation of energy theorem applied to fluids:
%  
%\begin{equation}
%  \rho\frac{\partial e}{\partial t} + P(\nabla\cdotp\vec v)=
%  \nabla\cdotp(k\nabla T)+\Phi
%\end{equation}
%
%  \textbf{Meaning:} The decrease in the internal energy of the fluid (i.e.\ the
%  total kinetic energy of all the fluid particles) in a fixed volume is the
%  work done by fluid's pressure forces, viscous forces (i.e.\ friction), the
%  heat transfer into the fluid, plus the change in energy from the flow of
%  fluid in and out of the volume.

%
%
%
\subsection{Bernoulli Equation}
For an ``ideal fluid flow'', we make the following assumptions to simplify the
Navier-Stokes equation:
\begin{enumerate}
\item The flow is \textbf{steady}
  \begin{itemize}
  \item Flow is ``time independent'', i.e.\ does not change with time
  \end{itemize}
\item The flow is \textbf{inviscid}
  \begin{itemize}
  \item The fluid has no viscosity
  \item No friction between the fluid and the surrounding, and therefore
  \item No shear stresses on the fluid
  \item Only forces are pressure at the surface, and body forces from gravity
  \end{itemize}
\item The flow is \textbf{incompressible}
  \begin{itemize}
  \item Density is constant throughout
  \end{itemize}
\item\textbf{No shaft work} done along the streamline
\item\textbf{No heat transfer} along the streamline
\end{enumerate}
Then the Navier-Stokes equations reduces to the \textbf{Bernoulli equation}.
For steady flow of an ideal fluid along any streamline:
\begin{equation}
  \boxed{
    p_1+\frac12\rho v_1^2+\rho gh_1=
    p_2+\frac12\rho v_2^2+\rho gh_2
  }
\end{equation}

%
%
%
\subsection{Streamlines}
\textbf{Streamlines} are the paths that fluid particles follow when they move.
Much like the gravitational and electric fields
%\footnote{Gravitational field is
%a topic covered briefly in AP Physics 1, and we will cover electric field later in this course}, this is a vector field.
\begin{itemize}
\item The velocity of the fluid flow is tangent to the streamline
\item Do not tell the magnitude of the velocity
\item Under ideal fluid flow, streamlines are always smooth
\end{itemize}
%\begin{center}
%  \pic{.35}{800px-AirfoilDeflectionLift_W3C}
%\end{center}

%
%
%{Real Fluid Flow}
%  In real fluid flow, streamlines are often not smooth
%  \begin{center}
%    \pic{.3}{stalling}
%    \pic{.362}{baseball}
%  \end{center}
%  Note that the Bernoulli equation does not capture a lot of real fluid flow
%  properties (re: our earlier assumptions)

%
%
%
\begin{equation}
  \boxed{
    p_1+\frac12\rho v_1^2+\rho gh_1= p_2+\frac12\rho v_2^2+\rho gh_2
  }
\end{equation}

%  \textbf{Dynamic pressure} is the kinetic energy per unit
%  volume\footnote{Anyone who took AP Physics 1 or Physics 11 may remember that
%  the drag force also contains the dynamic pressure term}:
%
\begin{equation}
  \frac12\rho v^2
\end{equation}
%  
%  \textbf{Hydrostatic pressure} is the gravitational potential energy per unit
%  volume:
%
\begin{equation}
  \rho gh
\end{equation}
The relationship between pressure and energy will be explained in the next
topic.

%
%
%
%{Bernoulli Equation}
%  Regions where Bernoulli equation is valid:
%  \begin{center}
%    \pic{.8}{bernoulli}
%  \end{center}

%
%
%
\subsection{Inlet Outlet Flow}
%  \begin{center}
%    \pic{.5}{physicsbook_fluids_graphik_26}
%  \end{center}
%  In this example, the mass flowing at the inlet is the same as the flow out of
%  it:
%
\begin{equation}
  \boxed{
    \rho_1 v_1A_1=\rho_2 v_2A_2
  }
\end{equation} 
For constant fluid density, the $\rho$ terms on both sides of the equation will
cancel:
\begin{equation}
  \boxed{
    v_1A_1=v_2A_2
  }
\end{equation}

%
%
%
%{Example: Multiple Inlet \& Outlets}
%
%    \column{.43\textwidth}
%    \begin{tikzpicture}[scale=.9]
%      \fill[blue!20!gray!30] rectangle (4,4);
%      \fill[blue!20!gray!30] (2.5,0) rectangle (3.25,-1);
%      \fill[blue!20!gray!30] (-1,.5) rectangle (0,1.5);
%      \fill[blue!20!gray!30] (4,3) rectangle (5,3.75);
%      \draw[very thick] (-1,1.5)--(0,1.5)--(0,4)--(4,4)--(4,3.75)--(5,3.75);
%      \draw[very thick] (-1,0.5)--(0,0.5)--(0,0)--(2.5,0)--(2.5,-1);
%      \draw[very thick] (3.25,-1)--(3.25,0)--(4,0)--(4,3)--(5,3);
%      \node[draw] at (-.5,2.25) (a) {1};
%      \node[draw] at (4.5,4.5) (b) {2};
%      \node[draw] at (4,-.5) (c) {3};
%      \draw(-.5,1.5)--(a);
%      \draw(4.5,3.75)--(b);
%      \draw(3.25,-.5)--(c);
%      \draw[vectors,blue] (-1.5,1)--(-.5,1);
%      \draw[vectors,blue] (4.5,3.375)--(5.5,3.375);
%      \draw[<->,very thick,blue] (2.875,-.5)--(2.875,-1.5);
%    \end{tikzpicture}
%
%    \column{.57\textwidth}
%    \textbf{Example:} Water at \SI{20}{\celsius} flows steadily through a
%    closed tank, as shown in the figure. As section 1,
%    $D_1=\SI6{\centi\metre}$ and the volume flow is
%    \SI{100}{\metre\cubed\per\hour}. At section 2, $D_2=\SI5{\centi\metre}$
%    and the average velocity is \SI8{\metre\per\second}. If
%    $D_3=\SI4{\centi\metre}$, what is
%    \begin{enumerate}
%    \item the flow rate $Q_3$ in \si{\metre\cubed\per\hour}?
%    \item the average $v_3$ in \si{\metre\per\second}?
%    \end{enumerate}

%
%
%
%{Example}
%  \textbf{Example:} Find a relation between the nozzle discharge velocity
%  $V$ and the tank free-surface height $h$. Assume frictionless flow.
%  \begin{center}
%    \pic{.35}{EGL}
%  \end{center}
%
%    \footnotesize The line labelled ``EGL'' is called the
%    ``energy grade line'', or the ``Bernoulli head'', given by the equation
%    $h_0=z+p/\rho g+v^2/2g$. In the region where Bernoulli equation is valid,
%    EGL is a constant.\par
