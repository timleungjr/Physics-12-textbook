\section*{Exercise Problems}

\begin{enumerate}[itemsep=6pt]

\item A planet moves faster in its orbit
  \begin{enumerate}
  \item when it is farthest from the Sun.
  \item the greater its mass.
  \item when it is in opposition.
  \item the farther it is from its satellites.
  \item when it is nearer the Sun.
  \end{enumerate}

\item Kepler's first law says that the planets move in elliptical orbits with
  the sun at one focus of the ellipse. What is at the other focus?
  \begin{enumerate}
  \item Empty space.
  \item The Earth
  \item The Moon.
  \item The planet in question.
  \end{enumerate}

\item The Earth is at an average distance of 1 AU from the Sun and has an
  orbital period of 1 year. Jupiter orbits the Sun at approximately 5 AU. How
  long is the orbital period of Jupiter?
  \begin{enumerate}
  \item 1 year
  \item 2 years
  \item 5 years
  \item 11 years
  \item 125 years
  \end{enumerate}
  
\item If a planet has an average distance from the sun (semi-major axis of its
  orbit) of 4 astronomical units, what is the period of its orbit? Hint:
  Use Kepler's third law.
  \begin{enumerate}
  \item $1$ year
  \item $4$ years
  \item $64$ years
  \item $12$ years
  \item $8$ years
  \end{enumerate}
  
\item The Earth and the moon apply a gravitational force to each other. Which
  of the statements is true?
  \begin{enumerate}
  \item Earth applies a greater force on the moon than the moon exerts on Earth.
  \item Earth applies a smaller force on the moon than the moon exerts on Earth.
  \item Earth applies a force on the moon, but the moon does not exert a force
    on Earth.
  \item Earth does not apply a force on the moon, but the moon exerts a force
    on Earth.
  \item The force Earth applies to the moon is equal and opposite to the force
    the moon applies to Earth.
  \end{enumerate}
  
\item Two masses exert a gravitational force $F$ on each other. If one of the
  masses is doubled, and the distance between the masses is tripled, the new
  force between them is
  \begin{enumerate}
  \item $6F$
  \item $2F/3$
  \item $2F/9$
  \item $3F/2$
  \item $4F/9$
  \end{enumerate}
  
\item What can be said about a satellite as it orbits the Earth at a constant
  speed? %\emph{Select two answers.}
  \begin{enumerate}
  \item The satellite's velocity is constant.
  \item The satellite's acceleration is constant.
  \item The satellite experiences acceleration towards the centre of the orbit.
  \item The satellite experiences acceleration away from the centre of the
    orbit.
  \item The satellite experiences acceleration tangent to the centre of the
    orbit.
  \end{enumerate}
  
\item When you shoot a cannonball directly upwards from the surface of the
  Earth with less than escape velocity, what will happen?
  \begin{enumerate}
  \item It will slow down, but will not fall back to Earth.
  \item It will keep moving at a constant speed and not fall back to Earth.
  \item It will slow down and eventually fall back to Earth.
  \item It will speed up as it moves away from Earth.
  \end{enumerate}

%  \item I throw a light plastic ball up the air and watch its motion. Which
%  stays constant?
%  \end{enumerate}
%    \item Its total mechanical energy.
%    \item Its kinetic energy.
%    \item Its gravitational potential energy.
%    \item Two of A, B and C are correct.
%    \item None of the other answers are correct.
%  \end{enumerate}

\item Newton discovered that gravity can be described as:
  \begin{enumerate}
  \item A spring-like connection between any two masses.
  \item A universal attraction between masses which gets stronger with
    distance.
  \item A force which is independent of the masses of the objects involved.
  \item An attraction between like electrical charges.
  \item A universal attraction between any two masses, which falls off as the
    square of their distance.
  \end{enumerate}

\item Pluto has a very elliptical orbit about the Sun, so sometimes it is
  closer to the Sun and sometimes it is farther away. Which stays constant?
  \begin{enumerate}
  \item Its gravitational potential energy.
  \item Its kinetic energy.
  \item Its total energy.
  \item Two of the above answers are correct.
  \item None of the other answers are correct.
  \end{enumerate}

\item If the force of gravity between a book of mass
  \SI{.500}{\kilo\gram} and a calculator of \SI{.100}{\kilo\gram} is
  \SI{1.5e-10}\newton, how far apart are they? 

\item Using the law of universal gravitation, find the location from Earth
  where the gravitational forces of Earth and the Moon balanced.

%  %\item The Pioneer 10 Spacecraft was the first to journey beyond Jupiter
%  %and is now well past Pluto. To escape from the solar system, how fast did
%  %Pioneer 10 have to be travelling as it passed the orbit of Jupiter? Assume
%  %that the mass of the solar system is essentially concentrated at the Sun. The
%  %radius of Jupiter's orbit is \SI{7.78e11}\metre.
%  %
%  %\includegraphics[width=2.25in]{pioneer1}

\item Mercury is the planet that is closest to the Sun. It has a mass of
  \SI{3.285e23}{\kilo\gram} and a radius of \SI{2.440e6}\metre.
  \begin{enumerate}[itemsep=3pt]
  \item What is the maximum speed of a satellite in a \emph{circular} orbit
    around Mercury?
  \item If the satellite is to stay in an elliptical orbit around Mercury,
    what orbital speed must it not allow to exceed? In which part of the orbit
    will the maximum speed occur?
  \end{enumerate}

\item A communications satellite is in geosynchronous orbit above Earth's
  equator.
  \begin{enumerate}[itemsep=3pt]
  \item What is the orbital period in seconds?
  \item What is the satellite's orbital speed?
  \item What is the altitude of the satellite?
  \end{enumerate}

\item A \SI{1.00e2}{\kilo\gram} space probe is in a circular orbit, 25 km
  above the surface of Titan, a moon of Saturn. If the radius of Titan is 2575
  km and its mass is \SI{1.346e23}{\kilo\gram}, determine the space probe's:
  \begin{enumerate}[itemsep=3pt]
  \item Orbital speed
  \item Orbital period
  \item Orbital kinetic energy
  \item Orbital gravitational potential energy
  \item Total orbital energy
  \item Binding (escape) energy
  \item Additional speed required for the space probe to break free from Titan
  \end{enumerate}

  %\item A \SI{550}{\kilo\gram} satellite launched upward from Earth's
  %surface reaches an orbit at a height of \SI{6000}{\kilo\metre}. Find:
  %\begin{enumerate}[itemsep=3pt]
  %  \item its change in gravitational potential energy
  %  \item its orbital kinetic energy
  %  \item its initial kinetic energy
  %\end{enumerate}
  %(Assume constant mass; hint: gain in $U_g$ is loss in $K$)
  
\item A \SI{2.0e4}{\kilo\gram} meteorite from outer space is heading towards
  Earth at \SI{2.1}{\kilo\metre\per\second}. It is on a path that will come
  within \SI{8500}{\kilo\metre} of Earth's centre.
  \begin{enumerate}[itemsep=3pt]
  \item Find the speed of the meteorite at its closest approach.
  \item Will the meteorite ever return to Earth's vicinity? Explain. (No
    additional calculations needed to find the answer.)
  \end{enumerate}
  This is a conservation of energy problem, since gravity is a conservative
  force.

  %  %\begin{enumerate}[itemsep=3pt]
%  %  \item Show that speed decreases as the radius of a satellite's orbit
%  %  increases.
%  %  \vspace{1.5in}
%  %  \item What effect does increasing an orbit's radius have on the period of
%  %  the satellite?
%  %  \vspace{1.5in}
%  %\end{enumerate}
%
%  %\item The Apollo Project was the first to put American astronauts on the
%  %moon. In total, six spacecrafts landed the moon in the 1960s. During the
%  %mission, the Apollo ``Command Module'' (CM) would typically stay in an orbit
%  %\SI{110}{\kilo\metre} above the lunar surface.
%  %\begin{enumerate}[itemsep=3pt]
%  %  \item Given the orbital altitude, how long would it take for the CM to
%  %  complete one orbit around the Moon? What is its speed in orbit?
%  %  
%  %  \item During the mission, the CM has to decrease its altitude to a mere
%  %  \SI{11}{\kilo\metre} orbit above the lunar surface so that the ``Lunar
%  %  Module'' (LM) can be safely detached from it. How much energy is released
%  %  during the descent? The combined mass of the CM and the CM is
%  %  \SI{28000}{\kilo\gram}.
%  %  \item How fast would the CM be moving at this low altitude?
%  %  \item In order to return to Earth, the CM needs to break free
%  %  from Moon's gravitational pull. What is the escape speed at this altitude?
%  %\end{enumerate}
%
%  \item As a member of the 2240 Olympic Committee, you are considering a new
%  sport: asteroid jumping. On Earth, world-class high jumpers routinely clear
%  \SI2\metre. Your job is to make sure athletes jumping from asteroids will
%  return to the asteroid. Make the simplifying assumption that asteroids are
%  spherical, with an average density of \SI{2.5e3}{\kilo\gram\per\metre\cubed}.
%  For safety, make sure that even a jumper capable of \SI3{\metre} on Earth
%  will return to the surface. What do you report for the minimum asteroid
%  diameter? \textbf{Answer to 3 significant figures.}
%  \vspace{\stretch1}
%  
%  %\item In the past, when the space shuttle delivered a crew to the
%  %International Space Station (ISS), it usually boosts the orbit of the station
%  %from about \SI{370}{\kilo\metre} to \SI{400}{\kilo\metre}. How much energy
%  %does the shuttle add to the station's orbit? (The mass of the ISS is
%  %approximately \SI{4.2e5}{\kilo\gram}.)
\end{enumerate}

