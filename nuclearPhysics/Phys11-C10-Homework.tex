\documentclass{../ossphysics}
\usepackage{mhchem}

\begin{document}
\setheader{Grade 11 Physics Class 10 Homework}
\hwtitle{11}{10}{Nuclear Physics}


\begin{questions}
  \question Three of the isotopes of radon ($Z=86$) gave 125, 134, and 136
  neutrons. Write the symbol for each.
  \vspace{\stretch1}

  \question Find the total binding energy of oxygen-16, given that is nuclear
  mass is \SI{15.9905}{u}.
  \vspace{\stretch1}
  
  \question Write the nuclear reaction equation for each a curium-248 atom
  undergoing alpha decay (refer to the periodic table).
  \vspace{\stretch1}
  
  \question Write the nuclear reaction equation for a gold-198 atom undergoing
  beta-negative decay.
  \vspace{\stretch1}

  \question Carbon-10 is a positron emitter with a \SI{19.3}{\second} half life.
  \begin{parts}
    \part Write an equation to describe its decay.
    \vspace{\stretch1}

    \part A \ce{^{10}_6C} sample has an initial mass of \SI{6.0}\gram. What will
    be its mass after \SI{1.0}{min}?
    \vspace{\stretch1}
  \end{parts}
  \newpage

  \question Determine the energy equivalent, in joules and electronvolts, of a
  single electron
  \vspace{\stretch1}
  
  \question Aluminum-26, which decays into Magnesium-26, has a half-life of
  approximately \SI{720000}{years}.
  \begin{parts}
    \part What type of decay does Al-26 undergo?
    \vspace{\stretch1}

    \part A moon rock has \SI3{\percent} of its original Al-26 mass. Determine
    the age of the moon rock.
    \vspace{\stretch1}

    \part Discuss any assumptions that must be made when using this method of
    dating.
    \vspace{\stretch1}
  \end{parts}

  \question Assuming that \SI{200}{\mega\electronvolt} per fission, determine
  the rate of fission events occurring each second in a reactor whose thermal
  power output is \SI{32}{\giga\watt}.
  \vspace{\stretch1}
  \newpage
  
  \question How much \ce{^{235}U} would be needed to fuel the reactor from the
  previous question for 1 year? (Note that this is an \emph{overestimate}
  because the fission of \ce{^{239}Pu} also contributes to the power output.)
  \vspace{\stretch1}
  
  \question The following shows the products of a uranium-238 decay series:

  \ce{^{238}_{92}U -> ^{234}_{90}Th -> ^{234}_{91}Pa -> ^{234}_{92}U -> ^{230}_{90}Th
  ->^{226}_{88}Ra}
  \begin{parts}
    \part Write a nuclear reaction equation for each stage of this series.
    Assume that beta decay reactions are beta-negative.
    \part Identify each reaction by type of decay. Explain your answers.
  \end{parts}
  \vspace{2.1in}
  
  \question Consider the carbon-nitrogen-oxygen cycle:

  \ce{^{12}_6C -> ^{13}_7N -> ^{13}_6C -> ^{14}_7N -> ^{15}_8O -> ^{15}_7N
    -> ^{12}_6C + ^4_2He}

  The first two reaction equations of this cycle are

  \ce{^{12}_6C + ^1_1H -> ^{13}_7N + energy}
  
  \ce{^13_7N -> ^{13}_6C + ^0_{+1}e + energy}
  \begin{parts}
    \part Which of these is a fusion reaction? Explain.
    \part Which of these is a beta decay reaction? What type of beta decay is
    it?
    \part Write the remaining reaction equations for this cycle and classify
    each by type of nuclear reaction. Assume that the beta decay reactions are
    of the same type as the one given above.
  \end{parts}
  \newpage

  \question Use the following values to answer (a) and (b):
  \begin{align*}
    m_{\text{H-1}} &=\SI{1.007825}u\\
    m_{\text{C-12}} &=\SI{12.000 00}u\\
    m_{\text{C-13}} &=\SI{13.003 35}u\\
    m_{\text{N-14}} &=\SI{14.003 07}u
  \end{align*}
  \begin{parts}
    \part Determine the amount of energy released in the third stage of the
    carbon-nitrogen-oxygen cycle:

    \ce{^{13}_6C + ^1_1H -> ^{14}_7N + energy}

    \part In the first stage of the carbon-nitrogen-oxygen cycle,
    \SI{1.95}{\mega\electronvolt} is produced per reaction:

    \ce{^{12}_6C + ^1_1H -> ^{13}_7N + energy}

    Use this information to determine the mass of nitrogen-13.
  \end{parts}
  \vspace{\stretch{1.5}}
  
  \question Neutron-induced fission of \ce{^{235}U} result in fission products
  iodine-139 and yttrium-95. How many neutrons are released? Include the
  reaction equation.
  \vspace{\stretch1}
  \newpage
  
  \question Most of an atom consists of
  \begin{choices}
    \choice empty space
    \choice electrons
    \choice protons
    \choice gravity
  \end{choices}
  
  \question The nucleus of an atom consists of which of the following particles?
  \begin{choices}
    \choice protons and electrons
    \choice protons and neutrons
    \choice neutrons and electrons
    \choice protons, neutrons, and electrons
  \end{choices}

  %\question Atoms that have electrons in their lowest energy levels are said to
  %be in their
  %\begin{choices}
  %  \choice normal state
  %  \choice level state
  %  \choice excited state
  %  \choice ground state
  %\end{choices}

  %\question When a uranium-238 atom undergoes alpha decay, what is the atomic
  %number of the new isotope?
  %\begin{choices}
  %  \choice 92
  %  \choice 90
  %  \choice 234
  %  \choice 242
  %\end{choices}

  \question What does an unified atomic mass unit represent?
  \begin{choices}
    \choice of the mass of an isolated hydrogen atom
    \choice the mass of a proton
    \choice $\dfrac1{12}$ of the mass of an isolated carbon-12 atom
    \choice the mass of a neutron
  \end{choices}
  
  \question A reaction in which the nucleus of an atom is split into smaller
  pieces is known as
  \begin{choices}
    \choice nuclear reduction
    \choice nuclear fusion
    \choice nuclear fission
    \choice atomic separation
  \end{choices}

  \question The half-life of radon-222 is \SI{3.82}{days}. How much of
  \SI{100}{\gram} sample of radon will be left after one week?
  \begin{choices}
    \choice\SI{28.1}\gram
    \choice\SI{83.4}\gram
    \choice\SI{50.0}\gram
    \choice\SI{32.2}\gram
  \end{choices}
  
  \question The amount of fuel necessary to establish a chain reaction is
  called the
  \begin{choices}
    \choice reaction mass
    \choice nuclear mass
    \choice fissile material
    \choice critical mass
  \end{choices}
\end{questions}
\end{document}
