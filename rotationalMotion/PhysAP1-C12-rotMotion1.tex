%\documentclass[12pt,compress,aspectratio=169]{beamer}
%\input{../mybeamer}
%\tikzstyle{every node}=[font=\footnotesize]
%
%\usetikzlibrary{patterns}
%
\chapter{Rigid-Body Rotational Motion}
%\subtitle{AP Physics 1}
%\input{../me}
%\input{../mycommands}

\section*{Before We Start}
Make sure that you have reviewed the following
\begin{itemize}
\item Kinematics and dynamics of uniform circular motion
\item Laws of motion 
\item Mechanical work and work-energy theorem
\end{itemize}


\section{Introduction}

Consider the uniform circular motion of an object, shown in
Fig.~\ref{fig:uniform-circ-motion}.
\begin{figure}[ht]
  \centering
  \begin{tikzpicture}[scale=.65]
    \draw[axes] (-3,0)--(3,0);
    \draw[axes] (0,-3)--(0,3);
    \draw[dashed] circle(2.5);
    \begin{scope}[rotate=38]
      \draw[vectors,blue] (2.5,0)--(2.5,1.5) node[above]{$v$};
      \draw[vectors,red] (2.5,0)--(1,0) node[left]{$F_c$};
      \draw[vectors] (2.5,0) arc (0:30:2.5) node[above left]{$\omega$};
      \draw[mass] (2.5,0) circle (.1);
    \end{scope}
  \end{tikzpicture}
  \caption{An object in uniform circular motion.}
  \label{fig:uniform-circ-motion}
\end{figure}
The object is acted on by a centripetal force $\bm F_c$, which is the net force
for a uniform circular motion. As centripetal foce is perpendicular to the
direction of motion, it does not do any mechanical work. Therefore, the object
moves with constant angular velocity $\omega$ and speed $v=\omega R$. We can
say that the \emph{rotational state} of the object does not change.
Unless/until the centripetal force changes in magnitude or direction, the
object will move in uniform circular motion forever. The bottom line is that
\textbf{the rotational motion of an object is not determined \emph{only} by
  what forces are acting it.}


\section{Torque}

%{Turning A Wrench} 

%    \pic1{wrench}
When tightening/loosening a nut by turning a wrench, we effectively change
its ``rotational state'' changes (i.e.\ it goee from not moving to moving).
Our experience already tells that in order to make the nut more, the applied
force must be applied at some distance away from the nut. How easy to turn the
nut depends on \emph{both} the distance as well as the amount of force.

\textbf{Torque}\footnote{Also known as the \textbf{moment of force}, or just
\textbf{moment}} is the tendency for a force to change the rotational motion
of a body.
%  \begin{itemize}
%  \item A force $\bm F$ acts at a point some position $\bm r$ from a
%    \textbf{pivot}, or \textbf{fulcrum}
%  \item There is an angle $\phi$ between $\bm F$ and $\bm r$
%  \item Example: the force to twist a screw
%  %\item In the example below, a force $\bm F$ is applied $\bm r$ away from
%  %  the pivot at an angle $\phi$. This generates a torque around the pivot.
%  \end{itemize}
The SI unit for torque is a \textbf{newton metre} (\si{\newton\metre}).
%  \begin{center}
%    \begin{tikzpicture}[scale=1.2]
%      \draw[thick,fill=lightgray] (-.13,-.13) rectangle (5.13,.13);
%      \draw[vectors,red] (0,0)--(5,0) node[midway,below]{\normalsize$\bm r$};
%      \draw[vectors,blue] (5,0)--(5.7,-1.5) node[right]{\normalsize$\bm F$};
%      \draw[dashed] (5,0)--(6,0);
%      \draw[axes] (5.7,0) arc(0:-atan(1.5/.7):.7)
%      node[midway,right]{\normalsize$\phi$};
%      \fill circle (.06) node[above=1]{\scriptsize\textbf{pivot}};
%    \end{tikzpicture}
%  \end{center}

We can express torque $\tau$ in terms of the force $\bm F$, the position
vector $\bm r$ and the angle $\phi$ between $\bm F$ and $\bm r$:
\begin{equation}
  \boxed{
    \tau=rF_\perp=rF\sin\phi
  }
\end{equation}
The subscript $\perp$ means that it is the force component that is
perpendicular to $\bm r$ that generates a torque; the parallel component does
not generate a torque.
%  \begin{center}
%    \begin{tabular}{l|c|c}
%      \rowcolor{pink}
%      \textbf{Quantity} & \textbf{Symbol} & \textbf{SI Unit} \\ \hline
%      Torque       & $\tau$ & \si{\newton\metre} \\
%      Force vector & $F$    & \si\newton \\
%      Distance from pivot to force & $r$    & \si\metre \\
%      Angle between force and position vectors & $\phi$ & radian
%    \end{tabular}
%  \end{center}

Let's consider cases where a force acts at a point away from the pivot to
generate a torque. In Fig.\ XXX, force is perpendicular to the position vector,
i.e.\ $\bm r\perp\bm F$. In this case, the torque is easy to calculate:
\begin{equation*}
  \tau = rF
\end{equation*}
The position vector $\bm r$ is known as the \textbf{moment arm}. In
Fig.\ XXX, force is applied at an angle $\phi$ to the position vector. Only the
perpendicular component ($F_\perp$) generates a torque; the parallel component
($F_\parallel$) does not. Therefore, the torque generated by the torque is:
\begin{equation*}
  \boxed{
    \tau = r{\color{blue}F_\perp}=r{\color{blue}F\sin\phi}
  }
\end{equation*}
Alternatively, we can also use the parallel component of the position vector,
i.e.\ $\bm r_\parallel$, as shown in Fig.\ XXX. In this case, $\bm F$ can be
moved along its line of action until it intersects the perpendicular component
of $\bm r$. The torque
generated is given by:
\begin{equation*}
  \boxed{
    \tau = {\color{red}r_\perp}F
    ={\color{red}r}F{\color{red}\sin\phi}
  }
\end{equation*}
The moment arm is $\bm r_\perp$.


\begin{figure}[ht]
  \centering
  \begin{subfigure}{.32\textwidth}
    \begin{tikzpicture}[scale=.6]
      \draw[thick,fill=lightgray] (-.2,-.2) rectangle (5.2,.2);
      \draw[vectors,red] (0,0)--(5,0) node[midway,below]{$\bm r$};
      \draw[vectors,blue] (5,0)--(5,2) node[above]{$\bm F$};
      \fill circle (.08) node[above]{\scriptsize pivot};
    \end{tikzpicture}
  \end{subfigure}
  \hspace{\stretch1}
  \begin{subfigure}{.32\textwidth}
    \centering
    \begin{tikzpicture}[scale=.6]
      \draw[thick,fill=lightgray](-.2,-.2) rectangle(5.2,.2);
      \begin{scope}[vectors]
        \draw[red] (0,0)--(5,0) node[midway,below]{$\bm r$};
        \draw[blue] (5,0)--(6.2,1.6) node[above]{$\bm F$};
        \draw[blue,dashed] (5,0)--(5,1.6) node[left]{$\bm F_\perp$};
        \draw[blue,dashed] (5,0)--(6.2,0) node[right]{$\bm F_\parallel$};
      \end{scope}
      \fill circle (.08) node[above]{\scriptsize pivot};
      \draw (5.6,0) arc(0:atan(4/3):.6) node[pos=2/3,right=0]{$\phi$};
    \end{tikzpicture}
  \end{subfigure}
  \hspace{\stretch1}
  \begin{subfigure}{.32\textwidth}
    \centering
    \begin{tikzpicture}[scale=.6]
      \draw[thick,fill=lightgray] (-.2,-.2) rectangle(5.2,.2);
      \draw[vectors,red] (0,0)--(5,0) node[midway,below]{$\bm r$};
      \draw[vectors,blue] (5,0)--(6.2,1.6) node[above]{$\bm F$};
      \begin{scope}[rotate=-atan(3/4)]
        \draw[vectors,red] (0,0)--(4,0) node[midway,below]{$\bm r_\perp$};
        \draw (4,-.5)--(4,5.5)
        node[pos=0,right]{\scriptsize line of action of $\bm F$};
        \draw (0,-1)--(0,2);
        \draw[vectors,blue,dash dot] (4,0)--(4,2) node[midway,right]{$\bm F$};
        \fill[blue] (4,0) circle (.04);
      \end{scope}
      \fill circle (.08) node[above]{\scriptsize pivot};
      \draw (5,0)--(6.3,0);
      \draw (5.6,0) arc(0:atan(4/3):.6) node[pos=2/3,right]{$\phi$};
      \draw (.6,0) arc(0:atan(4/3):.6) node[pos=2/3,right]{$\phi$};
    \end{tikzpicture}
  \end{subfigure}
\end{figure}
%
%
%
%{\emph{NOT} Generating a Torque}
No torque is generated if the line of action of the force goes through the
pivot. For example, in Fig.\ XXX, force is parallel to the position vector
$\bm r$ (i.e.\ $\theta=0$, therefore $\sin\theta=0$, resulting in $\tau=0$).
In Fig.\ XXX, the force $\bm F$ is applied at the pivot (i.e.\ $r=0$, and
therefore $\tau=0$).
\begin{figure}[ht]
  \centering
  \begin{subfigure}{.4\textwidth}
    \centering
    \begin{tikzpicture}[scale=.7]
      \draw[thick,fill=lightgray] (-.2,-.2) rectangle(5.2,.2);
      \draw[vectors,red] (0,0)--(5,0) node[midway,below]{$\bm r$};
      \draw[vectors,blue] (5,0)--(7,0) node[above]{$\bm F$};
      \draw (-1,0)--(9,0) node[above]{\scriptsize line of action};
      \fill circle (.08) node[above]{\scriptsize pivot};
      \fill[blue] (5,0) circle (.04);
    \end{tikzpicture}
  \end{subfigure}
  \begin{subfigure}{.4\textwidth}
    \centering
    \begin{tikzpicture}[scale=.7]
      \draw[thick,fill=lightgray] (-.2,-.2) rectangle(5.2,.2);
      \begin{scope}[rotate=-atan(3/4)]
        \draw (0,-1)--(0,3) node[right]{\scriptsize line of action};
        \draw[vectors,blue] (0,0)--(0,2) node[right]{$\bm F$};
      \end{scope}
      \fill circle (.08) node[above]{\scriptsize pivot};
    \end{tikzpicture}
  \end{subfigure}
\end{figure}

Net torque is the sum of all torques acting at a point. Keep in mind the sign
convention as torque is a vector:
\begin{equation}
  \boxed{
    \tau_\text{net}=\sum_{i=1}^N\tau_i=\tau_1 + \tau_2 +\cdots + \tau_N
  }
\end{equation}

%  \item A torque can be generated by a single force acting at a point, i.e.
%
%    \eq{-.15in}{
%      \tau_i = F_ir_i\sin\phi
%    }
%
%  \item\vspace{-.15in} \textbf{BUT}, most often, the forces that generate
%    torques are spread over an area, e.g.\ the normal force the head of a
%    screwdriver exerts on the screw
%  \item For rotational motion, we only care about the torque; in most cases, we
%    don't have to worry about exactly how the force generates the torque
%  \end{itemize}
%
%
\begin{example}
  Find the net torque on point C.
%  \begin{center}
%    \begin{tikzpicture}[scale=2.2]
%      \draw[thick,fill=blue!40] (-1.65,-.12) rectangle (1.65,.12);
%      \begin{scope}[thick,|<->|]
%        \draw (-1.5,-.27)--(1.5,-.27) node[midway,fill=black!2]{\SI{3.0}\metre};
%        \draw (-1.5,.26)--(0,.26) node[midway,fill=black!2]{\SI{1.5}\metre};
%      \end{scope}
%      \draw[dashed,thick] (-2.2,0)--(2.2,0);
%      \draw[dashed,thick] (0,0)--(0,.5);
%      \begin{scope}[vectors,orange]
%        \draw[rotate=-45] (0,0)--(0,1.5) node[right]{\SI{30}\newton};
%        \draw[rotate around={30:(-1.5,0)}] (-1.5,0)--(-2.5,0)
%        node[left]{\SI{20}\newton};
%        \draw[rotate around={-30:(1.5,0)}] (1.5,0)--(2.2,0)
%        node[right]{\SI{10}\newton};
%      \end{scope}
%      \begin{scope}[axes]
%        \draw (0,.4) arc (90:45:.4) node[pos=.7,above]{\ang{45}};
%        \draw (-1.9,0) arc (180:210:.4) node[midway,left] {\ang{30}};
%        \draw (1.9,0) arc (0:-30:.4) node[midway,right]{\ang{30}};
%      \end{scope}
%      \fill (-1.5,0) circle (.1) node[white]{\textbf{A}};
%      \fill circle (.1) node[white]{\textbf{B}};
%      \fill (1.5,0) circle (.1) node[white]{\textbf{C}};
%    \end{tikzpicture}
%  \end{center}
\end{example}




\section{Angular Momentum}

Consider a mass $m$ connected to a massless beam rotates with speed $v$ at a
distance $r$ from the center (shown on the right). It has an
\textbf{angular momentum} ($\bm L$). Expressed as a 1D vector, it is defined
as:
%  
%    \column{.77\textwidth}
    
\begin{equation}
  \boxed{
    L=rp_\perp=r(mv_\perp)=mr^2\omega
  }
\end{equation}
%    \begin{itemize}
%    \item The unit for angular momentum is a \textbf{kilogram metre squared per
%      second} (\si{\kilo\gram\metre\squared\per\second})
%    \item Angular momentum is a vector, with the $(+)$ direction
%      counterclockwise, and $(-)$ direction being clockwise.
%    \end{itemize}
%    
\begin{figure}[ht]
  \centering
  \begin{tikzpicture}[scale=2.5,rotate=70]
    \draw[line width=3] (0,0)--(2,0);
    \draw[mass] (2,0) circle (.05) node[right=1.5]{$m$};
    \fill circle (.04) node[right]{\scriptsize pivot};
      \draw[vectors,red] (0,0)--(2,0) node[midway,right]{$r$};
      \draw[vectors,violet] (2,0)--(2,.5) node[left]{$\bm v$};
  \end{tikzpicture}
\end{figure}



\section{Moment of Inertia}

Look again at the definition of angular momentum:

%  \eq{-.1in}{
%    L=\underbracket[1.3pt]{mr^2}_I\omega
%  }
    
The quantity $I=mr^2$ is called the \textbf{moment of inertia}, or
\textbf{rotational inertia}, with an SI unit of \textbf{kilogram metre
  squared} (\si{\kilo\gram\metre\squared}), and 

%  \eq{-.1in}{
%    \boxed{L=I\omega}
%  }

Moment of inertia can be considered to be an object's rotational equivalent of
mass. For a \emph{single particle} of $m$ rotating at a distance $r$ from the
pivot:
\begin{equation}
  \boxed{I=mr^2}
\end{equation}
For a \emph{collection of particles}, all rotating at the same angular
velocity $\omega$ about the pivot, each of mass $m_i$ at distance $r_i$ from
the pivot:
\begin{equation}
  \boxed{I=\sum m_ir_i^2}
\end{equation}
%
%  Note that while masses $m_i$ cannot change, its distance to the center of
%  rotation can
%  \begin{itemize}
%  \item e.g.\ When a figure skater starts to spin and draws their arms inward,
%    moment of inertia decreases.
%  \end{itemize}
%
%
%
%
%{Moment of Inertia}
%  \centering
%  \pic{.6}{mic}
%
%  In AP Physics 1, you are not required to remember these; they will be given
%  to you
%
%
%
%
%{Angular Momentum and Moment of Inertia}
%  Linear momentum $\bm p$ and angular momentum $L$ have very similar
%  expressions
%  
%  \eq{-.1in}{
%    \boxed{\bm p=m\bm v}\quad\quad\quad\boxed{L=I\omega}
%  }

Just as $\bm p$ describes the overall \emph{translational state} of a
physical system, $L$ describes its overall \emph{rotational state}




\section{Laws of Motion}

\subsection{First Law of Motion}
As we have already learned in Chapters~\ref{chapter:dynamics} and
\ref{chapter:momentum}, an object is in \textbf{translational equilibrium} is
when the net force acting on it is zero:
\begin{equation*}
  \bm F_\text{net}=\bm 0
\end{equation*}
The equilibrium condition does \emph{not} mean that the object has no
translational motion; it just means that the object's overall
\emph{translational state} is not changing (i.e.\ the translational momentum
$\bm p=m\bm v$ is constant). For objects with constant mass $m$, this means
that the object has a constant velocity (i.e.\ $\bm v=\text{constant}$) and
no acceleration (i.e.\ $\bm a=\bm 0$).

Likewise, an object is in \textbf{rotational equilibrium} when the net torque
acting on it is zero:
\begin{equation}
  \boxed{
    \tau_\text{net}=0
  }
\end{equation}
This does \emph{not} mean that the object has no rotational motion; it just
means that the object's overall \emph{rotational state} is not changing (i.e.\
angular momentum $L=I\omega$ is constant). For objects with constant moment of
inertia $I$, the object has a constant angular velocity (i.e.\
$\omega=\text{constant}$) and no angular acceleration (i.e.\ $\alpha=0$).




\subsection{Second Law of Motion}
The average net torque is the change of angular momentum over a finite time
interval:
%  \eq{-.1in}{
%    \tau_\text{net}=rF_\text{net}
%    =r\frac{\Delta p}{\Delta t}
%    =\frac{\Delta(rp)}{\Delta t}\;\;\longrightarrow\;\;
%    \boxed{
%      \tau=\frac{\Delta L}{\Delta t}
%    }
%  }
%  \begin{itemize}
%  \item If the net torque on a system is zero, then the rate of change
%    of angular momentum is zero, and we say that the angular momentum is
%    conserved. 
%  \item e.g.\ When a figure skater starts to spin and draws their arms inward,
%    moment of inertia decreases. Since angular momentum is conserved, there
%    must be an increase in $\omega$.
%  \end{itemize}


Again, as we have learned in Chapter~\ref{chapter:momentum}, for translational
motion, the net force $\bm F_\text{net}$ is rate of change of the object's
momentum $\bm p$:
\begin{equation*}
  \bm F_\text{net}=\frac{\Delta\bm p}{\Delta t}
\end{equation*}
For objects with constant mass $m$, this reduces to the very familiar
equation that we learned in Chapter~\ref{chapter:dynamics}:
\begin{equation*}
  \bm F_\text{net}=m\bm a
\end{equation*}
Likewise, for rotational motion, net torque $\tau_\text{net}$ is the rate of
change of angular momentum $L$:
\begin{equation}
  \boxed{
    \tau_\text{net}=\frac{\Delta L}{\Delta t}
  }
\end{equation}
For objects with constant moment of inertia $I$, this reduces to:
\begin{equation}
  \boxed{
    \tau_\text{net}=I\alpha
  }
\end{equation}



\section{Angular Momentum Conservation in Collisions}

\subsection{Angular Momentum in Translational Motion}
Even when there is no apparent rotational motion, it does not necessarily
mean that angular momentum is zero! In this case, mass $m$ travels along a
straight path at constant velocity (uniform motion), but the angular momentum
around point $P$ is not zero:
\begin{figure}[ht]
  \centering
  \begin{tikzpicture}
    \draw[dashed](-5,0)--(5,0);
    \draw[vectors,red] (-5,0)--(-3,0) node[below]{$\bm v$};
    \shade[ball color=red] (-5,0) circle (.2) node[above left,red]{$m$};
    \draw[vectors,blue] (0,-2)--(-5,0) node[midway,below left]{$\bm r$};
    
    \draw circle (.2) node[above left]{$m$};
    \draw[vectors,red!50] (.2,0)--(2,0) node[right]{$\bm v$};
    \draw[vectors,blue!50] (0,-2)--(0,0) node[midway,left]{$\bm r$};
      
    \draw (4,0) circle (.2) node[above left]{$m$};
    \draw[vectors,red!50] (4.2,0)--(6,0) node[right]{$\bm v$};
    \draw[vectors,blue!50] (0,-2)--(4,0) node[midway,above left]{$\bm r$};

    \fill (0,-2) circle (.05) node[below]{$P$};
  \end{tikzpicture}
\end{figure}
Since there is no force and no torque acting on the object, both the linear
momentum ($\bm p=m\bm v$) and angular momentum ($L=rp_\perp$) are constant.




%{Example Problem}
%  \textbf{Example:} A very slender skater extends her
%  arms\footnote{Both arms!}, holding a \SI{2.0}{\kilo\gram} mass in each hand.
%  She is rotating about a vertical axis at a given rate. She brings her arms
%  inward toward her body in such a way that the distance of each mass from the
%  axis changes from \SI{1.0}{\metre} to \SI{.50}\metre. Her rate of rotation
%  (neglecting her own mass) will?
%
%
%
%
%
%{Example Problem}
%  \textbf{Example:} A \SI{1.0}{\kilo\gram} mass swings in a vertical circle
%  after having been released from a horizontal position with zero initial
%  velocity. The mass is attached to a massless rigid rod of length
%  \SI{1.5}\metre. What is the angular momentum of the mass, when it is in its
%  lowest position?
%
%
%
%
%
%%{Solving Rotational Problems}
%%  When solving for rotational problems like the ones described in the previous
%%  sections:
%%  \begin{itemize}
%%  \item Draw a free-body diagram to account for all forces
%%  \item The direction of friction force is not always obvious
%%  \item The magnitude of any static friction force cannot be assumed to be at
%%    maximum.
%%  \item If the object is to change its rotational state, there must be a net
%%    torque causing it.
%%  \end{itemize}
%%
%
%
%
%%{Solving Rotational Problems}
%%  Once the free-body diagram is complete
%%  \begin{itemize}
%%  \item Breaks down the \emph{forces} into $\xx$, $\yy$ and $\zz$ components
%%  \item We have now three equations for translation, but it is likely that only
%%    \emph{one} direction will have forces:
%%
%%    \eq{-.3in}{
%%      \sum F_x=ma_x\quad\quad\sum F_y=ma_y\quad\quad\sum F_z=ma_z
%%    }
%%  \item And three equations for rotation, and torque is only applied in one
%%    direction (likely $\zz$):
%%    
%%    \eq{-.3in}{
%%      \sum\tau_x=I_x\alpha_x\quad\quad \sum\tau_y=I_y\alpha_y\quad\quad 
%%      \sum\tau_z=I_z\alpha_z
%%    }
%%  \end{itemize}
%%
%
%
%
%%{Solving Rotational Problems}
%%  For rotational motion dynamics equation:
%%  \begin{enumerate}
%%  \item Relate the force(s) that causes rotational motion to the net torque
%%
%%    \eq{-.1in}{
%%      \tau=Fr
%%    }
%%  \item Substitute the expression for moment of inertia (which has both mass
%%    and radius terms in it) into the equation for rotational motion
%%  \item Relate angular acceleration to linear acceleration, if applicable:
%%
%%    \eq{-.1in}{
%%      \alpha=\frac aR
%%    }
%%  \end{enumerate}
%%  Now there are two equations with force and acceleration terms. See handout
%%
%
%
%  
\section{Work \& Energy in Rotational Motion}

\subsection{Rotational Work}
For translational motion, the \emph{translational} work ($W_t$) done by a
constant force $\bm F$ displacing an object by $\Delta\bm r$ is given by:
\begin{equation*}
  W_t = \bm F\cdot \Delta\bm r = F\Delta r\cos\theta
\end{equation*}
For rotational motion, the \emph{rotational} work done by a constant
\emph{torque} $\tau$ displacing an object by an \emph{angular displacement}
of $\Delta\theta$ is given by:
\begin{equation}
  \boxed{
    W_r = \tau\Delta\theta
  }
\end{equation}
Note that $\Delta\theta$ is measured in \textbf{radians}.



\subsection{Rotational Kinetic Energy}

When the object is being constant net torque $\tau_\text{net}$, the crate 
rotates a constant angular acceleration $\alpha$, changing the angular velocity
from $\omega_i$ to $\omega_2$:
\begin{equation}
  \tau_\text{net}=I\alpha\quad\rightarrow\quad a=\frac{F_\text{net}}m
\end{equation}

Substituting the expression of acceleration into the kinematics equations,
and solving for the net work term $F_\text{net}\Delta d=W_\text{net}$:
\begin{align*}
  \omega_f^2 &=\omega_i^2+2{\color{blue}\alpha}\Delta\theta
  \quad\rightarrow\quad
  \omega_f^2 =\omega_f^2+2\left({\color{blue}\frac{\tau_\text{net}}I}\right)
  \Delta\theta\\
  \underbracket[1pt]{\tau_\text{net}\Delta\theta}_{W_\text{net}}
  &=\frac12I\left(\omega_f^2-\omega_i^2\right)
\end{align*}

When a net torque does work on an object, its angular speed $\omega$ changes,
and the work done is equal to a change in a motion quantity ``$K_r$'':
\begin{equation}
  W_\text{net} =
  \underbracket[1pt]{\frac12I\omega_f^2}_{\text{final }K_r} -
  \underbracket[1pt]{\frac12I\omega_i^2}_{\text{init. }K_r} = \Delta K_r
\end{equation}
This quantity $K$ is called the \textbf{rotational kinetic energy}, defined as:
\begin{equation}
  \boxed{
    K=\frac12I\omega^2
  }
\end{equation}

We can also try this, by using the definition of translational kinetic
energy to derive the rotational kinetic energy:
%  To find the kinetic energy of a rotating system of particles (discrete number
%  of particles, or continuous mass distribution), we sum the
%  kinetic energy of the individual particles:
\begin{equation}
  K=\sum_i\frac12m_iv_i^2=\frac12\left[\sum_i m_ir_i^2\right]\omega^2
  \quad\longrightarrow\quad
  \boxed{K=\frac12I\omega^2}
\end{equation}

The \textbf{work-energy theorem} for rotational work shows that the net
rotational work done on an object is \emph{always} equal to the change in its
rotational kinetic energy:
\begin{equation} 
  \boxed{W_\text{net} = \Delta K_r}
\end{equation}
As was the case for translational work, positive net work ($W_\text{net}>0$)
\emph{increases} kinetic energy (i.e.\ $\Delta K>0$) and the object's rotation
speeds up; while negative net work ($W_\text{net}<0$) \emph{decreases} kinetic
energy (i.e.\ $\Delta K<0$) and the rotational slows down.
%\item If net force is not doing work ($W_\text{net}=0$), then there is no
%    change in kinetic energy
Like for translational motion, the above equation applies even when the net
force is not constant, and reardless of what $\vec F_\text{net}$ consists of.


\subsection{Total Kinetic Energy of a Rotating System}

The total kinetic energy of a rotating system is the sum of its translational
and rotational kinetic energies at its centre of mass:
\begin{equation}
  \boxed{
    K=\frac12mv_\text{cm}^2+\frac12I_\text{cm}\omega^2
  }
\end{equation}  
In this case, $I_\text{cm}$ is evalulated at the centre of mass. For simple
problems, we only need to compute rotational kinetic energy at the pivot:
\begin{equation}
  \boxed{
    K=\frac12I_p\omega^2
  }
\end{equation}
In this case, the $I_p$ is calculated at the pivot. It is important to note
that, in general,
\begin{equation*}
  I_\text{cm}\neq I_p
\end{equation*}


%{Spinning Disk}
%  \begin{tikzpicture}
%    \draw[mass] circle (2);
%    \draw[thick] circle (.13);
%    \fill (0,0)--(.13,0) arc (0:90:.13) node[above=-2]{\scriptsize CM/pivot}
%    --(0,0);
%    \fill (0,0)--(-.13,0) arc (180:270:.13)--(0,0);
%    \draw[axes,rotate=-30] (1,0) arc (0:60:1) node[above]{$\omega$, $\alpha$};
%    \draw[axes,rotate=40] (0,0)--(-2,0) node[midway,right]{$R$};
%    \node at (0,-2.5) {Mass $M$};
%    \node at (0,-3.3) {$I=\dfrac12MR^2$};
%      %\only<2>{
%      %  \begin{scope}[violet]
%      %    \draw[thick,dash dot] (0,-.13)--(0,-2);
%      %    \foreach \y in {-.4,-.8,...,-2.1}\draw[vectors] (0,\y)--(-.8*\y,\y);
%      %    \node[right] at (1.6,-2) {$\bm v=\omega R$};
%      %  \end{scope}
%      %}
%  \end{tikzpicture}
%
%%    \only<2>{Speed at any point on this disk is proportional to its distance
%%      to the pivot:
%%
%%      \eq{-.1in}{
%%        v=\omega r \quad\text{for}\quad 0\leq r\leq R
%%      }
%%    }
%%    \only<3>{When it is spinning, it has an angular momentum of
%%
%%      \eq{-.1in}{
%%        L = I\omega
%%      }
%%
%%      and a rotational kinetic energy:
%%      
%%      \eq{-.1in}{
%%        K_r = \frac12I\omega^2
%%      }
%%    }
%%    
%%    \item Any angular acceleration that it may have will be due to the net
%%      torque at the pivot/CM:
%%      \begin{displaymath}
%%        \tau_\text{net}=\frac{\Delta L}{\Delta t}=I\alpha
%%      \end{displaymath}
%%    \end{itemize}


%{A Rotating Rod}
%%  
%%    \column{.3\textwidth}
%%    \centering
%  \begin{tikzpicture}
%    \draw[mass] (-.15,0) rectangle (.15,-5);
%    \draw[thick] (0,-2.5) circle (.13);
%    \fill (0,-2.5)--(.13,-2.5) node[right=-1]{\scriptsize CM} arc (0:90:.13)
%    --(0,-2.5);
%    \fill (0,-2.5)--(-.13,-2.5) arc (180:270:.13)--(0,-2.5);
%    \fill circle (.07) node[above]{\scriptsize pivot};
%    \draw[axes,rotate=-90] (.7,0) arc (0:180:.7)
%    node[left]{$\omega$, $\alpha$};
%    \draw[thick,|<->|] (-.5,0)--(-.5,-5) node[midway,fill=black!2]{$L$};
%%      \node at (0,-2.5) {Mass $M$};
%%      \node at (0,-3.3) {$I=\dfrac12MR^2$};
%%      \only<2>{
%%        \begin{scope}[violet]
%%          \draw[thick,dash dot] (0,-.13)--(0,-2);
%%          \foreach \y in {-.4,-.8,...,-2.1}\draw[vectors] (0,\y)--(-.8*\y,\y);
%%          \node[right] at (1.6,-2) {$\bm v=\omega R$};
%%        \end{scope}
%%      }
%  \end{tikzpicture}

