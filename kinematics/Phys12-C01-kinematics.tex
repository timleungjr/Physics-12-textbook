%\documentclass[12pt,compress,aspectratio=169]{beamer}
%\input{../mybeamer}
%
\chapter{Kinematics}
\label{chapter:kinematics}
%\subtitle{Unit 1: Fundamentals of Dynamics}
%\input{../term}
%\input{../mycommands}
%
%\begin{document}
%
%
%\section{Kinematics}
%  \textbf{Kinematics} describes the motion of points, bodies (objects), and
%  systems of bodies (groups of objects). It is the mathematical relationship
%  between



%\begin{frame}{Course Outline}
%  The course is divided into five major units. The first unit will take up
%  Classes 1 to 3.
%  \begin{enumerate}
%  \item<alert@1>Fundamentals of Dynamics
%  \item Energy and Momentum
%  \item Gravitational, Electric and Magnetic Fields
%  \item Wave Nature of Light
%  \item Modern Physics
%  \end{enumerate}
%
%
%
%
%\section{Files to Download from School Website}
%  There are many handouts and slides for download on the first day. Please
%  download them from the school website if you have not already done so.
%  \begin{itemize}
%  \item\texttt{Phys12-C01-courseOutline.pdf}
%  \item\texttt{Phys12-C01-equationSheet.pdf}
%  \item\texttt{Phys12-C01-kinematics-print.pdf}
%  \item\texttt{Phys12-C01-projectileMotion.pdf}
%%  \item\texttt{Phys12-C01-g.pdf}
%%  \item\texttt{Phys12-C01-sigFigs.pdf}
%  \item\texttt{Phys12-C01-HW.pdf}
%  \end{itemize}
%  Weekly homework question are posted on the school website, and also on the
%  Classkick app online. If you wish to print the class slides for yourself, we
%  recommend printing 4 to 6 slides per page.
%
%
%
%
\section{Frame of Reference}
A \textbf{frame of reference}\footnote{Or \textbf{reference frame}, or just
\textbf{frame}} is a \emph{coordinate system} in which physical
measurements are made. In classical mechanics, a frame of reference is
\begin{itemize}
\item the $x$- $y$- and $z$-axes in the Cartesian coordinate system
\end{itemize}
Later, when we study relativity, the frame of reference must also include
\begin{itemize}
\item a time axis
\end{itemize}




%\section{Frame of Reference}
Think of the frame of reference as a ``hypothetical mobile laboratory'' an
observer uses to make any physical measurements (e.g.\ mass, lengths, time).
At a minimum,
it must include:
\begin{itemize}
\item A set of rulers (i.e.\ a coordinate system) to measure lengths
\item A clock to measure the passage of time
  %\item A scale to compare forces
  %\item A balance to measure masses
\end{itemize}
We assume that this ``hypothetical laboratory'' is \emph{perfect}:
\begin{itemize}
\item The hypothetical instruments are not subjected to numerical errors
\item There is an instrument for whatever you want to measure
\item What matters is the \emph{motion} of the frame (at rest, uniform
  motion, accelerating), and how that motion affects the measurements
\end{itemize}




\subsection{Inertial Frame of Reference}
An \textbf{inertial frame of reference}\footnote{Also known as a
\textbf{rest frame}} is one that moves in uniform motion (constant
velocity, no acceleration)
\begin{itemize}
\item The frame of reference is not subjected to any net force
\item In all inertial frames of reference, the laws of motion are valid
\item Since all laws of motion are valid in all inertial frames of reference,
  \emph{any} inertial frame can be considered to be at rest (stationary)
\end{itemize}




The priniciple of relativity states that:
\begin{definition}
  \textbf{The Principle of Relativity:} The laws of motion must be
  obeyed in all inertial frames of reference.
\end{definition}
\begin{itemize}
\item All inertial frames are equally valid when determining the laws of
  motion
\item All inertial frames can be considered at rest
\item All motion is relative: there is no \emph{absolute motion} or
    \emph{absolute rest}
\end{itemize}
%
%
%
%
%\section{Inertial Frame of Reference}
Observer A is moving with a constant velocity relative to observer B
\begin{itemize}
\item A observes that the ball that he is throwing only has vertical motion
\item B observes that the ball is moving in a parabolic curve
\item A \& B agree on the \emph{equations} that govern the motion
\end{itemize}
\begin{center}
  \pic{.65}{kinematics/graphics/57}
\end{center}
Observer A makes the same observation regardless of whether he is moving
uniformly relative to the ground or not
\begin{itemize}
\item Valid for A to think that he is at rest, but B is moving
\item Also valid for B to think that he is at rest, but A is moving
\item Both A and B are inertial frames of reference
\end{itemize}
%  \begin{center}
%    \vspace{-.15in}
%    \pic{.65}{graphics/57}
%  \end{center}
%
%
%
%
\subsection{Non-inertial Frame of Reference}
A \textbf{non-inertial frame of reference} is one that is undergoing
acceleration (i.e.\ non-constant velocity)
\begin{itemize}
\item Require a \textbf{fictitious force}\footnote{Also known as a
\textbf{pseudo force}} in the FBD to account for the observations
  \begin{itemize}
  \item Hypothetical force
  \item Does not exist in inertial frame of reference
  \end{itemize}
\item Example: A car that is speeding up, slowing down, or turning
\end{itemize}
\begin{center}
  \pic{.4}{kinematics/graphics/man-in-accel-car}
\end{center}




%\section{Example: Reference Frame}
%  \begin{columns}
%    \column{.4\textwidth}
%    \pic1{graphics/capsule2}
%    
%    \column{.6\textwidth}
%    \textbf{Example:} Passengers in a high-speed elevator feel as though they
%    are being pressed heavily against the floor when the elevator starts moving
%    up. After the elevator reaches its maximum speed, the feeling disappears.
%  \end{columns}
%
%
%
%
%\section{Example: Reference Frame}
%  \begin{enumerate}
%  \item When do the elevator and passengers form 
%    \begin{itemize}
%    \item an inertial frame of reference?
%    \item a non-inertial frame of reference?
%    \end{itemize}
%  \item Before the elevator starts moving,
%    \begin{itemize}
%    \item what forces are acting on the passengers?
%    \item how large is the external (unbalanced) force?
%    \end{itemize}
%  \item Is a person standing outside the elevator in an inertial or
%    non-inertial frame of reference?
%  \end{enumerate}
%
%
%
%
%\section{Frame of Reference}
%  \textbf{Example:} Given the definition of an inertial frame of reference,
%  is Earth an inertial frame of reference?
%
%  \vspace{.3in}\uncover<2>{
%    \textbf{Answer:} No! Earth is rotating on its axis, and also orbiting in an
%    elliptical path around the Sun. However, we can adjust the value of $g$
%    slightly to account for the acceleration of Earth.
%  }
%
%
%
%
%\section{Motion Quantities}
%
%  Kinematics does not deal with the cause of motion.



\section{Kinematic Quantities}

\begin{itemize}
\item Position
\item Displacement
\item Distance
\item Velocity
\item Speed
\item Acceleration
\end{itemize}


\subsection{Position}

\textbf{Position} ($\mathbf r$) is the location of an object in a coordinate
system. It is a vector measured from the origin to the object. If the object
moves, $\mathbf r$ is a continuous function of time $t$:
\begin{equation}
  \mathbf r=\mathbf r(t)
\end{equation}
At this moment, it is not necessary to know what kind of function $\mathbf r(t)$
is; the motion of the object will be determined by the forces that act on the
object, and therefore what acceleration it has. All that matters is that
$\mathbf r(t)$ evolves with time, the object can be at only one position at any
time (hence it is a function), and that there are no gaps in the where the
object is (hence it is a continuous function). The SI unit for position is
\emph{metre} (\si\metre).

Position is a \emph{vector}, and therefore require a magnitude and direction.
For motion in a one-dimensional coordinate system, the position of the object
is a coordinate along the number line. The vector is an arrow drawn from
the origin to object itself, as shown in Figure~\ref{fig:1d-position}. In
the figure, the position of the object is positive.

\begin{figure}[ht]
  \centering
  \begin{tikzpicture}
    \draw[axes] (-3,0)--(3,0) node[pos=0,left]{$-$} node[right]{$+$};
    %\foreach \x in {-5,...,5} \draw (\x,.1)--(\x,-.1);
    \draw[thick] (0,.1)--(0,-.1) node[below]{$O$};
    \fill[red] (2.5,0) circle (.05);% node[below]{$A$};
    \draw[vectors,red] (0,0)--(2.5,0) node[midway,above]{$\mathbf r$};
  \end{tikzpicture}
  \caption{Position in a one-dimensional coordinate system}
  \label{fig:1d-position}
\end{figure}

In a two-dimesional coordinate system (e.g.\ $xy$-plane),
%\textbf{Position in 2D Coordinate System:} For two-dimensional motion,
there are several ways to describe an object's position. One way is to use the
$x$ and $y$ coordinates. The positions of the object at $P$ and $Q$ are:
\begin{align*}
  \vec r_P & =3\hat x + 2\hat y\\
  \vec r_Q & =-4\hat x + 3\hat y
\end{align*}
or the length of the straight line from the origin to
the position, and the angle it makes with the $x$ axis.
\begin{figure}[ht]
  \centering
  \begin{tikzpicture}
    \draw[help lines] (-3.3,-3.3) grid (3.3,3.3);
    \fill[red] (3,2) circle (.05) node[above]{$P$};
    \draw[vectors,red] (0,0)--(3,2) node[midway,above]{$\mathbf r$};
    \draw[axes] (-3.5,0)--(3.5,0) node[right]{$x$};
    \draw[axes] (0,-3.5)--(0,3.5) node[above]{$y$};
    \draw[axes] (1.5,0) arc (0:atan(2/3):1.5) node[midway,right]{$\theta$};
    \node[below left] at (0,0) {$O$};
  \end{tikzpicture}
  \caption{Position in a two-dimensional coordinate system}
\end{figure}



\subsection{Displacement}
\textbf{Displacement} ($\Delta\mathbf r$) is the \emph{change in position} when
an object moves through the coordinate system. Mathematically, displacement is
defined as the \emph{difference} between the initial position
$\mathbf r_1=\mathbf r(t_1)$ when motion begins, and the current position
$\mathbf r(t)$ of the object. Therefore, as the object moves, $\Delta\mathbf r$
is also a continuous function of time, i.e.:
%\begin{equation*}
%  \Delta\mathbf r=  \Delta\mathbf r(t)
%\end{equation*}
\begin{equation}
  \boxed{
    \Delta\mathbf r(t)=\mathbf r(t)-\mathbf r_1
  }
\end{equation}
Graphically, displacement is drawn as a vector pointing from the initial
position $\mathbf r_1$ towards the current/final position $\mathbf r$.
%\begin{figure}[ht]
%  \centering
%  \begin{tikzpicture}[scale=.5]
%    \draw[axes] (0,0)--(6,0) node[right]{$x$};
%    \draw[axes] (0,0)--(0,8) node[above]{$y$};
%    \draw[vectors,red] (0,0)--(4,1) node[midway,above]{$\mathbf r_1$};
%    \draw[vectors,red] (0,0)--(2,6) node[midway,left]{$\mathbf r_2$};
%    \draw[vectors,blue] (4,1)--(2,6) node[midway,right]{$\Delta\mathbf r$};
%  \end{tikzpicture}
%\end{figure}

\begin{figure}[ht]
  \centering
  \begin{tikzpicture}[scale=.8]
    \draw[axes] (0,0)--(7,0) node[right]{$x$};
    \draw[axes] (0,0)--(0,6.5) node[above]{$y$};
    \fill (4,1) circle (.08) node[below]{1};
    \fill (2,6) circle (.08) node[left]{5};
    \draw[vectors,red] (0,0)--(4,1) node[midway,above]{$\mathbf r_1$};
    \draw[vectors,red] (0,0)--(2,6) node[midway,left] {$\mathbf r_5$};
    \begin{scope}[rotate around={33:(4,1)}]
      \draw[vector,blue] (4,1)--+(2,0) node[midway,below]{$\Delta\mathbf r_1$};
      \fill (6,1) circle (.08) node[right]{2};
    \end{scope}
    \begin{scope}[rotate around={60:(4,1)}]
      \draw[vector,blue] (4,1)--+(3.35,0) node[midway,right]{$\Delta\mathbf r_2$};
      \fill (7.35,1) circle (.08) node[above]{3};
    \end{scope}
    \begin{scope}[rotate around={85:(4,1)}]
      \draw[vector,blue] (4,1)--+(2.8,0) node[pos=.7,left=0]{$\Delta\mathbf r_3$};
      \fill (6.8,1) circle (.08) node[above]{4};
    \end{scope}
    \draw[vectors,blue] (4,1)--(2,6) node[midway,left]{$\Delta\mathbf r_4$};
    %\fill (7,2) circle (.07);
    %\fill (6,4.5) circle (.07);
    %\fill[cyan] (3,3) circle (.07);
    %\fill[cyan] (4,6) circle (.07);
    \draw[very thick,dashed,->] (4,1) ..controls (7,2) and (6,4.5).. (5,4)
    ..controls (3,3) and (4,6).. (2,6) node[midway,right]{$\mathcal C$};
  \end{tikzpicture}
  \caption{The displacement of an object evolves with time as it moves.}
\end{figure}


The SI unit for displacement is also \textbf{metre} (\si\metre).



\textbf{Explain why this is a subtraction}


\subsection{Distance}

\textbf{Distance} ($s$) is a quantity that is \emph{similar} to
displacement. It is the length of the path $\mathcal C$ taken as an object
moves from initial position ($\mathbf r_1$) to its current/final position
($\mathbf r(t)$). Unlike displacement, distance is a \emph{length}, and therefore
it is \emph{scalar} quantity. For the same reason, distance is non-negative
(i.e.\ $s\geq 0$). As the object moves, distance is always increasing. Since
the length changes with time, distance is also a continuous function of time:
\begin{equation*}
  \boxed{s=s(t)}
\end{equation*}
%\item Depends on \emph{how} the object travels from $\mathbf r_1$ to
%  $\mathbf r_2$
Although the magnitude of the displacement vector $|\Delta\mathbf r|$ is also a
scalar, it is \emph{not} necessarily the same as distance. In classical
physics,  $s\geq |\Delta\mathbf r|$

%\begin{figure}
%  \centering
%  \begin{tikzpicture}[scale=.5]
%    \draw[axes] (0,0)--(6,0) node[right]{$x$};
%    \draw[axes] (0,0)--(0,8) node[above]{$y$};
%    \draw[vectors,red] (0,0)--(4,1) node[midway,above]{$\mathbf r_1$};
%    \draw[vectors,red] (0,0)--(2,6) node[midway,left] {$\mathbf r_2$};
%    \draw[vectors,blue] (4,1)--(2,6) node[midway,right]{$\Delta\mathbf r$};
%    \draw[very thick,dash dot] (4,1)..controls (6,5) and (5,7)..(2,6)
%    node[midway,right]{$s$};
%  \end{tikzpicture}
%\end{figure}




\subsection{Velocity}

\textbf{Average velocity} ($\mathbf v_\text{avg}$) is how quickly your position
changes \emph{over a finite time interval}. It is a vector quantity with an
SI unit of \textbf{metres per second} (\si{\metre\per\second}). The direction
of $\mathbf v_\text{avg}$ is the same as displacement $\Delta\mathbf d$, and it is
also a function of time:
\begin{equation}
  \boxed{
    \mathbf v_\text{avg}(t)
    =\frac{\Delta\mathbf r(t)}{\Delta t}
    =\frac{\mathbf r(t)-\mathbf r_1}{t-t_1}
  }
\end{equation}
where $\mathbf r_1=\mathbf r(t_1)$ is the initial position at initial time $t_1$

In contrast, \textbf{instantaneous velocity} ($\mathbf v$) is how quickly
your displacement is changing \emph{at a specific instance in time}
\begin{itemize}
\item Obtained by letting the time interval \emph{infinitesimally} small,
  i.e.\ $\Delta t\rightarrow 0$
\item Also called the \emph{rate of change in displacement}
\item Calculating instantaneous velocity may require calculus\footnote{For
those of you who know a bit of calculus, the definition of instantenous
velocity is:
\begin{displaymath}
  \mathbf v(t)=\frac{d\mathbf r}{dt}
\end{displaymath}}  
\end{itemize}




\subsection{Speed}

\textbf{Average speed} ($v$) is similar to average velocity, but instead of
using displacement, it is the distance ($s$) travelled over a \emph{finite}
time interval. Speed is a \emph{scalar}:
  
\begin{equation}
  \boxed{
    v_\text{avg}(t)
    =\frac{s(t)}{\Delta t}\geq 0
  }
\end{equation}
Similarly, \textbf{instantaneous speed} ($v$) is how quickly
distance is changing at a \emph{specific instance} in time.
\begin{itemize}
\item Since distance is always positive ($s\ge 0$), both average and
  instantaneous speeds must always be positive
\end{itemize}




\subsection{Acceleration}

\textbf{Average acceleration} ($\mathbf a_\text{avg}$) is how quickly the
instantaneous velocity vector changes over a \emph{finite} time interval,
with an SI unit of \textbf{metres per second squared}
(\si{\metre\per\second\squared}):
\begin{equation}
  \boxed{\mathbf a_\text{avg}(t)
    =\frac{\Delta\mathbf v(t)}{\Delta t}
    =\frac{\mathbf v(t)-\mathbf v_1}{t-t_1}
  }
\end{equation}
where $\mathbf v_1=\mathbf v(t_1)$ is the initial velocity $\mathbf v$ at initial time
$t_1$.

\textbf{Instantaneous acceleration} ($\mathbf a(t)$) is how quickly the velocity
vector is changing at a \emph{specific instance} in time

A few things to note about acceleration:
\begin{itemize}
\item i.e.\ \emph{the rate of change of instantaneous velocity}
\item A change in a vector ($\Delta\mathbf v$) can mean a change in magnitude
  and/or direction
\item There can be acceleration without any speeding up or slowing down!
\item Think  about what happens if a car is turning at constant speed
\end{itemize}




%\section{Working with Vectors}
%  Vectors obey the \emph{principle of superposition}, which means that they
%  \emph{add} together. Methods for adding vectors include:
%  \begin{itemize}
%  \item Using \textbf{Pythagorean theorem} (for vectors at right angles to
%    each other)
%  \item Using \textbf{cosine and sine laws}
%  \item Decomposing vectors into \textbf{components}, then reassemble them
%    using Pythagorean theorem
%  \end{itemize}
%  For 1D problems, ($+$) and ($-$) signs are sufficient to indicate direction
%  \begin{itemize}
%  \item Remember to indicate which way is positive though!
%  \end{itemize}
%  \textbf{WARNING:} When adding vectors, you
%  \underline{\textbf{\emph{must not}}} simply add the magnitudes of the vectors!



%
%\section{Basic Motion Graphs}
%We can describe \emph{one-dimensional} motion graphically using
%\textbf{motion graphs}, by plotting
%\begin{itemize}
%\item Position vs.\ time ($d$--$t$)
%\item Instantaneous velocity vs.\ time ($v$--$t$)
%\item Instantaneous acceleration vs.\ time ($a$--$t$)
%\end{itemize}



\section{Basic Motion Graphs}
In one-dimension, motion can 
%  %the kinematic equations\footnote{They can only
%  %  be used for constant acceleration!} can
also be expressed graphically using \textbf{motion graphs}. The most basic
motion graphs are motion quantities as functions of time:
\begin{itemize}[itemsep=3pt]
\item Position vs.\ time ($r$ vs.\ $t$)
\item Instantaneous velocity vs.\ time ($v$ vs.\ $t$)
\item Instantaneous acceleration vs.\ time ($a$ vs.\ $t$)
\end{itemize}
At the moment, we are interested in:
\begin{itemize}[itemsep=3pt]
\item What the graphs themselves tell us
\item What the slopes of the graphs tell us
\item What the areas under the graphs tell us
\end{itemize}



\subsection{Position vs.\ Time Graph}
The \emph{most} obvious choice for expressing 1D motion of an object
graphically is by plotting its position as a function of time ($r(t)$). An
example is shown in Fig.~\ref{fig:pos-time-graph}.
\begin{figure}[ht]
  \centering
  \begin{tikzpicture}[scale=1.5]
    \draw[axes] (0,0)--(3.5,0) node[right]{$t$};
    \draw[axes] (0,-.7)--(0,2.5) node[right]{$r$};
    \draw[functions,smooth,samples=30,domain=0:3]
    plot({\x},{-.2*\x^4+.5*\x^3+.4*\x^2-.5});
    
    \draw[gray] (1.5,0)--(1.5,1.1) node[pos=0,below,black]{$t_0$}
    --(0,1.1) node[left,black]{$r(t_0)$};
    \fill[red!80!black] (1.5,1.1) circle (.05);
  \end{tikzpicture}
  \caption{A typical position vs.\ time graph}
  \label{fig:pos-time-graph}
\end{figure}   
In a position vs.\ time graph, the horizontal ($x$) axis (independent variable)
is time $t$, while the the vertical ($y$) axis (dependent variable) is the
position $r$ measured from origin.
%\item Position in one-dimension can be $+/-$
%Generally, motion begins at $t=0$.
As shown in Figure~\ref{fig:pos-time-graph}, to find position at time $t_0$,
you can simply read the graph!
%\item Time only moves forward, but the graph does not explicitly tell you
%  so




\subsubsection{Slope of Secant: Average Velocity}
\textbf{Average velocity} of an object's motion is the
\emph{slope of the secant} of the position vs.\ time graph.
\begin{figure}[ht]
  \centering
  \begin{tikzpicture}[scale=1.1]
    \draw[axes] (0,0)--(0,3.3) node[above]{$r$};
    \draw[axes] (0,0)--(3,0) node[right]{$t$};
    \draw[functions,smooth,samples=20,domain=0:2.8]
    plot({\x},{-.2*\x^4+.5*\x^3+.4*\x^2+.4});
    \draw[thick,dash dot](.5,.55)--(2.5,2.9);
    \draw[thick] (.5,.55)--(2.5,.55)node[midway,above]{$\Delta t$}
    --(2.5,2.9) node[midway,right]{$\Delta r$};
    \draw[gray] (.5,.55)--(.5,-.1) node[below,black]{$t_1$};
    \draw[gray] (2.5,.55)--(2.5,-.1) node[below,black]{$t_2$};
    \draw[gray] (.5,.55)--(-.1,.55) node[left,black]{$r_1$};
    \draw[gray] (2.5,2.9)--(-.1,2.9) node[left,black]{$r_2$};
    \fill[red!80!black] (.5,.55) circle (.05);
    \fill[red!80!black] (2.5,2.9) circle (.05);
  \end{tikzpicture}
  \hspace{.1in}
  \begin{tikzpicture}[scale=1.1]
    \draw[axes] (0,0)--(0,3.3) node[above]{$r$};
    \draw[axes] (0,0)--(3,0) node[right]{$t$};
    \draw[functions] (0,.35)--(.5,.55)--(1,2)--(1.8,1.5)--(2.5,2.9);
    \draw[thick,dash dot] (.5,.55)--(2.5,2.9);
    \draw[thick] (.5,.55)--(2.5,.55)node[midway,above]{$\Delta t$}
    --(2.5,2.9) node[midway,right]{$\Delta r$};
    \draw[gray] (.5,.55)--(.5,-.1) node[below,black]{$t_1$};
    \draw[gray] (2.5,.55)--(2.5,-.1) node[below,black]{$t_2$};
    \draw[gray] (.5,.55)--(-.1,.55) node[left,black]{$r_1$};
    \draw[gray] (2.5,2.9)--(-.1,2.9) node[left,black]{$r_2$};
    \fill[red!80!black] (.5,.55) circle (.05);
    \fill[red!80!black] (2.5,2.9) circle (.05);
    \end{tikzpicture}
\end{figure}
%  \vspace{-.1in}Same average velocity in both graphs, but very different
%  motions
%  \begin{itemize}
%  \item ($+$) slope: motion in the ($+$) direction during the time interval
%  \item ($-$) slope: motion in the ($-$) direction during the time interval
%  \item Zero slope: no displacement over this time interval
%  \end{itemize}
%\end{frame}
%
%
%
\subsubsection{Instantaneous Velocity:} The instantaneous velocity of an object
is the \emph{slope of the tangent} to the curve of the position vs.\ time graph
at a specific time.
\begin{figure}[ht]
  \centering
  \begin{tikzpicture}[scale=.75]
    \draw[axes] (0,0)--(4,0) node[right] {$t$};
    \draw[axes] (0,0)--(0,4) node[right] {$d$};
    \draw[functions,smooth,samples=20,domain=.5:3.5]
    plot({\x},{.25*\x^2+.5});
    \fill[red!80!black] (2,1.5) circle (2.2pt);
    \draw[dotted,thick] (2,1.5)--(2,0) node[below] {$t_0$};
    
    \draw[smooth,samples=4,domain=.75:3.5,thick,dashed]
    plot({\x},{\x-.5});
    \draw[thick] (.75,.25)--(3.5,.25) node[pos=.65,above] {\scriptsize Run};
    \draw[thick] (3.5,.25)--(3.5,3) node[midway,right] {\scriptsize Rise};
    
  \end{tikzpicture}
\end{figure}

Like average velocity, the sign of the slope also indicates the direction of
motion. If position data is obtained experimentally, it may be difficult to
obtain an accurate value for instantaneous velocity.

%  What can we learn about instantaneous velocity from this position vs.\ time
%  graph?
%\begin{figure}
%  \begin{center}
%    \begin{tikzpicture}
%      \draw[axes] (0,0)--(3.5,0) node[right]{$t$};
%      \draw[axes] (0,-.7)--(0,2.5) node[right]{$d$};
%      \draw[functions,smooth,samples=30,domain=0:3]
%      plot({\x},{-.2*\x^4+.5*\x^3+.4*\x^2-.5});
%      \uncover<2->{
%        \begin{scope}[thick,cyan]
%          \draw (-.4,-.5)--+(.8,0) node[midway,below=-2]{$v=0$};
%          \draw (1.9,2.1)--+(.8,0) node[midway,above=-2]{$v=0$};
%          \draw (2.3,2.1)--(2.3,0) node[below=-2]{$t_1$};
%          \draw (0,-.5)--(0,0);
%        \end{scope}
%        \fill[cyan] (0,-.5) circle (.07);
%        \fill[cyan] (2.3,2.1) circle (.07);
%      %  \draw[dash dot,thick] (1.5,0)--(1.5,1.1) node[pos=0,below]{$t_0$}
%      %  --(0,1.1) node[left]{$d(t_0)$};
%      %  \fill[red!80!black] (1.5,1.1) circle (.05);
%      }
%      \uncover<3->{
%        \fill[pink!35,opacity=.3] (0,-.7) rectangle (2.3,2.3);
%        \draw[<-,thick,red] (.75,1) to[out=120,in=0] +(-1,.2)
%        node[left,text width=98,draw=red,fill=magenta!10]{\scriptsize
%          For $0<t<t_0$, velocity is positive ($v>0$) because the graph has a
%          positive slope\par};
%      }
%      \uncover<4->{
%        \fill[cyan!35,opacity=.3] (2.3,-.7) rectangle (3,2.3);
%        \draw[<-,thick,blue] (2.6,1) to[out=90,in=180] +(1.6,.4)
%        node[right,text width=86,draw=blue,fill=cyan!30]{\scriptsize
%          For $t>t_0$, velocity is negative ($v<0$) because the slope is
%          negative\par};
%      }
%    \end{tikzpicture}
%  \end{center}
%\end{frame}
%
%
%
%
\textbf{Acceleration:} Finding instantaneous acceleration $a(t)$ from a
position vs.\ time graph is \emph{very} difficult\footnote{If you already know
the function $r(t)$ exactly, then you can use calculus to find acceleration
exactly. In that case you don't really \emph{need} this graph in the first
place.}, but we can still find the \emph{sign} of acceleration based on whether
the graph opens up or down.
%In this example:
\begin{figure}[ht]
  \centering
  \begin{tikzpicture}
    \draw[axes] (0,0)--(3.5,0) node[right]{$t$};
    \draw[axes] (0,-.7)--(0,2.5) node[right]{$d$};
    \draw[functions,smooth,samples=30,domain=0:2.95]
    plot({\x},{-.2*\x^4+.5*\x^3+.4*\x^2-.5});
    %\uncover<2->{
      \fill[magenta!35,opacity=.3] (0,-.7) rectangle (1.47,2.3);
      \draw[<-,thick,red] (.75,.2) to[out=120,in=0] +(-1,1)
      node[left,text width=88,draw=red,fill=magenta!10]{\scriptsize
        For $t<t_0$, acceleration is positive ($a>0$) because the graph opens
        \underline{up}\par};
      \draw[thick,gray](1.47,1.03)--(1.47,0) node[below,black]{$t_0$};
      \fill[red!80!black] (1.47,1.03) circle (.055);
    %}
    %\uncover<3->{
      \draw[functions,smooth,samples=30,domain=1.48:2.95,blue]
      plot({\x},{-.2*\x^4+.5*\x^3+.4*\x^2-.5});
      \fill[red!80!black] (1.47,1.03) circle (.055);
      \fill[cyan!35,opacity=.3] (1.47,-.7) rectangle (3,2.3);
      \draw[<-,thick,blue] (2.25,1.8) to[out=270,in=180] +(1.6,-2.2)
      node[right,text width=88,draw=blue,fill=cyan!30]{\scriptsize
        For $t>t_0$, acceleration is negative ($a<0$) because the graph opens
        \underline{down}\par};
    %}
  \end{tikzpicture}
\end{figure}

%
%
%
\subsection{Velocity vs.\ Time Graph}

A less obvious choice for expressing 1D motion is by plotting
\emph{instantaneous} velocity as a function of time, i.e.\ $v=v(t)$. In
essence, we are plotting the slope of the position vs.\ time graph instead.
Using our example position vs.\ time graph:

\begin{figure}[ht]
  \centering
  \begin{tikzpicture}[scale=1.1]
    \draw[axes] (0,0)--(3.5,0) node[above]{$t$};
    \draw[axes] (0,-1)--(0,2.5) node[right]{$d$};
    \draw[functions,smooth,samples=30,domain=0:2.95,magenta]
    plot(\x,{-.2*\x^4+.5*\x^3+.4*\x^2-.5});

    \fill[violet] (1.3,.7) circle (.055);
    \draw[violet] (1.3,.7)--(1.3,0) node[below]{$t_0$};
    \draw[thick,violet,rotate around={atan(1.8):(1.3,.7)}] (.3,.7)--+(2,0)
    node[midway,above,rotate=atan(1.8)]{$m=v(t_0)$};
  \end{tikzpicture}
  \begin{tikzpicture}[scale=1.1]
    \draw[axes] (0,0)--(3.5,0) node[above]{$t$};
    \draw[axes] (0,-1)--(0,2.5) node[right]{$v$};
    \draw[smooth,samples=40,domain=0:2.5,functions,violet]
    plot(\x,{-.8*\x^3+1.5*\x^2+.8*\x});
    \draw[gray] (1.3,0)--(1.3,1.83) node[pos=0,below,black]{$t_0$}
    --(0,1.83) node[left,black]{$v(t_0)$};
    \fill[violet] (1.3,1.83) circle (.06);
  \end{tikzpicture}

\end{figure}

%
%
%
%\begin{frame}{Velocity vs.\ Time Graph}
%%  \begin{columns}
%%    \column{.7\textwidth}
%%    \begin{itemize}
%  The velocity vs.\ time graph shows how instantaneous velocity evolves with
%  time. In this example:
%%    \item<3->Slope of the secant: average acceleration
%%    \item<3->Slope of the tangent: instantaneous acceleration
%%    \end{itemize}
%%    
%%    \column{.3\textwidth}
%  \begin{center}
%    \begin{tikzpicture}
%      \draw[axes] (0,0)--(3,0) node[right=-2]{$t$};
%      \draw[axes] (0,-1.5)--(0,2.2) node[above=-2]{$v$};
%      \draw[smooth,samples=45,domain=0:2.55,very thick,violet]
%      plot(\x,{-.8*\x^3+1.5*\x^2+.8*\x});
%      \uncover<2->{
%        \fill[violet] (2.3,0) circle (.055) node[below left=-2]{$t_1$};
%        \fill[violet] circle (.055) node[left=-2]{$0$};
%        \draw[<-,violet,thick] (-.4,0)--+(-3,0)
%        node[text width=80,draw=violet,fill=violet!10]{\scriptsize
%          At $t=0$ and $t=t_1$, velocity is zero ($v=0$)\par};
%      }
%      \uncover<3->{
%        \draw[gray] (1.47,1.87)--(1.47,0) node[below,violet]{$t_0$};
%        \fill[violet] (1.47,1.87) circle (.055);
%        \draw[<-,thick,violet] (1.55,1.87) to[out=30,in=150] +(1.6,0)
%        node[right,text width=75,draw=violet,fill=violet!10]{\scriptsize
%          At $t=t_0$, velocity is maximum. The slope is zero at this point.
%          \par};
%      }
%    \end{tikzpicture}
%  \end{center}
%  Velocity is positive when the graph is above the time axis; and negative
%  when below the time axis
%
%%    \begin{tikzpicture}[scale=1.1]
%%      \draw[axes] (0,0)--(3.5,0) node[above]{$t$};
%%      \draw[axes] (0,-1.25)--(0,2.75) node[right]{$v$};
%%      \draw[functions,smooth,samples=40,domain=0:3]
%%      plot({\x},{1.35*(\x-1)*(\x-1)-1.1*\x+.5});
%%      \uncover<2>{
%%        \draw[smooth,samples=40,domain=.626:2.189,orange,very thick]
%%        plot({\x},{1.35*(\x-1)*(\x-1)-1.1*\x+.5});
%%        \draw[smooth,samples=40,domain=2.189:3,violet,very thick]
%%        plot({\x},{1.35*(\x-1)*(\x-1)-1.1*\x+.5});
%%        \draw[smooth,samples=40,domain=0:.626,violet,very thick]
%%        plot({\x},{1.35*(\x-1)*(\x-1)-1.1*\x+.5});
%%      }
%%    \end{tikzpicture}
%%  \end{columns}
%\end{frame}
%
%

\textbf{Average Acceleration:}
%  \begin{center}
%    \begin{tikzpicture}[scale=1.1]
%      \draw[axes] (0,0)--(3,0) node[right=-2]{$t$};
%      \draw[axes] (0,-1.5)--(0,2.2) node[above=-2]{$v$};
%      \draw[smooth,samples=45,domain=0:2.55,very thick,violet]
%      plot(\x,{-.8*\x^3+1.5*\x^2+.8*\x});
%      %\uncover<2->{
%      %  \fill[violet] (2.3,0) circle (.055) node[below left=-2]{$t_1$};
%      %  \fill[violet] (0,0) circle (.055) node[left=-2]{$0$};
%      %}
%    \end{tikzpicture}
%  \end{center}
%\end{frame}
%
%
%
\textbf{Instantaneous Acceleration:}
%  \begin{center}
%    \begin{tikzpicture}[scale=1.1]
%      \draw[axes] (0,0)--(3,0) node[right=-2]{$t$};
%      \draw[axes] (0,-1.5)--(0,2.2) node[above=-2]{$v$};
%      \draw[smooth,samples=45,domain=0:2.55,very thick,violet]
%      plot(\x,{-.8*\x^3+1.5*\x^2+.8*\x});
%      %\uncover<2->{
%      %  \fill[violet] (2.3,0) circle (.055) node[below left=-2]{$t_1$};
%      %  \fill[violet] (0,0) circle (.055) node[left=-2]{$0$};
%      %}
%    \end{tikzpicture}
%  \end{center}



\textbf{Displacement and position:} The area under the velocity vs.\ time graph
is the \emph{displacement} of the object. In example below, the shaded area is
the displacement between $t_1$ and $t_2$. If the position at $t_1$ (i.e.\
$d_1$) is also known, then we can find the position at $t_2$ (i.e.\
$d_2=d_1+\Delta d$).
\begin{figure}[ht]
  \centering
  \begin{tikzpicture}[scale=1.3]
    \draw[smooth,samples=30,domain=.5:2,gray,fill=lightgray]
    plot(\x,{-.8*\x^3+1.5*\x^2+.8*\x})--(2,0) node[below,black]{$t_2$}
    --(.5,0) node[below,black]{$t_1$}--cycle;

    \draw[smooth,samples=50,domain=0:2.4,functions,violet]
    plot(\x,{-.8*\x^3+1.5*\x^2+.8*\x});

    \draw[axes] (0,0)--(3.5,0) node[above]{$t$};
    \draw[axes] (0,-.5)--(0,2.5) node[right]{$v$};

    \node at (1.25,.7) {$\Delta d$};
  \end{tikzpicture}
  \caption{The area under the velocity vs.\ time graph between $t_1$ and $t_2$
    gives us the object's displacement during this time interval.}
  \label{fig:area-under-vt-graph}
\end{figure}
If the area is \emph{above} the $x$-axis (time axis), then displacement
is positive ($\Delta d>0$); if the area is \emph{below} the $x$-axis, then
displacement is negative ($\Delta d<0$).



\subsection{Acceleration vs. Time Graph}
In the same way that we convert a position vs.\ time graph to a velocity vs.\
time graph, we can also convert a velocity vs.\ time graph to an
\textbf{acceleration vs.\ time} graph, by plotting the slope of the tangent.
This graph shows how \emph{instantaneous} acceleration $a(t)$ evolves with
time. In the example below:
%    \centering
%    \begin{tikzpicture}[scale=1.1]
%      \draw[axes] (0,0)--(3,0) node[right]{$t$};
%      \draw[axes] (0,-1)--(0,2.5) node[right]{$v$};
%      \draw[smooth,samples=40,domain=0:2.4,functions,violet]
%      plot(\x,{-.8*\x^3+1.5*\x^2+.8*\x});
%      %\fill[violet] (1.3,.7) circle (.06);
%      %\draw[violet] (1.3,.7)--(1.3,0) node[below]{$t_0$};
%    \end{tikzpicture}
%    
%    \column{.3\textwidth}
%    \centering
%    \begin{tikzpicture}[scale=1.1]
%      \draw[axes] (0,0)--(3,0) node[right]{$t$};
%      \draw[axes] (0,-2.3)--(0,1.2) node[right]{$a$};
%      \draw[smooth,samples=40,domain=0:2.2,functions,orange]
%      plot(\x,{-1.2*\x^2+1.5*\x+.4});
%      %\draw[gray] (1.3,0)--(1.3,1.83) node[pos=0,below,black]{$t_0$}
%      %--(0,1.83) node[left,black]{$v(t_0)$};
%      %\fill[violet] (1.3,1.83) circle (.06);
%    \end{tikzpicture}
%  \end{columns}

%\begin{frame}{Acceleration vs.\ Time Graph}

%  \begin{center}
%    \begin{tikzpicture}[scale=1.1]
%      \draw[axes] (0,0)--(2.8,0) node[right=-2]{$t$};
%      \draw[axes] (0,-2)--(0,1.2) node[right=-2]{$a$};
%      \draw[smooth,samples=40,domain=0:2.2,functions,orange]
%      plot(\x,{-1.2*\x^2+1.5*\x+.4});
%      \uncover<2->{
%        \fill[pink!40,opacity=.4] (0,-2) rectangle (1.48,1);
%        \fill[orange] (0.63,.87) circle (.06);
%        \draw[gray] (.63,0)--(.63,.87) node[pos=0,below=-2,black]{$t_0$}
%        --(0,.87) node[left=-2,black]{$a_\text{max}$};
%        \fill[orange] (1.48,0) circle (.06) node[above,black]{$t_1$};
%        \draw[thick,magenta,<-] (.75,-1)--+(-1,0)
%        node[left,text width=80,draw=magenta]{\scriptsize
%          Acceleration is positive from $t=0\rightarrow t_1$, with a maximum
%          magnitude of $a_\text{max}$ at $t_0$\par};
%      }
%      \uncover<3->{
%        \fill[violet] (1.48,0) circle (.06) node[above]{$t_1$};
%        \draw[thick,violet,<-] (1.48,.4) to[out=90,in=180] (3,1)
%        node[right,text width=70,draw=violet,fill=violet!10]{\scriptsize
%          Acceleration is zero ($a=0$) at $t=t_1$\par};
%      }
%      \uncover<4->{
%        \fill[cyan!40,opacity=.4] (1.48,-2) rectangle (2.4,1);
%        \draw[thick,blue,<-] (2.1,-.8)--+(1,0)
%        node[right,text width=63,draw=blue,fill=cyan!10]{\scriptsize
%          Acceleration is negative for $t>t_1$\par};
%      }
%    \end{tikzpicture}
%  \end{center}
%  \uncover<4->{
%    \vspace{-.1in}Remember: Since acceleration is a vector, \emph{positive}
%    acceleration means acceleration \emph{in the positive
%    \underline{direction}}, but it does not necessarily mean the object speeds
%    up
%  }
%\end{frame}
%
%

\subsubsection{Slope of the Acceleration vs.\ Time Graph}
The slope of the acceleration vs.\ time graph is the rate of change of
acceleration, called \textbf{jerk}.\footnote{The slope of the tangent is
called \textbf{instantaneous jerk}, whereas the slope of the secant is the
\textbf{average jerk}.}. This is \emph{not} a topic that is covered in Grade
11 or 12 Physics.
\begin{figure}[ht]
  \centering
  \begin{tikzpicture}
    \draw[axes] (0,0)--(3,0) node[right]{$t$};
    \draw[axes] (0,-2)--(0,1.2) node[right]{$a$};
    \draw[smooth,samples=40,domain=0:2.2,functions,orange]
    plot(\x,{-1.2*\x^2+1.5*\x+.4});
    \fill[blue] (1,.7) circle (.055);
    \draw[blue,thick,rotate around={-atan(.9):(1,.7)}] (0,.7)--+(2.2,0);
  \end{tikzpicture}
\end{figure}



%\begin{frame}{Area Under the Acceleration vs.\ Time Graph}
%  The area under acceleration vs.\ time graph is the \emph{change in velocity}
%  $\Delta v$.
%  \begin{center}
%    \begin{tikzpicture}
%      \draw[axes] (0,0)--(3,0) node[right]{$t$};
%      \draw[axes] (0,-2)--(0,1.2) node[right]{$a$};
%      \draw[smooth,samples=40,domain=0:2.2,functions,orange]
%      plot(\x,{-1.2*\x^2+1.5*\x+.4});
%      \uncover<2->{
%        \draw[smooth,samples=40,domain=.3:1.3,thick,gray,fill=lightgray!50]
%        plot(\x,{-1.2*\x^2+1.5*\x+.4})--(1.3,0) node[below=-2,black]{$t_2$}
%        --(.3,0) node[below=-2,black]{$t_1$}--cycle;
%        \draw[thick,<-,black!80] (.8,.3)--+(-1.5,0)
%        node[text width=72,left,draw=black!80,fill=lightgray!50]{\scriptsize
%          $\Delta v>0$ from $t_1\rightarrow t_2$ because the area is above the
%          $x$ axis\par};
%      }
%      \uncover<3->{
%        \draw[smooth,samples=40,domain=1.7:2.1,thick,cyan,fill=cyan!20]
%        plot(\x,{-1.2*\x^2+1.5*\x+.4})--(2.1,0) node[above=-2]{$t_4$}
%        --(1.7,0) node[above=-2]{$t_3$}--cycle;
%        \draw[thick,<-,blue] (1.9,-.5)--+(2,0)
%        node[text width=75,right,draw=blue,fill=cyan!20]{\scriptsize
%          $\Delta v<0$ from $t_3\rightarrow t_4$ because the area is below the
%          $x$ axis\par};
%      }
%    \end{tikzpicture}
%  \end{center}
%  \begin{itemize}
%  \item If the area is \emph{above} the $x$-axis (time axis), then $\Delta v>0$
%  \item If the area is \emph{below} the $x$-axis, then $\Delta v<0$
%  \item Remember: $\Delta v>0$ does not necessarily mean that the object
%    speeds up; $\Delta v<0$ does not necessarily mean that it will slow down
%  \end{itemize}
%\end{frame}


\section{Uniform Motion}
\textbf{Uniform motion} is when the velocity vector is constant, and neither
its magnitude nor direction changes. In 1D, the motion graphs look like this:
\vspace{-.1in}\begin{center}
  \begin{tikzpicture}[scale=.6]
    \draw[axes] (0,0)--(4.5,0) node[right]{$t$};
    \draw[axes] (0,0)--(0,4.5) node[above]{$d$};
    \draw[functions] (0,.5)--(4,4);
  \end{tikzpicture}
  \hspace{.15in}
  \begin{tikzpicture}[scale=.6]
    \draw[axes] (0,0)--(4.5,0) node[right]{$t$};
    \draw[axes] (0,0)--(0,4.5) node[above]{$v$};
    \draw[functions] (0,2)--(4,2);
  \end{tikzpicture}
  \hspace{.15in}
  \begin{tikzpicture}[scale=.6]
    \draw[axes] (0,0)--(4.5,0) node[right]{$t$};
    \draw[axes] (0,0)--(0,4.5) node[above]{$a$};
    \draw[functions] (0,0)--(4,0);
  \end{tikzpicture}
\end{center}
\begin{itemize}
\item \vspace{-.15in}$d$--$t$ graph is a straight line
\item The slope of the $d$--$t$ graph, which is velocity $v$, is also constant
\item There is no acceleration, so $a=0$ for all $t$
\end{itemize}




\section{Uniform Acceleration}
When a constant net force acts on an object, it moves with a constant
non-zero acceleration, or \textbf{uniform acceleration}.
\begin{center}
  \begin{tikzpicture}[scale=.55]
    \draw[axes] (0,0)--(4.5,0) node[right]{$t$};
    \draw[axes] (0,-1.5)--(0,4.5) node[above]{$d$};
    \draw[functions,samples=10,domain=0:4] plot(\x,{.35*(\x-1)*(\x-1)+1});
  \end{tikzpicture}
  \hspace{.15in}
  \begin{tikzpicture}[scale=.55]
    \draw[axes] (0,0)--(4.5,0) node[right]{$t$};
    \draw[axes] (0,-1.5)--(0,4.5) node[above]{$v$};
    \draw[functions] (0,-1)--(4,2.5);
  \end{tikzpicture}
  \hspace{.15in}
  \begin{tikzpicture}[scale=.55]
    \draw[axes] (0,0)--(4.5,0) node[right]{$t$};
    \draw[axes] (0,-1.5)--(0,4.5) node[above]{$a$};
    \draw[functions] (0,1)--(4,1);
  \end{tikzpicture}
\end{center}
\begin{itemize}
\item The $d$ vs.\ $t$ graph is part of a \emph{parabola}
  \begin{itemize}
  \item opens \emph{up}, then acceleration is positive
  \item opens \emph{down}, then acceleration is negative
  \end{itemize}
\item The $v$ vs.\ $t$ graph is a straight line; the (constant) slope is the
  cceleration
\end{itemize}




\section{Area Under Motion Graphs}
\begin{center}
  \begin{tikzpicture}[scale=.55]
    \draw[axes] (0,0)--(4.5,0) node[right]{$t$};
    \draw[axes] (0,-1.5)--(0,4.5) node[above]{$d$};
    \draw[functions,samples=10,domain=0:4] plot(\x,{.35*(\x-1)*(\x-1)+1});
  \end{tikzpicture}
  \hspace{.15in}
  \begin{tikzpicture}[scale=.55]
    \draw[pink!40,fill=pink!40] (0,0)--(0,-1)--(1,0)--cycle;
    \draw[blue!20,fill=blue!20] (1,0)--(3.5,0)--(3.5,2.5)--cycle;
    \draw[axes] (0,0)--(4.5,0) node[right]{$t$};
    \draw[axes] (0,-1.5)--(0,4.5) node[above]{$v$};
    \draw[functions] (0,-1)--(4,3);
  \end{tikzpicture}
  \hspace{.15in}
  \begin{tikzpicture}[scale=.55]
    \fill[gray!40] rectangle (4,1) node[black,midway]{$\Delta v$};
    \draw[axes] (0,0)--(4.5,0) node[right]{$t$};
    \draw[axes] (0,-1.5)--(0,4.5) node[above]{$a$};
    \draw[functions] (0,1)--(4,1);
  \end{tikzpicture}
\end{center}
\begin{itemize}
\item The area under the $a$--$t$ graph is the change in velocity $\Delta v$
  \begin{itemize}
  \item If initial velocity is known, then we can plot $v$--$t$ graph based on
    this graph
  \end{itemize}
\item The area under the $v$--$t$ graph is the displacement $\Delta d$
  \begin{itemize}
  \item If the area is {\color{red!40}below} the $x$-axis (time axis), then
    displacement is negative;
  \item If the area is {\color{blue!20}above} the time axis, then
    displacement is positive
  \end{itemize}
\item The area under the $d$--$t$ graph has no physical meaning
\end{itemize}




\section{Simple Harmonic Motion}
In \textbf{simple harmonic motion}\footnote{Or \textbf{oscillatory motion},
or \textbf{vibration}}, displacement, velocity and acceleration are all
periodic functions, and none of them are constant!
\vspace{-.1in}
\begin{center}
  \begin{tikzpicture}[scale=.55]
    \draw[axes] (0,0)--(4.5,0) node[right]{$t$};
    \draw[axes] (0,-2.5)--(0,2.5) node[above]{$d$};
    \draw[functions,samples=50,domain=0:4] plot(\x,{2*cos(150*\x)});
  \end{tikzpicture}
  \hspace{.15in}
  \begin{tikzpicture}[scale=.55]
    \draw[axes] (0,0)--(4.5,0) node[right]{$t$};
    \draw[axes] (0,-2.5)--(0,2.5) node[above]{$v$};
    \draw[functions,samples=50,domain=0:4] plot(\x,{-2*sin(150*\x)});
  \end{tikzpicture}
  \hspace{.15in}
  \begin{tikzpicture}[scale=.55]
    \draw[axes] (0,0)--(4.5,0) node[right]{$t$};
    \draw[axes] (0,-2.5)--(0,2.5) node[above]{$a$};
    \draw[functions,samples=50,domain=0:4] plot(\x,{-2*cos(150*\x)});
  \end{tikzpicture}
\end{center}
We will discuss this topic later, in the next unit.



%\section{Example Problem}
%  \textbf{Example:} Expression the motion in a \emph{position-time}
%  graph and an \emph{acceleration-time} graph. Assume the object's initial
%  position is the origin of the coordinate system.
%  \begin{center}
%    \begin{tikzpicture}[scale=.75]
%      \draw[help lines] grid (12,4);
%      \draw[axes] (0,0)--(13,0) node[right]{$t$ (\si\second)};
%      \foreach \t in {0,...,12} {
%        \draw(\t,0)--(\t,-.15) node[below]{$\t$};
%      }
%      \draw[axes] (0,0)--(0,5) node[above]{$v$ (\si{\metre\per\second})};
%      \foreach \v in {0,...,4} {
%        \draw(0,\v,0)--(-.15,\v) node[left]{$\v$};
%      }
%      \draw[functions] (0,2)--(2.5,2)--(5,4)--(7.5,4)--(10,1)--(12,1);
%    \end{tikzpicture}
%  \end{center}
%
%
%
%
\section{1D Kinematic Equations}

\begin{align*}
  \Delta d &=v_1\Delta t + \frac12a\Delta t^2\\
  \Delta d &=v_2\Delta t - \frac12a\Delta t^2\\
  \Delta d &=\frac{v_1+v_2}2 \Delta t\\
  v_2 &= v_1+ a \Delta t\\
  v_2^2 &= v_1^2+ 2a \Delta d
\end{align*}

There are five motion quantities of interest:
\begin{center}
  \begin{tabular}{l|c|c}
    \rowcolor{pink}
    \textbf{Quantity} & \textbf{Symbol} & \textbf{SI Unit} \\ \hline
    Displacement & $\Delta d$ & \si{\metre} \\
    Initial (instantaneous) velocity & $v_1$ & \si{\metre\per\second} \\
    Final (instantaneous) velocity   & $v_2$ & \si{\metre\per\second} \\
    Acceleration (constant) & $a$    & \si{\metre\per\second\squared}\\
    Time interval & $\Delta t$ & \si\second
  \end{tabular}
\end{center}
Only valid for \underline{\textbf{constant acceleration}}



%\section{1D Kinematic Equations}
%  \begin{columns}
%    \column{.35\textwidth}
%    {\large
%      \begin{align*}
%        \Delta d &=v_1\Delta t + \frac12a\Delta t^2\\
%        \Delta d &=v_2\Delta t - \frac12a\Delta t^2\\
%        \Delta d &=\frac{v_1+v_2}2 \Delta t\\
%        v_2 &= v_1+ a \Delta t\\
%        v_2^2 &= v_1^2+ 2a \Delta d
%      \end{align*}
%    }
%    \column{.65\textwidth}
\begin{itemize}
\item For 1-object problems, you are usually given 3 of the 5 variables,
  and you are asked to find a 4th one
\item For 2-object problems, the motion of the two objects are connected by
  time interval $\Delta t$ and displacement $\Delta d$
\item For 2D or 3D problems, each direction should have its own kinematic
  equations
\end{itemize}



%\section{1D Kinematic Equations}
Kinematic equations \emph{cannot} be used when acceleration is non-uniform
(when non-constant forces act on an object):
\begin{itemize}
\item Aerodynamic forces
  \begin{itemize}
  \item Lift and drag forces
  \item proportional to $v^2$
  \end{itemize}
\item Spring force
  \begin{itemize}
  \item The force that a compressed/stretched spring applies to connected
    objects
  \item Proportional to spring displacement $\mathbf x$
  \end{itemize}
\item Dampers in springs
  \begin{itemize}
  \item Dampers are used to slow down the vibration of an object
  \item Generally proportional to $v$
  \end{itemize}
\end{itemize}  
%We will discuss more about forces later in this unit.




%\section{Relative Motion}
%  \begin{center}
%    \vspace{-.15in}
%    \pic{.65}{graphics/57}
%  \end{center}
%  \begin{itemize}
%  \item Observers (frames of reference) A and B measures different motion of the
%    ball because A and B a moving relative to each other
%  \item The instantaneous velocity of the ball at any time $t$, as measured by
%    A and B, is related by the instantaneous velocities of A and B relative to
%    each other
%  \end{itemize}
%
%
%
%
%\section{Relative Motion}
%  All velocities are measured \emph{relative} to a frame of reference.
%  Therefore, when expressing relative motion, we can use two subscripts:
%    
%  \eq{-.15in}{
%    \mathbf v_{AB}
%  }
%    
%  \vspace{-.15in}where $A$ represents the object, and $B$ represents the frame
%  of reference
%
%  \vspace{.25in}\textbf{Example:} the velocity of an airplane ($P$) travelling
%  at \SI{251}{\kilo\metre\per\hour} [N] relative to Earth ($E$) is expressed as:
%
%  \eq{-.2in}{
%    \mathbf v_{PE}=\SI{251}{\kilo\metre\per\hour}\text{ [N]}
%  }
%
%
%
%
%\section{Relative Motion}
%  \begin{itemize}
%  \item Different observers make different observations because they (their
%    frames of reference) are moving relative to each other.
%  \item In \emph{classical} mechanics, the different velocity measurements are
%    related by the \textbf{Galilean velocity addition rule}\footnote{This
%    equation was thought to be so obvious that no one bothered to give it a
%    name until Einstein showed that it is not valid near the speed of light}:
%    
%    \eq{-.1in}{
%      \boxed{\mathbf v_{AC}=\mathbf v_{AB}+\mathbf v_{BC}}
%    }
%
%    \vspace{-.1in}The velocity of $A$ relative to reference frame $C$ is the
%    velocity of $A$ relative to reference frame $B$, plus the velocity of $B$
%    relative to $C$.
%  \item Can only be used when velocity $v$ is small compared to the speed of
%    light $c$
%  \end{itemize}
%
%
%%\section{Relative Motion}
%%  If we add another reference frame ($D$), the equation becomes:
%%
%%  \eq{-.3in}{
%%    \mathbf v_{AD}=\mathbf v_{AB}+\mathbf v_{BC}+\mathbf v_{CD}
%%  }
%%
%
%
%
%\section{Relative Motion Example: Airplane in Air}
%  The velocity of the plane relative to the ground (``ground speed'') is
%  the velocity of the plane relative to the air (``air speed'') plus
%  the velocity of the air relative to the ground (``wind speed'').  
%  \begin{center}
%    \pic{.4}{graphics/Planewind}
%  \end{center}
%  The addition of velocities is exactly the same as any vector addition.

%\section{Projectile Motion}
%  A \textbf{projectile} is an object that is launched with an initial velocity
%  of $\mathbf v_1$ along a parabolic trajectory and accelerates only due to
%  gravity.
%  \begin{columns}[T]
%    \column{.3\textwidth}
%    \begin{tikzpicture}[scale=1.6]
%      \draw[axes] (0,0)--(2,0) node[right]{$x$};
%      \draw[axes] (0,0)--(0,2) node[above]{$y$};
%      \draw[dotted,domain=0:2.7,thick] plot (\x, {1.2*\x-.2*\x*\x});
%      \draw[vectors] (0,0)--(.75,.9) node[above]{$\mathbf v_1$};
%      \draw[vectors,red] (0,0)--(0,.9) node[midway,left]{$v_y$};
%      \draw[vectors,blue] (0,0)--(.75,0) node[midway,below]{$v_x$};
%      \draw[axes] (.5,0) arc (0:52:.5) node[pos=.6,right]{$\theta$};
%    \end{tikzpicture}
%
%    \column{.67\textwidth}
%    \begin{itemize}
%    \item $x$-axis: \emph{horizontal}, pointing \emph{forward}
%    \item $y$-axis: \emph{vertical}, pointing \emph{up}
%    \item Angle $\theta$ measured \emph{above} the horizontal (i.e.\ $\theta>0$
%      when thrown upwards; $\theta<0$ then thrown downwards)
%    \item The origin is usually where the projectile is launched
%    \end{itemize}
%  \end{columns}
%
%
%
%
%\section{Horizontal Direction}
%  The initial velocity $\mathbf v_1$ can be decomposed into its $x$ and $y$
%  components:
%
%  \vspace{-.25in}{\large
%    \begin{align*}
%      v_x &=v_1\cos\theta \\
%      v_y &=v_1\sin\theta
%      \end{align*}
%  }
%
%  There is no horizontal acceleration (i.e.\ $a_x=0$), therefore $v_x$ is
%  constant. The kinematic equations reduce to a single equation:
%
%  \eq{-.1in}{
%    \Delta x=v_x\Delta t=\left[v_0\cos\theta\right]\Delta t
%  }
%
%  \vspace{-.1in}where $\Delta x$ is the horizontal displacement.
%
%
%
%
%
%\section{Vertical Direction}
%  There is constant vertical acceleration due to gravity alone, i.e.\
%  $a_y=-g$. ($a_y$ is \emph{negative} due to the way we defined the
%  coordinate system, with the $y$-axis pointing up.) The most important
%  kinetic equation is this one:
%
%  \eq{-.1in}{
%    \Delta y = \left[v_1\sin\theta\right]\Delta t-\frac12g\Delta t^2
%  }
%
%  These two kinematic equations may also be useful:
%
%  \vspace{-.25in}{\large
%    \begin{align*}
%      v_y &= \left[v_1\sin\theta\right] -gt\\
%      v_y^2&=\left[v_1^2\sin^2\theta\right]-2g\Delta y
%    \end{align*}
%  }
%
%
%
%
%\section{Solving Projectile Motion Problems}
%  Horizontal and vertical motions are linearly independent, but variables are
%  shared in both directions:
%  \begin{itemize}
%  \item Time interval $\Delta t$
%  \item Launch angle $\theta$ (above the horizontal)
%  \item Initial speed $v_1$
%  \end{itemize}
%  
%  \vspace{.25in}When solving any projectile motion problems
%  \begin{itemize}
%  \item \emph{Two} equations with \emph{two} unknowns
%  \item If an object lands on an incline, there will be a third equation
%    relating $x$ and $y$
%  \end{itemize}
%
%
%
%
%\section{Symmetric Trajectory}
%  A projectile's trajectory is \emph{symmetric} if the object lands at the same
%  height as when it launched. The angle $\theta$ is measured above the
%  horizontal. The \textbf{time of flight} ($T$), \textbf{range} ($R$)
%  and \textbf{maximum height} ($H$) are, respectively,
%
%  \eq{-.1in}{
%    \boxed{T=\frac{2v_1\sin\theta}g}\quad\quad
%    \boxed{R=\frac{v_1^2\sin(2\theta)}g}\quad\quad
%    \boxed{H=\frac{v_1^2\sin^2\theta}{2g}}
%  }
%
%
%
%
%\section{Maximum Range}
%  \eq{-.1in}{
%    R=\frac{v_1^2\sin(2\theta)}g
%  }
%  
%  \begin{itemize}
%  \item Maximum range occurs at $\theta=\ang{45}$
%  \item For a given initial speed $v_0$ and range $R$, launch angle $\theta$ is
%    given by:
%    
%    \eq{-.1in}{
%      \theta_1=\frac12\sin^{-1}\left(\frac{Rg}{v_1^2}\right)
%    }
%
%  \item But there is another angle that \emph{gives the same range}!
%
%    \eq{-.1in}{
%      \theta_2=\ang{90}-\theta_1
%    }
%  \end{itemize}
%
%%
%%
%%\section{Projectile Motion}
%%  \begin{itemize}
%%  \item For projectile motion problems, resolve the problem into horizontal
%%    ($x$) and vertical ($y$) directions, and apply kinematic equations
%%    independently
%%  \item No horizontal acceleration ($a_x=0$), therefore kinematic equations
%%    reduce to a single equation:
%%    
%%    \eq{-.3in}{\Delta x=v_x\Delta t}
%%  \item Acceleration due to gravity only in the vertical ($y$) direction:
%%    
%%    \eq{-.25in}{\mathbf a_y=\mathbf g=\magdir{\SI{9.81}{\metre\per\second^2}}{down}}
%%
%%    \vspace{-.15in}We \emph{usually} define the (+) direction to be [up], so
%%    $a_y=-g=\SI{-9.81}{\metre\per\second^2}$ %, but it can change depending on
%%    %the problem
%%  \end{itemize}
%%
%%
%%
%%\section{Solving Projectile Motion Problems}
%%  \begin{itemize}
%%  \item There are variables the two directions
%%    \begin{itemize}
%%    \item Initial speed $v_i$
%%    \item Angle above the horizontal $\theta$ (appears in the initial velocities
%%      in both horizontal and vertical directions)
%%    \item Time of motion $\Delta t$
%%    \end{itemize}
%%  \item Have two equations with two unknowns
%%  \item In more difficult problems, $\Delta y$ and $\Delta x$ can be related
%%    geometrically, (so $3$ equations with $3$ unknowns)
%%  \end{itemize}
%%
%%
%%
%%
\begin{example}
  While hiking in the wilderness, you come to a cliff
  overlooking a river. A topographical map shows that the cliff is
  \SI{291}{\metre} high and the river is \SI{68.5}{\metre} wide at that
  point. You throw a rock directly forward from the top of the cliff, giving
  the rock a horizontal velocity of \SI{12.8}{\metre\per\second}.
  \begin{enumerate}
  \item Did the rock make it across the river?
  \item With what velocity did the rock hit the ground or water?
  \end{enumerate}
  
  \begin{center}
    \pic{.5}{kinematics/graphics/cliff}
  \end{center}
\end{example}



\begin{example}
  A golfer hits the golf ball off the tee, giving it an
  initial velocity of \SI{32.6}{\metre\per\second} at an angle of \ang{65} with
  the horizontal. The green where the golf ball lands is \SI{6.30}{\metre}
  higher than the tee, as shown in the illustration. Find the time interval
  when the golf ball was in the air, and the distance to the green.
  \begin{center}
    \pic{.5}{kinematics/graphics/golfer}
  \end{center}
\end{example}



\begin{example}
  You are playing tennis with a friend on tennis courts
  that are surrounded by a \SI{4.8}{\metre} fence. You opponent hits the ball
  over the fence and you offer to retrieve it. You find the ball at a distance
  of \SI{12.4}{\metre} on the other side of the fence. You throw the ball at an
  angle of \ang{55.} with the horizontal, giving it an initial velocity of
  \SI{12.1}{\metre\per\second}. The ball is \SI{1.05}{\metre} above the ground
  when you release it. Did the ball go over the fence, hit the fence, or hit
  the ground before it reached the fence?
\end{example}



%\section{Symmetric Trajectory}
%  Trajectory is \emph{symmetric} if the object lands at the same height as
%  when it started.
%  \begin{itemize}
%  \item Time of flight
%    \eq{-.1in}{t_\text{max}=\frac{2v_i\sin\theta} g}
%  \item Range
%    \eq{-.1in}{R=\frac{v_i^2\sin(2\theta)} g}
%  \item Maximum height
%    \eq{-.1in}{h_\text{max}=\frac{v_i^2\sin^2\theta}{2g}}
%  \end{itemize}
%  The angle $\theta$ is the \textbf{above the the horizontal}



\begin{example}
  A player kicks a football for the opening kickoff. He
  gives the ball an initial velocity of \SI{29}{m/s} at an angle of \ang{69}
  with the horizontal. Neglecting friction, determine the ball's maximum height,
  hang time and range?
\end{example}
