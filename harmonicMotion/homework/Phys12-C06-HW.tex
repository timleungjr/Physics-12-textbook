\section*{Problems}

\begin{enumerate}
%  \question A mass is attached to a spring and allowed to oscillate vertically.
%  Which of the following would \emph{not} change the period to the oscillation?
%  \begin{choices}
%    \choice Double the mass and double the spring constant
%    \choice Double the amplitude of vibration and the double the mass
%    \choice Double the gravitational field strength $g$ and double the mass
%    \choice Double the gravitational field strength $g$ and double the spring
%    constant
%    \choice Double the gravitational field strength $g$ and quadruple the mass
%  \end{choices}
%
%  \question A particle oscillated with simple harmonic motion with no damping.
%  Which of the following statements about the acceleration of the particle is
%  true?
%  \begin{choices}
%    \choice It has a value of \SI{9.8}{\metre\per\second\squared} when the
%    oscillation is vertical
%    \choice It is zero when the spring is minimum
%    \choice It is proportional to the frequency
%    \choice It is zero throughout the oscillation
%    \choice It is zero when the speed is maximum
%  \end{choices}
%
%  \question At what position does the mass attached to a spring in simple
%  harmonic motion have the greatest magnitude of acceleration? ($A$ is
%  amplitude)
%  \begin{choices}
%    \choice $-A$
%    \choice $-A/2$
%    \choice $0$
%    \choice $A/2$
%    \choice $A/4$
%  \end{choices}
%
%  \item A spring mass system oscillated up and down on Earth. When is the
%  kinetic energy the greatest?
%  \begin{choices}
%    \choice When it is passing through equilibrium
%    \choice At the top of its motion
%    \choice At the bottom of its motion
%    \choice When its gravitational potential energy is the greatest
%    \choice When its elastic energy is the greatest
%  \end{choices}
%  
%  \item An object with a mass $M$ is suspended from an elastic spring with a
%  spring constant $k$. The object oscillates with period $T$ on the surface of
%  Earth. If the oscillating system is moved to the surface of Moon, how it will
%  change the period of oscillations? Acceleration due to gravity on moon is
%  aproximately $\dfrac16 g_\text{Earth}$.
%  \begin{choices}
%    \choice The period is increased by factor of approximately $\sqrt6$
%    \choice The period is increased by factor of approximately 6
%    \choice The period is decreased by factor of approximately $\sqrt6$
%    \choice The period is decreased by factor of approximately 6
%    \choice The period remains the same
%  \end{choices}
%  \newpage
%  
%  \item A simple pendulum of mass $M$ and length $l$ is moved from the
%  Earth to the Moon. How does it change the period of oscillations?
%  \begin{choices}
%    \choice The period is increased by factor of approximately $\sqrt6$
%    \choice The period is increased by factor of approximately 6
%    \choice The period is decreased by factor of approximately $\sqrt6$
%    \choice The period is decreased by factor of approximately 6
%    \choice The period remains the same
%  \end{choices}
%  
%  \item A mass hangs from a vertical spring and is initially at rest. A
%  person then pulls down on the mass, stretching the spring. Does the total
%  mechanical energy of this system (the mass, the spring and Earth) increase,
%  decrease, or stay the same? Explain.
%  \vspace{\stretch 1}
%
%  \item A bungee jumper of mass \SI{75}{\kilo\gram} is standing on a
%  platform \SI{53}{\metre} above a river. The length of the unstretched bungee
%  cord is \SI{11}\metre. The spring constant of the cord is
%  \SI{65.5}{\newton\per\metre}. Calculate the jumper's speed at \SI{19}{\metre}
%  below the bridge on the first fall. 
%  \vspace{\stretch 2}
%  \newpage
%  
%  \item A model car of mass \SI{5.0}{\kilo\gram} slides down a frictionless
%  ramp into a spring with spring constant $k=\SI{4.9}{\kilo\newton\per\metre}$.
%  The spring experiences a maximum compression of \SI{22}{\centi\metre}.
%  \begin{center}
%    \pic{.35}{carramp}
%  \end{center}
%  \begin{enumerate}[itemsep=3pt]
%    \item Determine the height of the initial release point.
%    \item Calculate the speed of the model car when the spring has been
%    compressed 15 cm.
%    \item Determine the maximum acceleration of the car after it hits the
%    spring.
%  \end{enumerate}
%  \vspace{\stretch1}
%  
%%  \item Suppose you set a spring with spring constant
%%  \SI{4.5}{\newton\per\metre} into damped harmonic motion at noon, measuring
%%  its maximum displacement from equilibrium to be \SI{.75}\metre. When you
%%  return 15 minutes later, the spring is still oscillating, but its maximum
%%  displacement has decreased to \SI{.50}\metre.
%%  \begin{enumerate}[itemsep=3pt]
%%    \item Determine how much energy the system has lost.
%%    \item What is the power loss of the system?
%%  \end{enumerate}
%%  \vspace{\stretch1}
%%  \newpage
%  
%  \item A simple pendulum is \SI{5.0}{\metre} in length.
%  \begin{enumerate}[itemsep=3pt]
%    \item What is the period of simple harmonic motion for this pendulum if it
%    is located in an elevator accelerating upward at
%    \SI{5.0}{\metre\per\second\squared}?
%    \label{parta}
%
%    \item What is the answer to part (\ref{parta}) if the elevator is
%    accelerating downward at \SI{5.0}{\metre\per\second\squared}?
%    %\item What is the period of simple harmonic motion for this pendulum if it
%    %is placed in a truck that is accelerating horizontally at
%    %\SI{5.}{\metre\per\second\squared}?
%  \end{enumerate}
%  \vspace{1.5in}
%  
%%  \item Ball $1$ has a mass of 2.0 kg and is suspended with a 3.0 m rope
%%  from a post so that the ball is stationary. Ball 2 has a mass of 4.0 kg and
%%  is tied to another rope. The second rope also measures 3.0 m but is held at a
%%  \ang{60} angle, as shown in the figure below. When Ball 2 is released, it
%%  collides, head-on, with ball 1 in an elastic collision.
%%  \begin{enumerate}[itemsep=3pt]
%%  \item Calculate the speed of each ball immediately after the first collision.
%%  \item Calculate the maximum height of each ball after the first collision.
%%  \item If Ball 2 is allowed to oscillate freely after the collision, what is 
%%    its period of oscillation?
%%  \end{enumerate}
%%  \begin{center}
%%    \pic{.35}{pendulum}
%%  \end{center}
%%  \newpage
%  
%%  \item A 20 g particle moves in simple harmonic motion with a frequency of
%%  3.0 Hz and an amplitude of 5.0 cm.
%%  \begin{enumerate}[itemsep=3pt]
%%  \item Through what total distance does the particle move during one cycle of
%%    its motion?
%%  \item What is its maximum speed? Where does that occur?
%%  \item Find the maximum acceleration of the particle. Where does that occur?
%%  \end{enumerate}

\item Military specifications often call for electronic devices to be able to
  withstand accelerations of $10g$ (i.e.\
  \SI{98.1}{\metre\per\second\squared}). To make sure their products meet this
  specification, manufacturers test them using a shaking table that can vibrate
  a device at various specified frequencies and amplitudes. If a device is
  given a vibration of amplitude 1.5 cm, what should the frequency be?

\item In heavy seas, the bow of a ship undergoes a simple harmonic vertical
  pitching motion with a period of \SI{8.0}{\second} and an amplitude of
  \SI{2.0}\metre.
  \begin{enumerate}[itemsep=3pt]
  \item What is the maximum vertical velocity of the ship's bow?
  \item What is the maximum acceleration?
  \item An \SI{80}{\kilo\gram} sailor is standing on the scale in the bunkroom
    in the bow. What are the maximum and minimum readings on the scale, in
    newtons?
  \end{enumerate}

\item A block of wood slides on a frictionless horizontal surface. It is
  attached to a spring and oscillates with a period of $T=\SI{.80}\second$. A
  second block rests on top of the first block. The coefficient of static
  friction between thew two blocks is $\mu=0.25$.
  \begin{enumerate}[itemsep=3pt]
  \item If the amplitude of the oscillation is \SI1{\centi\metre}, will the
    block on top slip?
  \item What is the greatest amplitude of oscillation for which the top block
    will not slip?
  \end{enumerate}
%  \vspace{\stretch1}
%  
%%  \item An object of mass $m_1$ sliding on a frictionless horizontal surface is
%%  attached to a spring of force constant $k$. It oscillates with an amplitude
%%  $A$. When the spring is at its greatest extension and the mass is
%%  instantaneously at rest, a second object of mass $m_2$ is placed on top of it.
%%  \begin{enumerate}[itemsep=3pt]
%%  \item What is the smallest value the coefficient of static friction $\mu_s$
%%    can have if the second object is not to slip on the first?
%%  \item Explain how the total energy $E$, amplitude $A$, frequency $f$ and
%%    period $T$ of the system are changed by placing $m_2$ on $m_1$.
%%  \end{enumerate}
\end{enumerate}

