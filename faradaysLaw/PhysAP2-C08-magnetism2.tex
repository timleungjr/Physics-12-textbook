\chapter{Magnetic Induction}
\label{chapter:faradays-law}


\section{Faraday's Law}

A brief description of Michae Faraday's experiments here.

\subsection{Magnetic Flux}
%  \textbf{Question:} If a current-carrying wire can generate a magnetic field,
%  can a magnetic field affect the current in a wire?
%
%  \vspace{.3in}\textbf{Answer:} Yes, sort of\ldots
%
%  \vspace{.3in}To understand how to \emph{induce} a current by a magnetic field,
%  we need to look at ``fluxes''.

%    \pic1{flux2}
%  
%    The strict mathematical definition (i.e.\ using calculus) of
%    \textbf{magnetic flux} is defined as:
%    
\begin{equation}
  \Phi_m=\int_\mathcal{S}\vec B\cdot d\vec A
\end{equation}
%    For a uniform magnetic field, this simplies to:
\begin{equation}
  \boxed{ \Phi_m=\vec B\cdot\vec A }
\end{equation} 
%    where $\vec B$ is the magnetic field, and $\vec A$ is an area of wire where
%    the magnetic field passes through. The direction of the area vector is
%    pointing outwards


%
%
%
%{Magnetic Flux Over a Closed Surface}
%  The unit for magnetic flux is a ``weber'' (\si\weber), in honor of German
%  physicist Wilhelm Weber, who invented the electromagnetic telegraph with Carl
%  Gauss:
%
%  \eq{-.1in}{
%    \SI1\weber=\SI1{\tesla\metre\squared}
%  }
%  
The magnetic flux over a closed surface is always zero, indicating that
magnetic field lines can neither have beginnings nor ends. This is known as
Gauss's law for magnetism.
%  
%  \eq{-.1in}{
%    \oint_\mathcal{S}\vec B\cdot d\vec A=0
%  }



\subsection{Changing Magnetic Flux}

Changes to magnetic flux can be due to a number of reasons:
\begin{enumerate}
\item\textbf{Changing magnetic field strength:} If the magnetic field is
  created by a time-dependent source, for example, from an alternating
  current, i.e.
  \begin{displaymath}
    B=B(t)
  \end{displaymath}
\item\textbf{Changing orientation of magnetic field:} If the orientation of
  the wire loop is changing relative to the magnetic field with time, i.e.
  \begin{displaymath}
    \theta=\theta(t)
  \end{displaymath}
\item\textbf{Changing area:} If the surface area from which the flux is
  calculated is changing, i.e.
  \begin{displaymath}
    A=A(t)
  \end{displaymath}
\end{enumerate}

%{When Magnetic Flux is Changing}
%  \begin{itemize}
%  \item When the magnetic flux $\Phi_m$ is changing, an electromotive force
%    (\emph{emf}, $\mathcal E$) is created in the wire.
%  \item Unlike in a circuit, where the \emph{emf} is concentrated at the
%    terminals of the battery, the induced \emph{emf} is spread across the
%    entire wire.
%    \item Since \emph{emf} is work per unit charge, that means that there is an
%      electric field inside the wire to move the charges.
%    \end{itemize}

Faraday's law states that the rate of change of magnetic flux produces an
electromotive force:
\begin{equation}
  \boxed{
    \mathcal E={\color{red}-}\frac{\Delta\Phi_m}{\Delta t}
  }
\end{equation}
The negative sign {\textcolor{red}{highlighted in red}} is the result of Lenz's
law, which is related to the conservation energy.



\section{AC Generators}

%  A simple AC (alternating current) generator makes use of the fact that a 
%  coil rotating against a fixed magnetic field has a changing flux.
%  \begin{center}
%    \pic{.45}{generator}
%  \end{center}
%  Let's say the permanent magnets produce a uniform magnetic field $B$, and the
%  coil between them has $N$ turns, and an area $A$. Now let's say that the coil
%  is rotating with an angular frequency $\omega$.

%    \pic1{generator}
%
%    When the coil is turning at a constant rate $\omega$, the angle between the
%    coil and the magnetic field is:
%    
\begin{equation}
  \theta=\omega t+\theta_0
\end{equation} 
%
%    \vspace{-.1in}where $\theta_0$ is the initial angle (not very important
%    conceptually). The magnetic flux through the coil is:

\begin{equation}
  \Phi_m=NAB\cos\theta=NAB\cos(\omega t+\theta_0)
\end{equation}
%    
%    \vspace{-.2in}as the coil turns.

The electromotive force \emph{emf} $\mathcal E$ produced is therefore the
rate of change of the magnetic flux:
\begin{equation}
  \mathcal E =\underbracket[1pt]{NAB\omega}_{\mathcal E_0}\sin(\omega t+\delta)
\end{equation}


\section{Motional EMF}
%  {What happens when I slide the rod to the right?}

%    \pic1{motional-emf-1}
%
%    When sliding the rod to the right with speed $v$, the magnetic flux through
%    the loop (and its rate of change) is:
%
%    \vspace{-.3in}{\large
%      \begin{align*}
%        \Phi_m   &=BA=B\ell x\\
%        \mathcal E&=B\ell v
%      \end{align*}
%    }
%    
%    We can use the Lorentz force law on the charges on the rod to find the
%    direction of the current $I$.


%    \begin{tikzpicture}[scale=1.5,thick]
%      \draw (1.5,0) to[battery1,l_=$\mathcal E$] (1.5,1.5)--(0,1.5)
%      to[R,l_=$R$] (0,0)--(1.5,0);
%    \end{tikzpicture}
%    
%    \begin{itemize}
%    \item An equivalent circuit is shown on the left
%    \item The amount of current can be found using Ohm's law: $V=IR$
%    \item Note that the ``motional emf'' produced is spread over the entire
%      circuit
%      \begin{itemize}
%      \item In constrast, in a voltaic cell (or battery), the \emph{emf} is
%        concentrated between the two terminals.
%      \end{itemize}
%    \end{itemize}



\section{Lenz's Law}
%  Something very interesting happens when the current starts running on the
%  wire.
%  \begin{center}
%    \pic{.35}{motional-emf-2}
%  \end{center}
%  It produces an ``induced magnetic field'' out of the page, in the opposite
%  direction as the field that generated the current in the first place.

%
%
%
%  \begin{center}
%    \fbox{
%      \begin{minipage}{.7\textwidth}
%        \textbf{LENZ'S LAW}\\
%        The induced \emph{emf} and induced current are in such are
%        direction as to oppose the change that produces them
%      \end{minipage}
%    }
%  \end{center}
%
%  \vspace{.2in}So basically, the conservation of energy


%    \centering
%    \begin{tikzpicture}[thick]
%      \foreach \x in {1,...,6}
%      \foreach \y in {0,...,3} \node[cyan] at (\x,\y) {$\times$};
%      \draw[line width=.6mm](1.3,.5)--(6,.5);
%      \draw[line width=.6mm](1.3,2.5)--(6,2.5);
%      \draw[line width=1.7mm,red!60!black](4.5,.1)--(4.5,3.3);
%      \draw[<->](2.8,.53)--(2.8,2.47)node[midway,fill=black!2]{$\ell$};
%      \draw[|<->](1.3,.2)--(4.5,.2) node[pos=.42,fill=black!2]{$x$};
%      \draw[very thick,->](4,3.5)--(5,3.5) node[right]{$\vec v$};
%      \draw[very thick,->,orange](4.58,1.5)--(5.5,1.5) node[right]{$\vec F_a$};
%      \draw (1.3,.47) to[R=$R$] (1.3,2.53);
%      \node[cyan] at (1.5,3.2) {$\vec B_\text{in}$};
%    \end{tikzpicture}
%    
%    When sliding the rod to the right with constant velocity $\vec v$, magnetic
%    flux $\Phi_m$ through the circuit loop is:
%
\begin{equation}
  \Phi_m=B{\color{magenta}A}=B{\color{magenta}\ell x}
\end{equation}
%
%    \vspace{-.1in}Induced emf is:
%    
\begin{equation}
  \mathcal E
  =\frac{\Delta\Phi_m}{\Delta t}
  =\frac{B\ell{\color{violet}x}}{\color{violet}\Delta t}
  =B\ell{\color{violet}v}
\end{equation}
The induced current through the circuit can easily be caclulated using Ohm's
law:
\begin{equation}
  I=\frac{\mathcal E}R=\frac{B\ell v}R
\end{equation}


\subsection{Induced Current from Motional EMF}

%    \centering
%    \begin{tikzpicture}
%      \foreach \x in {1,...,6}
%      \foreach \y in {0,...,3} \node[cyan] at (\x,\y) {$\times$};
%      \draw[lightgray,line width=.6mm](1.3,.5)--(6,.5);
%      \draw[lightgray,line width=.6mm](1.3,2.5)--(6,2.5);
%      \draw[line width=1.7mm,lightgray](4.5,.1)--(4.5,3.3);
%      \draw[very thick,->](4,3.5)--(5,3.5) node[right]{$\vec v$};
%      \draw[thick] (1.3,.47) to[R=$R$] (1.3,2.53);
%      \node[cyan] at (1.5,3.2) {$\vec B_\text{in}$};
%      \draw[violet,ultra thick,->](4.5,.7)--(4.5,2.2) node[above]{$I$};
%      \draw[violet,ultra thick,->](4,2.5)--(2,2.5) node[midway,above]{$I$};
%      \draw[violet,ultra thick,->](2,.5)--(4,.5) node[midway,above]{$I$};
%    \end{tikzpicture}
%
%    Induced current $I$ runs counter-clockwise, because it must produce a
%    magnetic field that \emph{oppose the increase in magnetic flux}.
%
%    \vspace{.2in} The power dissipated by the resistor is:%(assuming that it is
%    %ohmic)
\begin{equation}
  P=I^2R=\left[\frac{B\ell v}R\right]^2R=\frac{B^2\ell^2v^2}R
\end{equation}
%
%    \textbf{But where did the energy come from?}


\subsection{Magnetic Force on Induced Current}

%    \centering
%    \begin{tikzpicture}
%      \foreach \x in {1,...,6}
%      \foreach \y in {0,...,3} \node[cyan] at (\x,\y) {$\times$};
%      \draw[line width=.6mm](1.3,.5)--(6,.5);
%      \draw[line width=.6mm](1.3,2.5)--(6,2.5);
%      \draw[line width=1.7mm,red!70!black](4.5,.1)--(4.5,3.3);
%      \draw[thick,<->](2.8,.53)--(2.8,2.47)node[midway,fill=black!2]{$\ell$};
%      \draw[very thick,->](4,3.5)--(5,3.5) node[right]{$\vec v$};
%      \draw[very thick,->,orange](4.58,1.5)--(5.5,1.5) node[right]{$\vec F_a$};
%      \draw[thick](1.3,.47) to[R=$R$] (1.3,2.53);
%      \node[cyan] at (1.5,3.2) {$\vec B_\text{in}$};
%      \foreach \yy in {.6,.9,...,2.5}
%      \draw[orange,very thick,->] (4.42,\yy)--(3.8,\yy);
%      \node[left,orange] at (3.9,1.5){$\vec F_m$};
%      \draw[black!10,ultra thick,->](4.5,.7)--(4.5,2.2) node[above]{$I$};
%    \end{tikzpicture}
%    
%    The induced current $I$ in the rod is moving in a magnetic field, therefore
%    it will experience a magnetic force $\vec F_m$ with a magnitude of:
%
\begin{equation}
  F_m=I\ell B = \left[\frac{B\ell v}R\right]\ell B
  =\frac{B^2\ell^2v}R
\end{equation}
For a constant $\vec v$, an external force of $F_a=F_m$ must be applied,
(see figure left). The power generated by the applied force is:
\begin{equation}
  P = F_av=F_mv=\left[\frac{B^2\ell^2v}R\right]v=\frac{B^2\ell^2v^2}R
\end{equation}



\section{Energy Conservation}

%    \centering
%    \begin{tikzpicture}
%      \foreach \x in {1,...,6}
%      \foreach \y in {0,...,3} \node[cyan] at (\x,\y) {$\times$};
%      \draw[line width=.6mm](1.3,.5)--(6,.5);
%      \draw[line width=.6mm](1.3,2.5)--(6,2.5);
%      \draw[line width=1.7mm,red!70!black](4.5,.1)--(4.5,3.3);
%      \draw[very thick,->](4,3.5)--(5,3.5) node[right]{$\vec v$};
%      \draw[very thick,->,orange](4.58,1.5)--(5.5,1.5) node[right]{$\vec F_a$};
%      \draw[thick](1.3,.47) to[R=$R$] (1.3,2.53);
%      \node[cyan] at (1.5,3.2) {$\vec B_\text{in}$};
%    \end{tikzpicture}

The power generated by the applied force is equal to the power dissipated by
the resistor that we have calculated:
\begin{equation}
  P = \frac{B^2\ell^2v^2}R
\end{equation}
Therefore, the energy in the circuit comes from the positive work done by the
applied force, and very importantly, our calculations confirm the law of
conservation of energy.

