\documentclass[12pt,compress,aspectratio=169]{beamer}
\input{../mybeamer}
\usepackage{hyperref}

\newcolumntype{g}{>{\columncolor{gray!15}}c}

\title{Class 10: Physics of Sound and Music}
\subtitle{AP Physics 2}
\input{../me}
\input{../mycommands}


\begin{document}

\begin{frame}
  \titlepage
\end{frame}



\section[Properties]{Properties of Sound}

\subsection{Compression/Rarefaction}

\begin{frame}{Sound Wave}
  \textbf{Sound wave}\footnote{Also known as an \textbf{acoustic wave}} is a:
  \begin{itemize}
  \item \textbf{mechanical wave}
    \begin{itemize}
    \item Requires a medium---air (or any gas), a liquid, or a solid
    \item Does not propagate in a vacuum
    \end{itemize}
  \item \textbf{longitudinal wave}
    \begin{itemize}
    \item The vibrations of the medium is \emph{parallel} to the direction of
      travel of the wave
    \end{itemize}
  \item \textbf{pressure wave}
    \begin{itemize}
    \item The crests and troughs of the wave corresponds to regions in the
      medium where the pressure is at the highest or lowest
    \end{itemize}
  \end{itemize}
\end{frame}



\begin{frame}{Example: Speaker}
  \begin{itemize}
  \item Air molecules near the speaker is disturbed by the vibration of the
    speaker cone
  \item The vibrations cause the air molecules to compress and
    expand, creating a wave that travels through air
  \end{itemize}
  \begin{center}
    \pic{.7}{images/speaker}
  \end{center}
\end{frame}



\begin{frame}{Longitudinal Wave Simulation}
  \begin{center}
    \pic{.7}{images/Lwave-v8-0}\\
    \href{https://ossfiles.ca/index.php/s/ec938CYeAo5LgtL}{[Link to Animation]}
  \end{center}
  This simulation was created by Professor Daniel A.\ Russell at the
  Pensylvannia State University.
  (\href{http://acs.psu.edu/drussell/demos.html}
  {acs.psu.edu/drussell/demos.html})
\end{frame}



\subsection{Speed of Sound}

\begin{frame}{Speed of Sound in Dry Air}
  The speed of sound $v_s$ in dry air can be calculated by this equation:

  \eq{-.1in}{
    \boxed{
      v_s=331.3\sqrt{1+\frac{T_C}{273.15}}=331.3\sqrt{\frac T{273.15}}
    }
  }

  where $T_c$ is the temperature in \emph{degree Celsius}, and $T$ is the
  absolute temperature in \emph{kelvin}. For temperatures near \SI0\celsius,
  the speed of sound can be further simplified by linearization of the above
  equation:

  \eq{-.1in}{
    v_s=331.3+0.606T_C
  }  
\end{frame}



\begin{frame}{Speed of Sound in Liquids and Solids}
  \begin{columns}
    \column{.6\textwidth}
    The speed of sound in a liquid depends on the ``bulk modulus'' $K$ of the
    liquid, and density $\rho$:
      
    \eq{-.13in}{
      v_s = \sqrt{\frac K\rho}
    }
   
    \vspace{-.07in}while the speed of sound in a solid depends on the
    ``Young's modulus'' $E$ of the solid and density $\rho$:

    \eq{-.13in}{
      v_s = \sqrt{\frac E\rho}
    }

    \vspace{-.07in}(You are not responsible for knowing these equations in
    AP Physics 2.)

    \column{.35\textwidth}
    {\small
      \begin{tabular}{l|c}
        \rowcolor{cyan!30}
        \textbf{Material} & \textbf{Speed} (\si{\metre\per\second}) \\
        \rowcolor{pink!70}
        \multicolumn{2}{c}{Gases (\SI0\celsius, \SI{101}{\kilo\pascal})} \\
        Carbon dioxide & 259 \\
        Oxygen         & 316 \\
        Air            & 331 \\
        Helium         & 965 \\
        \rowcolor{pink!70}
        \multicolumn{2}{c}{Liquids (\SI{20}\celsius)} \\
        Ethanol        & 1162 \\
        Fresh water    & 1482 \\
        Seawater       & 1440-1500 \\
        \rowcolor{pink!70}
        \multicolumn{2}{c}{Solids} \\
        Copper         & 5010 \\
        Glass          & 5640 \\
        Steel          & 5960
      \end{tabular}
    }
  \end{columns}
\end{frame}



\begin{frame}{Example Problem}
  \textbf{Example}: Suppose the room temperature of a classroom is
  \SI{21}\celsius. Calculate the speed of sound in the classroom.
\end{frame}



%\begin{frame}{Example Problem}
%  \textbf{Example}: The temperature is \SI{4.0}{\celsius} one morning as
%  Martita hikes through a canyon. She shouts at the canyon wall, and
%  \SI{2.8}{\second} later hears an echo. How far away is the canyon wall?
%\end{frame}



\subsection{Loudness}

\begin{frame}{Intensity of Sound}
  The average power transmitted $P_\text{avg}$ by a sound wave is proportional
  to the square of the ``pressure amplitude'' $\Delta p$, which is the pressure
  difference from atmospheric pressure.

  \eq{-.15in}{
    P\propto(\Delta p)^2
  }

  \vspace{-.2in}The \textbf{loudness} of sound, which is the sound wave's
  \textbf{intensity}, is the power of the sound wave divided by the area ($S$)
  that the sound wave passes through:

  \eq{-.1in}{
    \boxed{I=\frac PS}
  }
  
  \vspace{-.05in}When sound is emitted from a point source, the area that the
  wavefront passes through is $S=4\pi r^2$ (surface area of a sphere), where
  $r$ is the distance from the source.
\end{frame}



\begin{frame}{Intensity of Sound}
  As the sound wave travels outwards from a source, the surface area where the
  wavefront pass through increases with $r^2$. Therefore, intensity decreases:

  \eq{-.1in}{
    I\propto\frac1{r^2}
  }
  \begin{center}
    \pic{.38}{images/isqb}
  \end{center}
  As the intensity drops further away from the source, the amplitude $\Delta p$
  of the sound wave also decreases
\end{frame}



\begin{frame}{Threshold of Hearing}
  The lowest detectable sound intensity is called the \textbf{threshold of
    hearing} $I_0$, defined as:

  \eq{-.1in}{
    I_0=\SI{e-12}{\watt\per\metre\squared}
  }
  
  While the sound intensity at the \textbf{threshold of pain} can damage a
  human ear:

  \eq{-.1in}{
    I_p=\SI1{\watt\per\metre\squared}
  }
\end{frame}



\begin{frame}{The Decibel}
  The \textbf{decibel} is defined as the logarithm of the ratio intensity of
  sound $I$ compared to the threshold of hearing $I_0$:
  
  \eq{-.1in}{
    \beta=10\log_{10}\left[\frac I{I_0}\right]
  }
  \begin{itemize}
  \item The threshold of hearing is \SI0{dB} while the threshold of pain is
    \SI{120}{dB}
  \item Humans perceive a doubling of loudness when intensity is increased by a
    factor of 10 (i.e.\ \SI{10}{dB})
  \end{itemize}
\end{frame}



\begin{frame}{Sound Intensity in Decibels}
  \centering
  \begin{tabular}{l|c|c}
    \rowcolor{pink}
    \textbf{Source} & $I$ (\si{\watt\per\metre\squared}) & dB \\ \hline
    Threshold of Hearing      & \num{e-12} & 0   \\
    Rustling Leaves           & \num{e-11} & 10  \\
    Whisper                   & \num{e-10} & 20  \\
    Normal conversation       & \num{e-6}  & 60  \\
    Busy street traffic       & \num{e-5}  & 70  \\
    Vacuum cleaner            & \num{e-4}  & 80  \\
    Large orchestra           & \num{6e-3} & 98  \\
    MP3 player maximum Level  & \num{e-2}  & 100 \\
    Front row of rock concert & \num{e-1}  & 110 \\
    Military jet during take-off   & \num{e2} & 140 \\
    Instant perforation of eardrum & \num{e4} & 160
  \end{tabular}
\end{frame}



\subsection{Frequency/Pitch}

\begin{frame}{Frequency and Pitch}
  \begin{columns}
    \column{.65\textwidth}
    \begin{itemize}
    \item Frequencies of sound are referred to as its \textbf{pitch}
    \item Audible range for an adult human is approximately \SI{20}{\hertz} to
      \SI{20000}\hertz
    \item\textbf{Infrasound}: frequencies below audible range
    \item\textbf{Ultrasound}: frequencies above audible range, e.g.
      \begin{itemize}
      \item Dog whistles
      \item Medical ultrasound devices
      \end{itemize}
    \end{itemize}

    \column{.35\textwidth}
    \pic1{images/ultrasound}\\
    {\footnotesize Medical ultrasound devices usually use frequencies between
    \SI1{\mega\hertz} to \SI{20}{\mega\hertz}.\par}
  \end{columns}
\end{frame}



\section{Doppler Effect}

\subsection{Mach Number}

\begin{frame}{Mach Number}
  Speeds close to the speed of sound are often expressed in terms of its ratio
  to the speed of sound. This is called the \textbf{Mach number} ($M$):
  
  \eq{-.1in}{
    \boxed{M=\frac v{v_s}}
  }
  \begin{center}
    \begin{tabular}{l|c|c}
      \rowcolor{pink}
      \textbf{Quantity} & \textbf{Symbol} & \textbf{SI Unit} \\ \hline
      Mach number          & $M$   & (no units) \\
      Speed of the object  & $v$   & \si{\metre\per\second} \\
      Local speed of sound & $v_s$ & \si{\metre\per\second}
    \end{tabular}
  \end{center}
  \begin{itemize}
  \item Subsonic: $M<1$
  \item Supersonic: $1<M<5$
  \item Hypersonic: $M\geq5$
  \end{itemize}
\end{frame}



\begin{frame}{Sound from a Stationary Source}
  \begin{columns}
    \column{.38\textwidth}
    \begin{tikzpicture}[scale=.55]
      \begin{scope}[very thick]
        \draw[red!80] circle (1);
        \draw[red!60] circle (2);
        \draw[red!40] circle (3);
        \draw[red!20] circle (4);
      \end{scope}
      \fill circle (.075) node[right]{source};
      \draw[<->] (-1,0)--(-2,0) node[midway,below]{$\lambda$};
      \foreach \theta in {0,20,...,340}{
        \draw[axes,rotate=\theta] (3.7,0)--(4.7,0);
      }
    \end{tikzpicture}

    \column{.62\textwidth}
    When a sound is emitted from a stationary point source, the sound wave moves
    radially outward from the origin:
    \begin{itemize}
    \item Loudness (intensity) drops farther away from the source, proportional
      to $\dfrac1{r^2}$
    \item All points hear the same wavelength and frequency (pitch) of sound
    \end{itemize}
  \end{columns}
\end{frame}



\begin{frame}{Sound from a Moving Source}
  \vspace{.2in}
  \begin{columns}
    \column{.4\textwidth}
    \begin{tikzpicture}[scale=.7]
      \begin{scope}[very thick]
        \draw[red!20] circle (4);
        \draw[red!40] (0.5,0) circle (3);
        \draw[red!60] (1.0,0) circle (2);
        \draw[red!80] (1.5,0) circle (1);
      \end{scope}
      \draw[vectors] (0,0)--(2.7,0) node[above]{$v$};
      \fill circle (.075) node[below]{1};
      \fill (.5,0) circle (.075) node[below]{2};
      \fill (1,0) circle (.075) node[below]{3};
      \fill (1.5,0) circle (.075) node[below]{4};
      \draw[<->] (-4,0)--(-2.5,0) node[midway,above]{$\lambda_1$};
      \draw[<->] (4,0)--(3.5,0);
      \node (a) at (4.5,-1.2) {$\lambda_2$};
      \draw (3.75,-.1) to[out=270,in=180] (a);
    \end{tikzpicture}
      
    \column{.6\textwidth}
    When sound is emitted from a subsonic \emph{moving} source, the diagram
    looks different. In this case, the sound source is moving to the right,
    from 1 to 4
    \begin{itemize}
    \item When the source is moving \emph{towards the observer}, the
      wavelength $\lambda_2$ decreases, and the perceived frequency increases.
    \item When the source is moving \emph{away from the observer}, wavelength
      $\lambda_1$ increases, and the perceived frequency decreases.
    \end{itemize}
    This is called the \textbf{Doppler effect}.
  \end{columns}
\end{frame}



\begin{frame}{Doppler Effect}
  We experience Doppler effect every time an ambulance speeds by us with its
  sirens on.
  \begin{center}
    \pic{.6}{images/toronto-ambulance}
  \end{center}
  When it is moving towards us, the pitch of the siren is high, but
  the moment it passes us, the pitch decreases.
\end{frame}



\begin{frame}{Doppler Effect}
  When a wave source is moving at a speed $v_\text{src}$ and an observer is
  moving at $v_\text{ob}$, the perceived frequency $f'$ is shifted:

  \eq{-.1in}{
    \boxed{f'=\frac{v_s+v_\text{ob}}{v_s-v_\text{src}}f}
  }
  \begin{center}
    \begin{tabular}{l|c|c}
      \rowcolor{pink}
      \textbf{Quantity} & \textbf{Symbol} & \textbf{SI Unit}\\
      Perceived frequency & $f'$  & \si\hertz \\
      Source frequency    & $f$   & \si\hertz \\
      Speed of sound      & $v_s$ & \si{\metre\per\second}\\
      Speed of source & $v_\text{src}$ & \si{\metre\per\second}\\
      Speed of observer & $v_\text{ob}$ & \si{\metre\per\second}
    \end{tabular}
  \end{center}
  $v_\text{src}$ and $v_\text{ob}$ are positive when they move towards each
  other, and negative when they move away.
\end{frame}



\subsection{Sonic Boom}

\begin{frame}{Sound Source at Sonic Speed}
  \begin{columns}
    \column{.4\textwidth}
    \begin{tikzpicture}[scale=.55]
      \begin{scope}[very thick]
        \draw[red!20] circle (4);
        \draw[red!40] (1,0) circle (3);
        \draw[red!60] (2,0) circle (2);
        \draw[red!80] (3,0) circle (1);
      \end{scope}
      \draw[vectors] (0,0)--(4.5,0) node[right]{$v=v_s$};
      \fill circle (.075) node[below]{1};
      \fill (1,0) circle (.075) node[below]{2};
      \fill (2,0) circle (.075) node[below]{3};
      \fill (3,0) circle (.075) node[below]{4};
      \fill (4,0) circle (.075) node[below]{5};
      \draw[dashed] (4,7)--(4,-4)
      node[pos=.05,right]{\scriptsize No disturbance in front of the shock}
      node[pos=.05,left]{\scriptsize Disturbed air flow behind the shock};
    \end{tikzpicture}
      
    \column{.6\textwidth}
    When sound source is moving at the speed of sound ($M=1$):
    \begin{itemize}
    \item Wavefronts are bunched up just in front of the source
    \item Since sound wave is a pressure wave, right in front of the sound
      source, there is a large change in pressure (called a \textbf{shock
        wave})
    \item When the shock passes an observer, a loud bang can be heard (aka
      \textbf{sonic boom})
    \end{itemize}
  \end{columns}
\end{frame}



\begin{frame}{Sound from a Supersonic Source}
  \begin{columns}
    \column{.45\textwidth}
    \begin{tikzpicture}[scale=.55]
      \fill (3.5,-3) circle (.1) node[right]{observer};
      \begin{scope}[very thick]
        \draw[red!20] circle (4);
        \draw[red!40] (1.8,0) circle (3);
        \draw[red!60] (3.6,0) circle (2);
        \draw[red!80] (5.4,0) circle (1);
      \end{scope}
      \fill circle (.075) node[below]{1};
      \fill (1.8,0) circle (.075) node[below]{2};
      \fill (3.6,0) circle (.075) node[below]{3};
      \fill (5.4,0) circle (.075) node[below]{4};
      \fill (7.2,0) circle (.075) node[below]{5};

      \draw[vectors] (-3.5,0)--(9,0) node[above left]{$v>v_s$};
      \draw[dashed,rotate around={213.8:(7.2,0)}] (7.2,0)--(15.5,0);
      \draw[dashed,rotate around={146.3:(7.2,0)}] (7.2,0)--(17.5,0);
      \draw[axes] (-2.8,0) arc (180:146.3:10) node[pos=.4,right]{$\gamma$};
    \end{tikzpicture}
      
    \column{.55\textwidth}
    When sound source is moving at $M>1$, it out runs the sound that it makes:
    An \emph{oblique shock} is formed at an angle (called the
    \textbf{Mach angle}) given by:
    
    \eq{-.1in}{
      \gamma=\sin^{-1}\left(\frac 1M\right)
    }

    An observer does not hear the sound source until it has gone past!
  \end{columns}
\end{frame}



\begin{frame}{Bullet in Supersonic Flight}
  Generating a shock does not require an actual sound source. Any object
  moving through air creates a pressure disturbance. For example, a bullet in
  supersonic flight generates shock waves as it moves through air.
  \begin{columns}
    \column{.55\textwidth}
    \begin{itemize}
    \item The flow around this bullet is taken inside a \emph{shock tube} that
      generates a short burst of supersonic flow. A high-speed camera is used to
      take the photo.
    \item The Mach number changes as air flows around the bullet
    \end{itemize}

    \column{.45\textwidth}
    \begin{tikzpicture}
      \node at (0,0) {\pic{.9}{images/bullet2}};
      \node[thick,magenta,draw=magenta,fill=white] (A) at (2.6,1.4){
        \scriptsize shock wave};
      \draw[vectors,magenta] (A)--(.9,1.4);
    \end{tikzpicture}
  \end{columns}
\end{frame}



\begin{frame}{Duck in Water}
  A similar shock behaviour is observed when the duck swims in water, because
  the duck swims faster than the speed of the water wave, it also creates a
  cone shape.
  \begin{center}
    \begin{tikzpicture}
      \node at (0,0) {\pic{.6}{images/duck}};
      \node[thick,draw=blue,fill=white]
      at (2.3,.8) {\scriptsize undisturbed water};
      \node[thick,draw=blue,fill=white,above]
      (A) at (-2,1.5) {\scriptsize shock wave};
      \draw[vectors,blue] (A)--(-2,.3);
    \end{tikzpicture}
  \end{center}
\end{frame}



\section{Beats}

\begin{frame}{Visualizing Beat Frequency}
  Two waves ({\color{magenta}$\Psi_1$} and {\color{cyan}$\Psi_2$}) moving in the
  same medium\footnote{The two waves would be travelling at the same speed.}
  but with different frequencies\footnote{Think of two musical instruments
    playing out of tune from each other} go through regions of constructive and
  destructive interference:
  \begin{center}
    \begin{tikzpicture}[yscale=.66]
      \begin{scope}[->]
        \draw (0,0)--(11,0) node[right]{$x$};
        \draw (0,-5)--(0,2) node[right]{$y$};
        \draw (0,-3)--(11,-3) node[right]{$x$};
      \end{scope}
      \draw[vectors] (4,1.5)--(5,1.5) node[right]{$v$};
      \begin{scope}[smooth,samples=200,domain=0:10,thick]
        \draw[cyan] plot(\x,{sin(700*\x)});
        \draw[magenta] plot(\x,{sin(770*\x)});
        \draw[violet] plot(\x,{.8*(sin(700*\x)+sin(770*\x))-3});
      \end{scope}
      \begin{scope}[violet,dashed,smooth,samples=30,domain=0:10]
        \draw plot(\x,{ 1.6*cos(35*\x)-3});
        \draw plot(\x,{-1.6*cos(35*\x)-3});
      \end{scope}
      \node[right] at (10,1){\color{magenta}$\Psi_1$};
      \node[right] at (10,.4){\color{cyan}$\Psi_2$}; 
      \node[right] at (10,-2){\color{violet}$\Psi=\Psi_1+\Psi_2$}; 
      \foreach \x in {90,180,270} \draw[gray] (\x/35,-5)--(\x/35,1.5);
    \end{tikzpicture}
  \end{center}
\end{frame}



\begin{frame}{Beat Frequency}
  \textbf{Beat frequency} is the absolute value of the difference of the
  frequencies of the two component waves. At low frequencies, they sound like a
  pulsating ``whoomf''\footnote{If the beat frequency is in the audible
    range, it will be interpreted as an actual sound}. Musicians often use beat
  frequencies to determine if someone is playing in tune.

  \eq{-.1in}{
    \boxed{f_b=|f_2-f_1|}
  }

  \vspace{-.1in}
  \begin{center}
    \begin{tabular}{l|c|c}
      \rowcolor{pink}
      \textbf{Quantity} & \textbf{Symbol} & \textbf{SI Unit} \\ \hline
      Beat frequency              & $f_b$ & \si\hertz \\
      Frequencies component waves & $f_1$, $f_2$ & \si\hertz
    \end{tabular}
  \end{center}
  For those who are interested, the full derivation of the beat frequency can
  be found on the accompanied handout for this class, but it is not requised
  for Grade 11 Physics.
\end{frame}



\begin{frame}{Example Problem}
  \textbf{Example}: A tuning fork of unknown frequency is sounded at the same
  time as one of the frequency \SI{440}\hertz, resulting in the production of
  beats. Over 15 seconds, 45 beats are produced. What are the possible
  frequencies of the unknown tuning fork?
\end{frame}



\section{Harmonics}

\begin{frame}{Musical Instruments}
  When a musical instrument produces a sound at a certain frequency, it also
  produces many higher frequency sounds\footnote{the reason for this will
    become apparent later in the slides}
  \begin{itemize}
  \item The higher frequency sounds are \emph{whole-number multiples} of the
    lowest frequency
  \item e.g.\ a violin playing at \SI{440}{\hertz} produces sound waves at
    \begin{center}
      {\Large
        \vspace{.1in}
        \SI{440}\hertz, \SI{880}\hertz, \SI{1320}\hertz, \SI{1760}\hertz,
        \SI{2200}\hertz, \SI{2640}\hertz\ldots
        \vspace{.1in}
      }
    \end{center}
  \item Generally, the higher the frequency, the smaller the amplitude
  \item The overall quality of the sound comes from the sum of all the waves
    (principle of superposition). This is why a violin and a trumpet playing
    the same note will always sound different
  \end{itemize}
\end{frame}



\begin{frame}[t]{Harmonic Frequencies}
  The sound wave with the longest wavelength (and therefore lowest frequency)
  is called the \textbf{fundamental frequency}, or the \textbf{first harmonic}.
  \begin{center}
    \begin{tikzpicture}
      \draw[vectors] (1,1.3)--(1.5,1.3) node[right]{$v$};
      \draw[thick,blue,smooth,samples=30,domain=0:2.8] plot(\x,{sin(200*\x)});
    \end{tikzpicture}
    \hspace{.15in}
    \begin{tikzpicture}
      \draw[vectors] (1,1.3)--(1.5,1.3) node[right]{$v$};
      \draw[thick,red!20,smooth,samples=50,domain=0:2.8] plot(\x,{sin(400*\x)});
    \end{tikzpicture}
    \hspace{.15in}
    \begin{tikzpicture}
      \draw[vectors] (1,1.3)--(1.5,1.3) node[right]{$v$};
      \draw[thick,green!20,smooth,samples=70,domain=0:2.8]
      plot(\x,{sin(600*\x)});
    \end{tikzpicture}
  \end{center}
  Generally\footnote{This is, of course, not always the case. A \emph{muted}
  trumpet has its fundamental frequency mostly damped out.}, when a musical
  instrument produces a sound, the fundamental frequency is the one that is
  ``heard''
\end{frame}



\begin{frame}[t]{Harmonic Frequencies}
  A second wave is also created with half the wavelength and twice the
  frequency. It's called the \textbf{second harmonic}, or the
  \textbf{first overtone}.
  \begin{center}
    \begin{tikzpicture}
      \draw[vectors] (1,1.3)--(1.5,1.3) node[right]{$v$};
      \draw[thick,blue!20,smooth,samples=20,domain=0:2.8]
      plot(\x,{sin(200*\x)});
    \end{tikzpicture}
    \hspace{.15in}
    \begin{tikzpicture}
      \draw[vectors] (1,1.3)--(1.5,1.3) node[right]{$v$};
      \draw[thick,red,smooth,samples=50,domain=0:2.8] plot(\x,{sin(400*\x)});
    \end{tikzpicture}
    \hspace{.15in}
    \begin{tikzpicture}
      \draw[vectors] (1,1.3)--(1.5,1.3) node[right]{$v$};
      \draw[thick,green,smooth,samples=70,domain=0:2.8]
      plot(\x,{sin(600*\x)});
    \end{tikzpicture}
  \end{center}
  Beyond the 2nd harmonic, there are also the 3rd harmonic (2nd overtone),
  4th harmonic (3rd overtone), 5th harmonic (4th overtone) etc. Whole-number
  multiples of the fundamental frequency $f_1$ are its \textbf{harmonic
    frequencies}, and the $n$-th harmonic is:

  \eq{-.15in}{
    \boxed{f_n=nf_1}\quad\quad n=1,2,3,\ldots
  }
\end{frame}



\begin{frame}{Different Musical Instruments}
  Wave forms from different musical instruments at \SI{440}\hertz:
  \begin{center}
    \pic{.6}{images/F_InstrumentWaves}
  \end{center}
  The graph on the right correspond to the amplitudes at different frequencies.
  Note the peaks at regular intervals. Those are the harmonic frequencies.
\end{frame}



\section{Resonance Modes}

\subsection{Strings}

\begin{frame}{Standing Waves on Strings}
  \textbf{Resonance modes} are frequencies where a standing wave can be
  established. For a string of length $L$, this requires both ends of the
  string to be nodes. The \textbf{fundamental mode} (lowest frequency) occurs
  at $\lambda=2L$:
  \begin{center}
    \begin{tikzpicture}[scale=1.5]
      \draw[thick] (0,-.1)--(0,.1);
      \draw[thick] (pi,-.1)--(pi,.1);
      \draw[thick] (0,0)--(pi,0);
      \begin{scope}[smooth,samples=20]
        \draw[domain=0:pi,functions] plot(\x,{0.45*sin(180/pi*\x)});
        \draw[domain=pi:2.2*pi,dashed] plot(\x,{0.45*sin(180/pi*\x)});
        \draw[domain=0 :2.2*pi,dashed] plot(\x,{-.45*sin(180/pi*\x)});
      \end{scope}
      \draw[|<->|] (0,-.6)--(2*pi,-.6) node[midway,fill=white]{$\lambda=2L$};
      \fill circle (.04) node[below=3,fill=yellow!30]{N};
      \fill (pi/2,0) circle (.04) node[below=3,fill=pink!30]{A};
      \fill (pi,0) circle (.04) node[below=3,fill=yellow!30]{N};
    \end{tikzpicture}
  \end{center}
  
  \vspace{-.1in}The fundamental frequency can be calculated based on the
  relationship $v_\text{str}=f\lambda$:

  \eq{-.1in}{
    \boxed{
      f_1=\frac{v_\text{str}}\lambda=\frac{v_\text{str}}{2L}
      \quad\text{\normalsize where}\quad v_\text{str}=\sqrt{\frac{F_TL}m}
    }
  }
\end{frame}


\begin{frame}{Standing Waves On a String of Length $L$}
  The second resonance mode occurs when $\lambda=L$:

  \vspace{.2in}\begin{columns}
    \column{.4\textwidth}
    \centering
    \begin{tikzpicture}[scale=1.5]
      \draw[thick] (0,0)--(pi,0);
      \draw[thick] (0,-.1)--(0,.1);
      \draw[thick] (pi,-.1)--(pi,.1);
      \begin{scope}[smooth,samples=20,domain=0:pi]
        \draw[functions] plot(\x,{.4*sin(360/pi*\x)});
        \draw[dashed] plot(\x,{-.4*sin(360/pi*\x)});
      \end{scope}
      \fill (0,0) circle (.04) node[below]{N};
      \fill (pi,0) circle (.04) node[below]{N};
      \fill (pi/2,0) circle (.04) node[below]{N};
      \fill (pi/4,0) circle (.04) node[below]{A};
      \fill (3*pi/4,0) circle (.04) node[below]{A};
    \end{tikzpicture}
    
    \column{.6\textwidth}    
    \eq{-.01in}{
      f_2 = \frac v\lambda=\frac vL=2f_1
    }
  \end{columns}

  The third resonance mode occurs at $\lambda=\dfrac23L$:

  \begin{columns}
    \column{.4\textwidth}
    \centering
    \begin{tikzpicture}[scale=1.5]
      \draw[thick] (0,0)--(pi,0);
      \draw[thick] (0,-.1)--(0,.1);
      \draw[thick] (pi,-.1)--(pi,.1);
      \begin{scope}[smooth,domain=0:pi]
        \draw[samples=20,functions] plot(\x,{.4*sin(540/pi*\x)});
        \draw[samples=30,dashed] plot(\x,{-.4*sin(540/pi*\x)});
      \end{scope}
      \foreach \x in {0,pi/3,pi,2*pi/3}
      \fill (\x,0) circle (.04) node[below]{N};
      %\fill (0,0) circle (.04) node[below]{N};
      %\fill (pi,0) circle (.04) node[below]{N};
      %\fill (pi/3,0) circle (.04) node[below]{N};
      %\fill (2*pi/3,0) circle (.04) node[below]{N};
      \fill (pi/6,0) circle (.04) node[below]{A};
      \fill (pi/2,0) circle (.04) node[below]{A};
      \fill (5*pi/6,0) circle (.04) node[below]{A};
    \end{tikzpicture}
    
    \column{.6\textwidth}
      
    \eq{-.01in}{
      f_3=\frac v\lambda=\frac{3v}{2L}=3f_1
    }
  \end{columns}
\end{frame}



\begin{frame}{Standing Waves On a String of Length $L$}
  The $n$-th resonance mode of a wave on string is given by:

  \eq{-.1in}{
    \boxed{f_n=nf_1}\quad
    \text{\normalsize where}\quad
    \boxed{f_1=\frac v{2L}}
  }
  \begin{itemize}
  \item $n=1,2,3,\ldots$ is a whole-number multiple
  \item This equation is \emph{identical} to the equation for harmonic
    frequencies, meaning that on a string, every harmonic is a resonance
    frequency
  \item A vibrating string is said to have a ``full set of harmonics''
  \end{itemize}
\end{frame}


\subsection{Closed Pipes}

\begin{frame}{Standing Waves in a Closed Pipe of Length $L$}
  Similar standing-wave patterns can be found on pipes with both ends closed.
  Like strings, the boundary condition is that the closed ends of the pipes
  must be nodes.

  \vspace{.2in}
  \begin{columns}
    \column{.33\textwidth}
    \centering
    \begin{tikzpicture}[scale=1.3,yscale=.45]
      \foreach \x in {0,pi} \node[below,fill=yellow!40] at (\x,-1.1){N};
      \node[below,fill=pink!40] at (pi/2,-1.1){A};
      \begin{scope}[smooth,samples=20,domain=0:pi]
        \draw[functions] plot(\x,{sin(180/pi*\x)});
        \draw[dashed] plot(\x,{-1*sin(180/pi*\x)});
      \end{scope}
      \draw[thick] (0,-1) rectangle (pi,1);
      \node[above] at (pi/2,1){Fundamental Mode};
    \end{tikzpicture}

    \eq{-.25in}{
        f_1=\frac{v_s}\lambda=\frac{v_s}{2L}
    }
  
    \column{.33\textwidth}
    \centering
    \begin{tikzpicture}[scale=1.3,yscale=.45]
      \foreach \x in {0,pi/2,pi} \node[below,fill=yellow!40] at(\x,-1.1){N};
      \foreach \x in {pi/4,3*pi/4} \node[below,fill=pink!40] at(\x,-1.1){A};
      \begin{scope}[smooth,samples=20,domain=0:pi]
        \draw[functions] plot(\x,{sin(360/pi*\x)});
        \draw[dashed] plot(\x,{-1*sin(360/pi*\x)});
      \end{scope}
      \draw[thick] (0,-1) rectangle (pi,1);
      \node[above] at (pi/2,1){2nd Resonance Mode};
    \end{tikzpicture}

    \eq{-.25in}{
      f_2=
      \frac{v_s}\lambda=\frac{v_s}L=2f_1
    }
    
    \column{.33\textwidth}
    \centering
    \begin{tikzpicture}[scale=1.3,yscale=.45]
      \foreach \x in {0,pi/3,2*pi/3,pi}
      \node[below,fill=yellow!40] at (\x,-1.1){N};
      \foreach \x in {pi/6,pi/2,5*pi/6}
      \node[below,fill=pink!40] at (\x,-1.1){A};
      \begin{scope}[smooth,samples=20,domain=0:pi]
        \draw[functions] plot(\x,{sin(540/pi*\x)});
        \draw[dashed] plot(\x,{-1*sin(540/pi*\x)});
      \end{scope}
      \draw[thick] (0,-1) rectangle (pi,1);
      \node[above] at (pi/2,1){3rd Resonance Mode};
    \end{tikzpicture}

    \eq{-.25in}{
      f_3=\frac{3v_s}{2L}=3f_1
    }
  \end{columns}  
\end{frame}



\begin{frame}{Standing Waves in a Closed Pipe of Length $L$}
  For strings and pipes that are \emph{closed at both ends}, the $n$-th
  resonance mode is given by:
  
  \eq{-.1in}{
    \boxed{f_n=nf_1}\quad\text{\normalsize where}\quad
    \boxed{f_1=\frac{v_s}{2L}}
    \quad n=1,2,3,\ldots
  }
  \begin{itemize}
  \item The above equation is \emph{identical} to the equation for harmonic
    frequencies, i.e.\ every harmonic is a resonance mode
  \item The difference between a closed pipe and a string is that the wave
    speed is the speed of sound $v_s$ inside the pipe
  \end{itemize}
  No musical ``instruments'' are built this way, but you can use this to model
  standing wave patterns inside a concert hall.
\end{frame}



\subsection{Open Pipes}

\begin{frame}{Standing Waves in an Open Pipe of Length $L$}
  Some organ pipes and flute have open ends on both sides. In this case, the
  open ends are anti-nodes. The standing-wave pattern is similar to the closed
  pipes, but the locations of the nodes and anti-nodes are reversed.
  \begin{center}
    \begin{tikzpicture}[scale=1.4,yscale=.8]
      \draw[thick] (0,-.5)--(pi,-.5);
      \draw[thick] (0,.5)--(pi,.5);
      \foreach \x in {0,pi} \node[below,fill=pink!50] at (\x,-.55){A};
      \node[below,fill=yellow!50] at (pi/2,-.55){N};
      \begin{scope}[smooth,domain=0:pi,samples=15]
        \draw[functions] plot(\x,{.5*sin(180/pi*\x+90)});
        \draw[dashed] plot(\x,{-.5*sin(180/pi*\x+90)});
      \end{scope}
      \node[above] at (pi/2,.5){Fundamental Mode};
    \end{tikzpicture}
    \begin{tikzpicture}[scale=1.4,yscale=.8]
      \draw[thick] (0,-.5)--(pi,-.5);
      \draw[thick] (0,.5)--(pi,.5);
      \foreach \x in {0,pi/2,pi} \node[below,fill=pink!50] at (\x,-.55){A};
      \foreach \x in {pi/4,3*pi/4} \node[below,fill=yellow!50] at (\x,-.55){N};
      \begin{scope}[smooth,samples=20,domain=0:pi]
        \draw[functions] plot(\x,{.5*sin(360/pi*\x+90)});
        \draw[dashed] plot(\x,{-.5*sin(360/pi*\x+90)});
      \end{scope}
      \node[above] at (pi/2,.5){2nd Resonance Mode};
    \end{tikzpicture}
    \begin{tikzpicture}[scale=1.4,yscale=.8]
      \draw[thick] (0,-.5)--(pi,-.5);
      \draw[thick] (0,0.5)--(pi,0.5);
      \foreach \x in {0,pi/3,2*pi/3,pi}
      \node[below,fill=pink!50] at (\x,-.55){A};
      \foreach \x in {pi/6,pi/2,5*pi/6}
      \node[below,fill=yellow!50] at (\x,-.55){N};
      \begin{scope}[smooth,samples=30,domain=0:pi]
        \draw[functions] plot(\x,{.5*sin(540/pi*\x+90)});
        \draw[dashed] plot(\x,{-.5*sin(540/pi*\x+90)});
      \end{scope}
      \node[above] at (pi/2,.5){3rd Resonance Mode};
    \end{tikzpicture}
  \end{center}
  Like strings and closed pipes, open pipes also have a ``full set of
  harmonics''. The $n$-th resonance mode is given by:
  
  \eq{-.15in}{
    \boxed{f_n=nf_1}
    \quad n=1,2,3,\ldots
    \quad\text{\normalsize where}
    \quad\boxed{f_1=\frac{v_s}{2L}}
  }
\end{frame}



\begin{frame}{Example Problem}
  \textbf{Example}: An air column, open at both ends, has a fundamental
  resonance mode at \SI{330}\hertz.
  \begin{enumerate}
  \item What are the frequencies of the second and third resonance modes?
  \item If the speed of sound in air is \SI{344}{\metre\per\second}, what is the
    length of the air column?
  \end{enumerate}
\end{frame}



\subsection{Semi-Open Pipes}

\begin{frame}{Standing Waves in a Semi-Open Pipe of Length $L$}
  Not all configurations have a full set of harmonics. Most organ pipes, as
  well as most woodwind (e.g.\ clarinet, oboe, bassoon) and brass instruments
  (e.g.\ French horn, trumpet, trombone) have one closed end and one open end.
  \begin{itemize}
  \item The closed end is a node, while
  \item The open end is an antinode
  \end{itemize}
  \begin{center}
    \begin{tikzpicture}[xscale=1.4]
      \draw[thick] (pi,-.5)--(0,-.5) node[pos=0,below=1.5,fill=pink!50]{A}
      node[below=1.5,fill=yellow!50]{N}--(0,.5)--(pi,.5);
      \begin{scope}[smooth,samples=10,domain=0:pi]
        \draw[functions] plot(\x,{.5*sin(90/pi*\x)});
        \draw[dashed] plot(\x,{-.5*sin(90/pi*\x)});
      \end{scope}
      \node[above] at (pi/2,.5){Fundamental Mode};
    \end{tikzpicture}
    \begin{tikzpicture}[xscale=1.4]
      \draw[thick] (pi,-.5)--(0,-.5)--(0,.5)--(pi,.5);
      \foreach \x in {0,2*pi/3}\node[below,fill=yellow!50] at (\x,-.55){N};
      \foreach \x in {pi/3,pi} \node[below,fill=pink!50] at (\x,-.55){A};
      \begin{scope}[smooth,samples=20,domain=0:pi]
        \draw[functions] plot(\x,{.5*sin(270/pi*\x)});
        \draw[dashed] plot(\x,{-.5*sin(270/pi*\x)});
      \end{scope}
      \node[above] at (pi/2,.5){2nd Resonance Mode};
    \end{tikzpicture}
    \begin{tikzpicture}[xscale=1.4]
      \draw[thick] (pi,-.5)--(0,-.5)--(0,.5)--(pi,.5);
      \foreach\x in {0,2*pi/5,4*pi/5}\node[below,fill=yellow!50] at(\x,-.55){N};
      \foreach\x in {pi/5,3*pi/5,pi} \node[below,fill=pink!50] at (\x,-.55){A};
      \begin{scope}[smooth,samples=30,domain=0:pi]
        \draw[functions] plot(\x,{.5*sin(450/pi*\x)});
        \draw[dashed] plot(\x,{-.5*sin(450/pi*\x)});
      \end{scope}
      \node[above] at (pi/2,.5){3rd Resonance Mode};
    \end{tikzpicture}
  \end{center}
\end{frame}



\begin{frame}{Standing Waves in a Semi-Open Pipe of Length $L$}
  In this case, the fundamental mode occurs at $\lambda=4L$:
  \begin{center}
    \begin{tikzpicture}[xscale=1.1,yscale=.5]
      \draw[thick] (pi,-1)--(0,-1)--(0,1)--(pi,1);
      \begin{scope}[smooth,samples=25]
        \draw[domain=0:pi,functions] plot(\x,{sin(90/pi*\x)});
        \draw[domain=pi:4*pi,dashed] plot(\x,{sin(90/pi*\x)});
        \draw[domain=0:4*pi,dashed] plot(\x,{-1*sin(90/pi*\x)});
      \end{scope}
      \draw[|<->|] (0,1.55)--(pi,1.55) node[midway,fill=white]{$L$};
      \draw[|<->|] (0,-1.3)--(4*pi,-1.3) node[midway,fill=white]{$\lambda$};
    \end{tikzpicture}
  \end{center}
  
  Fundamental frequency $f_1$ differs from open-pipe and closed-pipe by a
  factor of 2:

  \eq{-.1in}{
    \boxed{f_1=\frac{v_s}\lambda=\frac{v_s}{4L}}
  }
    
  Fundamental mode $f_1$ is lower than open-ended and closed-ended pipes for
  the same length ; it has advantages when designing organ pipes that produces
  low frequencies.
\end{frame}



\begin{frame}{Standing Waves in a Semi-Open Pipe of Length $L$}
  A second resonance mode occurs at $\lambda=\dfrac43L$:
  \begin{columns}
    \column{.45\textwidth}
    \begin{tikzpicture}[xscale=1.5,yscale=.5]
      \begin{scope}[smooth,samples=15,domain=0:pi]
        \draw[functions] plot(\x,{sin(270/pi*\x)});
        \draw[dashed] plot(\x,{-1*sin(270/pi*\x)});
      \end{scope}
      \begin{scope}[smooth,dashed,samples=10,domain=pi:1.5*pi]
        \draw plot(\x,{-1*sin(270/pi*\x)});
        \draw plot(\x,{1*sin(270/pi*\x)});
      \end{scope}
      \draw[|<->|] (0,-1.6)--(4/3*pi,-1.6) node[midway,fill=white]{$\lambda$};
      \draw[thick] (pi,-1)--(0,-1)--(0,1)--(pi,1);
    \end{tikzpicture}
    
    \column{.55\textwidth}
    \eq{-.2in}{
      f_2=\frac{v_s}\lambda=\frac{3v_s}{4L}=3f_1
    }
  \end{columns}

  \vspace{.1in}And a third resonance mode occurs at $\lambda=\dfrac45L$:
  \begin{columns}
    \column{.45\textwidth}
    \begin{tikzpicture}[xscale=1.5,yscale=.5]
      \draw[thick] (pi,-1)--(0,-1)--(0,1)--(pi,1);
      \begin{scope}[smooth,samples=20,domain=0:pi]
        \draw[functions] plot(\x,{sin(450/pi*\x)});
        \draw[dashed] plot(\x,{-1*sin(450/pi*\x)});
      \end{scope}
      \draw[<->] (0,-1.6)--(.8*pi,-1.6) node[midway,fill=white]{$\lambda$};
    \end{tikzpicture}

    \column{.55\textwidth}
    \eq{0in}{
      f_3=\frac{v_s}\lambda=\frac{5v_s}{4L}=5f_1
    }
  \end{columns}
\end{frame}



\begin{frame}{Standing Waves in a Semi-Open Pipe of Length $L$}
  Semi-open pipes have an \emph{odd set of harmonics} because only
  \emph{odd-number} multiples of the fundamental can be resonance modes.

  \eq{-.1in}{
    \boxed{f_n=(2n-1)f_1}
    \quad n=1,2,3,\ldots
    \quad\text{\normalsize where}\quad
    \boxed{f_1=\frac{v_s}{4L}}
  }

  Remember: harmonic frequencies are multiples of the fundamental frequency.
  This means that:
  \begin{itemize}
  \item 2nd resonance mode = 3rd harmonic frequency
  \item 3rd resonance mode = 5th harmonic frequency
  \item 4th resonance mode = 7th harmonic frequency
  \end{itemize}
\end{frame}
\end{document}
