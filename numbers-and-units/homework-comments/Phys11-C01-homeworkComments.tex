\documentclass[12pt,compress,aspectratio=169]{beamer}
\input{../../mybeamer}
%
%\usepackage{cancel}
%
%\title{Class 1: Vectors and Motion Quantities}
%\subtitle{Unit 1: Motion}
%\input{../term}
\input{../../mycommands}


\begin{document}

%\begin{frame}
%  \titlepage
%\end{frame}
%
%\section[Intro]{Introduction}


\begin{frame}{When You Come In}
  When you come into the classroom, please
  \begin{enumerate}
  \item Pick up your marked copy of Homework \# 1 (\textbf{Don't get nervous!})
  \item Pick up a copy of the Homework \#3
  \item Submit Homework \#2
  \end{enumerate}
  The piles are labelled near the blackboard.
\end{frame}



\begin{frame}{Comments from Homework \#1}
  Homework \#1 has been marked during the week. Despite all the red ink in
  \emph{everyone's work}, the classes actually did quite well.

  \vspace{.2in}
  \begin{itemize}
  \item Homework is marked ``P'' (passed) or ``I'' (incomplete)
  \item For Homework \#1, your ``mark'' is based on whether you have tried most
    of the questions, and not necessarily whether you have the correct answers,
    i.e.\ the only way to have an ``I'' is to not do the question
  \item We will care more about the correct answers in later homework sets
  \end{itemize}
\end{frame}



\begin{frame}{Common Comments}
  The \emph{most} common comments all the teachers made for Homework \# 1
  is \textbf{the number of significant figures}.
  \begin{itemize}
  \item Most of your had too many sig.\ figs.\ in your answers.
  \item Remember: most of your calculations are multiplications and division.
  %  \begin{columns}
%    \column{.33\textwidth}
%    \centering
%    Counted:
%    \pic1{images/empty-classroom}
%
%    \column{.33\textwidth}
%    \centering
%    Measured:
%    \pic1{images/graduated-cylinder}
%
%    \column{.33\textwidth}
%    \centering
%    Estimated:
%    \pic1{images/crowded-beach}
%  \end{columns}
%
%  In physics, a quantity is either \emph{counted}, \emph{measured},
    %  or \emph{estimated}.
  \end{itemize}
\end{frame}



%\begin{frame}{Counted Quantities}
%  \begin{columns}
%    \column{.33\textwidth}
%    \pic1{images/empty-classroom}
%    
%    \column{.66\textwidth}
%    Examples of quantity that is \emph{counted}:
%    \begin{itemize}
%    \item There are 20 students and 1 teacher in an online Zoom class
%    \item There are 2 engines on a Boeing 777
%    \item 2 of the 15 cars in the parking lot are red
%    \end{itemize}
%    Quantity that are counted are usually expressed \emph{exactly} with
%    \emph{no uncertainties}. They are usually expressed as integers or
%    fractions (ratios of integers).
%  \end{columns}
%\end{frame}



\begin{frame}{Surface Area and Volume of Spheres}
  I am surprised with how many of you did not know the surface area and
  volume of spheres. In case this is something that you never learned (and
  didn't try to Google):
  \begin{itemize}
  \item\textbf{Surface area of a sphere:}

    \eq{-.15in}{
      \boxed{
        A=4\pi r^2
        }
    }

    \vspace{-.2in}It is \emph{not} the area of a circle! In general,
    we would avoid using 2 or 3 letters to indicate the variable name (e.g.\
    some students used ``\emph{SA}'') because it looks like many variables
    multiplied together.

  \item\textbf{Volume of a sphere:}

    \eq{-.1in}{
      \boxed{
        V=\frac43\pi r^3
      }
    }
  \end{itemize}
  If you didn't know, now is the time to 
\end{frame}



\begin{frame}{Best Practice for Presenting Your Work}
  The best practice that you should have for presenting your problem-solving
  work is to:
  \begin{itemize}
  \item Write out the equation that you are using
  \item Algebraically solve for the quantity that the question is asking
  \item Substitute numerical vales
  \item  Calculate final answer
  \item Round off to the appropriate number of significant figures
  \end{itemize}
\end{frame}



\begin{frame}{Presenting Your Work: An Example}
  7. How many grams of copper are required to make a \emph{hollow}
  spherical shell with an inner radius of $r_i=\SI{5.70}{\centi\metre}$ and
  outer radius of $r_o=\SI{5.75}{\centi\metre}$? The density of copper is
  $\rho=\SI{8.93}{\gram\per\centi\metre\cubed}$. (Mass and volume are related
  by $m=\rho V$.)
%  Quantities that are measured or estimated always have \emph{uncertainties};
%  the \emph{exact} answers are impossible to obtain. \emph{How} we write the
%  numbers conveys knowledge about those uncertainties.
%
%  \vspace{.15in}When a number is obtained through measurement or estimate,
%  usually the \emph{last} digit is the one that has uncertainties. For example,
%  the following two measurements represent different levels of uncertainty:
%
%  \eq{-.1in}{
%    \SI{103}\metre\quad\quad\text{and}
%    \quad\quad\SI{103.456}\metre
%  }
\end{frame}



%\begin{frame}{Uncertainties in Numbers}
%  \eq{-.3in}{
%    \SI{103}\metre
%  }
%  \begin{itemize}
%  \item\vspace{-.25in}Uncertainty is in the unit digit (accurate to within 1
%    metre)
%  \item Depending on how (with what tools, by whom, etc) the length was
%    measured, the actual value \emph{could} be \SI{102}{\metre} or
%    \SI{104}{\metre}
%  \end{itemize}
%
%  \eq{0in}{
%    \SI{103.456}\metre
%  }
%  \begin{itemize}
%  \item\vspace{-.25in}Uncertainty is in the 3rd decimal place (accurate to
%    about one millimetre)
%  \item The actual length could be \SI{103.457}{\metre} or \SI{103.455}{\metre}
%  \end{itemize}
%\end{frame}
%
%
%
%\begin{frame}{Managing Uncertainties}
%  When performing mathematical operations on numbers with uncertainties
%  (through measurements and/or estimations), we must manage the uncertainties
%  in the claculations.
%\end{frame}
%
%
%
%\begin{frame}{Uncertainties in Numbers: An Example}
%  \textbf{Example:} Given that $AB=\SI{103}\metre$ and
%  $BC=\SI{103}\metre$. What is $AC$?
%  \begin{center}
%    \begin{tikzpicture}
%      \draw[thick](0,0)--(2,0) node[pos=0,left]{$A$}
%      node[right]{$B$} node[midway,below]{\SI{103}\metre}
%      --(2,2) node[midway,right]{\SI{103}\metre} node[right]{$C$};
%      \draw[thick](1.85,0) rectangle(2,0.15);
%      \draw[very thick,red](0,0)--(2,2);
%    \end{tikzpicture}
%  \end{center}
%  In your math class, you will certainly be required to give the \emph{exact}
%  answer:
%
%  \eq{-.1in}{
%    \num{103}\sqrt 2\;\si\metre
%  }
%    
%  \vspace{-.2in}But $AB$, $BC$, as well as the right angle in the diagram are
%  \emph{measurements} with uncertainties. Then what is the best way to
%  express $AC$?
%\end{frame}
%
%
%
%\begin{frame}{Uncertainties in Numbers: An Example}
%  \begin{columns}
%    \column{.3\textwidth}
%    \vspace{.2in}
%    \begin{tikzpicture}[scale=1.35]
%      \draw[thick](0,0)--(2,0) node[pos=0,left]{$A$}
%      node[right]{$B$} node[midway,below]{\SI{103}\metre};
%      \draw[thick](2,0)--(2,2) node[midway,right]{\SI{103}\metre}
%      node[right]{$C$};
%      \draw[thick](1.85,0) rectangle(2,0.15);
%      \draw[very thick,red](0,0)--(2,2);
%    \end{tikzpicture}
%    
%    \column{.7\textwidth}
%    Is the answer still the same as in your math class?
%
%    \eq{-.1in}{
%      103\sqrt2\;\si\metre
%    }
%
%    \vspace{-.1in}Or should we round off to some decimal places, and our
%    answer is no longer ``exact''?
%
%    \eq{-.1in}{
%      \SI{145.6639969}\metre
%    }
%    
%    \vspace{-.1in}And if we round off decimal places, how many should we
%    keep? (In this particular example, the correct way to express the answer is
%    \SI{146}{\metre}, with \emph{no} decimal place. But how to we find out?
%  \end{columns}
%\end{frame}
%
%
%\subsection{Significant Figures}
%
%\begin{frame}{Significant Figures}
%  \vspace{.15in}We use a technique called \textbf{significant figures}
%  (or \textbf{sig.\ figs.\ }or \textbf{s.f.}).
%  \begin{itemize}
%  \item %The significant figures of a number are
%    Digits that carry meaning contributing to its measurement resolution
%  \item Also known as \textbf{significant digits}, \textbf{sig.\ digs.\ }or
%    \textbf{s.d.}
%  \item Used when a full error analysis cannot be performed
%  \end{itemize}
%
%  \vspace{.15in}First, we need to find out \emph{how many} significant figures
%  a measurement has. There are a few rules.
%\end{frame}
%
%
%
%\begin{frame}{When are Digits Significant?}
%  
%  \begin{block}{}
%    Non-zeros digits are always significant
%  \end{block}
%
%  \textbf{Example:}
%  \begin{itemize}
%  \item\SI{22}{\metre} has \emph{two} s.f., while
%  \item\SI{22.3}{\metre} has \emph{three} s.f.
%  \end{itemize}
%
%  \begin{block}{}
%    Zeroes placed before other digits (``leading zeros'') are not significant
%  \end{block}
%  \textbf{Example:}
%  \begin{itemize}
%  \item\SI{.046}{\metre\per\second} has \emph{two} s.f.
%  \item\SI{.00453}{\metre\per\second} has \emph{three} s.f.
%  \end{itemize}
%  The leading zeros in both cases do not contribute to the answer.
%\end{frame}
%
%
%\begin{frame}{When are Digits Significant?}
%  \begin{block}{}
%    Zeroes placed \emph{between} non-zero digits are significant
%  \end{block}
%
%  \textbf{Example:} \SI{4009}{\kilo\gram} has \emph{four} s.f.
%
%  \vspace{.25in}
%  \begin{block}{}
%    Zeroes placed behind a decimal point are significant except leading zeros
%  \end{block}
%
%  \textbf{Example:} \SI{7.90}{\watt} has \emph{three} s.f.
%\end{frame}
%
%
%
%\begin{frame}{When are Digits Significant?}
%  This last case is a bit contentious, and different teachers with different
%  backgrounds will interpret it differently:
%
%  \vspace{.1in}
%  \begin{block}{}
%    Zeroes at the end of a number are significant only if they are behind a
%    decimal point. Otherwise, it is impossible to tell.
%  \end{block}
%  
%  \vspace{.2in}\textbf{Example:} In the length \SI{8200}\metre, it is unclear
%  whether the zeroes are significant. The number of significant figures is at
%  least \emph{two} but could also be \emph{three} or \emph{four}.
%\end{frame}
%
%
%
%\begin{frame}{Significant Figures in Mathematical Operations}
%  \begin{block}{}
%    In calculations involving multiplication, division, trigonometric functions,
%    and square roots, the number of significant figures in the answer should
%    equal the least number of significant figures in any of the numbers being
%    multiplied or divided.
%  \end{block}
%
%  \textbf{Example:} When evaluating
%
%  \eq{-.15in}{
%    y=\sin(kx)
%  }
%    
%  where $k=\SI{.097}{\per\metre}$ (2 s.f.) and $x=\SI{4.73}\metre$ (3 s.f.),
%  the answer should have 2 s.f., i.e.:
%
%  \eq{-.1in}{
%    y=\sin(kx)=\num{.44}
%  }
%\end{frame}
%
%
%
%\begin{frame}{Significant Figures in Mathematical Operations}
%  \textbf{Example:} When multiplying 6.4 and 3.217 together, the answer
%  should have 2 significant figures:
%
%  \eq{-.3in}{
%    6.{\color{red}4}\times 3.21{\color{red}7}=2{\color{red}1}
%    }
%
%  \vspace{-.15in}The digits highlighted in red contain uncertainties. We can
%  show this by separating part of the numbers with certainties from the part
%  without:
%
%  \eq{-.2in}{
%    6.{\color{red}4}\times 3.21{\color{red}7}
%    = (6+{\color{red}0.4}) \times (3.21 + {\color{red}0.007})
%  }
%
%  \vspace{-.2in}Any number that is multiplied by 0.4 or 0.007 will
%  automatically be uncertain. When we put together all the numbers, we end up
%  with
%
%  \eq{-.2in}{
%    =2{\color{red}0.5888}
%  }
%
%  \vspace{-.2in}Since we expect only the \emph{last} digit to have
%  uncertainties, we round it off to the final answer of 21.
%\end{frame}
%
%
%
%\begin{frame}{Significant Figures in Mathematical Operations}
%  \begin{block}{}
%    For \textbf{additions and subtractions}: the number of \emph{decimal places}
%    in the answer equals the least number of \emph{decimal places} in any of the
%    numbers being added or subtracted.
%  \end{block}
%  
%  Using the numbers from the previous example, what if we add them together?
%
%  \eq{-.1in}{
%    6.{\color{red}4} + 3.21{\color{red}7} = 9.{\color{red}6}
%  }
%
%  Like multiplication/division, any number that is added to, or subtracted
%  from, another number that has uncertainties, the sum or difference will also
%  be uncertain.
%\end{frame}
%
%
%\begin{frame}{Integers and Irrational Numbers}
%  Integers and irrational numbers often appear in algebraic expressions of the
%  equations that we use in physics problems.
%  \begin{itemize}
%  \item Integers, the imaginary number, and ratios of integers have an infinite
%    number of significant figures because they do not have uncertainties:
%
%    \eq{-.1in}{
%      -1\quad5000\quad\frac32\quad i
%    }
%
%  \item Irrational numbers also have an infinite number of significant figures
%
%    \eq{-.1in}{
%      e\quad\pi\quad\sqrt5
%    }
%  \end{itemize}
%\end{frame}
%
%
%
%\subsection{Rounding}
%
%\begin{frame}{Rounding}
%  In the previous examples, we are left with more decimal places than needed.
%  Since we can only keep \emph{one} number with uncertainty, that means that
%  the answers have to be
%  \emph{rounded to the appropriate number of significant figures}.
%  
%  \vspace{.15in}Most problems in high-school physics have 2 or 3 significant
%  figures\footnote{of course there are exceptions}.
%\end{frame}
%
%
%
%\begin{frame}{Rules for Rounding}
%  \begin{columns}
%    \column{.6\textwidth}
%
%    To round to $n$ significant figures, check the $(n+1)$\textsuperscript{th}
%    digit onward. If they are
%    \begin{itemize}
%    \item \num{000000} to \num{499999}\ldots: round \textbf{down}
%    \item \num{500001} to \num{999999}\ldots: round \textbf{up}
%    \item \num{500}\ldots, then check $n$\textsuperscript{th} digit:
%      \begin{itemize}
%      \item if that number is odd, round up
%      \item if that number is even, round down
%      \end{itemize}
%    \end{itemize}
%    
%    \column{.4\textwidth}
%    \textbf{Example:} Rounding to three significant figures
%
%    \vspace{.15in}
%    \begin{tabular}{c|c}
%      \rowcolor{pink}
%      \textbf{Measurement} & \textbf{Round To} \\\hline
%      \num{1.2346} & 1.23 \\
%      \num{1.3478} & 1.35 \\
%      \num{2.4450} & 2.44 \\
%      \num{2.5752} & 2.58
%    \end{tabular}
%  \end{columns}
%  \vspace{.15in}This is the most common rounding method taught in \emph{most}
%  science and engineering textbooks, but there are variations between
%  disciplines.
%\end{frame}
%
%
%
%\begin{frame}{Keep One More Digit in Intermediate Answers}
%  When doing multi-step calculations, keep at least one more significant
%  figure\footnote{In fact, the best way is to keep \emph{all} the decimal
%    places in your calculator} in your intermediate results than needed in your
%  final answer.
%  \begin{itemize}
%  \item If the final answer requires two s.f., carry at least \emph{three}
%    s.f.\ during calculations
%  \item If you round off too soon, you discard information contained in the 3rd
%    digit; as a result, the 2nd digit in your final answer might be incorrect
%  \item This phenomenon is known as \textbf{round-off error}
%  \end{itemize}
%  \begin{block}{}
%    Different teachers have different approaches, so pay attention to
%    what they want!
%  \end{block}
%\end{frame}
%
%
%
%
%\begin{frame}{Scientific Notation}
%  In \textbf{scientific notation}, a number is written as a number between
%  1 and 10 that is multiplied by a power of 10. For example:
%
%  \eq{-.1in}{
%    \num{5326.6}\rightarrow\num{5.3266e3}
%  }
%
%  Because $\num{5326.6}=\num{5.3266}\times\num{1000}=\num{5.3266e3}$.
%  Another example:
%
%  \eq{-.1in}{
%    \num{.00654}\rightarrow\num{6.54e-3}
%  }
%\end{frame}
%
%
%
%\begin{frame}{Scientific Notation}
%  Scientific notation avoids confusions about significant figures, because all
%  digits, except for the first, are now after the decimal place, where zeros are
%  considered significant. For example, it is unclear how many significant
%  figures are in this time interval:
%
%  \eq{-.1in}{
%    \SI{84600}\second
%  }
%  
%  \vspace{-.15in}There could be 3, 4 or 5. However, if it is expressed using
%  scientific notation, the number of significant figures is clear:
%  
%  \eq{-.1in}{
%    \SI{8.460e4}\second
%  }
%
%  In this case, it is clear that the measurement is to 4 significant figures  
%\end{frame}
%
%
%
%\begin{frame}{Order of Magnitude}
%  When two numbers differ by a factor of 10, we say that they differ by one
%  \textbf{order of magnitude}, because the ratio of the number is $10^1$.
%  Similarly, \emph{two} orders of magnitude means a ratio of $10^2$.
%
%  \vspace{.15in}\textbf{Example:} In air, the speed of light $c$ is
%  approximately \emph{six} orders of magnitude faster than the speed of sound
%  $v_s$. We can also say that $v_s$ is approximately six orders of magnitude
%  slower than $c$.
%
%  \eq{-.1in}{
%    \frac c{v_s}=
%    \frac{\SI{3.00e8}{\metre\per\second}}
%         {\SI{3.43e2}{\metre\per\second}}
%    \approx 10^6
%  }
%  
%  Order-of-magnitude analysis is often used by scientists and engineers to
%  check whether the solution to a problem makes sense.
%\end{frame}
%
%
%
%\subsection{SI Units}
%
%\begin{frame}{SI Units}
%  The \textbf{International System of Units} (\textbf{SI}, from the French
%  \emph{Syst\`{e}me international d'unit\'{e}s}) is the modern form of the
%  metric system. SI units are based on seven base units:
%  \begin{columns}
%
%    \column{.4\textwidth}
%    \pic1{images/si-base}
%
%    \column{.6\textwidth}
%    \begin{tabular}{c|c|l}
%      \textbf{Symbol} & \textbf{Name} & \textbf{Quantity} \\ \hline
%      \si\second    & second  & time\\
%      \si\metre     & metre   & length\\
%      \si{\kilo\gram} & kilogram & mass\\
%      \si\ampere    & amp\`{e}re  & electric current\\
%      \si\kelvin    & kelvin  & temperature\\
%      \si\mol       & mole    & amount of substance\\
%      \si\candela   & candela & luminous intensity
%    \end{tabular}
%  \end{columns}
%\end{frame}
%
%
%
%\begin{frame}{SI Units}
%  All other units are derived from these seven base units. For example, the
%  units that you encounter include:
%  \begin{itemize}
%  \item Velocity, measured in \emph{metres per second} [\si{\metre\per\second}]
%  \item Force, measured in \emph{newtons} [\si\newton], defined as:
%    [\si{\kilo\gram\metre\per\second\squared}]
%  \item Work and energy, measured as \emph{joules} [\si\joule], defined as:
%    [\si{\kilo\gram\metre\squared\per\second\squared}]
%  \item And power is measured in \emph{watts} [\si\watt], defined as:
%    [\si{\kilo\gram\metre\squared\per\second\cubed}]
%  \end{itemize}
%  There will be more units in the course
%\end{frame}
%
%
%
%\begin{frame}{SI Unit Prefixes}
%  \begin{columns}
%    \column{.67\textwidth}
%    A \textbf{unit prefix} can be added to units of measurement to indicate
%    multiples of the units. Accepted SI unit prefixes indicate powers of 10.
%
%    \vspace{.1in}\textbf{Example:} A distance of $d=\SI{545}\metre$ can also
%    be written as:
%
%    \vspace{-.3in}{\large
%      \begin{align*}
%        d &= \SI{5.45e2}\metre\\
%        d &= \SI{.545}{\kilo\metre}
%      \end{align*}
%    }
%    %\vspace{-.25in}Common unit prefixes are shown on the table.
%    
%    \column{.3\textwidth}
%    \centering
%    \textbf{Common SI prefixes:}
%    
%    \begin{tabular}{|l|l|c|}
%      \hline
%      tera  & $10^{12}$ & T \\
%      giga  & $10^9$  & G \\
%      mega  & $10^6$  & M \\
%      kilo  & $10^3$  & k \\
%      centi & $10^{-2}$ & c \\
%      milli & $10^{-3}$ & m \\
%      micro & $10^{-6}$ & $\mu$ \\
%      nano  & $10^{-9}$ & n \\
%      \hline
%    \end{tabular}
%  \end{columns}
%\end{frame}
%
%
%
%\subsection{Unit Conversion}
%
%\begin{frame}{Unit Conversion}
%  Not all quantities are given to you are in SI units. For example, The SI unit
%  for speed is metres per second (\si{\metre\per\second}), but kilometres per
%  hour (\si{\kilo\metre\per\hour}) is commonly used around the world, while the
%  imperial unit miles per hour (\si{mph}) is used in the United States.
%  \begin{center}
%    \pic{.35}{images/MP9004482912}
%  \end{center}
%  As scientists, in order to communicate ideas with people who use different
%  units, we have to know how to convert them.
%\end{frame}
%
%
%
%\begin{frame}{Unit Conversion}
%  Generally, the conversion ratio is given to you (or you may be asked to search
%  for it online or in books). For example, to convert from
%  \si{\metre\per\second} to \si{\kilo\metre\per\hour}, the ratio is:
%
%  \eq{-.1in}{
%    \SI1{\metre\per\second}=\SI{3.6}{\kilo\metre\per\hour}
%  }
%
%  \vspace{-.15in}which I can re-write as
%
%  \eq{-.1in}{
%    \frac{\SI1{\metre\per\second}}{\SI{3.6}{\kilo\metre\per\hour}}=1
%    \quad\text{or}\quad
%    \frac{\SI{3.6}{\kilo\metre\per\hour}}{\SI1{\metre\per\second}}=1
%  }
%
%  The advantage of writing this way is that multiplying any numbers by 1 will
%  never change the number itself. This is the key to unit conversion.
%\end{frame}
%
%
%
%\begin{frame}{Unit Conversion}
%  If you wish to convert \SI{90}{\kilo\metre\per\hour} to
%  \si{\metre\per\second}, you can simply multiply by the ratio:
%
%  \eq{-.1in}{
%    90\;\cancel{\si{\kilo\metre\per\hour}}
%    \times
%    \boxed{
%      \frac{\SI1{\metre\per\second}}
%           {3.6\;\cancel{\si{\kilo\metre\per\hour}}}
%    }
%    =\SI{25}{\metre\per\second}
%  }
%
%  The \si{\kilo\metre\per\hour} units on the left-hand-side will cancel---yes,
%  units cancel too---leaving the same quantity in
%  \si{\metre\per\second}. This works backwards as well, for example, to convert
%  \SI{30.}{\metre\per\second} to \si{\kilo\metre\per\hour}:
%
%  \eq{-.1in}{
%    \num{30.0}\;\cancel{\si{\metre\per\second}}
%    \times
%    \boxed{
%      \frac{\SI{3.6}{\kilo\metre\per\hour}}
%           {\num1\;\cancel{\si{\metre\per\second}}}
%    }
%    =\SI{108}{\kilo\metre\per\hour}
%  }
%\end{frame}
%
%
%
%\begin{frame}{Accuracy vs.\ Precision}
%  When solving a problem in your physics homework, you have to deal with both
%  the \emph{accuracy} and \emph{precision} of the numerical values that you
%  are given. Using a dart board as an analogy:
%  \begin{center}
%    \pic{.9}{images/Accuracy-vs-precision1}
%  \end{center}
%  
%  \begin{itemize}
%  \item\textbf{Accuracy:} How close the darts are to the centre
%  \item\textbf{Precision:} How close the darts are to each other
%  \end{itemize}
%\end{frame}
%
%
%
%
%
%
%\begin{frame}{Kinematics}
%  \textbf{Kinematics} is a branch of mechanics that mathematically
%  \emph{describes} the motion of objects, by using these motion quantities:
%  \begin{itemize}
%  \item Position
%  \item Displacement
%  \item Distance
%  \item Velocity (instantaneous and average)
%  \item Speed
%  \item Acceleration (instantaneous and average)
%  \end{itemize}
%  Kinematics does not deal with \emph{what} causes motion (or what causes
%  motion to change), only the relationship between motion quantities
%\end{frame}
%
%
%
%\section{Vectors}
%
%\begin{frame}{Vectors vs.\ Scalars}
%  When describing a physical quantity, sometimes direction matters. Quantities
%  that require a direction called a \textbf{vector}.
%  \begin{itemize}
%  \item ``I am \SI{15}{\kilo\metre} from Meritus Academy''
%  \item Without knowing the direction, there is no way of knowing where I am
%  \end{itemize}
%
%  \vspace{.1in}For some other quantities, no direction is required. These are
%  called \emph{scalar} quantities.
%  \begin{itemize}
%  \item The mass of my bike is \SI{7.1}{\kilo\gram}
%  \item The current world record for the \SI{100}{\metre} dash is
%    \SI{9.58}\second
%  \end{itemize}
%\end{frame}
%
%
%
%\begin{frame}{Vector}
%  When direction matters, we use a mathematical concept called a \textbf{vector}
%  \begin{itemize}
%  \item A vector has two parts:
%    \begin{itemize}
%    \item\textbf{Magnitude:} how much of that quantity
%    \item\textbf{Direction:} which way
%    \end{itemize}
%  \item Usually represented by an arrow: the length is the magnitude
%    \begin{center}
%      \begin{tikzpicture}[rotate=25,scale=.75]
%        \draw (-.5,0)--(6,0) node[below right]{\scriptsize Line of Action};
%        \draw[vectors,red] (0,0)--(3,0) node[midway,above left]{$\vec A$};
%      \end{tikzpicture}
%    \end{center}
%  \item Vector quantities have an arrow written above the symbol:
%
%    \eq{-.1in}{\vec d\quad\vec v\quad\vec a\quad\vec F\ldots}
%  \end{itemize}
%\end{frame}
%
%
%
%\begin{frame}{Examples of Vector Quantities}
%    These are vector quantities that require direction:
%    \begin{itemize}
%    \item Position ($\vec d$)
%    \item Displacement ($\Delta \vec d$)
%    \item Velocity ($\vec v$)
%    \item Acceleration ($\vec a$)
%    \item Force ($\vec F$)
%    \item Magnetic Field ($\vec B$)
%    \end{itemize}
%\end{frame}
%
%
%\begin{frame}{Examples of Scalar Quantities}
%  These are scalar quantities we will encounter in Physics 11 that do not
%  require directions:
%  \begin{columns}[T]
%    \column{.33\textwidth}
%    \begin{itemize}
%    \item Distance ($s$)
%    \item Area ($A$)
%    \item Volume ($V$)
%    \item Speed ($v$)
%    \item Time ($t$)
%    \end{itemize}
%
%    \column{.33\textwidth}
%    \begin{itemize}
%    \item Mass ($m$)
%    \item Mechanical work ($W$)
%    \item Energy ($E$)
%    \item Heat ($Q$)
%    \item Frequency ($f$)
%    \end{itemize}
%
%    \column{.33\textwidth}
%    \begin{itemize}
%    \item Wavelength ($\lambda$)
%    \item Electric Charge ($q$)
%    \item Resistance ($R$)
%    \item Voltage ($V$)
%    \end{itemize}
%  \end{columns}
%\end{frame}
%
%
%
%\begin{frame}{Writing Vectors} %{Vectors in Two or More Dimensions}
%  Any vector can be expressed by explicitly stating their magnitude and
%  direction:
%  \begin{center}
%    {\large
%      magnitude [direction]
%    }
%  \end{center}
%
% \begin{itemize}
% \item If I walk \SI{22}{\kilo\metre} to the north to come to class at
%   Meritus Academy\footnote{Reality check: \SI{22}{\kilo\metre} south of
%     Meritus is in the \emph{lake}}, I move by:
%
%   \begin{center}
%     {\large
%       \SI{22}{\kilo\metre} [N]
%     }
%   \end{center}
% \item If I fly a small airplane north-west towards Sudbury, I travel with a
%   velocity of:
%   
%   \begin{center}
%     {\large
%       \SI{125}{\metre\per\second} [NW]
%     }
%   \end{center}
% \end{itemize}
%\end{frame}
%
%
%
%\begin{frame}{Vectors in One Dimension}
%  For vectors in 1D we only need to use a ($+$) and ($-$) signs to indicate
%  direction, like on a number line.
%  \begin{itemize}
%  \item Define a positive direction, e.g.
%    \begin{itemize}
%    \item If [North] is positive, then [South] is negative
%    \item If [up] is positive, then [down] is negative
%    \item If [left] is positive, then [right] is negative
%    \end{itemize}
%  \item When answering a problem, be sure to indicate which way is positive!
%  \item After that, use ($+$) and ($-$) signs to describe the direction. e.g.
%
%    \eq{-.15in}{
%      -\SI{28}{\metre\per\second}
%    }
%
%    \vspace{-.2in}
%    describes an object that is moving towards the ($-$) direction at
%    \SI{28}{\metre\per\second}. We need to know what the ($+$) direction is
%    before this has any meaning.
%  \end{itemize}
%\end{frame}
%
%
%\section{Motion Quantities}
%
%\subsection{Position}
%
%\begin{frame}{Position}
%
%  \textbf{Position} $\vec d$ is the location of any object in a coordinate
%  system
%  \begin{itemize}
%  \item A vector measured from the origin of the coordinate system (also called
%    the \textbf{reference point})
%  \item The coordinate system should be selected to simplify calculations
%  \item The symbol for position is sometimes $\vec x$ or $\vec r$
%  \item The SI unit for position is a \emph{metre} (\si\metre)
%  \end{itemize}
%\end{frame}
%
%
%
%\begin{frame}{Coordinate Systems}
%  \begin{columns}[t]
%    \column{.33\textwidth}
%    The 1D coordinate system is the number line.
%
%    \vspace{.5in}
%    \begin{center}
%      \begin{tikzpicture}
%        \draw[axes] (-2,0)--(2,0) node[right]{+};
%        \draw[thick] (0,.08)--(0,-.08) node[below=-1]{$O$};
%      \end{tikzpicture}
%    \end{center}
%    \vspace{.3in}
%    
%    \column{.33\textwidth}
%    The 2D coordinate system is the $xy$-plane.
%    \begin{center}
%      \begin{tikzpicture}
%        \draw[axes] (0,0)--(2,0) node[right]{$x$};
%        \draw[axes] (0,0)--(0,2) node[above]{$y$};
%      \end{tikzpicture}
%    \end{center}
%    
%    \column{.33\textwidth}
%    The 3D coordinate system is the Cartesian ($xyz$) space.
%    \begin{center}
%      \begin{tikzpicture}
%        \draw[axes] (0,0)--(1.6,-.5) node[right]{$y$};
%        \draw[axes] (0,0)--(-1,-.6) node[left]{$x$};
%        \draw[axes] (0,0)--(0,1.7) node[above]{$z$};
%      \end{tikzpicture}
%    \end{center}
%  \end{columns}
%  \vspace{.1in}Grade 11 Physics only focuses on one- and two-dimensional
%  problems
%\end{frame}
%
%
%
%\begin{frame}{Position in 1D}
%  A one-dimensional coordinate system is the number line, and the position
%  is the coordinate along the line. For example, if each tick represents
%  \SI1\metre, then
%
%  \vspace{.1in}
%  \begin{center}
%    \begin{tikzpicture}
%      \draw[axes] (-6,0)--(6,0) node[pos=0,left]{$-$} node[right]{$+$};
%      \foreach \x in {-5,...,5} \draw (\x,.1)--(\x,-.1);
%      \draw[thick] (0,.1)--(0,-.1) node[below]{$O$};
%      \fill[red] (3.2,0) circle (.05) node[below]{$A$};
%      \draw[vectors,red] (0,0)--(3.2,0) node[midway,above]{$\vec d_A$};
%      \fill[blue](-4,0) circle (.05) node[below]{$B$};
%      \draw[vectors,blue] (0,0)--(-4,0) node[midway,above]{$\vec d_B$};
%    \end{tikzpicture}
%  \end{center}
%  \begin{itemize}
%  \item The position of $A$ is {\color{red}$\vec d_A=+\SI{3.2}\metre$}; the
%    position of $B$ is {\color{blue}$\vec d_B=\SI{-4.0}\metre$}
%  \item The position vector is represented by an arrow drawn from the origin
%  \item The positive direction and the location of the origin must be known
%    \emph{a priori}\footnote{It means ahead of time}
%  \item For 1D, the arrow above the variable is usually omitted
%  \end{itemize}
%\end{frame}
%
%
%
%\begin{frame}{Position in 2D}
%  The two-dimensional coordinate system used in Grade 11 Physics is the
%  $xy$-plane, called the \textbf{cartesian coordinate system}.
%
%  \vspace{.2in}
%  \begin{columns}
%    \column{.3\textwidth}
%    \centering
%    \begin{tikzpicture}
%      \fill[red] (3,2) circle (.05) node[above]{$P(x,y)$};
%      \draw[vectors,red] (0,0)--(3,2) node[midway,above]{$\vec d_P$};
%      \draw[axes] (-.2,0)--(3.5,0) node[right]{$x$};
%      \draw[axes] (0,-.2)--(0,2.5) node[above]{$y$};
%      \draw[axes] (1.5,0) arc(0:atan(2/3):1.5) node[midway,right]{$\theta$};
%      \node[below left] at (0,0) {$O$};
%    \end{tikzpicture}
%
%    \column{.65\textwidth}
%    The position vector {\color{red}{$\vec d_P$}} is an arrow drawn from the
%    origin to $P$. There are two ways to describe the vector:
%    \begin{itemize}
%    \item By the $(x,y)$ coordinates, or
%    \item By the length of the arrow (magnitude) and the angle $\theta$ with
%      the $x$-axis (direction)
%    \end{itemize}
%    We will study 2D vectors in more depth in Classes 3 and 4
%  \end{columns}
%\end{frame}
%
%
%
%\subsection{Displacement}
%
%\begin{frame}{Displacement: Change in Position}
%  \textbf{Displacement} $\Delta\vec d$ is the change in position when an
%  object moves. It is the difference between the final position (where
%  motion ends) $\vec d_2$ and initial position (where motion begins)
%  $\vec d_1$:
%  
%  \eq{-.1in}{
%    \boxed{\Delta\vec d=\vec d_2-\vec d_1}
%  }
%  \begin{center}
%    \begin{tabular}{l|c|c}
%      \rowcolor{pink}
%      \textbf{Quantity} & \textbf{Symbol} & \textbf{SI Unit} \\ \hline
%      Displacement      & $\Delta\vec d$ & \si\metre\\
%      Initial position  & $\vec d_1$ & \si\metre \\
%      Final position    & $\vec d_2$ & \si\metre
%    \end{tabular}
%  \end{center}
%  \uncover<3>{
%    Displacement is posed as a \emph{subraction} of two vectors, but at this
%    time it may be unclear what this subtraction actually means (there is a 1D
%    example later)
%  }
%
%  \begin{tikzpicture}[overlay]
%    \uncover<2->{
%      \node[text width=106,draw=magenta,fill=magenta!5,text=magenta] (act) at
%      (2,3.6) {The Greek letter ``$\Delta$'' (``delta'') means ``change'' in
%        mathematics, so $\Delta\vec d$ literally means ``the change in the
%        position vector'', which is what displacement is};
%      \draw[axes,magenta] (act) to[out=20,in=180] +(4,1);
%    }
%  \end{tikzpicture}
%\end{frame}
%
%
%
%\begin{frame}{Displacement in 1D}
%  Expressing displacement in one dimension is straightforward. When an object
%  moves from position 1 to 2, then $\Delta\vec d$ is represented by the arrow
%  pointing from 1 towards 2:
%
%  \begin{center}
%    \begin{tikzpicture}
%      \draw[axes] (-6,0)--(6,0) node[pos=0,left]{$-$} node[right]{$+$};
%      \foreach \x in {-5,...,5} \draw (\x,.1)--(\x,-.1);
%      \draw[thick] (0,.1)--(0,-.15) node[below]{$O$};
%      \fill[red] (3.2,0) circle (.05) node[above]{1};
%      \fill[red] (-2.5,0) circle (.05) node[above]{2};
%      \draw[vectors,red] (3.2,0)--(-2.5,0) node[midway,above]{$\Delta\vec d$};
%      \draw[vectors,blue] (0,-.07)--(3.2,-.07) node[midway,below]{$\vec d_1$};
%      \draw[vectors,violet] (0,-.07)--(-2.5,-.07)
%      node[midway,below]{$\vec d_2$};
%    \end{tikzpicture}
%  \end{center}
%  \textbf{Example:} Using the equation from the last slide,
%  \begin{displaymath}
%    \Delta\vec d
%    ={\color{violet}\vec d_2}-{\color{blue}\vec d_1}
%    =(\SI{-2.5}\metre)-(\SI{3.2}\metre)=\boxed{{\color{orange}-}\SI{5.7}\metre}
%  \end{displaymath}
%  The displacement vector has a magnitude of \SI{5.7}{\metre} towards the
%  {\color{orange}negative direction}.
%\end{frame}
%
%
%
%\begin{frame}{Displacement in 2D}
%  \begin{columns}
%    \column{.6\textwidth}
%    Expressing displacement in 2D is more complicated, but still follows the
%    same idea as in 1D. When an object moves from position 1 to 2, then
%    $\Delta\vec d$ is represented by the arrow pointing from 1 towards 2.
%
%    \column{.35\textwidth}
%    \begin{tikzpicture}
%      \draw[vectors,magenta] (0,0)--(3,1) node[pos=.6,below]{$\vec d_1$};
%      \draw[vectors,magenta] (0,0)--(.5,2.5) node[midway,right] {$\vec d_2$};
%      \draw[vectors] (3,1)--(.5,2.5) node[midway,above=3]{$\Delta\vec d$};
%      \fill[magenta] (3,1) circle (.05) node[right]{1};
%      \fill[magenta] (.5,2.5) circle (.05) node[above]{2};
%      \draw[axes] (-.2,0)--(3.5,0) node[right]{$x$};
%      \draw[axes] (0,-.2)--(0,3.5) node[above]{$y$};
%      \node[below left] at (0,0) {$O$};
%    \end{tikzpicture}
%  \end{columns}
%\end{frame}
%
%
%
%\subsection{Distance}
%
%\begin{frame}{Distance}
%  Distance ($s$) is a motion quantity that is related to displacement
%  \begin{itemize}
%  \item Distance is a \emph{scalar} quantity (no direction)
%  \item The length of the path between two positions, and therefore
%  \item Depends on \emph{how} an object travels from one point to another
%  \item Non-negative ($s\geq 0$)
%  \end{itemize}
%  In the 1D example below, the displacement is \SI{-5.7}\metre, but the
%  distance travelled is much more than that.
%  \begin{center}
%    \begin{tikzpicture}
%      \draw[axes] (-6,0)--(6,0) node[pos=0,left]{$-$} node[right]{$+$};
%      \foreach \x in {-5,...,5} \draw (\x,.1)--(\x,-.1);
%      \draw[thick] (0,.1)--(0,-.1);
%      \fill[red] (3.2,0) circle (.05) node[above]{1};
%      \fill[red] (-2.5,0) circle (.05) node[above]{2};
%      \draw[vectors,red] (3.2,0)--(-2.5,0) node[midway,above]{$\Delta\vec d$};
%      \draw[very thick,blue] (3.2,-.2)--(1,-.2) arc(90:270:.1)
%      --(2.5,-.4) arc(90:-90:.1)--(-1.7,-.6) node[midway,below]{$s$}
%      arc(270:90:.1)--(0,-.4) arc(-90:90:.1)--(-2.5,-.2);
%    \end{tikzpicture}
%  \end{center}
%\end{frame}
%
%
%
%\begin{frame}{Distance}
%  \begin{columns}
%    \column{.35\textwidth}
%    \centering
%    \begin{tikzpicture}
%      \draw[vectors] (3,1)--(.5,2.5) node[midway,above=2.5]{$\Delta\vec d$};
%      \begin{scope}[very thick]
%        \draw[blue] (3,1) to[out=90,in=45] (.5,2.5);
%        \draw[green] (3,1) ..controls(2,.5) and (1,0).. (.5,2.5);
%        \draw[orange] (3,1) ..controls(6,2) and (1,4).. (.5,2.5);
%        \draw[violet] (3,1) ..controls(2.5,-2) and (0,0).. (.5,2.5);
%      \end{scope}
%      \fill (3,1) circle (.05) node[below right=-1]{1};
%      \fill (.5,2.5) circle (.05) node[left]{2};
%      \draw[axes] (-.2,0)--(3.5,0) node[right]{$x$};
%      \draw[axes] (0,-.2)--(0,3.5) node[above]{$y$};
%      \node[below left] at (0,0) {$O$};
%    \end{tikzpicture}
%
%    \column{.6\textwidth}
%    The {\color{violet}purple}, {\color{green}green}, {\color{blue}blue} and
%    {\color{orange}orange} paths all have the same displacement $\Delta\vec d$
%    (start at position 1 and ending at position 2), but their distances are
%    different
%  \end{columns}
%\end{frame}
%
%
%
%\begin{frame}{Example}
%  \begin{columns}
%    \column{.63\textwidth}
%    \textbf{Example:} Determine the 3 displacements between Freda's home and
%    the other three buildings:
%    \begin{itemize}
%    \item Home to school%: \uncover<2->{\magdir{\SI{141}\metre}{NE}}
%    \item Home to diner%: \uncover<3->{\magdir{\SI{200}\metre}{N}}
%    \item Home to sports complex%:
%      %\uncover<4->{\magdir{\SI{361}\metre}{N \ang{33.7} W}}
%    \end{itemize}
%    
%    \column{.37\textwidth}
%    \pic1{images/town-map}
%  \end{columns}
%\end{frame}
%
%
%
%\begin{frame}{Example}
%  \begin{columns}
%    \column{.63\textwidth}
%    \textbf{Example:} What is Freda's total distance and displacement if she
%    goes to school from home, then to the diner, then sports complex everyday?
%
%    \column{.37\textwidth}
%    \pic1{images/town-map}
%  \end{columns}
%\end{frame}
%
%
%
%\begin{frame}{Example}
%  \begin{columns}
%    \column{.63\textwidth}
%    Assuming that she goes home every night (a very reasonable assumption,
%    because if she doesn't, she can't start the next day from home again!)
%    \begin{itemize}
%    \item Displacement: $\Delta\vec d=\vec 0$
%    \item Total distance cannot be determined because we don't know what path
%      she takes
%    \end{itemize}
%
%    \column{.37\textwidth}
%    \pic1{images/town-map}
%  \end{columns}
%\end{frame}
%
%
%
%\begin{frame}{Another Simple Example}
%  \textbf{Example}: Dawn starts from her home, and bikes \SI{2.0}{\kilo\metre}
%  east, then \SI{3.0}{\kilo\metre} west.
%  \begin{enumerate}[(a)]
%  \item Draw a vector diagram, and
%  \item Find her displacement
%  \end{enumerate}
%
%  \vspace{.5in}
%  \textbf{Example}: Dawn starts from home again, but this time she goes
%  \SI{2.0}{\kilo\metre} north and then \SI{3.0}{\kilo\metre} east.
%  \begin{enumerate}[(a)]
%  \item Draw a vector diagram, and
%  \item Find her displacement
%  \end{enumerate}
%\end{frame}
%
%
%
%\subsection{Velocity}
%
%\begin{frame}{Velocity}
%  \textbf{Velocity} is how quickly displacement changes with time. Like
%  displacement, velocity is also a vector. \textbf{Average velocity} is defined
%  as:
%
%  \eq{-.1in}{
%    \boxed{
%      \vec v_\text{avg}
%      =\frac{\Delta\vec d}{\Delta t}
%      =\frac{\vec d_2-\vec d_1}{t_2-t_1}
%    }
%  }
%  \begin{center}
%    \begin{tabular}{l|c|l}
%      \rowcolor{pink}
%      \textbf{Quantity} & \textbf{Symbol} & \textbf{SI Unit} \\ \hline
%      Average velocity & $\vec v_\text{avg}$ &
%      \si{\metre\per\second} (metres per second)\\
%      Displacement     & $\Delta\vec d$ & \si{\metre} (metres) \\
%      Time interval    & $\Delta t$     & \si{\second} (seconds)
%    \end{tabular}
%  \end{center}
%  On the other hand, \textbf{instantaneous velocity} is how quickly
%  displacement is changing at an instance in time.
%\end{frame}
%
%
%
%\subsection{Speed}
%
%\begin{frame}{Speed}
%  Both velocity and speed are used to describe how \emph{fast} an object is
%  moving, but they are not the same! \textbf{Average speed} is how quickly
%  \emph{distance} is changing with time:
%
%  \eq{-.1in}{
%    \boxed{
%      v_\text{avg}
%      =\frac s{\Delta t}
%    }
%  }
%  \begin{center}
%    \begin{tabular}{l|c|l}
%      \rowcolor{pink}
%      \textbf{Quantity} & \textbf{Symbol} & \textbf{SI Unit} \\ \hline
%      Average speed & $v_\text{avg}$ &
%      \si{\metre\per\second} (metres per second)\\
%      Distance travelled & $s$        & \si{\metre} (metres) \\
%      Time interval      & $\Delta t$ & \si{\second} (seconds)
%    \end{tabular}
%  \end{center}
%  On the other hand, \textbf{instantaneous speed} is how quickly distance is
%  changing at an instance in time.
%\end{frame}
%
%
%
%\begin{frame}{Velocity vs.\ Speed}
%  Difference between average velocity and average speed:
%  \begin{itemize}
%  \item Velocity is a vector (needs a direction), but speed is a scalar (no
%    direction)
%  \item Since displacement is different from distance, an object that has
%    travelled a long distance but returns to the initial position will have
%    non-zero average speed but zero average velocity.
%  \end{itemize}
%  In contrast, instantaneous speed is just the magnitude of the instantaneous
%  velocity vector.
%\end{frame}
%
%
%
%\begin{frame}{Example Problems}
%  \textbf{Example}: Does speedometer of a car indicate speed or velocity?
%
%  \vspace{.5in}
%  \textbf{Example}: A student runs around a \SI{400}{\metre}
%  oval track in \SI{80}\second. Would the average velocity and average
%  speed be the same? Explain?
%\end{frame}
%
%
%
%\begin{frame}{Example Problem}
%  \textbf{Example}: A dragster in a race is timed at the 200 m and 400 m
%  points. The time are shown on the stopwatches in the diagram. Calculate the
%  average velocity for:
%  \begin{enumerate}
%  \item the first 200 m
%  \item the next 200 m, and
%  \item the entire race
%  \end{enumerate}
%  \begin{center}
%    \pic{.8}{images/dragrace}
%  \end{center}
%\end{frame}
%
%
%
%\subsection{Acceleration}
%
%\begin{frame}{Average Acceleration}
%  \textbf{Acceleration} is the rate of change of instantaneous velocity (how
%  quickly instantaneous velocity changes with time). It is also a vector:
%
%  \eq{-.1in}{
%    \boxed{
%      \vec a_\text{avg} = \frac{\Delta\vec v}{\Delta t}
%      = \frac{\vec v_2-\vec v_1}{t_2-t_1}
%    }
%  }
%  \begin{center}
%    \begin{tabular}{l|c|l}
%      \rowcolor{pink}
%      \textbf{Quantity} & \textbf{Symbol} & \textbf{SI Unit} \\ \hline
%      Average acceleration & $\vec a_\text{avg}$ &
%      \si{\metre\per\second\squared} (metres per second squared) \\
%      Initial and final velocities & $\vec v_1$, $\vec v_2$ &
%      \si{\metre\per\second} (metres per second) \\
%      Change in velocity & $\Delta\vec v$ &
%      \si{\metre\per\second} (metres per second) \\
%      Time interval      & $\Delta t$ & \si{\second} (seconds)
%    \end{tabular}
%  \end{center}
%  It may be easier to think of the unit for acceleration as ``metre per second,
%  per second''
%\end{frame}
%
%
%
%%\begin{frame}{Acceleration}
%%  \begin{itemize}
%%  \item A change in the velocity vector can mean a change in magnitude and/or
%%    direction
%%  \item There can be acceleration without any speeding up or slowing down!
%%  \item Think about what happens if a car is turning at constant speed
%%  \end{itemize}
%%\end{frame}
%
%
%
%\begin{frame}{Example Problems}
%  \textbf{Example}: A skater is going \SI{15}{\meter\per\second} at $t=0$.
%  After \SI{5.0}\second, he is going \SI{5.0}{\metre\per\second}. What is his
%  average acceleration?
%  
%  \vspace{.6in}
%  \textbf{Example}: A sports car is travelling in a straight line along a
%  highway at a speed of \SI{20}{\metre\per\second}. The driver steps on the
%  gas, and \SI{3.0}{\second} later, the car's speed is
%  \SI{32}{\metre\per\second}. Find its average acceleration.
%\end{frame}
%
%
%
%\section{Summary}
%
%\begin{frame}{Summary: Position in 1D}
%  For motion in one dimension, the position is like a point on the number line,
%  where:
%  \begin{itemize}
%  \item{\color{red}($+$) Position}: on the positive side of the origin
%  \item{\color{blue}($-$) Position}: on the negative side of the origin
%  \end{itemize}
%  \begin{center}
%    \begin{tikzpicture}
%      \draw[axes] (-6,0)--(6,0) node[pos=0,left] {$-$} node[right]{$+$};
%      \draw[thick] (0,.1)--(0,-.1) node[below]{0};
%
%      \fill[red] (3,0) circle (.05);
%      \draw[vectors,red] (0,0)--(3,0);
%      
%      \fill[blue] (-4,0) circle (.05);
%      \draw[vectors,blue] (0,0)--(-4,0);
%    \end{tikzpicture}
%  \end{center}
%  It doesn't matter which way the object is moving, only \emph{where} it is at
%  that moment
%\end{frame}
%
%
%
%\begin{frame}{Summary: Displacement \& Average Velocity} 
%  When expressing displacement of an object, it does not matter where the
%  object is relative to the reference point, only where it has gone
%  \begin{itemize}
%  \item{\color{red}($+$) Displacement}: moved in the positive direction
%  \item{\color{blue}($-$) Displacement}: moved in the negative direction
%  \end{itemize}
%  Average velocity is in the same direction as displacement
%  
%  \vspace{.2in}
%  \begin{center}
%    \begin{tikzpicture}
%      \draw[axes] (-6,0)--(6,0) node[pos=0,left]{$-$} node[right]{$+$};
%      \draw[thick] (-3,.1)--(-3,-.1) node[below]{0};
%
%      \draw[vectors,red] (-3.93,0)--(.5,0) node[midway,above]{$\Delta d$};
%      \fill[pink] (-4,0) circle (.05);
%      \fill[red] (.5,0) circle (.05);
%      
%      \draw[vectors,blue] (5,0)--(3,0) node[midway,above]{$\Delta d$};
%      \fill[blue!30] (5,0) circle (.05);
%      \fill[blue] (3,0) circle (.05);
%    \end{tikzpicture}
%  \end{center}
%\end{frame}
%
%
%
%\begin{frame}{Summary: Instantaneous Velocity}  
%  For intstantaneous velocity, it does not matter where the object is, only
%  where it is going
%  \begin{itemize}
%  \item{\color{red}($+$) velocity}: moving in the ($+$) direction
%  \item{\color{blue}($-$) velocity}: moving in the ($-$) direction
%  \end{itemize}
%  \begin{center}
%    \begin{tikzpicture}
%      \draw[axes] (-6,0)--(6,0) node[pos=0,left] {$-$} node[right]{$+$};
%      \draw[thick] (0,.1)--(0,-.1) node[below]{0};
%
%      \fill[red] (2,0) circle (.05);
%      \draw[vectors,red] (2,0)--(3.5,0) node[midway,above]{$\vec v>0$};
%
%      \fill[blue] (-.5,0) circle (.05);
%      \draw[vectors,blue] (-.5,0)--(-3,0) node[midway,above]{$\vec v<0$};
%    \end{tikzpicture}
%  \end{center}
%\end{frame}
%
%
%
%\begin{frame}{Summary: Acceleration}
%  Acceleration is a bit tricky; we have to know the velocity vector as well.
%  \begin{itemize}
%  \item If acceleration and velocity has the same sign, the object is speeding
%    up (speed increasing)
%  \item If the signs are opposite, then the object is slowing down
%  \end{itemize}
%\end{frame}
%
%
%
%\begin{frame}{Summary: Acceleration}
%  Example: Positive velocity
%  \begin{itemize}
%  \item{\color{red}($+$) velocity \& ($+$) acceleration:} moving in the
%    ($+$) direction \& speeding up
%  \item{\color{blue}($+$) velocity \& ($-$) acceleration:} moving in the
%    ($+$) direction \& slowing down
%  \end{itemize}
%  \begin{center}
%    \begin{tikzpicture}
%      \draw[axes] (-6,0)--(6,0) node[pos=0,left] {$-$} node[right]{$+$};
%      \draw[thick] (0,.1)--(0,-.1) node[below]{0};
%
%      \fill[red] (2,0) circle (.05);
%      \draw[vectors,red] (2,0)--(4.5,0) node[above]{$\vec v>0$};
%      \draw[vectors,red] (2.5,-1)--(3.5,-1) node[above]{$\vec a>0$};
%
%      \fill[blue] (-3,0) circle (.05);
%      \draw[vectors,blue] (-3,0)--(-1.5,0) node[above]{$\vec v>0$};
%      \draw[vectors,blue] (-2,-1)--(-3,-1) node[above]{$\vec a<0$};
%    \end{tikzpicture}
%  \end{center}
%  If acceleration has the same sign as velocity, then the object speeds up; if
%  the signs are opposite, the object slows down.
%\end{frame}
%
%
%
%\begin{frame}{Summary: Acceleration}
%  Example: Negative velocity
%  \begin{itemize}
%  \item{\color{red}($-$) velocity \& ($+$) acceleration:} moving in the
%    ($-$) direction \& slowing down
%  \item{\color{blue}($-$) velocity \& ($-$) acceleration:} moving in the
%    ($-$) direction \& speeding up
%  \end{itemize}
%  \begin{center}
%    \begin{tikzpicture}
%      \draw[axes] (-6,0)--(6,0) node[pos=0,left] {$-$} node[right]{$+$};
%      \draw[thick] (0,.1)--(0,-.1) node[below]{0};
%
%      \fill[red] (4.5,0) circle (.05);
%      \draw[vectors,red] (4.5,0)--(1,0) node[above]{$\vec v<0$};
%      \draw[vectors,red] (2.5,-1)--(4,-1) node[above]{$\vec a>0$};
%
%      \fill[blue] (-1,0) circle (.05);
%      \draw[vectors,blue] (-1,0)--(-3,0) node[above]{$\vec v<0$};
%      \draw[vectors,blue] (-1.5,-1)--(-2.5,-1) node[above]{$\vec a<0$};
%    \end{tikzpicture}
%  \end{center}
%\end{frame}
\end{document}

