\documentclass[12pt]{../../ossphysics}

\begin{document}
\setheader{Physics 12 Class 13 Homework}
\hwtitle{12}{13}{Thin-Film Interference}

\begin{questions}
  \question Thin films of soap sometime display an array of colours. This
  display is the result of:
  \begin{choices}
    \choice reflection, diffraction, and interference
    \choice reflection, refraction, and interference
    \choice reflection, refraction, and polarization
    \choice refraction, interference, and polarization
    \choice reflection, interference, and polarization
  \end{choices}

  \question A beam of light is unpolarized. This means that:
  \begin{choices}
    \choice vibrations are confined to a single plane
    \choice vibrations are occurring in all possible directions
    \choice vibrations are occurring in all directions perpendicular to the
    direction of light propagation
    \choice light has reflected from a horizontal surface
    \choice light has passed through a calcite crystal which causes double
    refraction
  \end{choices}

  \question Green light ($\lambda=\SI{550}{\nano\metre}$) is shone onto two soap
  films. An observer looking down at the soap films sees that soap film $X$
  appears uniformly green while soap film $Y$ shows green and black bands. If
  air is the medium on either side of the soap film, the best explanation of
  this pattern is that:
  \begin{choices}
    \choice film $X$ has a thickness much less than $\lambda$ and film $Y$ has
    a thickness of $\lambda/4$
    \choice film $X$ has a thickness of $\lambda/2$ and film $Y$ has a
    thickness of$\lambda/4$
    \choice film $X$ has a thickness of $\lambda/4$ and film $Y$ has a
    thickness of $\lambda/2$
    \choice film $X$ has consistent thickness throughout whereas film $Y$ has
    variable thickness
    \choice film $X$ has variable thickness whereas film $Y$ has consistent
    thickness throughout
  \end{choices}
  
  \question Which of the following correctly describes the motion of the
  electric and magnetic fields of a microwave transmitted by a cell phone?
  \begin{choices}
    \choice Both the electric and magnetic fields oscillate in the same plane
    and perpendicular to the direction of wave propagation.
    \choice Both the electric and magnetic fields oscillate perpendicular to
    each other and to the direction of wave propagation.
    \choice The electric field oscillates perpendicular to the direction of wave
    propagation. The magnetic field oscillates parallel to the direction
    of wave propagation.
    \choice Both the electric and magnetic fields oscillate parallel to the
    direction of wave propagation.
  \end{choices}
  
  \question The telescope on Mount Palomar has a diameter of 200 inches
  ($\SI1{in}=\SI{2.54e-2}{\metre}$). Suppose a double star were 4.0 light
  years away. Under ideal conditions, what must be the minimum separation of
  the 2 stars for their image to be resolved? (Use a wavelength of 400 nm.)
  \vspace{\stretch1}
  \newpage
  
  \question For a ruby laser of wavelength \SI{694}{\nano\metre}, the end of
  the ruby crystal is the aperture that determines the diameter of the light
  beam emitted. If the diameter is \SI{1.50}{\centi\metre} and the laser is
  aimed at the moon, find the approximate diameter of the light beam when it
  reaches the moon, assuming the spread is due solely to diffraction?
  \vspace{\stretch1}
  
%  \question You are on an international panel charged with allocation ``real estate''
%  for communication satellites in geosynchronous orbit. The panel needs to know
%  how many satellites could fit in geosynchronous orbit without receivers on
%  the ground picking up multiple signals. Assume that all satellites broadcast
%  at \SI{12}{\giga\hertz}. and that the receiver dishes are
%  \SI{45}{\centi\metre} in diameter. Begin by calculating the angular size of
%  the beam associated with such a receiver dish, defined as the width of the
%  central diffraction maximum. Use your result to find the number of satellites
%  allowed.
%  \vspace{\stretch{2}}
  
  \question An oil film with refractive index of $n=1.25$ floats on water. The
  film thickness varies from \SI{.80}{\micro\metre} to \SI{2.1}{\micro\metre}.
  If \SI{630}{\nano\metre} red light is incident normally on the film, how many
  locations will it undergo enhanced reflection?
  \vspace{\stretch1}

  \question Microwaves operate at a frequency of \SI{2.40}{\giga\hertz}. Find
  the minimum thickness for a plastic tray with a refractive index of 1.45 that
  will cause enhanced reflection of microwaves incident normal to the tray.
  \vspace{\stretch1}
  \newpage
  
  \question White light shines on a \SI{250}{\nano\metre} thick layer of diamond
  ($n=2.42$). What wavelength of \emph{visible light} is most strongly
  reflected?
  \vspace{\stretch1}
  
  \question As a soap bubble ($n=1.33$) evaporates and thins, the reflected
  colours gradually disappear.
  \begin{parts}
    \part What is the thickness as the last vestige of colour vanishes?
    \part What is the last colour seen?
  \end{parts}
  \vspace{\stretch1}

  \question A camera lens is made of glass with an index of refraction of 1.60.
  The lens is coated with magnesium fluoride film ($n=1.38$) to enhance its
  light transmission. This film is to produce zero reflection for light of
  wavelength \SI{540}{\nano\metre}. Treat the lens surface as a flat plane and
  the film as a uniformly thick flat film.
  \begin{parts}
    \part How thick must the film be to accomplish its objective in the first
    order?
    \part Would there be destructive interference for any other visible
    wavelengths?
  \end{parts}
  \vspace{\stretch1}
    
%  \question News media often conduct live interviews from locations halfway around the
%  world. There is obviously a time-lag between when a signal is sent and when
%  it is received.
%  \begin{parts}
%  \part Calculate how long the time-lag should be for a signal sent from
%    locations on Earth separated by \SI{2.00e4}{\kilo\metre}.
%  \part Suggest reasons why the actual time-lag differs from the value in (a).
%  \end{parts}
%  \vspace{\stretch1}
\end{questions}
\end{document}
