%{Curvilinear vs.\ Rectilinear Motion}
%  Kinematic quantities for rectilinear (translational) vs.\ curvilinear
%  (circular) motion are related:
%
%  \vspace{-.3in}{\large
%    \begin{align*}
%      x &\quad\rightarrow\quad \theta \\
%      v &\quad\rightarrow\quad \omega \\
%      a &\quad\rightarrow\quad \alpha
%    \end{align*}
%  }
%
%  Dynamics:
%  
%  \vspace{-.3in}{\large
%    \begin{align*}
%      m &\quad\rightarrow\quad I\\
%      F &\quad\rightarrow\quad\tau\\
%      p=mv &\quad\rightarrow\quad L=I\omega
%    \end{align*}
%  }
%\end{frame}
%
%
%
%{Laws of Motion}
%
%  The laws of motion are also related between translational and rotational
%  motion:
%  
%  \vspace{-.2in}{\large
%    \begin{align*}
%      F_\text{net}=\frac{\Delta p}{\Delta t} &\quad\rightarrow\quad
%      \tau_\text{net}=\frac{\Delta L}{\Delta t} \\
%      F_\text{net}=ma &\quad\rightarrow\quad
%      \tau_\text{net}=I\alpha
%    \end{align*}
%  }
%
%
%
%
%{Solving Rotational Problems}
%  When solving for rotational problems like the ones described in the previous
%  sections:
%  \begin{itemize}
%  \item Draw a free-body diagram to account for all forces
%  \item The direction of friction force is not always obvious
%  \item The magnitude of any static friction force cannot be assumed to be at
%    maximum.
%  \item If the object is to change its rotational state, there must be a net
%    torque causing it.
%  \end{itemize}
%
%
%
%
%{Solving Rotational Problems}
%  Once the free-body diagram is complete, the forces should break down into
%  their \emph{forces} into $\hat x$, $\hat y$ and $\hat z$ components. If the
%  axes are defined properly, only one direction should have acceleration
%  (usually $\hat x$), i.e.:
%  
%  \eq{-.1in}{
%    \sum F_x=ma \quad\quad \sum F_y=0
%  }
%
%  For rotational motion:
%    
%  \eq{-.1in}{
%    \sum\tau=I\alpha
%  }
%
%
%
%
\section{Pure Rolling Problems}
%
%{Motion with Both Translation and Rotation}
Often motion of an object includes both translational and rotational motion.
%  \begin{center}
%    \begin{tikzpicture}
%      \fill circle(1);
%      \fill[black!2] circle (.8);
%      \fill circle(.1);
%      \draw[thick](-3,-1)--(3,-1);
%      \draw[vectors,red] (1.25,0)--(2.5,0) node[right]{$v$};
%      \draw[vectors,red,rotate=-40] (0,1.1) arc (90:0:1.1) node[right]{$\omega$};
%    \end{tikzpicture}
%  \end{center}
%  For example, a tire that is rolling on the road without slipping has both
%  translational velocity $\bm v$ and angular velocity $\omega$.
%
%
%
%
%{No-Slip Rolling}

\begin{figure}[ht]
  \centering
  \begin{subfigure}{.2\linewidth}
    \centering
    \begin{tikzpicture}
      \fill circle (1);
      \fill[white] circle (.8);
      \fill circle (.1);
        
      \foreach \y in {-1,0,1}{
        \fill[red] (0,\y) circle (.05);
        \draw[vectors,red] (0,\y)--(1.5,\y) node[right]{$v$};
      }
    \end{tikzpicture}
    \caption{Translation}
  \end{subfigure}
  {\Huge$+$}
  \begin{subfigure}{.28\linewidth}
    \centering
    \begin{tikzpicture}
      \fill circle (1);
      \fill[white] circle (.8);
      \fill circle (.1);
      \draw[vectors,blue,rotate=-40] (0,1.1) arc (90:0:1.1)
      node[right]{$\omega$};
      \draw[vectors,blue] (0,1)--+(1.5,0) node[right]{$\omega R$};
      \draw[vectors,blue] (0,-1)--+(-1.5,0) node[left]{$\omega R$};
      \foreach \y in {-1,1} \fill[blue] (0,\y) circle (.05);
    \end{tikzpicture}
    \caption{Rotation}
  \end{subfigure}
  {\Huge$=$}
  \begin{subfigure}{.38\textwidth}
    \centering
    \begin{tikzpicture}
      \fill circle (1);
      \fill[white] circle (.8);
      \fill circle (.1);
       
      \foreach \y in {0,1} \fill[violet] (0,\y) circle (.05);
      \draw[vectors,violet] (0,1)--(3,1) node[right]{$v=2\omega R$};
      \draw[vectors,violet] (0,0)--(1.5,0) node[right]{$v=\omega R$};
      \fill[violet] (0,-1) circle (.05) node[right]{$v=0$};
    \end{tikzpicture}
    \caption{Total motion}
  \end{subfigure}
\end{figure}
%  At the contact point, if there is no slipping, the speed is zero.
%
%
%
%
%\subsection{Pure Rolling Problems}
%  \textbf{Example:} Consider a smooth solid sphere\footnote{Any object that can
%    roll with do! All that differs is the moment of inertia.} rolls along a
%  smooth surface without slipping
%  

%    \centering
%    \begin{tikzpicture}
%      \shade[ball color=gray!10] circle (1);
%      \draw[thick] (-2,-1)--(2,-1);
%      \draw[vectors] (1.2,0)--(2.2,0) node[above]{$v$};
%      \draw[vectors] (0,1.2) arc (90:60:1.2) node[midway,above]{$\omega$};
%      \draw[axes] (1,-2)--(1.5,-2) node[right]{$x$};
%      \draw[axes] (1,-2)--(1,-1.5) node[above]{$y$};
%      \fill circle(.05);
%
%      \draw[vectors,blue] (0,0)--(0,-1.5) node[below]{$\bm F_g$};
%      \draw[vectors,red] (-.03,-1)--(-.03,.5) node[above]{$\bm F_n$};
%    \end{tikzpicture}
%
%    \begin{itemize}
%    \item Assumptions:
%      \begin{itemize}
%      \item Both the sphere and the surface are both perfectly rigid (they
%        do not deform)
%      \item The sphere and the surface are both perfectly smooth without defects
%        even at the microscopic level
%      \end{itemize}
%    \item There are only two forces acting on the sphere:
%      \begin{itemize}
%      \item Gravitational force $\bm F_g$
%      \item Normal force $\bm F_n$
%      \end{itemize}
%    \item There is no friction
%    \end{itemize}
%  
%  \vspace{.2in}




The free-body diagram is simple enough that we can see that:
\begin{itemize}
\item There is no net force, therefore the translational state ($\bm v$) of
  the sphere is constant
%
%  \eq{-.1in}{
%    \sum\bm F=\bm 0\quad\quad \bm v=\text{constant}
%  }
\item Neither gravity or normal force generate a torque about the center of
  mass (CM), therefore there is no net torque, and the rotational state
  $\omega$ is constant:
%  
%  \eq{-.1in}{
%    \sum\tau=0 \quad\quad \omega=\text{constant}
%  }
\end{itemize}
In \emph{theory}, this sphere will roll along with angular speed $\omega$
and speed $v=\omega R$ forever.

\begin{figure}[ht]
  \centering
  \begin{tikzpicture}
    \shade[ball color=gray!10] circle (1);
    \draw[thick] (-2,-1)--(2,-1);
    \draw[vectors] (1.5,0)--(2.5,0) node[right]{$v$};
    \draw[vectors] (0,1.2) arc (90:60:1.2) node[midway,above]{$\omega$};
    \fill circle (.05) node[right]{CM};
    \draw[vectors,blue] (0,0)--(0,-1.5) node[below]{$\bm F_g$};
    \draw[vectors,red] (-.03,-1)--(-.03,.5) node[above]{$\bm F_n$};
  \end{tikzpicture}
\end{figure}




%{Reality: Rolling Resistance}
%  In reality, the rolling sphere will slow down and eventually come to a stop,
%  because \emph{nothing is perfectly rigid}: both the sphere and the surface
%  deform when they make contact
%  
%    \column{.7\textwidth}
%    \begin{itemize}
%    \item Example: a car's tires flatten when they make contact with the ground
%    \item The normal force is larger in magnitude on the front side than on the
%      other
%    \item $N$ exerts both a horizontal force to slow down the sphere, as well
%      as a torque to slow down its rotation
%    \item The normal force does not point toward the centre of mass because of the
%      deformation.
%    \item There may also be a frictional force that is generated because of the
%      deformation
%    \end{itemize}
%
%    \column{.3\textwidth}
%    \vspace{.3in}
%    \pic1{OAGZy}



\section{Rolling on an Inclined Surface}

Consider the same sphere of radius $R$, now rolling down a ramp of angle
$\phi$ without slippage.

\begin{figure}[ht]
  \centering
  \begin{tikzpicture}[scale=1.2,rotate=-30]
  \shade[ball color=gray!20] circle (1);
  \draw[thick] (-2,-1)--(4,-1);
  \draw[->,rotate=30] (0,0)--(-1,0) node[left]{$R$};
  \draw[vectors] (0,1.2) arc(90:60:1.2) node[right]{$\omega$};
  \draw[axes] (2,0)--(3,0) node[right]{$x$};
  \draw[axes] (2,0)--(2,1) node[above]{$y$};
  \fill circle (.05);
  \draw[vectors,blue,rotate=30] (0,0)--(0,-1.5) node[below]{$\bm F_g$};
  \draw[vectors,red] (0,-1)--(0,.5) node[above]{$\bm F_n$};
  \draw[vectors,orange] (0,-1)--(-1,-1) node[left]{$\bm f_s$};
  \draw[thick,rotate around={30:(4,-1)}] (4,-1)--(2,-1)
  node[pos=.6,above]{$\phi$};
\end{tikzpicture}

\end{figure}

Three forces act on the sphere as it rolls down the ramp
\begin{itemize}
\item Gravitational force ($\bm F_g=m\bm g$) acts at the centre of mass
\item Normal force ($\bm F_n$) acts at the point of contact
\item Static friction ($\bm f_s$) acts at the point of contact
\end{itemize}
Only static friction generates a torque about the centre of mass, in the clockwise
direction
\begin{itemize}
\item Without $f_s$, there would have been nothing that causes it to rotate.
\end{itemize}
%    \begin{tikzpicture}[scale=1.2,rotate=-30]
  \shade[ball color=gray!20] circle (1);
  \draw[thick] (-2,-1)--(4,-1);
  \draw[->,rotate=30] (0,0)--(-1,0) node[left]{$R$};
  \draw[vectors] (0,1.2) arc(90:60:1.2) node[right]{$\omega$};
  \draw[axes] (2,0)--(3,0) node[right]{$x$};
  \draw[axes] (2,0)--(2,1) node[above]{$y$};
  \fill circle (.05);
  \draw[vectors,blue,rotate=30] (0,0)--(0,-1.5) node[below]{$\bm F_g$};
  \draw[vectors,red] (0,-1)--(0,.5) node[above]{$\bm F_n$};
  \draw[vectors,orange] (0,-1)--(-1,-1) node[left]{$\bm f_s$};
  \draw[thick,rotate around={30:(4,-1)}] (4,-1)--(2,-1)
  node[pos=.6,above]{$\phi$};
\end{tikzpicture}


To solve this problem, there are three dynamics equations:
\begin{align}
  \sum F_x&=mg\sin\theta-f_s=ma\\
  \sum F_y&=F_n-mg\cos\theta=0\\
  \sum\tau &=Rf_s=I\alpha
\end{align}
At this point, the magnitude of static friction $f_s$ is \emph{not} known. The
coefficient of static friction ($\mu_s$) only tells us the \emph{maximum}
static friction, not the \emph{actual} friction. (We will instead use it to
check if the answer makes sense.)
  




%{Rolling on an Inclined Surface}
%  
%    \begin{tikzpicture}[scale=1.2,rotate=-30]
  \shade[ball color=gray!20] circle (1);
  \draw[thick] (-2,-1)--(4,-1);
  \draw[->,rotate=30] (0,0)--(-1,0) node[left]{$R$};
  \draw[vectors] (0,1.2) arc(90:60:1.2) node[right]{$\omega$};
  \draw[axes] (2,0)--(3,0) node[right]{$x$};
  \draw[axes] (2,0)--(2,1) node[above]{$y$};
  \fill circle (.05);
  \draw[vectors,blue,rotate=30] (0,0)--(0,-1.5) node[below]{$\bm F_g$};
  \draw[vectors,red] (0,-1)--(0,.5) node[above]{$\bm F_n$};
  \draw[vectors,orange] (0,-1)--(-1,-1) node[left]{$\bm f_s$};
  \draw[thick,rotate around={30:(4,-1)}] (4,-1)--(2,-1)
  node[pos=.6,above]{$\phi$};
\end{tikzpicture}

%    
%    For non-slip case, angular and translational acceleration are related using
%    relative motion:
%
%    \eq{-.2in}{ a=\alpha R}
%    
%    \vspace{-.2in}Solving for static friction:
%
%    \eq{-.1in}{
%      f_s=\frac{I\alpha}R=
%      \frac25mR^2\cdotp\frac aR\cdotp\frac1R=\frac25ma
%    }
%
%    It is substituted into the force equation in the $\hat x$ direction to
%    solve for the acceleration of the centre of mass along the ramp:
%
%    \eq{-.1in}{
%      mg\sin\theta-\frac25 ma=ma
%    }


%    \begin{tikzpicture}[scale=1.2,rotate=-30]
  \shade[ball color=gray!20] circle (1);
  \draw[thick] (-2,-1)--(4,-1);
  \draw[->,rotate=30] (0,0)--(-1,0) node[left]{$R$};
  \draw[vectors] (0,1.2) arc(90:60:1.2) node[right]{$\omega$};
  \draw[axes] (2,0)--(3,0) node[right]{$x$};
  \draw[axes] (2,0)--(2,1) node[above]{$y$};
  \fill circle (.05);
  \draw[vectors,blue,rotate=30] (0,0)--(0,-1.5) node[below]{$\bm F_g$};
  \draw[vectors,red] (0,-1)--(0,.5) node[above]{$\bm F_n$};
  \draw[vectors,orange] (0,-1)--(-1,-1) node[left]{$\bm f_s$};
  \draw[thick,rotate around={30:(4,-1)}] (4,-1)--(2,-1)
  node[pos=.6,above]{$\phi$};
\end{tikzpicture}

%    
%    The acceleration of the centre of mass is therefore:
%
%    \eq{-.1in}{
%      a=\frac57 g\sin\theta
%    }
%
%    \vspace{-.1in}For comparison, an object \emph{sliding} without friction
%    would have an acceleration of $a=g\sin\theta$ instead.
%    
%    \vspace{.2in}If the sphere starts from rest, the speed of the sphere when
%    it reaches the bottom of the ramp---a distance $d$ away---would be:
%
%    \eq{-.1in}{
%      v=\sqrt{2ad}=\sqrt{\frac{10}7gd\sin\theta}
%    }




\section{Solving Rotational Problems with Conservation of Energy}

The example with the rolling sphere down a ramp can be solved using the law of
conservation of energy as well. As the sphere rolls down the ramp, the work
done by the three forces acting on it.  
%    \begin{tikzpicture}[scale=1.2,rotate=-30]
  \shade[ball color=gray!20] circle (1);
  \draw[thick] (-2,-1)--(4,-1);
  \draw[->,rotate=30] (0,0)--(-1,0) node[left]{$R$};
  \draw[vectors] (0,1.2) arc(90:60:1.2) node[right]{$\omega$};
  \draw[axes] (2,0)--(3,0) node[right]{$x$};
  \draw[axes] (2,0)--(2,1) node[above]{$y$};
  \fill circle (.05);
  \draw[vectors,blue,rotate=30] (0,0)--(0,-1.5) node[below]{$\bm F_g$};
  \draw[vectors,red] (0,-1)--(0,.5) node[above]{$\bm F_n$};
  \draw[vectors,orange] (0,-1)--(-1,-1) node[left]{$\bm f_s$};
  \draw[thick,rotate around={30:(4,-1)}] (4,-1)--(2,-1)
  node[pos=.6,above]{$\phi$};
\end{tikzpicture}

\begin{itemize}
\item Gravity does \emph{positive translational work} on the sphere
\item Normal force does not do any translational work nor rotational work
\item Static friction ($\bm f_s$) does \emph{negative translation work}
  because the direction of $\bm f_s$ is opposite to the motion of the centre of mass
\item But static friction ($\bm f_s$) also does \emph{positive rotational
work}, because it generates a torque
\end{itemize}
%  
%
%    \begin{tikzpicture}[scale=1.2,rotate=-30]
  \shade[ball color=gray!20] circle (1);
  \draw[thick] (-2,-1)--(4,-1);
  \draw[->,rotate=30] (0,0)--(-1,0) node[left]{$R$};
  \draw[vectors] (0,1.2) arc(90:60:1.2) node[right]{$\omega$};
  \draw[axes] (2,0)--(3,0) node[right]{$x$};
  \draw[axes] (2,0)--(2,1) node[above]{$y$};
  \fill circle (.05);
  \draw[vectors,blue,rotate=30] (0,0)--(0,-1.5) node[below]{$\bm F_g$};
  \draw[vectors,red] (0,-1)--(0,.5) node[above]{$\bm F_n$};
  \draw[vectors,orange] (0,-1)--(-1,-1) node[left]{$\bm f_s$};
  \draw[thick,rotate around={30:(4,-1)}] (4,-1)--(2,-1)
  node[pos=.6,above]{$\phi$};
\end{tikzpicture}


The work done by static friction:
\begin{itemize}
\item Transforms energy from translational kinetic energy $K_t$ of the sphere
  into rotational kinetic energy $K_r$ of the sphere
\item is therefore \emph{internal} to the system
\end{itemize}
The total mechanical energy of the system is conserved when the sphere rolls
down the ramp:
\begin{equation}
  \boxed{
    K_t + K_r + U_g = \text{constant}
  }
\end{equation}
Using the conservation of energy, we can also find the same speed of the sphere
after rolling a distance $d$, as long as there is no slipping.
