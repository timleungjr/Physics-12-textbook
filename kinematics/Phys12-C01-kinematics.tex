%\documentclass[12pt,compress,aspectratio=169]{beamer}
%\input{../mybeamer}

\chapter{One-Dimensional Kinematics}
\label{chapter:kinematics}

%\section{Kinematics}
%  \textbf{Kinematics} describes the motion of points, bodies (objects), and
%  systems of bodies (groups of objects). It is the mathematical relationship
%  between





\section{Frame of Reference}
A \textbf{frame of reference}\footnote{Or \textbf{reference frame}, or just
\textbf{frame}} is a \emph{coordinate system} in which physical
measurements are made. In classical mechanics, a frame of reference is
\begin{itemize}
\item the $x$- $y$- and $z$-axes in the Cartesian coordinate system
\end{itemize}
Later, when we study relativity, the frame of reference must also include
\begin{itemize}
\item a time axis
\end{itemize}




%\section{Frame of Reference}
Think of the frame of reference as a ``hypothetical mobile laboratory'' an
observer uses to make any physical measurements (e.g.\ mass, lengths, time).
At a minimum,
it must include:
\begin{itemize}
\item A set of rulers (i.e.\ a coordinate system) to measure lengths
\item A clock to measure the passage of time
  %\item A scale to compare forces
  %\item A balance to measure masses
\end{itemize}
We assume that this ``hypothetical laboratory'' is \emph{perfect}:
\begin{itemize}
\item The hypothetical instruments are not subjected to numerical errors
\item There is an instrument for whatever you want to measure
\item What matters is the \emph{motion} of the frame (at rest, uniform
  motion, accelerating), and how that motion affects the measurements
\end{itemize}




\subsection{Inertial Frame of Reference}
An \textbf{inertial frame of reference}\footnote{Also known as a
\textbf{rest frame}} is one that moves in uniform motion (constant
velocity, no acceleration)
\begin{itemize}
\item The frame of reference is not subjected to any net force
\item In all inertial frames of reference, the laws of motion are valid
\item Since all laws of motion are valid in all inertial frames of reference,
  \emph{any} inertial frame can be considered to be at rest (stationary)
\end{itemize}




The priniciple of relativity states that:
\begin{definition}
  \textbf{The Principle of Relativity:} The laws of motion must be
  obeyed in all inertial frames of reference.
\end{definition}
\begin{itemize}
\item All inertial frames are equally valid when determining the laws of
  motion
\item All inertial frames can be considered at rest
\item All motion is relative: there is no \emph{absolute motion} or
    \emph{absolute rest}
\end{itemize}
%
%
%
%
%\section{Inertial Frame of Reference}
Observer A is moving with a constant velocity relative to observer B
\begin{itemize}
\item A observes that the ball that he is throwing only has vertical motion
\item B observes that the ball is moving in a parabolic curve
\item A \& B agree on the \emph{equations} that govern the motion
\end{itemize}
\begin{center}
  \pic{.65}{kinematics/graphics/57}
\end{center}
Observer A makes the same observation regardless of whether he is moving
uniformly relative to the ground or not
\begin{itemize}
\item Valid for A to think that he is at rest, but B is moving
\item Also valid for B to think that he is at rest, but A is moving
\item Both A and B are inertial frames of reference
\end{itemize}
%  \begin{center}
%    \vspace{-.15in}
%    \pic{.65}{graphics/57}
%  \end{center}
%
%
%
%
\subsection{Non-inertial Frame of Reference}
A \textbf{non-inertial frame of reference} is one that is undergoing
acceleration (i.e.\ non-constant velocity)
\begin{itemize}
\item Require a \textbf{fictitious force}\footnote{Also known as a
\textbf{pseudo force}} in the FBD to account for the observations
  \begin{itemize}
  \item Hypothetical force
  \item Does not exist in inertial frame of reference
  \end{itemize}
\item Example: A car that is speeding up, slowing down, or turning
\end{itemize}
\begin{center}
  \pic{.4}{kinematics/graphics/man-in-accel-car}
\end{center}




%\section{Example: Reference Frame}
%  \begin{columns}
%    \column{.4\textwidth}
%    \pic1{graphics/capsule2}
%    
%    \column{.6\textwidth}
%    \textbf{Example:} Passengers in a high-speed elevator feel as though they
%    are being pressed heavily against the floor when the elevator starts moving
%    up. After the elevator reaches its maximum speed, the feeling disappears.
%  \end{columns}
%
%
%
%
%\section{Example: Reference Frame}
%  \begin{enumerate}
%  \item When do the elevator and passengers form 
%    \begin{itemize}
%    \item an inertial frame of reference?
%    \item a non-inertial frame of reference?
%    \end{itemize}
%  \item Before the elevator starts moving,
%    \begin{itemize}
%    \item what forces are acting on the passengers?
%    \item how large is the external (unbalanced) force?
%    \end{itemize}
%  \item Is a person standing outside the elevator in an inertial or
%    non-inertial frame of reference?
%  \end{enumerate}
%
%
%
%
%\section{Frame of Reference}
%  \textbf{Example:} Given the definition of an inertial frame of reference,
%  is Earth an inertial frame of reference?
%
%  \vspace{.3in}\uncover<2>{
%    \textbf{Answer:} No! Earth is rotating on its axis, and also orbiting in an
%    elliptical path around the Sun. However, we can adjust the value of $g$
%    slightly to account for the acceleration of Earth.
%  }
%
%
%
%
%\section{Motion Quantities}
%
%  Kinematics does not deal with the cause of motion.



\section{Kinematic Quantities}

\begin{itemize}
\item Position
\item Displacement
\item Distance
\item Velocity
\item Speed
\item Acceleration
\end{itemize}


\subsection{Position}

\textbf{Position} ($\bm r$) is the location of an object inside a predefined
coordinate system. It is a vector measured from the origin to the object. If
the object moves, $\bm r$ is a continuous function of time $t$:
\begin{important-equation}
  \bm r=\bm r(t)
\end{important-equation}
At this moment, it is not necessary to know what kind of function $\bm r(t)$
is; the motion of the object will be determined by the forces that act on the
object, and therefore what acceleration it has. All that matters is that
$\bm r(t)$ evolves with time, the object can be at only one position at any
time (hence it is a function), and that there are no gaps in the where the
object is (hence it is a continuous function). The SI unit for position is
\emph{metre} (\si\metre).

Position is a \emph{vector}, and therefore require a magnitude and direction.
For motion in a one-dimensional coordinate system, the position of the object
is a coordinate along the number line. The vector is an arrow drawn from
the origin to object itself, as shown in Figure~\ref{fig:1d-position}. In
the figure, the position of the object is positive.

\begin{figure}[ht]
  \centering
  \begin{tikzpicture}
    \draw[axes] (-3,0)--(3,0) node[pos=0,left]{$-$} node[right]{$+$};
    %\foreach \x in {-5,...,5} \draw (\x,.1)--(\x,-.1);
    \draw[thick] (0,.1)--(0,-.1) node[below]{$O$};
    \fill[red] (2.5,0) circle (.05);% node[below]{$A$};
    \draw[vectors,red] (0,0)--(2.5,0) node[midway,above]{$\bm r$};
  \end{tikzpicture}
  \caption{Position in a one-dimensional coordinate system}
  \label{fig:1d-position}
\end{figure}

In a two-dimesional coordinate system (e.g.\ $xy$-plane),
%\textbf{Position in 2D Coordinate System:} For two-dimensional motion,
there are several ways to describe an object's position. One way is to use the
$x$ and $y$ coordinates. The positions of the object at $P$ and $Q$ are:
\begin{align*}
  \bm r_P & =3\hat x + 2\hat y\\
  \bm r_Q & =-4\hat x + 3\hat y
\end{align*}
or the length of the straight line from the origin to
the position, and the angle it makes with the $x$ axis.
\begin{figure}[ht]
  \centering
  \begin{tikzpicture}
    \draw[help lines] (-3.3,-3.3) grid (3.3,3.3);
    \fill[red] (3,2) circle (.05) node[above]{$P$};
    \draw[vectors,red] (0,0)--(3,2) node[midway,above]{$\bm r$};
    \draw[axes] (-3.5,0)--(3.5,0) node[right]{$x$};
    \draw[axes] (0,-3.5)--(0,3.5) node[above]{$y$};
    \draw[axes] (1.5,0) arc (0:atan(2/3):1.5) node[midway,right]{$\theta$};
    \node[below left] at (0,0) {$O$};
  \end{tikzpicture}
  \caption{Position in a two-dimensional coordinate system}
\end{figure}



\subsection{Displacement}
\textbf{Displacement} ($\Delta\bm r$) is the \emph{change in position} when
an object moves through the coordinate system. Mathematically, displacement is
defined as the \emph{difference} between the initial position
$\bm r_1=\bm r(t_1)$ when motion begins, and the current position
$\bm r(t)$ of the object. Therefore, as the object moves, $\Delta\bm r$
is also a continuous function of time, i.e.:
%\begin{equation*}
%  \Delta\bm r=  \Delta\bm r(t)
%\end{equation*}
\begin{important-equation}
  \boxed{
    \Delta\bm r(t)=\bm r(t)-\bm r_1
  }
\end{important-equation}
Since displacement is the difference between two positions, naturally, the SI
unit for displacement is also \emph{metre} (\si\metre). Graphically,
displacement is drawn as a vector pointing from the initial position $\bm r_1$
towards the current/final position $\bm r$. A two-dimensional example is shown
in Figure~\ref{fig:displacement-and-path}. An object moves along a curved path
($\mathcal C$) from position 1 ($\bm r_1$) to position 5 ($\bm r_5$). The
displacement vectors as it passes through to position 2, 3, 4, and 5 are shown
in blue as $\bm r_1\to\bm r_4$.
\begin{figure}[ht]
  \centering
  \begin{tikzpicture}
    \draw[axes] (0,0)--(7,0) node[right]{$x$};
    \draw[axes] (0,0)--(0,6.5) node[above]{$y$};
    \fill (4,1) circle (.08) node[below]{1};
    \fill (2,6) circle (.08) node[left]{5};
    \draw[vectors,red] (0,0)--(4,1) node[midway,above]{$\bm r_1$};
    \draw[vectors,red] (0,0)--(2,6) node[midway,left] {$\bm r_5$};
    \begin{scope}[rotate around={33:(4,1)}]
      \draw[vector,blue] (4,1)--+(2,0) node[midway,below]{$\Delta\bm r_1$};
      \fill (6,1) circle (.08) node[right]{2};
    \end{scope}
    \begin{scope}[rotate around={60:(4,1)}]
      \draw[vector,blue] (4,1)--+(3.35,0) node[midway,right]{$\Delta\bm r_2$};
      \fill (7.35,1) circle (.08) node[above]{3};
    \end{scope}
    \begin{scope}[rotate around={85:(4,1)}]
      \draw[vector,blue] (4,1)--+(2.8,0) node[pos=.7,left=0]{$\Delta\bm r_3$};
      \fill (6.8,1) circle (.08) node[above]{4};
    \end{scope}
    \draw[vectors,blue] (4,1)--(2,6) node[midway,left]{$\Delta\bm r_4$};
    \draw[very thick,dashed,->] (4,1) ..controls (7,2) and (6,4.5).. (5,4)
    ..controls (3,3) and (4,6).. (2,6) node[midway,right]{$\mathcal C$};
  \end{tikzpicture}
  \caption{The displacement of an object evolves with time as it moves.}
  \label{fig:displacement-and-path}
\end{figure}

\textbf{Explain why this is a subtraction} The graphical representation of
displacement does not necessarily tell us why


\subsection{Distance}

\textbf{Distance} ($s$) is a quantity that is \emph{similar} to
displacement. It is the length of the path $\mathcal C$ taken as an object
moves from initial position ($\bm r_1$) to its current/final position
($\bm r(t)$). Unlike displacement, distance is a \emph{length}, and therefore
it is \emph{scalar} quantity that does not require a direction. For the same
reason, distance is non-negative (i.e.\ $s\geq 0$). Since the length changes
with time, distance is also a continuous function of time:
\begin{equation}
  \boxed{s=s(t)}
\end{equation}
If motion begins at $t=0$, then the initial distance is zero, i.e.\ $s(0)=0$.
As the object moves, $s$ is always increasing. It should be noted that although
the magnitude of the displacement vector $|\Delta\bm r|$ is also a scalar, it
is \emph{not} necessarily the same as distance. In classical physics,
$s\geq |\Delta\bm r|$.

%\item Depends on \emph{how} the object travels from $\bm r_1$ to
%  $\bm r_2$


%\begin{figure}
%  \centering
%  \begin{tikzpicture}[scale=.5]
%    \draw[axes] (0,0)--(6,0) node[right]{$x$};
%    \draw[axes] (0,0)--(0,8) node[above]{$y$};
%    \draw[vectors,red] (0,0)--(4,1) node[midway,above]{$\bm r_1$};
%    \draw[vectors,red] (0,0)--(2,6) node[midway,left] {$\bm r_2$};
%    \draw[vectors,blue] (4,1)--(2,6) node[midway,right]{$\Delta\bm r$};
%    \draw[very thick,dash dot] (4,1)..controls (6,5) and (5,7)..(2,6)
%    node[midway,right]{$s$};
%  \end{tikzpicture}
%\end{figure}




\subsection{Velocity}

\textbf{Average velocity} ($\bm v_\text{avg}$) is how quickly your position
changes \emph{over a finite time interval}. It is a vector quantity with an
SI unit of \textbf{metres per second} (\si{\metre\per\second}). The direction
of $\bm v_\text{avg}$ is the same as displacement $\Delta\bm d$, and it is
also a function of time:
\begin{important-equation}
  \boxed{
    \bm v_\text{avg}(t)
    =\frac{\Delta\bm r(t)}{\Delta t}
    =\frac{\bm r(t)-\bm r_1}{t-t_1}
  }
\end{important-equation}
where $\bm r_1=\bm r(t_1)$ is the initial position at initial time $t_1$

In contrast, \textbf{instantaneous velocity} ($\bm v$) is how quickly
your displacement is changing \emph{at a specific instance in time}
\begin{itemize}
\item Obtained by letting the time interval \emph{infinitesimally} small,
  i.e.\ $\Delta t\rightarrow 0$
\item Also called the \emph{rate of change in displacement}
\item Calculating instantaneous velocity may require calculus\footnote{For
those of you who know a bit of calculus, the definition of instantenous
velocity is:
\begin{displaymath}
  \bm v(t)=\frac{d\bm r}{dt}
\end{displaymath}}  
\end{itemize}




\subsection{Speed}

\textbf{Average speed} ($v$) is similar to average velocity, but instead of
using displacement, it is the distance ($s$) travelled over a \emph{finite}
time interval. Speed is a \emph{scalar}:
  
\begin{important-equation}
  \boxed{
    v_\text{avg}(t)
    =\frac{s(t)}{\Delta t}\geq 0
  }
\end{important-equation}
Similarly, \textbf{instantaneous speed} ($v$) is how quickly
distance is changing at a \emph{specific instance} in time.
\begin{itemize}
\item Since distance is always positive ($s\ge 0$), both average and
  instantaneous speeds must always be positive
\end{itemize}




\subsection{Acceleration}

\textbf{Average acceleration} ($\bm a_\text{avg}$) is how quickly the
instantaneous velocity vector changes over a \emph{finite} time interval,
with an SI unit of \textbf{metres per second squared}
(\si{\metre\per\second\squared}):
\begin{important-equation}
  \boxed{\bm a_\text{avg}(t)
    =\frac{\Delta\bm v(t)}{\Delta t}
    =\frac{\bm v(t)-\bm v_1}{t-t_1}
  }
\end{important-equation}
where $\bm v_1=\bm v(t_1)$ is the initial velocity $\bm v$ at initial time
$t_1$.

\textbf{Instantaneous acceleration} ($\bm a(t)$) is how quickly the velocity
vector is changing at a \emph{specific instance} in time

A few things to note about acceleration:
\begin{itemize}
\item i.e.\ \emph{the rate of change of instantaneous velocity}
\item A change in a vector ($\Delta\bm v$) can mean a change in magnitude
  and/or direction
\item There can be acceleration without any speeding up or slowing down!
\item Think  about what happens if a car is turning at constant speed
\end{itemize}




%\section{Working with Vectors}
%  Vectors obey the \emph{principle of superposition}, which means that they
%  \emph{add} together. Methods for adding vectors include:
%  \begin{itemize}
%  \item Using \textbf{Pythagorean theorem} (for vectors at right angles to
%    each other)
%  \item Using \textbf{cosine and sine laws}
%  \item Decomposing vectors into \textbf{components}, then reassemble them
%    using Pythagorean theorem
%  \end{itemize}
%  For 1D problems, ($+$) and ($-$) signs are sufficient to indicate direction
%  \begin{itemize}
%  \item Remember to indicate which way is positive though!
%  \end{itemize}
%  \textbf{WARNING:} When adding vectors, you
%  \underline{\textbf{\emph{must not}}} simply add the magnitudes of the vectors!


\section{One-Dimensional Kinematic Equations}

\begin{align}
  \Delta d &=v_1\Delta t + \frac12a\Delta t^2\\
  \Delta d &=v_2\Delta t - \frac12a\Delta t^2\\
  \Delta d &=\frac{v_1+v_2}2 \Delta t\\
  v_2 &= v_1+ a \Delta t\\
  v_2^2 &= v_1^2+ 2a \Delta d
\end{align}

There are five motion quantities of interest:
\begin{center}
  \begin{tabular}{l|c|c}
    \rowcolor{pink}
    \textbf{Quantity} & \textbf{Symbol} & \textbf{SI Unit} \\ \hline
    Displacement & $\Delta d$ & \si{\metre} \\
    Initial (instantaneous) velocity & $v_1$ & \si{\metre\per\second} \\
    Final (instantaneous) velocity   & $v_2$ & \si{\metre\per\second} \\
    Acceleration (constant) & $a$    & \si{\metre\per\second\squared}\\
    Time interval & $\Delta t$ & \si\second
  \end{tabular}
\end{center}
Only valid for \underline{\textbf{constant acceleration}}


%\section{1D Kinematic Equations}
%\section{Relative Motion}
%  \begin{center}
%    \vspace{-.15in}
%    \pic{.65}{graphics/57}
%  \end{center}
%  \begin{itemize}
%  \item Observers (frames of reference) A and B measures different motion of the
%    ball because A and B a moving relative to each other
%  \item The instantaneous velocity of the ball at any time $t$, as measured by
%    A and B, is related by the instantaneous velocities of A and B relative to
%    each other
%  \end{itemize}
%
%
%
%
%  \begin{columns}
%    \column{.35\textwidth}
%    {\large
%      \begin{align*}
%        \Delta d &=v_1\Delta t + \frac12a\Delta t^2\\
%        \Delta d &=v_2\Delta t - \frac12a\Delta t^2\\
%        \Delta d &=\frac{v_1+v_2}2 \Delta t\\
%        v_2 &= v_1+ a \Delta t\\
%        v_2^2 &= v_1^2+ 2a \Delta d
%      \end{align*}
%    }
%    \column{.65\textwidth}
%\begin{itemize}
%\item For 1-object problems, you are usually given 3 of the 5 variables,
%  and you are asked to find a 4th one
%\item For 2-object problems, the motion of the two objects are connected by
%  time interval $\Delta t$ and displacement $\Delta d$
%\item For 2D or 3D problems, each direction should have its own kinematic
%  equations
%\end{itemize}


The one-dimensional kinematic equations are generally derived using calculus.
In fact, with calculus, this is a fairly straightforward exercise. However, in
light of this being an algebra based physics textbook, we  will show how to
derive these equations without calculus.

The equation that is the arguably the simplest to derive is Eq. 1.6, which
came directly from the definition of average acceleration. When acceleration
is constant, $a_\text{avg}=\text{constant}=a$, and assuming that motion starts
at $t=0$, we can simply rearrange the terms:
\begin{equation*}
  a_\text{avg} =\frac{\Delta v}{\Delta t}\quad\longrightarrow\quad
  a =\frac{v_2-v_1}{\Delta t}\quad\longrightarrow\quad
  \boxed{v_2 =v_1+a\Delta t}
\end{equation*}
Now we turn to the definition of average velocity, which we have defined
earlier in Eq. 1.3. When acceleration is constant, average velocity is the
arithmetic average between the initial and final velocities
\begin{equation*}
  v_\text{avg}=\frac{v_1+v_2}2
\end{equation*}
Combining these two equations, and again, assuming that motion begins at $t = 0$, we can simply rearrange the terms:
\begin{equation*}
  v_\text{avg}=\frac{\Delta r}{\Delta t}\quad\longrightarrow\quad
  \frac{v_1+v_2}2=\frac{\Delta r}{\Delta t}\quad\longrightarrow\quad
  \boxed{
    \Delta r=\frac{v_1+v_2}2\Delta t
  }
\end{equation*}
Next, we substitute the expression for $v_2$ in Eq 1.16 into Eq. 1.17, we
derive Eq. 1.3:

Likewise, we can solve for $v_1$ in Eq 1.16 and substitute it into Eq. 1.17, we
derive Eq. 1.4.

The remaining question requires a little bit more thought. When



\subsection{Limitations of the Kinematic Equations}
Kinematic equations \emph{cannot} be used when acceleration is non-uniform
(when non-constant forces act on an object):
\begin{itemize}
\item Aerodynamic forces
  \begin{itemize}
  \item Lift and drag forces
  \item proportional to $v^2$
  \end{itemize}
\item Spring force
  \begin{itemize}
  \item The force that a compressed/stretched spring applies to connected
    objects
  \item Proportional to spring displacement $\bm x$
  \end{itemize}
\item Dampers in springs
  \begin{itemize}
  \item Dampers are used to slow down the vibration of an object
  \item Generally proportional to $v$
  \end{itemize}
\end{itemize}  
%We will discuss more about forces later in this unit.




\section{Basic Motion Graphs}
Motion in a one-dimension coordinate system can be expressed graphically using
\textbf{motion graphs}. The most basic motion graphs are motion quantities as
functions of time:
\begin{itemize}[itemsep=3pt]
\item Position vs.\ time ($r$ vs.\ $t$)
\item Instantaneous velocity vs.\ time ($v$ vs.\ $t$)
\item Instantaneous acceleration vs.\ time ($a$ vs.\ $t$)
\end{itemize}
At the moment, we are interested in:
\begin{itemize}[itemsep=3pt]
\item What the graphs themselves tell us
\item What the slopes of the graphs tell us
\item What the concavity of the graphs tell us
\item What the areas under the graphs tell us
\end{itemize}



\subsection{Position vs.\ Time Graph}
The \emph{most} obvious choice for expressing 1D motion of an object
graphically is by plotting its position as a function of time ($r(t)$). An
example is shown in Fig.~\ref{fig:pos-time-graph}.
\begin{figure}[ht]
  \centering
  \begin{tikzpicture}[scale=1.5]
    \draw[axes] (0,0)--(3.5,0) node[right]{$t$};
    \draw[axes] (0,-.7)--(0,2.5) node[right]{$r$};
    \draw[functions,smooth,samples=30,domain=0:3]
    plot({\x},{-.2*\x^4+.5*\x^3+.4*\x^2-.5});
    
    \draw[gray] (1.5,0)--(1.5,1.1) node[pos=0,below,black]{$t_0$}
    --(0,1.1) node[left,black]{$r(t_0)$};
    \fill[red] (1.5,1.1) circle (.05);
  \end{tikzpicture}
  \caption{An example position vs.\ time graph}
  \label{fig:pos-time-graph}
\end{figure}   
In a position vs.\ time graph, the horizontal ($x$) axis (independent variable)
is time $t$, while the the vertical ($y$) axis (dependent variable) is the
position $r$ measured from origin.


\subsubsection*{Slope of Secant: Average Velocity}
Average velocity $v_\text{avg}$ of an object's motion is the \emph{slope} of the
secant line in the position vs.\ time graph. The slope of a line is defined as
\emph{rise over run}. For a position vs.\ time graph, the rise is the
displacement ($\Delta r$) while the run is the time interval ($\Delta t$). Then,
\begin{equation*}
  \text{slope}=\frac{\text{rise}}{\text{run}}=\frac{\Delta r}{\Delta t}
  =v_\text{avg}
\end{equation*}
and the slope is the average velocity by definition. When caclulating
$v_\text{avg}$, we are only interested at the position at the \emph{beginning}
of the time interval $t_1$ (let's call this initial position $r_1$) and the
position at the \emph{end} of the time interval $t_2$ (let's call the final
position $r_2$). If the slope is positive between $t_1\to t_2$, then the object
has a positive dispacement ($\Delta r>0$) and a positive average velocity
($v_\text{avg}>0$). Conversely, if the slope is negative between $t_1\to t_2$,
then the object has a negative displacement ($\Delta r<0$) and a negative
average velocity ($v_\text{avg}<0$). If the slope is zero, it means that the
object has returned to its initial position, and there is zero displacement
($\Delta r=0$) and zero average velocity ($v_\text{avg}=0$).

\begin{figure}[ht]
  \centering
  \begin{tikzpicture}[scale=1.3]
    \draw[axes] (0,0)--(0,3.3) node[above]{$r$};
    \draw[axes] (0,0)--(3,0) node[right]{$t$};
    \draw[functions,smooth,samples=20,domain=0:2.8]
    plot({\x},{-.2*\x^4+.5*\x^3+.4*\x^2+.4});
    \draw[thick,dash dot](.5,.55)--(2.5,2.9);
    \draw[thick] (.5,.55)--(2.5,.55)node[midway,above]{$\Delta t$}
    --(2.5,2.9) node[midway,right]{$\Delta r$};
    \draw[gray] (.5,.55)--(.5,-.1) node[below,black]{$t_1$};
    \draw[gray] (2.5,.55)--(2.5,-.1) node[below,black]{$t_2$};
    \draw[gray] (.5,.55)--(-.1,.55) node[left,black]{$r_1$};
    \draw[gray] (2.5,2.9)--(-.1,2.9) node[left,black]{$r_2$};
    \fill[red!80!black] (.5,.55) circle (.05);
    \fill[red!80!black] (2.5,2.9) circle (.05);
  \end{tikzpicture}
  \hspace{.1in}
  \begin{tikzpicture}[scale=1.3]
    \draw[axes] (0,0)--(0,3.3) node[above]{$r$};
    \draw[axes] (0,0)--(3,0) node[right]{$t$};
    \draw[functions] (0,.35)--(.5,.55)--(1,2)--(1.8,1.5)--(2.5,2.9);
    \draw[thick,dash dot] (.5,.55)--(2.5,2.9);
    \draw[thick] (.5,.55)--(2.5,.55)node[midway,above]{$\Delta t$}
    --(2.5,2.9) node[midway,right]{$\Delta r$};
    \draw[gray] (.5,.55)--(.5,-.1) node[below,black]{$t_1$};
    \draw[gray] (2.5,.55)--(2.5,-.1) node[below,black]{$t_2$};
    \draw[gray] (.5,.55)--(-.1,.55) node[left,black]{$r_1$};
    \draw[gray] (2.5,2.9)--(-.1,2.9) node[left,black]{$r_2$};
    \fill[red!80!black] (.5,.55) circle (.05);
    \fill[red!80!black] (2.5,2.9) circle (.05);
  \end{tikzpicture}
  \caption{Two motions that have the same average velocity}
  \label{fig:average-velocity}
\end{figure}

%  \vspace{-.1in}Same average velocity in both graphs, but very different
%  motions

%


\subsubsection*{Slope of Tangent: Instantaneous Velocity}

The instantaneous velocity of an object is the \emph{slope of the tangent} to
the curve of the position vs.\ time graph at a specific time. We can
obtain/estimate the slope of the tangent by starting with a secant line, and
then gradually bringing the two points in time together.\footnote{Anyone with a
background in calculus should immediately understand that we are taking the
limit as $\Delta t\to0$, which means that instantaneous velocity is the first
time derivative of position. If you don't have a background in calculus, fear
not, as this will become very obvious the moment you learn what a
``derivative'' is in your calculus class.} As was in the previous case, a
positive slope means that the object is travelling towards the positive
direction with a positive velocity at time $t$ (i.e. $v(t)>0$); a negative
slope means that the object is travelling towards the negative direction with
a negative velocity at $t$ (i.e. $v(t)<0$); a slope of zero means that the
object is momentarily at rest (i.e. $v(t)=0$).

Again, using the position vs. time graph that we have first examined in
Fig.~\ref{fig:pos-time-graph}, now shown in
Fig.~\ref{fig:instant-v-in-pos-time-graph}. We can see that the object is
momentarily stationary at $t=0$ and $t=t_1$; positive velocity between
$0<t<t_1$; and negative velocity from $t>t_1$.

\begin{figure}[ht]
  \centering
  \begin{tikzpicture}[scale=1.3]
    \draw[axes] (0,0)--(3.5,0) node[right]{$t$};
    \draw[axes] (0,-.7)--(0,2.5) node[right]{$d$};
    \draw[functions,smooth,samples=30,domain=0:3]
    plot({\x},{-.2*\x^4+.5*\x^3+.4*\x^2-.5});

    \begin{scope}[thick,cyan]
      \draw (-.4,-.5)--+(.8,0) node[midway,below=-2]{$v=0$};
      \draw (1.9,2.1)--+(.8,0) node[midway,above=-2]{$v=0$};
      \draw (2.3,2.1)--(2.3,0) node[below=-2]{$t_1$};
      \draw (0,-.5)--(0,0);
    \end{scope}
    \fill[cyan] (0,-.5) circle (.07);
    \fill[cyan] (2.3,2.1) circle (.07);

    \fill[pink!35,opacity=.3] (0,-.7) rectangle (2.3,2.3);
    \draw[<-,thick,red] (.75,1) to[out=120,in=0] +(-1,.2)
    node[left,text width=98,draw=red,fill=magenta!10]{\scriptsize
      For $0<t<t_0$, velocity is positive ($v>0$) because the graph has a
      positive slope\par};
    
    \fill[cyan!35,opacity=.3] (2.3,-.7) rectangle (3,2.3);
    \draw[<-,thick,blue] (2.6,1) to[out=90,in=180] +(1.6,.4)
    node[right,text width=86,draw=blue,fill=cyan!30]{\scriptsize
      For $t>t_0$, velocity is negative ($v<0$) because the slope is
      negative\par};
  \end{tikzpicture}
  \caption{Obtaining instantaneous information from a position vs.\ time
    graph}
  \label{fig:instant-v-in-pos-time-graph}
\end{figure}


%\begin{figure}[ht]
%  \centering
%  \begin{tikzpicture}[scale=.75]
%    \draw[axes] (0,0)--(4,0) node[right] {$t$};
%    \draw[axes] (0,0)--(0,4) node[right] {$d$};
%    \draw[functions,smooth,samples=20,domain=.5:3.5]
%    plot({\x},{.25*\x^2+.5});
%    \fill[red!80!black] (2,1.5) circle (2.2pt);
%    \draw[dotted,thick] (2,1.5)--(2,0) node[below] {$t_0$};
%    
%    \draw[smooth,samples=4,domain=.75:3.5,thick,dashed]
%    plot({\x},{\x-.5});
%    \draw[thick] (.75,.25)--(3.5,.25) node[pos=.65,above] {\scriptsize Run};
%    \draw[thick] (3.5,.25)--(3.5,3) node[midway,right] {\scriptsize Rise};    
%  \end{tikzpicture}
%\end{figure}
%
%Like average velocity, the sign of the slope also indicates the direction of
%motion. If position data is obtained experimentally, it may be difficult to
%obtain an accurate value for instantaneous velocity.

%%  What can we learn about instantaneous velocity from this position vs.\ time
%%  graph?


%%
%%
%%
%%



%READY\begin{frame}{Instantaneous Velocity}
%READY  The \textbf{instantaneous velocity} of an object is the
%READY  \emph{slope of the tangent} to the curve of the position vs.\ time graph at a
%READY  specific time.
%READY  \begin{center}
%READY    \vspace{-.15in}
%READY    \begin{tikzpicture}[scale=.75]
%READY      \draw[axes] (0,0)--(4,0) node[right] {$t$};
%READY      \draw[axes] (0,0)--(0,4) node[right] {$d$};
%READY      \draw[functions,smooth,samples=20,domain=.5:3.5]
%READY        plot({\x},{.25*\x^2+.5});
%READY      \fill[red!80!black] (2,1.5) circle (2.2pt);
%READY      \draw[dotted,thick] (2,1.5)--(2,0) node[below] {$t_0$};
%READY      \uncover<2->{
%READY        \draw[smooth,samples=4,domain=.75:3.5,thick,dashed]
%READY        plot({\x},{\x-.5});
%READY        \draw[thick] (.75,.25)--(3.5,.25) node[pos=.65,above] {Run};
%READY        \draw[thick] (3.5,.25)--(3.5,3) node[midway,right] {Rise};
%READY      }
%READY    \end{tikzpicture}
%READY  \end{center}
%READY  \vspace{-.1in}Like average velocity, the sign ($+$/$-$) of the slope
%READY  indicates the direction of motion. If position data is obtained
%READY  experimentally, it may be difficult to obtain an accurate value for
%READY  instantaneous velocity.
%READY
%READY
%READY
%READY
%READY\begin{frame}{Instantaneous Velocity}
%READY  What can we learn about instantaneous velocity from \emph{this} position vs.\
%READY  time graph?
%READY  \begin{center}
%READY    \begin{tikzpicture}
%READY      \draw[axes] (0,0)--(3.5,0) node[right]{$t$};
%READY      \draw[axes] (0,-.7)--(0,2.5) node[right]{$d$};
%READY      \draw[functions,smooth,samples=30,domain=0:3]
%READY      plot(\x,{-.2*\x^4+.5*\x^3+.4*\x^2-.5});
%READY      \uncover<2->{
%READY        \begin{scope}[orange]
%READY          \draw[thick] (-.4,-.5)--+(.8,0) node[midway,below=-2]{$v=0$};
%READY          \draw[thick] (1.9,2.1)--+(.8,0) node[midway,above=-2]{$v=0$};
%READY          \draw (2.3,2.1)--(2.3,0) node[below=-2]{$t_1$};
%READY          \draw (0,-.5)--(0,0);
%READY        \end{scope}
%READY        \fill[orange] (0,-.5) circle (.05);
%READY        \fill[orange] (2.3,2.1) circle (.05);
%READY        \draw[<-,thick,orange] (2.3,-.5) to[out=270,in=200] +(1,-.5)
%READY        node[right,text width=128,draw=orange,fill=orange!10]{At $t=0$ and
%READY          $t=t_1$, velocity is zero ($v=0$) because the graph has zero slope};
%READY
%READY        \fill[pink,opacity=.3] (0,-.7) rectangle (2.3,2.3);
%READY        \draw[<-,thick,red] (.75,1) to[out=120,in=0] +(-1,.2)
%READY        node[left,text width=106,draw=red,fill=magenta!10]{Between $0<t<t_1$,
%READY          velocity is positive ($v>0$) because the graph has a positive slope};
%READY
%READY        \fill[cyan!35,opacity=.3] (2.3,-.7) rectangle (3,2.3);
%READY        \draw[<-,thick,blue] (2.8,1.7) to[out=70,in=150] +(1,0)
%READY        node[right,text width=99,draw=blue,fill=cyan!30]{For $t>t_1$, velocity
%READY          is negative ($v<0$) because the slope is negative};
%READY      }
%READY    \end{tikzpicture}
%READY  \end{center}



\subsubsection*{Concavity: Direction of Instantaneous Acceleration}

Finding instantaneous acceleration $a(t)$ from a position vs.\ time graph is
\emph{very} difficult.\footnote{Of course, if you already know the function
$r(t)$ analytically, and if you know calculus, then you can differentiate the
function twice to find the exact analytical solution to acceleration $a(t)$. In
that case you don't really \emph{need} this graph in the first place.}, but we
can still find the \emph{direction} of acceleration based on the concavity of
the graph, i.e.\ whether the graph opens up or down. If the graph
\emph{concaves up} (it ``opens up''), then acceleration is towards the positive
direction ($a>0$); if the graph \emph{concaves down} (it ``opens down''), then
acceleration is towards the negative direction ($a<0$).

As an example, in Fig.~\ref{fig:pos-time-graph-concavity}, which is the same
position vs.\ time graph from Fig.~\ref{fig:pos-time-graph}.
Here, motion begins at $t=0$ with a positive acceleration. We know this to be
true because the slope (instantaneous velocity) becomes higher
(``more positive'') over time. The graph concaves up. Then at $t=t_d$,
acceleration is zero. This point on the graph is came the inflection point.
From $t>t_d$, the graph concaves down, and acceleration is negative. Bear in
mind that velocity is still positive until a later time (at $t=t_c$, which was
discussed in the previous section).

\begin{figure}[ht]
  \centering
  \begin{tikzpicture}[scale=1.3]
    \draw[axes] (0,0)--(3.5,0) node[right]{$t$};
    \draw[axes] (0,-.7)--(0,2.5) node[right]{$d$};
    \draw[functions,smooth,samples=30,domain=0:2.95]
    plot({\x},{-.2*\x^4+.5*\x^3+.4*\x^2-.5});
    \fill[magenta!35,opacity=.3] (0,-.7) rectangle (1.47,2.3);
    \draw[<-,thick,red] (.75,.2) to[out=120,in=0] +(-1,1)
    node[left,text width=88,draw=red,fill=magenta!10]{For $t<t_0$, acceleration
      is positive ($a>0$) because the graph opens \underline{up}};
    \draw[thick,gray](1.47,1.03)--(1.47,0) node[below,black]{$t_0$};
    \fill[red!80!black] (1.47,1.03) circle (.055);

    \draw[functions,smooth,samples=30,domain=1.48:2.95,blue]
    plot({\x},{-.2*\x^4+.5*\x^3+.4*\x^2-.5});
    \fill[red!80!black] (1.47,1.03) circle (.055);
    \fill[cyan!35,opacity=.3] (1.47,-.7) rectangle (3,2.3);
    \draw[<-,thick,blue] (2.25,1.8) to[out=270,in=180] +(1.6,-2.2)
    node[right,text width=88,draw=blue,fill=cyan!30]{For $t>t_0$, acceleration
      is negative ($a<0$) because the graph opens \underline{down}};
  \end{tikzpicture}
  \caption{Finding instantaneous acceleration information from position vs.\
    time graph}
  \label{fig:pos-time-graph-concavity}
\end{figure}


\subsection{Velocity vs.\ Time Graph}

A less obvious choice for expressing 1D motion is by plotting
\emph{instantaneous} velocity as a function of time, i.e.\ $v=v(t)$. In
essence, we are plotting the slope of the position vs.\ time graph instead.
Note that describing motion using velocity vs.\ time graph is not at all
unusual. For example, or submarine can easily measure its motion (velocity)
relative to the ocean, by directly measuring how fast it's propellers are
turning. Using our example position vs.\ time graph:

\begin{figure}[ht]
  \centering
  \begin{tikzpicture}[scale=1.1]
    \draw[axes] (0,0)--(3.5,0) node[above]{$t$};
    \draw[axes] (0,-1)--(0,2.5) node[right]{$d$};
    \draw[functions,smooth,samples=30,domain=0:2.95,magenta]
    plot(\x,{-.2*\x^4+.5*\x^3+.4*\x^2-.5});

    \fill[violet] (1.3,.7) circle (.055);
    \draw[violet] (1.3,.7)--(1.3,0) node[below]{$t_0$};
    \draw[thick,violet,rotate around={atan(1.8):(1.3,.7)}] (.3,.7)--+(2,0)
    node[midway,above,rotate=atan(1.8)]{$m=v(t_0)$};
  \end{tikzpicture}
  \begin{tikzpicture}[scale=1.1]
    \draw[axes] (0,0)--(3.5,0) node[above]{$t$};
    \draw[axes] (0,-1)--(0,2.5) node[right]{$v$};
    \draw[smooth,samples=40,domain=0:2.5,functions,violet]
    plot(\x,{-.8*\x^3+1.5*\x^2+.8*\x});
    \draw[gray] (1.3,0)--(1.3,1.83) node[pos=0,below,black]{$t_0$}
    --(0,1.83) node[left,black]{$v(t_0)$};
    \fill[violet] (1.3,1.83) circle (.06);
  \end{tikzpicture}
\end{figure}


%READY\subsection[$v$ vs.\ $t$]{Velocity vs.\ Time Graph}
%READY
%READY\begin{frame}{Velocity vs.\ Time Graph}
%READY  A less obvious choice for expressing 1D motion is by plotting
%READY  \emph{instantaneous} velocity as a function of time, i.e.\ $v=v(t)$. Now we
%READY  plot the slope of the position vs.\ time graph instead. Using our example
%READY  position vs.\ time graph:
%READY  \begin{columns}[T]
%READY    \column{.3\textwidth}
%READY    \centering
%READY    \begin{tikzpicture}[scale=1.1]
%READY      \draw[axes] (0,0)--(3.5,0) node[above]{$t$};
%READY      \draw[axes] (0,-1)--(0,2.5) node[right]{$d$};
%READY      \draw[functions,smooth,samples=30,domain=0:2.95,magenta]
%READY      plot(\x,{-.2*\x^4+.5*\x^3+.4*\x^2-.5});
%READY      \uncover<2->{
%READY        \fill[violet] (1.3,.7) circle (.055);
%READY        \draw[violet] (1.3,.7)--(1.3,0) node[below]{$t_0$};
%READY        \draw[thick,violet,rotate around={atan(1.8):(1.3,.7)}] (.3,.7)--+(2,0)
%READY        node[midway,above,rotate=atan(1.8)]{$m=v(t_0)$};
%READY      }
%READY    \end{tikzpicture}
%READY    
%READY    \column{.3\textwidth}
%READY    \centering
%READY    \uncover<2->{
%READY      \begin{tikzpicture}[scale=1.1]
%READY        \draw[axes] (0,0)--(3.5,0) node[above]{$t$};
%READY        \draw[axes] (0,-1)--(0,2.5) node[right]{$v$};
%READY        \draw[smooth,samples=40,domain=0:2.5,functions,violet]
%READY        plot(\x,{-.8*\x^3+1.5*\x^2+.8*\x});
%READY        \draw[gray] (1.3,0)--(1.3,1.83) node[pos=0,below,black]{$t_0$}
%READY        --(0,1.83) node[left,black]{$v(t_0)$};
%READY        \fill[violet] (1.3,1.83) circle (.06);
%READY      \end{tikzpicture}
%READY    }
%READY  \end{columns}
%READY  %\vspace{-.1in}In many situations, the velocity information is directly
%READY  %obtained (e.g.\ experimentally), instead of relying on position vs.\ time
%READY  %graph.
%READY
%READY
%READY
%READY
%READY\begin{frame}{Velocity vs.\ Time Graph}
%READY  The velocity vs.\ time graph shows how instantaneous velocity evolves with
%READY  time. In this example, which corresponds to the $d$ vs.\ $t$ graph from the
%READY  previous slides:
%READY  \begin{center}
%READY    \begin{tikzpicture}
%READY      \draw[axes] (0,0)--(3,0) node[right=-2]{$t$};
%READY      \draw[axes] (0,-1.5)--(0,2.2) node[above=-2]{$v$};
%READY      \draw[smooth,samples=45,domain=0:2.55,very thick,violet]
%READY      plot(\x,{-.8*\x^3+1.5*\x^2+.8*\x});
%READY      \uncover<2->{
%READY        \fill[orange] (2.3,0) circle (.055) node[below left=-2]{$t_1$};
%READY        \fill[orange] circle (.055) node[left=-2]{$0$};
%READY        \draw[<-,orange,thick] (-.4,0)--+(-3,0)
%READY        node[text width=160,draw=orange,fill=orange!10]{
%READY          At $t=0$ and $t=t_1$, velocity is zero ($v=0$). This corresponds to
%READY          when the $d$ vs.\ $t$ graph has zero slope};
%READY      }
%READY      \uncover<3->{
%READY        \fill[magenta!30,opacity=.3] (0,-1.5) rectangle (2.3,2);
%READY        \draw[<-,thick,red] (1.1,-.5)--+(0,-1.7)
%READY        node[text width=260,draw=red,fill=magenta!10]{For $0<t<t_1$, velocity
%READY          is positive ($v>0$) because the graph is above the time axis. This
%READY          corresponds to when the slope of the $d$ vs.\ $t$ graph is positive};
%READY      }
%READY      \uncover<4->{
%READY        \fill[blue!30,opacity=.3] (2.3,-1.5) rectangle (2.6,2);
%READY        \draw[<-,thick,blue] (2.4,.5) to[out=45,in=135] +(1.3,-.5)
%READY        node[right,text width=160,draw=blue,fill=cyan!20]{For $t>t_1$, velocity
%READY          is negative ($v<0$) because the graph is below the time axis. This
%READY          coresponds to when the slope of the $d$ vs.\ $t$ graph is negative};
%READY      }
%READY%      \uncover<3->{
%READY%        \draw[gray] (1.47,1.87)--(1.47,0) node[below,violet]{$t_0$};
%READY%        \fill[violet] (1.47,1.87) circle (.055);
%READY%        \draw[<-,thick,violet] (1.55,1.87) to[out=30,in=150] +(1.6,0)
%READY%        node[right,text width=75,draw=violet,fill=violet!10]{\scriptsize
%READY%          At $t=t_0$, velocity is maximum. The slope is zero at this point.
%READY%          \par};
%READY%      }
%READY    \end{tikzpicture}
%READY  \end{center}
%READY%  Velocity is positive when the graph is above the time axis; and negative
%READY%  when below the time axis
%READY
%READY%    \begin{tikzpicture}[scale=1.1]
%READY%      \draw[axes] (0,0)--(3.5,0) node[above]{$t$};
%READY%      \draw[axes] (0,-1.25)--(0,2.75) node[right]{$v$};
%READY%      \draw[functions,smooth,samples=40,domain=0:3]
%READY%      plot({\x},{1.35*(\x-1)*(\x-1)-1.1*\x+.5});
%READY%      \uncover<2>{
%READY%        \draw[smooth,samples=40,domain=.626:2.189,orange,very thick]
%READY%        plot({\x},{1.35*(\x-1)*(\x-1)-1.1*\x+.5});
%READY%        \draw[smooth,samples=40,domain=2.189:3,violet,very thick]
%READY%        plot({\x},{1.35*(\x-1)*(\x-1)-1.1*\x+.5});
%READY%        \draw[smooth,samples=40,domain=0:.626,violet,very thick]
%READY%        plot({\x},{1.35*(\x-1)*(\x-1)-1.1*\x+.5});
%READY%      }
%READY%    \end{tikzpicture}
%READY%  \end{columns}
%READY


\subsubsection*{Slope of Secant: Average Acceleration}
The \emph{slope of the secant} of the velocity vs.\ time graph is the
\textbf{average acceleration} of the motion.
%READY  \begin{center}
%READY    \begin{tikzpicture}[scale=1.1]
%READY      \draw[axes] (0,0)--(3,0) node[right=-2]{$t$};
%READY      \draw[axes] (0,-1.5)--(0,2.2) node[above=-2]{$v$};
%READY      \draw[smooth,samples=45,domain=0:2.55,very thick,violet]
%READY      plot(\x,{-.8*\x^3+1.5*\x^2+.8*\x});
%READY
%READY      \draw[very thick,dashed] (.5,.675)--(1.75,1.71);
%READY      \fill (.5,.675) circle (.05);
%READY      \draw[gray] (.5,0)--(.5,.675) node[pos=0,below]{$t_1$}--(0,.675)
%READY      node[left]{$v_1$};
%READY      \fill (1.75,1.71) circle (.05);
%READY      \draw[gray] (1.75,0)--(1.75,1.71) node[pos=0,below]{$t_2$}--(0,1.71)
%READY      node[left]{$v_2$};
%READY      
%READY      \draw[<->] (.5,.675)--(1.75,.675) node[midway,below]{$\Delta t$};
%READY      \draw[<->] (1.75,.675)--(1.75,1.71) node[midway,right]{$\Delta v$};
%READY    \end{tikzpicture}
%READY  \end{center}



\subsubsection*{Slope of Tangent: Instantaneous Acceleration}

The slope of the tangent of the velocity vs.\ time graph is the instantaneous
acceleration of the object.
%READY%  \begin{center}
%READY%    \begin{tikzpicture}
%READY%      \draw[axes] (0,0)--(3,0) node[right=-2]{$t$};
%READY%      \draw[axes] (0,-1.5)--(0,2.2) node[above=-2]{$v$};
%READY%      \draw[smooth,samples=45,domain=0:2.55,very thick,violet]
%READY%      plot(\x,{-.8*\x^3+1.5*\x^2+.8*\x});
%READY%      %\uncover<2->{
%READY%      %  \fill[violet] (2.3,0) circle (.055) node[below left=-2]{$t_1$};
%READY%      %  \fill[violet] (0,0) circle (.055) node[left=-2]{$0$};
%READY%      %}
%READY%    \end{tikzpicture}
%READY%  \end{center}



\subsubsection*{Area Under the Graph: Displacement}
The area under the velocity vs.\ time graph is the \emph{displacement} of the
object. %In example below, the shaded area is the displacement between
%READY  $t_1$ and $t_2$. If the position at $t_1$ (i.e.\ $d_1=d(t_1)$) is known, then
%READY  we can find the position at $t_2$ (i.e.\ $d_2=d_1+\Delta d$).
%READY  \begin{center}
%READY    \begin{tikzpicture}[scale=1.1]
%READY      \draw[axes] (0,0)--(3.5,0) node[above]{$t$};
%READY      \draw[axes] (0,-1)--(0,2.5) node[right]{$v$};
%READY      \draw[smooth,samples=50,domain=0:2.4,functions,violet]
%READY      plot(\x,{-.8*\x^3+1.5*\x^2+.8*\x});
%READY      \draw[smooth,samples=30,domain=.5:2,thick,gray,fill=lightgray]
%READY      plot(\x,{-.8*\x^3+1.5*\x^2+.8*\x})--(2,0) node[below,black]{$t_2$}
%READY      --(.5,0) node[below,black]{$t_1$}--cycle;
%READY      \node at (1.25,.7) {$\Delta d$};
%READY    \end{tikzpicture}
%READY  \end{center}
%READY  \begin{itemize}
%READY  \item If the area is \emph{above} the $x$-axis (time axis), then displacement
%READY    is positive ($\Delta d>0$)
%READY  \item If the area is \emph{below} the $x$-axis, then displacement is negative
%READY    ($\Delta d<0$)
%READY  \end{itemize}



\subsection{Acceleration vs.\ Time Graph}
%READY
%READY\begin{frame}{(Instantaneous) Acceleration vs.\ Time Graph}
In the same way that we convert a position vs.\ time graph to a velocity vs.\
time graph, we can also convert a velocity vs.\ time graph to an
\textbf{acceleration vs.\ time} graph, by plotting the slope of the tangent
%READY  \begin{columns}
%READY    \column{.3\textwidth}
%READY    \centering
%READY    \begin{tikzpicture}[scale=1.1]
%READY      \draw[axes] (0,0)--(3,0) node[right]{$t$};
%READY      \draw[axes] (0,-1)--(0,2.5) node[right]{$v$};
%READY      \draw[smooth,samples=40,domain=0:2.4,functions,violet]
%READY      plot(\x,{-.8*\x^3+1.5*\x^2+.8*\x});
%READY      %\fill[violet] (1.3,.7) circle (.06);
%READY      %\draw[violet] (1.3,.7)--(1.3,0) node[below]{$t_0$};
%READY    \end{tikzpicture}
%READY    
%READY    \column{.3\textwidth}
%READY    \centering
%READY    \begin{tikzpicture}[scale=1.1]
%READY      \draw[axes] (0,0)--(3,0) node[right]{$t$};
%READY      \draw[axes] (0,-2.3)--(0,1.2) node[right]{$a$};
%READY      \draw[smooth,samples=40,domain=0:2.2,functions,orange]
%READY      plot(\x,{-1.2*\x^2+1.5*\x+.4});
%READY      %\draw[gray] (1.3,0)--(1.3,1.83) node[pos=0,below,black]{$t_0$}
%READY      %--(0,1.83) node[left,black]{$v(t_0)$};
%READY      %\fill[violet] (1.3,1.83) circle (.06);
%READY    \end{tikzpicture}
%READY  \end{columns}
%READY
%READY
%READY
%READY
%READY\begin{frame}{Acceleration vs.\ Time Graph}
%READY  The acceleration vs.\ time graph shows how \emph{instantaneous} acceleration
%READY  $a(t)$ evolves with time. In the example below:
%READY  \begin{center}
%READY    \begin{tikzpicture}
%READY      \draw[axes] (0,0)--(2.8,0) node[right=-2]{$t$};
%READY      \draw[axes] (0,-2)--(0,1.2) node[right=-2]{$a$};
%READY      \draw[smooth,samples=40,domain=0:2.2,functions,orange]
%READY      plot(\x,{-1.2*\x^2+1.5*\x+.4});
%READY      \uncover<2->{
%READY        \fill[pink!40,opacity=.4] (0,-2) rectangle (1.48,1);
%READY        \fill[orange] (0.63,.87) circle (.06);
%READY        \draw[gray] (.63,0)--(.63,.87) node[pos=0,below=-2,black]{$t_0$}
%READY        --(0,.87) node[left=-2,black]{$a_\text{max}$};
%READY        \fill[orange] (1.48,0) circle (.06) node[above,black]{$t_1$};
%READY        \draw[thick,magenta,<-] (.75,-1)--+(-1,0)
%READY        node[left,text width=165,draw=magenta]{Acceleration is positive between
%READY          $0<t<t_1$, with a maximum magnitude of $a_\text{max}$ at $t_0$};
%READY
%READY        \fill[violet] (1.48,0) circle (.06) node[above]{$t_1$};
%READY        \draw[thick,violet,<-] (1.48,.4) to[out=90,in=180] (3,1)
%READY        node[right,text width=142,draw=violet,fill=violet!10]{Acceleration is
%READY          zero ($a=0$) at $t=t_1$. Note that it does \emph{not} mean that the
%READY          object is stationary at this time};
%READY
%READY        \fill[cyan!40,opacity=.4] (1.48,-2) rectangle (2.4,1);
%READY        \draw[thick,blue,<-] (2.1,-.8)--+(1,0)
%READY        node[right,text width=142,draw=blue,fill=cyan!10]{Acceleration is
%READY          negative for $t>t_1$ because the graph is below the time axis};
%READY      }
%READY    \end{tikzpicture}
%READY  \end{center}
%READY  \uncover<2>{
%READY    Remember: Since acceleration is a vector, \emph{positive}
%READY    acceleration means acceleration \emph{in the positive
%READY    \underline{direction}}, but it does not necessarily mean the object speeds
%READY    up
%READY  }


\subsubsection*{Slope of the Acceleration vs.\ Time Graph}
The slope of the acceleration vs.\ time graph is the rate of change of
acceleration, called \textbf{jerk}. The slope of the tangent is called
\textbf{instantaneous jerk}, whereas the slope of the secant is the
\textbf{average jerk}/
%READY%  This is \emph{not} a topic that is covered in Grade 11 or 12 Physics.
%READY%  \begin{center}
%READY%    \begin{tikzpicture}
%READY%      \draw[axes] (0,0)--(3,0) node[right]{$t$};
%READY%      \draw[axes] (0,-2)--(0,1.2) node[right]{$a$};
%READY%      \draw[smooth,samples=40,domain=0:2.2,functions,orange]
%READY%      plot(\x,{-1.2*\x^2+1.5*\x+.4});
%READY%      \fill[blue] (1,.7) circle (.055);
%READY%      \draw[blue,thick,rotate around={-atan(.9):(1,.7)}] (0,.7)--+(2.2,0);
%READY%    \end{tikzpicture}
%READY%  \end{center}

\subsubsection*{Area Under the Acceleration vs.\ Time Graph}
The area under acceleration vs.\ time graph is the \emph{change in velocity}
$\Delta v$.
%READY  \begin{center}
%READY    \begin{tikzpicture}
%READY      \draw[axes] (0,0)--(3,0) node[right]{$t$};
%READY      \draw[axes] (0,-2)--(0,1.2) node[right]{$a$};
%READY      \draw[smooth,samples=40,domain=0:2.2,functions,orange]
%READY      plot(\x,{-1.2*\x^2+1.5*\x+.4});
%READY      \uncover<2->{
%READY        \draw[smooth,samples=40,domain=.3:1.3,thick,gray,fill=lightgray!50]
%READY        plot(\x,{-1.2*\x^2+1.5*\x+.4})--(1.3,0) node[below=-2,black]{$t_2$}
%READY        --(.3,0) node[below=-2,black]{$t_1$}--cycle;
%READY        \draw[thick,<-,black!80] (.8,.3)--+(-1.5,0)
%READY        node[text width=110,left,draw=black!80,fill=lightgray!50]{$\Delta v>0$
%READY          from $t_1\to t_2$ because the area is above the time axis};
%READY      }
%READY      \uncover<3->{
%READY        \draw[smooth,samples=40,domain=1.7:2.1,thick,cyan,fill=cyan!20]
%READY        plot(\x,{-1.2*\x^2+1.5*\x+.4})--(2.1,0) node[above=-2]{$t_4$}
%READY        --(1.7,0) node[above=-2]{$t_3$}--cycle;
%READY        \draw[thick,<-,blue] (1.9,-.5)--+(2,0)
%READY        node[text width=110,right,draw=blue,fill=cyan!20]{$\Delta v<0$ from
%READY          $t_3\to t_4$ because the area is below the time axis};
%READY      }
%READY    \end{tikzpicture}
%READY  \end{center}
%READY  \begin{itemize}
%READY  \item If the area is \emph{above} the $x$-axis (time axis), then $\Delta v>0$
%READY  \item If the area is \emph{below} the $x$-axis, then $\Delta v<0$
%READY  \item Remember: $\Delta v>0$ does not necessarily mean that the object
%READY    speeds up; $\Delta v<0$ does not necessarily mean that it will slow down
%READY  \end{itemize}

%READY\subsection{Uniform Motion}
%READY
%READY\begin{frame}{Uniform Motion}
%READY  When the velocity is constant , the
%READY  resulting motion is called \textbf{uniform motion}. In 1D, motion graphs for
%READY  uniform motion look like this:
%READY  \begin{center}
%READY    \begin{tikzpicture}[scale=.6]
%READY      \draw[axes] (0,0)--(4.5,0) node[right]{$t$};
%READY      \draw[axes] (0,0)--(0,4.5) node[right]{$d$};
%READY      \draw[functions] (0,.5)--(4,4)
%READY      node[midway,sloped,above=-3]{
%READY        $\text{slope}=v=\text{constant}$};
%READY      \fill[red!80!black] (0,.5) circle (.1) node[left]{$d_1$};
%READY    \end{tikzpicture}
%READY    \hspace{.1in}
%READY    \begin{tikzpicture}[scale=.6]
%READY      \draw[axes] (0,0)--(4.5,0) node[right]{$t$};
%READY      \draw[axes] (0,0)--(0,4.5) node[right]{$v$};
%READY      \draw[functions] (0,2)--(4,2)
%READY      node[midway,above=-3]{$\text{slope}=a=0$};
%READY      \fill[red!80!black] (0,2) circle (.1) node[left]{$v$};
%READY    \end{tikzpicture}
%READY    \hspace{.1in}
%READY    \begin{tikzpicture}[scale=.6]
%READY      \draw[axes] (0,0)--(4.5,0) node[right]{$t$};
%READY      \draw[axes] (0,0)--(0,4.5) node[right]{$a$};
%READY      \draw[functions] (0,0)--(4,0);
%READY      \fill[red!80!black] circle (.1) node[left]{$a=0$};
%READY    \end{tikzpicture}
%READY  \end{center}
%READY  \begin{itemize}
%READY  \item \vspace{-.15in}$d$--$t$ graph is a straight line
%READY  \item The slope of the $d$--$t$ graph, which is velocity, is constant. The
%READY    sign of the slope indicates the direction of motion
%READY  \item There is no acceleration, so $a=0$ for all $t$
%READY  \end{itemize}
%READY
%READY
%READY
%\subsection{Uniform Acceleration}
%
%\begin{frame}{Uniform Acceleration}
%  A constant non-zero acceleration is called \textbf{uniform acceleration}.
%  It is due to a constant net force acting on the object.
%
%  \vspace{-.2in}\begin{columns}[T]
%    \column{.33\textwidth}
%    \begin{center}
%      \begin{tikzpicture}[scale=.6]
%        \draw[axes] (0,0)--(4.5,0) node[right]{$t$};
%        \draw[axes] (0,-1.1)--(0,3.5) node[right]{$d$};
%        \draw[functions,smooth,samples=10,domain=0:4]
%        plot({\x},{.35*(\x-1)*(\x-1)-1});
%        \fill[red!80!black] (0,-.65) circle (.1) node[left]{$d_1$};
%      \end{tikzpicture}
%    \end{center}
%    
%    {\footnotesize Position $d(t)$ is described by the quadratic function in
%      time:
%      \begin{displaymath}
%        d(t)=d_1+v_1t+\frac12at^2
%      \end{displaymath}
%      \par}
%    
%    \column{.33\textwidth}
%    \begin{center}
%      \begin{tikzpicture}[scale=.6]
%        \draw[axes] (0,0)--(4.5,0) node[right]{$t$};
%        \draw[axes] (0,-1.1)--(0,3.5) node[right]{$v$};
%        \draw[functions] (0,-.8)--(4,3) node[midway,sloped,above=-3]{
%          $\text{slope}=a=\text{constant}$};
%        \fill[red!80!black] (0,-.8) circle (.1) node[left]{$v_1$};
%      \end{tikzpicture}
%    \end{center}
%    
%    {\footnotesize Velocity $v(t)$ is described by the linear function in time:
%      \begin{displaymath}
%        v(t)=v_1+at
%      \end{displaymath}
%      \par}
%
%    \column{.33\textwidth}
%    \begin{center}
%      \begin{tikzpicture}[scale=.6]
%        \draw[axes] (0,0)--(4.5,0) node[right]{$t$};
%        \draw[axes] (0,-1.1)--(0,3.5) node[right]{$a$};
%        \draw[functions] (0,1)--(4,1);
%        \fill[red!80!black] (0,1) circle (.1) node[left]{$a$};
%      \end{tikzpicture}
%    \end{center}
%
%    {\footnotesize Acceleration $a(t)$ is just a constant:
%      \begin{displaymath}
%        a(t)=a
%      \end{displaymath}
%      \par}
%
%  \end{columns}
%  \begin{itemize}
%  \item\vspace{-.1in}The $d$--$t$ graph is a \emph{parabola}\footnote{Just
%    because a graph \emph{looks like} a parabola isn't necessarily mean that it
%    is!}
%    \begin{itemize}
%    \item If the parabola opens up, then acceleration is positive
%    \item If the parabola opens down, then acceleration is negative
%    \end{itemize}
%  \item The $v$--$t$ graph is a straight line; the slope is the acceleration
%  \end{itemize}
%
%
%
%\subsection{Position vs.\ Time Graph}


%%\item Position in one-dimension can be $+/-$
%%Generally, motion begins at $t=0$.
%As shown in Figure~\ref{fig:pos-time-graph}, to find position at time $t_0$,
%you can simply read the graph!
%%\item Time only moves forward, but the graph does not explicitly tell you
%%  so
%
%
%
%

%
%%
%%
%%
%%\begin{frame}{Velocity vs.\ Time Graph}
%%%  \begin{columns}
%%%    \column{.7\textwidth}
%%%    \begin{itemize}
%%  The velocity vs.\ time graph shows how instantaneous velocity evolves with
%%  time. In this example:
%%%    \item<3->Slope of the secant: average acceleration
%%%    \item<3->Slope of the tangent: instantaneous acceleration
%%%    \end{itemize}
%%%    
%%%    \column{.3\textwidth}
%%  \begin{center}
%%    \begin{tikzpicture}
%%      \draw[axes] (0,0)--(3,0) node[right=-2]{$t$};
%%      \draw[axes] (0,-1.5)--(0,2.2) node[above=-2]{$v$};
%%      \draw[smooth,samples=45,domain=0:2.55,very thick,violet]
%%      plot(\x,{-.8*\x^3+1.5*\x^2+.8*\x});
%%      \uncover<2->{
%%        \fill[violet] (2.3,0) circle (.055) node[below left=-2]{$t_1$};
%%        \fill[violet] circle (.055) node[left=-2]{$0$};
%%        \draw[<-,violet,thick] (-.4,0)--+(-3,0)
%%        node[text width=80,draw=violet,fill=violet!10]{\scriptsize
%%          At $t=0$ and $t=t_1$, velocity is zero ($v=0$)\par};
%%      }
%%      \uncover<3->{
%%        \draw[gray] (1.47,1.87)--(1.47,0) node[below,violet]{$t_0$};
%%        \fill[violet] (1.47,1.87) circle (.055);
%%        \draw[<-,thick,violet] (1.55,1.87) to[out=30,in=150] +(1.6,0)
%%        node[right,text width=75,draw=violet,fill=violet!10]{\scriptsize
%%          At $t=t_0$, velocity is maximum. The slope is zero at this point.
%%          \par};
%%      }
%%    \end{tikzpicture}
%%  \end{center}
%%  Velocity is positive when the graph is above the time axis; and negative
%%  when below the time axis
%%
%%%    \begin{tikzpicture}[scale=1.1]
%%%      \draw[axes] (0,0)--(3.5,0) node[above]{$t$};
%%%      \draw[axes] (0,-1.25)--(0,2.75) node[right]{$v$};
%%%      \draw[functions,smooth,samples=40,domain=0:3]
%%%      plot({\x},{1.35*(\x-1)*(\x-1)-1.1*\x+.5});
%%%      \uncover<2>{
%%%        \draw[smooth,samples=40,domain=.626:2.189,orange,very thick]
%%%        plot({\x},{1.35*(\x-1)*(\x-1)-1.1*\x+.5});
%%%        \draw[smooth,samples=40,domain=2.189:3,violet,very thick]
%%%        plot({\x},{1.35*(\x-1)*(\x-1)-1.1*\x+.5});
%%%        \draw[smooth,samples=40,domain=0:.626,violet,very thick]
%%%        plot({\x},{1.35*(\x-1)*(\x-1)-1.1*\x+.5});
%%%      }
%%%    \end{tikzpicture}
%%%  \end{columns}
%%
%%
%%
%
%\textbf{Average Acceleration:}
%%  \begin{center}
%%    \begin{tikzpicture}[scale=1.1]
%%      \draw[axes] (0,0)--(3,0) node[right=-2]{$t$};
%%      \draw[axes] (0,-1.5)--(0,2.2) node[above=-2]{$v$};
%%      \draw[smooth,samples=45,domain=0:2.55,very thick,violet]
%%      plot(\x,{-.8*\x^3+1.5*\x^2+.8*\x});
%%      %\uncover<2->{
%%      %  \fill[violet] (2.3,0) circle (.055) node[below left=-2]{$t_1$};
%%      %  \fill[violet] (0,0) circle (.055) node[left=-2]{$0$};
%%      %}
%%    \end{tikzpicture}
%%  \end{center}
%%
%%
%%
%%
%\textbf{Instantaneous Acceleration:}
%%  \begin{center}
%%    \begin{tikzpicture}[scale=1.1]
%%      \draw[axes] (0,0)--(3,0) node[right=-2]{$t$};
%%      \draw[axes] (0,-1.5)--(0,2.2) node[above=-2]{$v$};
%%      \draw[smooth,samples=45,domain=0:2.55,very thick,violet]
%%      plot(\x,{-.8*\x^3+1.5*\x^2+.8*\x});
%%      %\uncover<2->{
%%      %  \fill[violet] (2.3,0) circle (.055) node[below left=-2]{$t_1$};
%%      %  \fill[violet] (0,0) circle (.055) node[left=-2]{$0$};
%%      %}
%%    \end{tikzpicture}
%%  \end{center}
%
%
%
%\textbf{Displacement and position:} The area under the velocity vs.\ time graph
%is the \emph{displacement} of the object. In example below, the shaded area is
%the displacement between $t_1$ and $t_2$. If the position at $t_1$ (i.e.\
%$d_1$) is also known, then we can find the position at $t_2$ (i.e.\
%$d_2=d_1+\Delta d$).
%\begin{figure}[ht]
%  \centering
%  \begin{tikzpicture}[scale=1.3]
%    \draw[smooth,samples=30,domain=.5:2,gray,fill=lightgray]
%    plot(\x,{-.8*\x^3+1.5*\x^2+.8*\x})--(2,0) node[below,black]{$t_2$}
%    --(.5,0) node[below,black]{$t_1$}--cycle;
%
%    \draw[smooth,samples=50,domain=0:2.4,functions,violet]
%    plot(\x,{-.8*\x^3+1.5*\x^2+.8*\x});
%
%    \draw[axes] (0,0)--(3.5,0) node[above]{$t$};
%    \draw[axes] (0,-.5)--(0,2.5) node[right]{$v$};
%
%    \node at (1.25,.7) {$\Delta d$};
%  \end{tikzpicture}
%  \caption{The area under the velocity vs.\ time graph between $t_1$ and $t_2$
%    gives us the object's displacement during this time interval.}
%  \label{fig:area-under-vt-graph}
%\end{figure}
%If the area is \emph{above} the $x$-axis (time axis), then displacement
%is positive ($\Delta d>0$); if the area is \emph{below} the $x$-axis, then
%displacement is negative ($\Delta d<0$).
%
%
%
%\subsection{Acceleration vs. Time Graph}
%In the same way that we convert a position vs.\ time graph to a velocity vs.\
%time graph, we can also convert a velocity vs.\ time graph to an
%\textbf{acceleration vs.\ time} graph, by plotting the slope of the tangent.
%This graph shows how \emph{instantaneous} acceleration $a(t)$ evolves with
%time. In the example below:
%%    \centering
%%    \begin{tikzpicture}[scale=1.1]
%%      \draw[axes] (0,0)--(3,0) node[right]{$t$};
%%      \draw[axes] (0,-1)--(0,2.5) node[right]{$v$};
%%      \draw[smooth,samples=40,domain=0:2.4,functions,violet]
%%      plot(\x,{-.8*\x^3+1.5*\x^2+.8*\x});
%%      %\fill[violet] (1.3,.7) circle (.06);
%%      %\draw[violet] (1.3,.7)--(1.3,0) node[below]{$t_0$};
%%    \end{tikzpicture}
%%    
%%    \column{.3\textwidth}
%%    \centering
%%    \begin{tikzpicture}[scale=1.1]
%%      \draw[axes] (0,0)--(3,0) node[right]{$t$};
%%      \draw[axes] (0,-2.3)--(0,1.2) node[right]{$a$};
%%      \draw[smooth,samples=40,domain=0:2.2,functions,orange]
%%      plot(\x,{-1.2*\x^2+1.5*\x+.4});
%%      %\draw[gray] (1.3,0)--(1.3,1.83) node[pos=0,below,black]{$t_0$}
%%      %--(0,1.83) node[left,black]{$v(t_0)$};
%%      %\fill[violet] (1.3,1.83) circle (.06);
%%    \end{tikzpicture}
%%  \end{columns}
%
%%\begin{frame}{Acceleration vs.\ Time Graph}
%
%%  \begin{center}
%%    \begin{tikzpicture}[scale=1.1]
%%      \draw[axes] (0,0)--(2.8,0) node[right=-2]{$t$};
%%      \draw[axes] (0,-2)--(0,1.2) node[right=-2]{$a$};
%%      \draw[smooth,samples=40,domain=0:2.2,functions,orange]
%%      plot(\x,{-1.2*\x^2+1.5*\x+.4});
%%      \uncover<2->{
%%        \fill[pink!40,opacity=.4] (0,-2) rectangle (1.48,1);
%%        \fill[orange] (0.63,.87) circle (.06);
%%        \draw[gray] (.63,0)--(.63,.87) node[pos=0,below=-2,black]{$t_0$}
%%        --(0,.87) node[left=-2,black]{$a_\text{max}$};
%%        \fill[orange] (1.48,0) circle (.06) node[above,black]{$t_1$};
%%        \draw[thick,magenta,<-] (.75,-1)--+(-1,0)
%%        node[left,text width=80,draw=magenta]{\scriptsize
%%          Acceleration is positive from $t=0\rightarrow t_1$, with a maximum
%%          magnitude of $a_\text{max}$ at $t_0$\par};
%%      }
%%      \uncover<3->{
%%        \fill[violet] (1.48,0) circle (.06) node[above]{$t_1$};
%%        \draw[thick,violet,<-] (1.48,.4) to[out=90,in=180] (3,1)
%%        node[right,text width=70,draw=violet,fill=violet!10]{\scriptsize
%%          Acceleration is zero ($a=0$) at $t=t_1$\par};
%%      }
%%      \uncover<4->{
%%        \fill[cyan!40,opacity=.4] (1.48,-2) rectangle (2.4,1);
%%        \draw[thick,blue,<-] (2.1,-.8)--+(1,0)
%%        node[right,text width=63,draw=blue,fill=cyan!10]{\scriptsize
%%          Acceleration is negative for $t>t_1$\par};
%%      }
%%    \end{tikzpicture}
%%  \end{center}
%%  \uncover<4->{
%%    \vspace{-.1in}Remember: Since acceleration is a vector, \emph{positive}
%%    acceleration means acceleration \emph{in the positive
%%    \underline{direction}}, but it does not necessarily mean the object speeds
%%    up
%%  }
%%
%%
%%
%
%\subsubsection{Slope of the Acceleration vs.\ Time Graph}
%The slope of the acceleration vs.\ time graph is the rate of change of
%acceleration, called \textbf{jerk}.\footnote{The slope of the tangent is
%called \textbf{instantaneous jerk}, whereas the slope of the secant is the
%\textbf{average jerk}.}. This is \emph{not} a topic that is covered in Grade
%11 or 12 Physics.
%\begin{figure}[ht]
%  \centering
%  \begin{tikzpicture}
%    \draw[axes] (0,0)--(3,0) node[right]{$t$};
%    \draw[axes] (0,-2)--(0,1.2) node[right]{$a$};
%    \draw[smooth,samples=40,domain=0:2.2,functions,orange]
%    plot(\x,{-1.2*\x^2+1.5*\x+.4});
%    \fill[blue] (1,.7) circle (.055);
%    \draw[blue,thick,rotate around={-atan(.9):(1,.7)}] (0,.7)--+(2.2,0);
%  \end{tikzpicture}
%\end{figure}
%
%
%
%%\begin{frame}{Area Under the Acceleration vs.\ Time Graph}
%%  The area under acceleration vs.\ time graph is the \emph{change in velocity}
%%  $\Delta v$.
%%  \begin{center}
%%    \begin{tikzpicture}
%%      \draw[axes] (0,0)--(3,0) node[right]{$t$};
%%      \draw[axes] (0,-2)--(0,1.2) node[right]{$a$};
%%      \draw[smooth,samples=40,domain=0:2.2,functions,orange]
%%      plot(\x,{-1.2*\x^2+1.5*\x+.4});
%%      \uncover<2->{
%%        \draw[smooth,samples=40,domain=.3:1.3,thick,gray,fill=lightgray!50]
%%        plot(\x,{-1.2*\x^2+1.5*\x+.4})--(1.3,0) node[below=-2,black]{$t_2$}
%%        --(.3,0) node[below=-2,black]{$t_1$}--cycle;
%%        \draw[thick,<-,black!80] (.8,.3)--+(-1.5,0)
%%        node[text width=72,left,draw=black!80,fill=lightgray!50]{\scriptsize
%%          $\Delta v>0$ from $t_1\rightarrow t_2$ because the area is above the
%%          $x$ axis\par};
%%      }
%%      \uncover<3->{
%%        \draw[smooth,samples=40,domain=1.7:2.1,thick,cyan,fill=cyan!20]
%%        plot(\x,{-1.2*\x^2+1.5*\x+.4})--(2.1,0) node[above=-2]{$t_4$}
%%        --(1.7,0) node[above=-2]{$t_3$}--cycle;
%%        \draw[thick,<-,blue] (1.9,-.5)--+(2,0)
%%        node[text width=75,right,draw=blue,fill=cyan!20]{\scriptsize
%%          $\Delta v<0$ from $t_3\rightarrow t_4$ because the area is below the
%%          $x$ axis\par};
%%      }
%%    \end{tikzpicture}
%%  \end{center}
%%  \begin{itemize}
%%  \item If the area is \emph{above} the $x$-axis (time axis), then $\Delta v>0$
%%  \item If the area is \emph{below} the $x$-axis, then $\Delta v<0$
%%  \item Remember: $\Delta v>0$ does not necessarily mean that the object
%%    speeds up; $\Delta v<0$ does not necessarily mean that it will slow down
%%  \end{itemize}
%%


\subsection{Uniform Motion}

When an object moves along a straight line with \emph{constant} velocity
(i.e.\ both magnitude and direction of velocity are constant), the object's
motion is called \textbf{uniform motion}. The motion graphs for uniform
motion is shown in Fig.~\ref{fig:uniform-motion-graphs}. The position vs.\ time
graph for a uniform motion is a straight line, and its slope is the (constant)
velocity. Fig. 1.10a shows a uniform motion with a constant positive velocity.
Because velocity is constant, velocity vs.\ time graph is a horizontal straight
line, and the (constant) functional value of that graph is the (constant)
velocity of the object. An object in uniform motion has zero acceleration,
therefore the slope of the velocity versus time graph is 0, and the
acceleration versus time graph is a horizontal line that coincides with the
$x$-axis (time axis).
\begin{figure}[ht]
  \centering
  \begin{subfigure}{.3\textwidth}
    \centering
    \begin{tikzpicture}[scale=.8]
      \draw[axes] (0,0)--(4.5,0) node[right]{$t$};
      \draw[axes] (0,0)--(0,4.5) node[above]{$r$};
      \draw[functions] (0,.5)--(4,4);
    \end{tikzpicture}
    \caption{Position vs.\ time}
  \end{subfigure}
  \begin{subfigure}{.3\textwidth}
    \centering
    \begin{tikzpicture}[scale=.8]
      \draw[axes] (0,0)--(4.5,0) node[right]{$t$};
      \draw[axes] (0,0)--(0,4.5) node[above]{$v$};
      \draw[functions] (0,2)--(4,2);
    \end{tikzpicture}
    \caption{Velocity vs.\ time}
  \end{subfigure}
  \begin{subfigure}{.3\textwidth}
    \centering
    \begin{tikzpicture}[scale=.8]
      \draw[axes] (0,0)--(4.5,0) node[right]{$t$};
      \draw[axes] (0,0)--(0,4.5) node[above]{$a$};
      \draw[functions] (0,0)--(4,0);
    \end{tikzpicture}
    \caption{Acceleration vs.\ time}
  \end{subfigure}
  \caption{Basic motion graphs for uniform motion}
  \label{fig:uniform-motion-graphs}
\end{figure}
%\begin{itemize}
%\item \vspace{-.15in}$d$--$t$ graph is a straight line
%\item The slope of the $d$--$t$ graph, which is velocity $v$, is also constant
%\item There is no acceleration, so $a=0$ for all $t$
%\end{itemize}




\subsection{Uniform Acceleration}
When a constant net force acts on an object, it moves with a constant
non-zero acceleration, or \textbf{uniform acceleration}.
\begin{figure}[ht]
  \centering
  \begin{subfigure}{.3\textwidth}
    \centering
    \begin{tikzpicture}[scale=.8]
      \draw[axes] (0,0)--(4.5,0) node[right]{$t$};
      \draw[axes] (0,-1.5)--(0,4.5) node[above]{$r$};
      \draw[functions,samples=20,domain=0:4,smooth] plot(\x,{.35*(\x-1)*(\x-1)+1});
    \end{tikzpicture}
  \end{subfigure}
  \begin{subfigure}{.3\textwidth}
    \centering
    \begin{tikzpicture}[scale=.8]
      \draw[axes] (0,0)--(4.5,0) node[right]{$t$};
      \draw[axes] (0,-1.5)--(0,4.5) node[above]{$v$};
      \draw[functions] (0,-1)--(4,2.5);
    \end{tikzpicture}
  \end{subfigure}
  \begin{subfigure}{.3\textwidth}
    \centering
    \begin{tikzpicture}[scale=.8]
      \draw[axes] (0,0)--(4.5,0) node[right]{$t$};
      \draw[axes] (0,-1.5)--(0,4.5) node[above]{$a$};
      \draw[functions] (0,1)--(4,1);
    \end{tikzpicture}
  \end{subfigure}
  \caption{Motion graphs for uniformly accelerated motion}
  \label{fig:uniform-acceleration-graphs}
\end{figure}
\begin{itemize}
\item The $d$ vs.\ $t$ graph is part of a \emph{parabola}
  \begin{itemize}
  \item opens \emph{up}, then acceleration is positive
  \item opens \emph{down}, then acceleration is negative
  \end{itemize}
\item The $v$ vs.\ $t$ graph is a straight line; the (constant) slope is the
  cceleration
\end{itemize}




%\section{Area Under Motion Graphs}
%\begin{center}
%  \begin{tikzpicture}[scale=.55]
%    \draw[axes] (0,0)--(4.5,0) node[right]{$t$};
%    \draw[axes] (0,-1.5)--(0,4.5) node[above]{$d$};
%    \draw[functions,samples=10,domain=0:4] plot(\x,{.35*(\x-1)*(\x-1)+1});
%  \end{tikzpicture}
%  \hspace{.15in}
%  \begin{tikzpicture}[scale=.55]
%    \draw[pink!40,fill=pink!40] (0,0)--(0,-1)--(1,0)--cycle;
%    \draw[blue!20,fill=blue!20] (1,0)--(3.5,0)--(3.5,2.5)--cycle;
%    \draw[axes] (0,0)--(4.5,0) node[right]{$t$};
%    \draw[axes] (0,-1.5)--(0,4.5) node[above]{$v$};
%    \draw[functions] (0,-1)--(4,3);
%  \end{tikzpicture}
%  \hspace{.15in}
%  \begin{tikzpicture}[scale=.55]
%    \fill[gray!40] rectangle (4,1) node[black,midway]{$\Delta v$};
%    \draw[axes] (0,0)--(4.5,0) node[right]{$t$};
%    \draw[axes] (0,-1.5)--(0,4.5) node[above]{$a$};
%    \draw[functions] (0,1)--(4,1);
%  \end{tikzpicture}
%\end{center}
%\begin{itemize}
%\item The area under the $a$--$t$ graph is the change in velocity $\Delta v$
%  \begin{itemize}
%  \item If initial velocity is known, then we can plot $v$--$t$ graph based on
%    this graph
%  \end{itemize}
%\item The area under the $v$--$t$ graph is the displacement $\Delta d$
%  \begin{itemize}
%  \item If the area is {\color{red!40}below} the $x$-axis (time axis), then
%    displacement is negative;
%  \item If the area is {\color{blue!20}above} the time axis, then
%    displacement is positive
%  \end{itemize}
%\item The area under the $d$--$t$ graph has no physical meaning
%\end{itemize}




\subsection{Simple Harmonic Motion}
In \textbf{simple harmonic motion}\footnote{Or \textbf{oscillatory motion},
or \textbf{vibration}}, displacement, velocity and acceleration are all
periodic functions, and none of them are constant!

\begin{figure}[ht]
  \centering
  \begin{subfigure}{.3\textwidth}
    \begin{tikzpicture}[scale=.8]
      \draw[axes] (0,0)--(4.5,0) node[right]{$t$};
      \draw[axes] (0,-2.5)--(0,2.5) node[above]{$d$};
      \draw[functions,samples=50,domain=0:4] plot(\x,{2*cos(150*\x)});
    \end{tikzpicture}
  \end{subfigure}
  \begin{subfigure}{.3\textwidth}
    \begin{tikzpicture}[scale=.8]
      \draw[axes] (0,0)--(4.5,0) node[right]{$t$};
      \draw[axes] (0,-2.5)--(0,2.5) node[above]{$v$};
      \draw[functions,samples=50,domain=0:4] plot(\x,{-2*sin(150*\x)});
    \end{tikzpicture}
    \end{subfigure}
  \begin{subfigure}{.3\textwidth}
    \begin{tikzpicture}[scale=.8]
      \draw[axes] (0,0)--(4.5,0) node[right]{$t$};
      \draw[axes] (0,-2.5)--(0,2.5) node[above]{$a$};
      \draw[functions,samples=50,domain=0:4] plot(\x,{-2*cos(150*\x)});
    \end{tikzpicture}
  \end{subfigure}
  \caption{Motion graphs for simple harmonic motion}
\end{figure}
%We will discuss this topic later, in the next unit.



\begin{example}
  Express the motion in a \emph{position vs.\ time} graph
  and an \emph{acceleration vs.\ time} graph. Assume the object's initial
  position is the origin of the coordinate system.
  \begin{center}
    \begin{tikzpicture}
      \draw[help lines] grid (12,4);
      \draw[axes] (0,0)--(13,0) node[right]{$t$ [\si\second]};
      \foreach \t in {0,...,12} \draw(\t,0)--(\t,-.15) node[below]{$\t$};
      \draw[axes] (0,0)--(0,5) node[above]{$v$ [\si{\metre\per\second}]};
      \foreach \v in {0,...,4} \draw(0,\v,0)--(-.15,\v) node[left]{$\v$};

      \draw[functions] (0,2)--(2.5,2)--(5,4)--(7.5,4)--(10,1)--(12,1);
    \end{tikzpicture}
  \end{center}
\end{example}






\section{Advanced Graphing Techniques}

\subsection{Advanced Graphing Technique: $x$ vs.\ $t^2$}

A toy cart rolls down a ramp from rest (i.e.\ $v_0=0$). At regular distances,
time is recorded. In this case, the acceleration of the cart \emph{should} be
constant. Can we determine the car's acceleration?
\begin{figure}[ht]
  \centering
  \begin{tikzpicture}[scale=.7,rotate around={-10:(9,0)}]
    \draw[thick] (0,0)--+(8.8,0);
    \draw[thick] (.3,.1) rectangle +(.8,.2);
    \draw[thick,fill=black!2] (.5,.1) circle (.1);
    \draw[thick,fill=black!2] (.9,.1) circle (.1);
    \foreach \x in {.7,1.2,...,8.5}
    \draw[thick](\x,0)--+(0,-.15); % node[below]{\tiny$\x$};
  \end{tikzpicture}
\end{figure}

Based on the data, we can plot an $x$--$t$ graph. Unfortunately this graph
would not be very helpful in determining  acceleration.

\begin{tabular}{c|c}
  $x$ & $t$ \\\hline
  $x_0$ & 0 \\
  $x_1$ & $t_1$\\
  \vdots & \vdots \\
  $x_N$ & $t_N$
\end{tabular}

\begin{center}
  \begin{tikzpicture}
    \draw[axes](0,0)--(0,2) node[right]{$x$};
    \draw[axes](0,0)--(2,0) node[right]{$t$};
    \draw[smooth,domain={0:1.8},very thick] plot(\x,{.4*\x*\x+.3});
  \end{tikzpicture}
\end{center}


If, instead, we plot $x$--$t^2$, the graph would be \emph{linear}, and
the slope will tell us about the acceleration.

\begin{tabular}{c|c}
  $x$ & $t^2$ \\\hline
  $x_0$ & 0 \\
  $x_1$ & $t_1^2$\\
  \vdots & \vdots \\
  $x_N$ & $t_N^2$
\end{tabular}

\begin{center}
  \begin{tikzpicture}
    \draw[axes] (0,0)--(0,2) node[right]{$x$};
    \draw[axes] (0,0)--(2,0) node[right]{$t^2$};
    \draw[very thick] (0,.3)--(1.8,1.8);
  \end{tikzpicture}
\end{center}



When we plot $x$ as a function of $t^2$ instead of a function of $t$, we are
essentially doing this to the kinematic equation (from a few slide ago):
\begin{equation}
  \underbracket{x}_y=\underbracket{x_0}_b +
  \underbracket{\frac12a}_m\underbracket{t^2}_x
\end{equation}
and turning it into a linear function in the form $y=mx+b$ that is familiar to
everyone.
\begin{itemize}
\item The slope of the $x$ vs. $t^2$ graph is $\dfrac12a$ (or acceleration
  is 2 times the slope)
\item If the $x$ vs.\ $t^2$ graph is \emph{not} linear, we will know that
  our assumption of constant acceleration was incorrect, and that there are
  other factors that we neglected
\end{itemize}



\subsection{Advanced Graphing Technique: $v^2$ vs.\ $\Delta x$}

A toy cart rolls down a ramp. At regular positions $x$ along the ramp, the
cart's velocity $v$ is recorded instead. Again, acceleration \emph{should}
be constant. Can we determine the car's acceleration from the data?
\begin{figure}[ht]
  \centering
  \begin{tikzpicture}[scale=.7,rotate around={-10:(9,0)}]
    \draw[thick] (0,0)--+(8.8,0);
    \draw[thick] (.3,.1) rectangle +(.8,.2);
    \draw[thick,fill=black!2] (.5,.1) circle (.1);
    \draw[thick,fill=black!2] (.9,.1) circle (.1);
    \foreach \x in {.7,1.2,...,8.5}
    \draw[thick](\x,0)--+(0,-.15); % node[below]{\tiny$\x$};
  \end{tikzpicture}
\end{figure}
      
Plotting the data would give us a $v$--$x$ graph. Unfortunately (again) this
graph would not help us find the acceleration.

\begin{tabular}{c|c}
  $x$ & $v$ \\\hline
  $x_0$ & $v_0$ \\
  $x_1$ & $v_1$\\
  \vdots & \vdots \\
  $x_N$ & $v_N$
\end{tabular}

\begin{center}
  \begin{tikzpicture}
    \draw[axes] (0,0)--(0,2) node[right]{$v$};
    \draw[axes] (0,0)--(2,0) node[right]{$x$};
    \draw[smooth,domain={0:1.8},samples=40,very thick]
    plot(\x,{\x^.5+.3});
  \end{tikzpicture}
\end{center}
But if we plot $v^2$ vs.\ $\Delta x$, the graph would be \emph{linear},
and the slope will give us information about acceleration.

\begin{tabular}{c|c}
  $v^2$ & $\Delta x$ \\\hline
  $v_0$ & 0 \\
  $v_1$ & $x_1-x_0$\\
  \vdots & \vdots \\
  $v_N$ & $x_N-x_0$
\end{tabular}

\begin{center}
  \begin{tikzpicture}
    \draw[axes] (0,0)--(0,2) node[right]{$v^2$};
    \draw[axes] (0,0)--(2,0) node[right]{$\Delta x$};
    \draw[very thick] (0,.3)--(1.8,1.8);
  \end{tikzpicture}
\end{center}

If velocity and position are given, then the relationship would be best
expressed using this kinematic equation (we have studied a few slide ago):
\begin{equation}
  \underbracket[1pt]{v^2}_y=\underbracket[1pt]{v_0^2}_b+
  \underbracket[1pt]{2a}_m
  \underbracket[1pt]{(x-x_0)}_x
\end{equation}
by plotting $v^2$ vs. $\Delta x=x-x_0$, we again have a linear function in the
form of $y=mx+b$.
\begin{itemize}
\item The slope of the graph is two times the acceleration $m=2a$ (i.e.\
  acceleration is one half of the slope).
\item The square of the initial velocity ($v_0^2$) is the $y$-intercept
\item \emph{If} our assumption was incorrect (i.e.\ acceleration is \emph{not}
  constant) then the line would not be straight
\end{itemize}




\subsection{Graphing ``Linear'' Functions: Example}
  
This concept extends to graphing other physical relationships not related to
kinematics. For example, to find the index of refraction of a material using
Snell's law, we can plot $\sin\theta_1$ vs.\ $\sin\theta_2$ (instead than
$\theta_1$ vs.\ $\theta_2$). The slope is the ratio of the indices $n_1/n_2$:
\begin{equation}
  \underbracket[1pt]{\sin\theta_2}_y=
  \underbracket[1pt]{\left[\frac{n_1}{n_2}\right]}_m
  \underbracket[1pt]{\sin\theta_1}_x
\end{equation}
In an experiment, we can vary the incident angle $\theta_1$ in a medium with
a known index ($n_1$), and measure the corresponding refracted angle
$\theta_2$. Once the slope of the graph ($n_1/n_2$) is known, we can find
$n_2$.




\subsection*{Graphing ``Linear'' Functions: Another Example}
To relate the period of oscillation (the time it takes to swing back and
forth once) of a simple pendulum to the length of the pendulum, plot $T$
vs.\ $\sqrt\ell$:
\begin{equation}
  \underbracket[1pt]{T}_y=
  \underbracket[1pt]{\frac{2\pi}{\sqrt g}}_m
  \underbracket[1pt]{\sqrt\ell}_x
\end{equation}
or alternatively, by squaring both sides of the equation to relate $T^2$ to
$\ell$:
\begin{equation}
  \underbracket[1pt]{T^2}_y=
  \underbracket[1pt]{\left[\frac{4\pi^2}g\right]}_m
  \underbracket[1pt]{\ell}_x
\end{equation}
We can change the length of the pendulum $\ell$ and measure the period $T$ of
the oscillation. This relationship will be studied in the last topic of this
course, in Class 15.





\section{A Further Discussion on Acceleration}

Many concepts that we study in physics have strict \emph{formal} definitions
(when we solve problems), as well as \emph{casual} definitions (when we
communicate with non-scientists). Concepts presented in this book that fall into
this category include:
\begin{itemize}
\item Distance
\item{\color{red}Acceleration}
\item Work
\item Energy
\item Power
\end{itemize}

The formal definition of acceleration is clear: acceleration is \emph{the
rate of change of instantaneous velocity}, i.e.\ how quickly instantaneous
velocity changes with time:

%$  \eq{-.1in}{
%$    \boxed{
%$      \bm a_\text{avg}(t) = \frac{\Delta\bm v(t)}{\Delta t}
%$      = \frac{\bm v(t)-\bm v(t_0)}{t-t_0}
%$    }
%$  }
%$  %where $\bm v_i=\bm v(t_i)$ is the initial velocity.
\begin{itemize}
\item A change in the velocity vector can mean a change in magnitude and/or
  direction
\item There is acceleration as long as $\Delta\bm v\neq 0$
\item There can be acceleration without any speeding up or slowing down
\item A car driven at constant speed around a circular track experiences
  acceleration
\end{itemize}
The reason for acceleration is because an imbalance of forces (called the
\textbf{net force}) acts on an object\footnote{This is the second law of
motion}. The direction of acceleration is the same as the net force.

Like all vectors in 1D, the direction of acceleration is either
\emph{positive} or \emph{negative}, assuming that the positive direction is
known ahead of time.
\begin{itemize}
\item\textbf{Positive/negative acceleration} means acceleration in either the
  positive/negative direction
\item Since acceleration is caused by a force, a positive/negative
  acceleration means that the object is being \emph{pushed} in that direction
\item Whether the object speeds up or slows down depends on the velocity
  vector when the acceleration begins, as shown in the slides for Class 1.
\end{itemize}

But we must also be aware of the \emph{casual} definition of acceleration in
everyday conversations as well:
\begin{itemize}
\item\textbf{Acceleration:} the rate of increase in \emph{speed}
  (``speeding up'')
\item\textbf{Deceleration:} the rate of decrease in \emph{speed}
  (``slowing down'')
\end{itemize}
Since velocity is a vector (needs a direction) quantity while speed is a
scalar (no direction) quantity, therefore the formal and casual definitions
of acceleration can potentially be describing something very different.

With this in mind, let's clear up two misconceptions about acceleration.
\begin{center}
  \vspace{.1in}\fcolorbox{black}{yellow!15}{
    \begin{minipage}{.65\linewidth}
      \textbf{Misconception \#1:} There is no such thing as
      \emph{de}celeration.
    \end{minipage}
  }
\end{center}
%$  \vspace{.1in}
\begin{itemize}
\item\textbf{THIS IS CORRECT:} Under the \emph{formal} definition,
  ``deceleration'' is not a proper kinematic quantity. \emph{Any} change in
  velocity results in an acceleration
\item\textbf{BUT PAY ATTENTION:} \emph{Deceleration} is an \emph{acceptable}
  term often used in the scientific/engineering/technical
  community\footnote{Go ahead, ask \emph{any} engineer!}
\item We have already used this term in a homework question
\end{itemize}


\begin{center}
  \vspace{.1in}\fcolorbox{black}{yellow!15}{
    \begin{minipage}{.64\linewidth}
      \textbf{Misconception \#2:} Positive acceleration means speeding up;
      negative acceleration means slowing down.
    \end{minipage}
  }
\end{center}
\textbf{THIS IS ONLY A SPECIAL CASE:}
\begin{itemize}
\item For 1D problems with one object moving \emph{only} forward, it is
  customary to define the forward direction to be positive (and therefore
  backwards is negative)
\item In this \emph{special} case (and only in this special case!) this
  statement is correct
\end{itemize}
\textbf{IN GENERAL:} This statement does not make any sense when
\begin{itemize}
\item The object alternates in direction of its motion (e.g.\ vibrations)
\item There are multiple objects moving both directions (e.g.\ traffic on
  a highway)
\item Motion in two or higher dimensions
\end{itemize}

Scientists \& engineers communicate with other scientists \& engineers, and
also with others with different technical and non-technical backgrounds
\begin{itemize}
\item Understand the difference between the formal \& casual definitions of
  acceleration (and other physical quantities that have multiple meaning)
  \item Check your work to see whether solving questions using either
    definitions result in different answers
  \item Know your audience: are \emph{they} communicating with you using the
    formal definition?
  \end{itemize}









\chapter{Two Dimensional Kinematics}
\label{chapter:2d-kinematics}

As we begin to describe motion in higher dimensions, it is important to extend
our understanding of vectors into a two-dimensional space. In this section, we
will learn some basic two-dimensional vector arithmetic (decomposition,
addition, subtraction and multiplication) that can apply to any kind of
vectors. Note that once we become familiar with two-dimensional vectors,
extending our understanding to three dimensions is much easier.



\section{Two-Dimensional Coordinate System}

The coordinate system for two dimensional motion is the Cartesian $xy$-plane,
as we have discussed briefly in the previous chapter. Again, we attribute the
$x$ and $y$ axes based on the physical directions. For example, for motion on
level ground, we may wish to align the $x$ axis to the east, and the $y$ axis
to the north. For projectile motions, which will be discussed in details in
Section 2.3, we may align the $x$-axis to the forward direction, and the
$y$-axis to upward direction. There are also instances where a generic
$xy$-plane would suffice. Like one-dimensional motion, the origin of the
coordinate system (the reference point) is selected for convenience to
simplify calculations.



\section{Describing a 2D Vector}

Once the two-dimensional coordinate system has been set, \emph{any} vector
inside the coordinate system is a two-dimensional vector. Like vectors in
higher dimensions are also represented as arrows. The magnitude of the vector
is the length of the arrow; the direction of the vector can be described by the
angle it makes with the $x$-axis. We can explicitly state the magnitude of the
vector, with the direction inside a bracket:
\begin{center}
  magnitude [direction]
\end{center}
Fig.~\ref{fig:2d-vectors} shows four examples of 2D vectors on the xy plane.
Vectors $\bm A$ and $\bm B$ have both $x$ and $y$ ``components''; vector
$\bm C$ is only along the $+x$ axis; and vector $\bm D$ is entirely along the
$-y$ axis.
\begin{figure}[ht]
  \centering
  \begin{tikzpicture}[scale=.7]
    \draw[help lines,gray] (-5,-3) grid (5,4);
    \draw[axes] (-5.3,0)--(5.3,0) node[right]{$x$};
    \draw[axes] (0,-3.3)--(0,4.3) node[right]{$y$};
    \draw[vectors,red] (0,0)--(4,3) node[right]{$\bm A$};
    \draw[axes] (1.5,0) arc (0:atan(3/4):1.5)
    node[midway,right]{\SI{37}\degree};
    \draw[vectors,red] (-2,3)--+(-2,-2) node[left]{$\bm B$};
    \draw[thick,dash dot] (-2,3)--+(1.25,0);
    \draw[axes] (-1,3) arc (0:225:1) node[pos=.75,left]{\SI{225}\degree};
    \draw[vectors,red] (2,-1)--+(2,0) node[below]{$\bm C$};
    \draw[vectors,red] (0,-1)--+(0,-1.5) node[right]{$\bm D$};
  \end{tikzpicture}
  \caption{Examples of two-dimensional vectors}
  \label{fig:2d-vectors}
\end{figure}
We can express these vectors by directly stating their magnitudes and
directions:
\begin{align*}
  \bm A &= 5.0\;[\SI{37}\degree]\\
  \bm B &= 1.4\;[\SI{135}\degree]\\
  \bm C &= 2.0\;[\SI0\degree]\\
  \bm D &= 1.5\;[\SI{270}\degree]
\end{align*}
In introductory-level physics courses, this is often the preferred way of
expressing 2D vectors, because it can be understood intuitively. However, in
high-level physics courses, it is often more convenient to express 2D vectors
in component form.

In Cartesian space, regardless of where the vector is drawn, as long they have
the same magnitude and direction (i.e.\ making the same angle with the
$x$-axis), the vectors are the same. Vectors do not have to be drawn from the
origin. (In practice, for example, while position vectors must always drawn
from the origin, displacement vectors do not.) As shown in
Fig.~\ref{fig:3-same-vectors}, the vectors $\bm A$, $\bm A'$, and $\bm A''$ all
have the same length, and make the same angle with the $x$-axis, therefore all
three vectors are equal, i.e. $\bm A=\bm A'=\bm A''$.
\begin{figure}[ht]
  \centering
  \begin{tikzpicture}[scale=.5]
    \draw[axes] (-3,0)--(10,0) node[right] {$x$};
    \draw[axes] (0,0)--(0,5) node[above] {$y$};
    \draw[vectors,violet] (0,0)--(4,3) node[above]{$\bm A$};
    \draw[vectors,magenta] (5,1.5)--(9,4.5) node[above]{$\bm A'$};
    \draw[vectors,orange] (-2,2)--(2,5) node[above]{$\bm A''$};
    \draw[dashed,thick] (-2,2)--(0,0)--(5,1.5);
    \draw[dashed,thick] (2,5)--(4,3)--(9,4.5);
  \end{tikzpicture}
  \caption{Three equal vectors in a two-dimensional cartesian space}
  \label{fig:3-same-vectors}
\end{figure}



\section{Vector Decomposition}
In the previous section, we alluded to the fact that some 2D vectors have
components. These components are the \textbf{projection} of the vector on to
the axes. For our current purpose, we can think of the projection as the
shadow that the vector


%\section{Vectors in Two or More Dimensions}
%
%A vector is usually expressed as a line segment with an arrowhead. This
%vector can be \emph{any} kind of a vector\footnote{In this course, we will
%study position, displacement,velocity, or acceleration, force, and
%gravitational field vectors.}; the mathematical principles are the same.

%\begin{itemize}
%\item\textbf{Magnitude} $|\bm A|$: the length of the line
%\item\textbf{Direction} $\phi$: where the arrow is pointing at
%\end{itemize}

Consider the vector $\bm A$ shown in Fig.~\ref{fig:2d-decomposition}.
\begin{figure}[ht]
  \centering
  \begin{tikzpicture}[scale=.9]
    \draw[help lines] grid (5,4);
    \draw[axes] (0,0)--(5.5,0) node[right] {$x$};
    \draw[axes] (0,0)--(0,4.5) node[above] {$y$};
    \draw[axes] (2,0) arc (0:atan(3/4):2) node[midway,right] {$\phi$};
    \draw[vectors] (0,0)--(4,3) node[midway,above] {$\bm A$};
    \draw[vectors,magenta] (0,0)--(0,3) node[midway,left] {$\bm A_y$};
    \draw[vectors,dashed,magenta] (4,0)--(4,3) node[midway,right]{$\bm A_y$};
    \draw[vectors,cyan] (0,0)--(4,0) node[midway,below]{$\bm A_x$};
  \end{tikzpicture}
  \caption{An example two-dimensional vector and its $x$ and $y$ components.}
  \label{fig:2d-decomposition}
\end{figure}
%\begin{figure}[ht]
%  \centering
%  \begin{tikzpicture}[scale=.8]
%    \draw[help lines] grid (5,4);
%    \draw[axes] (0,0)--(5.5,0) node[right]{$x$};
%    \draw[axes] (0,0)--(0,4.5) node[above]{$y$};
%    \draw[axes] (2,0) arc (0:atan(3/4):2) node[midway,right]{$\phi$};
%    \draw[vectors,red] (0,0)--(4,3) node[midway,above] {$\bm A$};
%  \end{tikzpicture}

%\end{figure}

%\section{Vector Decomposition}
%
%In \textbf{vector decomposition}, vectors are exprssed by breaking down into
%their components along the axes of the coordinate system.
%\begin{columns}
%  \column{.27\textwidth}
%  \input{2d-vector}
%  
%  \column{.73\textwidth}
%  \begin{itemize}
%  \item A 2D vector $\bm A$ can be ``decomposed'' into two 1D components:
%    \begin{itemize}
%      \item $x$-component ({\color{cyan}$\bm A_x$}) along the $x$ direction
%      \item $y$-component ({\color{magenta}$\bm A_y$}) along the $y$
%        direction
%      \end{itemize}
%    \item The components are \emph{projections} of the vector onto the axes
%    \item Analogy: Think of shining a flashlight on an arrow. The projection
%      is the shadow cast on the ground ($x$-axis) or a wall ($y$-axis)
%    \end{itemize}
%  \end{columns}
%
%
%
%
%    \input{2d-vector}
%
We can express this vector as a sum of the component vectors:
\begin{equation*}
  \bm A = \bm A_x + \bm A_y
\end{equation*}
Or more commonly,
\begin{equation*}
  \bm A = A_x\hat{\bm x} + A_y\hat{\bm y}
\end{equation*}
where $\hat{\bm x}$ and $\hat{\bm y}$ are \textbf{unit vectors} (magnitude of 1)
representing the ``directions of the $x$ and $y$ axes''. The vectors shown
in Fig.~\ref{fig:2d-vectors} can now be expressed in \emph{component form}:
\begin{align*}
  \bm A &= 5\;[\SI{37}\degree] = 4\hat{\bm x} + 3\hat{\bm y}\\
  \bm B &= 1.4\;[\SI{135}\degree] = -2\hat{\bm x} + -2\hat{\bm y}\\
  \bm C &= 2\;[\SI0\degree] = 2\hat{\bm x}\\
  \bm D &= 1.5\;[\SI{270}\degree] = -1.5\hat{\bm y}
\end{align*}
Later, we will show in Section~\ref{sec:vector-addition} why the component form
is written as a sum.


If components $A_x$ and $A_y$ are known, then the vector's magnitude $|\bm A|$
(written as the \emph{absolute value} of the vector) and direction $\phi$ can
be calculated using Pythagorean theorem and basic trigonometry, respectively:

\begin{align}
  |\bm{A}| &=\sqrt{A_x^2+A_y^2}\\
  \phi &=\tan^{-1}\left(\frac{A_y}{A_x}\right)
\end{align}

Conversely, if magnitude $|\bm A|$ and direction $\phi$ are known, then
components $\bm A_x$ and $\bm A_y$ can be easily obtained using the sine and
cosine functions:
\begin{align}
  |\bm A_x| &= |\bm A| \cos\phi\\
  |\bm A_y| &= |\bm A| \sin\phi
\end{align}
Note: A 2D vector can be fully described using any 2 of the parameters: $A_x$,
$A_y$, $A$ or $\phi$




\begin{example}
  Write the displacement vector $\Delta\bm r$ with a magnitude of
  \SI{64}{\metre} at an angle of \ang{120} with the $x$-axis in component form.
  \begin{center}
    \begin{tikzpicture}[scale=.45]
      \draw[axes] (-5,0)--(5,0) node[right] {$x$};
      \draw[axes] ( 0,0)--(0,7) node[above] {$y$};
      \draw[axes] (2,0) arc (1:120:2) node[midway,right=2]{\ang{120}};
      \draw[vectors,rotate=120,blue] (0,0)--(6.4,0) node[above]{$\Delta\bm r$};
    \end{tikzpicture}
  \end{center}
\end{example}



\section{Vector Addition}
\label{sec:vector-addition}

\begin{displaymath}
  {\color{orange}\bm C} = {\color{red}\bm A} + {\color{blue}\bm B}
\end{displaymath}
The vector sum can be visualized by lining up the head of one vector with
the toe of the next one, shown graphically on the right. Like scalars, the
order of the addition does not matter.
\begin{figure}
  \centering
  \begin{tikzpicture}[scale=.5]
    \draw[help lines] (-4,0) grid (4,8);
    \draw[axes] (-4,0)--(4.5,0) node[right]{$x$};
    \draw[axes] (0,0)--(0,8.5) node[above]{$y$};
    \draw[vectors,red] (0,0)--(2,5) node[midway,right]{$\bm A$};
    \draw[vectors,blue] (0,0)--(-3,2) node[midway,left=3]{$\bm B$};
    \draw[vectors,orange] (0,0)--(-1,7) node[midway,left]{$\bm C$};
    \draw[vectors,dashed,blue] (2,5)--(-1,7) node[midway,above]{$\bm B$};
    \draw[vectors,dashed,red] (-3,2)--(-1,7) node[midway,left]{$\bm A$};
  \end{tikzpicture}
\end{figure}

The sum of two vectors can be calculated by individually adding the components:
\begin{equation}
  \bm C=\bm A + \bm B
  =(A_x+B_x)\hat{\bm x} + (A_y+B_y)\hat{\bm y} + (A_z+B_z)\hat{\bm z}
\end{equation}
Not surprisingly:
\begin{equation}
  \bm A + \bm B=\bm B + \bm A
\end{equation}


\section{Vector Subtraction} %: Head to Head}

Consider the difference of two vectors:

\begin{displaymath}
  {\color{orange}\bm D} ={\color{red}\bm A}-{\color{blue}\bm B}
\end{displaymath}
    
It may be easier to express the subtraction as an addition, i.e.:
    
\begin{displaymath}
  {\color{red}\bm A}={\color{orange}\bm D}+{\color{blue}\bm B}
\end{displaymath}

    
\begin{figure}[ht]
  \centering
  \begin{tikzpicture}[scale=.6]
    \draw[help lines] grid (7,6);
    \draw[axes] (0,0)--(8,0) node[right]{$x$};
    \draw[axes] (4,0)--(4,6.5) node[above]{$y$};
    \draw[vectors,red] (4,0)--(6,5) node[midway,right]{$\bm A$};
    \draw[vectors,blue] (4,0)--(1,2) node[midway,above]{$\bm B$};
    \draw[vectors,orange] (1,2)--(6,5) node[midway,above]{$\bm D$};
  \end{tikzpicture}
\end{figure}
  %\vspace{.2in}Vector subtraction is used in calculating displacement
  %$\Delta\bm r=\bm r-\bm r_0$, and the change in velocity (for calculating
  %acceleration) $\Delta\bm v=\bm v-\bm v_0$.


Like the sum, the difference between two vectors can be calculated by
individually subtracting their components:

%  \eq{-.1in}{
%    \boxed{\bm D=\bm A - \bm B
%      =(A_x-B_x)\hat x + (A_y-B_y)\hat y + (A_z-B_z)\hat z
%    }
%  }
%  
%  Not surprisingly:
%
%  \eq{-.1in}{
%    \boxed{\bm D=\bm A - \bm B= -(\bm B - \bm A)}
%    }
%
%
%
%
%\begin{frame}{Displacement}
%  
%  \begin{columns}
%    \column{.7\textwidth}
%    Displacement, which is defined mathematically as:
%    
%    \eq{-.1in}{
%      \Delta\bm r(t)=\bm r(t) -\bm r_0
%    }
%
%    can now be expressed by individually adding or subtracting each component
%    independently:
%    
%    \eq{-.2in}{
%      \Delta\bm r(t)=
%      [x(t)-x_0]\hat x+[y(t)-y_0]\hat y +[z(t)-z_0]\hat z
%    }
%    
%    \column{.3\textwidth}
%    \centering
%    \begin{tikzpicture}[scale=.5]
%      \draw[axes] (0,0)--(6,0) node[right]{$x$};
%      \draw[axes] (0,0)--(0,7) node[above]{$y$};
%      \draw[vectors,red] (0,0)--(5,2) node[midway,above]{$\bm r_0$};
%      \draw[vectors,red] (0,0)--(2,6) node[midway,left]{$\bm r$};
%      \draw[vectors,blue] (5,2)--(2,6) node[midway,right]{$\Delta\bm r$};
%    \end{tikzpicture}
%  \end{columns}
%
%
%
%
%\begin{frame}{Average Velocity}
%  The average velocity of $\bm v_\text{avg}$ of an object can also be expressed
%  by components
%
%  \eq{-.1in}{
%    \bm v_\text{avg}= \frac{\Delta\bm r}{\Delta t}=
%    \frac{\Delta x}{\Delta t}\hat x +
%    \frac{\Delta y}{\Delta t}\hat y +
%    \frac{\Delta z}{\Delta t}\hat z
%  }
%
%  where $\Delta x$, $\Delta y$ and $\Delta z$ are the displacement components
%  in the $\hat x$, $\hat y$ and $\hat z$ directions respectively, i.e.:
%
%  \eq{-.1in}{
%    \Delta\bm r = (\Delta x)\hat x +(\Delta y)\hat y +(\Delta z)\hat z
%  }
%  
%
%
%
%
%\begin{frame}{Average Acceleration}
%  In the same way that average velocity can be separated into components,
%  the average acceleration can also be calculated the same way:
%
%  \eq{-.1in}{
%    \bm a_\text{avg} = \frac{\Delta\bm v}{\Delta t}
%    =\frac{\Delta v_x}{\Delta t}\hat x 
%    +\frac{\Delta v_y}{\Delta t}\hat y
%    +\frac{\Delta v_z}{\Delta t}\hat z
%  }
%  
%  where $\Delta v_x$, $\Delta v_y$ and $\Delta v_z$ are the components of the
%  change in velocity in the $\hat x$, $\hat y$ and $\hat z$ directions
%  respectively, i.e.:
%
%  \eq{-.1in}{
%    \Delta\bm v = (\Delta v_x)\hat x +(\Delta v_y)\hat y+(\Delta v_z)\hat z
%  }




\section{Vector Multiplication \& Division}
%
%\begin{frame}{Vector Multiplication \& Division by Scalar}
%  \begin{center}
%    \begin{tikzpicture}[scale=.7]
%      \draw[help lines] grid (5,3);
%      \draw[axes] (0,0)--(5.5,0) node[right]{$x$};
%      \draw[axes] (0,0)--(0,3.5) node[above]{$y$};
%      \draw[vectors,red] (0,0)--(2,1) node[right]{$\bm A$};
%    \end{tikzpicture}
%    \hspace{.15in}
%    \begin{tikzpicture}[scale=.7]
%      \draw[help lines] grid (5,3);
%      \draw[axes] (0,0)--(5.5,0) node[right]{$x$};
%      \draw[axes] (0,0)--(0,3.5) node[above]{$y$};
%      \draw[vectors,red] (0,0)--(4,2) node[right]{$2\bm A$};
%    \end{tikzpicture}
%  \end{center}
%  When multiplying/dividing a vector by a scalar $k$, multiply/divide each
%  components:
%
%  \eq{-.1in}{
%    k\bm A=(kA_x)\hat x + (kA_y)\hat y + (kA_z)\hat z
%  }
%
%  Direction of the vector reverses if multiplied by a negative number (i.e.\
%  $k<0$)
%
%
%
%
%\begin{frame}{Vector Multiplications}
%  Later, in mechanical work and energy (Classes \#7 and \#8), we will use a
%  vector multiplication procedure call the \textbf{dot product}, or
%  \textbf{inner product}, which is the scalar product of two vectors:
%
%  \eq{-.1in}{
%    C=\bm A\cdot\bm B=A_xB_x+A_yB_y+A_zB_z=AB\cos\theta
%  }
%  
%  \vspace{-.1in}In rotational motion (Classes \#12 and \#13), specifically on
%  the discussion on \emph{torque}, we will also study another vector
%  multiplication called the \textbf{cross product}, which is the vector product
%  of two vectors:
%
%  \eq{-.1in}{
%    \bm C = \bm A\times\bm B\quad\quad |\bm C|=|\bm A||\bm B|\sin\theta
%  }
%
%  \vspace{-.1in}{\footnotesize There is a 3rd multiplication procedure called
%    the ``outer product'', which is the matrix product of two vectors; it is
%    not used in any AP Physics exams.\par}
%
%
%
%
%\section{Other Ways to Describe Vectors}
%
%\begin{frame}{Alternative Forms to Express Vectors}
%  Vectors can also be written in the following alternative format:
%  \begin{center}
%    {\large\textbf{magnitude [direction]}}
%  \end{center}
%  This form is especially common in high-school level physics courses. This
%  way of writing vectors makes it easy to visualize a 2D vector, but it also
%  makes arithmetic operations difficult.
%
%
%
%
%\begin{frame}{Cartesian Coordinate System}
%  \begin{columns}
%    \column{.65\textwidth}
%    When directions are expressed in a 2D Cartesian coordinate system (i.e.\
%    the $xy$-plane), the direction is expressed using the
%    \textbf{standard angle}
%    \begin{itemize}
%    \item The angle is positive when measured counter-clockwise from the
%      positive $x$ axis
%    \item the angle is negative when measured clockwise from the positive $x$
%      axis
%    \end{itemize}
%
%    \column{.35\textwidth}
%    \begin{tikzpicture}[scale=.6]
%      \draw[axes] (-4,0)--(4,0) node[right]{$x$};
%      \draw[axes] (0,-4)--(0,4) node[above]{$y$};
%
%      \draw[vectors,red,rotate=65] (0,0)--(3.7,0);
%      \draw[axes,red] (2,0) arc (0:65:2) node[midway,right]{\ang{65}};
%
%      \draw[vectors,orange,rotate=-53] (0,0)--(3.7,0);
%      \draw[axes,orange] (2.5,0) arc (0:-53:2.5)
%      node[midway,right]{\ang{-53}};
%            
%      \draw[vectors,blue!80!black,rotate=140] (0,0)--(3.7,0);
%      \draw[axes,blue!80!black] (1.5,0) arc (0:140:1.5)
%      node[pos=.65,above]{\ang{140}};
%    \end{tikzpicture}
%  \end{columns}
%
%
%
%\section{Compass Directions}
%
%\begin{frame}{Compass Directions}
%  For any compass direction, there are four ways to express it. For example,
%  the velocity vector $\bm v$ can be expressed two ways by measuring the
%  direction from East towards North:
%  \begin{columns}
%    \column{.65\textwidth}
%    \vspace{-.1in}{\large
%      \begin{align*}
%        \bm v &=\SI{25}{\metre\per\second}\text{ [E \ang{35} N]}\\
%        \bm v &=\SI{25}{\metre\per\second}\text{ [\ang{35} N of E]}\\
%      \end{align*}
%    }
%    
%    \vspace{-.25in}\hspace{-.02in}But it can also be measured from North towards
%    East:
%
%    \vspace{-.25in}{\large
%      \begin{align*}
%        \bm v &=\SI{25}{\metre\per\second}\text{ [N \ang{55} E]}\\
%        \bm v &=\SI{25}{\metre\per\second}\text{ [\ang{55} E of N]}\\
%      \end{align*}
%    }
%    
%    \column{.3\textwidth}
%    \centering
%    \begin{tikzpicture}[scale=1.3]
%      \draw[axes] (-1,0)--(2,0) node[right]{E};
%      \draw[axes] (0,-1)--(0,2) node[above]{N};
%      \draw[vectors,rotate=35,cyan] (0,0)--(1.75,0)
%      node[above]{$\bm v=\SI{25}{\metre\per\second}$};
%      \draw[axes,cyan] (1,0) arc (0:35:1) node[midway,right]{\ang{35}};
%    \end{tikzpicture}
%  \end{columns}
%
%
%
%
%\begin{frame}{Compass Directions: A Warning}
%  \begin{columns}
%    \column{.6\textwidth}
%    In everyday life, you may be able to get away with vague descriptions like
%    ``Brampton is northwest of Toronto'', but in problem-solving in physics,
%    care must be taken when using directions like [NE], [SE], [SW] or [NW].
%    \textbf{They can only be used if the angle is \ang{45}.}
%
%    \column{.4\textwidth}
%    \centering
%    \begin{tikzpicture}[scale=1.2]
%      \draw[axes] (0,0)--(3,0) node[right]{E};
%      \draw[axes] (0,0)--(0,3) node[above]{N};
%      \draw[vectors,magenta,rotate=45] (0,0)--(3,0) node[right]{[NE]};
%      \draw[axes] (1,0) arc(0:45:1) node[above]{\ang{45}};
%      \draw[vectors,cyan,rotate=26] (0,0)--(3,0) node[right]{[E \ang{26} N]};
%      \draw[axes] (2,0) arc(0:26:2) node[midway,right]{\ang{26}};
%    \end{tikzpicture}
%    
%    These are two different directions!\\
%    You must not confuse them!
%  \end{columns}
%
%
%
%
%%\begin{frame}{Bearing}
%%  In navigation\footnote{This applies to both aircraft and ships}, directions
%%  are often expressed in terms of \textbf{bearing}. The bearing of the object
%%  is measured clockwise from the north (i.e.\ towards the east).



\begin{example}
  Dawn starts from home and bikes \SI{3.0}{\kilo\metre} to the north and then
  \SI{4.0}{\kilo\metre} towards the east. What is her total displacement?
\end{example}



\begin{example}
  A kayaker sets out for a paddle on a lake. She heads west, but is blown off
  course by a strong wind. After \SI{1.0}\hour, she
  arrives at a lighthouse \SI{12}{\kilo\metre} southwest of her starting point.
  She waits for the wind to die down; then paddles towards the setting sun and
  arrives at a small island \SI{8.0}{\kilo\metre} west of the lighthouse. In the
  calm of evening, the kayaker plans to paddle straight back to her starting
  point.
  \begin{enumerate}
  \item Determine her displacement from her starting point to the island.
  \item In which direction should she now head and how far will she have to
    paddle to go?
  \end{enumerate}
\end{example}



\begin{example}
  A water-skier begins his ride by being pulled straight
  behind the boat. Initially, he has the same velocity as the boat
  \SI{50}{\kilo\metre\per\hour} towards the north. Once up to speed, the
  water-skier takes control and cuts out to the side. In cutting out to the
  side, the water-skier changes his velocity in both magnitude and direction.
  His new velocity is \SI{60}{\kilo\metre\per\hour} at a direction of
  [N \ang{60} E]. Find the water-skier's change in velocity. 
\end{example}



%\begin{frame}{Different Ways to Write Vectors}
%  There are two ways to write variables that are vector quantities, depending
%  on the book, teacher etc. Most books and technical journals \emph{print} with
%  a bold font:
%
%  \eq{-.1in}{
%    \mathbf r\quad\Delta\mathbf r\quad\mathbf v\quad\mathbf a\quad
%    \mathbf F\quad\mathbf B\quad\mathbf E
%  }
%
%  In introductory physics courses and books, or when \emph{writing} a vector
%  on paper, we usually put an arrow on top of the variable instead:
%
%  \eq{-.1in}{
%    \bm r\quad\Delta\bm r\quad\bm v\quad\bm a\quad\bm F\quad
%    \bm B\quad\bm E
%  }
%
%  (AP Physics exams previously used the bold font, but switched to the
%  arrow-on-top format a few years ago.)
%
%
%
%
%\begin{frame}{Different Ways to Write Vector Components}
%  As for the vector itself, there are several ways to write the components:
%
%  \vspace{.1in}
%  \begin{columns}[T]
%    \column{.29\textwidth}
%    As coordinates:
%
%    \eq{-.1in}{
%      r(3.5,1.2,6.5)
%    }
%
%    \vspace{.08in}{\footnotesize This format is most common in describing
%      position vectors, but it is used for other types of vectors as well.\par}
%    
%    \column{.39\textwidth}
%    As a sum of vector components:
% 
%    \eq{-.1in}{
%      \bm r =3.5\hat x +1.2\hat y +6.5\hat z
%    }
%
%    \vspace{.08in}{\footnotesize Sometimes the unit vectors are
%      $(\hat\imath,\hat\jmath,\hat k)$, or $(\hat e_x,\hat e_y,\hat e_z)$\par}
%
%    \column{.3\textwidth}
%    Matrix form:
%
%    \eq{-.1in}{
%      \bm r=
%        \begin{bmatrix}
%          3.5 \\
%          1.2 \\
%          6.5
%        \end{bmatrix}
%    }
%    
%    {\footnotesize This format is most commonly used in linear algebra, when
%      solving a system of equations\par}
%  \end{columns}
%  
%  \vspace{.2in}All of these different styles of notation can be extended to
%  vectors of arbitrary number of dimensions






\section{Relative Motion}
All velocities are measured \emph{relative} to a frame of reference.
Therefore, when expressing relative motion, we can use two subscripts:
%    
%  \eq{-.15in}{
%    \bm v_{AB}
%  }
%    
%  \vspace{-.15in}where $A$ represents the object, and $B$ represents the frame
%  of reference
%
%  \vspace{.25in}\textbf{Example:} the velocity of an airplane ($P$) travelling
%  at \SI{251}{\kilo\metre\per\hour} [N] relative to Earth ($E$) is expressed as:
%
%  \eq{-.2in}{
%    \bm v_{PE}=\SI{251}{\kilo\metre\per\hour}\text{ [N]}
%  }
%
%
%
%
%\section{Relative Motion}
%  \begin{itemize}
%  \item Different observers make different observations because they (their
%    frames of reference) are moving relative to each other.
%  \item In \emph{classical} mechanics, the different velocity measurements are
%    related by the \textbf{Galilean velocity addition rule}\footnote{This
%    equation was thought to be so obvious that no one bothered to give it a
%    name until Einstein showed that it is not valid near the speed of light}:
%    
%    \eq{-.1in}{
%      \boxed{\bm v_{AC}=\bm v_{AB}+\bm v_{BC}}
%    }
%
%    \vspace{-.1in}The velocity of $A$ relative to reference frame $C$ is the
%    velocity of $A$ relative to reference frame $B$, plus the velocity of $B$
%    relative to $C$.
%  \item Can only be used when velocity $v$ is small compared to the speed of
%    light $c$
%  \end{itemize}
%
%
%%\section{Relative Motion}
%%  If we add another reference frame ($D$), the equation becomes:
%%
%%  \eq{-.3in}{
%%    \bm v_{AD}=\bm v_{AB}+\bm v_{BC}+\bm v_{CD}
%%  }
%%
%
%
%
%\section{Relative Motion Example: Airplane in Air}
%  The velocity of the plane relative to the ground (``ground speed'') is
%  the velocity of the plane relative to the air (``air speed'') plus
%  the velocity of the air relative to the ground (``wind speed'').  
%  \begin{center}
%    \pic{.4}{graphics/Planewind}
%  \end{center}
%  The addition of velocities is exactly the same as any vector addition.

%\section{Projectile Motion}
%  A \textbf{projectile} is an object that is launched with an initial velocity
%  of $\bm v_1$ along a parabolic trajectory and accelerates only due to
%  gravity.
%  \begin{columns}[T]
%    \column{.3\textwidth}
%    \begin{tikzpicture}[scale=1.6]
%      \draw[axes] (0,0)--(2,0) node[right]{$x$};
%      \draw[axes] (0,0)--(0,2) node[above]{$y$};
%      \draw[dotted,domain=0:2.7,thick] plot (\x, {1.2*\x-.2*\x*\x});
%      \draw[vectors] (0,0)--(.75,.9) node[above]{$\bm v_1$};
%      \draw[vectors,red] (0,0)--(0,.9) node[midway,left]{$v_y$};
%      \draw[vectors,blue] (0,0)--(.75,0) node[midway,below]{$v_x$};
%      \draw[axes] (.5,0) arc (0:52:.5) node[pos=.6,right]{$\theta$};
%    \end{tikzpicture}
%
%    \column{.67\textwidth}
%    \begin{itemize}
%    \item $x$-axis: \emph{horizontal}, pointing \emph{forward}
%    \item $y$-axis: \emph{vertical}, pointing \emph{up}
%    \item Angle $\theta$ measured \emph{above} the horizontal (i.e.\ $\theta>0$
%      when thrown upwards; $\theta<0$ then thrown downwards)
%    \item The origin is usually where the projectile is launched
%    \end{itemize}
%  \end{columns}
%
%
%
%
%\section{Horizontal Direction}
%  The initial velocity $\bm v_1$ can be decomposed into its $x$ and $y$
%  components:
%
%  \vspace{-.25in}{\large
%    \begin{align*}
%      v_x &=v_1\cos\theta \\
%      v_y &=v_1\sin\theta
%      \end{align*}
%  }
%
%  There is no horizontal acceleration (i.e.\ $a_x=0$), therefore $v_x$ is
%  constant. The kinematic equations reduce to a single equation:
%
%  \eq{-.1in}{
%    \Delta x=v_x\Delta t=\left[v_0\cos\theta\right]\Delta t
%  }
%
%  \vspace{-.1in}where $\Delta x$ is the horizontal displacement.
%
%
%
%
%
%\section{Vertical Direction}
%  There is constant vertical acceleration due to gravity alone, i.e.\
%  $a_y=-g$. ($a_y$ is \emph{negative} due to the way we defined the
%  coordinate system, with the $y$-axis pointing up.) The most important
%  kinetic equation is this one:
%
%  \eq{-.1in}{
%    \Delta y = \left[v_1\sin\theta\right]\Delta t-\frac12g\Delta t^2
%  }
%
%  These two kinematic equations may also be useful:
%
%  \vspace{-.25in}{\large
%    \begin{align*}
%      v_y &= \left[v_1\sin\theta\right] -gt\\
%      v_y^2&=\left[v_1^2\sin^2\theta\right]-2g\Delta y
%    \end{align*}
%  }
%
%
%
%
%\section{Solving Projectile Motion Problems}
%  Horizontal and vertical motions are linearly independent, but variables are
%  shared in both directions:
%  \begin{itemize}
%  \item Time interval $\Delta t$
%  \item Launch angle $\theta$ (above the horizontal)
%  \item Initial speed $v_1$
%  \end{itemize}
%  
%  \vspace{.25in}When solving any projectile motion problems
%  \begin{itemize}
%  \item \emph{Two} equations with \emph{two} unknowns
%  \item If an object lands on an incline, there will be a third equation
%    relating $x$ and $y$
%  \end{itemize}
%
%
%
%
%\section{Symmetric Trajectory}
%  A projectile's trajectory is \emph{symmetric} if the object lands at the same
%  height as when it launched. The angle $\theta$ is measured above the
%  horizontal. The \textbf{time of flight} ($T$), \textbf{range} ($R$)
%  and \textbf{maximum height} ($H$) are, respectively,
%
%  \eq{-.1in}{
%    \boxed{T=\frac{2v_1\sin\theta}g}\quad\quad
%    \boxed{R=\frac{v_1^2\sin(2\theta)}g}\quad\quad
%    \boxed{H=\frac{v_1^2\sin^2\theta}{2g}}
%  }
%
%
%
%
%\section{Maximum Range}
%  \eq{-.1in}{
%    R=\frac{v_1^2\sin(2\theta)}g
%  }
%  
%  \begin{itemize}
%  \item Maximum range occurs at $\theta=\ang{45}$
%  \item For a given initial speed $v_0$ and range $R$, launch angle $\theta$ is
%    given by:
%    
%    \eq{-.1in}{
%      \theta_1=\frac12\sin^{-1}\left(\frac{Rg}{v_1^2}\right)
%    }
%
%  \item But there is another angle that \emph{gives the same range}!
%
%    \eq{-.1in}{
%      \theta_2=\ang{90}-\theta_1
%    }
%  \end{itemize}
%
%%
%%
%%\section{Projectile Motion}
%%  \begin{itemize}
%%  \item For projectile motion problems, resolve the problem into horizontal
%%    ($x$) and vertical ($y$) directions, and apply kinematic equations
%%    independently
%%  \item No horizontal acceleration ($a_x=0$), therefore kinematic equations
%%    reduce to a single equation:
%%    
%%    \eq{-.3in}{\Delta x=v_x\Delta t}
%%  \item Acceleration due to gravity only in the vertical ($y$) direction:
%%    
%%    \eq{-.25in}{\bm a_y=\bm g=\magdir{\SI{9.81}{\metre\per\second^2}}{down}}
%%
%%    \vspace{-.15in}We \emph{usually} define the (+) direction to be [up], so
%%    $a_y=-g=\SI{-9.81}{\metre\per\second^2}$ %, but it can change depending on
%%    %the problem
%%  \end{itemize}
%%
%%
%%
%%\section{Solving Projectile Motion Problems}
%%  \begin{itemize}
%%  \item There are variables the two directions
%%    \begin{itemize}
%%    \item Initial speed $v_i$
%%    \item Angle above the horizontal $\theta$ (appears in the initial velocities
%%      in both horizontal and vertical directions)
%%    \item Time of motion $\Delta t$
%%    \end{itemize}
%%  \item Have two equations with two unknowns
%%  \item In more difficult problems, $\Delta y$ and $\Delta x$ can be related
%%    geometrically, (so $3$ equations with $3$ unknowns)
%%  \end{itemize}
%%
%%
%%
%%
\begin{example}
  While hiking in the wilderness, you come to a cliff overlooking a river. A
  topographical map shows that the cliff is
  \SI{291}{\metre} high and the river is \SI{68.5}{\metre} wide at that
  point. You throw a rock directly forward from the top of the cliff, giving
  the rock a horizontal velocity of \SI{12.8}{\metre\per\second}.
  \begin{enumerate}
  \item Did the rock make it across the river?
  \item With what velocity did the rock hit the ground or water?
  \end{enumerate}
  
  \begin{center}
    \pic{.5}{kinematics/graphics/cliff}
  \end{center}
\end{example}



\begin{example}
  A golfer hits the golf ball off the tee, giving it an
  initial velocity of \SI{32.6}{\metre\per\second} at an angle of \ang{65} with
  the horizontal. The green where the golf ball lands is \SI{6.30}{\metre}
  higher than the tee, as shown in the illustration. Find the time interval
  when the golf ball was in the air, and the distance to the green.
  \begin{center}
    \pic{.5}{kinematics/graphics/golfer}
  \end{center}
\end{example}



\begin{example}
  You are playing tennis with a friend on tennis courts
  that are surrounded by a \SI{4.8}{\metre} fence. You opponent hits the ball
  over the fence and you offer to retrieve it. You find the ball at a distance
  of \SI{12.4}{\metre} on the other side of the fence. You throw the ball at an
  angle of \ang{55.} with the horizontal, giving it an initial velocity of
  \SI{12.1}{\metre\per\second}. The ball is \SI{1.05}{\metre} above the ground
  when you release it. Did the ball go over the fence, hit the fence, or hit
  the ground before it reached the fence?
\end{example}



%\section{Symmetric Trajectory}
%  Trajectory is \emph{symmetric} if the object lands at the same height as
%  when it started.
%  \begin{itemize}
%  \item Time of flight
%    \eq{-.1in}{t_\text{max}=\frac{2v_i\sin\theta} g}
%  \item Range
%    \eq{-.1in}{R=\frac{v_i^2\sin(2\theta)} g}
%  \item Maximum height
%    \eq{-.1in}{h_\text{max}=\frac{v_i^2\sin^2\theta}{2g}}
%  \end{itemize}
%  The angle $\theta$ is the \textbf{above the the horizontal}



\begin{example}
  A player kicks a football for the opening kickoff. He
  gives the ball an initial velocity of \SI{29}{m/s} at an angle of \ang{69}
  with the horizontal. Neglecting friction, determine the ball's maximum height,
  hang time and range?
\end{example}
