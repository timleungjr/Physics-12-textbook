\chapter{Momentum, Impulse and Collisions}
\label{chapter:momentum}

%\newcommand{\half}{\ensuremath\frac12}
%
%
%
%\section{Momentum}
%
%\begin{frame}{A New Concept: Momentum}
%  You can't stop a bullet train or a bullet by yourself. Why not?
%  \begin{center}
%    \pic{.7}{graphics/japanese-high-speed-electric-long-train-website-header}\\
%    \pic{.7}{graphics/618520462903640723}
%  \end{center}
%
%
%
%
%\begin{frame}{A New Concept: Momentum}
%  What makes it so difficult to stop a train, or a speeding bullet, or a car?
%  \begin{itemize}
%  \item Both the train, the speeding bullet and the car have a lot of
%    \emph{momentum}
%  \item Momentum is related to both the \emph{mass} and \emph{velocity} of
%    an object
%  \item ``Mass in motion'': the tendency for the object to remain in the
%    same state of motion
%  \end{itemize}
%
%
%
%
%%\begin{frame}{Start with Inertia}
%  \begin{block}{Inertia}
%    The \emph{tendency} of a mass stay in its original state unless an external
%    force act upon it. If a mass is in motion, then it will stay in motion. If
%    a mass is at rest, then it will stay at rest. An object has inertia simply
%    because it has mass. This has nothing to do with velocity or motion.
%  \end{block}
%
%  The measurement of inertia is mass, and the SI unit is kilogram $kg$.

\section{Momentum}
\textbf{Momentum}\footnote{or more accurately, \textbf{translational
  momentum}, or \textbf{linear momentum}. For rotational motions, there is
also a similar concept called \textbf{angular momentum} which will be
studied in Chapter \ref{chapter:rotMotion}} ($\bm p$) of an object in motion
is defined as the product of the object's mass and its velocity. It represents
the amount of motion (how much is moving, and how fast) an object has:
\begin{important-equation}
  \text{Translational momentum:}\quad
  \bm p=m\bm v
\end{important-equation}
%    \begin{tabular}{l|c|c}
%      \rowcolor{pink}
%      \textbf{Quantity} & \textbf{Symbol} & \textbf{SI Unit} \\ \hline
%      Momentum          & $\bm p$        & \si{\kilo\gram\metre\per\second}\\
%      Mass              & $m$             & \si{\kilo\gram} \\
%      Velocity          & $\bm v$        & \si{\metre\per\second}
%    \end{tabular}
%  \end{center}
Momentum is a \emph{vector}; the direction of the momentum vector is the
direction of the velocity vector. Curiously, momentum does not have a special
unit; the SI unit for momentum is just \emph{kilogram metre per second}
(\si{\kilo\gram\metre\per\second})---the unit for mass times the unit for
velocity.
%The SI unit of momentum is
%\textbf{kilogram metre per second} (\si{\kilo\gram\metre\per\second}).


\begin{example}
  Determine the magnitude of momentum of a \SI{170}{\gram} hockey puck
  traveling across the ice at a velocity of \SI{15.5}{\metre\per\second}.
\end{example}

\section{Impulse}
When a force is applied to the an object over a time interval, an
\textbf{impulse} $\bm J$ is generated by that force. For a constant or average
force $\bm F$, the impulse generated over the time interval $\Delta t$ is
defined as:
\begin{important-equation}
  \text{Impulse:}\quad
  \bm J=\bm F\Delta t
\end{important-equation}
%  \begin{center}
%    \begin{tabular}{l|c|c}
%      \rowcolor{pink}
%      \textbf{Quantity} & \textbf{Symbol} & \textbf{SI Unit} \\ \hline
%      Impulse           & $\bm J$    & \si{\newton\second}\\
%      Average force     & $\bm F$    & \si\newton \\
%      Time interval     & $\Delta t$  & \si\second
%    \end{tabular}
%  \end{center}
Impulse $\bm J$ is a vector in the same direction as the force vector. It 
depends on the time that the force is applied, but not necessarily on the
motion of the object. The SI unit for impulse is a \emph{newton second}
(\si{\newton\second}). Like all vectors, $\bm J$ can be evaluated separately
by components:
\begin{equation}
  \bm J = J_x\hat{\bm x} + J_y\hat{\bm y} + J_z\hat{\bm z}
  =
  (F_x\Delta t)\hat{\bm x}
  + (F_y\Delta t)\hat{\bm y}
  + (F_z\Delta t)\hat{\bm z}
\end{equation}  

\begin{remark}
  For those of you with a background in integral calculus, for non-constant
  forces, impulse is given by the integral:
  \begin{equation*}
    \bm J=\int_{t_1}^{t_2}\bm F\dl t
  \end{equation*}
\end{remark}


\textbf{Impulse Generated by a Specific Force}
\begin{itemize}
\item Every force acting on an object generates an impulse
\item Can calculate the impulse generated by each force over the time interval
\end{itemize}

\textbf{Net Impulse on a Object}
\begin{itemize}
\item The vector sum of all the impulses on an object
\item The impulse generated by the net force, i.e.
  \begin{equation}
    \boxed{
      \bm J_\text{net} = \bm F_\text{net}\Delta t
    }
  \end{equation}
\end{itemize} 

\begin{example}
  Consider a box being pushed up a ramp. The impulses on the box by each of
  the forces can be calculated over a time interval $\Delta t$ of interest:
  \begin{center}
    \begin{tikzpicture}[rotate=30]
      \draw[thick] (0,0)--+(3,0);
      \draw[thick,rotate=-30] (0,0)--+(3,0);
      \draw[mass] (1,0) rectangle +(1,1);
      \fill[red] (1.5,.5) circle (.07);
      \draw[vectors,red] (1.5,.5)--+(1.5,0) node[right]{$\bm F_A$};
      \draw[vectors,red] (1.5,.5)--+(-1,0) node[left]{$\bm f_k$};
      \draw[vectors,red] (1.5,.5)--+(0,1.2) node[left]{$\bm F_N$};
      \draw[vectors,red,rotate around={-30:(1.5,.5)}]
      (1.5,.5)--+(0,-1) node[right]{$\bm F_g$};
    \end{tikzpicture}
  \end{center}
  \begin{itemize}
  \item Impulse from applied force: $\bm J_A =\bm F_A\Delta t$
  \item Impulse from kinetic friction: $\bm J_f =\bm f_k\Delta t$
  \item Impulse from normal force: $\bm J_N =\bm F_N\Delta t$
  \item Impulse from gravity: $\bm J_g =\bm F_g\Delta t$
  \end{itemize}
  
  The net impulse is the sum of all the impulses acting on the object:
  \begin{align*}
    \bm J_\text{net} &=\sum\bm J_i=\bm J_A+\bm J_f +\bm J_N + \bm J_g\\
    &=(\bm F_A+\bm f_k+\bm F_N+\bm F_g)\Delta t=\bm F_\text{net}\Delta t
  \end{align*}
\end{example}


\subsection{Impulse Generated by Non-Constant Force}


\section{First \& Second Laws of Motion}
We can relate the net force to the change in momentum using the second law
of motion.
\begin{equation}
  \bm F_\text{net}=m\bm a
  =m\frac{\Delta\bm v}{\Delta t}
  =\frac{\Delta(m\bm v)}{\Delta t}
  =\frac{\Delta\bm p}{\Delta t}
\end{equation}

This leads to the the general form of the \emph{first} law of motion:
\begin{definition}
  \textbf{Law \#1: The net momentum of an object or a system objects remains
    constant until a net external force is applied to the system:}
  \begin{equation}
    \bm F_\text{net}= \sum_i\frac{\Delta p_i}{\Delta t}=0
    \quad\longleftrightarrow\quad
    \sum_i\bm p_i=\text{constant}
  \end{equation}
\end{definition}
where as in the second law of motion:
\begin{definition}
  \textbf{Law \#2: The net force on an object is equal to the rate of change of
    its momentum}:
  \begin{equation}
    \bm F_\text{net}=\frac{\Delta\bm p}{\Delta t}
  \end{equation}
\end{definition}
The general form is necessary when mass changes with time.

Multiplying both sides of the second law of motion by $\Delta t$, we can see
that: net impulse $\bm J_\text{net}=\bm F_\text{net}\Delta t$ is equal to the
change in momentum $\Delta\bm p$. This is called the \textbf{momentum-impulse
  principle}:
\begin{important-equation}
  \text{Momentum-impulse principle:}\quad
  \bm J_\text{net}=\Delta\bm p=\bm p_2-\bm p_1
\end{important-equation}


\begin{example}[Golf Club]
  If a golf club exerts an average force of \SI{5.25e3}{\newton} [W] on a golf
  ball over a time interval of \SI{5.45e-4}\second, what is the impulse of the
  interaction?
\end{example}
%
%
%
%\begin{frame}{Why Average Force?}

%    \pic1{graphics/impulse1}
%
%    Why did the previous example use \textbf{average force}?




\begin{example}[A Slightly Longer Example]
  A student practices her tennis volleys by hitting a tennis ball against a
  wall.
  \begin{itemize}
  \item If the \SI{.060}{\kilo\gram} ball travels \SI{48}{\metre\per\second}
    before hitting the wall and then bounces directly backward at
    \SI{35}{\metre\per\second}, what is the impulse of the interaction?
  \item If the duration of the interaction if \SI{25}{\milli\second}, what is
    the average force exerted on the ball by the wall?
  \end{itemize}
\end{example}


\section{Momentum and Kinetic Energy}
We can express kinetic energy ($K$) in terms of the magnitude of momentum
($p=|\bm p|$):
\begin{equation}
  K =\frac12mv^2 =\frac12\left[\frac{m^2}m\right]{v^2}
\end{equation}

Noting that $m^2v^2=p^2$, we have:
\begin{important-equation}
  K=\frac{p^2}{2m}
  \label{k-and-p}
\end{important-equation}

Eq.~\ref{k-and-p} show that whenever momentum changes in \emph{magnitude},
kinetic energy also changes. But we must also recognize that kinetic energy is
a scalar quantity, while momentum $\bm p$ is the vector, therefore a momentum
vector that changes only in \emph{direction} does not change in kinetic energy,
which means that a net force may impart a net impulse ($\bm J_\text{net}$) on
the object, while not doing any net work ($W_\text{net}$) on it.



\section{Conservation of Momentum}

We begin the study of collisions by considering two objects---let's call them
objects 1 and 2 with masses $m_1$ and $m_2$---colliding in
isolation\footnote{This means that the collision is an isolated system. The
concept of an isolated system is discussed in dynamics, and also in
conservation of energy. In an isolated system, there are no external forces,
and therefore no external work can be done to the system. By the law of
conservation of energy $\Delta E_\text{system}=W_\text{ext}$, the total energy
of the system must remain constant}, as shown in Fig.~\ref{collision1}.
\begin{figure}[h]
  \centering
  \begin{tikzpicture}[scale=1.3]
    \draw[thick,fill=lightgray] circle (.3);
    \draw[axes,gray] (-1.5,0)--(-.4,0);
    \draw[gray,fill=gray!20,dash dot] (-1.8,0) circle (.3) node{$m_1$};
    \begin{scope}[rotate=-35]
      \draw[thick,fill=lightgray] (.58,0) circle (.3);
      \draw[axes,gray] (2,0)--(1.4,0);
      \draw[gray,fill=gray!20,dash dot] (2.3,0) circle (.3) node{$m_2$};
      \draw[ultra thick] (.29,.1)--(.29,-.1);
      \draw[vectors,magenta] (.29,0)--+(1,0) node[pos=1.2]{$F_n$};
      \draw[vectors,red] (.29,0)--+(-1,0) node[pos=1.2]{$F_n$};
      \draw[vectors,blue](.27,0)--+(0,.7) node[pos=1.2]{$F_f$};
      \draw[vectors,cyan](.31,0)--+(0,-.7) node[pos=1.2]{$F_f$};
    \end{scope}
  \end{tikzpicture}
  \caption{Collision of two objects}
  \label{collision1}
\end{figure}
For simplicity, we will assume that the forces the objects exert on each other
during the collision do not cause them to rotate.
\begin{remark}
  There is a good reason that we make the assumption that the collision does
  not cause additional rotation. A rotating object contains additional kinetic
  energy.%, we will neglect that for simplicity.
  This means that the forces do not generate torques on the
  objects. Alternatively, we can also think of objects with very small moment
  of inertia, so that the rotational kinetic energy is negligible.
\end{remark}

The forces that the objects exert on each other may include, but not limited to:
\begin{itemize}[itemsep=4pt]
\item Normal force
\item Static friction
\item Kinetic friction
\item Spring force
\item Gravitational force
\item Electrostatic force (if the objects are charged)
\item Magnetic force (if the objects are magnetic)
\end{itemize}
If we assert that the third law of motion (Section ~\ref{sec:3rd-law}) is
valid, then the forces that objects 1 and 2 exert on each other
(the ``action and reaction pair of forces'') must be equal in magnitude and
opposite in direction. Since the individual forces exerted on objects 1 and 2
are equal in magnitude and opposite in direction, both objects experience a net
force that is also equal in magnitude and opposite in direction:
\begin{equation}
  \bm F_1=-\bm F_2
  \label{net-force}
\end{equation}
Since the forces are exerted on both objects over the same time interval, the
\emph{net} impulse on $m_1$ and $m_2$ are also equal and opposite:
\begin{equation}
  \bm J_1=-\bm J_2
  \label{net-impulse}
\end{equation}
Finally, as net impulse is equal to the change in an object's momentum by the
momentum-impulse principle ($\bm J_\text{net}=\Delta\bm p$), the change in
momentum on objects 1 and 2 must also be equal in magnitude and opposite in
direction:
\begin{equation}
  \Delta\bm p_1=-\Delta \bm p_2
  \label{equal-opposite-p}
\end{equation}
We can rewrite Eq.~\ref{equal-opposite-p} to say that \emph{the total change in
momentum of the system is zero}:
\begin{equation}
  \Delta\bm p_1+\Delta\bm p_2=\bm 0
\end{equation}  
In other words, the total momentum of the system is conserved during a
collision. For a collision of $N$ objects, the net, or total, momentum before
the collision is the same as the net momentum after the collision:
\begin{important-equation}
  \sum_{i=1}^N\bm p_i=\sum_{i=1}^N\bm p_i'
\end{important-equation}
For a collision (or explosion) of two objects, momentum is conserved at the
time of interaction:
\begin{equation}
  m_1\bm v_1+m_2\bm v_2=m_1\bm v_1'+m_2\bm v_2'
  \label{eq:momentum-same}
\end{equation}
It's important to note that since momentum is a vector, the collision
%/explosion
problem \emph{must} be solved using vector arithmetic.

%  \begin{center}
%    \begin{tabular}{l|c|c}
%      \rowcolor{pink}
%      \textbf{Quantity} & \textbf{Symbol} & \textbf{SI Unit} \\ \hline
%      Masses of the objects & $m_1$, $m_2$ & \si{\kilo\gram} \\
%      Initial velocities & $\bm v_1$, $\bm v_2$ &
%      \si{\metre\per\second} \\
%      Final velocities & $\bm v_1'$, $\bm v_2'$ &
%      \si{\metre\per\second}
%    \end{tabular}
%  \end{center}




\section{Third Law of Motion}

Of course, that momentum is conserved during a collision is to be expected. In
fact, the derivation of the conservation of momentum gives us insight into the
nature of third law of motion.

Here, in working out the conservation of momentum, we began with the third law
of motion, and arriving at the first law of motion. But how did we know that
the action and reaction forces are equal and opposite in the first place? We
know that because of the first law of motion. In other words, the third law of
motion must be true because the first law of motion is true. Hence the third
law of motion is an application of the first law of motion.


On first glance, the first law of motion deals with something completely
different. But here, we see that when a system of object is isolated (i.e.\
there are no \emph{external} forces), the total momentum of the system is
constant. The forces that the two objects ($m_1$ and $m_2$) apply on each other
are \emph{internal} to the system.

Since the net momentum is constant, any change in the momentum of $m_1$ must be
offset by the change in momentum in $m_2$ (Eq.~\ref{equal-opposite-p}). Since,
by the momentum-pulse principle,  the change in momentum is equal to the net
impulse, $m_1$ and $m_2$ must experience equal and opposite net impulses
(Eq.~\ref{net-impulse}). And since both $m_1$ and $m_2$ experience forces over
the same time interval, the \emph{net} force on $m_1$ and $m_2$ are
also equal and opposite (Eq.~\ref{net-force}). If the net force on $m_1$ and
$m_2$ are equal and opposite, it is necessary that any forces that the objects
exert on each exert onto each other must be equal and opposite. In other words,
\textbf{the third law of motion must be true because it is an application of
   the 1st law of motion}.


%\textbf{Conservation of momentum} is derived through the third law of motion.
%Consider the interaction of two objects in isolation (let's call them objects A
%and B), they apply forces on each other. These forces may include, but not
%limited to:
%If we assert that the third law of motion (Section ~\ref{sec:3rd-law}) is valid,
%the forces that they exert on each other must be equal in magnitude and
%opposite in direction; both objects experience a net force that is also equal
%in magnitude and opposite in direction:
%\begin{equation}
%  \bm F_A = -\bm F_B
%\end{equation}
%Both objects experiences these forces over the same time interval, therefore
%the net impulse on object A ($\bm J_A$) is also equal in magnitude and opposite
%in direction to the net impulse of object B ($\bm J_B$), i.e.
%\begin{equation}
%  \bm J_A = -\bm J_B
%\end{equation}
%Consequently, the change in momentum in object A is equal and opposite to that
%of object B:
%\begin{equation}
%  \Delta\bm p_A = -\Delta\bm p_B
%\end{equation}
%Therefore, in an isolated system that consists only of objects A and B, the net
%(total) change in the system's momentum is zero:
%\begin{equation}
%  \Delta\bm p_B + \Delta\bm p_B = 0
%  \label{eq:no-momentum-change}
%\end{equation}
%In other words, momentum is conserved. Of course, this is to be expected, since
%the general form of the \emph{first} law of motion shows that momentum must be
%conserved in the absence of a net force.
%
%When we solve problems, we can rewrite Eq.~\ref{eq:no-momentum-change} as
%\begin{important-equation}
%  \text{Conservation of momentum:}\quad
%  \sum_i\bm p_i = \sum_i\bm p_i'
%\end{important-equation}
%We can apply Eq.~\ref{eq:momentum-same} to both collision and explosion
%problems. (Explosion are really just collisions in reverse.)
%
%\footnote{which itself is an application of the first law of motion}.
%When objects interact in isolation\footnote{Think of the isolated systems
%discussed in the previous class}, the total momentum \emph{before} the
%interaction is the same as \emph{after} the interaction:
%
%  \eq{-.13in}{
%    \boxed{
%      \sum_i\bm p_i=\sum_i\bm p_i'
%    }
%  }
%
%  \vspace{-.1in}Examples:
%  \begin{itemize}
%  \item Collision of two or more objects
%  \item A rocket expelling gas from the engine nozzle
%  \item Skaters pushing on each other on a friction-less ice surface
%  \item An exploding bomb
%  \end{itemize}
%
%
%
%
%\begin{frame}{Conservation of Momentum}
%  In Grade 12 Physics, we will limit our discussions primarily to only
%  2 objects. For a collision or explosion between two objects $A$ and $B$:
%
%  \eq{-.1in}{
%    \boxed{m_A\bm v_A+m_B\bm v_B=m_A\bm v_A'+m_B\bm v_B'}
%  }
%  \begin{center}
%    \begin{tabular}{l|c|c}
%      \rowcolor{pink}
%      \textbf{Quantity}      & \textbf{Symbol} & \textbf{SI Unit} \\ \hline
%      Masses of $A$ and $B$             & $m_A$, $m_B$ & \si{\kilo\gram} \\
%      Initial velocities of $A$ and $B$ & $\bm v_A$, $\bm v_B$ &
%      \si{\metre\per\second} \\
%      Final velocities of $A$ and $B$   & $\bm v_A'$, $\bm v_B'$ &
%      \si{\metre\per\second} \\
%    \end{tabular}
%  \end{center}



\begin{example}
  A \SI{1.75e4}{\kilo\gram} boxcar is rolling down a track towards a stationary
  boxcar that has a mass of \SI{2.00e4}{\kilo\gram}. Just before the collision,
  the first boxcar is moving east at \SI{5.45}{\metre\per\second}. When the
  boxcars collide, they lock together and continue to down the track. What is
  the velocity of the two boxcars immediately after the collision?
  \begin{center}
    \pic{.45}{momentum/graphics/boxcars}
  \end{center}
\end{example}



%\begin{example}
%  Two people A and B stand in a canoe on top of the water.
%  Find the velocity of the canoe and person B at the instant that person A
%  start to take a step, if her velocity is \SI{.75}{\metre\per\second}
%  [forward]. Assume person A has a mass of 65 kg and the combined mass of the
%  canoe, A and B, is \SI{115}{\kilo\gram}.
%  \begin{center}
%    \pic{.45}{graphics/canoe}
%  \end{center}
%\end{example}



%  What happens if I let go of this balloon?
%  \begin{center}
%    \pic{.5}{graphics/Balloons4}
%  \end{center}
%  We will talk about this more in Unit 3 when we talk about propulsion in space


\subsection{Glancing Collisions}
A glancing collision involves motion in 2D. Since momentum is a vector, the
calculation involves some vector arithmetic.
  
\begin{example}[Billiard Ball Problem]
  \label{eg:billiard}
  A billiard ball of mass \SI{.17}{\kilo\gram} (``cue ball'') moves with a
  velocity of \SI{1.3}{\metre\per\second} towards a stationary billiard ball
  (``eight ball'') of mass \SI{.16}{\kilo\gram}, and strikes it with a
  glancing blow. The cue ball moves off at an angle of \ang{30} clockwise
  from its original direction, with a speed of \SI{.96}{\metre\per\second}.
  What are the magnitude and direction of the final velocity of the eight ball?
  \begin{center}
    \begin{tikzpicture}[scale=.9]
      \draw[dashed] (-3,0)--(0,0);
      \draw[vector] (-3,0)--+(1.5,0) node[above]{$\bm v_c$};
      \shade[balloon1] (-3,0) circle (.3) node[above=7]{$m_c$};
      \shade[balloon2] circle (.3) node[above=7]{$m_8$};
      \begin{scope}[rotate=-30]
        \draw[dashed] (0,0)--+(2,0);
        \draw[vector] (2,0)--+(1.2,0) node[pos=1.15]{$\bm v_c'$};
        \shade[balloon1] (2,0) circle (.3) node[above=7]{$m_c$};
      \end{scope}
       \begin{scope}[rotate=40]
        \draw[dashed] (0,0)--+(2,0);
        \draw[vector] (2,0)--+(1.2,0) node[pos=1.2]{$\bm v_8'=?$};
        \shade[balloon2] (2,0) circle (.3) node[above=7]{$m_8$};
      \end{scope}
    \end{tikzpicture}
  \end{center}
  
  \textbf{Solution:} Since this is \emph{clearly} a two dimensional problem,
  the most straightforward approach is to split the conservation of momentum
  problem into $x$ and $y$ components. We start by defining the $x$ axis to be
  along the original direction of the cue ball's motion. From the conservation
  of momentum equation, we separate the momentum vectors $\bm p_c$ and
  $\bm p_c'$ into components. Here we are writing the vectors in matrix form.
  The top row is the $x$ component; the bottom row is the $y$ component:
  \begin{displaymath}
    \bm p_c =\bm p_c' + \bm p_8'\quad\longrightarrow\quad
    \left[\begin{matrix}
        m_cv_c\\0
      \end{matrix}\right]
    =
    \left[\begin{matrix}
        m_cv_c'\cos\theta\\
        -m_cv_c'\sin\theta
      \end{matrix}\right]
    + m_8\bm v_8'
  \end{displaymath}
  %Note that the $y$ component for $\bm p_c'$ is negative because after the
  %collision, its $y$ component is along the $-y$ direction.
  Solving for $\bm v_8'$ (move all the terms to the other side of the equation,
  then factor out the $m_c$ terms, and then divide by $m_8$), we get:
  \begin{displaymath}
    \bm v_8'=\frac{m_c}{m_8}
    \left[\begin{matrix}
        v_c-v_c'\cos\theta\\
        v_c'\sin\theta
      \end{matrix}\right]
    =\frac{0.165}{0.155}    
    \left[\begin{matrix}
        1.25-0.956\cos(\ang{29.7})\\
        0.956'\sin(\ang{29.7})
      \end{matrix}\right]
    =
    \left[\begin{matrix}
        0.44666\\
        0.50422
      \end{matrix}\right]
    \si{\metre\per\second}
  \end{displaymath}
  Now we can find the magnitude and direction of the vector:
  
  \textbf{Alternate Solution:} We can also solve this problem by
  ``solving a triangle''. We recognize than the conservation of momentum
  gives us a triangle like this:
  \begin{center}
    \begin{tikzpicture}[vectors,scale=2]
      \draw (0,0)--(2,0) node[midway,above]{$\bm p_c$};
      \draw (0,0)--(1.3,-.75) node[midway,below left]{$\bm p_c'$};
      \draw (1.3,-.75)--(2,0) node[midway,below right]{$\bm p_8'$};
      \draw[axes] (.7,0) arc (0:-30:.7) node[midway,right]{$\theta=\ang{30}$};
    \end{tikzpicture}
  \end{center}
  The magnitude of the vectors $\bm p_c$ and $\bm p_c'$ can be calculated
  using given quantities:
  \begin{align*}
    p_c & = m_cv_c=(0.165)(1.25)=\\
    p_c' & = m_cv_c'=(0.165)(0.956)=
  \end{align*}
  The magnitude of $\bm p_8'$ can easily be calculated using the cosine law:
  \begin{displaymath}
    p_8'=\sqrt{p_c^2+p_c'^2-2p_cp_c'\cos\theta}=
  \end{displaymath}
  And the magnitude of velocity of the eight-ball can be obtained by dividing by
  the mass of the eight-ball:
  \begin{displaymath}
    v_8'=\frac{p_8'}{m_8}=
  \end{displaymath}
  Which, not surprisingly, is the same as in our first solution!
\end{example}

\section{A Full View of Collision}
There are two types of collisions: \textbf{elastic collision} and
\textbf{inelastic collision}, which depends on what kinds of forces the objects
apply to each other during the collision. To better understand the similarities
and differences between elastic and inelastic collisins, we must first review
the properties of conservative and non-conservative forces, as related to the
conservation of mechanical energy.

For conservative forces:
\begin{itemize}
\item There is a related potential energy. For example, gravitational force
  ($\bm F_g$) is related to gravitational potential energy ($U_g$); spring
  force ($\bm F_s$) is related to the elastic potential energy ($U_e$)
\item Positive work by a conservative force transforms potential energy into
  kinetic energy by the same amount, while negative work transforms kinetic
  energy into potential energy by the same amount.
\item Work done by a conservative force is path independent.
\item The list of conservative forces is \emph{very} short. It includes
  \begin{itemize}[nosep]
  \item Gravitational force
  \item Spring force from a stretched/compressed spring\footnote{Spring
  force is only considered to be conservative when applied to \emph{ideal}
  springs that obeys Hooke's law ($\bm F_s=-k\bm x$) perfectly.}
  \item Electric force, between charged particles
  \item Magnetic force, between magnets and moving charged particlesp
  \item Nuclear forces
  \end{itemize}
\end{itemize}
In contrast, for non-conservative forces:
\begin{itemize}
\item There is no related potential energy. Work done by a non-conservative
  force always transforms one form of kinetic energy into another, or from the
  kinetic energy of one object to another, or transforms kinetic energy into
  thermal energy.
\item Work done by a non-conservative force is generally path dependent.
\end{itemize}
When two objects collide in isolation (i.e.\ an isolated system with no
external work $W_\text{net}=0$), by the law of conservation of
energy,
%\footnote{In classical mechanics, the law of conservation of energy is
%most properly expressed as \emph{the change in the total energy of a system is
%equal to the net external work done to the system}:
%\begin{displaymath}
%  \Delta E_\text{system}=W_\text{ext}
%\end{displaymath}
%The consequence of the law of conservation energy is that well-known statement
%that \emph{energy cannot be created or destroyed; it can only change in form}.
%},
the total system energy is conserved. Within this isolated system, the
forces that objects 1 and 2 apply to each other do work to each other:
\begin{itemize}
\item Work done by conservative forces transforms kinetic energies into
  potential energies, and then from potential back to kinetic energies, while
\item Wokr done by non-conservative forces transform kinetic energies into
  thermal energies, and/or other forms of kinetic energies
\end{itemize}
%Elastic collisions
%\begin{itemize}
%\item Momentum is conserved
%\item Kinetic energy is conserved
%    %\begin{itemize}
%    %\item During the collision, kinetic energy is first transformed into a
%    %  potential energy (e.g.\ compressing a spring), and then 
%    %\item All of the energy is then released back as kinetic energy
%    %\end{itemize}
%\item Often the two objects don't actually make
%  contact with each other
%\item\textbf{You must NOT assume that a collision is elastic unless you are
%  told!}
%\end{itemize}
%

\subsection{Elastic Collisions}
If the forces that two objects apply to each other are entirely conservative
(or if there are non-conservative forces, but they do not do work), then,
during the collision,
\begin{itemize}[itemsep=6pt]
\item At the start of the collision, when the two objects come together,
  conservative forces do negative work: kinetic energy of the objects is
  transformed into potential energies
\item At the end of the collision, when the objects separate from each
  other, conservative forces do positive work: potential energies stored
  between the objects are transformed back into kinetic energy
\end{itemize}

When the masses collide, kinetic energy is transformed into elastic potential
energy:
\begin{figure}[ht]
  \centering
  \begin{tikzpicture}
    \draw[thick,fill=magenta!30] (-1.8,0) circle (.3) node{$m_1$};
    \draw[thick,
      decoration={aspect=.5,segment length=3, amplitude=3, coil},
      decorate] (-1.5,0)--+(.8,0);
    \draw[vectors] (-.7,0)--+(1,0) node[midway,above]{$\bm v_1$};
    
    \draw[mass] (2.3,0) circle (.3) node{$m_2$};
    \draw[vectors] (2,0)--+(-1,0) node[midway,above]{$\bm v_2$};
  \end{tikzpicture}
  \caption{Kinetic energy is converted into elastic potential energy during an
    elastic collision}
\end{figure}
When charged particles collide, kinetic energy is transformed into electrical
potential energy:
\begin{figure}[ht]
  \centering
  \begin{tikzpicture}
    \draw[thick,fill=magenta!30] (-1.8,0) circle (.3) node{$q_1$};
    \draw[vectors] (-1.5,0)--+(1,0) node[midway,above]{$\bm v_1$};
    
    \draw[mass] (2.3,0) circle (.3) node{$q_2$};
    \draw[vectors] (2,0)--+(-1,0) node[midway,above]{$\bm v_2$};
  \end{tikzpicture}
  \caption{Kinetic energy is converted into electric potential energy in the
  elastic collision of two electrically charged particles}
\end{figure}
When the masses ``collide'', kinetic energy is transformed
into gravitational potential energy.
\begin{figure}[ht]
  \centering
  \begin{tikzpicture}[thick,scale=1.2]
    \draw (1,0)--+(6,0);
    \draw[fill=magenta!30] (2,0)--(2,.5)--(2.5,.5) to[out=270,in=10] (2,0);
    \draw[vectors] (2.43,.25)--+(1,0) node[midway,above]{$\bm v_1$};
    
    \draw[mass] (4,0) to[out=10,in=270] (5.5,1)--(6,1)--(6,0)--(4,0);
    \node at (5.7,.5) {$m_2$};
    \node at (2.2,.3) {$m_1$};
    \draw[vectors] (5.05,.3)--+(-1,0) node[midway,above]{$\bm v_2$};
  \end{tikzpicture}
  \caption{Kinetic energy is converted into gravitational potential energy in
    the elastic ``collision'' of two masses}
\end{figure}


%\section{Elastic vs.\ Inelastic Collisions}
%
%\begin{frame}{A Full View of Collision}
%  There are two types of collisions: \textbf{elastic collision} and
%  \textbf{inelastic collision}
%
%  \vspace{.2in}Elastic collisions
%  \begin{itemize}
%  \item Momentum is conserved
%  \item Kinetic energy is conserved
%    \begin{itemize}
%    \item During the collision, kinetic energy is first transformed into a
%      potential energy (e.g.\ compressing a spring), and then 
%    \item All of the energy is then released back as kinetic energy
%    \end{itemize}
%  \item Often the two objects don't actually make
%    contact with each other
%  \item\textbf{You must NOT assume that a collision is elastic unless you are
%    told!}
%  \end{itemize}
%
%
%
%\begin{frame}{Collisions}
%  Inelastic collisions (the majority of all collisions)
%  \begin{itemize}
%  \item Momentum is conserved
%  \item Kinetic energy is lost due to heat, friction or sound
%  \item Energy is conserved to motion just before and/or just after the
%    collision (depends on the situation)
%  \end{itemize}
%
%  \vspace{.15in}A special case of inelastic collision is a \textbf{completely
%    inelastic collision}\footnote{also known as a \textbf{perfectly inelastic
%    collision}} where the objects stick together after the collision
%  \begin{itemize}
%  \item Momentum is conserved
%  \item Most of the kinetic energy is lost
%  \end{itemize}
%


%\section{Elastic Collision}
%
%In an \textbf{elastic collision}, in addition to the conservation of momentum,
%we also require that kinetic energy be conserved before and after the
%collision:
%\begin{equation}
%  \sum K_i = \sum K_i'
%\end{equation}
%Kinetic energy can only be conserved if the work done on the objects during
%the collision are entirely by conservative forces. This can happen when,
%during the collision,
%\begin{itemize}[itemsep=4pt]
%\item The object apply \emph{only} conservative forces to each other.
%\item The objects apply both conservative and non-conservative forces to each
%  other, but only conservative forces do work.
%\end{itemize}
%These potential energies
%that are stored are based on the conservative forces that the objects apply to
%each other. They can be:
%\begin{itemize}[itemsep=4pt]
%\item\emph{elastic} (e.g.\ a spring that compresses when the two objects
%  collide),
%\item\emph{electric} (e.g.\ two charged particles moving towards each other),
%\item\emph{gravitational} (e.g.\ gravitational sling shots for satellites and
%  deep-space vehicles), or
%\item\emph{magnetic} (e.g.\ north poles of two magnets are repelled from each
%  other)
%\end{itemize}
%When two objects interact in isolation, the total kinetic energies before and
%after the interaction is written as:
%%The result is that the total kinetic
%%energies of both objects before the collision is the same as the total
%%afterwards:
%\begin{equation}
%  \frac12m_1v_1^2 + \frac12m_2v_2^2 = \frac12m_1v_1'^2+\frac12 m_2v_2'^2
%  \label{eq:K1}
%\end{equation}
%Note that unlike Eq.~\ref{eq:momentum-same}, which is a vector equation,
%Eq.~\ref{eq:K1} is a scalar equation.
%
%%
%An \textbf{elastic collision} is a special case in which both \emph{momentum}
%and \emph{kinetic energy} are conserved. Here, we will derive the equations for
%\emph{one-dimension} elastic collisions between \emph{two} objects.
%, as shown in Fig.~\ref{fig:1d}.
%\begin{figure}[ht]
%  \centering
%  \begin{tikzpicture}[scale=.8]
%    \tikzstyle{balloon1}=[ball color=red];
%    \tikzstyle{balloon2}=[ball color=blue];
%    \shade[balloon1](0,0) circle(1) node[white]{$m_1$};
%    \draw[very thick,->](1,0)--(2.5,0) node[midway,above]{$v_1$};
%    \shade[balloon2](7,0) circle(.6) node[white]{$m_2$};
%    \draw[very thick,->](6.4,0)--(5.2,0) node[midway,above]{$v_2$};
%  \end{tikzpicture}
%  \caption{Collision of two objects in one dimension.}
%  \label{fig:1d}
%\end{figure}
%
%conservation of momentum equation can be expressed as:
%\begin{equation}
%  m_1\bm v_1+m_2\bm v_2=m_1\bm v_1'+m_2\bm v_2'
%  \label{eq:mom1}
%\end{equation}
%where $m_1$ and $m_2$ are the masses of objects 1 and 2, $v_1$ and $v_2$ are,
%respectively, the initial velocities of the masses, and $v_1'$ and $v_2'$ are
%their final velocities.
%
%\textbf{Kinetic energy is conserved in an elastic collision} from the fact that
%during the collision, conserative forces are present to transform kinetic
%energy into  potential energy. 
%
%In most cases, the objects never touch each other. During the collision, these
%conservative forces transform \emph{all} the potential energy back into kinetic
%energy, and at the end of the interaction, those same conservative fores
%release the energy back as kinetic energy.
%\begin{remark}
%  One of the most common comments by students is that Eq.~\ref{eq:K1} is the
%  \emph{same} as Eq.~\ref{eq:momentum-same} (i.e.\ they are always satisfied
%  together). Clearly this is not the case! The momentum equation is
%  \emph{linear} in $\bm v$, while kinetic energy is \emph{quadratic} in $v$.
%\end{remark}



\subsubsection{Derivation of One-Dimensional Elastic Collision Equations}
To derive the equation for one-dimensional elastic collisions, we collect
all the $m_1$ and $m_2$ terms in Eq.~\ref{eq:momentum-same} together. First,
we collect all the $m_1$ term on the left, and the $m_2$ terms on the right.
After factoring the mass terms, Eq.~\ref{eq:momentum-same} becomes:
\begin{equation}
  m_1(v_1-v_1')=m_2(v_2'-v_2)
  \label{eq:mom2a}
\end{equation}
We can repeat the same process for kinetic energy in Eq.~\ref{eq:K1}, by
cancelling the $\frac12$ factors from every term, and then collect all the
$m_1$ term on the left, and the $m_2$ terms on the right. After factoring the
mass terms, Eq.~\ref{eq:K1} becomes:
\begin{equation}
  m_1(v_1^2-v_1'^2) = m_2(v_2'^2-v_2^2)
  \label{eq:K2a}
\end{equation}
The only difference between Eq.~\ref{eq:mom2a} and Eq.~\ref{eq:K2} is that
speed term in the kinetic energy is squared but not in the momentum equation.
Noting that the speed terms on both sides of the equation are just difference
of two squares\footnote{Remember the difference of squares can be factored as:
$(a^2-b^2)=(a+b)(a-b)$ in case you have forgotten. I sympathize with you if
you have indeed forgotten: there are so many things to remember from so many
different courses!}, we can factor the terms in Eq.~\ref{eq:K2a} into:
\begin{equation}
  m_1(v_1-v_1')(v_1+v_1') = m_2(v_2'+v_2)(v_2'-v_2)
  \label{eq:K2}
\end{equation}
When we divide Eq.~\ref{eq:K2} by Eq.~\ref{eq:mom2a}:
\begin{equation*}
  \frac{m_1(v_1-v_1')(v_1+v_1')}{m_1(v_1-v_1')}
  =
  \frac{m_2(v_2'+v_2)(v_2'-v_2)}{m_2(v_2'-v_2)}
\end{equation*}
The mass terms ($m_1$ and $m_2$), as well as velocity terms ($v_1-v_1'$ and
$v_2'-v_2$) cancel, leading to a very simple relationship between the initial
and final velocities of objects 1 and 2. We can rearrange terms to express
the final velocities $v_1'$ and $v_2'$ in terms of other velocities:
\begin{align}
  \nonumber
  v_1+v_1' &= v_2'+v_2\\
  v_2'&=v_1+v_1'-v_2\quad\text{(solving for $v_2'$)}\label{eq:relvel1}\\
  v_1'&=v_2+v_2'-v_1\quad\text{(solving for $v_1'$)}\label{eq:relvel}
\end{align}
To solve for the final velocity of object 1, we substitute Eq.~\ref{eq:relvel1}
back into Eq.~\ref{eq:mom2a}. This time, we collect all $v_1$, $v_2$ and $v_1'$
terms together:
\begin{align}
  \nonumber
  m_1(v_1-v_1')&=m_2(v_1+v_1'-v_2-v_2)\\
  \nonumber
  m_1v_1-m_1v_1'&=m_2v_1+m_2v_1'-2m_2v_2\\
  (m_1-m_2)v_1+2m_2v_2&=(m_2+m_1)v_1'
  \label{eq:mom3}
\end{align}
By dividing every term in Eq.~\ref{eq:mom3} by the total mass ($m_2+m_1$), we 
obtain the algebraic expression for the final velocity of object 1 after the
elastic collision. Likewise, substituting Eq.~\ref{eq:relvel} back into
Eq.~\ref{eq:mom2a} and following the same procedure, we obtain the algebraic
expression for the final velocity of object 2 after the collision. The two
equations are shown together here.  Even if the derivation had been
uninspiring, our final equations are very interesting:
\begin{eqBox}
  \begin{align}
    v_1' &=
    \left[\frac{m_1-m_2}{m_1+m_2}\right]v_1 +
    \left[\frac{2m_1}{m_1+m_2}\right]v_2    \label{eq:mom4}\\
    v_2' &=
    \left[\frac{2m_1}{m_1+m_2}\right]v_1 +
    \left[\frac{m_1-m_2}{m_1+m_2}\right]v_2     \label{eq:mom5}
  \end{align}
\end{eqBox}
%One special case is if the second object 2 is initially stationary, i.e.\
%$v_2=0$. In such a case, the $v_2$ terms are dropped from
%Eqs.~\ref{eq:mom4} and \ref{eq:mom5}, reducing the equations to:
%\begin{align}
%  v_1'&=\left(\frac{m_1-m_2}{m_1+m_2}\right)v_1\\
%  v_2'&=\left(\frac{2m_1}{m_1+m_2}\right)v_1
%  \label{eg:simple1}
%\end{align}


\begin{example}[Billiard Problem from Example~\ref{eg:billiard}]
  In the billiard ball examle that we have studied earlier, is the collision
  elatic?

  \textbf{Solution:}
\end{example}



\subsection{Special Cases for One-Dimensional Elastic Collisions}
There are a few different possibilities for outcomes:
\begin{itemize}[leftmargin=15pt]
\item\textbf{Two equal masses:} If the masses are equal, i.e.\ $m_1=m_2=m$,
  then Eqs.~\ref{eq:mom4} and \ref{eq:mom5} become:
  \begin{align*}
    v_1'&=\left(\frac{m_1-m_2}{m_1+m_2}\right)v
    =\left(\frac{m-m}{m+m}\right)v=0\\
    v_2'&=\left(\frac{2m_1}{m_1+m_2}\right)v
    =\left(\frac{2m}{m+m}\right)v=v
  \end{align*}
  All the momentum and energy from object 1 is transferred to object 2. Object
  1 stops, while object 2 continues to move at $v$, as shown in
  Fig.~\ref{fig:same-mass}. This behaviour is most notably expressed in a
  \emph{Newton's cradle}.\footnote{Note that the Newton's cradle is \emph{not}
    an elastic collision. The fact that you can \emph{hear} the metal balls
    colliding means that energy has escaped the system as a sound wave.}
  \begin{figure}[ht]
    \centering
    \begin{tikzpicture}[scale=.8]
      %\tikzstyle{balloon1}=[ball color=red];
      %\tikzstyle{balloon2}=[ball color=blue];
      \shade[balloon1] (-1,0) circle (1) node[white]{$m$};
      \draw[vectors] (0,0)--(1.5,0) node[midway,above]{$v$};
      \shade[balloon2] (4,0) circle (1) node[white]{$m$};
      
      \shade[balloon1] (2,-2.5) circle (1) node[white]{$m$};
      \draw[dashed] (4,-2.5) circle(1);
      \draw[dashed] (5,-2.5)--(6,-2.5);
      \shade[balloon2] (7,-2.5) circle (1) node[white]{$m$};
      \draw[vectors] (8,-2.5)--(9.5,-2.5) node[midway,above]{$v$};
    \end{tikzpicture}
    \caption{When a moving object collides with a stationary object of equal
      mass, all the momentum and energy are transferred.}
    \label{fig:same-mass}
  \end{figure}
\item\textbf{Large Object Colliding With Small Object:} In this second case,
  object 1 is much more massive than the second object, i.e.\ $m_1\gg m_2$. We
  can then effectively ``ignore'' the $m_2$ terms in Eqs.~\ref{eq:mom4} and
  \ref{eq:mom5}. The equations then become:
  \begin{align*}
    v_1'&=\left(\frac{m_1-m_2}{m_1+m_2}\right)v=\left(\frac{m_1}{m_1}\right)v
    =v\\
    v_2'&=\left(\frac{2m_1}{m_1+m_2}\right)v=\left(\frac{2m_1}{m_1}\right)v=2v
  \end{align*}
  Object 1 continues to move at nearly its initial speed $v$, while object 2
  picks up \emph{twice} the speed of object 1, as shown in
  Fig~\ref{fig:big-small}. This kind of collision is often used in
  gravitational slingshots, which forms the basis for deep-space exploration
  probes that need to travel through the solar system.
  \begin{figure}[ht]
    \centering
    \begin{tikzpicture}[scale=.8]
      %\tikzstyle{balloon1}=[ball color=red];
      %\tikzstyle{balloon2}=[ball color=blue];
      \shade[balloon1] circle (1.5) node[white]{$m_1$};
      \draw[vectors] (1.5,0)--(3,0) node[midway,above]{$v$};
      \shade[balloon2] (7,0) circle (.6) node[white]{$m_2$};
      
      \shade[balloon1](5.5,-2.5) circle(1.5) node[white]{$m_1$};
      \draw[vectors] (7,-2.5)--(8.5,-2.5) node[midway,above]{$v$};
      \draw[dashed] (7,-2.5) circle (.6);
      \draw[dashed] (7.6,-2.5)--(9.4,-2.5);
      \shade[balloon2] (10,-2.5) circle (.6) node[white]{$m_2$};
      \draw[vectors] (10.6,-2.5)--(13.6,-2.5) node[midway,above]{$2v$};
    \end{tikzpicture}
    \caption{When a large object collides with a stationary small object, 
    it does not slow down, but the smaller object gains twice the speed.}
    \label{fig:big-small}
  \end{figure}
\item\textbf{Small Object Colliding With Large Object:} In this final case,
  object 1 has a much smaller mass than object 2, i.e.\ $m_1\ll m_2$. Like the
  previous case, we can effectively ignore the smaller mass, which is $m_1$ in
  this case. Then, Eqs.~\ref{eq:mom4} and \ref{eq:mom5} reduces to:
  \begin{align*}
    v_1'&=\left(\frac{m_1-m_2}{m_1+m_2}\right)v=\left(\frac{-m_2}{m_2}\right)v
    =-v\\
    v_2'&=\left(\frac{2m_1}{m_1+m_2}\right)v=\left(\frac{2m_1}{m_2}\right)v=0
  \end{align*}
  Object 2 continues to be stationary, while object 1 bounces back at its
  original speed, as shown in Fig~\ref{fig:big-small}.
  \begin{figure}[ht]
    \centering
    \begin{subfigure}{.45\textwidth}
      \centering
      \begin{tikzpicture}[scale=.8]
        \draw[vectors] (0,0)--(2,0) node[right]{$v$};
        \shade[balloon1] circle (.35) node[white]{$m_1$};
        \shade[balloon2] (5,0) circle (1.5) node[white]{$m_2$};
      \end{tikzpicture}
    \end{subfigure}
    \begin{subfigure}{.45\textwidth}
      \centering
      \begin{tikzpicture}[scale=.8]
        \shade[balloon1] (4,-3.3) circle(.35) node[white]{$m_1$};
        \draw[vectors] (3.5,-3.3)--(2,-3.3) node[left]{$v'$};
        \shade[balloon2] (7,-3.3) circle (1.5) node[white]{$m_2$};
      \end{tikzpicture}
    \end{subfigure}
    \caption{When a small objects collides with a stationary large object, it
      bounces back with the same speed, while the larger object remains
      stationary.}
  \end{figure}
\end{itemize}




\section{Example Problems}

\begin{example}
  A \SI{.0520}{\kilo\gram} golf ball is moving velocity of
  \SI{2.10}{\metre\per\second} when it collides, head on, with a stationary
  \SI{.155}{\kilo\gram} billiard  ball. If the golf ball rolls directly
  backwards with a velocity of \SI{-1.04}{\metre\per\second}, is the collision
  elastic?
\end{example}
%  \vspace{.4in}\textcolor{gray}{
%    To solve this type of problem, using the conservation of momentum to find
%    the velocities of the golf ball and the billiard ball. Then, sum the total
%    kinetic energies of both balls, and compare that to the total kinetic
%    energy of the balls before the collision.
%  }
%
%
%\begin{example}
%  A car (\SI{1000}{\kilo\gram}) travels at a speed of
%  \SI{20}{\metre\per\second} towards a stationary truck
%  (\SI{3000}{\kilo\gram}). The car rear ends the truck
%  elastically\footnote{Reality check: For this example we are not concerned
%  with the \emph{implausibility} of such a collision; we just want to figure
%  out what happens \emph{if} the collision actually occurs.}. What are the
%  velocities of the car and truck after the collision?
%\end{example}
%
%
%
%\begin{frame}{Solving the Example Problem}
%  In order to solve this question we need \emph{both} the conservation of
%  (kinetic) energy and the conservation of momentum. Here, $A$ is the car, $B$
%  is the truck.
%
%  \vspace{-.2in}{\large
%    \begin{align*}
%      m_Av_A+m_Bv_B&=m_Av_A'+m_Bv_B'\\
%      \half m_Av_A^2+\half m_Bv_B^2&=\half m_Av_A'^2+\half m_Bv_B'^2
%    \end{align*}
%  }
%
%  %For generalization, we won't assume whether any object is stationary or in
%  %motion.
%
%
%
%
%\begin{frame}{Solving the Example Problem}
%  \begin{columns}[T]
%    \column{.45\textwidth}
%    \underline{\textbf{Momentum Equation}}
%
%    \vspace{-.1in}
%    \begin{displaymath}
%      m_Av_A+m_Bv_B=m_Av_A'+m_Bv_B'
%    \end{displaymath}
%%    We can eliminate $v_B$, since the truck wasn't moving:
%%    \begin{displaymath}
%%      m_Av_A=m_Av_A'+m_Bv_B'
%%    \end{displaymath}
%
%    \vspace{.02in}Move all $m_A$ terms to the left, and $m_B$ terms to the
%    right :
%    \begin{equation}
%      \boxed{m_A(v_A-v_A')=m_B(v_B'-v_B)}
%    \end{equation}
%
%    \column{.55\textwidth}
%    \uncover<2>{
%      \underline{\textbf{Kinetic Energy Equation}}
%      \begin{displaymath}
%        \half m_Av_A^2+\half m_Bv_B^2=\half m_Av_A'^2+\half m_Bv_B'^2
%      \end{displaymath}
%      Multiply every term by 2, and then move $m_A$ terms to the left, and
%      $m_B$ terms to the right:
%      \begin{equation}
%        \boxed{m_A(v_A^2-v_A'^2)=m_B(v_B'^2-v_B^2)}
%      \end{equation}
%    }
%  \end{columns}
%
%
%
%
%\begin{frame}{Solving the Example Problem}
%  Dividing (2) by (1), we get:
%  \begin{displaymath}
%    \frac{(2)}{(1)}=\frac{m_A(v_A^2-v_A'^2)}{m_A(v_A-v_A')}
%    =\frac{m_B(v_B'^2-v_B^2)}{m_B(v_B'-v_B)}
%  \end{displaymath}
%  We can cancel out the $m_A$ and $m_B$ terms on both sides, then factor the
%  difference of two squares on top:
%  \begin{displaymath}
%    \frac{(v_A+v_A')(v_A-v_A')}{(v_A-v_A')}=
%    \frac{(v_B'+v_B)(v_B'-v_B)}{(v_B'-v_B)}
%  \end{displaymath}
%  Now we get an equation relating the velocities that can be substituted back
%  to (1) and (2):
%  \begin{displaymath}
%    v_A + v_A'= v_B + v_B'
%  \end{displaymath}
%
%
%
%
%\begin{frame}{Solving the Example Problem}
%  We arrive at these two equations for final velocities that applies to
%  \emph{all} one-dimensional elastic collisions:
%  
%  \eq{-.2in}{
%    \boxed{v_A'=\frac{m_A-m_B}{m_A+m_B}v_A+\frac{2m_B}{m_A+m_B}v_B}\;\;
%    \boxed{v_B'=\frac{m_B-m_A}{m_A+m_B}v_B+\frac{2m_A}{m_A+m_B}v_A}
%  }
%
%  %\vspace{-.1in}These equations apply to \emph{all} one-dimensional elastic collisions.
%  Substituting values for $m_A$, $m_B$ and $v_A$ in this example, we get:
%  \begin{align*}
%    v_A'&=\frac{m_A-m_B}{m_A+m_B}v_A=\frac{(1000-3000)}{(1000+3000)}\times 20
%    = \boxed{\SI{-10}{\metre\per\second}}\\
%    v_B'&=\frac{2m_A}{m_A+m_B}v_A=\frac{(2\times 1000)}{(1000+3000)}\times 20
%    = \boxed{\SI{10}{\metre\per\second}}
%  \end{align*}
%
%
%
%
%\begin{frame}[t]{This example tells us much more!}
%  What happens when the two objects have the same mass, i.e.\ $m_A=m_B=m$?
%
%  \vspace{-.2in}{\large
%    \begin{align*}
%      v_A'&=\cancel{\frac{m_A-m_B}{m_A+m_B}v_A}+\frac{2m_B}{m_A+m_B}v_B\\
%      v_B'&=\cancel{\frac{m_B-m_A}{m_A+m_B}v_B}+\frac{2m_A}{m_A+m_B}v_A
%    \end{align*}
%    %\boxed{v_A'=\frac{m_A-m_B}{m_A+m_B}v_A}\quad
%    %\boxed{v_B'=\frac{2m_A}{m_A+m_B}v_A}
%  }
%  
%  \vspace{-.15in}
%  \begin{columns}
%    \column{.5\textwidth}
%    \begin{align*}
%      v_A'&=\frac{2m_B}{m_A+m_B}v_A=\frac{2m}{m+m}v_A\\
%      v_A'&=v_B
%    \end{align*}
%
%    \column{.5\textwidth}
%    \begin{align*}
%      v_B'&=\frac{2m_A}{m_A+m_B}v_A=\frac{2m}{m+m}v_A\\
%      v_B'&=v_A
%    \end{align*}
%  \end{columns}
%
%  \vspace{.1in}If both masses are the same, then they exchange their
%  velocities!
%
%
%
%
%\begin{frame}[t]{This example tells us much more!}
%  When $A$ is much more massive than $B$, i.e.\ $m_A\gg m_B$, and assuming that
%  $B$ is initially at rest, we can effectively ``ignore'' $m_B$:
%  
%  \eq{-.1in}{
%    \boxed{v_A'=\frac{m_A-m_B}{m_A+m_B}v_A}\quad
%    \boxed{v_B'=\frac{2m_A}{m_A+m_B}v_A}
%  }  
%  \vspace{-.1in}
%  \begin{columns}
%    \column{.5\textwidth}
%    \begin{align*}
%      v_A'&=\frac{m_A-m_B}{m_A+m_B}v_A\\
%      v_A'&\approx\frac{m_A}{m_A}v_A\\
%      v_A'&\approx v_A
%    \end{align*}
%
%    \column{.5\textwidth}
%    \begin{align*}
%      v_B'&=\frac{2m_A}{m_A+m_B}v_A\\
%      v_B'&\approx\frac{2m_A}{m_A}v_A\\
%      v_B'&\approx 2v_A
%    \end{align*}
%  \end{columns}
%  
%  \vspace{.1in}$A$ continues to move like nothing happened, but object $B$ is
%  pushed to move at an twice the speed of $A$.
%
%
%
%
%\begin{frame}[t]{This example tells us much more!}
%  Similarly, when $B$ is more massive, i.e.\ $m_A\ll m_B$, and $B$ is initially
%  at rest, then we `ignore'' $m_A$ instead:
%
%  \eq{-.1in}{
%    \boxed{v_A'=\frac{m_A-m_B}{m_A+m_B}v_A}\quad
%    \boxed{v_B'=\frac{2m_A}{m_A+m_B}v_A}
%  }
%  \vspace{-.1in}
%  \begin{columns}
%    \column{.5\textwidth}
%    \begin{align*}
%      v_A'&=\frac{m_A-m_B}{m_A+m_B}v_A\\
%      v_A'&\approx\frac{-m_B}{m_B}v_A\\
%      v_A'&\approx-v_A
%    \end{align*}
%    \column{.5\textwidth}
%    \begin{align*}
%      v_B'&=\frac{2m_A}{m_A+m_B}v_A\\
%      v_B'&\approx\frac 0{m_B}v_A\\
%      v_B'&\approx 0
%    \end{align*}
%  \end{columns}
%  
%  \vspace{.1in}$A$ bounces off $B$, and travels in the opposite direction
%  with the same speed.
%
%
%
%
%\section{Practice Problems}

\begin{example}
  A forensic expert needs to find the velocity of a bullet fired from a gun in
  order to predict the trajectory of a bullet. He fires a \SI{5.5}{\gram}
  bullet into a ballistic pendulum with a mass of \SI{1.75}{\kilo\gram}. The
  pendulum swing to a height of \SI{12.5}{\centi\metre} above its rest position
  before dropping back down. What is the velocity of the bullet just before it
  hit and became embedded in the pendulum bob?
  \begin{center}
    \pic{.4}{momentum/graphics/bullet}
  \end{center}
\end{example}



\begin{example}
  A block of wood with a mass of \SI{.500}{\kilo\gram} slides across the floor
  towards a \SI{3.50}{\kilo\gram} block of wood. Just before the collision, the
  small block is travelling at \SI{3.15}{\metre\per\second}. Because some nails
  are sticking out of the blocks, the blocks stick together when they collide.
  Scratch marks on the floor show that they slid \SI{2.63}{\centi\metre} before
  coming to a stop. What is the coefficient of friction between the wooden
  blocks and the floor?
\end{example}
%  \vspace{.4in}\textcolor{gray}{We won't use kinematic or dynamic equations to
%    solve this problem. Instead, we'll use conservation momentum equation and
%    the work kinetic energy theorem.}
%

