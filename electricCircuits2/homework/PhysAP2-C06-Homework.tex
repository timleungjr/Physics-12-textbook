\documentclass{../../oss-apphys-exam}

\begin{document}
%\genheader

\gentitle{6}{DC CIRCUIT ANALYSIS, PART 2}
\classkickMCinstructions

%\textbf{Questions \ref{cap1}--\ref{cap2}}
%\begin{center}
%  \begin{tikzpicture}[american voltages,scale=1.1]
%    \draw(0,0) to[battery1,l=\SI{120}\volt] (0,2)
%    to[C=\SI2{\micro\farad}](2,2)
%    to[C,l_=\SI8{\micro\farad}](2,0)
%    to[C=\SI4{\micro\farad}](0,0);
%    \draw(2,2)--(3,2) to[C=\SI8{\micro\farad}] (3,0)--(2,0);
%  \end{tikzpicture}
%\end{center}
  
\begin{questions}  
%  \question The equivalent capacitance of this circuit is
%  \begin{choices}
%    \choice$7/4$ \si{\micro\farad}
%    \choice$4/7$ \si{\micro\farad}
%    \choice$16/13$ \si{\micro\farad}
%    \choice\SI{10}{\micro\farad}
%    \choice\SI{22}{\micro\farad}
%  \end{choices}
%  \label{cap1}
%    
%  \question The charge stored on the \SI2{\micro\farad} capacitor is most nearly
%  \begin{choices}
%    \choice\SI6{\micro\coulomb}
%    \choice\SI{12}{\micro\coulomb}
%    \choice\SI{22}{\micro\coulomb}
%    \choice\SI{36}{\micro\coulomb}
%    \choice\SI{147}{\micro\coulomb}
%  \end{choices}
%  \label{cap2}
  
  \question A capacitor $C_0$ is connected to a battery and stores charge. If
  the space between the capacitor plates is filled with oil, which of the
  following quantities increase?
  \begin{choices}
    \choice Capacitance and voltage across the plates
    \choice Charge and voltage across the plates
    \choice Capacitance and electric field between the plates
    \choice Capacitance and charge on the plates
    \choice Electric field between the plates and voltage across the plates
  \end{choices}

  \question The circuit shown in the figure has two resistors, an uncharged
  capacitor, a battery, two ammeters, and a switch initially in the open
  position. What will happen to the current measured in the ammeters from the
  instant the switch is closed to a long time after the switch is closed?
  \uplevel{
    \vspace{-.22in}
    \cpic{.3}{circuit1}
  }
  \begin{tabular}{cll}
    & \underline{Ammeter 1} & \underline{Ammeter 2} \\
    (A) & Reading remains constant & Reading remains constant \\
    (B) & Reading remains constant & Reading will change\\
    (C) & Reading will change      & Reading remains constant \\
    (D) & Reading will change      & Reading will change
  \end{tabular}

  \question Four identical capacitors with a plate area of $A$, a distance
  between the plates of $d$, and a dielectric constant $\kappa$ are connected
  to a battery, a resistor, and a switch in series. The switch is closed for a
  long time. The total energy stored in the set of four capacitors is $U$. The
  four capacitors in series are to be replaced with a single capacitor that will
  store the same energy as the four-capacitor set. Which capacitor
  geometry will accomplish this? Select two answers.

  \begin{tabular}{clll}
    & \underline{Dielectric Constant} & \underline{Plate Area}
    & \underline{Distance Between Plates} \\
    (A) & $2\kappa$        & $2A$        & $d$ \\
    (B) & $\kappa$         & $2A$        & $2d$ \\
    (C) & $\dfrac12\kappa$ & $\dfrac12A$ & $d$ \\
    (D) & $\kappa$         & $A$         & $4d$ \\
  \end{tabular}
  \newpage
  
  \question A battery of voltage $V_0$ is attached to two parallel conducting
  plates. Charge is distributed on the plates, and then the battery is
  removed. A dielectric is then inserted between the plates, filling the
  space. Which of the following decreases after the battery is removed and the
  dielectric is inserted to fill the space between the plates?
  \begin{choices}
    \choice Capacitance
    \choice Charge on the plates
    \choice Net electric field between the plates
    \choice Area of the plates
    \choice Separation distance between the plates
  \end{choices}
  
  \question Circuit I and Circuit II shown each consist of a capacitor and a
  resistor. A battery is connected across $a$ and $b$, and then removed. Which
  of the following statements is true of the circuits?
  \begin{center}
    \begin{tikzpicture}
      \node[above] at (1,2.5){Circuit I};
      \draw(0,1.25) to[short,*-](0,2.5)--(2,2.5) to[C=$C$](2,1.25)
      to[R=$R$](2,0)--(0,0) to[short,-*](0,0.75);
      \node at (-0.3,0.75) {$a$};
      \node at (-0.3,1.25) {$b$};
    \end{tikzpicture}
    \hspace{.5in}
    \begin{tikzpicture}
      \node[above] at (1.5,2){Circuit II};
      \draw(0,1.25) to[short,*-](0,2)--(1.5,2) to[R=$R$](1.5,0)--(0,0)
      to[short,-*](0,0.75);
      \draw(1.5,2)--(3,2) to[C=$C$](3,0)--(1.5,0);
      \node at (-0.3,0.75) {$a$};
      \node at (-0.3,1.25) {$b$};
    \end{tikzpicture}
  \end{center}
  \begin{choices}
    \choice Circuit I and Circuit II will both retain stored energy when the
    battery is removed.
    \choice Neither Circuit I nor Circuit II will retain stored energy when
    the battery is removed.
    \choice Only Circuit I will retain stored energy when the battery is
    removed.
    \choice Only Circuit II will retain stored energy when the battery is
    removed.
    \choice Current will continue to flow in both circuits after the battery
    is removed.
  \end{choices}
  
  \uplevel{
    \textbf{Questions \ref{temp1}--\ref{temp2}}: A resistor with a resistance
    of $R$ is sealed in a closed cylindrical container. The gas inside the
    cylinder has an initial pressure of $P$, an initial volume of $V$, and
    number of atoms $N$. A battery of emf ($\varepsilon$) is connected to the
    resistor through an airtight piston of mass $m$ fitted inside the cylinder.
    The piston is able to move up and down as energy is supplied to the gas by
    the electrical circuit.
    \cpic{.17}{temperature}
  }

  \question Which of the following is an expression of the change in temperature
  $\Delta T$ of the gas while energy is supplied to the gas by the resistor?
  \label{temp1}
  \begin{choices}
    \choice $\Delta T=\dfrac{2\varepsilon^2}{3Nk_bR}$
    \choice $\Delta T=\dfrac{2\varepsilon^2t}{3Nk_bR}$
    \choice $\Delta T=\dfrac{2\varepsilon t}{3Nk_bR}$
    \choice $\Delta T=\dfrac{2\varepsilon}{3Nk_bR}$
  \end{choices}
  \newpage
  
  \question The temperature of the gas, as a function of time for the single
  resistor circuit, is shown in the figure. The apparatus is now opened, and a
  second identical resistor $R$ is connected in parallel to the circuit inside
  the cylinder. The apparatus is then resealed and reset to the original
  conditions. How will this affect the temperature versus time graph?
  \label{temp2}
  \begin{center}
    \begin{tikzpicture}[scale=1.5]
      \draw[axes] (0,0)--(2,0) node[right]{$t$};
      \draw[axes] (0,0)--(0,2) node[right]{$T$};
      \draw[very thick,rotate=45] (.2,0)--(2.3,0);
    \end{tikzpicture}
  \end{center}
  \begin{choices}
    \choice The slope of the graph quadruples.
    \choice The slope of the graph doubles.
    \choice The slope of the graph is cut in half.
    \choice The slope of the graph is one-quarter as large.
  \end{choices}
  \newpage

  \classkickFRQinstructions
  
  \uplevel{
    \centering
    \begin{tikzpicture}[scale=1.4,american voltages]
      \draw[thick] (2,0)
      to[battery1,l=\SI{12}\volt] (0,0)
      to[short] (0,1.5)
      to[R,l=\mbox{$R_1=\SI{15}\ohm$}] (2,1.5)--(2,0);
      \draw[thick](0,1.5)
      to[short] (0,3)
      to[C,l=\SI{240}{\micro\farad}] (1,3)
      to[R=\mbox{$R_2=\SI{10}\ohm$}] (2,3)--(2,1.5);
    \end{tikzpicture}
  }
  \question The figure above shows a circuit with two resistors, a battery, a
  capacitor, and a switch. Originally, the switch is open, and the capacitor is
  uncharged.

  \begin{parts}
    \part Complete the voltage-current-resistance-power (VIRP) chart for
    the circuit immediately after the switch is closed.
    \bgroup
    \def\arraystretch{1.8}
    
    {\large
      \begin{center}
        \begin{tabular}{l|m{1.2cm}|m{1.2cm}|m{1.2cm}|m{1.2cm}}
          \hline
          \textbf{Location} &
          \emph V (\si{\volt}) &
          \emph I (\si{\ampere}) &
          \emph R (\si{\ohm}) &
          \emph P (\si{\watt}) \\ \hline
          1 & & & 15 & \\ \hline
          2 & & & 10 & \\ \hline
          Total for Circuit & 12 & & & \\\hline
        \end{tabular}
      \end{center}
    }
    \egroup
    
    \part Complete the voltage-current-resistance-power (VIRP) chart for
    the circuit after the switch is closed for a long time.
    \bgroup
    \def\arraystretch{1.8}
    
    {\large
      \begin{center}
        \begin{tabular}{l|m{1.2cm}|m{1.2cm}|m{1.2cm}|m{1.2cm}}
          \hline
          \textbf{Location} &
          \emph V (\si\volt) &
          \emph I (\si\ampere) &
          \emph R (\si\ohm) &
          \emph P (\si\watt) \\ \hline
          1 & & & 15 & \\ \hline
          2 & & & 10 & \\ \hline
          Total for Circuit & 12 & & & \\\hline
        \end{tabular}
      \end{center}
    }
    \egroup

    \part What is the energy stored in the capacitor after the switch has
    been closed a long time?
  \end{parts}
  \newpage

  
  %TAKEN FROM 2003 AP PHYSICS B FREE-RESPONSE QUESTION #2
  \uplevel{
    \cpic{.3}{rc-circuit}
  }
  \question A circuit contains two resistors (\SI{10}{\ohm} and \SI{20}\ohm)
  and two capacitors (\SI{12}{\micro\farad} and \SI6{\micro\farad}) connected to
  a \SI6{\volt} battery, as shown in the diagram above. The circuit has been
  connected for a long time.
  \begin{parts}
    \part Calculate the total capacitance of the circuit.
    \vspace{\stretch1}
    
    \part Calculate the current in the \SI{10}{\ohm} resistor.
    \vspace{\stretch1}
    
    \part Calculate the potential difference between points $A$ and $B$.
    \vspace{\stretch1}
    
    \part Calculate the charge stored on one plate of the \SI6{\micro\farad}
    capacitor.
    \vspace{\stretch1}
    
    \part The wire is cut at point $P$. Will the potential difference between
    points $A$ and $B$ increase, decrease, or remain the same?

    \vspace{.1in}
    \underline{\hspace{.5in}} increase\hspace{.3in}
    \underline{\hspace{.5in}} decrease\hspace{.3in}
    \underline{\hspace{.5in}} remain the same
    
    \vspace{.1in}Justify your answer.
    \vspace{\stretch1}
  \end{parts}
  \newpage

  %TAKEN FROM 2015 AP PHYSICS 2 EXAM FREE-RESPONSE QUESTION #2
  \uplevel{
    \cpic{.45}{bulbs}
  }
  \question A battery of emf $\mathcal E$ and negligible internal resistance,
  three identical incandescent lightbulbs, and a switch $S$ that is initially
  open are connected in the circuit shown above. The bulbs each have resistance
  $R$. Students make predictions about what happens to the brightness of the
  bulbs after the switch is closed.
  \begin{parts}
    \part A student makes the following prediction about bulb 1: ``Bulb 1 will
    decrease in brightness when the switch is closed.''
    \begin{subparts}
      \subpart Do you agree or disagree with the student's prediction about
      bulb 1? Qualitatively explain your reasoning.
      \vspace{\stretch1}
      
      \subpart Before the switch is closed, the power expended by bulb 1 is
      $P_1$. Derive an expression for the power $P_\text{new}$ expended by bulb 1
      after the switch is closed in terms of $P_1$.
      \vspace{\stretch1}
      
      \subpart How does the result of your derivation in part (a)ii relate to
      your explanation in part (a)i?
      \vspace{\stretch1}
    \end{subparts}
    \part A student makes the following prediction about bulb 2: ``Bulb 2 will
    decrease in brightness after the switch is closed.''
    \begin{subparts}
      \subpart Do you agree or disagree with the student's prediction about bulb
      2? Explain your reasoning in words.
      \vspace{\stretch1}
      
      \subpart Justify your explanation with a calculation.
      \vspace{\stretch1}
    \end{subparts}
    \newpage
    
    \uplevel{
      While the switch is open, bulb 3 is replaced with an uncharged capacitor.
      The switch is then closed.
    }

    \part How does the brightness of bulb 1 compare to the brightness of bulb 2
    immediately after the switch is closed? Justify your answer.
    \vspace{\stretch1}
    
    \part How does the brightness of bulb 1 compare to the brightness of bulb 2
    a long time after the switch is closed? Justify your answer.
    \vspace{\stretch3}
  \end{parts}
\end{questions}
\end{document}
