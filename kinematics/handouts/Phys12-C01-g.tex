\documentclass[12pt,compress,aspectratio=169]{beamer}
\usetheme{Singapore}
\setbeamersize{text margin left=.5cm,text margin right=.5cm}
\setbeamertemplate{navigation symbols}{} % suppress nav bar
\usepackage[lf]{carlito}
\usepackage{tikz}
\usepackage{mathpazo}
\usepackage{xcolor,colortbl}
\usepackage{siunitx}

\sisetup{
  detect-all,
  per-mode=symbol
}


\title{Acceleration Due To Gravity}
\subtitle{Unit 1: Fundamentals of Dynamics}
\author{Grade 12 Physics}
\institute{Olympiads School}
\date{Updated: Fall 2022}

\newcommand{\eq}[2]{
  \vspace{#1}{\large\begin{displaymath}#2\end{displaymath}}
}

\begin{document}


\begin{frame}
  \titlepage
\end{frame}



\begin{frame}{What is $g$}
  In physics, we use the symbol $\vec g$ to represent the \textbf{acceleration
    due to gravity} for any free-falling objects. On/near the surface of Earth,
  it has an \emph{average} value of

  \eq{-.15in}{
    \boxed{
      \vec g=\SI{9.81}{\metre\per\second\squared}\text{ [down]}
    }
  }

  Since acceleration is a vector quantity, the direction of $\vec g$ is
  important: it always points down by definition.
\end{frame}



\begin{frame}{Law of Universal Gravitation}
  The value of $g$ is obtained by grouping terms from the \textbf{law of
    universal gravitation}, which describes the force of attraction between
  massive objects. The magnitude of the gravitational force is given by:

  \eq{-.1in}{
    \boxed{F_g=\frac{Gm_1m_2}{r^2}}
  }
  \begin{center}
    \begin{tabular}{l|c|c}
      \rowcolor{pink}
      \textbf{Quantity} & \textbf{Symbol} & \textbf{SI Unit} \\ \hline
      Magnitude of gravitational force & $F_g$   & \si\newton\\
      Universal gravitational constant & $G$   & \si{N.m^2/kg^2} \\
      Point masses                     & $m_1$, $m_2$ & \si{\kilo\gram} \\
      Distance between two masses      & $r$   & \si\metre
    \end{tabular}
  \end{center}
  The \textbf{universal gravitational constant} has a value of
  $G=\SI{6.674e-11}{N.m^2/kg^2}$.
\end{frame}



\begin{frame}{How Do We Get The Value of $g$?}
  To find the value of $g$, we group the variables in the universal gravitation
  equation:
    
  \eq{-.2in}{
    F_g=\underbrace{\left[\frac{Gm_s}{r^2}\right]}_{=g}m=mg
  }

  \vspace{-.1in} On/near Earth's surface, we use Earth's mass
  $m_s=m_E=\SI{5.972e24}{\kilo\gram}$ and the average value of Earth's radius
  $r=r_E=\SI{6.371e6}{\metre}$ to find the value of $g$:

  \eq{-.2in}{
    g\approx\SI{9.82}{\metre\per\second\squared}
  }

  \vspace{-.15in}This is slightly \emph{higher} than the accepted value of
  $9.81$.
\end{frame}



\begin{frame}{Earth Is Non-Inertial}
  This small discrepency comes from the fact that Earth is a non-initial frame
  of reference, because:
  \begin{itemize}
  \item Earth rotates on an axis with a period of \SI1\day
  \item Earth orbits around the Sun with a period of \SI{365.25}\day
  \item The solar system orbits around the centre of the galaxy (Milky Way) with
    an unknown period
  \end{itemize}
  Of these factors, the Earth's own rotation (not surprisingly) has the
  strongest influence.
\end{frame}



\begin{frame}{Making Earth an ``Inertial'' Frame of Reference}
  Because of Earth's own rotation, observers on Earth experience a small
  \textbf{centrifugal force} in the opposite direction of gravity.
  \begin{itemize}
  \item Centrifugal force is not a real force
  \item It is a \textbf{pseudo force}\footnote{Also known as a
    \textbf{fictitious force}} that is only experienced by observers in
    non-inertial frames of reference
  \end{itemize}
  Therefore observers on Earth (non-inertial frames of reference)
  measure a slight
  decrease in $g$ because of this centrifugal force, hence we use a
  \emph{corrected} value of \SI{9.81}{\metre\per\second\squared}. This way, we
  can ``pretend'' that Earth is an inertial frame of reference.
\end{frame}



\begin{frame}{Value of $g$ At Various Parts of Earth}
  Additionally, the Earth is not a perfect sphere. Depending on where you are,
  the value of $g$ changes slightly.
  \begin{center}
    \begin{tabular}{l|c|c|c}
      \rowcolor{pink}
      \textbf{Location} & $g$ (\si{m/s^2}) & Altitude (\si{m}) & $r$ (\si{km})\\
      \hline
      North Pole                     & $9.8322$ & $0$      & $6357$ \\
      Equator                        & $9.7805$ & $0$      & $6378$ \\
      Peak of Mt. Everest            & $9.7647$ & $8850$   & $6387$ \\
      Bottom of Mariana Ocean Trench & $9.8331$ & \num{-11034} & $6367$ \\
      International Space Station    & $9.0795$ & \num{250000} & $6628$
    \end{tabular}
  \end{center}
  $r$ is the distance to the centre of Earth.
\end{frame}



\begin{frame}{Using $g$ in Kinematics and Dynamics}
  Some important notes when solving questions:
  \begin{itemize}
  \item $\vec g$ is a vector, and the direction is \emph{always} pointing down
  \item The \emph{magnitude} of this vector, i.e.\
    $g=|\vec g|=\SI{9.81}{\metre\per\second\squared}$, \emph{not}
    \SI{-9.81}{\metre\per\second\squared}.
    \begin{itemize}
    \item Consistent with basic understanding of vectors, the magnitude must
      always non-negative\footnote{It means that it can be positive, or zero}
    \item We ``use'' \SI{-9.81}{\metre\per\second\squared} in problem solving
      because (most notably) for projectile motions, $y$-axis generally is
      defined pointing up, while $g$ points down by definition (i.e.\ in the
      negative direction)
    \item The negative sign in \SI{-9.81}{\metre\per\second\squared}
      actually presents the direction of the vector
    \end{itemize}
  \end{itemize}
%  This way of defining the axes is the standard (``right-handed'') Cartesian
%  coordinate system that is most familiar in math classes. It also allows for
%  most mathematical tools for vectors to be used easily.
%\end{frame}
%
%
%
%\begin{frame}{Using $g$ in Kinematics}
%  When the $y$-axis points upwards, acceleration due to gravity, which points
%  \emph{down}, is therefore negative, i.e.\
%  
%  \eq{-.2in}{
%    a_y=-g=\SI{-9.81}{\metre\per\second^2}
%  }
%
%  It is important to keep in mind that \SI{9.81}{\metre\per\second^2} is the
%  \emph{magnitude} of the acceleration vector, while the negative sign
%  indicates that it is in the negative \emph{direction} (down).
\end{frame}
\end{document}
