%\chapter{Introduction to Quantum Mechanics}
\chapter{Quantum Mechanics}
\label{chapter:quantum}


\begin{definition}
  \textbf{Anyone who is not shocked by quantum theory has not understood it.}

  \vspace{.1in}\flushright{- Niels Bohr}
\end{definition}



%\section[Intro]{Introduction}

Light must be a wave. After all, it has all the properties of waves:
\begin{itemize}
\item Refraction
\item Interference
\item Diffraction\ldots
\end{itemize}
We even know what \emph{kind} of a wave it is
\begin{itemize}
\item Electromagnetic (``EM'') wave
\item A transverse wave
\item Same as: radio waves, microwave, infrared, ultraviolet, x-ray\ldots
\item Travels in vacuum with a speed of \SI{2.998e8}{\metre\per\second},
  independent of the velocity of the object emitting the light
\end{itemize}
But now we're going to find out that things aren't as simple as it seems.
%
%
%
%
\section{Blackbody Radiation}
It has been well understood by blacksmiths for centuries that when metals are
heated, the brightness and colour of its glow depends only on its temperature,
and not the type of metal itself. An example is shown in
Fig.~\ref{fig:glowing-metal}. However, there was no physics that could explain
this phenomenon, until the concept of \textbf{blackbody radiation} was
proposed by Gustav Kirchhoff in 1860, about the same time when Maxwell's
equations for electrodynamics were developed.
\begin{figure}[ht]
  \centering
  \pic{.75}{quantumMechanics/graphics/Glowing_metal}
  \caption{Metals glow when heated}
  \label{fig:glowing-metal}
\end{figure}
A blackbody is an idealized object that absorbs all incident electromagnetic
(EM) radiation, regardless of frequency or angle of incidence. We can think of
the blackbody as a box (a ``cavity'') with a mirror inside, and a hole where
EM radiation is allowed into it. Some of the light reflects inside the cavity,
and some gets absorbed by the blackbody. Eventually \emph{all} the light inside
the cavity is absorbed. This is shown schematically in Fig.~\ref{fig:cavity}.
\begin{figure}[ht]
  \centering
  \pic{.3}{quantumMechanics/graphics/blackbody1}
  \caption{A blackbody modelled as a cavity}
  \label{fig:cavity}
\end{figure}

A blackbody is always in thermal equilibrium with its surroundings (see
Chapter~\ref{chapter:thermodynamics}), therefore all of the absorbed energy
would be then \emph{immediately} radiated back as EM radiation.\footnote{The
term ``blackbody'' came about because an object at room temperature, most of
the EM radiation emitted and is in the infrared wavelengths that cannot be
perceived by the human eye, therefore the object appears ``black''.} The
spectral distribution depends only on temperature, as observed. Radiation
spontaneously emitted by many ordinary objects can be approximated as blackbody
radiation.

The spectral radiance curve for the blackbody is shown in
Fig.~\ref{fig:spectral-radiance}.
\begin{figure}[ht]
  \centering
  \begin{tikzpicture}[scale=1.2]
    \def\N{100}
    \def\xmax{3100}
    \def\ymax{1.36e10}

    % RAINBOW
    \foreach \lambda in {430,432,...,750}{
      \definecolor{tmpcolor}{wave}{\lambda}
      \colorlet{mycolor}[rgb]{tmpcolor}
      \draw[thick,mycolor,opacity=.3]
      (\lambda/435+.41,.39)--(\lambda/435+.41,5);
    }
  
    \begin{axis}[
        every axis plot/.style={
          thick,mark=none,samples=\N,domain=5:\xmax,smooth},
        xmin=(-.05*\xmax), xmax=(1.05*\xmax),
        ymin=(-.08*\ymax), ymax=(1.08*\ymax),
        restrict y to domain=0:\ymax,
        axis lines=middle,
        %axis line style=thick,
        %tick style={black,thick},
        max space between ticks=26,
        xlabel={Wavelength $\lambda$ [nm]},
        ylabel={Power $P$ [kW/sr\,m$^2$\,nm]},
        xlabel style={at={(rel axis cs:.5,0)},below=3pt},
        ylabel style={at={(rel axis cs:-.132,.5)},rotate=90},
        width=10cm, height=7cm,
        tick scale binop=\times,
        every y tick scale label/.style={at={(rel axis cs:0,1)},anchor=south}
      ]
      
      % PLANCK
      \addplot[red] {planck(x,3000)};
      \addplot[orange] {planck(x,4000)};
      \addplot[blue] {planck(x,5000)};
      
      % MAXIMUM (Wien's displacement law)
      \addplot[very thick,variable=T,domain=2200:5200]
      ({lampeak(T)},{planck(lampeak(T),T)});
    \end{axis}
      
    % LABELS
    \node[scale=.8,red] at (3,.85) {\SI{3000}\kelvin};
    \node[scale=.8,orange] at (2.6,2) {\SI{4000}\kelvin};
    \node[scale=.8,blue] at (2.8,4) {\SI{5000}\kelvin};

    \fill (2.77,0.71) circle (2pt);
    \fill (2.17,1.81) circle (2pt);
    \fill (1.82,4.75) circle (2pt);
  \end{tikzpicture}
  \caption{Spectral radiance of the blackbody at various temperatures}
  \label{fig:spectral-radiance}
\end{figure}
As temperature of the blackbody increases, EM radiation intensity increases
(i.e.\ the object glows brighter) while the peak shifts towards shorter
wavelengths (i.e.\ colour become more bluish). The peak of the curve is
predicted by \textbf{Wien's displacement law}, shown as the black line on the
Fig.~\ref{fig:spectral-radiance}.




\subsection{Rayleigh-Jeans Law}

By 1899, the best explanation of the blackbody radiation is the
\textbf{Rayleigh-Jeans law}, which was based on established and well-proven
physics: standing waves, Maxwell's equations for electrodynamics, and
thermodynamics.\footnote{If you are interested, and have background in
calculus, you can find a fairly simple explanation via
\href{https://hyperphysics.gsu.edu/hbase/mod6.html}{hyperphysics.gsu.edu/hbase/mod6.html})}

Curiously, Rayleigh-Jeans law predict an interesting relationship between
energy per unit volume at any temperature $T$ to be inversely proportional to
the 4th power of wavelength by:
\begin{equation}
  P(\lambda,T)=\frac{8\pi k_BT}{\lambda^4}
\end{equation}
The law agrees with experimental results for long wavelengths (low frequency),
but shorter wavelengths (e.g.\ ultraviolet waves) deviates wildly from observed
behaviour, and the equation predicts that the shortest wavelengths would have
infinite intensity, as shown in Fig.~\ref{fig:rayleigh-jeans}. This was known
as the \emph{ultraviolet catastrophe}.
\begin{figure}[ht]
  \centering
  \begin{tikzpicture}
    \def\N{100}
    \def\xmax{3100}
    \def\ymax{1.36e10}

    % RAINBOW
    \foreach \lambda in {430,432,...,750}{
      \definecolor{tmpcolor}{wave}{\lambda}
      \colorlet{mycolor}[rgb]{tmpcolor}
      \draw[thick,mycolor,opacity=.3]
      (\lambda/435+.41,.39)--(\lambda/435+.41,5);
    }
  
    \begin{axis}[
        every axis plot/.style={
          mark=none,samples=\N,domain=5:\xmax,smooth},
        xmin=(-.05*\xmax), xmax=(1.05*\xmax),
        ymin=(-.08*\ymax), ymax=(1.08*\ymax),
        restrict y to domain=0:\ymax,
        axis lines=middle,
        axis line style=thick,
        tick style={black,thick},
        max space between ticks=26,
        xlabel={Wavelength $\lambda$ (nm)},
        ylabel={Power $P$ (kW/sr\,m$^2$\,nm)},
        xlabel style={at={(rel axis cs:.5,0)},below=2pt},
        ylabel style={at={(rel axis cs:-.12,.5)},rotate=90},
        width=10cm, height=7cm,
        tick scale binop=\times,
        every y tick scale label/.style={at={(rel axis cs:0,1)},anchor=south}
      ]
      
      % PLANCK
      \addplot[black] {planck(x,5000)};
      \addplot[red,ultra thick,domain=1000:3500] {rayleighjeans(x,5000)};
    \end{axis}
      
    % LABELS
    \node[scale=.7] at (2.8,4) {\SI{5000}\kelvin};
    \node[red,scale=.7] at (5.6,3) {\SI{5000}{\kelvin} Rayleigh-Jeans};
  \end{tikzpicture}
  \caption{Comparision of Rayleigh-Jeans law prediction at \SI{5000}{\kelvin}
    with observed spectral radiance}
  \label{fig:rayleigh-jeans}
\end{figure}

The equation that \emph{best} predicted the spectral radiance for blackbody
radiation was \textbf{Wien's distribution law}, or
\textbf{Wien's approximation}, as shown in Fig.~\ref{fig:wiens-approx}. It fits
experimental data for shorter wavelengths, but deviates slightly for longer
wavelengths (lower frequencies). This law is an ``empirical function'' to fit
the data.

\begin{figure}[ht]
  \centering
  \begin{tikzpicture}
    \def\N{100}
    \def\xmax{3100}
    \def\ymax{1.36e10}

    % RAINBOW
    \foreach \lambda in {430,432,...,750}{
      \definecolor{tmpcolor}{wave}{\lambda}
      \colorlet{mycolor}[rgb]{tmpcolor}
      \draw[thick,mycolor,opacity=.03]
      (\lambda/435+.41,.39)--(\lambda/435+.41,5);
    }
  
    \begin{axis}[
        every axis plot/.style={
          mark=none,samples=\N,domain=5:\xmax,smooth},
        title={Spectral Radiance at 5000 K},
        xmin=(-.02*\xmax), xmax=(1.02*\xmax),
        ymin=(-.05*\ymax), ymax=(1.08*\ymax),
        restrict y to domain=0:\ymax,
        axis lines=middle,
        %axis line style=thick,
        %tick style={black,thick},
        max space between ticks=26,
        xlabel={Wavelength $\lambda$ [nm]},
        ylabel={Power $P$ [kW/sr\,m$^2$\,nm]},
        xlabel style={at={(rel axis cs:.5,0)},below=3pt},
        ylabel style={at={(rel axis cs:-.12,.5)},rotate=90},
        width=10cm, height=7cm,
        tick scale binop=\times,
        every y tick scale label/.style={at={(rel axis cs:0,1)},anchor=south}
      ]
      
      % PLANCK
      \addplot[black] {planck(x,5000)};
      \addplot[very thick,red] {wien(x,5000)};
    \end{axis}
      
    % LABELS
    \node[scale=.8,red] at (2.8,2) {Wien};
  \end{tikzpicture}
  \caption{Wien's distribution law prediction for the spectral radiance
    at \SI{5000}\kelvin.}
  \label{fig:wiens-approx}
\end{figure}


\section{Quantization of Energy}

In 1900, German physicist Max Planck began searching for a solution to the
blackbody radiation problem
%\begin{columns}
%    \column{.16\textwidth}
%    \pic{1.2}{20973-050-F6EEBFF1}\\
%    {\footnotesize Max Planck}
%    
%    \column{.8\textwidth}
\begin{itemize}
\item First improved on Wien's distributon law with better agreement with
  experimental data
\item Then derived the distribution function $P(\lambda,T)$, called
  \textbf{Planck's law}:
  \begin{equation}
    P(\lambda,T)=\frac{2hc^2}{\lambda^5}
    \dfrac1{\exp\left[\frac{hc}{\lambda k_BT}\right]-1}
  \end{equation}
\item Then he searched for a way to explain the physics
\item Turned to a new branch of physics called
  \textbf{statistical mechanics} (it is how the Maxwell-Boltzmann function
  is derived in Class \#1)
\end{itemize}

A blackbody radiates because the walls are composed of subatomic electric
oscillators (``resonators'')
\begin{itemize}
\item The nature of these resonators were unknown
\item Billions of resonators vibrating at different frequencies, and
  therefore
\item Emitting radiation at those frequencies (from Class \#9, the frequency
  of disturbance at the source is the frequency of the wave)
\item In classical physics, the resonators can have any value of energy, and
  change its amplitude continuously
\end{itemize}
For Planck's law to make sense, Planck had to argue that energy emitted each
resonator must be \emph{discrete}
\begin{itemize}
\item When energy is emitted from a resonator, it drops to the next lower
  (discrete) energy level
\end{itemize}

The total energy of \emph{any} harmonic oscillator can only be whole-number
multiples of $hf$:
\begin{equation}
  \boxed{E_\text{res}=nhf}\quad n=0,1,2,\ldots
\end{equation}
\begin{center}
  \begin{tabular}{l|c|c}
    \rowcolor{pink}
    \textbf{Quantity} & \textbf{Symbol} & \textbf{SI Unit} \\ \hline
    Energy of the resonator & $E_\text{res}$ & \si\joule\\
    Energy level            & $n$ & (no unit)\\
    Planck's constant      & $h$ & \si{\joule\second}\\
    Frequency of resonator & $f$ & \si\hertz
  \end{tabular}
\end{center}
\textbf{Planck's constant} is experimentally determined to be
$h=\SI{6.626e-34}{\joule\second}$

\begin{figure}[ht]
  \centering
  \begin{tikzpicture}[xscale=.5,yscale=.7]
    \draw[very thick] (0,0)--(4,0);
    \node at (-.5,0){$0$};
    \node at (4.7,0){$0$};
    \foreach \y in {1,...,4} {
      \draw[very thick] (0,\y)--(4,\y);
      \node at (-.5,\y){$\y$};
      \node at (4.7,\y){$\y hf$};
    }
    \foreach \y in {0,...,3} \draw[very thick,red,<->] (\y+.5,\y)--+(0,1);
    
    \draw[vectors] (4.7,4.5)--+(0,1.5) node[above]{Energy $E$};
    \draw[vectors] (-.5,4.5)--+(0,1.5) node[above]{to $n=\infty$};
  \end{tikzpicture}
\end{figure}
Although Planck derived the correct equation for blackbody radiation (and
\emph{accidentally} began the field of quantum mechanics), in his view, using
statistical mechanics was only a mathematical trick, and he did not
investigate further.



\subsection{Energy Quantization in Ordinary Harmonic Oscillators}

Planck envisioned that the quantization of energy applies to \emph{all} harmonic
oscillators, rather than just for the blackbody radiation. To understand how
energy quantization applies to ordinary harmonic oscillators, let us
consider the frictionless horizontal spring-mass system that we studied in
Chapters~\ref{chapter:energy} and 7 (harmonic motion), which is shown again in
Fig.~\ref{fig:horizontal-spring-mass1-again}.
\begin{figure}[hbt]
  \centering
  \begin{tikzpicture}[scale=1.3]
    \draw[mass] (5,.5) rectangle (6,1.5);
    \draw[thick,decorate,
      decoration={aspect=.4,segment length=8,amplitude=8,coil}] (0,1)--(5,1);
    \fill[pattern=north east lines] (6.5,.5)--(6.5,.3)--(-0.2,.3)
    --(-.2,2)--(0,2)--(0,.5)--cycle;
    \draw[thick] (6.5,.5)--(0,.5)--(0,2);
    \draw[axes] (6.5,1)--+(.8,0) node[right]{$+$};
    \draw[dashed] (3.5,-.5)--+(0,3) node[above]{Unstretched/Equilibrium};
    \draw[vector] (3.5,0)--(5,0) node[midway,below]{$x$};
    \fill[red] (5.5,1) circle (.08);
    \draw[vector,red] (5.5,1)--(5.5,0) node[below]{$\bm F_g$};
    \draw[vector,red] (5.5,1)--(5.5,2) node[above]{$\bm F_n$};
    \draw[vector,red] (5.5,1)--(4.5,1) node[above]{$\bm F_s$};
  \end{tikzpicture}
  \caption{A horizontal spring-mass system with no friction, drag and damping
    forces.}
  \label{fig:horizontal-spring-mass1-again}
\end{figure}
In Section \ref{sec:horizontal-spring-mass} we derived the equation for the
natural frequency ($f_0$) of such a system, which depends on the spring
constant $k$ of the spring, and the mass $m$ connected to it:
\begin{equation*}
  f_0 = \frac1{2\pi}\sqrt{\frac km}
\end{equation*}
When the mass oscillates, its total mechanical energy is the mass's kinetic
energy, plus the elastic potential energy stored in the spring, as we have
already shown in Eq.~\ref{eq:total-energy}.
\begin{equation*}
  E_\text{total} = K + U_e
  \label{eq:total-energy}
\end{equation*}
At maximum spring displacement (i.e.\ at amplitude $A$), all of the system's
energy is in elastic potential energy:
\begin{equation*}
  E_\text{total} = \frac12kA^2
\end{equation*}
We can then relate the total energy to the amplitude of the harmonic motion.
Solving for the amplitude, we have:
\begin{equation}
  A=\sqrt{\frac{2E}k}
  \label{eq:amplitude-from-energy}
\end{equation}
In classical physics, the pendulum can have \emph{any} energy levels: both the
mass's kinetic and elastic potential energies are continuous functions, and
therefore any amplitudes are plausible. However, in Planck's view, the energies
of this system must also be quantized.

For a hypothetical system with $k=\SI{987}{\newton\per\metre}$, and
$m=\SI1{\kilo\gram}$, we will have a natural frequency of
$f_0=\SI{5.00}\hertz$. As it is oscillating at its natural frequency, the
allowable energies are:
\begin{align*}
  E_0 &= 0\\
  E_1 &= hf_0  = \SI{3.32e-33}\joule\\
  E_2 &= 2hf_0 = \SI{6.63e-33}\joule\\
  E_3 &= 3hf_0 = \SI{9.94e-33}\joule
\end{align*}
The gap between different energy levels is $\Delta E=hf=\SI{3.32e-33}\joule$.
Energies between these discrete levels (i.e.\ between $E_1$ and $E_2$, or
between $E_2$ and $E_3$) are not allowed. This means that the possible
amplitudes of the mass can be calculated by substituting the energy levels into
Eq.~\ref{eq:amplitude-from-energy}:
\begin{align*}
  A_0 &= 0\\
  A_1 &= \cdots\\
  A_2 &= \cdots\\
  A_3 &= \cdots
\end{align*}
For a spring-mass system with such a low natural frequency, the energy gap---and
therefore the gaps between amplitudes---is laughably small, which is why energy
quantization cannot be observed in a macroscopic scale. But for a charged
particle emitting light, it will have to vibrate at a frequency between
\num{4e14} to \SI{7e14}\hertz. All of a sudden, the gap $\Delta E$ is no longer
small.

When this mass is oscillating, frictional forces removes energy from the
system. In Chapter 7, we studied the use of a ``damping force'' $F_D=-bv$ to
consolidate the effects of kinetic friction, drag and viscous damping. When
this damping force is present, the total energy $E$ of the system decreases
with time $t$ as an exponential decay function:
\begin{equation}
  E(t)=E_0e^{-\frac bmt} %=E_0e^{-\frac{t}{\tau}}
\end{equation}
where $E_0$ is the initial energy of the system, and $b$ is the damping
constant. For simplicity, we are not showing the derivation of this equation,
except to point out that $E(t)$ is a continuous function in time, as shown in
Fig.\ 15.4a. However, if the energy levels of this harmonic oscillator is
indeed quantized, then $E(t)$ would, in fact, decrease as a step function, as
shown in .%Fig.\ 15.4b.

\section{Photoelectric Effect}

%\section{Maxwell's Equations: Classical Laws of Electrodynamics}  
%%    \pic{.97}{graphics/PORTRAIT-James-Clerk-Maxwell}\\  
%%    {\footnotesize James Clerk Maxwell\par}

Maxwell's equations show that an oscillating charge generates a fluctuating
electric field and magnetic field, which travels through space as an
``electromagnetic wave'' with speed:
%\begin{equation*}
%  c=\frac1{\sqrt{\varepsilon_0\mu_0}}=\SI{299792458}{\meter\per\second}
%\end{equation*}
By 1862, the speed of light was already measured to within \SI{.6}{\percent} of
this value\footnote{By L\'{e}on Foucault using a rotating mirror experiment.
His experimental value was \num{298000}$\pm 500$ \si{\metre\per\second}}. Does
that mean that light is an EM wave?
%
%
%
%

%%\begin{frame}{Proving Light as an Electromagnetic Wave}
Proving that light is an EM wave requires an alternating current with a
%frequency of \SI{e14}{\per\second}.
%\begin{itemize}
%\item Physicists already knew the wavelengths of visible light (e.g.\ through
%  double-slit interference patterns)
%\item The frequency can be determined simply applying $c=f\lambda$
%\end{itemize}
%
%Technology of Maxwell's time could only generate frequencies
%around \SI{e8}{\per\second}
%\begin{itemize}
%\item Much higher than the \SI{60}{\per\second} that our electrical outlet
%  uses
%\item But still one million (\num{e6}) times too low
%\end{itemize}






%%\begin{frame}{Heinrich Hertz}
%%  
%%    \column{.65\textwidth}
%%    \begin{itemize}
%%    \item German physicist (1857-1894)
%%    \item Devised the ``spark gap experiment'' to generate high frequencies
%%    \item The unit for frequency is named after him in his honour
%%    \end{itemize}
%%
%%    \column{.35\textwidth}
%%    \pic{1}{graphics/Heinrich_Rudolf_Hertz}
%%  
%%
%
%
%
%
%\begin{frame}{The Spark Gap Experiment}
German physicist Heinrich Hertz (1857-1894) devised a ``spark gap experiment''
to generate high frequencies required
\begin{figure}[ht]
  \centering
  \pic{.5}{quantumMechanics/graphics/Hertz_exp_2}
\end{figure}
\begin{itemize}
\item Produced EM waves with frequency \num{e14} oscillations per second
\item Showed that light waves have the same wavelengths as predicted by
  Maxwell's equations
\end{itemize}

Hertz left a terse remark in his results:
\begin{center}
  \fbox{
    \begin{minipage}{.75\textwidth}
      \emph{It is essential that the pole surfaces of the spark gap should be
      frequently re-polished to ensure reliable operation of the spark.}
      \end{minipage}
    }
  \end{center}
It had been observed that electrons were leaving the metal surface of the
spark balls (caused by the UV radiation), causing oxidation. This was known
as the \textbf{photoelectric effect}. Hertz and other physicists who repeated
his experiments did not have a satisfactory explanation using classical physics
This is, in fact, the first evidence that light is \emph{not} a wave after all.

Modern demonstrations of the photoelectric effect is generally shown using an
apparatus shown in Fig.~\ref{fig:photoelectric-exp}. When EM waves hits the
(metallic) emitter plate, electrons are knocked off the surface with some
kinetic energy, generating a current.
\begin{figure}[ht]
  \centering
  \begin{tikzpicture}[scale=.7,american voltages]
    \draw[thick,green!70!black,smooth,samples=30,domain=0:3,rotate=40,<-]
    plot(\x,{.15*sin(240*\x)}) node[above]{EM};
    \draw[thick] (.5,1)--(1.5,2);
    \draw[thick] (2,1)--(3,2);
    \draw[thick] (.5,1)--(0,1) arc (90:270:1)--(6,-1) arc (-90:90:1)--(2,1);
    \draw[ultra thick,blue!80!black] (0,.8)--(0,-.8)
    node[below,black]{Emitter};
    \draw[ultra thick,red!80!black] (6,.8)--(6,-.8)
    node[below,black]{Collector};
    \draw[thick] (0,0)--(-1.5,0)--(-1.5,-2.5)
    to[V,l_={Variable voltage}](3,-2.5)
    to[rmeter,t=A,l_={Ammeter}] (7.5,-2.5) --(7.5,0)--(6,0);
    \begin{scope}[blue!40]
      \fill (3,-.5) circle (.1);  \draw[vectors] (3,.-.5)--(4,-.5);
      \fill (1,0) circle (.1);    \draw[vectors] (1,0)--(2,0);
      \fill (2.2,.3) circle (.1); \draw[vectors] (2.2,.3)--(3.2,.3);
      \fill (4.5,.4) circle (.1); \draw[vectors] (4.5,.4)--(5.5,.4);
    \end{scope}
  \end{tikzpicture}
  \caption{Modern apparatus for demonstrating photoelectric effect}
  \label{fig:photoelectric-exp}
\end{figure}
The current can be stopped by adjusting the voltage, called the
\textbf{stopping voltage}, allowing us to measure the kinetic energy of the
\textbf{photo electrons}.

From the classical electrodynamics, energy is transferred continuously from the
light wave to the electrons on the metal surface. Maxwell's equation shows that
the intensity\footnote{Intensity of a wave is defined as the power transmitted
by a wave divided by the area that the wavefront passes through} of the
electromagnetic wave is defined as
\begin{equation*}
  I=\frac{cB_0^2}{2\mu}
\end{equation*}
The power transmitted by the wave depends on the square of the amplitude.
Therefore, based on classical electrodynamics, the kinetic energy of the
photo-electrons should be proportional to the square of the ``brightness''
of the light (i.e.\ intensity), but not the frequency of the light. Therefore,
increasing intensity of the light should:
\begin{itemize}
\item Increases the number of electron emitted
\item Increases the electrons' kinetic energy
\end{itemize}
Even a dim light would \emph{eventually} transfer enough energy to an electron
be emitted. However, \textbf{this is not what is happening!}

\begin{figure}[ht]
  \centering
  \pic{.9}{quantumMechanics/graphics/73bacc9f2bf571752483a89ef6c61a94f07470f7}
\end{figure}
Experimental results show that increasing intensity of light knocked off more
electrons from the metal, but does \emph{not} change their maximum kinetic
energy $K_\text{max}$ of the photo electrons. Curiously, increasing the
\emph{frequency} of the light did change $K_\text{max}$, although below a
certain frequency, \emph{no} electrons were emitted.




%\section{Einstein}
%
%%\begin{frame}{1905: \emph{Annus Mirabilis} (The Miraculous Year)}
%%  {The Year That Einstein Became Very Famous}
%%  \begin{itemize}
%%  \item<alert@2>\textbf{Photoelectric effect:} 
%%  \item\textbf{Brownian motion:} ``On the Motion of Small Particles Suspended
%%    in a Stationary Liquid, as Required by the Molecular Kinetic Theory of
%%    Heat''
%%  \item<alert@1>\textbf{Special relativity:} ``On the Electrodynamics of Moving
%%    Bodies''
%%  \item\textbf{Mass-energy equivalence:} ``Does the inertia of a body depend
%%    upon its energy content?''
%%  \end{itemize}




\subsection{The Photon: Packets of Energy}
In the paper, \emph{On a Heuristic Viewpoint Concerning the Production and
Transformation of Light}, published in 1905, Albert Einstein proposed a
solution to the photoelectric effect: that it can only be explained if light
behaves \emph{not} as a continuous wave, but rather, \emph{a collection of
discrete energy particles} that he called \textbf{photons}, each carrying energy
$E=hf$. In Einstein's model, the photons collide with the electrons, and are
then all of their energy are absorbed into the electron without delay.

The energy of the photon is determined by its \emph{frequency}, in agreement
with Planck's energy quantization equation, therefore, increasing the frequency
of the light increases the kinetic energy $K$ of the ejected photo electrons,
in agreement with experimental results. Additionally, the intensity
(brightness) of light is related to the number of photons. Therefore,
increasing the brightness of the light results in a higher number of electrons
being knocked off the metal surface. This is also in agreement with
experimental results.

The relationship between kinetic energy of electrons and the photons can be
summarized in a simple linear function:
\begin{important-equation}
  \text{Photoelectric effect:}
  \quad
  K_\text{max}=
  \begin{cases}
    hf-\varphi & \text{if}\;\;hf>\varphi\\
    0          & \text{otherwise}
  \end{cases}
  \label{eq:photo-electric-effect}
\end{important-equation}
where $K_\text{max}$ is the maximum kinetic energy of the ``photo-electrons''
ejected from the metal; $h$ is Planck's constant; $f$ is the frequency of the
incident electromagnetic radiation; and $\varphi$ a property of the material,
called the \text{work function}.
%The ``work function'' is a property of the metal that 
The bottom line is that classical electrodynamics concept cannot be applied in
photoelectric effect, and must be replaced with quantum mechanics.




\subsection{Work Function $\varphi$}

\textbf{Work function} $\varphi$ is the minimum energy required to remove an
electron from the metal to a point just outside the metal surface. In other
words, it determines how much energy is absorbed by the metal until an electron
can be knocked off its surface. The lowest frequency at which electrons are
ejected is called the \textbf{threshold frequency} $f_0$. At this frequency,
the photoelectrons have zero kinetic energy. Using
\ref{eq:photo-electric-effect}, we find that expression for threshold
frequency:
\begin{important-equation}
  f_0=\frac{\varphi}h
\end{important-equation}
The slope of the graph is $h$ independent of the metal.

An example for zinc is shown as an example in Fig.~\ref{fig:zinc}. Zinc has a
work function of \SI{4.3}{\electronvolt}, and therefore a threshold frequency of
\SI{1.04e15}\hertz. The frequency is in the ultra-violet range, and therefore
visible light will not cause electrons to be ejected.
\begin{figure}[ht]
  \centering
  \begin{tikzpicture}[xscale=.5,yscale=.65]
    \foreach \wav in {420, 422,...,740}{
      \definecolor{tmpcolor}{wave}{\wav}
      \colorlet{mycolor}[rgb]{tmpcolor}
      \fill[fill=mycolor] ({3000/\wav},-5) rectangle +(.1,7);
    }
    \draw[axes] (0,0)--(15.7,0) node[above]{$f$ ($\times\SI{e14}{\hertz}$)};
    \draw[axes] (0,-5.2)--(0,2.7) node[right]{$\varphi$ ($\si{\electronvolt})$};
    \foreach \x in {2,4,...,14} \draw[thick](\x,0)--(\x,-.2) node[below]{$\x$};
    \foreach \y in {-5,...,2} \draw[thick](0,\y)--(-.2,\y) node[left]{$\y$};

    \draw[ultra thick,red] (0,0)--(10.4,0) node[above]{$f_0$};
    \draw[ultra thick,red,rotate around={22.46:(10.4,0)}]
    (10.4,0)--+(5,0) node[right]{Zn};
    \draw[ultra thick,dashed] (0,-4.3)--(10.4,0);
  \end{tikzpicture}
  \caption{Kinetic energy of photo-electrons for zinc.}
  \label{fig:zinc}
\end{figure}

\begin{table}[ht]
  \centering
  \begin{tabular}{l|c}
    \rowcolor{pink}
    \textbf{Metal} & \textbf{Work function} (\si\electronvolt) \\ \hline
    Aluminum & 4.28 \\ \hline
    Calcium  & 2.87 \\ \hline
    Cesium   & 2.14 \\ \hline
    Copper   & 4.65 \\ \hline
    Iron     & 4.50 \\ \hline
    Lead     & 4.25 \\ \hline
    Lithium  & 2.90 \\ \hline
    Nickel   & 5.15 \\ \hline
    Platinum & 5.65 \\ \hline
    Potassium & 2.30 \\ \hline
    Tin      & 4.42 \\ \hline
    Tungsten & 4.55 \\ \hline
    Zinc     & 4.33 \\
  \end{tabular}
  \caption{Work function for various metals}
  \label{tabl:work-function}
\end{table}


\section{Compton Scattering}

American physicist Arthur Compton\footnote{1892--1962; Nobel Prize winner in
1927} studied x-ray scattering by free electrons. In this case, the
``free electrons'' are actually valence electrons in atoms that have
\emph{very} low ionization energy, so any electromagnetic radiation will knock
them off the atom. After hitting the electrons, the x-ray radiation is
scattered into different angles. The electrons ejected from the atoms are
called \textbf{recoil electrons}.
\begin{center}
  \begin{tikzpicture}
  \draw (0,0)--(-2.5,0);
  \draw[dashed] (0,0)--(2,0);
  \draw[functions,red,smooth,samples=60,domain=-2:-.5]
  plot(\x,{.25*sin(800*\x)});
  \draw[vectors,red] (-1.8,0)--(-.7,0) node[midway,below=9]{incident x-ray};
      
  \begin{scope}[rotate=40]
    \draw (0,0)--(2,0);
    \draw[vectors] (2,0)--+(.7,0);
    \draw[thick,fill=violet] (2,0) circle (.07)
    node[right=3]{recoil electron};
  \end{scope}
  \draw[axes] (1.5,0) arc (0:40:1.5) node[midway,right]{$\phi$};
      
  \begin{scope}[rotate=-50]
    \draw (0,0)--(2,0);
    \draw[functions,orange,smooth,samples=60,domain=.25:1.75]
    plot(\x,{.2*sin(600*\x)});
    \draw[vectors,orange] (.5,0)--(1.5,0)
    node[midway,below=9,rotate=-50]{scattered x-ray};
  \end{scope}
  \draw[axes] (1.2,0) arc (0:-50:1.2) node[midway,right]{$\theta$};
  
  \draw[thick,fill=cyan!90!black] circle (.15) node[above=5]{atom};
\end{tikzpicture}

\end{center}
It was quickly discovered that \emph{classical} theory of electrodynamics
cannot account for the scattering behaviour.
    
Classical theory predicted that:
\begin{itemize}
\item No change in the frequency and wavelength of the scattered x-ray
  (because frequency of a wave should only depend on the source)
\item The motion of the ``recoil'' electrons should be random
\end{itemize}
But what really happened was that:
\begin{itemize}
\item Frequency shift in scattered x-ray depends on scattering angle
  $\theta$
\item Energy of the recoil electrons depends on the recoil angle $\phi$
\end{itemize}


\subsection{Side Track: Using Mass-Energy Equivalence}
Compton turned Einstein's discovery of mass-energy equivalance in 1905, which
drew a link between relativity and quantum mechanics:
\begin{equation}
  E=mc^2
\end{equation}
where $E$ is the total energy of an object, $m$ is the relativistic mass, and
$c$ is the speed of light in a vacuum. This equation can be written in the
\textbf{energy-momentum relation} form:
\begin{equation}
  E^2=p^2c^2+m_0^2c^4
\end{equation}
where $m_0$ is the rest mass of the object. This equation predicted that a
\emph{massless} particle, like a photon, should also have a momentum:
\begin{important-equation}
  \text{Momentum of a photon:}\quad
  p=\frac Ec=\frac{hf}c=\frac h\lambda
\end{important-equation}
By treating the x-ray as particles (photons) with momentum, the electron
scattering problem becomes a basic 2D elastic collision (similar to the
problems studied in Class \#5). In this collision, both momentum and energy
are conserved:
\begin{equation}
  \bm p_i=\bm p_e + \bm p_f\quad\text{and}\quad
  E_i = E + E_f
\end{equation}
\begin{center}
  \begin{tikzpicture}
    \draw (0,0)--(-2.5,0);
    \draw[dashed] (0,0)--(2,0);
    \draw[vectors,red] (-1.8,0)--+(1,0)
    node[midway,below]{\small incident photon};
    \draw[thick,fill=red] (-1.8,0) circle (.1);
      
    \begin{scope}[rotate=40]
      \draw (0,0)--(2,0);
      \draw[vectors] (2,0)--+(.7,0);
      \draw[thick,fill=violet] (2,0) circle (.07)
      node[right=3]{recoil electron};
    \end{scope}
    \draw[axes] (1.5,0) arc (0:40:1.5) node[midway,right]{$\phi$};
      
    \begin{scope}[rotate=-50]
      \draw (0,0)--(2,0);
      \draw[vectors,orange] (.75,0)--+(1,0)
      node[midway,below,rotate=-50]{\small scattered photon};
      \draw[thick,fill=orange] (.75,0) circle (.1);
    \end{scope}
    \draw[axes] (1.2,0) arc (0:-50:1.2) node[midway,right]{$\theta$};
      
    \draw[thick,fill=cyan!90!black] circle (.15) node[above=5]{atom};

    \node[text width=100,color=red] at (-1.8,-.9){\small
      $p_i=\dfrac{E_i}c=\dfrac{hf_i}c=\dfrac h{\lambda_i}$};

    \node[text width=100,color=orange] at (2.5,-.9){\small
      $p_f=\dfrac{E_f}c=\dfrac{hf_f}c=\dfrac h{\lambda_f}$};

    \node[text width=100,color=violet] at (0,1.3){\small
      $p_e=\dfrac{\sqrt{E^2-(m_ec^2)^2}}c$};
  \end{tikzpicture}
\end{center}
The change in wavelength in the x-ray photons is given by:
\begin{equation}
  \Delta\lambda=\lambda_f-\lambda_i=\frac h{m_ec}\left(1-\cos\theta\right)
\end{equation}
The conclusion from the Compton scattering experiment is that, in addition to
energy, photons also have a momentum that is proportional to Planck's
constant and inversely proportional to its wavelength:

%%%    \pic{1.2}{graphics/Arthur_Compton_1927}
%%%
%%%    {\footnotesize Arthur H.\ Compton\par}
%
%American physicist Arthur Compton\footnote{1892--1962; Nobel Prize winner in
%1927}, was studying x-ray scattering by free electrons
%\begin{itemize}
%\item Classical theory cannot account for the scattering behaviour
%\item Frequency shift only depends on scattering angle
%\item Prediction possible if treating the x-ray as photons with
%  momentum, just like a particle
%\item In the \textbf{momentum-energy relation} by Einstein\footnote{From
%the mass-energy equivalence $E=m'c^2$ which can be also expressed as
%$E^2=p^2c^2+m^2c^4$}, a massless particle ($m=0$) has an energy related
%  to its momentum by $E=pc$.
%\end{itemize}
%
%\begin{equation}
%  \boxed{
%    p=\frac Ec=\frac{hf}c=\frac h\lambda
%  }
%\end{equation}
%
%
%
%
%%\begin{frame}{Compton Scattering}
%By treating x-ray as photons (i.e.\ a particle) with momentum $p=h/\lambda$,
%Compton showed that the behaviour of the scattered x-ray and the recoil
%electron is consistent with an \emph{elastic collision}\footnote{Recall from
%Unit 2 that in an elastic collision, both momentum and kinetic energy are
%conserved} between the x-ray photon and the electron.
%\begin{center}
%  \pic{.45}{quantumMechanics/graphics/compton2}
%\end{center}
%%
%%
%%
%%
%%\begin{frame}{Momentum of a Photon}
%The momentum of a photon is proportional to Planck's constant and inversely
%proportional to its wavelength:

%\begin{equation}
%  \boxed{p=\frac h\lambda}
%\end{equation}
%\begin{center}
%  \begin{tabular}{l|c|c}
%    \rowcolor{pink}
%    \textbf{Quantity} & \textbf{Symbol} & \textbf{SI Unit} \\ \hline
%    Momentum          & $p$       & \si{\kilo\gram\metre\per\second} \\
%    Planck's constant & $h$       & \si{\joule\second} \\
%    Wavelength        & $\lambda$ & \si\metre
%  \end{tabular}
%\end{center}



\begin{example}
  Calculate the momentum of a photon of light that has
  frequency of \SI{5.09e14}\hertz.
\end{example}



\section{Matter Waves}

%\begin{frame}{Matter Waves: The De Broglie Hypothesis}
%%  
%%    \pic{1}{graphics/76562-004-66881FD5}\\
%%    {\footnotesize Louis De Broglie}

If electromagnetic waves are really particles of energy, then are particles
(e.g.\ electrons) waves?
\begin{itemize}
\item Louis Broglie\footnote{1892--1987; Nobel Prize winner in 1929. De Broglie
is the only physicist to have won a Nobel Prize for work done in a Ph.D.\
thesis},
  while completing his Ph.D.\ in 1924, proposed a hypothesis: a particle can
  also have a wavelength
\item Confirmed, accidentally, by the Davisson-Germer Experiment in 1927
  (beam of electron scattering on nickel crystal surface)
\end{itemize}




%\begin{frame}{Electron Interference}
%  
%    \column{.84\textwidth}
%    If I perform a double-slit experiment with a beam of electrons, will I get
%    an interference pattern?
%\begin{figure}[ht]
%  \centering
%  \pic{.6}{graphics/CNX_Chem_06_03_Electrnin}
%\end{figure}

\begin{figure}[ht]
  \centering
  \pic1{quantumMechanics/graphics/206px-Double-slit_experiment_results_Tanamura_2}
  \caption{Interference pattern shown for electrons sent through a double-slit
    appratus}
  \label{fig:electron-2slit}
\end{figure}

If matter, like an electron, is also a wave, then it should have a wavelength
too. We can solve momentum equation to find $\lambda$:
\begin{equation}
    p=\frac h\lambda\;\;\rightarrow\;\;
    \lambda=\frac hp\;\;\rightarrow\;\;\boxed{\lambda=\frac h{mv}}
\end{equation}
%  \begin{center}
%    \begin{tabular}{l|c|c}
%      \rowcolor{pink}
%      \textbf{Quantity} & \textbf{Symbol} & \textbf{SI Unit} \\ \hline
%      Wavelength of a particle & $\lambda$ & \si\metre \\
%      Planck's constant & $h$       & \si{\joule\second} \\
%      Mass              & $m$       & \si{\kilo\gram} \\
%      Velocity          & $v$       & \si{\metre\per\second}
%    \end{tabular}
%  \end{center}
This applies if momentum is non-relativistic.





\begin{example}
  Calculate the wavelength of an electron moving with a velocity of
  \SI{6.39e6}{\metre\per\second}.
\end{example}


%\begin{frame}{Particle-Wave Duality}
%  \textbf{The Copenhagen interpretation:}
%  \begin{itemize}
%  \item Accepted view: wave-particle duality
%  \item An experiment can either show:
%    \begin{itemize}
%    \item The wave nature: diffraction, refraction (e.g.\ light)
%    \item The particle nature: scattering (Compton effect), photoelectric
%      effect
%    \item But not both.
%    \end{itemize}
%  \end{itemize}



\section{Uncertainty Principle}
If a particle is a wave, how can you tell where it is? Consider the
one-dimensional wave shown in Fig.~\ref{fig:1d-wave}.
\begin{figure}[ht]
  \centering
  $\Psi(x,t)=\cos(kx-\omega t)$\\
  \begin{tikzpicture}[scale=1.3]
    \draw[xscale=.3,function,smooth,samples=300,domain=-18:18]
    plot(\x,{sin(180*\x)});
  \end{tikzpicture}
  \caption{A one-dimensional harmonic wave with a single wavelength}
  \label{fig:1d-wave}
\end{figure}
At any time $t$, the wave/particle can be described by the wave function
$\Psi(x,t)$. The function has a single wavelength of $\lambda$ (therefore a
single value of momentum $p=\frac h\lambda$), but it has no distinguishing
features that can tell you the particle's position $x$. In fact, the wave
functoin seems to suggest that the particle is \emph{everywhere}. Therefore we
conclude that, \textbf{when we have precise knowledge of a particle wave's
  \emph{momentum}, we have no knowledge of its \emph{position}.}

On the other extreme, consider a particle/wave defined as a
\emph{delta function}, as shown in Fig.~\ref{fig:1d-delta-wave}.
\begin{figure}[ht]
  \centering
  \vspace{-.3in}
  \begin{tikzpicture}[scale=1.2]
    \draw[function] (-5,0)--(-.04,0)--(0,2.3)--(.04,0)--(5,0);
    %\draw[function,smooth,samples=300,domain=-5:5]
    %plot(\x,{2*exp(-\x*\x/0.005)});
    \draw[axes] (3,2) to[out=180,in=90] (0,2.4);
    %\draw[axes] (3,2) to[out=270,in=270] (0,0);
    \node[right] at (3,2){The particle is here!};
  \end{tikzpicture}
  \caption{A one-dimensional delta wave}
  \label{fig:1d-delta-wave}
\end{figure}
The particle's position $x$ is well-defined but its wavelength $\lambda$ (and
therefore momentum $p$) is undefined. (Recall the definition of a wavelength:
it is the shortest distance between two points in a wave that is in phase. In
the case of a pulse wave, no two points are in phase.) The wave function
provides no information about the particle's velocity. Therefore, we conclude
that \textbf{when we have precise knowledge of a particle wave's
  \emph{position},  we have no knowledge of \emph{where} it is going.}

However, if a moving particle has small variations (uncertainties) in its
momentum $p$ (wavelength $\lambda$), as shown Fig.~\ref{fig:packet}.
\begin{figure}[ht]
  \centering
  $\cos(280x)+\cos(281x)+\cdots+\cos(300x)+\cdots+\cos(319x)+\cos(320x)$\\
  \begin{tikzpicture}[scale=1.2]
    \draw[xscale=.3,yscale=.07,function,smooth,samples=360,domain=-15:15]
    plot(\x,{.5*(
      sin(300*\x)+
      sin(301*\x)+ sin(302*\x)+ sin(303*\x)+sin(304*\x)+sin(305*\x)+
      sin(306*\x)+ sin(307*\x)+ sin(308*\x)+sin(309*\x)+sin(310*\x)+
      sin(311*\x)+ sin(312*\x)+ sin(313*\x)+sin(314*\x)+sin(315*\x)+
      sin(316*\x)+ sin(317*\x)+ sin(318*\x)+sin(319*\x)+sin(320*\x)+
      sin(280*\x)+ sin(281*\x)+ sin(282*\x)+sin(283*\x)+sin(284*\x)+
      sin(285*\x)+ sin(286*\x)+ sin(287*\x)+sin(288*\x)+sin(289*\x)+
      sin(290*\x)+ sin(291*\x)+ sin(292*\x)+sin(293*\x)+sin(294*\x)+
      sin(295*\x)+ sin(296*\x)+ sin(297*\x)+sin(298*\x)+sin(299*\x)
      )});
  \end{tikzpicture}
  \caption{A one-dimensional wave packet}
  \label{fig:packet}
\end{figure}
When we add up the different waves together, we begin to see a \textbf{packet}
forming.
\textbf{To gain knowledge about the \emph{position} of a particle, we must
  give up knowledge about its \emph{momentum}, and vice versa.}




%\begin{frame}{Uncertainty Principle}
Werner Heisenberg\footnote{1901--1976, Nobel Prize winner in 1932} showed that
because of wave properties of particles, an observer can never be completely
certain of \emph{both} an object's momentum $p$ and position $x$. The limit
to any measurement is given by the \textbf{uncertainty principle}:
\begin{equation}
  \boxed{\sigma_p\sigma_x\geq \frac h{4\pi}}
\end{equation}
\begin{center}
  \begin{tabular}{l|c|c}
    \rowcolor{pink}
    \textbf{Quantity} & \textbf{Symbol} & \textbf{SI Unit} \\ \hline
    Uncertainty in momentum & $\sigma_p$ & \si{\kilo\gram\metre\per\second}\\
    Uncertainty in position & $\sigma_x$ & \si\metre \\
    Planck's constant       & $h$        & \si{\joule\second}
  \end{tabular}
\end{center}
The more you know about an object's position, the less you know about its
momentum, and vice versa.




\section{Bohr Atomic Model}

At the beginning of the 20th Century, a deeper understanding of the atom begin
to take shape.
\begin{figure}[ht]
  \centering
  \begin{subfigure}{.45\textwidth}
    \centering
    \begin{tikzpicture}
      \filldraw[draw=pink,
        even odd rule,
        inner color=pink,
        outer color=pink!40] circle (2);
      \draw[dotted,thin] (-1.3,0)--(1,0);
      \draw[dotted,thin] (-1.3,0)--(0,-1.3);
      \draw[dotted,thin] (1,0)--(0,-1.3);
      
      \draw[dotted,thin] (-1.3,0)--(.3,.8);
      \draw[dotted,thin] (0,-1.3)--(.3,.8);
      \draw[dotted,thin] (1,0)--(.3,.8);
    
      \filldraw[cyan] (-1.3,0) circle (.1);
      \filldraw[cyan] (1,0) circle (.1);
      \filldraw[cyan] (0,-1.3) circle (.1);
      \filldraw[cyan] (.3,.8) circle (.1);
    \end{tikzpicture}
    \caption{``Plum pudding model'' of J.\ J.\ Thomson in 1904}
    \label{fig:plum-pudding}
  \end{subfigure}
  \begin{subfigure}{.45\textwidth}
  \centering
  \begin{tikzpicture}
    \draw[gray,thin] circle (2);
    \fill[red] circle (.2) node[white,midway]{$+$};
    \foreach \theta in {0,135,...,430}{
      \draw[thick,rotate=\theta] ellipse (2 and .5);
      \filldraw[cyan,rotate=\theta] (-1.3,-.38) circle (.1);
    }
  \end{tikzpicture}
  \caption{``Orbital model'' of Ernest Rutherford from 1911}
  \label{fig:rutherford-model}
  \end{subfigure}
\end{figure}
The plum-pudding model accounts for the existence of the electron, but is
inconsistent with subsequent discovery of the nucleus.


Consistent with the discovery of the nucleus, but difficult to accept because
it assumes that the atom is mostly empty space. Rutherford's orbital model is
more advanced than J.\ J.\ Thomson's model, but it is still inconsistent with
Maxwell's equation:
\begin{itemize}
\item As electrons orbit the nucleus in circular motion, they must have a
  centripetal acceleration
\item Accelerating electrons emit electromagnetic radiation, and lose energy
\end{itemize}
Maxwell's equations predict that electrons' orbits must eventually collapse
\begin{center}
  \pic{.3}{quantumMechanics/graphics/decay}
\end{center}



\subsection{Hydyogen Emission Spectrum}
When a gas is heated, it emits EM radiation at specific wavelengths. For
hydrogen, there are 4 specific wavelengths that are visible to the human eyes.
\begin{center}
  %\pic{.4}{graphics/hydrogen-spectra1}
  \pic{.6}{quantumMechanics/graphics/dab881c4db4639afd5967d5ada40d4d1}
\end{center}
Other gases have different wavelengths (and therefore colours). This is a
topic studied in depth in Chemistry
%\end{frame}
%
%
%\begin{frame}{Hydyogen Emission Spectrum}
The actual colours of the hydrogen emission spectrum looks like this:
%\footnote{And let's be very clear that, despite being able to ``find''
%diagrams of these lines online, there is absolutely no ``purple'' in the
%colours; violet is a blue light, not purple}:
\begin{center}
  \pic{.7}{quantumMechanics/graphics/Visible_spectrum_of_hydrogen}
\end{center}
In 1885, Johann Balmer derived an empirical formula for the wavelengths of
these lights:
\begin{equation}
  \frac1\lambda=R_\text{H}\left[\frac1{2^2}-\frac1{n^2} \right]
\end{equation}
where $n=3,4,\ldots$ is a whole number, and
$R_\text{H}=\SI{1.09677583e7}{\per\metre}$ is the \textbf{Rydberg constant} for
hydrogen. These emission lines are called the \textbf{Balmer series}. Other
wavelengths \emph{outside} of the visible light spectrum were also discovered,
giving the Rydberg formula for all hydrogen-like\footnote{A hydrogen-like atom
is any atom or ion with a single valence electron} gas:
\begin{equation}
  \frac1\lambda=RZ^2\left[\frac1{n_1^2}-\frac1{n_2^2} \right]
\end{equation}



\subsection{Bohr Atomic Model}
In 1918, Niels Bohr\footnote{1885--1962; Nobel Prize winner in 1922}
postulated\footnote{Without evidence, of course!} that there must be
``stationary'' orbits that electrons can orbit in without radiation. Bohr
assumed that:
\begin{itemize}
\item Electrons are held in the atom by electrostatic force %(studied in Class
  %\#8)
\item The electrons are in circular motion, with the electrostatic force
  providing the centripetal forces
\item The ratio of electric potential energy to kinetic energy
  of the electron is $U_q=-2K$
\end{itemize}
%The hydrogen emission spectrum corresponds to energy change when electrons
%dropped from one quantum energy level to the next, emitting photons of
%specific frequencies
Bohr proposed that the energy levels of the electrons must be:
%\begin{equation}
%  \boxed{
%    E_n=-\frac{k^2e^4m}{2\hbar^2}\frac{Z^2}{n^2}
%  }
%\end{equation}
\begin{center}
  \begin{tabular}{l|c|c}
    \rowcolor{pink}
    \textbf{Quantity} & \textbf{Symbol} & \textbf{SI Unit} \\ \hline
    Energy at level $n$ & $E_n$ & \si\joule \\
    Coulomb's constant & $k$ & \si{N.m^2/C^2} \\
    Elementary charge  & $e$ & \si\coulomb \\
    Atomic mass        & $m$ & \si{\kilo\gram} \\
%    Reduced Planck's constant & $\hbar$ & \si{\joule\second}\\
    Atomic number      & $Z$ & integer; no units\\
    Energy level       & $n$ & integer; no units
  \end{tabular}
\end{center}  
%From the wave-particle duality perspective, the ``orbits'' correspond more to
%a standing wave around the nucleus (a standing wave does not lose energy)
%
%
%
%\begin{frame}{Bohr Atomic Model}
%
%  \eq{-.0in}{
%    \boxed{E_n=-\frac{k^2e^4m}{2\hbar^2}\frac{Z^2}{n^2}}
%  }
%
The Bohr model is successful in describing the hydrogen atom and singly-ionized
helium atom, but it fails for heavier atoms. Simplifying the equation for the
hydrogen atom:
\begin{equation}
  E_n=-\frac{13.6}{n^2}
\end{equation}
where $E_n$ is in electron volts (\si{\electronvolt})





\begin{figure}[ht]
  \centering
  \begin{tikzpicture}[scale=1.3]
    \draw[thick] (-2,1)--+(2.5,0) node[right]{$n_i$};
    \draw[thick] (-2,-1)--+(2.5,0) node[right]{$n_f$};
    \draw[thick,fill=violet] (-1,1) circle (.05);
    \draw[thick,fill=violet!20] (-1,-1) circle (.05);
    \draw[vectors](-1,.95)--(-1,-.95);
    \draw[xscale=.1,yscale=.25,thick,red,smooth,samples=300,domain=-9:9]
    plot(\x,{.1*(
      sin(221*\x)+sin(223*\x)+sin(225*\x)+
      sin(226*\x)+sin(228*\x)+sin(230*\x)+
      sin(231*\x)+sin(233*\x)+sin(235*\x)+
      sin(236*\x)+sin(238*\x)+sin(240*\x)+
      sin(241*\x)+sin(243*\x)+sin(245*\x)+
      sin(246*\x)+sin(248*\x)+sin(250*\x)+
      sin(251*\x)+sin(253*\x)+sin(255*\x)+
      sin(256*\x)+sin(258*\x)
    });
    \draw[vectors,red] (1,0)--+(.7,0) node[right]{photon};
  \end{tikzpicture}
\end{figure}
When electrons drop from one energy ($n_i$) to another ($n_f$), a photon is
emitted, with energy (measured in electronvolts):
\begin{equation}
  \Delta E=13.6\left[\frac1{n_f^2}-\frac1{n_i^2} \right]
\end{equation}
The frequency and wavelength corresponding to the emitted photon are given by:
\begin{equation}
  f=\frac{\Delta E}h\quad\to\quad
  \lambda=\frac cf
\end{equation}

Aside from the ``Balmer series'' of hydrogen spectrum, there are 5 other series
that are not in the visible-light spectrum.

\begin{itemize}
\item\textbf{Lyman series} (ultraviolet):
  \begin{itemize}
  \item the EM emissions when the electrons drop from a higher energy state
    ($E_n$) to the ground state $n=1$ (i.e.\ $E_1$)
  \end{itemize}
\item\textbf{Balmer series}: from $E_n$ to $E_2$ (visible light)
\item\textbf{Paschen series}: from $E_n$ to $E_3$ (infrared)
\item\textbf{Brackket series}: from $E_n$ to $E_4$ (infrared)
\item\textbf{Pfund series}: from $E_n$ to $E_5$ (infrared)
\item\textbf{Hymphreys series}: from $E_n$ to $E_6$ (infrared)
\end{itemize}

\begin{center}
  \pic{.4}{quantumMechanics/graphics/400px-Hydrogen_transitions}
\end{center}

When Bohr developed the hydrogen atom model, De Broglie's hypothesis has not
been formulated, but it gives us a glimpse of what Bohr is missing:
\begin{itemize}
\item The ``stationary orbits'' correspond to a standing wave around the
  nucleus %, called a resonance mode
\item A standing wave does not lose energy
\item For example, electron standing wave modes $n=1,\ldots,6$
\end{itemize}

\begin{figure}[ht]
  \centering
  \begin{subfigure}{.3\textwidth}
    \centering
    \begin{tikzpicture}[scale=.6]
      \begin{polaraxis}[
          axis lines=none,
          line width=3
        ]
        \addplot[draw=black,thick,domain=0:360,samples=360] {1}; 
        \addplot[domain=0:360,samples=360] {.2*cos(x)+1};
        \addplot[dashed,domain=0:360,samples=360] {-.2*cos(x)+1};
      \end{polaraxis}
    \end{tikzpicture}
    \caption{$n=1$}
  \end{subfigure}
  \begin{subfigure}{.3\textwidth}
    \centering
    \begin{tikzpicture}[scale=.6]
      \begin{polaraxis}[
          axis lines=none,
          line width=3
        ]
        \addplot[draw=black,thick,domain=0:360,samples=360] {1}; 
        \addplot[domain=0:360,samples=360] {.2*cos(2*x)+1};
        \addplot[dashed,domain=0:360,samples=360] {-.2*cos(2*x)+1};
      \end{polaraxis}
    \end{tikzpicture}
    \caption{$n=2$}
  \end{subfigure}
  \begin{subfigure}{.3\textwidth}
    \centering
    \begin{tikzpicture}[scale=.6]
      \begin{polaraxis}[
          axis lines=none,
          line width=3
        ]
        \addplot[draw=black,thick,domain=0:360,samples=360] {1}; 
        \addplot[domain=0:360,samples=360] {.2*cos(3*x)+1};
        \addplot[dashed,domain=0:360,samples=360] {-.2*cos(3*x)+1};
      \end{polaraxis}
    \end{tikzpicture}
    \caption{$n=3$}
  \end{subfigure}
  \begin{subfigure}{.3\textwidth}
    \centering
    \begin{tikzpicture}[scale=.6]
      \begin{polaraxis}[
          axis lines=none,
          line width=3
        ]
        \addplot[draw=black,thick,domain=0:360,samples=360] {1}; 
        \addplot[domain=0:360,samples=360] {.2*cos(4*x)+1};
        \addplot[dashed,domain=0:360,samples=360] {-.2*cos(4*x)+1};
      \end{polaraxis}
    \end{tikzpicture}
    \caption{$n=4$}
  \end{subfigure}
  \begin{subfigure}{.3\textwidth}
    \centering
    \begin{tikzpicture}[scale=.6]
      \begin{polaraxis}[
          axis lines=none,
          line width=3
        ]
        \addplot[draw=black,thick,domain=0:360,samples=360] {1}; 
        \addplot[domain=0:360,samples=360] {.2*cos(5*x)+1};
        \addplot[dashed,domain=0:360,samples=360] {-.2*cos(5*x)+1};
      \end{polaraxis}
    \end{tikzpicture}
    \caption{$n=5$}
  \end{subfigure}
  \begin{subfigure}{.3\textwidth}
    \centering
    \begin{tikzpicture}[scale=.6]
      \begin{polaraxis}[
          axis lines=none,
          line width=3
        ]
        \addplot[draw=black,thick,domain=0:360,samples=360] {1}; 
        \addplot[domain=0:360,samples=360] {.2*cos(6*x)+1};
        \addplot[dashed,domain=0:360,samples=360] {-.2*cos(6*x)+1};
      \end{polaraxis}
    \end{tikzpicture}
    \caption{$n=6$}
  \end{subfigure}
  \caption{The first 6 resonance mode for electrons in a hydrogen atom}
  
\end{figure}





\section{Particle in a Box}

\begin{figure}[ht]
  \centering
  \begin{tikzpicture}
    \foreach \y in {1,...,5}{
      \draw[dashed](0,\y-.5)--(3,\y-.5) node[pos=0,left]{$n=\y$};
      \draw[very thick,violet,smooth,samples=80,domain=0.01:2.99]
      plot({\x},{.4*sin(60*\y*\x)+\y-.5});
    }
    \fill[gray!40] (-.1,5) rectangle (0,0);
    \fill[gray!40] (-.1,0) rectangle (3.1,-.1);
    \fill[gray!40] (3,-.1) rectangle (3.1,5);
    \draw[thick] (0,5)
    --(0,0) node[pos=0,above]{$\psi(x)$} node[below]{0}
    --(3,0) node[below]{$L$}--(3,5);
  \end{tikzpicture}
  \caption{Standing-wave resonance modes for a particle inside a 1D box}
  \label{fig:particle-in-a-box}
\end{figure}

A particle confined to a 1D box behaves like a standing wave, as shown in
Fig.~\ref{fig:particle-in-a-box}. The resonance modes (frequencies where a
stable standing wave exists) correspond to the wavelengths:
\begin{equation}
  \lambda=\frac{2L}n\quad n=1,2,3\ldots
\end{equation}
and the momentum of the particle at the $n$-th mode is:
\begin{equation}
  p_n=\frac h\lambda=\frac{nh}{2L}
\end{equation}



%\begin{figure}[ht]
%  \centering
%  \begin{tikzpicture}
%    \foreach \y in {1,...,5}{
%      \draw[dashed](0,\y-.5)--(3,\y-.5) node[pos=0,left]{$n=\y$};
%      \draw[very thick,violet,smooth,samples=80,domain=0.01:2.99]
%      plot({\x},{.4*sin(60*\y*\x)+\y-.5});
%    }
%    \fill[gray!40](-.1,5)rectangle(0,0);
%    \fill[gray!40](-.1,0)rectangle(3.1,-.1);
%    \fill[gray!40](3,-.1)rectangle(3.1,5);
%    \draw[thick](0,5)--(0,0) node[pos=0,above]{$\psi(x)$}
%    node[below]{0}--(3,0)node[below]{$L$}--(3,5);
%  \end{tikzpicture}
%\end{figure}   

Kinetic energy of the particle can be expressed in terms of momentum:
\begin{equation}
  E_n=\frac12mv^2=\frac{p_n^2}{2m}=\frac{n^2h^2}{8mL^2}
\end{equation}
If a particle is a standing wave, then kinetic energy of the particle can
never be zero, therefore
\begin{itemize}
\item The particle cannot have zero velocity
\item The lowest energy level ($n=1$) is called the
  \textbf{zero-point energy}
    \end{itemize}

\begin{example}
  A \SI{.150}{\kilo\gram} billiard ball is confined to the pool table
  \SI{1.42}{\metre} wide. How long (in seconds) will it take to travel from one
  side of the table to the other side at fundamental mode?
\end{example}



%\section{Schr\"{o}dinger}
%
%\begin{frame}{What Else Can You Learn from Quantum Mechanics}
%  {Schr\"{o}dinger's Equation}
%  \begin{itemize}
%  \item The differential equations that governs the quantum state of a
%    quantum system changes in time
%  \item It's like what Newton's second law of motion and conservation of
%    energy to classical mechanics
%  \item Gives full details of the behaviour of electrons in atoms
%  \item ``Schr\"{o}dinger's Cat'' thought experiment
%  \end{itemize}
