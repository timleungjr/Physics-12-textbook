\documentclass[11pt,letterpaper]{article}
\usepackage[margin=0.7in,letterpaper]{geometry}
%\usepackage{tikz,graphicx,wrapfig}
\usepackage{mathpazo}
\usepackage[scaled]{helvet}
\usepackage{enumitem}
\usepackage{amsmath}
\usepackage{cancel}
\renewcommand{\familydefault}{\sfdefault}
\newcommand{\e}[1]{\times 10^{#1}}
\newcommand{\half}{\frac{1}{2}}
\newcommand{\dg}{^\circ}
\newcommand{\magdir}[2]{#1\;[\textrm{#2}]}
\newcommand{\mb}[1]{\mathbf{#1}}
\newcommand{\andthen}{\quad\longrightarrow\quad}
\newcommand{\answer}[1]{
  \fbox{
    \begin{minipage}{0.98\textwidth}
      \textbf{Answer:}
      #1
    \end{minipage}
  }
  \vspace{0.05in}
}

\setlength{\parindent}{0pt}
\begin{document}

\begin{center}
  {\Large\textbf{Physics 12 Unit 2 (Momentum and Energy) In-class Examples}}
\end{center}

\textbf{Example 1:} Determine the momentum of a $0.300kg$ hockey puck
travelling across the ice at a velocity of $\magdir{5.55m/s}{N}$.\\
\answer{
  \begin{displaymath}
    \mb{p}=m\mb{v}=0.300kg\times 5.55m/s=\magdir{1.67kg\cdot m/s}{N}
  \end{displaymath}
}

\textbf{Example 2:} If a golf club exerts an average force of
$\magdir{5.25\e{3}N}{W}$ on a golf ball over a time interval of $5.45\e{-4}s$,
what is the impulse of the interaction?\\
\answer{
  \begin{displaymath}
    \mb{J}=\Delta\mb{p}=\mb{F}\Delta t=
    \magdir{5.25\e{3}N}{W}\times 5.45\e{-4}s
    =\magdir{2.86kg\cdot m/s}{W}
  \end{displaymath}
}

\textbf{Example 3:} A student practices her tennis volleys by hitting a tennis
ball against a wall.
\begin{itemize}[noitemsep]
\item If the $0.060kg$ ball travels $48m/s$ before hitting the wall and then
  bounces directly backwards at $35m/s$, what is the impulse of the interaction?
\item If the duration of the interaction if $25ms$, what is the average force
  exerted on the ball by the wall?
\end{itemize}
\answer{Let's call the direction towards the wall positive; backwards is
  negative.
  \begin{displaymath}
    J=\Delta p=m(v_2-v_1)=0.060kg\times\left(-35-48\right)
    =\boxed{-4.9kg\cdot m/s}\;\;\text{or}\;\;
    \boxed{\magdir{4.9kg\cdot m/s}{backwards}}
  \end{displaymath}
  To find the force:
  \begin{displaymath}
    J=F\Delta t\andthen F=\frac{J}{\Delta t}=\frac{-4.9kg\cdot m/s}{0.025s}
    =\boxed{-1.97\e{2}N}\;\;\text{or}\;\;
    \boxed{\magdir{1.97\e{2}N}{backwards}}
  \end{displaymath}
}

\textbf{Example 4:} A $1.75\e{4}kg$ boxcar is rolling down a track towards a
stationary boxcar that has a mass of $2.00\e{4}kg$. Just before the collision,
the first boxcar is moving east at $5.45m/s$. when the boxcars collide, they
lock together and continue to down the track. What is the velocity of the two
boxcars immediately after the collision?\\
\answer{
  Momentum equation. B is the stationary boxcar
  \begin{align*}
    m_Av_A+\cancel{m_Bv_B}&=m_Av_A'+m_Bv_B'
    \quad\textrm{where}\quad v_A'=v_B'=v'\\
    m_Av_A&=(m_A+m_B)v'\\
    v'&=\frac{m_Av_A}{m_A+m_B}
    =\frac{2.00\e{4}\times 5.45}{(1.75+2.00)\e{4}}
    =\boxed{\magdir{2.91m/s}{forward}}
  \end{align*}
  Should point out that this is a ``completely inelastic'' collision.
}

\textbf{Example 5:} If two people A and B stand in a canoe on top of the
water. Find the velocity of the canoe and person B at the instant that person A
start to take a step, if her velocity is $\magdir{0.75m/s}{forward}$.
Assume person A has a mass of $65kg$ and the combined mass of the canoe, A and
B is $115kg$.\\
\answer{The mass of person B plus the canoe is $115-65=50kg$. Momentum equation:
  \begin{align*}
    \cancel{m_Av_A}+\cancel{m_Bv_B}&=m_Av_A'+m_Bv_B'\\
    v_B'&=\frac{-m_Av_A'}{m_B}
    =\frac{-65\times 0.75}{50}
    =-0.975=\boxed{\magdir{0.975m/s}{backwards}}
  \end{align*}
  Remember that until A starts walking, neither A nor B are in motion, i.e.\
  the momentum $p_A$ and $p_B$ are both zero at the beginning. Also note that
  ``backwards'' is from the point of view of A.
}

\newpage 
\textbf{Example 6:} A $0.0520kg$ golf ball is moving east with a velocity of
$2.10m/s$ when it collides, head on, with a stationary $0.155kg$ billiard ball.
If the golf ball rolls directly backward with a velocity of $-1.04m/s$, was the
collision elastic?\\
\answer{Let east to be positive direction. ``G'' is the golf ball; ``B'' is
  the billiard ball. Momentum equation:
  \begin{align*}
    m_Gv_G+\cancel{m_Bv_B}&=m_Gv_G'+m_Bv_B'\\
    v_B'&=\frac{m_G(v_G-v_G')}{m_B}
    =\frac{0.0520\times\left(2.10-(-1.04)\right)}{0.155}=\magdir{1.05m/s}{E}
  \end{align*}
  Now that we know the velocity of the billiard ball, we can compute the
  initial and final kinetic energy:
  \begin{align*}
    \text{Before:}\quad&\half m_Gv_G^2=\half\times 0.052\times 2.10^2
    =0.1147J\\
    \text{After:}\quad&\half m_Gv_G'^2+\half m_Bv_B'^2
    =\half\times 0.052\times (-1.04)^2+\half\times 0.155\times 1.05^2=
    0.1135J
  \end{align*}
  There is less kinetic energy after the collision, therefore it's
  \textbf{not elastic}.
}

\textbf{Example 8:} A forensic expert needed to find the velocity of a bullet
fired from a gun in order to predict the trajectory of a bullet. He fired a
$5.50g$ bullet into a ballistic pendulum with a bob that had a mass of
$1.75kg$. The pendulum swung to a height of $12.5cm$ above its rest position
before dropping back down. What was the velocity of the bullet just before it
hit and became embedded in the pendulum bob?\\
\answer{We work backwards to solve the problem. Assuming that the pendulum
  doesn't lose any energy, if it swings to a height of $12.5cm=0.125m$, that
  means that right after the bullet strikes it, it had kinetic energy:
  \begin{displaymath}
    E_k=\half\cancel{m_{b+p}}v_\mathrm{after}^2=\cancel{m_{b+p}}g\Delta h\andthen
    v_\mathrm{after}=\sqrt{2g\Delta h}=\sqrt{2\times 9.81\times 0.125}=1.566m/s
  \end{displaymath}
  Now solve the momentum equation for when the bullet hits the pendulum:
  \begin{align*}
    m_bv_b+\cancel{m_pv_p}&=m_bv_b'+m_pv_p'
    \quad\text{where}\quad v_b'=v_p=v_\mathrm{after}\\
    v_b&=\frac{(m_b+m_p)v_\mathrm{after}}{m_b}
    =\frac{(1.75+0.0055)\times 1.566}{0.0055}=\boxed{500m/s}
  \end{align*}
  $500m/s=1800km/h$ is typical for a bullet fired from a high-powered rifle.
}

\textbf{Example 9:} A block of wood with a mass of $0.500kg$ slides across the
floor toward a $3.50kg$ block of wood. Just before the collision, the small
block is travelling at $3.15m/s$. Because some nails are sticking out of the
blocks, the blocks stick together when they collide. Scratch marks on the floor
show that they slid $2.63cm$ before coming to a stop. What is the coefficient
of friction between the wooden blocks and the floor?\\
\answer{Start with the collision. Momentum equation:
  \begin{align*}
    m_av_a+\cancel{m_bv_b}&=m_av_a'+m_bv_b'
    \quad\text{where}\quad v_b'=v_p=v'\\
    v'&=\frac{m_av_a}{m_a+m_b}
    =\frac{0.500\times 3.15}{3.50+0.500}=0.39375m/s
  \end{align*}
  Now that we have the velocity after the collision, we can use kinematic
  equations to compute the acceleration:
  \begin{displaymath}
    \cancel{v_2^2}=v_1^2+2a\Delta d\andthen
    a=-\frac{v_1^2}{2\Delta d}=-\frac{0.39375^2}{2\times 0.0263}=-2.9475m/s^2
  \end{displaymath}
  The acceleration is due to friction, so we can find the friction force and
  coefficient:
  \begin{displaymath}
    F_f=ma\andthen \mu F_n=ma\andthen \mu\cancel{m}g=\cancel{m}a\andthen
    \mu=\frac{a}{g}=\frac{2.9475}{9.81}=\boxed{0.300}
  \end{displaymath}
}

\textbf{Example 10:} A woman pushes a lawnmower with a force of $150N$ at an
angle of $35\dg$ down from the horizontal. The lawn is $10.0m$ wide and
required $15$ complete trips across the back. How much work does she do?\\
\answer{Complete trip is $10.0\times 2=20.0m$, so $\Delta d=300m$.
  \[ W=F\Delta d\cos35\dg=150\times 300\times\cos35\dg=\boxed{36.8\e{4}J}\]
}

\textbf{Example 11:} You drive a nail horizontally into a wall, using a
$0.448kg$ hammerhead. If the hammerhead is moving horizontally at $5.5m/s$ and
in one blow drives the nail into the wall a distance of $3.4cm$, determine the
average force acting on 
\begin{itemize}[noitemsep]
\item The hammerhead
\item The nail
\end{itemize}
\answer{The work done on the hammerhead is the change in it's kinetic energy.
  After the hammerhead drives the nail into the wall, it has zero speed
  $v_2=0$:
  \[ W=\Delta E_k=\cancel{\half mv_2^2}-\half mv_1^2
  =-\half\times 0.448\times 5.5^2=-6.776J\]
  Work is negative, i.e.\ force is opposite the direction of travel $\Delta d$.
  \[ W=F_\mathrm{hammerhead}\Delta d\andthen
  F_\mathrm{hammerhead}=\frac{W}{\Delta d}=\frac{-6776}{0.034}
  =\boxed{-199N}\;\;\text{or}\;\;\boxed{\magdir{199N}{backwards}}\]
  Applying Newton's 3rd law, the force on the nail is equal in magnitude and
  opposite in direction, i.e.:
  \[ F_\mathrm{nail}=\boxed{\magdir{199N}{forward}}\]
}

\textbf{Example 12:} A gas-powered winch on a rescue helicopter does
$4.20\e{3}J$ of work while lifting a $50.0kg$ swimmer at a constant speed up
from the ocean. Through what height was the swimmer lifted?\\
\answer{
  \begin{displaymath}
    W=\Delta E_g=mg\Delta h\andthen
    \Delta h=\frac{W}{mg}=\frac{4.20\e{3}}{50.0\times 9.81}=\boxed{8.56m}
  \end{displaymath}
}

\textbf{Example 13:} A skier is gliding along with a speed of $2.00m/s$ at the
top of a ski hill, $40.0m$ high. The skier then begins to slide down the icy
(friction-less) hill.
\begin{itemize}[noitemsep]
\item What will be the skier's speed at a height of $25.0m$?
\item At what height will the skier have a speed of $10.0m/s$?
\end{itemize}
\answer{Assuming there are no other losses in energy, then at $25,0m$ height:
  \begin{align*}
    \half\cancel{m}v_1^2+\cancel{m}gh_1&=
    \half\cancel{m}v_2^2+\cancel{m}gh_2\\
    v_2&=\sqrt{v_1^2+2g(h_1-h_2)}=\boxed{17.3m/s}
  \end{align*}
  When the velocity is $10.0m/s$:
  \begin{align*}
    \half\cancel{m}v_1^2+\cancel{m}gh_1&=
    \half\cancel{m}v_2^2+\cancel{m}gh_2\\
    h_2&=\frac{1}{2g}(v_1^2-v_2^2)+h_1=\boxed{35.0m}
  \end{align*}
  
}

\textbf{Example 14:} A typical compound archery bow requires a force of
$133N$ to hold an arrow at ``full draw'' (pulled back $71cm$). Assuming that
the bow obeys Hooke's law, what is its spring constant?\\
\answer{
  \begin{displaymath}
    F_a=kx\andthen k=\frac{F_a}{x}=\frac{133}{0.71}=\boxed{187N/m}
  \end{displaymath}
}
  
\textbf{Example 15:} A spring with spring constant of $75N/m$ is resting on a
table.
\begin{itemize}[noitemsep]
\item If the spring is compressed a distance of $28cm$, what is the increase in
  its potential energy?
\item What force must be applied to hold the spring in this position?
\end{itemize}
\answer{Elastic potential energy:
  \begin{displaymath}
    E_e=\half kx^2=\half\times 75\times 0.28^2=\boxed{2.94J}
  \end{displaymath}
  Applied force:
  \begin{displaymath}
    F_a=kx=75\times 0.28=\boxed{21N}
  \end{displaymath}
}

\textbf{Example 16:} A low-friction cart with a mass of $0.25kg$ travels along
a horizontal track and collides head on with a spring that has a spring
constant of $155N/m$. if the spring was compressed by $6.0cm$, how fast was the
cart initially travelling?\\
\answer{When the spring is fully compressed, the cart will no longer have any
  speed, i.e.\ all the kinetic energy from the cart is transferred to the
  elastic potential of the spring:
  \begin{displaymath}
    \half mv^2=\half kx^2\andthen
    v=\sqrt{\frac{kx^2}{m}}=\sqrt{\frac{155\times 0.06^2}{0.25}}=\boxed{1.49m/s}
  \end{displaymath}
}

\textbf{Example 17:} A freight elevator car with a total mass of $100.0kg$
is moving downward at $3.00m/s$, when the cable snaps. The car falls $4.00m$
onto a huge spring with a spring constant of $8.00\e{3}N/m$. By how much
will the spring be compressed when the car reaches zero velocity?\\
\answer{The total energy of the elevator is conserved, i.e.
  \begin{displaymath}
    \half mv_1^2+mg\Delta h_1+\cancel{\half kx_1^2}=
    \cancel{\half mv_2^2}+mg\Delta h_2+\half kx_2^2
  \end{displaymath}
  Note that initially, the spring is not compressed ($x_1=0$), and after the
  elevator falls onto the spring, it no longer has any speed ($v_2=0$).
  If we assume that $h=0$ at the moment that the elevator touches the
  spring, then $h_1=4.0$, $h_2=-x$. Substituting values, we have:
  \begin{align*}
    \half mv_1^2+mg\Delta h_1&= -mgx+\half kx_2^2\\
    \half(100)(3.00^2)+(100)(9.81)(4.00)&=-(100)(9.81)x+\half(8.00\e{3})x_2^2\\
    4000x^2-981x-4374&=0\\
    x&=\boxed{1.18m}\;\;\text{or}\;\;\cancel{-0.93m}
  \end{align*}
  I can easily selected $h=0$ to be the point where the spring is fully
  compressed, in that case, $h_2=0$ and $h_1=4.0+x$. I'll have to solve a
  different quadratic equation, buy I'll get the same answer for $x$. In fact,
  I can also say that $h=0$ to be the initial height, i.e. $h_1=0$, and
  $h_2=-4.0-x$. I will still have the same answer!
}

\end{document}
