\chapter{Magnetism}


%\section{Magnetic Field from Permanent Magnets}
%  Your experience with magnetism is most likely with \textbf{permanent
%    magnets}, for example a bar magnet (below), or a horseshoe magnet, or even
%  a fridge magnet.
%  \begin{center}
%    \pic{.3}{graphics/Bar_magnet_crop}
%  \end{center}
%  Permanent magnets have a \textbf{north pole} and a \textbf{south pole}, which
%  are related to the north and south poles of Earth.
%
%
%
%
%\section{Magnetic Field}
%  Permanent magnets generate a \textbf{magnetic field}. Outside the magnet,
%  field lines run from the north pole toward the south pole; inside the magnet,
%  from south towards the north.
%  \begin{center}
%    \pic{.34}{graphics/iron-filings}
%  \end{center}
%  Magnetic field lines do not have beginnings or ends
%
%
%
%
%  In permanent magnets, like poles repel, while opposite poles attract. The
%  force that magnets exert on each other is called the \textbf{magnetic force}.
%  \begin{center}
%    \begin{tikzpicture}[scale=.6]
%      \fill[red] (1.5,0) rectangle (3,1);
%      \draw[thick] rectangle (3,1);
%      \node[left] at (3,.5){\Large N};
%      \node[right] at (0,.5){\Large S};
%      \draw[vectors] (3,.5)--(5,.5) node[midway,above]{$F_m$};
%      
%      \fill[red] (9.5,0) rectangle (11,1);
%      \draw[thick] (8,0) rectangle (11,1);
%      \node[left] at (11,.5){\Large N};
%      \node[right] at (8,.5){\Large S};
%      \draw[vectors] (8,.5)--(6,.5) node[midway,above]{$F_m$};
%    \end{tikzpicture}
%
%    \begin{tikzpicture}[scale=.6]
%      \fill[red] (1.5,0) rectangle (3,1);
%      \draw[thick] rectangle (3,1);
%      \node[left] at (3,.5){\Large N};
%      \node[right] at (0,.5){\Large S};
%      \draw[vectors] (0,.5)--(-2,.5) node[midway,above]{$F_m$};
%        
%      \fill[red] (9.5,0) rectangle (8,1);
%      \draw[thick] (8,0) rectangle (11,1);
%      \node[left] at (11,.5){\Large S};
%      \node[right] at (8,.5){\Large N};
%      \draw[vectors] (11,.5)--(13,.5) node[midway,above]{$F_m$};
%    \end{tikzpicture}
%
%    \begin{tikzpicture}[scale=.6]
%      \fill[red] (1.5,0) rectangle (0,1);
%      \draw[thick] rectangle (3,1);
%      \node[left] at (3,.5){\Large S};
%      \node[right] at (0,.5){\Large N};
%      \draw[vectors] (0,.5)--(-2,.5) node[midway,above]{$F_m$};
%        
%      \fill[red] (9.5,0) rectangle (11,1);
%      \draw[thick] (8,0) rectangle (11,1);
%      \node[left] at (11,.5){\Large N};
%      \node[right] at (8,.5){\Large S};
%      \draw[vectors] (11,.5)--(13,.5) node[midway,above]{$F_m$};
%    \end{tikzpicture}
%  \end{center}
%  When a magnet is in the magnetic field created by another magnet, the
%  magnetic force is generated. The strength of the magnetic force of varies
%  inversely as the square of the distance between the magnets (inverse square
%  law)
%
%
%
%
%\section{Magnetic Force}
%  
%    \pic1{graphics/visualizing-magnetic-fields-magnet-compasses-brightened}
%
%    Objects that are not magnets, but made from material that can be magnetized
%    (e.g.\ iron, nickel, cobalt) also experience a magnetic force when in a
%    magnetic field
%    \begin{itemize}
%    \item Compass needles orient themselves in the direction of the magnetic
%      field
%    \item The atoms \emph{temporarily} align themselves to the external
%      magnetic field, and therefore becomes a magnet
%    \end{itemize}
%  
%
%
%
%
%
%
%\section{Electromagnets}
A popular experiment in magnetism in elementary schools:
\begin{itemize}
\item Wrap a copper wire around an iron nail/screw
\item Connect the ends of the wire to a battery
\item When the circuit is closed, the nail becomes magnetized, picks up
  small paper clips
\item As soon as the current stops, the nail stops being magnetic
\end{itemize}
\begin{center}
  \pic{.5}{magneticField/graphics/IMG_2343}
\end{center}




\section{Magnetic Field Generated By an Infinitely Long Wire}
%  
%    \pic{1}{graphics/magcur2}
%    
For an \emph{infinitely-long straight} wire carrying a current, the magnetic
field outside wire has a magnitude of:
\begin{equation}
  \boxed{
    B=\frac{\mu_0I}{2\pi r}
  }
\end{equation}
%    \begin{center}
%      \begin{tabular}{l|c|c}
%        \rowcolor{pink}
%        \textbf{Quantity} & \textbf{Symbol} & \textbf{SI Unit} \\ \hline
%        Magnitude of magnetic field & $B$ & \si\tesla \\
%        Electric current            & $I$ & \si\ampere \\
%        Radial distance from the wire & $r$ & \si\metre \\
%        Permeability of free space & $\mu_0$ & \si{\tesla\metre\per\ampere}
%      \end{tabular}
%    \end{center}
%    The constant $\mu_0=4\pi\times\SI{e-7}{T.m/A}$ is the
%    \textbf{permeability of free space}
%  
%
%
%
%
%\section{Magnetic Field Generated By a Wire}
%  
%    \pic{1}{graphics/magcur2}
%    
%    \begin{itemize}
%    \item Both electric current ($\bm I$) and magnetic field ($\bm B$) are
%      vectors
%    \item Direction of $\bm I$ based on movement of positive charges
%      \begin{itemize}
%      \item In an \emph{actual} wire, negative charges (electrons) are moving
%        in the opposite direction
%      \end{itemize}
%    \item Direction of $\bm B$ is determined using \textbf{right hand rule}
%    \end{itemize}
%  
%
%
%
%%\section{Current-Carrying Wire Loop}
%%  
%%    \pic{1}{graphics/curloo}
%%
%%    When we shape the current-carrying wire into a loop, we can (again) use
%%    the Biot-Savart law to find the magnetic field away from it.
%%
%%    \vspace{.2in}
%%    One loop isn't very interesting (except when you're integrating Biot-Savart
%%    law) but what if we have many loops
%%  
%%
%
%
%
%\section{Wounding Wires Into a Coil}
%  When a current-carrying wire is wound into a coil, it becomes a
%  \textbf{solenoid}
%  \begin{itemize}
%  \item The magnetic field from a solenoid is very similar to a bar magnet,
%    with a north and south pole
%  \item Magnetic field inside the solenoid is uniform
%  \item Magnetic field strength can be increased by the addition of an iron core
%  \end{itemize}
%  \begin{center}
%    \pic{.5}{graphics/barsol}
%  \end{center}
%
%
%
%
\section{Magnetic Field Inside a Solenoid}

\begin{figure}[ht]
  \centering
  \pic{.75}{magneticField/graphics/magneticfield4}
\end{figure}
The magnetic field $\bm B$ \emph{inside} a solenoid is uniform (i.e.\
constant), with its strength given by:
\begin{equation}
  \boxed{B=\frac{\mu NI}\ell}
\end{equation}
where $\mu$ is the effective permeability, $N$ is the number of turns in the
solenoid, $I$ is the amount of current through the solenoid, and $\ell$ is the
length of the solenoid. The direction of $\bm B$ is determined by right-hand
rule.
%\begin{center}
%  \begin{tabular}{l|c|c}
%    \rowcolor{pink}
%    \textbf{Quantity} & \textbf{Symbol} & \textbf{SI Unit} \\ \hline
%    Magnetic field intensity    & $B$    & \si\tesla \\
%    Number of turns in the coil & $N$    & \\
%    Length of the solenoid      & $\ell$ & \si\metre \\
%    Electric current            & $I$    & \si\ampere \\
%    Effective permeability      & $\mu$  & \si{\tesla\metre\per\ampere}
%  \end{tabular}
%\end{center}
%  
%
%
%
%
%%\section{A Practical Solenoid}
A practical solenoid, shown in Fig.~\ref{fig:practical-solenoid}, usually has
hundreds of turns:
\begin{figure}[ht]
  \centering
  \pic{.45}{magneticField/1020201515330450255}
  \caption{A practical solenoid used in high-school and university experiments}
  \label{fig:practical-solenoid}
\end{figure}

%%  \vspace{-.2in}
%%  This ``air core'' coil is used for high school and university experiments. It
%%  has approximately 600 turns of copper wire wound around a plastic core.
%%
%
%
%
%\section{Magnetic Fields from Permanent Magnets}
%
%\section{Permanent Magnets}
%  Permanent magnets is also based on the motion of charges.
%  \begin{enumerate}
%  \item Electrons in an atom \emph{spin}. A spinning electron generates a tiny
%    magnetic field. An electron also spins around the nucleus, generating an
%    orbital magnetic field. (This behaviour is best explained by quantum
%    mechanics)
%    \begin{center}
%      \pic{.3}{graphics/Electron-spin}
%    \end{center}
%    In most full ``shells'', the spin of these electrons are paired, so the
%    magnetic fields cancel each other.
%  \end{enumerate}
%
%
%
%
%\section{Permanent Magnets}
%  \begin{enumerate}
%    \setcounter{enumi}{1}
%  \item However, electron ``orbits'' are not always filled, therefore some
%    atoms create a (very small) magnetic field.
%    \begin{center}
%      \pic{.4}{graphics/paramagnetic}
%    \end{center}
%    The atoms that have unpaired electrons are \textbf{paramagnetic} because
%    they are attracted to magnets; atoms that have no unpaired electrons are
%    \textbf{diamagnetic}.
%  \end{enumerate}
%
%
%
%
%\section{Permanent Magnets}
%  \begin{enumerate}
%    \setcounter{enumi}{2}
%  \item While many atoms exhibit paramagnetism, they do not necessarily make
%    good magnets, because the atoms are most likely arranged such that the
%    magnetic fields from each atom cancel.
%    \begin{center}
%      \begin{tikzpicture}
%        \foreach \x in {0,2,...,8}{
%          \shade[ball color=gray] (\x,0) circle (.2);
%          \draw[vectors] (\x,-.5)--(\x,.7);
%        }
%        \foreach \x in {1,3,5,7}{
%          \shade[ball color=red] (\x,0) circle (.2);
%          \draw[vectors] (\x,.5)--(\x,-.7);
%        }
%      \end{tikzpicture}
%    \end{center}
%    When they do not cancel, then they can become magnets. This is called
%    \textbf{ferromagnetism}:
%    \begin{center}
%      \begin{tikzpicture}
%        \foreach \x in {0,1,...,8}{
%          \shade[ball color=gray] (\x,0) circle (.2);
%          \draw[vectors] (\x,-.5)--(\x,.7);
%        }
%      \end{tikzpicture}
%    \end{center}
%    Transitional elements such as iron, nickel and cobalt, and some of their
%    alloys exhibit ferromagnetism.
%  \end{enumerate}
%
%
%
%
%\section{Permanent Magnets}
%  \begin{enumerate}
%    \setcounter{enumi}{3}
%  \item Atoms in these ferromagnetic materials are arranged in ``domains''
%    where the atoms' magnetic moments are aligned. In the presence of a strong
%    external magnetic field, these domains will line up, creating a permanent
%    magnet.
%    \begin{center}
%      \pic{.6}{graphics/domain}
%    \end{center}
%  \end{enumerate}
%
%
%
%    
%\section{Earth}
%  Earth is also a ``permanent'' magnet, with the \emph{magnetic south pole}
%  located near the geographic north pole, and the \emph{magnetic north pole}
%  located near the geographic south pole. The poles are tilted by
%  $\approx\ang{11}$ from the spin axis.
%  \begin{center}
%    \pic{.5}{graphics/mearthbar}
%  \end{center}
%  \vspace{-.15in}The exact nature of Earth's magnetic field is not known,
%  although it may be related to ``generator effect'' from Earth's rotation,
%  circulating the outer-core fluid around.
%
%
%
%
\section{Magnetic Force}

%\section{So What Does the Magnetic Field Do?}
%  
%    \begin{center}
%      Gravitational Field $\bm g$
%    \end{center}
%    \begin{itemize}
%    \item Generated by massive objects
%    \item Affects massive objects
%    \end{itemize}
%
%    \begin{center}
%      Electric Field $\bm E$
%    \end{center}
%    \begin{itemize}
%    \item Generated by charged particles
%    \item Affects charged particles
%    \end{itemize}
%
%    \begin{center}
%      Magnetic Field $\bm B$
%    \end{center}
%    \begin{itemize}
%    \item Generated by \emph{moving} charged particles
%    \item Affects moving charged particles
%    \end{itemize}
%  
%
%
%
%
%\section{Force on a Moving Charge in a Magnetic Field}
When a moving charge ($q$) enters a magnetic field ($\bm B$) with a velocity
$\bm v$, the magnetic field exerts a force ($\bm F_m$) on the charge, with
a magnitude of:
\begin{equation}
  \boxed{
    F_m=qvB\sin\theta
  }
\end{equation}
where $F_m$ is the magnitude of magnetic force on the moving charge, $q$ is
the amount of electrical charge in motion; $v$ is the speed of the moving
charged particle; $B$ is the strength (magnitude) of the external magnetic
field; and $\theta$ is the angle between velocity and magnetic field
vectors. The direction of $\bm F_m$ is perpendicular to both $\bm v$ and
$\bm B$. 
\begin{remark}
  For those who have a stronger background with vectors, the magnetic force
  is more appropriately expressed in terms of the cross product:
  \begin{equation*}
    \bm F_m=q(\bm v\times\bm B)
  \end{equation*}
  Note what whenever a ``right-hand rule'' is invoked in this chapter, there
  is actually a cross product being applied.
\end{remark}



%\section{Convention for Diagrams}
%  \begin{center}
%    \pic{.7}{graphics/sign-convention}
%  \end{center}



\begin{example}
  A particle carrying a charge of \SI{2.5e-6}{\coulomb} enters a magnetic field
  travelling at \SI{3.4e5}{\metre\per\second} to the right of the page. If a
  uniform magnetic field is pointing directly into the page and has a strength
  of \SI{.50}\tesla, what is the magnitude and direction of the force acting on
  the charge as it just enters the magnetic field?
  \begin{center}
    \begin{tikzpicture}[scale=.75]
      \foreach \x in {0,.7,...,4.2}{
        \foreach \y in {0,.7,...,4.2} \node[violet] at (\x,\y) {$\times$};
      }
      \draw[red!70!black,thick] (0,2.5) circle (.2) node{$+$};
      \draw[red!70!black,vectors] (.2,2.5)--(1.7,2.5) node[right]{$\bm v$};
    \end{tikzpicture}
  \end{center}
\end{example}  
%
%

%
%\section{Force on a Current-Carrying Conductor in a Magnetic Field}
Likewise, the magnetic field also exerts a force on a conductor carrying a
current. The magnitude of the magnetic force $\bm F_m$ on the a
current-carrying wire in a magnitude field $\bm B$ is given by:
\begin{equation}
  \boxed{F_m=I\ell B\sin\theta}
\end{equation}
where $I$ is the amount of current in the wire, $\ell$ is the length of
the conducting wire inside the magnitude field, and $\theta$ is the angle
between the direction of (conventional) current and the magnitude field.
\begin{remark}
  Again, for those who have a stronger background with vectors, the magnetic
  force on the current-carrying wire is more appropriately expressed in terms
  of the cross product:
  \begin{equation*}
    \bm F_m=I(\bm\ell\times\bm B)
  \end{equation*}
  Further, if you have some background in integral calculus, the above equation
  is more appropropriately expresssed as:
  \begin{equation*}
    \bm F_m=\int d\bm F_m \quad\text{where}\quad
    d\bm F_m=I(\bm d\ell\times\bm B)
  \end{equation*}
\end{remark}

%      \textbf{Quantity} & \textbf{Symbol} & \textbf{SI Unit} \\ \hline
%      Magnetic force on the conductor   & $F_m$ & \si\newton \\
%      Electric current in the conductor & $I$   & \si\ampere \\
%      Length of the conductor inside magnetic field & $\ell$ & \si\metre \\
%      Magnetic field strength           & $B$   & \si\tesla \\
%      Angle between conductor and magnetic field & $\theta$ &
%    \end{tabular}
%  \end{center}
%  This should not come as a surprise as a current is just a stream of charged
%  particles
%
%
%
%
%\section{Right Hand Rule for Induced Magnetic Force}
%  \begin{center}
%    \pic{.55}{graphics/right-hand-rule}
%  \end{center}
%  \begin{itemize}
%  \item\vspace{-.2in}The direction of current is the \emph{conventional
%    current}, which assumes the flow of positive charges. In fact, in a
%    conductor, negatively charged electrons flow in the opposite direction.
%  \item Single charged particle follows the same convention.
%  \end{itemize}




\begin{example}
  A wire segment of length \SI{40}{\centi\metre}, carrying a current of
  \SI{12}\ampere, crosses a magnetic field of \SI{.75}{\tesla} [up] at an
  angle of [up \ang{40} right].  What magnetic force is exerted on the wire?
\end{example}


\section{Definition of Charge and Current}
The definition of one \emph{amp\`{e}re} of current actually comes from the
above example. When two wires are placed at \SI1{\metre} apart, with the same
amount of current flowing in the same direction. When the current is
\SI1{\ampere}, they will exert a magnetic force of \SI{2e-7}{\newton} of
attractive force per metre of wire on each other.
\begin{figure}[ht]
  \centering
  \begin{tikzpicture}[scale=.8]
    \foreach \x in {0,10}{
      \draw[dashed] (\x,-2.5)--(\x,2.5);
      \draw[vectors,red] (\x,-2)--(\x,2) node[above]{$I$};
    }
    \draw[<->] (0,-1)--(10,-1) node[midway,fill=white]{\SI1\metre};
    \draw[vector] (0,0)--(1.5,0) node[right]{$\vec F_m/\ell=\SI{2e-7}\newton$};
    \draw[vector] (10,0)--(8.5,0) node[left]{$\vec F_m/\ell=\SI{2e-7}\newton$};
  \end{tikzpicture}
\end{figure}
From the definition of the amp\`{e}re for current, we can then define the
coulomb of charge as the amount of charge that 1 ampere of current passes
through a wire in 1 second:
\begin{equation*}
  \SI1{\coulomb}=\SI1{\ampere}\times\SI1{\second}
\end{equation*}

\section{Circular Motion Caused by a Magnetic Field}
When a charged particle enters a magnetic field at right angle, the resulting
magnetic force $\bm F_m$ perpendicular to both velocity $\bm v$ and magnetic
field $\bm B$. If motion is confined to a 2D plane, this results in a uniform
circular motion, where centripetal force $\bm F_c$ is provided by the magnetic
force $\bm F_m$. We can solve for the radius $r$ of the motion:
\begin{equation}
  F_c=ma_c\;\;\rightarrow\;\;
  F_m=\frac{mv^2}r\;\;\rightarrow\;\;
  qvB=\frac{mv^2}r\;\;\rightarrow\;\;
  \boxed{r=\frac{mv}{qB}}
\end{equation}




\section{Mass Spectrometer}

A \textbf{mass spectrometer} is used to determine the composition of a
substance by separating particles of different mass, and measuring their
\emph{charge-to-mass} ratios. It consists of three major components:
\begin{itemize}
\item Particle accelerator
\item Velocity selector
\item Mass separator
\end{itemize}
%  \begin{center}
%    \begin{tikzpicture}[scale=.9]
%      % Particle accelerator plates
%      \draw[thick] (2,0)--(2,3) to[battery] (0,3)--(0,0);
%      \foreach \y in {.2,.65,...,2.7}
%      \draw[vectors,red!80!black] (0,\y)--(2,\y);
%      \node[red!80!black] at (.5,1.8) {$\bm E$};
%      \node at (1,-.5) {Particle Accelerator};
%      \node at (1,3.75) {$\Delta V$};
%      % Ion leaving the particle accelerator
%      \fill[blue!50!black] (.05,1) circle (.05);
%      \begin{scope}[very thick,blue!50!black]
%        \draw[->] (.05,1)--(1.2,1);
%        \draw (1.1,1)--(2.5,1);
%      \end{scope}
%      % Velocity selector plates
%      \foreach \x in {3.25,3.75,...,7}
%      \draw[vectors,red!80!black] (\x,2)--(\x,0);
%      \node at (5,-.5) {Velocity Selector};
%      \foreach \x in {3,3.5,...,7}{
%        \foreach \y in {-.25,.25,...,2.25} \node[cyan] at (\x,\y) {$\times$};
%      }
%      \node[above,cyan] at (5,2.25) {$\bm B_\text{in}$};
%      \node[red!80!black] at (3,1.75){$\bm E$};
%      \draw[very thick] (3,0)--(7,0) node[right]{$-$};
%      \draw[very thick] (3,2)--(7,2) node[right]{$+$};
%      
%      % Ion travelling through the velocity selector plates
%      \draw[vectors,blue!50!black] (2.5,1)--(8.75,1);
%      \begin{scope}[thick,blue!60!black,smooth,domain=3:7.5,dashed,->]
%        \draw[samples=10] plot({\x},{.03*(\x-3)*(\x-3)+1});
%        \draw[samples=10] plot({\x},{-.03*(\x-3)*(\x-3)+1});
%      \end{scope}
%      \node[right=-1,blue!50!black] at (7.5,1.6){\tiny faster ions};
%      \node[right=-1,blue!50!black] at (7.5,0.4){\tiny slower ions};
%      % Mass separation
%      \draw[thick] (8.75,0) rectangle (10.75,4);
%      \node at (9.75,-.5) {Mass Separator};
%      \begin{scope}[vectors,blue!50!black]
%        \draw (8.75,1) arc (-90:90:.7);
%        \draw (8.75,1) arc (-90:90:1);
%        \draw (8.75,1) arc (-90:90:1.2);
%      \end{scope}
%      \foreach \x in {8.5,9,...,11}{
%        \foreach \y in {-.25,.25,...,4.25} \node[cyan] at (\x,\y) {$\times$};
%      }
%      \node[cyan] at (10,3.5){$\bm B_\text{in}$};
%    \end{tikzpicture}
%  \end{center}
%
%
%
%
\subsection{Simple Particle Accelerator}

\begin{figure}[ht]
  \centering
  \begin{tikzpicture}[scale=1.4]
    \draw[thick] (2,0)--(2,.95);
    \draw[thick] (2,1.05)--(2,3) to[battery,l_=$\Delta V$] (0,3)--(0,0);
    \foreach \y in {.2,.65,...,2.7}
    \draw[vectors,red!80!black] (0,\y)--(2,\y);
    \node[red!80!black] at (.5,1.75) {$\bm E$};
    \node[below] at (1,0) {Particle Accelerator};
    
    \begin{scope}[blue!50!black]
      \fill (.05,1) circle (.05);
      \draw[vectors] (.05,1)--(1.2,1) node[below]{$+$ ions};
      \draw[very thick] (1.1,1)--(2.5,1);
    \end{scope}
  \end{tikzpicture}
  \caption{A simple particle accelerator}
\end{figure}

%    \begin{itemize}
%    \item The simplest particle accelerator can be made of:
%      \begin{itemize}
%    \item A pair of parallel plates
%    \item An accelerating potential difference (voltage)
%    \item A particle (ion) source
%      \end{itemize}
%    \item Ionized particles pass through the plates and get accelerated by the
%      electric field then get shot out of the other end of the plate. By the
%      conservation of energy:
%
%      \eq{-.1in}{
%        K=-\Delta U_q\quad\rightarrow\quad \frac12mv^2=q\Delta V
%      }
%    \end{itemize}
%  
%
%
%
%
\subsection{Velocity Selector}


\begin{figure}
  \centering
  \begin{tikzpicture}[scale=1.3]
    \foreach \x in {3,3.5,...,7}{
      \foreach \y in {-.25,.25,...,2.25} \node[cyan] at (\x,\y) {$\times$};
    }
    % Velocity selector plates
    \foreach \x in {3.25,3.75,...,7}
    \draw[vectors,red!80!black] (\x,2)--(\x,0);
    \node[red!80!black] at (3,1.75){$\bm E$};
    \node[above,cyan] at (5,2.25) {$\bm B_\text{in}$};
    % Ion travelling through the velocity selector plates
    \begin{scope}[vectors,blue!50!black]
      \draw (2.5,1)--(8.75,1);
      \begin{scope}[smooth,samples=10,domain=3:7.5,dotted]
        \draw plot(\x,{.03*(\x-3)*(\x-3)+1});
        \draw plot(\x,{-.03*(\x-3)*(\x-3)+1});
      \end{scope}
    \end{scope}
    \node[blue!50!black] at (8.2,1.6){faster ions};
    \node[blue!50!black] at (8.25,.4){slower ions};
    
    \draw[very thick] (3,0)--(7,0) node[right]{$-$};
    \draw[very thick] (3,2)--(7,2) node[right]{$+$};
  \end{tikzpicture}
\end{figure}
%  \begin{itemize}
%  \item Filters the beam of particles to let particles with the same velocity
%    to pass through
%  \item Consists of a crossed (perpendicular) electric and magnetic field
%  \item When electric and magnetic forces are balanced, particle travels
%    straight through
%  \end{itemize}
%
%
%
%
The charged particle travels straight through when magnetic force $\bm F_m$
and electric force $\bm F_q$ are balanced:
\begin{equation*}
   F_m=F_q
\end{equation*}
Substituting the expressions for $F_m$ and $F_q$, we can solve for $v$:
\begin{equation}
  qvB =qE\quad\rightarrow\quad\boxed{v = \frac EB}
\end{equation}
What particle velocity can go straight through can be controlled by adjusting
the relative strength of the electric field $\bm E$ and magnetic field $\bm B$.
\textbf{This velocity does not depend on the charge or the mass of the charged
  particle.}



\subsection{Mass Separator}
%
\begin{figure}
  \centering
  \begin{tikzpicture}[scale=1.3]
    \foreach \x in {8.5,9,...,11}{
      \foreach \y in {-.25,.25,...,4.25} \node[cyan] at (\x,\y) {$\times$};
    }
    % Mass separation
    \draw[thick] (8.75,0) rectangle (10.75,4);
    \foreach \radius in {.7,1,1.2}
    \draw[vectors,blue!50!black] (8.75,1) arc (-90:90:\radius);
    \node[cyan] at (10,3.5){$\bm B_\text{in}$};
    \node[blue!50!black] at (8,2.4){lower mass};
    \node[blue!50!black] at (7.9,3.4){higher mass};
  \end{tikzpicture}
\end{figure}

%    Finally, particles of different masses are separated by going into circular
%    motion inside a magnetic field, where
%
%    \eq{-.15in}{
%      F_c=F_m
%    }
%
%    \vspace{-.13in}Substitute the expressions for $F_c$ and $F_m$, then solve
%    for $m$:
%
%    \eq{-.13in}{
%      \frac{mv^2}r=qvB\quad\rightarrow\quad\boxed{\frac qm = \frac v{rB}}
%    }
%
%    The composition of the substance is determined by measuring the relative
%    amount of particles with different radii.
%  
%
%
%
%%  \textbf{Example}: A positive ion, having a charge of \SI{3.2e-19}{\coulomb},
%%  enters at the extreme left of the parallel plate assembly associated with the
%%  velocity selector and mass spectrometer shown previously.
%%  \begin{itemize}
%%  \item If the potential difference across the simple accelerator is
%%    \SI{1.2e3}{\volt}, what is the kinetic energy of the particle as it leaves
%%    through the hole in the right plate?
%%  \end{itemize}
%%
%%
%%\section{Example Problem (cont.)}
%%  \begin{itemize}
%%  \item The parallel plates of the velocity selector are separated by
%%    \SI{12}{\milli\metre} and have an electric potential difference across
%%    them of \SI{360}{\volt}. If a magnetic field of strength
%%    \SI{.10}{\tesla} is applied at right angles to the
%%    electric field, what is the speed of the particles that will be selected to
%%    pass on the mass spectrometer?
%%  \item When these particles then enter the mass spectrometer, which shares a
%%    magnetic field with the velocity selector, the radius of the resulting
%%    circular path followed by the particles is \SI{6.3}{\centi\metre}. What is
%%    the mass of the charged particles?
%%  \end{itemize}
%%
%%
%%\section{The Cockcroft-Walton Proton Accelerator}
%%    \begin{itemize}
%%    \item Capable of \SI{1}{\mega eV} (mega electron volt)
%%    \item $\SI{1}{eV}=\SI{1.6021e-19}{J}$
%%    \end{itemize}
%%
%%    \pic{1}{graphics/proton-accel}
%%  
%%
%%
%%
%%\section{The Cyclotron}
%%  \begin{itemize}
%%  \item Large number of small increases in potential.
%%  \item \SI{30}{\mega eV} is very common 
%%  \end{itemize}
%%  \begin{center}
%%    \pic{.5}{graphics/cyclotron}
%%  \end{center}
