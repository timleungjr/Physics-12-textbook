\documentclass{../../oss-apphys-exam}
\begin{document}
%\genheader

\gentitle{5}{DC CIRCUIT ANALYSIS, PART 1}

\classkickMCinstructions

\textbf{Questions \ref{ohmic1}--\ref{ohmic2}}: The figure shows current as a
function of electric potential difference for a resistor and bulb.
\cpic{.35}{ohmic}

\begin{questions}
  \question Are the devices ohmic?
  \label{ohmic1}
  
  \begin{tabular}{ccc}
    & \underline{Resistor} & \underline{Bulb}\\ \hline
    (A) & Ohmic & Ohmic \\
    (B) & Ohmic & Non-Ohmic\\
    (C) & Non-Ohmic & Ohmic\\
    (D) & Non-Ohmic & Non-Ohmic\\
  \end{tabular}
  
  \question If the resistor and bulb are connected in parallel to a 10.0 V
  battery, what is the total current passing through the system?
  \begin{choices}
    \choice 0.5 A
    \choice 0.7 A
    \choice 1.0 A
    \choice 1.7 A
  \end{choices}

  \question With the resistor and bulb still connected in parallel to the 10.0 V
  battery, what is the total power dissipated by the bulb and resistor?
  \begin{choices}
    \choice 0.042 W
    \choice 10 W
    \choice 17 W
    \choice 24 W
  \end{choices}
  
  \question What is the equivalent resistance of the bulb and resistor while
  still connected in parallel to the 10.0 V battery?
  \begin{choices}
    \choice\SI{.17}\ohm
    \choice\SI{5.9}\ohm
    \choice\SI{17}\ohm
    \choice\SI{24}\ohm
  \end{choices}

  \question The bulb and resistor are removed and reconnected in series to the
  10.0 V battery. What is the total current passing through the system?
  \label{ohmic2}
  \begin{choices}
    \choice 0.41 A
    \choice 0.50 A
    \choice 1.0 A
    \choice 1.7 A
  \end{choices}
  \newpage
  
  \uplevel{
    \textbf{Questions \ref{batt1}--\ref{batt2}}: Two batteries and two resistors
    are connected in a circuit, as shown in the figure. The currents through
    $R_1$, $R_2$, and $\varepsilon_2$ are shown.
    \cpic{.3}{2batteries}
  }
  
  \question Which of the following is a proper application of conservation laws
  to this circuit? \emph{Select two answers.}
  \label{batt1}
  \begin{choices}
    \choice $\varepsilon_1 - I_1 R_1 - I_2 R_2 = 0$
    \choice $\varepsilon_2 - \varepsilon_1 - I_1 R_1 = 0$
    \choice $I_1 + I_2 - I_3 = 0$
    \choice $I_2 + I_3 - I_1 = 0$
  \end{choices}

  \question The resistors $R_1$ and $R_2$ have the same resistance. If the
  potential differences of the batteries are $\varepsilon_1=\SI9\volt$ and
  $\varepsilon_2=\SI6\volt$, which resistor will have the most current passing
  through it?
  \label{batt2}
  \begin{choices}
    \choice $R_1$
    \choice $R_2$
    \choice $R_1$ and $R_2$ have the same current.
    \choice It is not possible to determine the currents through the resistors
    without more information.
  \end{choices}

  \uplevel{
    \textbf{Questions \ref{identical1}--\ref{identical2}}: Four identical
    resistors of resistance Rare connected to a battery, as shown in the
    figure. Ammeters $A_1$ and $A_2$ measure currents of 1.2 A and 0.4 A,
    respectively.
    \cpic{.38}{identical}
  }

  \question What are the currents measured by ammeters $A_3$ and $A_4$?
  \label{identical1}
  
  \begin{tabular}{ccc}
    & $A_3$ & $A_4$ \\ \hline
    (A) & \SI{.4}\ampere & \SI{.4}\ampere \\
    (B) & \SI{.8}\ampere & \SI{.4}\ampere \\
    (C) & \SI{.4}\ampere & \SI{1.2}\ampere \\
    (A) & \SI{.8}\ampere & \SI{1.2}\ampere
  \end{tabular}

  \question What is the equivalent resistance of the circuit?
  \label{identical2}
  \begin{choices}
    \choice $\dfrac14R$
    \choice $\dfrac43R$
    \choice $\dfrac52R$
    \choice $4R$
  \end{choices}

  \question A student is given a battery with an unknown emf ($\varepsilon$)
  and an internal resistance of $r$. The student sets up a circuit with a known
  resistor and switch, as shown in the figure. Which measurements should the
  student make to find the values of both $\varepsilon$ and $r$?
  \emph{Select two answers.}
  \cpic{.28}{internal}
  \begin{choices}
    \choice With the switch open, measure the potential difference between
    points 1 and 2 and the current at point 1.
    \choice With the switch closed, measure the potential difference between
    points 1 and 2 and the current at point 1.
    \choice With the switch open, measure the potential difference between
    points 1 and 2. Close the switch and measure the current at point 1.
    \choice With the switch open, measure the potential difference between
    points 1 and 2. Close the switch and measure the potential difference
    between points 1 and 2.
  \end{choices}
  \newpage

  \fullwidth{
    \classkickFRQinstructions
  }
  
  % TAKEN FROM THE 2017 AP PHYSICS 2 EXAM FREE-RESPONSE QUESTION #2

  \question A group of students is given several long, thick, cylindrical
  conducting rods of the same unknown material with various lengths and
  diameters and asked to experimentally determine the resistivity of the
  material using a graph. The available equipment includes a voltmeter, an
  ammeter, connecting wires, a variable-output DC power supply, and a metric
  ruler.
  \begin{parts}
    \part Describe a procedure the students could use to collect the data
    needed to create the graph, including the measurements to be taken and a
    labeled diagram of the circuit to be used. Include enough detail that
    another student could follow the procedure and obtain similar data.

    Draw a labeled diagram here.
    \vspace{\stretch1}
      
    Write your procedure here.
    \vspace{\stretch1}
      
    \part Describe how the data could be graphed in a way that is useful
    for determining the resistivity of the material. Describe how the graph
    could be analyzed to calculate the resistivity.
    \vspace{\stretch1}
    \newpage
    
    \uplevel{
      The students are now given a rectangular rod of the material, as shown
      below, whose dimensions are not known. The students are asked to
      experimentally determine the resistance of the rod. They obtain the data
      in the table below for the potential difference $\Delta V$ across the rod
      and the current $I$ in it.
      \begin{center}
        \vspace{.1in}
        \pic{.65}{rod}
        \begin{tabular}{|c|c|c|c|c|c|c|}
          \hline
          $\Delta V$ (\si\volt) & 6.0 & 5.0 & 3.5 & 2.5 & 2.0 & 1.5 \\
          \hline
          $I$ (\si\ampere) & 0.078 & 0.070 & 0.044 & 0.036 & 0.027 & 0.018 \\
          \hline
        \end{tabular}
      \end{center}
    }

    \part On the axes below, plot the data so that the resistance of the
    rectangular rod can be determined from a best-fit line. Label and scale the
    axes. Use the best-fit line to determine the resistance of the rod, clearly
    showing your calculations.

    \uplevel{
      \centering
      \begin{tikzpicture}
        \draw[lightgray,step=.2] grid (8,8);
        \draw[thick] grid (8,8);
        \draw[axes] (0,0)--(8.5,0);
        \draw[axes] (0,0)--(0,8.5);
      \end{tikzpicture}
      \vspace{1in}
    }
    \uplevel{
      After completing their calculations, the students begin to consider the
      factors that might have produced uncertainties in their results.
    }
    \part The students realize that they did not take into account the
    internal resistance of the power supply. Briefly describe how this would
    affect their value of the resistance of the rectangular rod. Explain your
    reasoning.
    \vspace{\stretch1}
    
    \part The students realize that they did not take into account a possible
    change in the temperature of the cylindrical rods. Should the students be
    concerned about this? Explain why or why not.
    \vspace{\stretch1}
  \end{parts}
  \newpage
  
  \uplevel{
    \cpic{.38}{XYZ}
  }
  \question\vspace{-.1in} Four lightbulbs are connected in a circuit with a
  \SI{24}{\volt} battery as shown above.
  \begin{parts}
    \part
    \begin{subparts}
      \subpart Determine the average potential energy change of an electron as
      it moves from point $Z$ to point $X$.
      \vspace{\stretch1}
      
      \subpart Indicate whether the electron gains or loses potential energy as
      it moves from point $Z$ to point $X$.

      \vspace{.15in}
      \underline{\hspace{.4in}} Gains energy
      \vspace{.15in}
      \underline{\hspace{.4in}} Loses energy
    \end{subparts}

    \part Calculate the equivalent resistance of the circuit.
    \vspace{\stretch1}
    
    \part Calculate the magnitude of the current through point $Y$.
    \vspace{\stretch1}
    
    \part Indicate on the diagram the direction of the current through point
    $Y$.
    \vspace{\stretch1}
    
    %\part Calculate the energy dissipated in the 12 W bulb in 5.0 s.
    %\vspace{\stretch1}
    
    \part Rank the bulbs in order of brightness, with 1 being the brightest. If
    any bulbs have the same brightness, give them the same ranking.

    \vspace{.15in}
    \underline{\hspace{.4in}} Bulb A\hspace{1in}
    \underline{\hspace{.4in}} Bulb B\hspace{1in}
    \underline{\hspace{.4in}} Bulb C\hspace{1in}
    \underline{\hspace{.4in}} Bulb D
  \end{parts}
\end{questions}
\end{document}
