\chapter{Work and Energy}
\label{chapter:energy}


The topic of work and energy is, perhaps, the most important topic in
fundamental physics. It is also a concept that is most
misunderstood\footnote{And in the opinion of the author, often badly taught as
a result.}, as especially in the area of conservation of energy, and so it is
often presented in a way that is confusing to students at many levels. In this
document, we aim to approach this topic from a logical perspective that
combines knowledge in kinematics as well as dynamics.

We start with some definition at are (unfortunately) not very useful:
\begin{itemize}[nosep]
\item \textbf{Energy} is the ability to do work.
\item \textbf{Work} is the mechanism in which energy is transformed.
\end{itemize}
At a minimum, these definitions at least tell two things:
\begin{itemize}
\item The concepts of work and energy cannot be separated: Defining work
  without leading to energy is rather pointless; defining energy without first
  referring to work makes no sense.
\item Word definitions are not enough. We also need to have \emph{mathematical}
  definitions as well.
\end{itemize}
It is customary to start with the definition of mechanical work, and use it
to define different forms of energies.

\section{Mechanical Work}
\label{sec:mechwork}
An infinitesimal amount of \textbf{mechanical work} $\dl W$ is done when a
force $\bm F$ displaces an object by an infinitesimal amount
$\dl\bm x$. If the force moves an object along the path $\mathcal C$, the
total work $W$ done by the force is defined by the integral:
\begin{equation}
  \boxed{
    W=\int_{\mathcal C}\dl W=\int_{\mathcal C}\bm F(\bm x)\cdot\dl\bm x
  }
  \label{work-definition}
\end{equation}
While both force and displacement are vectors, work is a scalar quantity. Work
by $\bm F$ can be positive or negative depending on the dot product. In
general, the amount of work done ($W$) depends on the path $\mathcal C$. %From
%Eq.~(\ref{work-definition}), we recognize that
No work is done if the force is perpendicular to displacement, (i.e.\
$\bm F\cdot\dl\bm x=0$) which means that the force did not \emph{cause}
the displacement, or if the object does not move, (i.e.\
$\dl\bm x=\bm 0$), or if no force is applied during motion (i.e.\
$\bm F=\bm 0$).

For motion confined to one direction along a one-dimensional coordinate
system, Eq.~(\ref{work-definition}) reduces to\footnote{Note that direction
still matters for $F$ and $x$, even in 1D, in that there is still a positive
and negative direction (if $F$ and $\dl x$ are in the same direction, then
$W>0$; and if $F$ and $\dl x$ are in opposite direction, then $W<0$).}:
\begin{equation}
  \boxed{
    W=\int_{x_0}^{x_1} F(x)\dl x
  }
\end{equation}
For a constant force that moves an object along a straight path, the integral
simplifies to just the dot product of two vectors\endnote{We must remember how
  to express the dot products. If you know the angles between the vectors,
  then:
  \begin{equation*}
    \bm A\cdot\bm B= AB\cos\theta
  \end{equation*}
  And if you know the components of the vectors, then
  \begin{equation*}
    \bm A\cdot\bm B= A_xB_x+A_yB_y+A_zB_z
  \end{equation*}
}:
\begin{equation}
  \boxed{
    W = \bm F\cdot\Delta\bm x=F\Delta x\cos\theta
  }
  \label{eq:no-integration}
\end{equation}
where $\theta$ is the angle between the force and displacement vectors. We can
also use the above equation if the force $\bm F$ is \emph{averaged} over
the displacement, i.e.\
\begin{equation}
  \bm F=\bm F_\text{avg}=\frac{\int_{x_0}^{x_1} F(x)\dl x}{x_1-x_0}
\end{equation}
\begin{remark}%sidenote}
  We need a short discussion about what this \emph{average} is. In this case,
  the force is averaged over the \emph{displacement} that the object travels.
  We are not concerned about how long it takes for the object to move. In
  contrast, when computing the \emph{impulse} generated from an average force
  over a time interval, which is related to the change in the \emph{momentum}
  of an object, the force is averaged over \emph{time}:
  \begin{displaymath}
    \overline{\bm F}=\frac{\int_{t_0}^{t_1} \bm F(t)\dl t}{t_1-t_0}
  \end{displaymath}
  (We purposely use different notation to distinguish different types of
  averaging.) Time averaging does not consider the actual motion of the
  object. This time averaging is also how we get the expressions for average
  velocity and average acceleration in kinematics, when we describe the motion
  of a single object:
  \begin{displaymath}
    \overline{\bm v}=\frac{\int_{t_0}^{t_1} \bm v(t)\dl t}{t_1-t_0}
    \quad\text{and}\quad
    \overline{\bm a}=\frac{\int_{t_0}^{t_1} \bm a(t)\dl t}{t_1-t_0}
  \end{displaymath}
  Later in this chapter, we will also discuss the concept of average power,
  which uses a time averaging. However, there is still a third kind of
  averaging that we will touch on in this chapter. In thermodynamics of gases,
  when we calculate the average speed and kinetic energies of many gas
  molecules, the average is the arithmetic mean:
  \begin{displaymath}
    \langle v\rangle = \frac{\sum v_i}N
  \end{displaymath}
  And if the speeds are given in a probability density function $f(v)$,
  \begin{displaymath}
    \langle v\rangle = \int_0^\infty vf(v)f\dl v
  \end{displaymath}
  This type of averaging is called an \textbf{ensemble average}. While it is
  crucial to know the difference between different types of averaging, when
  solving problems, it is usually very clear from the context. 
\end{remark}%sidenote}

At this moment, it is unclear what the force $\bm F$ is. Whereas we use the
\emph{net force} when calculating acceleration in dynamics problems, when
calculating the work done by ``a force'', it could mean:
\begin{enumerate}[leftmargin=12pt]
\item\textbf{Work done by a \emph{specific} force.} As you have no doubt seen
  in dynamics problems, there are usually multiple forces acting on an object.
  We can calculate, based on the motion of the object, how much work is done by
  each force. Calculating the work done by a specific force allows us to study
  the \emph{exact} mechanism in how energy is transformed.
  
\item\textbf{Work done by the \emph{net} force}, in other words, the \emph{sum}
  of all the work done by each force. This is also called the \textbf{net work}
  $W_\text{net}$:
  \begin{equation*}
    W_\text{net}
    = \sum_i W_i
    =\int_{\mathcal C}\bm F_\text{net}(\bm x)\cdot\dl\bm x
  \end{equation*}
  The net work allows us to study what is the overall conservation of energy
  to the entire object.
\end{enumerate}

\begin{example}
  A worker pushes a heavy crate up a ramp with a varying applied force. The
  free-body diagram for this problem is shown below. Here, four forces act on
  the crate: gravity $\bm F_g$, normal force $\bm F_n$, kinetic
  friction $\bm f$, and applied force $\bm F_a(x)$, which is expressed
  as a function of the crate's displacement as it moves.
  \begin{center}
    \begin{tikzpicture}
      \draw[thick] (0,0)--(5,0);
      \draw[axes] (3,0) arc (0:25:3) node[midway,right]{$\theta$};
      \begin{scope}[rotate=25]
        \draw[thick] (0,0)--({5/cos(25)},0);
        \draw[thick,|->] (3.5,1.75)--+(2,0) node[right]{$\Delta x$};
        \draw[mass] (1.5,0) rectangle (3.5,1.5);
        \fill (2.5,.75) circle (.07);
        \draw[vector] (2.5,.75)--+(1.5,0) node[right]{$\bm F_a(x)$};
        \draw[vector] (2.5,.75)--+(0,1.5) node[above]{$\bm F_n$};
        \draw[vector,rotate around={-25:(2.5,.75)}]
        (2.5,.75)--+(0,-1.25) node[below]{$\bm F_g$};
        \draw[vector] (2.5,.75)--+(-1,0) node[left]{$\bm f$};
      \end{scope}
    \end{tikzpicture}
  \end{center}
  We can calculate the work done by each force:
  \begin{itemize}[leftmargin=12pt]
  \item Work done by the normal force is zero, because $\bm F_n$ is
    perpendicular to the direction of motion:
    \begin{equation*}
      W_n=\int\underbrace{\bm F_n\cdot\dl\bm x}_{=0}=0
    \end{equation*}
  \item Work done by kinetic friction is negative, because $\bm f$ is in
    the opposite direction to motion:
    \begin{equation*}
      W_f=\int\bm f\cdot\dl\bm x=-\int f\dl x=-\mu_k F_n\Delta x<0
    \end{equation*}
  \item Work done by gravity is also negative, because the component of
    gravity along the direction of motion is in the opposite direction.
    (This is badly written!!)
  \item The work done by the applied force is positive, because applied force
    is in the same direction as motion.
    \begin{equation*}
      W_a(x)=\int_{x_0}^x F_a\dl x>0
    \end{equation*}
  \end{itemize}
  The total work (i.e.\ the net work) done by summing the work done by each
  force:
  \begin{align*}
    W_\text{net} &=W_a+W_n+W_g +W_f\\
    &=\int\left(\bm F_a+\bm F_n+\bm F_g+\bm F_f\right)
    \cdot\dl\bm x =\int\bm F_\text{net}\cdot\dl\bm x
  \end{align*}
\end{example}



\section{Kinetic Energy}
\label{sec:transKE}
When a net force acts on an object (with constant mass) to accelerate it, work
is done, For motion in one dimension, the total/net work done is given by:
\footnote{When you do this problem ``properly'' in 2D or 3D, the only
difference is the dot product, which only \emph{slightly} increases the
complexity of the problem from the 1D case:
\begin{equation*}
  W_\text{net}
  = \int\bm F_\text{net}(\bm x)\cdot\dl\bm x
  = \cdots = m\int\bm v\cdot\dl\bm v
\end{equation*}
The difference in the dot product is a lot easier to evaluate than you might
think:
\begin{align*}
  m\int\bm v\cdot\dl\bm v
  &=m\left(\int v_x\dl v_x + \int v_y\dl v_y + \int v_z\dl v_z\right)\\
  &=\left(\frac12mv_x^2+\frac12mv_y^2 + \frac12mv_z^2\right)
  =\frac12mv^2
\end{align*}
where $v^2=v_x^2+v_y^2+v_z^2$. This is, of course, the same result that we
got from the one-dimensional case.}
\begin{equation}
  W_\text{net}
  =\int_{x_0}^{x_1}F_\text{net}(x)\dl x
  =\int ma\dl x
  =m\int\diff vt\dl x
\end{equation}
Since both $v(t)$ and $x(t)$ are continuously differentiable in time, we can
switch the order of the differentiation:
\begin{equation*}
  =m\int\diff xt\dl v=m\int_{v_0}^{v_1}v\dl v
\end{equation*}
The limits of integration switch from the initial and final position ($x_0$ and
$x_1$) to the initial and final velocities, where $v_0=v(x_0)$ and
$v_1=v(x_1)$. Evaluating this integral, we have:
\begin{equation*}
  =m\int_{v_0}^{v_1}v\dl v
  =\frac12mv^2\Big|^{v_1}_{v_0}
  =\frac12mv_1^2-\frac12mv_0^2
  =\Delta K
\end{equation*}
where $K$ is defined as the \textbf{kinetic energy}\footnote{This is more
specifically called the \emph{translational} kinetic energy, and it should be
distinguished from the \emph{rotational kinetic energy} which is used when an
object is rotating about a pivot.}:
\begin{equation}
  \boxed{
    K=\frac12mv^2
  }
\end{equation}
The definition of kinetic energy came from this integration: when the net force
on an object is doing work, that work is equal to the change in
\emph{something}, and we \emph{define} that quantity as the kinetic energy.
This is known as the \textbf{work-energy theorem}\footnote{Also known as the
\textbf{work-energy principle}, and \textbf{work-energy relationship}}:
\begin{equation}
  \boxed{
    W_\text{net}=\Delta K
  }
  \label{work-energy-theorem}
\end{equation}
When multiple forces act on an object, \emph{positive} net work will
\emph{increase} the kinetic energy of the object, while \emph{negative} net
work will decreases kinetic energy. Eq.~\ref{work-energy-theorem} applies
regardless of \emph{what} the net force is comprised of. Very importantly, the
work-energy theorem turns a potentially difficult integration problem
(integrating $W_\text{net}$) into a simple algebraic expression ($\Delta K$).
\begin{example}
  A net force $F=4x$ (in newtons) acts on an object of mass \SI2{\kilo\gram} as
  it moves along the $x$-axis from $x=1$ to $x=\SI5\metre$. Given that the
  object is at rest at $x=1$,
  \begin{enumerate}%[(a)]
  \item Calculate the net work
  \item What is the final speed of the object?
  \end{enumerate}
\end{example}



\section{Potential Energies}

Unlike kinetic energy, forms of energy that can be stored are called
\textbf{potential energy}.

\subsection{Gravitational Force \& Gravitational Potential Energy}
\label{section:gravitationalPE}

Consider an object that is dropped (free-falling) under the force of gravity
over a distance of $\Delta x$, shown in Fig.~\ref{fig:falling1}.
\begin{figure}[ht]
  \centering
  \begin{tikzpicture}[scale=.6]
    \draw[thick,fill=gray!10] (7.75,0) arc (75:105:30);
    \draw[dashed] (-5,1.05)--+(10,0);
    \draw[dashed] (-5,6)--+(10,0);
    \draw[dashed] (-5,3)--+(10,0);
    \draw[mass] (0,6) circle (.2) node[right=2]{$m$};
    \draw[vector,red] (0,6)--+(0,-2) node[right]{$\bm F_g$};
    \draw[vector] (-.5,6)--+(0,-3) node[midway,left]{$\Delta x$};
    \draw[vector,blue] (-1.7,1.05)--+(0,4.95) node[midway,left]{$h_0$};
    \draw[vector,blue] (-1,1)--+(0,2) node[midway,right]{$h_1$};
  \end{tikzpicture}
  \caption{Gravitational force doing positive work on a free-falling object.}
  \label{fig:falling1}
\end{figure}

When displacement $\Delta\bm x$ is small, acceleration due to gravity
$\bm g$ can be considered to be constant, therefore the gravitational
force $\bm F_g=m\bm g$ is constant. Work done by gravity can be
calculated using Eq.~\ref{eq:no-integration}, and no calculus is needed. Since
both $\bm F_g$ and $\Delta\bm x$ are in the same direction, work done
by gravity ($W_g$) is positive. From the work-energy theorem
(Eq.~\ref{work-energy-theorem}), there is an increase in kinetic energy, and
the object speeds up:
\begin{equation*}
  W_g=mg\Delta x>0 \quad\longrightarrow\quad \Delta K > 0
\end{equation*}
This is consistent with our understanding of kinematics and dynamics. But the
work done by gravitational force can also be expressed in terms of the change
in height. Using ground as the reference level (i.e.\ $h=0$), the work done by
gravity can be written as:
\begin{equation*}
  W_g = mg(h_0-h_1)
\end{equation*}
We can further modify this equation:
\begin{align}
  W_g &= mg(h_0-h_1)\nonumber\\
  & = -mg(h_1-h_0)\nonumber\\
  & = -(mgh_1-mgh_0)\nonumber\\
  W_g &= -\Delta U_g
\end{align}
where $U_g$ is defined as the \textbf{gravitational potential energy}:
\begin{equation}
  \boxed{U_g=mgh}
\end{equation}
Since the choice of the reference level (where we define $h=0$) is arbitrary,
we are more interested in the \emph{change} in gravitational potential energy,
which is related to the work done by gravity:
\begin{equation}
  \boxed{
    W_g=-\Delta U_g
  }\quad\text{where}\quad
  \boxed{
    \Delta U_g=mg\Delta h
  }
  \label{work-potential-energy}
\end{equation}
In Eq.~\ref{work-potential-energy}, we note some special relationships between
the work done by gravity ($W_g$) and the change in the gravitational potential
energy ($U_g$):
\begin{definition}
  \begin{itemize}[itemsep=3pt,leftmargin=15pt]
  \item When work done by gravtational force is \emph{positive} (i.e.\
    $W_g>0$), there is a \emph{decrease} in gravitational potential energy by
    the same amount ($\Delta U_g>0$), while
  \item When work done by graviational force is \emph{negative} (i.e.\
    $W_g<0$), there is an \emph{increase} in gravitational potential energy by
    the same amount ($\Delta U_g>0$)
  \item The work by gravity is \emph{path independent}: $W_g$ depends on the end
    points $h_0$ and $h_1$, but not \emph{how} it goes from
    $h_0\rightarrow h_1$
  \item Only work done by gravity can affect $U_g$
  \end{itemize}
\end{definition}
As you can see in the example in Fig.~\ref{path-independence}, the work done by
gravity is the same in all cases. That is not to say that there are no other
forces acting on the object; we have merely isolated the work done by the
gravitational force alone.
\begin{figure}[ht]
  \centering
  \begin{tikzpicture}[scale=.95]
    \draw[thick,dashed] (0,0)--+(11,0) node[right]{$h_0$};
    \draw[thick,dashed] (0,-2)--+(11,0) node[right]{$h_1$};
    
    \fill[red] (.7,0) circle (.07);
    \draw[vector,red] (.7,0)--+(0,-2);
    \draw[thick,red] (.7,-2) circle (.07);
    \node[below,red] at (.7,-2.3){Dropped};
    
    \fill[violet] (3.5,0) circle (.07);
    \draw[vector,violet] (3.5,0)--+(0,1.2) arc(180:0:.05)--+(0,-3.2);
    \draw[thick,violet] (3.6,-2) circle (.07);
    \node[below,violet] at (3.55,-2.3){Thrown straight up};
      
    \fill[orange] (5.5,0) circle (.07);
    \draw[vector,orange] (5.5,0) to[out=50,in=120] +(2,-2);
    \draw[thick,orange] (7.5,-2) circle (.07);
    \node[below,orange] at (6.5,-2.3){Projectile};
      
    \fill[magenta] (8,0) circle (.07);
    \draw[vector,magenta] (8,0) to[out=-50,in=230] +(2.5,-2);
    \draw[thick,magenta] (10.5,-2) circle (.07);
    \node[below,magenta] at (9.25,-2.3){Arbitrary surface};
  \end{tikzpicture}  
  \caption{Work done by gravity ($W_g$) is the same in all above cases because
    they all have the same initial and final height.}
  \label{path-independence}
\end{figure}

\textbf{Gravity} is the mutually attractive force between
massive objects. The magnitude of gravitational force between two
\textbf{point masses} is proportional to their masses ($m_1$, $m_2$), and
inversely proportional to the square of the distance ($r$) between them:
\begin{equation}
  \boxed{F_g=\frac{Gm_1m_2}{r^2}}
\end{equation}
where $G=\SI{6.67e-11}{N.m^2/kg^2}$ is the \textbf{universal gravitational
  constant}
\begin{figure}[ht]
  \centering
  \begin{tikzpicture}[scale=.65]
    \draw[vector,red] (0,0)--(2,0) node[right]{$\bm F_g$};
    \draw[vector,blue] (8,0)--(6,0) node[left]{$\bm F_g$};
    \shade[balloon1] circle (.7) node[white]{$m_1$};
    \shade[balloon2] (8,0) circle (1) node[white]{$m_2$};
    \draw[dashed] (0,0)--(0,-1.5);
    \draw[dashed] (8,0)--(8,-1.5);
    \draw[<->,thick] (0,-1.3)--(8,-1.3) node[midway,fill=white]{$r$};
  \end{tikzpicture}
  \caption{Gravity is a mutual attraction between massive objects}
\end{figure}




The expression for \textbf{gravitational potential energy} can be obtained
from the law of universal gravitation using basic integral calculus:

\begin{equation}
  W_g = \int_{\bm r_1}^{\bm r_2}\bm F_g\cdot\dl\bm r
  = Gm_1m_2 \int_{r_1}^{r_2}\frac{\dl r}{r^2} = -\Delta U_g
\end{equation}
where we again define the gravitational potential energy stored between two
point masses $m_1$ and $m_2$:
\begin{equation}
  \boxed{U_g=-\frac{Gm_1m_2}r}
\end{equation}
$U_g$ is the work required to move two objects from $r$ to $\infty$. $U_g=0$ at
$r=\infty$ and \emph{decrease} as $r$ decreases




%\begin{frame}{Gravitational Potential Energy}
%  Since $g$ is not a constant, we use an equation consistent with the law of
%  universal gravity to obtain the general expression for
%  \textbf{gravitational potential energy} stored between a system of two
%  masses:
%  
%  \eq{-.05in}{
%    \boxed{U_g=-\frac{Gm_1m_2}r}
%  }
%  \begin{center}
%    \begin{tabular}{l|c|c}
%      \rowcolor{pink}
%      \textbf{Quantity} & \textbf{Symbol} & \textbf{SI Unit} \\ \hline
%      Gravitational potential energy & $U_g$ & \si\joule \\
%      Point masses & $m_1$, $m_2$ & \si{\kilo\gram} \\
%      Distance between centres of mass & $r$ & \si\metre \\
%      Universal gravitational constant & $G$ & \si{N.m^2/kg^2}
%    \end{tabular}
%  \end{center}
%  The ``reference level'' is chosen at infinity (i.e.\ $U_g=0$ at $r=\infty$)
%  and \emph{decrease} as $r$ decreases
%\end{frame}

As we look at work done by other forces, we will begin to see the same pattern
emerge for other forces.



\subsection{Spring Force \& Elastic Potential Energy}
The \textbf{spring force}, or \textbf{elastic force} $\bm F_e$ is the force
that a compressed/stretched spring exerts on the object connected to it, shown
in Fig.~\ref{hooke1}. An \emph{ideal} spring obeys \textbf{Hooke's law}, which
states that the spring force is proportional to the amount of spring
displacement $\bm x$, and acts in the opposition to the displacement:
\begin{equation}
  \boxed{
    \bm F_e=-k\bm x
  }
\end{equation}
The constant $k$ is called the \textbf{spring constant}\footnote{The spring
constant is also called the \textbf{force constant}, \textbf{Hooke's constant},
and in many engineering textbooks, \textbf{spring rate}.}, with a unit of
\emph{newton per meter} (\si{\newton\per\metre}). The spring constant
represents the stiffness of the spring, and it depends on the geometry of the
spring, as well as the material that it is made of.
\begin{figure}[ht]
  \centering
  \begin{tikzpicture}
    \draw[mass] (5,.5) rectangle (6,1.5);
    \draw[thick,
      decoration={aspect=.6,segment length=5mm, amplitude=2.5mm, coil},
      decorate] (0,1)--(5,1);
    \fill[pattern=north east lines] (-.2,0) rectangle (0,2);
    \draw[thick] (0,.0)--(0,2);
    \fill[red] (5.5,1) circle (.06);
    \draw[vector,red] (5.5,1)--(4,1) node[above]{$\bm F_e$};
    \draw[dashed] (3,0)--(3,2) node[above]{unstretched/equilibrium};
    \draw[vector] (3,.3)--(5,.3) node[midway,below]{$\bm x$};
  \end{tikzpicture}
  \hspace{.2in}
  \begin{tikzpicture}
    \fill[pattern=north east lines] (-.2,0) rectangle (0,2);
    \draw[thick] (0,0)--(0,2);
    \draw[dashed] (3,0)--(3,2) node[above]{unstretched/equilibrium};;
    \draw[mass] (1.5,.5) rectangle (2.5,1.5);
    \draw[thick,decorate,
      decoration={aspect=.3,segment length=1.5mm, amplitude=2.5mm, coil}]
    (0,1)--(1.5,1);
    \draw[vector] (3,.3)--(1.5,.3) node[midway,below]{$\bm x$};
    \fill[red] (2,1) circle (.06);
    \draw[vector,red] (2,1)--(3,1) node[above]{$\bm F_e$};
  \end{tikzpicture}
  \caption{Direction of spring force is always opposite to spring
    displacement.}
  \label{hooke1}
\end{figure}

As the spring force moves an object connected to the spring, the work done by
the spring force on the object is:
\begin{equation}
  W_e=\int_{x_0}^{x_1}F_e\dl x =-k\int_{x_0}^{x_1} x\dl x
  =-\frac12kx^2\Big|^{x_1}_{x_0}=-\Delta U_e
  \label{eq:spring-pot1}
\end{equation}
where $U_e$ is now defined as the \textbf{elastic potential energy}:
\begin{equation}
  \boxed{
    U_e=\frac12kx^2
  }
\end{equation}
Crucially, the work done by the spring force is related to the elastic
potential energy by:
\begin{equation}
  \boxed{
    W_e=-\Delta U_e
  }
\end{equation}
The integration in Eq.~\ref{eq:spring-pot1} shows the same properties as in
the work done by gravitational force that was previously shown in
Section \ref{section:gravitationalPE}:
%\begin{definition}
%  \begin{itemize}[itemsep=3pt,leftmargin=12pt]%[nosep]%,leftmargin=10pt]
%  \item When work done by the spring force is \emph{positive}, (i.e.\
%    $W_e>0$), there is a \emph{decrease} in spring potential energy
%    (i.e.\ $U_e<0$) by the same amount, while
%  \item When work done by the spring force is \emph{negative}, (i.e.\
%    $W_e<0$), there is a \emph{increase} in spring potential energy
%    (i.e.\ $U_e>0$) by the same amount
%  \item Work by the spring force is \emph{path independent}: $W_e$ depends on
%    the end points $x_0$ and $x_1$, but not \emph{how} the object moves from
%    $x_0$ to $x_1$
%  \item Only work done by $\bm F_e$ can affect $U_e$
%  \end{itemize}
%\end{definition}


\subsection{Electrostatic Force \& Electric Potential Energy}
Similar to the attractive force between masses, there is also a mutually
attractive or repulsive force between charged particles, given by
\textbf{Coulomb's law}:
\begin{equation}
  \boxed{
    \bm F_q=\frac{kq_1q_2}{r^2}\hat{\bm r}
  }
\end{equation}

The integral is nearly identical to that for the gravitational force:
\begin{equation}
  W_q=\int\bm F_q\cdot\dl\bm r
  =kq_1q_2\int_{r_0}^{r_1}\frac{\dl r}{r^2}
  =-\frac{kq_1q_2} r\Big|^{r_2}_{r_1}=-\Delta U_q
\end{equation}
where $U_q$ is the \textbf{electric potential energy} that is stored between
the two point charges, defined as:
\begin{equation}
  \boxed{
    U_q = \frac{kq_1q_2}r
  }
\end{equation}

Again, we see the same properties that we have observed 

%\begin{definition}
%  \begin{itemize}[nosep,leftmargin=10pt]
%  \item\emph{Positive} work by the electric force \emph{decreases} electric
%    potential energy, while
%  \item\emph{Negative} work by the electric force \emph{increases} electric
%    potential energy
%  \item $W_q$ depends on the end points $r_0$ and $r_1$, but not \emph{how}
%    it went from $r_0\rightarrow r_1$
%  \item Only work done by $\bm F_q$ can affect $U_q$
%  \end{itemize}
%\end{definition}


\section{Conservative Forces}

These forces are called \textbf{conservative forces}
\begin{itemize}[nosep]
\item Gravitational force $\bm F_g$
\item Spring force $\bm F_e$
\item Electrostatic force $\bm F_q$
\item Magnetic force $\bm F_m$
\item Nuclear forces
\end{itemize}
Because they shared these common properties:
\begin{itemize}[nosep]
\item The work done by these forces relate to a change of a potential energy
  \begin{itemize}[nosep]
  \item Positive work decreases this related potential energy
  \item Negative work increases this related potential energy
  \end{itemize}
\item The work done by a conservative force is \emph{path independent}, in that
  it depends only on end points, but not \emph{how} it gets from one end point
  to the other
\end{itemize}



By the fundamental theorem of calculus, any conservative forces $\bm F$
must be the negative gradient of the potential energies:
%\begin{equation}
%  \boxed{
%    \bm F=-\nabla U=-\diffp Ux\iii-\diffp Uy\jjj-\diffp Uz\kkk
%  }
%\end{equation}
In one-dimension, the gradient operator becomes just:
\begin{equation}
  \boxed{
    F=-\diff Ux
  }
\end{equation}
The direction of a conservative force \emph{always} decreases the potential
energy. (Pay attention to the negative sign. Students often forget it.)



%\section{Energy Diagrams}
%A plot of potential energy ($U$) vs.\ position ($x$) for a conservative force
%\begin{figure}[ht]
%  \centering
%  \begin{tikzpicture}[scale=.8]
%    \draw[axes] (0,0)--(10,0) node[right]{$x$};
%    \draw[axes] (0,0)--(0,5) node[right]{$U(x)$};
%    \draw[very thick] (.2,4.5) to[out=-70,in=180](1.5,2.5)
%    to[out=0,in=180] (2.5,1)
%    --(5.5,1) node[midway,below]{Neutral equilibrium}
%    to[out=0,in=180] (7,3.5)
%    to[out=0,in=180] (8,2.5)
%    to[out=0,in=250] (9.5,4.5);
%    \fill (1.5,2.5) circle (.07) node[below]{$A$};
%    \fill (7,3.5) circle (.07) node[below]{$B$};
%    \fill (8,2.5) circle (.07) node[above]{$C$};
%    \draw[<-] (1.6,2.7)--(2,3.3) node[right]{Unstable equilibrium};
%    \draw[<-] (6.9,3.6)--(6.4,4.2) node[left]{Unstable equilibrium};
%    \draw[<-] (8,2.4)--(8,1.5) node[below]{Stable equilibrium};
%  \end{tikzpicture}
%\end{figure}
%
%  The expressions for potential energies also come from integrating the work
%  equation, in that work equals to the change in \emph{something}, and we
%  called that potential energy. Therefore:
%
%  \eq{-.2in}{
%    \boxed{
%      W_c=-\Delta U
%    }
%  }
%  \begin{itemize}
%  \item\vspace{-.15in}$\Delta U$ can be positive or negative depending on the
%    direction of the (conservative) force
%  \item Positive work \emph{decreases} the related potential energy
%  \item Negative work \emph{increases} the related potential energy
%  \end{itemize}

%  Positive work done by conservative forces on an object does two things:
%  \begin{enumerate}[1.]
%  \item Decrease its potential energy, while
%  \item Increase its kinetic energy by the same amount
%  \end{enumerate}
%  Mathematically, this shows that mechanical energy must \emph{always} be
%  conserved when there are only conservative forces:
%
%  \eq{-.1in}{
%    W_c=-\Delta U = \Delta K \quad\longrightarrow\quad
%    \Delta K + \Delta U =0
%  }
%
%  That's why those forces are called conservative forces, and they form the
%  basis for conservation of energy.


\section{Non-Conservative Forces}
The majority of the common forces encountered in introductory physics courses
are generally \textbf{non-conservative}. They include, but not limited to:
\begin{itemize}[nosep]
\item Applied force
\item Tension force
\item Normal force
\item Static friction
\item Kinetic friction
\item Aerodynamic lift and drag 
\end{itemize}
The work-energy theorem (Eq.~\ref{work-energy-theorem}) still applies for
non-conservative forces. However, the work done by non-conservative forces
differs from conservative forces in that:
\begin{itemize}[leftmargin=12pt,itemsep=4pt]
\item There is \textbf{no related potential energies}: the work done by a
  non-conservative force transform energy from one form of kinetic energy to
  another
\item The work is \textbf{path dependent}
\end{itemize}


\subsection{Work by Static Friction}
We can illustrate the work done by a non-conservative force in general, by
examining how work is done by static friction, using a standard multi-body
problem studied in dynamics.\footnote{The focus of the dynamics study is often
to calculate the maximum acceleration $\bm a_\text{max}$ of the blocks
without them sliding against each other, and the maximum applied force
$\bm F_\text{max}$ associated with that maximum acceleration.}

Two blocks with masses $m_1$ and $m_2$, stacked vertically, move to the right
without slipping by an external applied force $\bm F$ applied to $m_1$, as
shown in Fig.~\ref{stacked1}. The coefficient of static friction between the
two blocks is $\mu_s$, but the contact between $m_1$ and the table is
frictionless. The focus of \emph{this} example is to find the work done by the
forces between the blocks as the blocks accelerate.
\begin{figure}[ht]
  \centering
  \begin{tikzpicture}
    \fill[pattern=north east lines] rectangle (6,-.2);
    \draw[thick] (0,0)--(6,0);
    \draw[thick] (1,0) rectangle (4,1.2) node[midway]{$m_1$};
    \draw[thick] (1.7,1.2) rectangle (3.3,2) node[midway]{$m_2$};
    \draw[vector] (4,.6)--+(1.5,0) node[right] {$\bm F$};
    \draw[<-] (.95,.05) to[out=150,in=0] (0,.5) node[left]{frictionless};
    \draw[<-] (3.35,1.25) to[out=20, in=180] (4.5,1.8) node[right]{$\mu_s$};
  \end{tikzpicture}
  \caption{An external force is applied to accelerate two stacked blocks to
    the right.}
  \label{stacked1}
\end{figure}

The free-body diagrams of the blocks are shown in Fig.~\ref{stacked-fbd}.
(The forces highlighted in the same colour are action-reaction pairs.) The
static friction between $m_1$ and $m_2$ is $\bm f$.
\begin{figure}[ht]
  \centering
  \begin{tikzpicture}
    \fill circle (.1);
    \draw[vector] (0,0)--(0,-1) node[below]{$\bm F_{g2}$};
    \draw[vector] (0,0)--(0, 1) node[above,fill=pink!30]{$\bm N_{12}$};
    \draw[vector] (0,0)--(1.5,0) node[right,fill=yellow!30]{$\bm f$};

    \fill (5,0) circle (.1);
    \draw[vector] (4.97,0)--+(0,-1) node[below left]{$\bm F_{g1}$};
    \draw[vector] (5.03,0)--+(0,-1)
    node[below right,fill=pink!30]{$\bm N_{12}$};
    \draw[vector] (5,0)--+(0,1) node[above]{$\bm N_1$};
    \draw[vector] (5,0)--+(-1.5,0) node[left,fill=yellow!30]{$\bm f$};
    \draw[vector] (5,0)--+(2,0) node[right]{$\bm F$};
  \end{tikzpicture}
  \caption{Free-body diagrams of the stacked masses.}
  \label{stacked-fbd}
\end{figure}

The top block $m_2$ accelerates to the right because there is static
friction $\bm f$ at the interface with $m_1$. (Without friction, $m_2$
would just slide off $m_1$ when the force is applied.) Static friction is
the \emph{only} force doing (positive) work on $m_2$. Normal force
$\bm N_{12}$ and gravitational force $\bm F_{g2}$ are perpendicular to
motion, and therefore, do not do any work.
%, and the work done by $\bm f$ is
%\emph{positive}.
As static friction pulls $m_2$ to the right, it accelerates and gains kinetic
energy, consistent with the work-energy theorem
(Eq.~\ref{work-energy-theorem}), and kinematics and dynamics. The change in
kinetic energy in $m_2$ as it moves from $x_0\longrightarrow x_1$ is
\begin{equation*}
  \Delta K_2=\int_{x_0}^{x_1} f\dl x
\end{equation*}
Note that since we are not assuming whether applied force $\bm F$ is
constant, we cannot assume whether static friction is constant.

On the bottom block $m_1$, as it moves to the right, applied force $\bm F$
does positive work, while in this case, static friction $\bm f$ does
\emph{negative} work. At a minimum, we conclude that energy is transferred
from $m_1$ to $m_2$, but we can do a more thorough (although still very simple)
analysis to find out how much.

Without static friction
%between $m_1$ and $m_2$,
the net force on $m_1$ would have just been $\bm F$, and the change in
kinetic energy moving from $x_0\longrightarrow x_1$ to the right is simply
\begin{equation*}
  \Delta K_1 = \int_{x_0}^{x_1} F\dl x\quad\quad\text{(no friction)}
\end{equation*}
However, with static friction present, the gain in kinetic energy is reduced:
\begin{equation*}
  \Delta K_1 = \int_{x_0}^{x_1} F_\text{net}\dl x =
  \int_{x_0}^{x_1} (F-f)\dl x = \int_{x_0}^{x_1} F\dl x -
  \underbrace{\int_{x_0}^{x_1} f\dl x}_{\Delta K_2}
\end{equation*}
Work done by static friction reduced the kinetic energy of $m_1$ by the same
amount that is gained by $m_2$. We therefore conclude that work done by
friction transfers kinetic energy from one block to another.




\subsection{Work by Kinetic Friction}





\section{Internal Energy}
Consider a container of gas of mass $M$ that oscillates vertically from the
ceiling by a spring with stiffness $k$. The container moves at speed $v$ at a
height $h$ above Earth (shown in Fig.~\ref{fig:gas}).
\begin{figure}[ht]
  \centering
  \begin{tikzpicture}[scale=.7]
    \begin{scope}[thick]
      \draw (-1,5)--(3,5);
      \draw (-1,-4)--(3,-4) node[right]{$h=0$};
    \draw[dashed] (-1,2.8)--+(4,0) node[right]{Unstretched};
    \draw[decoration={aspect=.4, segment length=5, amplitude=6, coil},
      decorate] (1,5)--(1,2.1) node[midway,right=4]{$k$};
    \draw[vector] (.5,2.8)--(.5,2.1) node[midway,left]{$\bm x$};
    \draw[fill=gray!10] (-.1,-.1) rectangle (2.1,2.1);
    \draw[vector] (2.5,2)--(2.5,0) node[midway,right]{$\bm v$};
    \draw[|<-] (-.5,1)--(-.5,-4) node[midway,left]{$h$};
    \end{scope}
    \foreach \i in {1,...,90} \fill[red] (rand+1,rand+1) circle (.05);
  \end{tikzpicture}
  \caption{A container of gas, suspended vertically, and moving above Earth}
    %has kinetic energy, gravitational potential energy, elastic potential
    %energy, as well as internal energies}
  \label{fig:gas}
\end{figure}
During this oscillation, the container has a bulk kinetic energy of
\begin{equation*}
  K=\dfrac12 Mv^2
\end{equation*}
a gravitational potential energy\footnote{Using the ground level as the
reference ($h=0$)} of
\begin{equation*}
  U_g=Mgh
\end{equation*}
where $h$ is the height of the container above Earth. The is also an elastic
potential energy of
\begin{equation*}
  U_e=\frac12kx^2
\end{equation*}
stored when the spring is stretched or compressed. But the random motion of
the air molecules in the container also contribute to an additional energy,
called the \textbf{internal energy} $E_\text{int}$, or \textbf{thermal energy}.

Internal energy of a system of molecules is the sum of all their kinetic and
potential energies at the \emph{microscopic} level:
\begin{equation}
  \boxed{
    E_\text{int}=K_\text{micro} + U_\text{micro}
  }
\end{equation}
As the name suggests, $E_\text{int}$ is proportional to molecules'
\textbf{absolute temperature}, measured in \emph{kelvin}.
\begin{remark}
  For those who are keen to know, for a monatomic or ideal gas, the internal
  energy comes entirely from the kinetic energy from the 3 degrees of
  translational freedom, and is given by
  \begin{equation*}
    E_\text{int}=\dfrac32Nk_bT
  \end{equation*}
  where $N$ is the number of molecules, $k_b$ is the Boltzmann's constant.
  For a diatomic gas, there are 3 degrees of translational freedom, and 2
  degrees of rotational freedom, and the internal energy is given by:
  \begin{equation*}
    E_\text{int}\approx\dfrac52NkT
  \end{equation*}
  For some solids, there are 3 degrees of translational freedom, and 3 degrees
  of degrees of vibrational freedom, and therefore the internal energy is:
  \begin{equation*}
    E_\text{int}\approx3Nk_bT
  \end{equation*}
\end{remark}%sidenote}
%\end{figure}
Discussions on thermal/internal energy and the behaviour of gases and solids
are part of a much larger discipline within physics called
\textbf{thermodynamics}, but it is outside the scope of this book.



\section{Conservation of Energy}
Conservation of energy is most often stated using a statement that even a lay
person with no physics background watching a movie may be familiar with:
\begin{definition}
  \textbf{Conservation of Energy (non-technical version):} Energy cannot
  be created or destroyed; it only changes in form.
\end{definition}
While this statement certainly is not incorrect---indeed, it captures the
\emph{essence} of how energy is conserved---the actual law for the conservation
of energy is more nuanced. In application, it is often a tedius exercise in
bookkeeping:
\begin{definition}
  \textbf{Law of conservation of energy:} The change in the total energy
  of a system is equal to the net external work done to the system.
\end{definition}
We can express the law of conservation of energy with the simple equation:
\begin{equation}
  \boxed{
    \Delta E_\text{sys}=W_\text{ext}
  }
  \label{eq:energy-conservation-law}
\end{equation}



\subsection{System}

Before we can show how energy \emph{must} be conserved, we must first define
what a system is. A \emph{system} of objects is defined in the same way as in
solving dynamics problems: 
%(specifically, the multi-body problems studied in
%Section~\ref{sec:multibody}):
a system is a predefined collection of objects that apply forces on each other.
Because object apply forces on each other\footnote{This statement may be
plainly obvious to some, but not to
many others: the third law of motion is at work here when objects applies
forces on each other.}, they may also do work on each other. A system may be:
\begin{itemize}[leftmargin=12pt,topsep=0pt,itemsep=3pt]
\item\textbf{Isolated:} All the forces that act on the objects in the system
  act on each other (third law of motion) and are therefore \emph{internal}.
  There are no external forces.
\item\textbf{Closed:} There are external forces, but they do not do any work.
\item\textbf{Open:} There are external forces, and they do mechanical work
  (``external work'') to the objects in the system
\end{itemize}

The system's energy include both all the kinetic and potential energies of the
objects (collectively known as the \textbf{mechanical energy}), as well as the
internal/thermal energies of the objects:
\begin{equation}
  \boxed{
    E_\text{sys}
    =\underbracket[1.3pt]{\sum K+\sum U}_\text{mechanical energy}+
    \sum E_\text{int}
  }
  \label{eq:system-energy}
\end{equation}
The \emph{change} in system energy must therefore be the total change of all
the energies in the systems. i.e.:
\begin{equation}
  \boxed{
    \Delta E_\text{sys}
    =\sum\Delta K+\sum\Delta U+\sum\Delta E_\text{int}
  }
  \label{eq:system-energy-change}
\end{equation}
Substituting the expression from Eq.~(\ref{eq:system-energy}) into
(\ref{eq:energy-conservation-law}), we get the ``change'' in system energy:
\begin{equation}
  \boxed{
    \sum\Delta K + \sum\Delta U + \sum\Delta E_\text{int}=W_\text{ext}
  }
\end{equation}
The sign convention of the external work $W_\text{ext}$ is the same as how it
is defined earlier in %Section~\ref{sec:mechwork}.
Section~\ref{sec:mechwork}. When external work is:

\begin{itemize}[leftmargin=12pt,nosep]
\item\textbf{ositive:} work is done \emph{to} the system to the surrounding;
  the system energy therefore \emph{increases}
\item\textbf{negative:} work is done \emph{by} the system to the surrounding;
  the system energy therefore \emph{decreases}
\end{itemize}
In isolated systems that does not interact with the outside (and therefore
no external work can be done to the system), the law of conservation of energy
further simplifies to
\begin{equation}
  \boxed{
    \sum\Delta K+\sum\Delta U+\sum\Delta E_\text{int}=0
  }
\end{equation}

%  In almost all of the problem encountered in AP Physics C, there will be no
%  change in the internal energy of the system, and conservation of energy
%  reduces to:
%  
%  \eq{-.1in}{
%    \boxed{ \Delta K + \Delta U = W_\text{ext} }\quad\rightarrow\quad
%    \boxed{ U_1 + K_1 + W_\text{ext} = U_2 + K_2 }
%  }
%

\section{Energy Conservation in a Gravitational System}


\subsection{Free-Fall}
Assuming that there are no friction, drag (air resistance)%see Section~\ref{sec:drag}),
or other damping forces present, a free-falling object (shown in
Fig.~\ref{fig:earth}) forms an isolated system with Earth. 
\begin{figure}[ht]
  \centering
  \begin{tikzpicture}[scale=.65]
    \draw[thick,fill=cyan!5] (7.75,0) arc (75:105:30);
    \draw[mass] (0,7) circle (.2) node[right=3]{$m$};
    \draw[vector,red] (0,7)--+(0,-2) node[below=-2]{$\bm F_g$};
    \draw[vector,red] (0,.5)--+(0,2) node[above=-2]{$\bm F_g$};
  \end{tikzpicture}
  \caption{A free-falling object forms an isolated system with Earth}
  \label{fig:earth}
\end{figure}
This system consists of only Earth and the mass $m$, and therefore the energy
of the system is the kinetic energy of the mass ($K$) and the gravitational
potential energy ($U_g$)\footnote{$U_g$ is often described as ``the
gravitational potential energy of the object''. Strictly speaking, this is
incorrect.
%this is a topic that will be studied in more detail in
%Section~\ref{section:gravity}.
However, this will not affect how the calculations is done, and in the
opinion of the author, poses no major conceptual issue. You are, of course,
welcome to disagree.
} stored between the mass and Earth.

The only force that in the system is the gravitational force $\bm F_g$,
which (by the third law of motion) acts on \emph{both} the mass and Earth, and
is therefore \emph{internal} to the system. Since there are no other external
forces, this is an isolated system.
As the object falls, the gravitational force is doing work on both the mass as
well as Earth\footnote{Due to the mass of Earth, its displacement due
to this gravitational force is laughably insignificant. We will therefore only
focus on the motion of the mass}. Since gravitational force is conservative,
\begin{itemize}[leftmargin=15pt,topsep=0pt,itemsep=3pt]
\item If the mass falls towards Earth, the \emph{positive} work by gravity
  transforms gravitational potential energy into kinetic energy of the mass by
  the same amount
\item If the free-``falling'' mass moves away from Earth, the \emph{negative}
  work by gravity transforms kinetic energy of the mass into gravitational
  potential energy by the same amount
\end{itemize}
and since $\bm F_g$ is internal to the system, the change in the system's
energy is zero.
\begin{equation}
  \Delta K + \Delta U_g=0
\end{equation}
In practice, we can also use the following equation for problem solving:
\begin{equation}
  \underbracket[1.3pt]{K+U_g}_\text{initial state}
  =\underbracket[1.3pt]{K'+U_g'}_\text{final state}
\end{equation}

\begin{example}
  A ball is thrown upwards with an initial speed of \SI5{\metre\per\second}
  from the third-floor balcony of an apartment building, \SI{10}{\metre} above
  the ground. If friction and air resistance can be ignored, with what speed
  would it hit the ground?
  
  \textbf{Solution:} This is a simple problem that any beginner
  physics student would have solved in kinematics, but we will solve using
  conservation of energy instead.
  Since there is no friction and drag, the
  total mechanical energy is conserved between when the ball is thrown, and the
  moment just before it hits the ground:
  \begin{displaymath}
    \underbracket[1.3pt]{K + U_g}_\text{thrown} =
    \underbracket[1.3pt]{K' + U_g'}_\text{just before hitting the ground}
  \end{displaymath}
  Setting $h=0$ at ground level (our reference level), and substituting the
  expressions for $K$ and $U_g$ the above equattion, we have
  \begin{displaymath}
    \frac12 mv^2 + mgh = \frac12 mv'^2
  \end{displaymath}
  Cancelling mass terms, then solving for $v'$ and substituting numerical
  values, we have:
  \begin{displaymath}
    v'= \sqrt{v^2+2gh}=\sqrt{5^2+2\cdot9.8\cdot10}=
  \end{displaymath}
  The above equation is, of course, identical to the one-dimensional kinematic
  equation (as one should expect!). But there are some differences:
  \begin{itemize}[itemsep=4pt,leftmargin=12pt]
  \item This is a scalar problem. We actually don't know explicitly the
    direction of the ball when it hits the ground (although it is plainly
    obvious)
  \item We don't know how long it took to hit the ground.
  \item All we know is the condition at the end points.
  \end{itemize}
  %What is remarkable about this example is that even though the derivation
  %of the energy conservation is complex, the application is simple algebra.
\end{example}

\begin{example}
  Let's try something \emph{slightly} more difficult for a bit more insight.

  \vspace{4pt}\textbf{Question:} A ball is thrown with an upward angle of
  \SI{60}{\degree} with an initial speed of \SI5{\metre\per\second} from the
  same third-floor balcony, \SI{10}{\metre} above the ground. If friction and
  air resistance can be ignored, with what speed would it hit the ground?
  \begin{center}
    \begin{tikzpicture}
      \draw[thick] (-1,0)--(0,0)--(0,-2)--(4,-2);
    \end{tikzpicture}
  \end{center}
  \vspace{4pt}\textbf{Solution:}
  %This highlights the difference between solving
  %a problem by kinematics vs.\ solving it by conservation of energy.
  Again, since there is no friction and drag, the total energy is conserved
  between when the ball is thrown ($K+U_g$), and just before it hits the ground
  ($K'+U_g'$):
  \begin{displaymath}
    K + U_g = K' + U_g'
  \end{displaymath}
  Setting $h=0$ at ground level and substituting the expressions for $K$ and
  $U_g$ into the above equation, as we did in the previous example:
  \begin{displaymath}
    \frac12 mv^2 + mgh = \frac12 mv'^2
  \end{displaymath}
  Again, cancelling mass terms, then solving for $v'$ and finally substituting
  numerical values:
  \begin{displaymath}
    v'= \sqrt{v^2+2gh}=\sqrt{5^2+2\cdot9.8\cdot10}=
  \end{displaymath}
  This is of course the same answer as the previous example. When we solve this
  problem using 2D kinematics, we would have to split the problem into
  horizontal and vertical motion. We would have to find the vertical velocity
  when the ball hits the ground, then combine with the horizontal velocity
  to find the \emph{speed}. Using conservation of energy allows us to solve a
  scalar problem instead of a vector problem, but\ldots
\end{example}
Talk about why we don't include thermal energy in the examples.


\subsection{Arbitrary Ramps}
Another type of gravitational system that is similar to the free-falling
object is an object that is sliding along an arbitrary ramp without friction,
drag, or other damping forces, as shown in Fig.~\ref{fig:ramp-system}.
%Assuming that there is no friction or drag, energy is also conserved for an
%object sliding down an arbitrarily-shaped ramp:
\begin{figure}[ht]
  \centering
  \begin{tikzpicture}
    \draw[thick] (0,4) to[out=-30,in=180] (3,1) to[out=0,in=180] (5,3)
    to[out=0,in=170] (8,0) to[out=-10,in=180] (10,0);
    \draw[mass,rotate around={-60.5:(1,2.93)}] (1,2.93) rectangle +(.6,.6);
    \fill[red] (1.4,2.8) circle (.08);
    \draw[vector,red] (1.4,2.8)--+(0,-1.5) node[left]{$\bm F_g$};
    \draw[vector,red,rotate around={30:(1.4,2.8)}] (1.4,2.8)--+(1.4,0)
    node[right]{$\bm F_n$};
  \end{tikzpicture}
  \caption{A mass sliding along an arbitrary ramp forms an isolated system
    with Earth.}
  \label{fig:ramp-system}
\end{figure}
Like the free-fall case, the system also consists of the mass and Earth, and
the system energy consists of the kinetic energy $K$ of the mass, and the
gravitational potential energy $U_g$ stored between the mass and Earth. The
two object interact \emph{with each other} via the gravitational force
$\bm F_g$ and the normal force $\bm F_n$ at the interface at the ramp.
Since these two forces act on both objects, it is \emph{internal} to the
system. There are no other forces from the outside, and the system is isolated.

As the mass slides down (or up!) along the ramp, normal force $\bm F_n$ is
always perpendicular to motion, therefore the only force that does work is
gravity $\bm F_g$. And since gravitational force is conservative,
\begin{itemize}[leftmargin=15pt]
\item when the mass slides downward, the \emph{positive} work done by gravity
  transforms gravitational potential energy into kinetic energy by the same
  amount
\item when the mass slides upwards, the \emph{negative} work done by gravity
  transforms kinetic energy of the mass into gravitational potential energy by
  the same amount
\end{itemize}
Similar to the free-falling case. Therefore the net change in the system energy
is zero, i.e. the total system energy is constant
\begin{equation}
  \Delta K + \Delta U_g=0\quad\text{or}\quad
  K+U_g=\text{constant}
\end{equation}
The shape of the ramp does not matter, only the initial and final height
relative to the reference.

\begin{example}
  A skier glides with a speed of \SI{2.0}{\metre\per\second} at the top of a
  ski hill, \SI{40}{\metre} high. She then begins to slide down
  the icy (i.e.\ frictionless) hill.
  %\footnote{In reality, there will always be
  %\emph{some} friction and drag as she slides down. In that case, we will also
  %need to know the non-conservative work done by friction.}
  \begin{enumerate}[leftmargin=12pt]
  \item What is the skier's speed at a height of \SI{25}\metre?
  \item At what height does the skier have a speed of
    \SI{10}{\metre\per\second}?
  \end{enumerate}
\end{example}



\section{Spring-Mass Systems}

\subsection{Horizontal Spring-Mass System}
In a horizontal spring-mass system, a mass $m$ sits on a level horizontal
surface. One end of a spring is attached to the mass, and the other to a fixed
point, thus allowing it to vibrate horizontally, as shown in
Fig.~\ref{fig:hspring-mass}.
\begin{figure}[ht]
  \centering
  \begin{tikzpicture}
    \draw[mass] (5,.5) rectangle (6,1.5);
    \draw[thick,decorate,
      decoration={coil,amplitude=6,aspect=.5,segment length=6}] (0,1)--(5,1);
    \fill[pattern=north east lines] (6.5,.5)--(6.5,.3)--(-.2,.3)
    --(-.2,2)--(0,2)--(0,.5)--cycle;
    \draw[very thick] (0,2)--(0,.5)--(6.5,.5);
    \draw[vector,red] (5.5,1)--(5.5,0) node[below]{$\bm F_g$};
    \draw[vector,red] (5.5,1)--(5.5,2) node[above]{$\bm F_n$};
    \draw[vector,red] (5.5,1)--(4.5,1) node[above]{$\bm F_e$};
    \fill[red] (5.5,1) circle (.06);
  \end{tikzpicture}
  \caption{A horizontal spring-mass system without friction or drag is a
  closed system}
  \label{fig:hspring-mass}
\end{figure}
Assuming that there are no friction, drag or other damping forces present, this
horizontal spring-mass system is a closed system: it consists of the mass and
the spring. (Earth is not part of the system.) Therefore the system's total
mechanical energy consists of the kinetic energy of the mass ($K$) and the
elastic potential energy stored in the spring ($U_e$). Two \emph{external}
forces act on the system: the gravitational force ($\bm F_g$) and normal
force ($\bm F_n$) which act on the mass. However, these forces do not do
any work because they are perpendicular to the motion of the mass. Therefore
there is no external work, and the total energy of the system is constant:
\begin{equation}
  K+U_e=\text{constant}
\end{equation}


\begin{example}
  A toy cart with a mass of \SI{.25}{\kilo\gram} travels
  along a frictionless horizontal track and collides head on with a spring that
  has a spring constant of \SI{155}{\newton\per\metre}. If the spring is
  compressed by \SI{6.0}{\centi\metre}, how fast is the cart initially
  travelling?
  \begin{center}
    \begin{tikzpicture}
      \fill[blue!30] rectangle (10,-.3);
      \draw (0,0)--(10,0);
      \draw[fill=brown] (.5,.2) rectangle (3,.5) node[midway,above=3]{$m$};
      \draw[fill=gray] (1,.2) circle (.2);
      \draw[fill=gray] (2.5,.2) circle (.2);
      \fill (1,.2) circle (.06);
      \fill (2.5,.2) circle (.06);
      \draw[vector] (3,.35)--(4.5,.35) node[above]{$v$};
      \draw[pattern=north east lines] (9.3,0) rectangle (10,1.7);
      \draw[ultra thick,decorate,
        decoration={aspect=.4,segment length=1.5mm, amplitude=2mm, coil}]
      (6.3,.4)--(9.3,.4) node[midway,above=3]{spring};
    \end{tikzpicture}
  \end{center}
\end{example}

\subsection{Vertical Spring-Mass System}
In a vertical spring-mass system, a mass is suspended vertically by a
massless spring, as shown in Fig.~\ref{fig:vspring-mass}.
\begin{figure}[ht]
  \centering
  \begin{tikzpicture}
    \draw[mass] (.5,1.5) rectangle (1.5,2.5);% node[midway]{$m$};
    \draw[thick,decorate,decoration={
        aspect=.5,segment length=7, amplitude=5,coil}] (1,5)--(1,2.5); 
    \fill[pattern=north east lines] (0,5) rectangle (2,5.2);
    \draw[very thick] (0,5)--(2,5);
    \draw[vector,red] (1,2)--(1,1) node[right]{$\bm F_g$};
    \draw[vector,red] (1,2)--(1,3) node[right]{$\bm F_e$};
    \fill[red] (1,2) circle (.05);
  \end{tikzpicture}
  \caption{A vertical spring-mass system without friction or drag is an
  isolated system.}
  \label{fig:vspring-mass}
\end{figure}
The mass is allowed to oscillate vertically. This system consists of the mass,
the spring and Earth, therefore, the total energy of the system includes the
kinetic energy of the mass ($K$), the gravitational potential energy stored
between the mass and Earth ($U_g$), and the elastic potential energy stored in
the spring ($U_e$). Assuming that there are no friction, drag or other damping
forces in the spring, the only forces are the spring force $\bm F_e$ and
the gravitational force $\bm F_g$, both of which are \emph{internal} to the
system. Therefore, there are no external forces, and the vertical spring-mass
system is an isolated system. The total energy remains constant as the mass
oscillates vertically.
\begin{equation}
  K + U_g + U_e=\text{constant}
\end{equation}


\begin{example}
  Even when the mass isn't always attached to the spring.
\end{example}


\section{Simple Pendulum}
\label{sec:simple-pendulum-energy}
In a simple pendulum, a mass is hung from a massless string from a fixed point,
and is allowed to swing back and forth, as shown in
Fig.~\ref{fig:pendulum-system}.
%    \begin{itemize}
%    \item Gravity ($m\bm g$), which is conservative, is the only force that
%      does work
%    \item Tension ($\bm F_T$), which is non-conservative, does not do work on the
%      pendulum because it is always perpendicular to the motion of the pendulum
%      bob
%    \end{itemize}
\begin{figure}[ht]
  \centering
  \begin{tikzpicture}
    \fill[pattern=north east lines] (-1,0) rectangle (1,.2);
    \draw[thick] (-1,0)--(1,0);
    \begin{scope}[rotate=15]
      \draw[thick] (0,0)--(0,-5);
      \draw[mass] (0,-5) circle (.2) node[right=4]{$m$};
      \draw[red,vector] (0,-5)--(0,-3.5) node[left]{$\bm F_T$};
      \draw[red,vector,rotate around={-15:(0,-5)}] (0,-5)--(0,-6.3)
      node[below]{$\bm F_g$};
    \end{scope}
    \draw[dashed,thin] (0,0)--(0,-5);
    \draw[axes] (0,-2) arc (270:285:2) node[midway,below]{$\phi$};
  \end{tikzpicture}
  \caption{A simple pendulum system is a closed system}
  \label{fig:pendulum-system}
\end{figure}
The system consists of the pendulum bob (the mass) and Earth. As such, the
total energy in the system include the kinetic energy of the mass ($K$), as
well as the gravitational potential energy stored between the mass and Earth
($U_g$). Assuming that there are no friction, drag or other damping forces in
the spring, there are only two forces act on the pendulum bob: the tension force
$\bm F_T$, and the gravitational force $\bm F_g$. Gravitational force
is an internal force (because it acts on both the pendulum bob and Earth), but
tension force is an \emph{external} force. However, since $\bm F_T$ is
always perpendicular to the motion of the pendulum bob, it does not do any
work, and the system is therefore a closed system. The total energy of the
system is constant:
\begin{equation}
  K + U_g =K'+U_g'
\end{equation}
    

\begin{example}
  What is remarkable about this example is that even though the derivation
  of the energy conservation is complex, the application is simple algebra.
\end{example}


\section{Accounting for Changing Internal Energy}
Energy is \emph{always} conserved as long as the system is defined properly.

In this case, the system consists of a mass, a spring, Earth and all the air
molecules inside the box:
\begin{figure}[ht]
  \centering
  \begin{tikzpicture}[scale=.8,thick]
    \fill[pattern=north east lines] rectangle (5,4);
    \draw rectangle (5,4);
    \draw[fill=blue!5] (.2,.2) rectangle (4.8,3.8);
    \draw[thick,decorate,
      decoration={aspect=.3,segment length=2mm, amplitude=2.5mm, coil}]
    (2.5,3.8)--(2.5,2.25) node[midway,right=4]{$k$};
    \draw[mass] (2,2.25) rectangle (3,1.25) node[midway]{$m$};
  \end{tikzpicture}
  \caption{In this isolated system, mechanical energy decreases in time while
    internal/thermal energy increases.}
  \label{fig:isolated-with-thermal}
\end{figure}
The energies of this system include the kinetic energy of the mass ($K$), the
gravitational potential energy ($U_g$) between the mass and Earth, the elastic
potential energy ($U_e$) stored in the spring, as well as the internal/thermal
energy ($E_\text{int}$) of the air molecules and the mass.

As the mass vibrates, friction and drag with air slows it down, converting the
kinetic energy of the mass into the internal energy of the air. Total energy
is conserved even as the mass stops moving.
\begin{equation}
  K + E_\text{int}+U_g+U_e=\text{constant}
\end{equation}

%  \vspace{.2in}Non-conservative forces doing work are \emph{internal} to the
%  system, and therefore energy is still conserved. (Work done by friction
%  transform from the kinetic energy of the mass to the kinetic energy of the
%  air molecules.)


\section{Isolated vs.\ Open System}
Accounting for the change in the internal energy of the air molecules is not
always practical, especially when the air molecules are not confined to a box.
\begin{figure}[ht]
  \centering
  \begin{tikzpicture}[thick]
    \fill[blue!5] (-3,0) rectangle (8,4);
    \draw (-3,4)--(8,4);
    \draw[
      decoration={aspect=.3,segment length=2mm, amplitude=2.5mm, coil},
      decorate] (2.5,4)--(2.5,2.25) node[midway,right=4]{$k$};
    \draw[mass] (2,2.25) rectangle (3,1.25) node[midway]{$m$};
  \end{tikzpicture}
  \caption{A system with friction is drag is usually treated as an open
    system.}
  \label{fig:open-system}
\end{figure}

The solution:
\begin{itemize}
\item Take the air molecule out of the \emph{system}
\item No longer an isolated system
\end{itemize}
The negative work done by kinetic friction and drag (collectively known as
$W_f$) are treated as external work between initial and final states
\begin{equation}
  \underbrace{K + U_g + U_e}_\text{initial state} + W_f=
  \underbrace{K' + U_g' + U_e'}_\text{final state}
\end{equation}

%  If \emph{only} conservative forces are doing work, mechanical energy (i.e.\
%  $K+U$) is always conserved:
%
%  \eq{-.2in}{
%    \boxed{K+U =K'+U'}
%  }
%  
%  When external non-conservative forces are also doing work, instead of
%  \emph{trying} to isolate the system, we can instead calculate the work done
%  by them $W_{nc}$ and add it to the total energy of the system
%    
%  \eq{-.2in}{
%    \boxed{K+U+W_{nc}=K'+U'}
%  }

%\begin{example}
%  A mass $m$ is dropped from a height of $h$ above the equilibrium position of
%  a spring. Set up the equation that determines the spring's compression $d$
%  when the object is instantaneously at rest.
%  \begin{center}
%    \pic{.8}{../energy-calculus/spring-example1}
%  \end{center}
%\end{example}

%  \textbf{Example 3:} A mass $m$ is pulled a distance $d$ up an incline (angle
%  of elevation $\theta$) at constant speed using a rope that is parallel to
%  the incline. The coefficient of friction is $\mu_k$.
%  \begin{enumerate}[(a)]
%  \item What is the magnitude of the tension force in the rope?
%  \item What is the magnitude of the normal force?
%  \item What is the work done by the normal force?
%  \item What is the work done by friction?
%  \item What is the work done by the tension force?
%  \item What is the net work?
%  \item What is the change in total mechanical energy?
%  \item Show that $\Delta E_{mech}=W_{nc}$.
%  \end{enumerate}


\begin{example}
  A spring with a spring constant of \SI{200}{\newton\per\metre} is used to
  send a box sliding across a rough horizontal surface. The coefficient of
  kinetic friction between the box and the horizontal surface is $\mu=0.25$.
  The box, with a mass of \SI{500}\gram, is pushed against the horizontal
  spring, compressing it by \SI{12}{\centi\metre}. It is then released. The box
  reaches maximum speed at equilibrium, when the forces on the box are
  balanced. (Note that because of the kinetic friction on the box, the
  equilibrium position is \emph{not} when the spring has fully decompressed.)
  \begin{center}
    \begin{tikzpicture}[scale=1.5]
      \draw[thick,
        decoration={aspect=.4,segment length=4, amplitude=7, coil},
        decorate] (0,.4)--(1.5,.4);
      \draw[mass] (1.5,0) rectangle (2.2,.8) node[midway]{$m$};
      \draw[ultra thick] (0,1.25)--(0,0)--(6,0);
      \draw[|<-,thick] (2.2,-.2)--(4.5,-.2)
      node[midway,below]{\SI{12}{\centi\metre}};
      \draw[dash dot,thick] (4.5,-.5)--(4.5,1.25)
      node[above]{Unstretched position};
    \end{tikzpicture}
  \end{center}
  \begin{enumerate}[leftmargin=15pt]%,label={\Alph.}]
  %\item What is the speed of the box at the instant the spring is passing
  %  the point of being compressed by \SI8{\centi\metre}? (It has now moved
  %  \SI4{\centi\metre} away from the initial compression.)
    
  \item Draw a free-body diagram of the box when it is at equilibrium $x_E$,
    and calulate the spring displacement at equilibrium.
    
  \item At this equilibrium position, the box's speed is at its maximum.
    What is this maximum speed?

  \item The box loses contact with the spring after the spring has fully
    decompressed. It then slides along the same surface (with $\mu=0.25$)
    until frictional force slows the box to a stop. How far will the box
    travel after leaving the spring?
  \end{enumerate}
  \textbf{Solutions:} This is an interesting problem because of the presence of
  friction force.

  \begin{enumerate}[leftmargin=15pt]%,label={\Alph.}]
  \item To find the equilibrium position, we have to draw a free-body diagram:
    \begin{center}
      \begin{tikzpicture}[scale=.8,vector]
        \fill circle (.07);
        \draw (0,0)--+(0,-1.5) node[below]{$\vec F_g$};
        \draw (0,0)--+(0,1.5) node[above]{$\vec F_n$};
        \draw (0,0)--+(2,0) node[below]{$\vec F_e$};
        \draw (0,0)--+(-2,0) node[below]{$\vec f$};
      \end{tikzpicture}
    \end{center}
    It should be obvious that the magnitude of normal and gravitational forces
    are the same: $F_n=F_g=mg$. At this point, we have
    \begin{align*}
      f &=F_e\quad\rightarrow\quad \mu F_n=kx_E
      \quad\rightarrow\quad \mu mg=kx_E\\
      x_E&=\frac{\mu mg}k=\frac{.25\cdot0.5\cdot9.81}{200}=\SI{0.000631}\metre
    \end{align*}
    (What is even more interesting is that, if the box doesn't leave the spring,
    it will oscillate back and forth, with its amplitude decreasing each time.
    The equilibrium position changes every oscillation.)
  \item Now we can use the conservation energy to find the speed at
    equilibrium. Like the horizontal spring-mass example, our system consists
    of the mass and the spring. Therefore system energy (like the horizontal
    spring-mass) is the kinetic energy of the box ($K$) and the elastic
    potential energy of the spring ($U_e$). Friction is an external force, and
    the conservation of energy now becomes
    \begin{equation*}
      \Delta K + \Delta U_e = W_\text{fric}
    \end{equation*}
    We can substitute the change in kinetic energy (which is just $K'$ at
    equilibrum, since initially the box is at rest) and the change in potential
    energy. Frictional work is negative, and since kinetic friction is constant,
    there is no need to integrate:
    \begin{equation*}
      \frac12mv^2 + \left[\frac12kx_E^2-\frac12kx^2\right] = -\mu mg(x-x_E)
    \end{equation*}
    Solving for $v$ and substituting numerical values, we find the maximum
    speed of the box:
    \begin{align*}
      v &= \sqrt{\frac km\left(x^2-x_E^2\right)-2\mu g(x-x_E)}\\
      &= \sqrt{\frac{200}{.5}\left(.12^2-.000631^2\right)
        -2\cdot.25\cdot9.81(.12-0.00063)}\\
    \end{align*}
  \end{enumerate}
\end{example}

\section{Power \& Efficiency}
\textbf{Power} is the \emph{rate} at which work is done, i.e.\ the rate at
which energy is being transformed:
\begin{equation}
  \boxed{P(t) = \diff Wt}\quad\quad
  \boxed{\overline P = \frac W{\Delta t}}
\end{equation}
%\begin{center}
%  \begin{tabular}{l|c|c}
%    \rowcolor{pink}
%    \textbf{Quantity}  & \textbf{Symbol} & \textbf{SI Unit} \\ \hline
%    Instantaneous and average power & $P$, $\overline P$ & \si\watt \\
%    Work done          & $W$ & \si\joule \\
%    Time interval      & $\Delta t$ & \si\second
%  \end{tabular}
%\end{center}
In engineering, power is often more critical than the actual amount of work
done.

If a force is used to push an object at a constant velocity, the
power produced by the force is:
\begin{equation}
  P=\diff Wt=\frac{\bm F\cdot\dl\bm x}{\dl t}
  =\bm F\cdot\diff{\bm x}t
  \quad\rightarrow\quad
  \boxed{P=\bm F\cdot\bm v}
\end{equation}
Application: aerodynamics
\begin{itemize}
\item When an object moves through air, the applied force must overcome air
  resistance (drag force), which is proportional with $v^2$
\item Therefore ``aerodynamic power'' must scale with $v^3$ (i.e.\ doubling
  your speed requires $2^3=8$ times more power)
\item Important when aerodynamic forces dominate
\end{itemize}

\textbf{Efficiency} is the ratio of useful energy or work output to the total
energy or work input
\begin{equation}
  \boxed{ \eta = \frac{E_o}{E_i}\times\SI{100}\percent }\quad
  \boxed{ \eta = \frac{W_o}{W_i}\times\SI{100}\percent }
\end{equation}
\begin{center}
  \begin{tabular}{l|c|c}
    \rowcolor{pink}
    \textbf{Quantity} & \textbf{Symbol} & \textbf{SI Unit} \\ \hline
    Useful output energy & $E_o$  & \si\joule \\
    Input energy         & $E_i$  & \si\joule \\
    Useful output work   & $W_o$  & \si\joule \\
    Input work           & $W_i$  & \si\joule \\
    Efficiency           & $\eta$ & no units
  \end{tabular}
\end{center}
Efficiency is always $0\leq\eta<\SI{100}\percent$
%\theendnotes


%\section{Exercise Questions}
%
%\section*{Multiple-Choice Questions}
%
%\begin{multicols*}{2}
%  \begin{enumerate}[leftmargin=12pt,itemsep=4pt]
%  \item A \SI1{\kilo\gram} ball is thrown vertically downward from a
%    \SI{50}{\metre} high tower with an initial speed of \SI4{\metre\per\second}.
%    Just before striking the ground, the speed of the ball is
%    \SI{20}{\metre\per\second}. The energy lost to air friction is most nearly
%    \begin{enumerate}[nosep,topsep=0pt]
%      \item\SI{101}\joule
%      \item\SI{210}\joule
%      \item\SI{308}\joule
%      \item\SI{406}\joule
%      \item\SI{508}\joule
%    \end{enumerate}
%    
%    %TL%  \textbf{Questions \ref{q:pq1}--\ref{q:pq2}}:
%  \item A \SI2{\kilo\gram} projectile is launched with a speed of
%    \SI{20}{\metre\per\second} from horizontal ground at an angle of \ang{37}
%    to the horizontal, as shown. Point $P$ is at the top of the path, and point
%    $Q$ is at the end of the path, just before the projectile again reaches the
%    ground. The kinetic energy of the projectile at point $P$ is
%    %  \cpic{.35}{symprojectile}
%    %    \begin{choices}
%    %      \choice\SI{108}\joule
%    %      \choice\SI{225}\joule
%    %      \choice\SI{256}\joule
%    %      \choice\SI{400}\joule
%    %      \choice\SI{525}\joule
%    %    \end{choices}
%    
%  \item The kinetic energy of the projectile at point $Q$ from the previous
%    question is
%    %    \begin{choices}
%    %      \choice\SI{108}\joule
%    %      \choice\SI{225}\joule
%    %      \choice\SI{256}\joule
%    %      \choice\SI{400}\joule
%    %      \choice\SI{525}\joule
%    %    \end{choices}
%
%  \item If a projectile thrown directly upward reaches a maximum height
%    $h$ and spends a total time in the air of $T$, then returning to the
%    original location, the average power of the gravitational force is
%    \begin{enumerate}[nosep,topsep=0pt]
%    \item $P=2mgh/T$
%    \item $P=-2mgh/T$
%    \item 0
%    \item $P=mgh/T$
%    \item $P=-mgh/T$
%    \end{enumerate}
%    
%  \item Given that the constant net force on an object and the object's 
%    displacement, which of the following quantities can be calculated?
%    \begin{enumerate}[nosep,topsep=0pt]
%    \item the net change in the object's velocity
%    \item the net change in the object's mechanical energy
%    \item the average acceleration
%    \item the net change in the object's kinetic energy
%    \item the net change in the object's potential energy
%    \end{enumerate}
%    
%  \item A conservative force acting on an object varies with the equation
%    $F(x)=-3x^2-2x-4$, where force is in newtons and displacement is in metres.
%    The potential energy at $x=\SI2\metre$ is
%    \begin{enumerate}[nosep,topsep=0pt]
%    \item zero
%    \item\SI{20}\joule
%    \item\SI{40}\joule
%    \item\SI{-20}\joule
%    \item\SI{-40}\joule
%    \end{enumerate}
%    %\columnbreak
%    
%  \item The only force acting on an object is given by the equation
%    $F(x)=2-4x$, where the force is measured in newtons and position in meters.
%    What is the change in the object's kinetic energy as it moves from $x=2$ to
%    $x=1$?
%    \begin{enumerate}[nosep,topsep=0pt]
%    \item +\SI4\joule
%    \item \SI{-4}\joule
%    \item +\SI2\joule
%    \item \SI{-2}\joule
%    \item +\SI8\joule
%    \end{enumerate}
%    
%  \item The potential energy of an object varies with the equation
%    $U(x)=2x^2+x-6$, where force is in newtons and displacement is in meters. A
%    force $F$ vs.\ displacement $x$ graph would yield which of the following?
%    \begin{enumerate}[nosep,topsep=0pt]
%    \item A straight, horizontal line
%    \item A parabola
%    \item An exponential decay curve
%    \item A straight line with a positive slope
%    \item A straight line with a negative slope
%    \end{enumerate}
%    
%  \item A particle of mass $m$ moves according to the displacement
%    equation $x=2t^{5/2}$. The kinetic energy of the particle as a function of
%    time is
%    \begin{enumerate}[nosep,topsep=0pt]
%    \item $10mt^{5/2}$
%    \item $10mt^{3/2}$
%    \item $\dfrac{25}2mt^3$
%    \item $5mt^2$
%    \item $2mt^{3/2}$
%    \end{enumerate}
%    
%  \item An electron travels in a circle around a hydrogen nucleus at a
%    very high speed. The work done by the electrostatic force acting on the
%    electron after one complete revolution is
%    \begin{enumerate}[nosep,topsep=0pt]
%    \item zero
%    \item positive
%    \item negative
%    \item equal to the kinetic energy of the electron
%    \item equal to the potential energy of the electron
%    \end{enumerate}
%    
%  \item An object is moved from rest at point $P$ to rest at point $Q$ in
%    a gravitational field. The net work against the gravitational field depends
%    on the
%    \begin{enumerate}[nosep,topsep=0pt]
%    \item mass of the object and the positions of $P$ and $Q$
%    \item mass of the object only
%    \item positions of $P$ and $Q$ only
%    \item length moved between points $P$ and $Q$
%    \item coefficient of friction
%    \end{enumerate}
%    
%  \item Two blocks of mass $m_A$ and $m_B$ are connected by a string that
%    passes over a light pulley. The mass of $A$ is larger than the mass of $B$.
%    The speed of mass $A$ just before reaching the floor is:
%    \begin{center}
%      \begin{tikzpicture}[scale=.7,thick]
%        \draw circle(1);
%        \draw (-1,0)--(-1,-1.5);
%        \draw (-1.5,-1.5) rectangle (-.5,-2.5) node[midway]{$A$};
%        \draw (1,0)--(1,-3);
%        \draw (1.5,-3) rectangle (.5,-4)  node[midway]{$B$};
%        \draw (-1.8,-2.5)--(-2.2,-2.5);
%        \draw[<->] (-2,-2.5)--(-2,-5.5) node[midway,fill=white]{$D$};
%        \fill[gray!50] (-2.3,-5.7) rectangle (2,-5.5);
%        \draw (-2.3,-5.5)--(2,-5.5);
%      \end{tikzpicture}
%    \end{center}
%    \begin{enumerate}[nosep,topsep=0pt]
%    \item $\sqrt{\dfrac{2(m_A-m_B)}{m_A+m_B}gD}$
%    \item $\sqrt{\dfrac{2(m_A+m_B)}{m_A-m_B}gD}$
%    \item $\sqrt{\dfrac{2m_A}{m_A+m_B}gD}$
%    \item $\sqrt{\dfrac{2m_B}{m_A+m_B}gD}$
%    \item $\sqrt{\dfrac{2m_A}{m_B}gD}$
%    \end{enumerate}
%    
%%TL    \uplevel{
%%TL      \textbf{Questions \ref{q:well1}--\ref{q:well2}}: Consider the potential
%%TL      energy function shown below.
%%TL      \cpic{.28}{potential-well}
%%TL    }
%%TL
%%TL    \question Assuming that no non-conservative forces are present, if a
%%TL    particle of mass $m$ is released from position $x_0$, what is the maximum
%%TL    speed it will achieve?
%%TL    \label{q:well1}
%%TL    \begin{choices}
%%TL      \choice $\sqrt{4U_0/m}$
%%TL      \choice $\sqrt{2U_0/m}$
%%TL      \choice $\sqrt{U_0/m}$
%%TL      \choice $\sqrt{U_0/2m}$
%%TL      \choice The particle will achieve no maximum speed but instead will
%%TL      continue to accelerate indefinitely.
%%TL    \end{choices}
%%TL    
%%TL    \question Which of the following is the most accurate description of the
%%TL    system introduced in the previous question?
%%TL    \label{q:well2}
%%TL    \begin{choices}
%%TL      \choice A stable equilibrium
%%TL      \choice An unstable equilibrium
%%TL      \choice A neutral equilibrium
%%TL      \choice A bound system
%%TL      \choice There is a linear restoring force
%%TL    \end{choices}
%%TL    \columnbreak
%
%  \item A pendulum bob of mass $m$ is released from rest as shown in the
%    figure below. What is the tension in the string as the pendulum swings
%    through the lowest point of its motion?
%    \begin{center}
%      \begin{tikzpicture}
%        \draw[thick,dashed] (0,0)--(0,-3.5);
%        \begin{scope}[rotate=60]
%          \draw[very thick] (0,0)--(0,-4);
%          \fill (0,-4) circle (.13) node[right]{$\;m$};
%          \draw[<->,thick] (.5,0)--(.5,-4) node[midway,above right]{$l$};
%        \end{scope}
%        \draw[<->,thick] (0,-1) arc (270:330:1) node[pos=.6,below]{\ang{60}};
%      \end{tikzpicture}
%    \end{center}
%%    \begin{choices}
%%      \choice $T=\dfrac12mg$
%%      \choice $T=mg$
%%      \choice $T=\dfrac32mg$
%%      \choice $T=2mg$
%%      \choice None of the above
%%    \end{choices}
%
%  \item A force is applied to a block of mass $m$ at a downward angle of
%    $\theta$ to the vertical as shown. The block moves with a constant speed
%    across a rough floor for a distance $x$. The work done by the applied force
%    on the block is
%    \begin{center}
%      \begin{tikzpicture}
%        \fill[gray!50] (0,-.15) rectangle (5,0);
%        \draw[thick] (0,0)--(5,0);
%        \draw[thick] (1.5,0) rectangle (3,1);
%        \draw[<-,thick] (3,1)--(4.3,2.3);
%        \draw[thick,dashed] (2.5,2.3)--(4.3,2.3)--(4.3,.8);
%        \draw[thick] (4.3,1.6) arc (270:225:.7)
%        node[midway,below left]{$\theta$};
%      \end{tikzpicture}
%    \end{center}
%%    \begin{choices}
%%      \choice $Fx\sin\theta$
%%      \choice $Fx\cos\theta$
%%      \choice $Fmx\sin\theta$
%%      \choice $Fmx\cos\theta$
%%      \choice zero
%%    \end{choices}
%
%%TL%    \uplevel{
%%TL%      \textbf{Questions \ref{sphere1}--\ref{sphere2}}:      
%%TL%      A small block rests on the top of a smooth sphere of radius $R$ when it
%%TL%      is given a light tap so that it just begins sliding on the sphere. When
%%TL%      the block reaches the angle $\theta$, it loses contact with the surface
%%TL%      of the sphere.
%%TL%      \begin{center}
%%TL%        \begin{tikzpicture}[scale=.9]
%%TL%          \draw[thick]circle (2);
%%TL%          \fill circle (.05);
%%TL%          \draw[axes] (0,0)--(0,2) node[midway,left]{$R$};
%%TL%          \draw[thick] (-.2,2) rectangle(.2,2.4);
%%TL%          \begin{scope}[rotate around={-40:(0,0)}]
%%TL%            \draw[thick,dashed] (0,0)--(0,2);
%%TL%            \draw[thick] (-.2,2) rectangle (.2,2.4);
%%TL%            \draw[vector] (0,2.2)--(.8,2.2) node[right]{$\mb{v}$};
%%TL%          \end{scope}
%%TL%          \draw[thick] (0,.6) arc (90:50:.6) node[midway,above]{$\theta$};
%%TL%        \end{tikzpicture}
%%TL%      \end{center}
%%TL%    }
%%TL%
%%TL%    \question The kinetic energy of the block as it leaves the surface of the
%%TL%    sphere is
%%TL%    \label{sphere1}
%%TL%    \begin{choices}
%%TL%      \choice $mgR$
%%TL%      \choice $mgR\cos\theta$
%%TL%      \choice $mgR\sin\theta$
%%TL%      \choice $mg(R-R\cos\theta)$
%%TL%      \choice $mg(R-R\sin\theta)$
%%TL%    \end{choices}
%%TL%
%%TL%    \question The speed of the block as it leaves the surface of the sphere is
%%TL%    \label{sphere2}
%%TL%    \begin{choices}
%%TL%      \choice $\sqrt{2g}m$
%%TL%      \choice $\sqrt{2gR}m$
%%TL%      \choice $\sqrt{2gR\cos\theta}$
%%TL%      \choice $\sqrt{2gR(1-\cos\theta)}$
%%TL%      \choice $\sqrt{2gR(1-\sin\theta)}$
%%TL%    \end{choices}
%    
%  \item A machine can lift large weights according to the power equation
%    $P(t)=4t^3+3t^2-2$, where power is in watts and time is in seconds. The
%    energy expended by the machine from $t=0$ to $t=\SI{10}{\second}$ is
%    %\begin{choices}
%    %  \choice \SI{1260}\joule
%    %  \choice \SI{3630}\joule
%    %  \choice \SI{9240}\joule
%    %  \choice \SI{10980}\joule
%    %  \choice \SI{18150}\joule
%    %\end{choices}
%  \end{enumerate}
%\end{multicols*}
%
%\subsection*{Problem-Solving Questions}
%
%%\begin{enumerate}
%%TL  \setcounter{question}{\value{lastmc}}
%%TL
%%TL  % TAKEN FROM THE 2003 AP PHYSICS C MECHANICS FREE-RESPONSE QUESTION MECH.1
%%TL  \question The \SI{100}{\kilo\gram} box shown below is being pulled along the
%%TL  $x$-axis by a student. The box slides across a rough surface, and its
%%TL  position $x$ varies with time $t$ according to the equation
%%TL  $x= 0.5t^3+2t$, where $x$ is in meters and $t$ is in seconds.
%%TL  \cpic{.3}{student}
%%TL  \begin{parts}
%%TL    \part Determine the speed of the box at time $t=0$.
%%TL    \vspace{\stretch1}
%%TL    
%%TL    \part Determine the kinetic energy of the box as a function of time $t$.
%%TL    \vspace{\stretch1}
%%TL      
%%TL    \part Determine the net force acting on the box as a function of time $t$.
%%TL    \vspace{\stretch1}
%%TL
%%TL    \part Determine the power being delivered to the box as a function of time
%%TL    $t$.
%%TL    \vspace{\stretch1}
%%TL    
%%TL    \part Calculate the net work done on the box in the interval $t=0$ to
%%TL    $t=\SI2\second$.
%%TL    \vspace{\stretch1}
%%TL  \end{parts}
%%TL  \newpage
%%TL  
%%TL  % THIS QUESTION ORIGINALLY CAME FROM ONE OF THE AP PHYSICS REFERENCE BOOKS.
%%TL  % IT IS AN "OKAY" QUESTION, BUT I WANT TO AVOID IT AS MUCH AS I CAN MOVING
%%TL  % FORWARD.
%%TL%  \question A mass $m$ is placed on an incline of angle $\theta$ at a distance
%%TL%  $d$ from the end of a spring as shown below. The coefficient of kinetic
%%TL%  friction between the mass and the plane is $\mu$.
%%TL%  \cpic{.3}{ramp1}
%%TL%  \begin{parts}
%%TL%    \part The mass is released from rest at the position shown. Using Newton's
%%TL%    laws, calculate the block's speed when it reaches the spring.
%%TL%    
%%TL%    \part Using energy conservation, alculate the block's speed when it reaches
%%TL%    the spring.
%%TL%    
%%TL%  \part The spring has spring constant $k$. At what value $x$ of the compression
%%TL%    of the spring does the object reach its maximum speed?
%%TL%  \end{parts}
%%TL%  \newpage
%%TL  
%%TL  % TAKEN FROM THE 2009 AP PHYSICS C EXAM FREE-RESPONSE QUETION MECH 1
%%TL  \question A \SI{3.0}{\kilo\gram} object is moving along the $x$-axis in a
%%TL  region where its potential energy as a function of $x$ is given as
%%TL  $U(x)=4.0x^2$, where $U$ is in joules and $x$ is in meters. When the object
%%TL  passes the point $x=\SI{-0.50}\metre$, its velocity is
%%TL  $+$\SI{2.0}{\metre\per\second}. All forces acting on the object are
%%TL  conservative.
%%TL  \begin{parts}
%%TL    \part Calculate the total mechanical energy of the object.
%%TL    \vspace{\stretch1}
%%TL    
%%TL    \part Calculate the $x$-coordinate of any points at which the object has
%%TL    zero kinetic energy.
%%TL    \vspace{\stretch1}
%%TL    
%%TL    \part Calculate the magnitude of the momentum of the object at
%%TL    $x=\SI{.60}\metre$.
%%TL    \vspace{\stretch1}
%%TL    
%%TL    \part Calculate the magnitude of the acceleration of the object as it passes
%%TL    $x=\SI{.60}\metre$.
%%TL    \vspace{\stretch1}
%%TL    \newpage
%%TL    
%%TL    \part On the axes below, sketch graphs of the object's position $x$ versus
%%TL    time $t$ and kinetic energy $K$ versus time $t$. Assume that $x=0$ at time
%%TL    $t=0$. The two graphs should cover the same time interval and use the same
%%TL    scale on the horizontal axes.
%%TL    \begin{center}
%%TL      \begin{tikzpicture}
%%TL        \draw[axes] (-1,0)--(10,0) node[right]{$t$};
%%TL        \draw[axes] (0,-3)--(0,3) node[above]{$x$};
%%TL      \end{tikzpicture}
%%TL      
%%TL      \vspace{.4in}
%%TL      \begin{tikzpicture}
%%TL        \draw[axes] (-1,0)--(10,0) node[right]{$t$};
%%TL        \draw[axes] (0,-3)--(0,3) node[above]{$K$};
%%TL      \end{tikzpicture}
%%TL    \end{center}
%%TL  \end{parts}
%%TL  \newpage
%%TL  
%%TL  % TAKEN FROM THE 2007 AP PHYSICS C MECHANICS FREE-RESPONSE QUESTION MECH 3
%%TL  \uplevel{
%%TL    \cpic{.7}{track1}
%%TL  }
%%TL  \question The apparatus above is used to study conservation of mechanical
%%TL  energy. A spring of force constant \SI{40}{\newton\per\metre} is held
%%TL  horizontal over a horizontal air track, with one end attached to the air
%%TL  track. A light string is attached to the other end of the spring and connects
%%TL  it to a glider of mass $m$. The glider is pulled to stretch the spring an
%%TL  amount $x$ from equilibrium and then released. Before reaching the photogate,
%%TL  the glider attains its maximum speed and the string becomes slack. The
%%TL  photogate measures the time t that it takes the small block on top of the
%%TL  glider to pass through. Information about the distance $x$ and the speed
%%TL  $v$ of the glider as it passes through the photogate are given below.
%%TL  \begin{center}
%%TL    \begin{tabular}{|c|c|c|c|c|}
%%TL      \hline
%%TL      Trial \#  & Extension of the Spring $x$ (m) &
%%TL      Speed of Glider &
%%TL      Extension Squared &
%%TL      Speed Squared \\
%%TL      & $x$ (\si{\metre}) & $v$ (\si{\metre\per\second}) &
%%TL      $x^2$ (\si{\metre\squared}) & $v^2$ (\si{m^2/s^2}) \\
%%TL      \hline
%%TL      1 & \num{0.30e-1} & 0.47 & \num{0.09e-2} & 0.22\\\hline
%%TL      2 & \num{0.60e-1} & 0.87 & \num{0.36e-2} & 0.76\\\hline
%%TL      3 & \num{0.90e-1} & 1.3  & \num{0.81e-2} & 1.7\\\hline
%%TL      4 & \num{1.2 e-1} & 1.6  & \num{1.4 e-2} & 2.6\\\hline
%%TL      5 & \num{1.5 e-1} & 2.2  & \num{2.3 e-2} & 4.8\\\hline
%%TL    \end{tabular}
%%TL  \end{center}
%%TL  \begin{parts}
%%TL    \part Assuming no energy is lost, write the equation for conservation of
%%TL    mechanical energy that would apply to this situation.
%%TL    \vspace{\stretch1}
%%TL    
%%TL    \part On the grid below, plot $v^2$ versus $x^2$. Label the axes,
%%TL    including units and scale.
%%TL    
%%TL    \vspace{.2in}
%%TL    \begin{center}
%%TL      \begin{tikzpicture}[yscale=.3,xscale=.45]
%%TL        \draw[vector] (0,0)--(34,0);
%%TL        \draw[vector] (0,0)--(0,27);
%%TL        \draw[gray] grid (32,25);
%%TL        \draw[very thick,step=5] grid (32,25);
%%TL      \end{tikzpicture}
%%TL    \end{center}
%%TL    \part Draw a best-fit straight line through the data.
%%TL    \newpage
%%TL    
%%TL    \part Use the best-fit line to obtain the mass $m$ of the glider.
%%TL    \vspace{\stretch1}
%%TL    
%%TL    \part The track is now tilted at an angle $\theta$ as shown below. When the
%%TL    spring is unstretched, the center of the glider is a height $h$ above the
%%TL    photogate. The experiment is repeated with a variety of values of $x$.
%%TL    \cpic{.7}{track2}
%%TL    
%%TL    \part Assuming no energy is lost, write the new equation for conservation of
%%TL    mechanical energy that would apply to this situation.
%%TL    \vspace{\stretch1}
%%TL    
%%TL    \part Will the graph of $v^2$ versus $x^2$ for this new experiment be a
%%TL    straight line? 
%%TL
%%TL    \vspace{.2in}
%%TL    \underline{\hspace{.5in}} Yes\hspace{.5in}
%%TL    \underline{\hspace{.5in}} No
%%TL
%%TL    \vspace{.2in}Justify your answer.
%%TL    \vspace{\stretch1}
%%TL  \end{parts}
%%TL  \newpage
%%TL
%%TL%  % TAKEN FROM 2012 AP PHYSICS C MECHANICS EXAM FREE-RESPONSE QUESTION MECH 2
%%TL%  \question You are to perform an experiment investigating the conservation of
%%TL%  mechanical energy involving a transformation from initial gravitational
%%TL%  potential energy to translational kinetic energy.
%%TL%  \begin{parts}
%%TL%    \part You are given the equipment listed below, all the supports required
%%TL%    to hold the equipment, and a lab table. On the list below, indicate each
%%TL%    piece of equipment you would use by checking the line next to each item.
%%TL%
%%TL%    \begin{tabular}{lll}
%%TL%      \underline{\hspace{.3in}} Track &
%%TL%      \underline{\hspace{.3in}} Meterstick &
%%TL%      \underline{\hspace{.3in}} Set of objects of different masses\\
%%TL%      \underline{\hspace{.3in}} Cart &
%%TL%      \underline{\hspace{.3in}} Electronic balance &
%%TL%      \underline{\hspace{.3in}} Lightweight low-friction pulley\\
%%TL%      \underline{\hspace{.3in}} String &
%%TL%      \underline{\hspace{.3in}} Stopwatch & \\
%%TL%    \end{tabular}
%%TL%
%%TL%    \part Outline a procedure for performing the experiment. Include a diagram
%%TL%    of your experimental setup. Label the equipment in your diagram. Also
%%TL%    include a description of the measurements you would make and a symbol for
%%TL%    each measurement.
%%TL%    \label{procedure}
%%TL%    
%%TL%    \part Give a detailed account of the calculations of gravitational
%%TL%    potential energy and translational kinetic energy both before and after the
%%TL%    transformation, in terms of the quantities measured in part
%%TL%    (\ref{procedure}).
%%TL%
%%TL%    \part After your first trial, your calculations show that the energy
%%TL%    increased during the experiment. Assuming you made no mathematical errors,
%%TL%    give a reasonable explanation for this result.
%%TL%
%%TL%    \part On all other trials, your calculations show that the energy decreased
%%TL%    during the experiment. Assuming you made no mathematical errors, give a
%%TL%    reasonable physical explanation for the fact that the average energy you
%%TL%    determined decreased. Include references to conservative and
%%TL%    nonconservative forces, as appropriate.
%%TL%  \end{parts}
%%\end{enumerate}
%
