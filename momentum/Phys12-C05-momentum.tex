\chapter{Momentum, Impulse and Collisions}
\label{chapter:momentum}

%\newcommand{\half}{\ensuremath\frac12}
%
%
%
%\section{Momentum}
%
%\begin{frame}{A New Concept: Momentum}
%  You can't stop a bullet train or a bullet by yourself. Why not?
%  \begin{center}
%    \pic{.7}{graphics/japanese-high-speed-electric-long-train-website-header}\\
%    \pic{.7}{graphics/618520462903640723}
%  \end{center}
%
%
%
%
%\begin{frame}{A New Concept: Momentum}
%  What makes it so difficult to stop a train, or a speeding bullet, or a car?
%  \begin{itemize}
%  \item Both the train, the speeding bullet and the car have a lot of
%    \emph{momentum}
%  \item Momentum is related to both the \emph{mass} and \emph{velocity} of
%    an object
%  \item ``Mass in motion'': the tendency for the object to remain in the
%    same state of motion
%  \end{itemize}
%
%
%
%
%%\begin{frame}{Start with Inertia}
%%  \begin{block}{Inertia}
%%    The \emph{tendency} of a mass stay in its original state unless an external
%%    force act upon it. If a mass is in motion, then it will stay in motion. If
%%    a mass is at rest, then it will stay at rest. An object has inertia simply
%%    because it has mass. This has nothing to do with velocity or motion.
%%  \end{block}
%%
%%  The measurement of inertia is mass, and the SI unit is kilogram $kg$.

\section{Momentum}
\textbf{Momentum}\footnote{This is more accurately called
\textbf{translational momentum} or \textbf{linear momentum}.} is the product
of an object's mass $m$ and its velocity $\bm v$:
\begin{equation}
  \boxed{\bm p=m\bm v}
\end{equation}
%    \begin{tabular}{l|c|c}
%      \rowcolor{pink}
%      \textbf{Quantity} & \textbf{Symbol} & \textbf{SI Unit} \\ \hline
%      Momentum          & $\bm p$        & \si{\kilo\gram\metre\per\second}\\
%      Mass              & $m$             & \si{\kilo\gram} \\
%      Velocity          & $\bm v$        & \si{\metre\per\second}
%    \end{tabular}
%  \end{center}

Momentum is a \emph{vector}; the direction of the momentum vector is the
direction of the velocity vector.



%
%\begin{frame}{Simple Example: Hockey Puck}
%  \textbf{Example:} Determine the momentum of a \SI{.300}{\kilo\gram} hockey
%  puck travelling across the ice at a velocity of
%  \SI{5.55}{\metre\per\second} [N].
%
%
%
%
\section{Impulse}
The \textbf{impulse} $\bm J$ generated by a constant/average force $\bm F$
over time interval $\Delta t$ is defined as:
\begin{equation}
  \boxed{\bm J=\bm F\Delta t}
\end{equation}
%  \begin{center}
%    \begin{tabular}{l|c|c}
%      \rowcolor{pink}
%      \textbf{Quantity} & \textbf{Symbol} & \textbf{SI Unit} \\ \hline
%      Impulse           & $\bm J$    & \si{\newton\second}\\
%      Average force     & $\bm F$    & \si\newton \\
%      Time interval     & $\Delta t$  & \si\second
%    \end{tabular}
%  \end{center}
The direction of the Impulse vector is the same as the force vector. Impulse
from a specific force depends on the time interval that the force is applied,
but not on whether the object moves.


\begin{remark}
  If you with a background in integral calculus, for non-constant forces,
  impulse is given by the integral:
  \begin{equation*}
    \bm J=\int_{t_1}^{t_2}\bm F dt
  \end{equation*}
\end{remark}
%  Remember that calculus is not required for Grade 12 Physics.
%
%
%
\textbf{Impulse Generated by a Specific Force}
\begin{itemize}
\item Every force acting on an object generates an impulse
\item Can calculate the impulse generated by each force over the time interval
\end{itemize}

\textbf{Net Impulse on a Object}
\begin{itemize}
\item The vector sum of all the impulses on an object
\item The impulse generated by the net force, i.e.
  \begin{equation}
    \boxed{
      \bm J_\text{net} = \bm F_\text{net}\Delta t
    }
  \end{equation}
\end{itemize} 

\begin{example}
  Consider a box being pushed up a ramp. The impulses on the box by each of
  the forces can be calculated over a time interval $\Delta t$ of interest:
  \begin{center}
    \begin{tikzpicture}[rotate=30]
      \draw[thick] (0,0)--+(3,0);
      \draw[thick,rotate=-30] (0,0)--+(3,0);
      \draw[mass] (1,0) rectangle +(1,1);
      \fill[red] (1.5,.5) circle (.07);
      \draw[vectors,red] (1.5,.5)--+(1.5,0) node[right]{$\bm F_A$};
      \draw[vectors,red] (1.5,.5)--+(-1,0) node[left]{$\bm f_k$};
      \draw[vectors,red] (1.5,.5)--+(0,1.2) node[left]{$\bm F_N$};
      \draw[vectors,red,rotate around={-30:(1.5,.5)}]
      (1.5,.5)--+(0,-1) node[right]{$\bm F_g$};
    \end{tikzpicture}
  \end{center}
  \begin{itemize}
  \item Impulse from applied force: $\bm J_A =\bm F_A\Delta t$
  \item Impulse from kinetic friction: $\bm J_f =\bm f_k\Delta t$
  \item Impulse from normal force: $\bm J_N =\bm F_N\Delta t$
  \item Impulse from gravity: $\bm J_g =\bm F_g\Delta t$
  \end{itemize}
  
  The net impulse is the sum of all the impulses acting on the object:
  \begin{align*}
    \bm J_\text{net} &=\sum\bm J_i=\bm J_A+\bm J_f +\bm J_N + \bm J_g\\
    &=(\bm F_A+\bm f_k+\bm F_N+\bm F_g)\Delta t=\bm F_\text{net}\Delta t
  \end{align*}
\end{example}




\section{Impulse \& The Laws of Motion}
%  We can relate the net force to the change in momentum using the second law
%  of motion.
%
%  \eq{-.1in}{
%    \bm F_\text{net}=m\bm a
%    =m\frac{\Delta\bm v}{\Delta t}
%    =\frac{\Delta(m\bm v)}{\Delta t}
%    =\frac{\Delta\bm p}{\Delta t}
%  }

In fact, this is the general form of the second law of motion: \textbf{net
  force is the \emph{rate of change of momentum}}:
\begin{equation}
  \boxed{
    \bm F_\text{net}=\frac{\Delta\bm p}{\Delta t}
  }
\end{equation}  
The general form is necessary when mass changes with time.
%
%
%
%
%\begin{frame}{Momentum-Impulse Theorem}
%  Multiplying both sides of the second law of motion by $\Delta t$, we can see
%  that: %net
%  %impulse $\bm J_\text{net}=\bm F_\text{net}\Delta t$ is equal to the change
%  %in momentum $\Delta\bm p$:
%
%  \eq{-.1in}{
%    \bm F_\text{net}\Delta t = \Delta\bm p
%    %\quad\longrightarrow\quad
%    %\boxed{\bm J_\text{net}=\Delta\bm p=\bm p_2-\bm p_1}      
%  }

This is called the \textbf{momentum-impulse theorem}. The net
impulse %\footnote{i.e.\ impulse generated by the net force, or the vector sum
%of the impulses from all forces}
on an object is the object's change in momentum:
\begin{equation}
  \boxed{
    \bm J_\text{net}=\Delta\bm p=\bm p_2-\bm p_1
  }      
\end{equation}



\begin{example}[Golf Club]
  If a golf club exerts an average force of \SI{5.25e3}{\newton} [W] on a golf
  ball over a time interval of \SI{5.45e-4}\second, what is the impulse of the
  interaction?
\end{example}
%
%
%
%\begin{frame}{Why Average Force?}

%    \pic1{graphics/impulse1}
%
%    Why did the previous example use \textbf{average force}?




\begin{example}[A Slightly Longer Example]
  A student practices her tennis volleys by hitting a tennis ball against a
  wall.
  \begin{itemize}
  \item If the \SI{.060}{\kilo\gram} ball travels \SI{48}{\metre\per\second}
    before hitting the wall and then bounces directly backward at
    \SI{35}{\metre\per\second}, what is the impulse of the interaction?
  \item If the duration of the interaction if \SI{25}{\milli\second}, what is
    the average force exerted on the ball by the wall?
  \end{itemize}
\end{example}



\section{Conservation of Momentum}

\textbf{Conservation of momentum} is derived through the third law of
motion\footnote{which itself is an application of the first law of motion}.
When objects interact in isolation\footnote{Think of the isolated systems
discussed in the previous class}, the total momentum \emph{before} the
interaction is the same as \emph{after} the interaction:
%
%  \eq{-.13in}{
%    \boxed{
%      \sum_i\bm p_i=\sum_i\bm p_i'
%    }
%  }
%
%  \vspace{-.1in}Examples:
%  \begin{itemize}
%  \item Collision of two or more objects
%  \item A rocket expelling gas from the engine nozzle
%  \item Skaters pushing on each other on a friction-less ice surface
%  \item An exploding bomb
%  \end{itemize}
%
%
%
%
%\begin{frame}{Conservation of Momentum}
%  In Grade 12 Physics, we will limit our discussions primarily to only
%  2 objects. For a collision or explosion between two objects $A$ and $B$:
%
%  \eq{-.1in}{
%    \boxed{m_A\bm v_A+m_B\bm v_B=m_A\bm v_A'+m_B\bm v_B'}
%  }
%  \begin{center}
%    \begin{tabular}{l|c|c}
%      \rowcolor{pink}
%      \textbf{Quantity}      & \textbf{Symbol} & \textbf{SI Unit} \\ \hline
%      Masses of $A$ and $B$             & $m_A$, $m_B$ & \si{\kilo\gram} \\
%      Initial velocities of $A$ and $B$ & $\bm v_A$, $\bm v_B$ &
%      \si{\metre\per\second} \\
%      Final velocities of $A$ and $B$   & $\bm v_A'$, $\bm v_B'$ &
%      \si{\metre\per\second} \\
%    \end{tabular}
%  \end{center}
%  \textbf{Important note:} since momentum is a vector, the collision/explosion
%  problem \emph{must} be solve using vector arithmetic.



\begin{example}
  A \SI{1.75e4}{\kilo\gram} boxcar is rolling down a track towards a stationary
  boxcar that has a mass of \SI{2.00e4}{\kilo\gram}. Just before the collision,
  the first boxcar is moving east at \SI{5.45}{\metre\per\second}. When the
  boxcars collide, they lock together and continue to down the track. What is
  the velocity of the two boxcars immediately after the collision?
  \begin{center}
    \pic{.45}{momentum/graphics/boxcars}
  \end{center}
\end{example}
%
%
%
%%\begin{frame}{Example Problem: Canoe}
%%  \textbf{Example:} Two people A and B stand in a canoe on top of the water.
%%  Find the velocity of the canoe and person B at the instant that person A
%%  start to take a step, if her velocity is \SI{.75}{\metre\per\second}
%%  [forward]. Assume person A has a mass of 65 kg and the combined mass of the
%%  canoe, A and B, is \SI{115}{\kilo\gram}.
%%  \begin{center}
%%    \pic{.45}{graphics/canoe}
%%  \end{center}
%%
%
%
%
%%\begin{frame}{Conservation of Momentum}
%%  What happens if I let go of this balloon?
%%  \begin{center}
%%    \pic{.5}{graphics/Balloons4}
%%  \end{center}
%%  We will talk about this more in Unit 3 when we talk about propulsion in space
%%
%
%
%
\subsection{Glancing Collisions}
%  A glancing collision involves motion in 2D. Since momentum is a vector, the
%  calculation involves some vector arithmetic.
%  
%  \vspace{.3in}\textbf{Example:} A billiard ball of mass \SI{.165}{\kilo\gram}
%  (``cue ball'') moves with a velocity of \SI{1.25}{\metre\per\second} towards
%  a stationary billiard ball (``eight ball'') of mass \SI{.155}{\kilo\gram},
%  and strikes it with a glancing blow. The cue ball moves off at an angle of
%  \ang{29.7} clockwise from its original direction, with a speed of
%  \SI{.956}{\metre\per\second}.
%  \begin{enumerate}[(a)]
%  \item What is the final velocity of the eight ball?
%  \item Is the collision elastic?
%  \end{enumerate}
%
%  \vspace{.3in}We cannot answer the second part of the question yet (be patient,
%  we'll come back to this example later), but we can definitely solve the first
%  part.
%
%
%
%
\section{Elastic vs.\ Inelastic Collisions}
%
%\begin{frame}{A Full View of Collision}
%  There are two types of collisions: \textbf{elastic collision} and
%  \textbf{inelastic collision}
%
%  \vspace{.2in}Elastic collisions
%  \begin{itemize}
%  \item Momentum is conserved
%  \item Kinetic energy is conserved
%    \begin{itemize}
%    \item During the collision, kinetic energy is first transformed into a
%      potential energy (e.g.\ compressing a spring), and then 
%    \item All of the energy is then released back as kinetic energy
%    \end{itemize}
%  \item Often the two objects don't actually make
%    contact with each other
%  \item\textbf{You must NOT assume that a collision is elastic unless you are
%    told!}
%  \end{itemize}
%
%
%
%\begin{frame}{Collisions}
%  Inelastic collisions (the majority of all collisions)
%  \begin{itemize}
%  \item Momentum is conserved
%  \item Kinetic energy is lost due to heat, friction or sound
%  \item Energy is conserved to motion just before and/or just after the
%    collision (depends on the situation)
%  \end{itemize}
%
%  \vspace{.15in}A special case of inelastic collision is a \textbf{completely
%    inelastic collision}\footnote{also known as a \textbf{perfectly inelastic
%    collision}} where the objects stick together after the collision
%  \begin{itemize}
%  \item Momentum is conserved
%  \item Most of the kinetic energy is lost
%  \end{itemize}
%


\section{Elastic Collision}

An \textbf{elastic collision} is a special case in which both \emph{momentum}
and \emph{kinetic energy} are conserved. Here, we will derive the equations for
\emph{one-dimension} elastic collisions between \emph{two} objects.
%, as shown in Fig.~\ref{fig:1d}.
%\begin{figure}[ht]
%  \centering
%  \begin{tikzpicture}[scale=.8]
%    \tikzstyle{balloon1}=[ball color=red];
%    \tikzstyle{balloon2}=[ball color=blue];
%    \shade[balloon1](0,0) circle(1) node[white]{$m_1$};
%    \draw[very thick,->](1,0)--(2.5,0) node[midway,above]{$v_1$};
%    \shade[balloon2](7,0) circle(.6) node[white]{$m_2$};
%    \draw[very thick,->](6.4,0)--(5.2,0) node[midway,above]{$v_2$};
%  \end{tikzpicture}
%  \caption{Collision of two objects in one dimension.}
%  \label{fig:1d}
%\end{figure}

\textbf{Conservation of momentum} is the direct consequence of the third law of
motion. During the collision, the two objects exert an equal and opposite force
on each other. Therefore, over the time interval $\Delta t$ of the collision,
an equal and opposite impulse ($\bm J=\bm F\Delta t$) is applied to each
object. The resulting change of momentum of each obiect is therefore equal and
opposite, and the net change of momentum is zero. In this case, the equation for
conservation of momentum equation can be expressed as:
\begin{equation}
  m_1\bm v_1+m_2\bm v_2=m_1\bm v_1'+m_2\bm v_2'
  \label{eq:mom1}
\end{equation}
where $m_1$ and $m_2$ are the masses of objects 1 and 2, $v_1$ and $v_2$ are,
respectively, the initial velocities of the masses, and $v_1'$ and $v_2'$ are
their final velocities.

\textbf{Kinetic energy is conserved in an elastic collision} from the fact that
during the collision, conserative forces are present to transform kinetic
energy into  potential energy. This potential energy can be:
\begin{itemize}[itemsep=6pt]
\item\emph{elastic} (e.g.\ a spring that compresses when the two objects
  collide),
\item\emph{electric} (e.g.\ two charged particles moving towards each other),
\item\emph{gravitational} (e.g.\ gravitational sling shots for satellites and
  deep-space vehicles), or
\item\emph{magnetic} (e.g.\ north poles of two magnets are repelled from each
  other)
\end{itemize}
In most cases, the objects never touch each other. During the collision, these
conservative forces transform \emph{all} the potential energy back into kinetic
energy, and at the end of the interaction, those same conservative fores
release the energy back as kinetic energy. The result is that the total kinetic
energies of both objects before the collision is the same as the total
afterwards:
\begin{equation}
  \frac12m_1v_1^2 + \frac12m_2v_2^2 = \frac12m_1v_1'^2+\frac12 m_2v_2'^2
  \label{eq:K1}
\end{equation}
Note that unlike Eq.~\ref{eq:mom1}, which is a vector equation, Eq.~\ref{eq:K1}
is a scalar equation.
\begin{remark}
  One of the most common comments by students is that Eq.~\ref{eq:K1} is the
  \emph{same} as Eq.~\ref{eq:mom1} (i.e.\ they are always satisfied together).
  Clearly this is not the case! The momentum equation is \emph{linear} in $v$,
  while kinetic energy is \emph{quadratic} in $v$.
\end{remark}

\subsection{Derivation of One-Dimensional Elastic Collision Equations}
To derive the equation for one-dimension elastic collisions, we collect
all the $m_1$ and $m_2$ terms in Eq.~\ref{eq:mom1} together. We can then move
all terms to the left hand side so that the right-hand side will be $1$, as
shown below:
\begin{align}
  m_1(v_1-v_1')&=m_2(v_2'-v_2) \label{eq:mom2a}\\
  \frac{m_1(v_1-v_1')}{m_2(v_2'-v_2)}&=1 \label{eq:mom2}
\end{align}
Likewise, in the kinetic energy equation (Eq.\ \ref{eq:K1}), we can cancel all
the $\dfrac12$ factors from every term. Using the same approach
as the momentum equation, we collect the $m_1$ and $m_2$ terms together, and
then we factor the differences of squares terms\footnote{Remember the
difference of squares: $(a^2-b^2)=(a+b)(a-b)$ in case you have forgotten. I
sympathsize with you if you have indeed forgotten: there are so many things
to remember from so many different courses!} on both sides of the equation.
Then the  terms are rearranged:
\begin{align}
  \nonumber m_1(v_1^2-v_1'^2) &= m_2(v_2'^2-v_2^2)\\
  \nonumber m_1(v_1-v_1')(v_1+v_1') &= m_2(v_2'+v_2)(v_2'-v_2)\\
  \tikz[baseline]{
    \node[fill=blue!20] (n1) {
      $\left[\dfrac{m_1(v_1-v_1')}{m_2(v_2'-v_2)}\right]$
    }
  }(v_1+v_1') &= (v_2'+v_2)
  \label{eq:K2}
\end{align}
Note that the highlighted term in Eq.~\ref{eq:K2} is just the left-hand-side
of Eq.~\ref{eq:mom2}, which is equal to $1$. We can therefore cancel all mass
terms and express the final velocities $v_1'$ and $v_2'$ in terms of other
velocities:
\begin{align}
  \nonumber
  v_1+v_1' &= v_2'+v_2\\
  v_2'&=v_1+v_1'-v_2\quad\text{(solving for $v_2'$)}\label{eq:relvel1}\\
  v_1'&=v_2+v_2'-v_1\quad\text{(solving for $v_1'$)}\label{eq:relvel}
\end{align}
To solve for the final velocity of object 1, Eq.~\ref{eq:relvel1} is then
substituted back into Eq.~\ref{eq:mom2a}. This time, we collect all $v_1$,
$v_2$ and $v_1'$ terms together:
\begin{align}
  \nonumber
  m_1(v_1-v_1')&=m_2(v_1+v_1'-v_2-v_2)\\
  \nonumber
  m_1v_1-m_1v_1'&=m_2v_1+m_2v_1'-2m_2v_2\\
  (m_1-m_2)v_1+2m_2v_2&=(m_2+m_1)v_1'
  \label{eq:mom3}
\end{align}
By dividing every term in Eq.~\ref{eq:mom3} by the total mass ($m_2+m_1$), we 
obtain the final expression for the final velocity of object $1$:
\begin{equation}
  \boxed{
    v_1'=
    \left(\frac{m_1-m_2}{m_1+m_2}\right)v_1 +
    \left(\frac{2m_1}{m_1+m_2}\right)v_2
  }
  \label{eq:mom4}
\end{equation}
Likewise, Eq.~\ref{eq:relvel} can be substituted back into Eq.~\ref{eq:mom2a}
to obtain the expression for the final velocity of object 2:
\begin{equation}
  \boxed{
    v_2'=
    \left(\frac{2m_1}{m_1+m_2}\right)v_1 +
    \left(\frac{m_1-m_2}{m_1+m_2}\right)v_2
  }
  \label{eq:mom5}
\end{equation}
One special case is if the second object 2 is initially stationary, i.e.\
$v_2=0$. In such a case, the $v_2$ terms are dropped from
Eqs.~\ref{eq:mom4} and \ref{eq:mom5}, reducing the equations to:
\begin{align}
  v_1'&=\left(\frac{m_1-m_2}{m_1+m_2}\right)v_1\\
  v_2'&=\left(\frac{2m_1}{m_1+m_2}\right)v_1
  \label{eg:simple1}
\end{align}



\subsection{Special Cases for One-Dimensional Elastic Collisions}
There are a few different possibilities for outcomes:
\begin{itemize}[leftmargin=15pt]
\item\textbf{Two equal masses:} If the masses are equal, i.e.\ $m_1=m_2=m$,
  then Eq.~\ref{eg:simple1} becomes:
  \begin{align*}
    v_1'&=\left(\frac{m_1-m_2}{m_1+m_2}\right)v
    =\left(\frac{m-m}{m+m}\right)v=0\\
    v_2'&=\left(\frac{2m_1}{m_1+m_2}\right)v
    =\left(\frac{2m}{m+m}\right)v=v
  \end{align*}
  All the momentum and energy from object 1 is transferred to object 2. Object
  1 stops, while object 2 continues to move at $v$, as shown in
  Fig.~\ref{fig:same-mass}. This behaviour is most notably expressed in a
  \emph{Newton's cradle}.\footnote{Note that the Newton's cradle is \emph{not}
    an elastic collision. The fact that you can \emph{hear} the metal balls
    colliding means that energy has escaped the system as a sound wave.}
  \begin{figure}[ht]
    \centering
    \begin{tikzpicture}[scale=.8]
      %\tikzstyle{balloon1}=[ball color=red];
      %\tikzstyle{balloon2}=[ball color=blue];
      \shade[balloon1] (-1,0) circle (1) node[white]{$m$};
      \draw[vectors] (0,0)--(1.5,0) node[midway,above]{$v$};
      \shade[balloon2] (4,0) circle (1) node[white]{$m$};
      
      \shade[balloon1] (2,-2.5) circle (1) node[white]{$m$};
      \draw[dashed] (4,-2.5) circle(1);
      \draw[dashed] (5,-2.5)--(6,-2.5);
      \shade[balloon2] (7,-2.5) circle (1) node[white]{$m$};
      \draw[vectors] (8,-2.5)--(9.5,-2.5) node[midway,above]{$v$};
    \end{tikzpicture}
    \caption{When a moving object collides with a stationary object of equal
      mass, all the momentum and energy are transferred.}
    \label{fig:same-mass}
  \end{figure}
\item\textbf{Large Object Colliding With Small Object:} In this second case,
  object 1 is much more massive than the second object, i.e.\ $m_1\gg m_2$. We
  can then effectively ``ignore'' the $m_2$ terms in Eq.~\ref{eg:simple1}. The
  equations then become:
  \begin{align*}
    v_1'&=\left(\frac{m_1-m_2}{m_1+m_2}\right)v=\left(\frac{m_1}{m_1}\right)v
    =v\\
    v_2'&=\left(\frac{2m_1}{m_1+m_2}\right)v=\left(\frac{2m_1}{m_1}\right)v=2v
  \end{align*}
  Object 1 continues to move at nearly its initial speed $v$, while object 2
  picks up \emph{twice} the speed of object 1, as shown in
  Fig~\ref{fig:big-small}. This kind of collision is often used in
  gravitational slingshots, which forms the basis for deep-space exploration
  probes that need to travel through the solar system.
  \begin{figure}[ht]
    \centering
    \begin{tikzpicture}[scale=.8]
      %\tikzstyle{balloon1}=[ball color=red];
      %\tikzstyle{balloon2}=[ball color=blue];
      \shade[balloon1] circle (1.5) node[white]{$m_1$};
      \draw[vectors] (1.5,0)--(3,0) node[midway,above]{$v$};
      \shade[balloon2] (7,0) circle (.6) node[white]{$m_2$};
      
      \shade[balloon1](5.5,-2.5) circle(1.5) node[white]{$m_1$};
      \draw[vectors] (7,-2.5)--(8.5,-2.5) node[midway,above]{$v$};
      \draw[dashed] (7,-2.5) circle (.6);
      \draw[dashed] (7.6,-2.5)--(9.4,-2.5);
      \shade[balloon2] (10,-2.5) circle (.6) node[white]{$m_2$};
      \draw[vectors] (10.6,-2.5)--(13.6,-2.5) node[midway,above]{$2v$};
    \end{tikzpicture}
    \caption{When a large object collides with a stationary small object, 
    it does not slow down, but the smaller object gains twice the speed.}
    \label{fig:big-small}
  \end{figure}
\item\textbf{Small Object Colliding With Large Object:} In this final case,
  object 1 has a much smaller mass than object 2, i.e.\ $m_1\ll m_2$. Like the
  previous case, we can effectively ignore the smaller mass, which is $m_1$ in
  this case. Then, Eq.~\ref{eg:simple1} reduces to:
  \begin{align*}
    v_1'&=\left(\frac{m_1-m_2}{m_1+m_2}\right)v=\left(\frac{-m_2}{m_2}\right)v
    =-v\\
    v_2'&=\left(\frac{2m_1}{m_1+m_2}\right)v=\left(\frac{2m_1}{m_2}\right)v=0
  \end{align*}
  Object 2 continues to be stationary, while object 1 bounces back at its
  original speed, as shown in Fig~\ref{fig:big-small}.
  \begin{figure}[ht]
    \centering
    \begin{tikzpicture}[scale=.8]
      %\tikzstyle{balloon1}=[ball color=red];
      %\tikzstyle{balloon2}=[ball color=blue];
      \shade[balloon1] circle (.5) node[white]{$m_1$};
      \draw[vectors] (.5,0)--(2,0) node[right]{$v$};
      \shade[balloon2] (7,0) circle (1.5) node[white]{$m_2$};
    
      \shade[balloon1] (4,-3.3) circle(.5) node[white]{$m_1$};
      \draw[vectors] (3.5,-3.3)--(2,-3.3) node[left]{$v$};
      \shade[balloon2] (7,-3.3) circle (1.5) node[white]{$m_2$};
    \end{tikzpicture}
    \caption{When a small objects collides with a stationary large object, it
      bounces back with the same speed, while the larger object remains
      stationary.}
  \end{figure}
\end{itemize}




\section{Example Problems}

\begin{example}
  A \SI{.0520}{\kilo\gram} golf ball is moving velocity of
  \SI{2.10}{\metre\per\second} when it collides, head on, with a stationary
  \SI{.155}{\kilo\gram} billiard  ball. If the golf ball rolls directly
  backwards with a velocity of \SI{-1.04}{\metre\per\second}, is the collision
  elastic?
\end{example}
%  \vspace{.4in}\textcolor{gray}{
%    To solve this type of problem, using the conservation of momentum to find
%    the velocities of the golf ball and the billiard ball. Then, sum the total
%    kinetic energies of both balls, and compare that to the total kinetic
%    energy of the balls before the collision.
%  }


\begin{example}
  A car (\SI{1000}{\kilo\gram}) travels at a speed of
  \SI{20}{\metre\per\second} towards a stationary truck
  (\SI{3000}{\kilo\gram}). The car rear ends the truck
  elastically\footnote{Reality check: For this example we are not concerned
  with the \emph{implausibility} of such a collision; we just want to figure
  out what happens \emph{if} the collision actually occurs.}. What are the
  velocities of the car and truck after the collision?
\end{example}
%
%
%
%\begin{frame}{Solving the Example Problem}
%  In order to solve this question we need \emph{both} the conservation of
%  (kinetic) energy and the conservation of momentum. Here, $A$ is the car, $B$
%  is the truck.
%
%  \vspace{-.2in}{\large
%    \begin{align*}
%      m_Av_A+m_Bv_B&=m_Av_A'+m_Bv_B'\\
%      \half m_Av_A^2+\half m_Bv_B^2&=\half m_Av_A'^2+\half m_Bv_B'^2
%    \end{align*}
%  }
%
%  %For generalization, we won't assume whether any object is stationary or in
%  %motion.
%
%
%
%
%\begin{frame}{Solving the Example Problem}
%  \begin{columns}[T]
%    \column{.45\textwidth}
%    \underline{\textbf{Momentum Equation}}
%
%    \vspace{-.1in}
%    \begin{displaymath}
%      m_Av_A+m_Bv_B=m_Av_A'+m_Bv_B'
%    \end{displaymath}
%%    We can eliminate $v_B$, since the truck wasn't moving:
%%    \begin{displaymath}
%%      m_Av_A=m_Av_A'+m_Bv_B'
%%    \end{displaymath}
%
%    \vspace{.02in}Move all $m_A$ terms to the left, and $m_B$ terms to the
%    right :
%    \begin{equation}
%      \boxed{m_A(v_A-v_A')=m_B(v_B'-v_B)}
%    \end{equation}
%
%    \column{.55\textwidth}
%    \uncover<2>{
%      \underline{\textbf{Kinetic Energy Equation}}
%      \begin{displaymath}
%        \half m_Av_A^2+\half m_Bv_B^2=\half m_Av_A'^2+\half m_Bv_B'^2
%      \end{displaymath}
%      Multiply every term by 2, and then move $m_A$ terms to the left, and
%      $m_B$ terms to the right:
%      \begin{equation}
%        \boxed{m_A(v_A^2-v_A'^2)=m_B(v_B'^2-v_B^2)}
%      \end{equation}
%    }
%  \end{columns}
%
%
%
%
%\begin{frame}{Solving the Example Problem}
%  Dividing (2) by (1), we get:
%  \begin{displaymath}
%    \frac{(2)}{(1)}=\frac{m_A(v_A^2-v_A'^2)}{m_A(v_A-v_A')}
%    =\frac{m_B(v_B'^2-v_B^2)}{m_B(v_B'-v_B)}
%  \end{displaymath}
%  We can cancel out the $m_A$ and $m_B$ terms on both sides, then factor the
%  difference of two squares on top:
%  \begin{displaymath}
%    \frac{(v_A+v_A')(v_A-v_A')}{(v_A-v_A')}=
%    \frac{(v_B'+v_B)(v_B'-v_B)}{(v_B'-v_B)}
%  \end{displaymath}
%  Now we get an equation relating the velocities that can be substituted back
%  to (1) and (2):
%  \begin{displaymath}
%    v_A + v_A'= v_B + v_B'
%  \end{displaymath}
%
%
%
%
%\begin{frame}{Solving the Example Problem}
%  We arrive at these two equations for final velocities that applies to
%  \emph{all} one-dimensional elastic collisions:
%  
%  \eq{-.2in}{
%    \boxed{v_A'=\frac{m_A-m_B}{m_A+m_B}v_A+\frac{2m_B}{m_A+m_B}v_B}\;\;
%    \boxed{v_B'=\frac{m_B-m_A}{m_A+m_B}v_B+\frac{2m_A}{m_A+m_B}v_A}
%  }
%
%  %\vspace{-.1in}These equations apply to \emph{all} one-dimensional elastic collisions.
%  Substituting values for $m_A$, $m_B$ and $v_A$ in this example, we get:
%  \begin{align*}
%    v_A'&=\frac{m_A-m_B}{m_A+m_B}v_A=\frac{(1000-3000)}{(1000+3000)}\times 20
%    = \boxed{\SI{-10}{\metre\per\second}}\\
%    v_B'&=\frac{2m_A}{m_A+m_B}v_A=\frac{(2\times 1000)}{(1000+3000)}\times 20
%    = \boxed{\SI{10}{\metre\per\second}}
%  \end{align*}
%
%
%
%
%\begin{frame}[t]{This example tells us much more!}
%  What happens when the two objects have the same mass, i.e.\ $m_A=m_B=m$?
%
%  \vspace{-.2in}{\large
%    \begin{align*}
%      v_A'&=\cancel{\frac{m_A-m_B}{m_A+m_B}v_A}+\frac{2m_B}{m_A+m_B}v_B\\
%      v_B'&=\cancel{\frac{m_B-m_A}{m_A+m_B}v_B}+\frac{2m_A}{m_A+m_B}v_A
%    \end{align*}
%    %\boxed{v_A'=\frac{m_A-m_B}{m_A+m_B}v_A}\quad
%    %\boxed{v_B'=\frac{2m_A}{m_A+m_B}v_A}
%  }
%  
%  \vspace{-.15in}
%  \begin{columns}
%    \column{.5\textwidth}
%    \begin{align*}
%      v_A'&=\frac{2m_B}{m_A+m_B}v_A=\frac{2m}{m+m}v_A\\
%      v_A'&=v_B
%    \end{align*}
%
%    \column{.5\textwidth}
%    \begin{align*}
%      v_B'&=\frac{2m_A}{m_A+m_B}v_A=\frac{2m}{m+m}v_A\\
%      v_B'&=v_A
%    \end{align*}
%  \end{columns}
%
%  \vspace{.1in}If both masses are the same, then they exchange their
%  velocities!
%
%
%
%
%\begin{frame}[t]{This example tells us much more!}
%  When $A$ is much more massive than $B$, i.e.\ $m_A\gg m_B$, and assuming that
%  $B$ is initially at rest, we can effectively ``ignore'' $m_B$:
%  
%  \eq{-.1in}{
%    \boxed{v_A'=\frac{m_A-m_B}{m_A+m_B}v_A}\quad
%    \boxed{v_B'=\frac{2m_A}{m_A+m_B}v_A}
%  }  
%  \vspace{-.1in}
%  \begin{columns}
%    \column{.5\textwidth}
%    \begin{align*}
%      v_A'&=\frac{m_A-m_B}{m_A+m_B}v_A\\
%      v_A'&\approx\frac{m_A}{m_A}v_A\\
%      v_A'&\approx v_A
%    \end{align*}
%
%    \column{.5\textwidth}
%    \begin{align*}
%      v_B'&=\frac{2m_A}{m_A+m_B}v_A\\
%      v_B'&\approx\frac{2m_A}{m_A}v_A\\
%      v_B'&\approx 2v_A
%    \end{align*}
%  \end{columns}
%  
%  \vspace{.1in}$A$ continues to move like nothing happened, but object $B$ is
%  pushed to move at an twice the speed of $A$.
%
%
%
%
%\begin{frame}[t]{This example tells us much more!}
%  Similarly, when $B$ is more massive, i.e.\ $m_A\ll m_B$, and $B$ is initially
%  at rest, then we `ignore'' $m_A$ instead:
%
%  \eq{-.1in}{
%    \boxed{v_A'=\frac{m_A-m_B}{m_A+m_B}v_A}\quad
%    \boxed{v_B'=\frac{2m_A}{m_A+m_B}v_A}
%  }
%  \vspace{-.1in}
%  \begin{columns}
%    \column{.5\textwidth}
%    \begin{align*}
%      v_A'&=\frac{m_A-m_B}{m_A+m_B}v_A\\
%      v_A'&\approx\frac{-m_B}{m_B}v_A\\
%      v_A'&\approx-v_A
%    \end{align*}
%    \column{.5\textwidth}
%    \begin{align*}
%      v_B'&=\frac{2m_A}{m_A+m_B}v_A\\
%      v_B'&\approx\frac 0{m_B}v_A\\
%      v_B'&\approx 0
%    \end{align*}
%  \end{columns}
%  
%  \vspace{.1in}$A$ bounces off $B$, and travels in the opposite direction
%  with the same speed.
%
%
%
%
%\section{Practice Problems}

\begin{example}
  A forensic expert needs to find the velocity of a bullet fired from a gun in
  order to predict the trajectory of a bullet. He fires a \SI{5.5}{\gram}
  bullet into a ballistic pendulum with a mass of \SI{1.75}{\kilo\gram}. The
  pendulum swing to a height of \SI{12.5}{\centi\metre} above its rest position
  before dropping back down. What is the velocity of the bullet just before it
  hit and became embedded in the pendulum bob?
  \begin{center}
    \pic{.4}{momentum/graphics/bullet}
  \end{center}
\end{example}



\begin{example}
  A block of wood with a mass of \SI{.500}{\kilo\gram} slides across the floor
  towards a \SI{3.50}{\kilo\gram} block of wood. Just before the collision, the
  small block is travelling at \SI{3.15}{\metre\per\second}. Because some nails
  are sticking out of the blocks, the blocks stick together when they collide.
  Scratch marks on the floor show that they slid \SI{2.63}{\centi\metre} before
  coming to a stop. What is the coefficient of friction between the wooden
  blocks and the floor?
\end{example}
%  \vspace{.4in}\textcolor{gray}{We won't use kinematic or dynamic equations to
%    solve this problem. Instead, we'll use conservation momentum equation and
%    the work kinetic energy theorem.}
%

