\documentclass[12pt,compress,aspectratio=169]{beamer}
\input{../mybeamer}

\title{Class 15: Power Generation}
\subtitle{Unit 5: Electricity and Magnetism}
\input{../term}
\input{../mycommands}


\begin{document}

\begin{frame}
  \titlepage
\end{frame}




\begin{frame}{Direct and Alternating Current}
  In a \textbf{direct current} (\textbf{DC}), the flow of charges is always in
  the same direction, but the current itself does not have to be constant in
  time.
  \begin{center}
    \begin{tikzpicture}[scale=.6]
      \draw[functions] (0,3)--(4,3);
      \draw[axes] (0,0)--(4,0) node[right]{$t$};
      \draw[axes] (0,0)--(0,4) node[above]{$I$};
    \end{tikzpicture}
    \begin{tikzpicture}[scale=.6]
      \draw[smooth,samples=20,domain=0:3.5,functions]
      plot(\x,{3*(exp(-.8*\x))});
      \draw[axes] (0,0)--(4,0) node[right]{$t$};
      \draw[axes] (0,0)--(0,4) node[above]{$I$};
    \end{tikzpicture}
    \begin{tikzpicture}[scale=.6]
      \draw[smooth,samples=20,domain=0:3.5,functions] plot(\x,{sin(120*\x)+2});
      \draw[axes] (0,0)--(4,0) node[right]{$t$};
      \draw[axes] (0,0)--(0,4) node[above]{$I$};
    \end{tikzpicture}
  \end{center}
\end{frame}



\begin{frame}{Direct and Alternating Current}
  In an \textbf{alternating current} (AC), the flow of charges changes
  direction, usually as a sinusoidal function of time.
  \begin{center}
    \begin{tikzpicture}[scale=.6]
      \draw[axes] (0,0)--(5,0) node[right]{$t$};
      \draw[axes] (0,-2)--(0,2) node[above]{$I$};
      \draw[smooth,samples=50,domain=0:4,functions] plot(\x,{1.5*sin(150*\x)});
    \end{tikzpicture}
    \begin{tikzpicture}[scale=.6]
      \draw[axes] (0,0)--(5,0) node[right]{$t$};
      \draw[axes] (0,-2)--(0,2) node[above]{$V$};
      \draw[smooth,samples=40,domain=0:4,functions] plot(\x,{1.5*sin(150*\x)});
    \end{tikzpicture}
  \end{center}
  The power outlet in North America are all AC, with a root-mean-square voltage
  of \SI{120}\volt, and a frequency of \SI{60}\hertz. AC current are important
  in the power generation.
\end{frame}



\begin{frame}{AC Generator}
  \begin{columns}
    \column{.5\textwidth}
    \pic1{graphics/generator}

    \column{.5\textwidth}
    \begin{itemize}
    \item When the coil is turning, the charge carriers inside the coil move
      relative to the magnetic field
    \item The charges experience a magnetic force, creating an electromotive
      force
%    \item When the coil is turning, the orientation between the coil and the
%      magnetic field changes
%    \item Because the \emph{magnetic flux} is changing, a potential difference
%      (voltage) is created in the coil itself
    \item When connected to a resistor, it acts as an AC power source
    \end{itemize}
  \end{columns}
\end{frame}



\section{Transformers}

\begin{frame}[t]{How Transformers Work}
  \begin{center}
    \pic{.8}{graphics/transformers}
  \end{center}
  \begin{itemize}
  \item Consists of two coils (\emph{primary} and \emph{secondary})
  \item Think of each coil as solenoids
  \item The coils are not electrically connected (i.e.\ no wires connecting the
    two circuits)
  \item Both coils may be wrapped around a square soft-iron core (shown above)
  \end{itemize}
\end{frame}



\begin{frame}[t]{How Transformers Work}
  \begin{center}
    \pic{.8}{graphics/transformers}
  \end{center}
  \begin{itemize}
  \item The primary coil is connected to a AC power source
  \item When current flows through the primary circuit, a magnetic field is
    generated in the primary coil
  \item Because the current alternates, the magnetic field inside the primary
    coil is constantly changing in strength and direction
  \end{itemize}
\end{frame}



\begin{frame}[t]{How Transformers Work}
  \begin{center}
    \pic{.8}{graphics/transformers}
  \end{center}
  \begin{itemize}
  \item A magnetic field is also generated inside the entire soft-iron core
  \item The magnetic field in the iron core oscillates with the same
    frequency as the primary coil
  \item A current is induced in the secondary coil because the magnetic field
    through the coil (i.e.\ the iron core) is changing
  \end{itemize}
\end{frame}



\begin{frame}[t]{How Transformers Work}
  \begin{center}
    \pic{.8}{graphics/transformers}
  \end{center}
  \begin{itemize}
  \item An alternating voltage is generated in the secondary coil with the same
    frequency as the primary coil
  \item The voltage in the secondary coil depends on the
    ratio of windings in both coils
    \begin{itemize}
    \item \textbf{(a)} is called a \emph{step-down} transformer, while
    \item \textbf{(b)} is called a \emph{step-up} transformer
    \end{itemize}
  \end{itemize}
\end{frame}



\begin{frame}{Transformers in Circuit Diagrams}
  The circuit diagram for a transformer consists of two coils. The vertical
  line between the coil indicate whether this transformer has a soft-iron core:

  \vspace{.2in}
  \begin{columns}
    \column{.5\textwidth}
    \centering
    \begin{tikzpicture}[scale=2,thick]
      \draw (0,0) node[transformer] (T) {}
      (T.A2) to [short,o-] ++ (-1,0) coordinate (aux)
      to [sI] (aux |- T.A1)
      to [short,-o] (T.A1);
    \end{tikzpicture}

    Air core

    \column{.5\textwidth}
    \centering
    \begin{tikzpicture}[scale=2,thick]
      \draw (0,0) node[transformer] (T) {}
      (T.A2) to [short,o-] ++ (-1,0) coordinate (aux)
      to [sI] (aux |- T.A1)
      to [short,-o] (T.A1);
      \draw[thick] ( .03,.3)--( .03,-.3);
      \draw[thick] (-.03,.3)--(-.03,-.3);
    \end{tikzpicture}

    Iron core
  \end{columns}
\end{frame}



\begin{frame}{Transformers}
  Assuming that the electric potential energy (and therefore power) in the
  primary coil is transmitted to the secondary coil, and using the equation for
  power in a circuit:

  \eq{-.1in}{
    \Delta E_p=\Delta E_s
    \quad\to\quad
    P_p=P_s
    \quad\to\quad
    \boxed{I_pV_p = I_sV_s}
  }
\end{frame}



\begin{frame}{Transformers}
  Voltage, current in the two coils (primary and secondary) are related to the
  number of windings:
  
  \eq{-.1in}{
    \boxed{\frac{V_p}{V_s}=\frac{I_s}{I_p}=\frac{N_p}{N_s}}
  }
  \begin{center}
    \begin{tabular}{l|c|c}
      \rowcolor{pink}
      \textbf{Quantity} & \textbf{Symbol} & \textbf{SI Unit} \\ \hline
      Number of windings in primary \& secondary coils & $N_p$, $N_s$ & N/A \\
      Current in primary \& secondary coils & $I_p$, $I_s$ & \si\ampere \\
      Voltage across primary \& secondary coils & $V_p$, $V_s$ & \si\volt
    \end{tabular}
  \end{center}
  These two equations assume that the transformer is \SI{100}{\percent}
  efficient. In practice, the efficiency of transformers is
  $\approx\SI{90}\percent$.
\end{frame}



%\begin{frame}{Transformers}
%  \begin{itemize}
%  \item Because only a \emph{changing} magnetic field can induce current, the
%    transformer can only be used with alternating current
%  \end{itemize}
%\end{frame}



\begin{frame}{Power Grid}
  Between where power is generated, and your electrical outlet, AC power
  usually goes through a number of step-up and step-down transformers. This is
  usually referred to as a \textbf{power grid}.
  \begin{center}
    \pic{.7}{graphics/powerline}
  \end{center}
\end{frame}



\section{Power Generation}

\begin{frame}{Power Generation}
  How do we generate the high-voltage AC in the first place then?
\end{frame}



\subsection{Fossil Fuels}

\begin{frame}{Fossil Fuels: The Dominant Fuel Source}
  \begin{center}
    \pic{.5}{graphics/gasoline-engine}
  \end{center}
  \begin{itemize}
  \item Example: crude oil, coal,natural gas
  \item Crude oil is \emph{refined} be separating the different hydrocarbon
    fuels (gasoline, kerosene, diesel etc)
  \item Combustion of the fuel releases thermal energy which turns a shaft
    connected to an AC generator to generates electricity
  \end{itemize}
\end{frame}



\begin{frame}{Fossil Fuels: The Dominant Fuel Source}
  \begin{columns}
    \column{.7\textwidth}
    Advantages:
    \begin{itemize}
    \item Relatively inexpensive to process
    \end{itemize}
    Disadvantages:
    \begin{itemize}
    \item Releases harmful by-products
    \item Source of greenhouse gases, especially carbon monoxide and carbon
      dioxide (effects last for decades)
    \item Contribute to global warming and climate change
    \end{itemize}
    
    \column{.29\textwidth}
    \pic1{graphics/rig}
  \end{columns}
\end{frame}



\begin{frame}{Fossil Fuels: A Crisis}
  \begin{center}
    \pic{.5}{graphics/Oil-Drilling-and-7-Year-Low-Oil-Prices}
  \end{center}
  \begin{itemize}
  \item Oil ``production'' (i.e.\ extraction of crude oil) has peaked
  \item Largest reservoir of crude oil located in geo-politically unstable
    areas (Middle East, Central America, Africa)
  \end{itemize}
\end{frame}



\subsection{Wind}

\begin{frame}{Wind Power}
  \begin{itemize}
  \item Have been used for centuries in windy areas around the world
  \item Turbine blades are highly optimized like wings on airplanes to generate
    as much lift force as possible
  \item The turbine turns an AC generator (mounted on the wind urbine itself)
    to create electricity
  \end{itemize}
  \begin{center}
    \pic{.63}{graphics/windfarm}
  \end{center}
\end{frame}



\begin{frame}{Wind Power}
  Advantages:
  \begin{itemize}
  \item Zero emissions of greenhouse gases
  \item Limited potential for accidents
  \end{itemize}

  Disadvantages:
  \begin{itemize}
  \item Expensive to build
  \item Low energy density--requires a large site for a large city
  \item Need sustained wind of sufficient strength
  \item Unpredictable
  \item Kill birds?
  \end{itemize}
\end{frame}



\begin{frame}{Wind Power: A Local Example}
  \begin{center}
    \pic{.5}{graphics/CrombieCruise-108}
  \end{center}
  The wind turbine at the CNE can power about 300 homes on windy days (when it
  is working)
\end{frame}



\subsection{Hydroelectric}

\begin{frame}{Hydroelectric Power}
  The Hoover Dam in the United States was the first large-scale hydroelectric
  project in the world
  \begin{center}
    \pic{.45}{graphics/hooverdam}
  \end{center}
  It is a popular power generating method in Canada, United States and in Europe
\end{frame}



\begin{frame}{Hydroelectric Power}
  \begin{center}
    \pic{.65}{graphics/Hydroelectric-generating-station1}
  \end{center}
  \begin{itemize}
  \item\vspace{-.15in}At the bottom of the reservoir, water pressure is high
  \item Water is pushed through the \textbf{penstock}, gaining kinetic energy
  \item The high-speed water pushes the turbine, which then generates a
    high-voltage AC
  \end{itemize}
\end{frame}



\begin{frame}{Hydroelectric Power}
  Advantages:
  \begin{itemize}
  \item Efficient
  \item Reliable
  \item Carbon neutral: operation does not generate greenhouse gases
  \item Limited environmental impact after the operation of the facility begins
  \item Well-developed technologies and infrastructure
  \end{itemize}
  Disadvantages:
  \begin{itemize}
  \item Large ecological damage during construction
  \end{itemize}
  %What other advantages and/or disadvantages can you think of?
\end{frame}



\subsection{Geothermal}

\begin{frame}{Geothermal Power}
  \begin{columns}[T]
    \column{.45\textwidth}
    \pic1{graphics/Geothermal}
    
    \column{.55\textwidth}
    \begin{itemize}
    \item Hot steam (high temperature and high pressure) is pumped from
      underground streams
    \item The high pressure steam is used to push a turbine, turning kinetic
      energy into electricity
    \item The cooled water is then pumped back into the steam
    \item Used widely in Iceland
    \end{itemize}
  \end{columns}
\end{frame}


\begin{frame}{Geothermal Power}
  \begin{columns}
    \column{.45\textwidth}
    \pic1{graphics/Geothermal}
    
    \column{.55\textwidth}
    Advantages:
    \begin{itemize}
    \item Carbon neutral
    \item Unlimited supply (unless the whole core of the earth cools down)
    \end{itemize}

    Disadvantages:
    \begin{itemize}
    \item Limited location
    \item Naturally occurring, dissolved corrosive salts can cause problems for
      equipment
    \item Potential for toxic gas discharge (e.g.\ hydrogen sulphide)
    \end{itemize}
  \end{columns}
\end{frame}



\begin{frame}{Geothermal Power}
  On a smaller scale, there are also home-based geothermal system for heating
  and cooling. They take advantage of the fact that the temperature below
  ground is generally held to a consistent temperature throughout the year.
  \begin{center}
    \pic{.45}{graphics/heat-cool-1024x773}
  \end{center}
\end{frame}



\subsection{Solar}

\begin{frame}{Solar Power: No Generators Needed}
  \begin{center}
    \pic{.6}{graphics/PV-cell}
  \end{center}
  \begin{itemize}
  \item\vspace{-.15in}Solar power is based on \textbf{photoelectric effect},
    where light at specific frequencies knock off electrons from metals
  \item A \textbf{photo-voltaic cell} combines layers of metal that have too
    many electrons with layers that lacks them.
  \item When light hits the metal, a current is generated
  \end{itemize}
\end{frame}



\begin{frame}{Solar Power}
  \begin{center}
    \pic{.4}{graphics/Solar-Farms-Splash1}
  \end{center}
  \vspace{-.1in}Advantages:
  \begin{itemize}
  \item Portable---it requires no infrastructure (e.g.\ pipelines, transformers)
  \item Zero emission---it generates no greenhouse gases
  \end{itemize}
  Disadvantages:
  \begin{itemize}
  \item Expensive
  \item Low efficiency: ``Terrestrial grade'' solar cells are only
    $\approx\SI{18}{\percent}$ efficient
  \item Potentially toxic chemicals to make 
  \item Significant land area required (what's the work around?)
  \end{itemize}
\end{frame}



\begin{frame}{Solar Power}
  Practicality aside, you can even powered a car with solar power.
  \begin{columns}[T]
    \column{.4\textwidth}
    \pic1{graphics/SOLARCAR}
    
    \column{.6\textwidth}
    \begin{itemize}
    \item At peak power, the on-board solar panels generate 
      $\approx\SI{1200}\watt$ of power (the power consumed by a hair dryer)
    \item Most solar cars have very efficient electric motors, and very little
      aerodynamic loses
    \item Well-designed solar cars can run on solar power alone (no battery
      assistance) at \SI{80}{\kilo\metre\per\hour}
    \end{itemize}
  \end{columns}
  
  The Honda solar car in the photo cost several million dollars to
  develop and build.
\end{frame}



\subsection{Nuclear}

\begin{frame}{Nuclear Power}
  Energy is released by nuclear fission (e.g.\ uranium, plutonium)
  \begin{columns}
    \column{.65\textwidth}
    \begin{itemize}
    \item Example: Canada's own CANDU reactors
    \item Advantages:
      \begin{itemize}
      \item Fuels are relatively easy to transport
      \item Radiation from \emph{normal} operation of a power plant less
        than a coal power plan
      \end{itemize}
    \item Disadvantages:
      \begin{itemize}
      \item Nuclear waste: fission bi-products are also
        radioactive; require careful storage
      \item Accident could happen (regardless how small the possibility is)
      \item Very expensive to build (rate of return is lower than other power
        sources)
      \end{itemize}
    \end{itemize}

    \column{.35\textwidth}
    \centering
    \pic1{graphics/candu}\\
    {\scriptsize CANDU reactor}\par
  \end{columns}
\end{frame}

\end{document}
