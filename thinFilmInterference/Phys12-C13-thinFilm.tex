\chapter{Thin-Film Interference} % \& Electromagnetic Waves}



%\section{Thin-Film Interference}
\textbf{Thin-film interference} occurs when light reflected/refracted at
the upper \& lower boundaries of a thin film of an indexed
material\footnote{An ``indexed material'' means a material that has an index
of refraction $n>1$} interfere with one another
\begin{itemize}
\item The film is a few wavelengths in thickness
\item The thickness determines whether the interference is constructive or
  destructive
\item When white light is incident on the film, some colours are enhanced
  while others are reduced
\end{itemize}




%\section{Thin-Film Interference}
%  \begin{center}
%    \pic{.4}{graphics/soap-bubble}\hspace{.01in}
%    \pic{.277}{graphics/oil-film}\hspace{.01in}
%    \pic{.2265}{graphics/camera-lens}
%  \end{center}
%  Examples:
%  \begin{itemize}
%  \item Soap bubbles
%  \item Oil films on water
%  \item Anti-reflective coatings on glasses and camera lenses
%  \end{itemize}



\section{Path Difference}

To study the thin-film interference problem, we begin with a set up shown in
Fig.~\ref{fig:thin-film1}.
\begin{figure}[ht]
  \centering
  \begin{tikzpicture}[scale=.9]
    \fill[cyan!20] rectangle (6,3);
    \draw (0,0)--(6,0);
    \draw (0,3)--(6,3);
    \draw[<->] (5.5,0)--(5.5,3) node[midway,fill=cyan!20]{$t$};
    \draw[ultra thick] (1.6,5)--(2,3);
    \draw[vectors] (2,3)--(2.4,5);
    \draw[vectors,dash dot] (2,3)--(2.1,0)--(2.2,3)--(2.6,5);
    \node at (.25,1.5){$n$};
    \node at (.25,3.25){$n_u$};
    \node at (.25,-.25){$n_b$};
  \end{tikzpicture}
  \caption{Basic configuration for the thin-film interference problem}
  \label{fig:thin-film1}
\end{figure}
A thin film of index $n$ is placed between two other indexed material ($n_u$
above the film; $n_b$ below the film). The thickness of the film $t$ is assumed
to be in the other of the wavelength of the light.

When light shines from above the film with a frequency $f$ and wavelength
$\lambda$, the light would reflect and refract at the top interface, as denoted
by the solid line. At the same time, some light is also refracted into the
film, as denoted by the dotted line. As light enters the thin film from above,
frequency remains the same, but since the index of refraction changes from
$n_u$ to $n$, the wavelength changes from $\lambda$ to $\lambda'$. The
wavelenghts are related by the equation:
%$c=f\lambda$. Therefore:
\begin{equation*}
  \lambda'=\frac{n_u}n\lambda
\end{equation*}
For the analysis in this chapter, we assume that the medium above the film is
either air or vacuum, we can use $n_u=1$ for simplicity, and the above equation
reduces to:
\begin{equation}
  \lambda'=\frac\lambda{n}
\end{equation}

Inside the film, the refracted light continues to travel towards the bottom,
where it hits the lower interface. Again, the light would reflected back into
the film, and also refracted into the third medium (we are not concerned about
this beam at the moment). The reflected light once again hits the top
interface, and refracts back in to the first medium.

For small incident angles (i.e.\ $\theta_1\approx 0$), the path difference
$\Gamma$ between the solid and dotted lines is approximately twice the
thickness $t$ of the film:
\begin{equation*}
  \Gamma\approx2t
\end{equation*}
As was in the last chapter, $\Gamma$ determines the conditions for
constructive and destructive interference.




\section{Soap Bubble}
Light travels through air and strikes a soap film (with $n_\text{soap}$). On
either side of the soap film is air, as shown in Figure~\ref{fig:soap-bubble}.
\begin{figure}[ht]
  \centering
  \begin{tikzpicture}[scale=.8]
    \fill[cyan!20] rectangle (6,3);
    \draw (0,0)--(6,0);
    \draw (0,3)--(6,3);
    \draw[<->] (5.5,0)--(5.5,3) node[midway,fill=cyan!20]{$t$};
    \draw[vectors,red] (1.6,5)--(2,3) node[left]{$\pi$}--(2.4,5);
    \draw[vectors,teal] (1.55,5)--(1.95,3)--(2.1,0)--(2.2,3)--(2.6,5);
    \node at (-.3,1.5){$n$};
    \node at (-.3,3.4){$n_u=1$};
    \node at (-.3,-.4){$n_b=1$};
  \end{tikzpicture}
  \caption{Thin-film interference of the soap bubble}
  \label{fig:soap-bubble}
\end{figure}
At the upper interface, reflected light has a \ang{180} ($\pi$ radian)
phase shift, as $n_\text{air}<n_\text{soap}$, i.e.\ light reflects from a
fast to slow medium. At the lower interface: the reflected wave has no phase
shift (slow to fast medium)

Because of the single phase shift at a top interface, a
\textbf{constructive maximum} occurs if path difference ($\Gamma=2t$) is a
half-number ($m-\frac12$) multiple of wavelength ($\lambda$):
\begin{equation}
  \Gamma=2t=\left(m-\frac12\right)\lambda'
  \quad\rightarrow\quad
  \boxed{
    2n_\text{soap}t=\left(m-\frac12\right)\lambda
  }
\end{equation}
while \textbf{destructive minimum} occurs if $\Gamma$ is a whole-number
multiple of wavelength:
\begin{equation}
  \boxed{
    2n_\text{soap}t=m\lambda
  }
\end{equation}
where $m=1,2,3\ldots$
%
%
%
%
%\section{Soap Bubble}
%  \begin{center}
%    \pic{.45}{graphics/soap-bubble}
%  \end{center}

Because the condition for interference depends on the incident wavelength, when
incident light is a white-light (broadband), some colours experience
constructive interference, while other colours, destructive interference.
The colour pattern comes from the variations of thickness of the film, which is
not constant, and also changing with time




\section{Oil Film on Water}
\begin{figure}[ht]
  \centering
  \begin{tikzpicture}[scale=.7]
    \fill[cyan!20] rectangle (6,3);
    \draw (0,0)--(6,0);
    \draw (0,3)--(6,3);
    \draw[<->] (5.5,0)--(5.5,3) node[midway,fill=cyan!20]{$t$};
    \draw[vectors,red] (1.6,5)--(2,3) node[left]{$\pi$}--(2.4,5);
    \draw[vectors,teal] (1.55,5)--(1.95,3)--(2.1,0)--(2.2,3)--(2.6,5);
    \node at (-.3,1.5){$n_\text{oil}$};
    \node at (-.3,3.4){$n_\text{air}=1$};
    \node at (-.3,-.4){$n_\text{water}=1.33$};
  \end{tikzpicture}
\end{figure}
Light travels through air and strikes a oil film (with $n_\text{oil}$). Below
the oil film is water, which has a lower index of refraction than oil
($n_\text{oil}>n_\text{water}$). At the upper interface: Reflected light has a
phase shift of \ang{180} ($\pi$) because $n_\text{air}<n_\text{oil}$ (fast to
slow medium); at the lower interface: Reflected light has no phase shifts (slow
to fast medium)

Because there is only one phase shift, like the soap bubble, a
\textbf{constructive maximum} occurs if $\Gamma$ is a half-number multiple of
wavelength:
\begin{equation}
  \boxed{
    2n_\text{oil}t=\left(m-\frac12\right)\lambda
  }
\end{equation}
while a \textbf{destructive minimum} occurs if $\Gamma$ is a whole-number
multiple of wavelength:
\begin{equation}
  \boxed{
    2n_\text{oil}t=m\lambda
  }
\end{equation}
where $m=1,2,3\ldots$ assuming that the index of refraction of the oil film
is \emph{higher} than water.


\section{Anti-Reflective Coating}
\begin{figure}[ht]
  \centering
  \begin{tikzpicture}[scale=.65]
    \fill[cyan!20] rectangle (6,3);
    \draw (0,0)--(6,0);
    \draw (0,3)--(6,3);
    \draw[<->] (5.5,0)--(5.5,3) node[midway,fill=cyan!20]{$t$};
    \draw[vectors,red] (1.6,5)--(2,3) node[left]{$\pi$}--(2.4,5);
    \draw[vectors,teal] (1.55,5)--(1.95,3)--(2.1,0) node[left]{$\pi$}--
    (2.2,3)--(2.6,5);
    \node at (-.3,1.5){$n$};
    \node at (-.3,3.4){$n_\text{air}=1$};
    \node at (-.3,-.4){$n_\text{glass}$};
  \end{tikzpicture}
\end{figure}

The anti-reflective coating on eyeglasses and camera lenses has an index of
refraction that is lower than glass (but higher than air):
\begin{equation}
  n_\text{air}<n_\text{coating}<n_\text{glass}
\end{equation}
In this case, light reflects with a phase shift of $\pi$ on both the upper and
lower interfaces, because in both cases, the light is travelling from a fast to
a slow medium

The interference conditions for anti-reflective coating is \emph{opposite}
to the soap bubble and oil film, because the phase shifts occur on both
boundaries. A \emph{constructive maximum} occurs if the path difference is a
whole-number multiple of wavelength:
\begin{equation}
  \boxed{
    2nt=m\lambda
  }
\end{equation}
while a \emph{destructive minimum} occurs if the path difference is a
half-number multiple of wavelength:
\begin{equation}
  \boxed{
    2nt=\left(m-\frac12\right)\lambda
  }
\end{equation}
where $m=1,2,3\ldots$



\chapter{Electromagnetic Wave}

The final question for this unit: If light is a wave, then what \emph{kind} of
wave is it?
%
%
%
%
%\section{New Physics: Maxwell's Equations}
%  \begin{columns}
%    \column{.3\textwidth}
%    \centering
%    \pic1{graphics/PORTRAIT-James-Clerk-Maxwell}\\
%    James Clerk Maxwell
%    
%    \column{.7\textwidth}
%    \begin{itemize}
%    \item Classical laws of electrodynamics
%    \item Published in 1861 and 1862
%    \item Explains the relationship between
%      \begin{itemize}
%      \item Electricity
%      \item Electric Circuits
%      \item Magnetism
%      \item Optics
%      \end{itemize}
%    \item Previously these disciplines are thought to be separate and not
%      related
%    \end{itemize}
%  \end{columns}
%
%
%
%
\section{Maxwell's Equations}
\begin{align*}
  \nabla\cdot\mathbf E &=\frac\rho{\varepsilon_0}\\
  \nabla\cdot\mathbf B &= 0\\
  \nabla\times\mathbf E &=-\frac{\partial\mathbf B}{\partial t}\\
  \nabla\times\mathbf B &=
  -\mu_0\mathbf J+\mu_0\varepsilon_0\frac{\partial\mathbf E}{\partial t}
\end{align*}

%    \begin{tikzpicture}[overlay]
%      \node[text width=115,draw=violet,fill=violet!5,text=violet] (G1) at
%      (2,3.5) {
\textbf{Gauss's law for electricity:} Electric fields must begin and/or end at
a charge.

%      \draw[axes,violet] (G1) to[out=50,in=180] (4.9,4.5);
%
%      \node[text width=120,draw=orange,fill=orange!5,text=orange] (G2) at
%      (11.22,4.2) {
\textbf{Gauss's law for magnetism:} Magnetic field lines do not have a
beginning or an end.
%      };
%      \draw[axes,orange] (G2) to[out=180,in=0] (7.1,3.5);
%
%      \node[text width=115,draw=green!70!black,fill=green!5,text=green!70!black]
%      (F) at (2,1.2)
\textbf{Faraday's law:} Fluctuations in the magnetic field in time produces an
electric field that varies in space.

%      
%      \draw[axes,green!70!black] (F) to[out=60,in=180] (4.7,2.4);
%
%      \node[text width=147,draw=red!70!black,fill=red!5,text=red!70!black]
%      (A) at (11.7,2.6)
\textbf{Ampere's law:} Fluctuations in the electric field in time produces a
magnetic field that varies in space.

%      \draw[axes,red!70!black] (A) to[out=270,in=0] (10.2,1);
%    \end{tikzpicture}
%  }

That's a lot of symbols that you won't recognize. Solving them require
\emph{a lot} of difficult calculus that even most science students in
university don't need to learn. (\textbf{i.e.\ you don't need to learn this})


\section{Major Findings of Maxwell's Equations}
Disturbances in the electric and magnetic fields propagate as a wave with
defined speed (``speed of light''):
\begin{equation}
  \boxed{
    c=\frac1{\sqrt{\varepsilon_0\mu_0}} = \SI{2.998e8}{\metre\per\second}
  }
\end{equation}

\begin{definition}
  \textbf{Permittivity of free space} $\varepsilon_0$: the ability of a
  vacuum to resist the formation of an electric field within it. The
  constant is related to the Coulomb constant.
\end{definition}

\begin{definition}
  \textbf{Permeability of free space} $\mu_0$: A measure of the ability
  of a vacuum to become magnetized.
\end{definition}
\begin{itemize}
\item Scientist have previously measured the speed of light to good accuracy
\item Maxwell's equations show that light is (probably) an electromagnetic
  (``EM'') wave
\end{itemize}
%
%
%
%
\section{The Electromagnetic Spectrum}
\begin{figure}[ht]
  \centering
  \pic1{thinFilmInterference/graphics/electromagneticspectrum-141b490bac872789434}
\end{figure}
%
%
%
%
%\section{Polarization}
%
\section{On Polarization of Light}
%  \begin{itemize}
%  \item Light is an electromagnetic wave, generated by
%    \begin{itemize}
%    \item An oscillating charged particle (e.g. shaking an electron violently)
%    \item An alternating (``A/C'') current (i.e.\ many oscillating charges)
%    \item Through black-body radiation
%    \end{itemize}
%  \item EM waves have both an oscillating electric field ($\mathbf E$) and
%    magnetic field ($\mathbf B$), because
%    \begin{itemize}
%    \item A charged particle creates an electric field, and
%    \item A moving charged particle creates a magnetic field
%    \end{itemize}
%  \item $\mathbf E$ and $\mathbf B$ are always perpendicular to one another,
%    according to Maxwell's equations
%  \end{itemize}
%

%  Charged particles can vibrate in any direction, so the oscillating $\mathbf E$
%  and $\mathbf B$ can look quite chaotic. We can only guarantee that, $\mathbf E$ and
%  $\mathbf B$ are:
%  \begin{itemize}
%  \item Always perpendicular to each other
%  \item Always perpendicular to the direction of wave travel
%  \item This kind of light (or general EM wave) is ``unpolarized''
%  \item Most EM waves you experience in life are this kind:
%  \end{itemize}
%  \begin{center}
%    \vspace{-.1in}\pic{.7}{graphics/T1Zlt}
%  \end{center}

%  But if we can confine $\mathbf E$ and $\mathbf B$ to one plane, then we have a
%  ``polarized'' light:
%  \begin{center}
%    \pic{.4}{graphics/em-20field}
%  \end{center}
%  There are a few ways to do this\ldots
%
%
%
%
%
%\section{Polarization of Light Using Polarizer}
%  A polarizer is really just a grill that only lets in vibration in one
%  direction through:
%  \begin{center}
%    \pic{.4}{graphics/polarizerfencemodel600}
%  \end{center}
%  \begin{itemize}
%  \item The incoming wave can be vibrating in any direction, but outgoing wave
%    only vibrates in one direction.
%  \item Sunglasses with polarizing lens
%  \item Polarizer filters on cameras
%  \end{itemize}
%
%
%
%
%%\section{Polarization by Reflection}
%%  \begin{columns}
%%    \column{.35\textwidth}
%%    \pic1{graphics/01fig16}
%%    
%%    \column{.65\textwidth}
%%    At \textbf{Brewster's angle}, the light reflected off a medium (e.g.\
%%    glass, water) is also polarized
%%
%%    \eq{-.1in}{
%%      \theta_B =\tan^{-1}\left(\frac{n_2}{n_1}\right)
%%    }
%%    \begin{itemize}
%%    \item Incident light is non-polarized
%%    \item Reflected light is polarized
%%    \item Refracted light is partially polarized
%%    \item For water ($n=1.33$), $\theta_B=\ang{53}$
%%    \item For glass ($n=1.5$), $\theta_B=\ang{56}$
%%    \end{itemize}
%%  \end{columns}

