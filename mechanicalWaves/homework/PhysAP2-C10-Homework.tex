\documentclass{../../oss-apphys-exam}

\begin{document}
\gentitle{10}{MECHANICAL WAVES}

%\classkickMCinstructions

\begin{questions}
  \question Of the following properties of a wave, the one that is
  \emph{independent} of the others is its \underline{\hspace{1in}}.
  \begin{choices}
    \choice amplitude
    \choice speed
    \choice wavelength
    \choice frequency
    \choice period
  \end{choices}

  \question Waves transmit \underline{\hspace{1in}} from one place to another.
  \begin{choices}
    \choice mass
    \choice amplitude
    \choice wavelength
    \choice energy
  \end{choices}

  \question Which of the following is an example of a longitudinal wave?
  \begin{choices}
    \choice Water wave
    \choice Microwave
    \choice Sound wave
    \choice Radio wave
    \choice X-ray
  \end{choices}
    
  \question Which of the following distances describes the amplitude of a wave?
  \begin{choices}
    \choice Crest to trough
    \choice Crest to crest
    \choice Trough to trough
    \choice Top of crest to bottom of trough
    \choice Crest to equilibrium position
  \end{choices}

  \question Which of the following measurements is used to find the wavelength?
  \begin{choices}
    \choice Crest to zero displacement
    \choice Crest to trough
    \choice Trough to zero displacement
    \choice Trough to crest
    \choice Crest to crest
  \end{choices}  

  \question A wave has a frequency of \SI{100}{\hertz} and a wavelength of
  \SI1{\metre}. What is the speed of the wave?
  \begin{choices}
    \choice\SI{.01}{\metre\per\second}
    \choice\SI{1}{\metre\per\second}
    \choice\SI{10}{\metre\per\second}
    \choice\SI{100}{\metre\per\second}
    \choice\SI{1000}{\metre\per\second}
  \end{choices}
    
  \question Which of the following quantities remains constant as a mechanical
  wave travels from one type of spring into another?
  \begin{choices}
    \choice Frequency
    \choice Wavelength
    \choice Speed
    \choice Amplitude
    \choice Spring constant
    \end{choices}
  \newpage
    
%  \question A jackhammer operator wears a set of protective headphones. Through
%  the headphones, a sound wave is broadcast that is \ang{180} out of phase with
%  the jackhammer sound wave. The result is that he does not hear the sound of
%  the jackhammer. These two sound waves are an example of which of the
%  following?
%  \begin{choices}
%    \choice Standing wave
%    \choice Transverse wave
%    \choice Destructive interference
%    \choice Constructive interference
%    \choice Doppler effect
%  \end{choices}
      
  \question Two waves have the same frequency. What other characteristic must be
  the same for these waves?
  \begin{choices}
    \choice Speed
    \choice Period
    \choice Amplitude
    \choice Intensity
    \choice Wavelength
  \end{choices}

  \question A child dips her finger repeatedly into the water to make waves. If
  she dips her finger more frequently, the wavelength \underline{\hspace{.5in}}
  and the speed \underline{\hspace{.5in}}.
  \begin{choices}
    \choice Increases; decreases
    \choice Decreases; increases
    \choice Increases; stays the same
    \choice Decreases; stays the same
    \choice Stays the same; increases
  \end{choices}
  
  \question As a wave is formed, what is the relationship between the wavelength
  and frequency?
  \begin{choices}
    \choice Linearly related and directly proportional
    \choice Linearly related but not directly proportional
    \choice Inversely proportional
    \choice Parabolic
    \choice Exponential
  \end{choices}
  
  \uplevel{  
    \textbf{Questions \ref{des1}--\ref{des2}} use the following figure:
    \begin{center}
      \begin{tikzpicture}
        \draw[thick] (0,0)--(1,0)--(1.5,1.5)--(2,0)--(5,0)--(6.5,-2/3)
        --(8,0)--(9,0);
        \draw[vectors] (2.2,.5)--(3.2,.5);
        \draw[vectors] (4.8,-.5)--(3.8,-.5);
        \draw[thick,<->|] (.8,0)--(.8,1.5) node[midway,left]{$+A$};
        \draw[thick,<->|] (8.2,0)--(8.2,-2/3) node[midway,right]{$-A/3$};
      \end{tikzpicture}
    \end{center}
  }

  \question Two waves are traveling on a string. The directions and amplitude
  of each wave are shown in the figure. When the two waves meet, what will be
  the amplitude of the resulting wave?
  \label{des1}
  \begin{choices}
    \choice $-4A/3$
    \choice $-2A/3$
    \choice $0$
    \choice $2A/3$
    \choice $4A/3$
  \end{choices}
    
  \question The figure depicts which of the following phenomena?
  \begin{choices}
    \choice Standing wave
    \choice Transverse wave
    \choice Destructive interference
    \choice Constructive interference
    \choice Doppler effect
  \end{choices}
    
  \question After the waves interact, what will happen?
  \label{des2}
  \begin{choices}
    \choice One wave ($2A/3$) will travel to the right.
    \choice One wave ($-2A/3$) will travel to the left.
    \choice There will be no more waves.
    \choice One wave ($+A$) will travel to the right, while one wave ($-A/3$)
    will travel to the left.
    \choice One wave ($-A$) will travel to the right, while one wave ($+A/3$)
    will travel to the left.
  \end{choices}
  \newpage
      
%  \question Which of the following best describes a wave?
%  \begin{choices}
%    \choice pattern resembling a sine wave
%    \choice An object that oscillates back and forth at a characteristic
%    frequency
%    \choice A disturbance that carries energy and momentum from one place to
%    another with the transfer of mass
%    \choice A disturbance that carries energy and momentum from one place to
%    another without the transfer of mass
%    \choice An oscillating electric and magnetic field that cannot travel
%    through a vacuum
%  \end{choices}

  \question While relaxing at a wave pool, you notice the wave machine making 12
  waves in \SI{40}{\second} and the wave crests are \SI{3.6}{\metre} apart.
  \begin{parts}
    \part Determine the speed that the waves must be traveling.
    \vspace{\stretch1}
    
    \part Your friend tells you that he can make the waves travel faster by
    increasing the frequency to 2 waves per second. Would you agree? Explain.
    What would change in the wave if the frequency increases, and what would be
    the new value?
    \vspace{\stretch{1.5}}
  \end{parts}
  
  \question A steel piano wire is \SI{.70}{\metre} long and has a mass of
  \SI{5.0}\gram. It is stretched with a tension of \SI{500}\newton.
  \begin{parts}
    \part What is the speed of the transverse wave on the wire?
    \vspace{\stretch1}
    
    \part To reduce the speed by a factor of 2 without changing the tension,
    what mass of copper wire would have to be wrapped around the steel wire?
    \vspace{\stretch1}
  \end{parts}
  
  \question The E-string on a violin produces a frequency of \SI{660}\hertz,
  and the wavelength of the standing wave on the string is
  \SI{650}{\milli\metre},
  \begin{parts}
    \part Estimate the speed of the travelling wave on the violin.
    \vspace{\stretch{.7}}
    
    \part If that E-string has a tension of \SI{82.3}\newton, what is the
    density ($\rho$) of the metal that the string is made of? Assume that the
    string is perfectly circular, with a radius of \SI{.27}{\milli\metre}.
    Mass of an object $m=\rho V$, and the volume of a cylinder is $V=\pi r^2L$.
    (Hint: solve the problem algebraically first, and then substitute numerical
    values. Some terms will automagically cancel.)
    \vspace{\stretch2}
  \end{parts}
  \newpage



  % TAKEN FROM THE 2016 AP PHYSICS 1 EXAM FREE-RESPONSE QUESTION #5
  \uplevel{
%    \classkickFRQinstructions   
    \cpic{.3}{dangling}
  }
  \question The figure above on the left shows a uniformly thick rope hanging
  vertically from an oscillator that is turned off. When the oscillator is on
  and set at a certain frequency, the rope forms the standing wave shown above
  on the right. $P$ and $Q$ are two points on the rope.
  \begin{parts}
    \part The tension at point $P$ is greater than the tension at point $Q$.
    Briefly explain why.
    \vspace{\stretch1}

    \part A student hypothesizes that increasing the tension in a rope increases
    the speed at which waves travel along the rope. In a clear, coherent
    paragraph-length response that may also contain figures and/or equations,
    explain why the standing wave shown above supports the student’s hypothesis.
    \vspace{\stretch1}
  \end{parts}
  \newpage

  % TAKEN FROM THE 2018 AP PHYSICS 1 EXAM FREE-RESPONSE QUESTION #4
  \question A transverse wave travels to the right along a string.
  \begin{parts}
    \part Two dots have been painted on the string. In the diagrams below, those
    dots are labeled $P$ and $Q$.
    \label{parta}
    
    \begin{subparts}
      \subpart The figure below shows the string at an instant in time. At the
      instant shown, dot $P$ has maximum displacement and dot $Q$ has zero
      displacement from equilibrium. At each of $P$ and $Q$, draw an arrow
      indicating the direction of the instantaneous velocity of that dot. If
      either dot has zero velocity, write ``$v=0$'' next to the dot.
      \cpic{.4}{wave}
      
      \subpart The figure below shows the string at the same instant as shown in
      part (\ref{parta})i. At each of $P$ and $Q$, draw an arrow indicating the
      direction of the instantaneous acceleration of that dot. If either dot
      has zero acceleration, write ``$a=0$'' next to the dot.
      \cpic{.4}{wave}  
    \end{subparts}
    
    \uplevel{
      The figure below represents the string at time $t=0$, the same instant as
      shown in part (\ref{parta}) when dot $P$ is at it maximum displacement
      from equilibrium. For simplicity, dot $Q$ is not shown.
      \cpic{.4}{wave2}
    }
    \part 
    \begin{subparts}
      \subpart On the grid below, draw the string at a later time $t=T/4$, where
      $T$ is the period of the wave.
%      \begin{center}
%         \begin{tikzpicture}[yscale=.8,xscale=.85]
%           \draw[gray] (0,-2) grid(8,2);
%           \draw[vectors](0,-2)--(0,2.5) node[above]{$y$ (cm)};
%           \draw[vectors](0,0)--(8.5,0) node[right]{$x$ (cm)};
%         \end{tikzpicture}
%      \end{center}
      \cpic{.45}{grid}
      
      \subpart On your drawing above, draw a dot to indicate the position of dot
      $P$ on the string at time $t=T/4$ and clearly label the dot with the
      letter $P$.
    \end{subparts}
    
    \part Now consider the wave at time $t=T$. Determine the distance traveled
    (not the displacement) by dot $P$ between times $t=0$ and $t=T$.
  \end{parts}
  \newpage
  
  % TAKEN FROM THE 2018 AP PHYSICS 1 EXAM FREE-RESPONSE QUESTION #4
  \uplevel{
    \cpic{.65}{string-mass}
  }
  \question The figure above shows a string with one end attached to an
  oscillator and the other end attached to a block. The string passes over a
  massless pulley that turns with negligible friction. Four such strings, $A$,
  $B$, $C$, and $D$, are set up side by side, as shown in the diagram below.
  Each oscillator is adjusted to vibrate the string at its fundamental
  frequency $f$. The distance between each oscillator and pulley $L$ is the
  same, and the mass $M$ of each block is the same. However, the fundamental
  frequency of each string is different.
  \cpic{.65}{topview}

  The equation for the velocity $v$ of a wave on a string is
  $v=\sqrt{\dfrac{F_T}{m/L}}$, where $F_T$ is the tension of the string and
  $m/L$ is the mass per unit length (linear mass density) of the string.
  \begin{parts}
    \part What is different about the four strings shown above that would result
    in their having different fundamental frequencies? Explain how you arrived
    at your answer.
    \vspace{\stretch1}
    
    \part A student graphs frequency as a function of the inverse of the linear
    mass density. Will the graph be linear? Explain how you arrived at your
    answer.
    \vspace{\stretch1}
    \newpage

    \part The frequency of the oscillator connected to string $D$ is changed so
    that the string vibrates in its second harmonic. On the side view of string
    $D$ below, mark and label the points on the string that have the greatest
    average vertical speed.
    \cpic{.65}{sideview}
  \end{parts}
\end{questions}
\end{document}
