\documentclass[letterpaper,11pt]{article}
\usepackage[margin=1in]{geometry}
\usepackage{enumitem}
  
\begin{document}
\begin{center}
  \Large\textbf{NOTICE FOR AP PHYSICS 1 CLASS 13}
\end{center}
Please note that there is no new topics presented for Class 13. Instead, there
will be an in-class practice test. Here are some information about test:
\begin{itemize}[leftmargin=18pt]
\item The test is divided into two sections: a set of multiple-choice
  questions, followed by a (very) short break, then free-response (i.e.\
  problem-solving) questions. During the multiple-choice section, you will not
  be able to see the free-response questions. One the multiple-choice section is
  finished, you are not allowed to go back and work on those questions. 
\item For the multiple-choice questions:
  \begin{itemize}[leftmargin=20pt]
  \item Answer all questions on Classkick by clicking in the selection box.
    \textbf{Do \emph{not} use the pen or highlighter function. It will
      \emph{not} register and I will mark it as no submission.}
  \item You do not need to show your work.
  \item Like the actual AP Physics 1 exam, you will not be penalized for wrong
    answers, so if you are stuck, your objective will be to eliminate as many
    impossible answers as you can, and then improve your odds for guessing.
  \item In an actual AP Physics 1 exam, you will have 90 minutes to complete 50
    questions. There are far fewer questions for the practice test, and it will
    take a lot less time.
  \end{itemize}
\item For the free-response questions:
  \begin{itemize}[leftmargin=20pt]
  \item Show your work. The correct answers only account for about $1/3$ of the
    total marks. Start with the relevant concepts and equations, then solve for
    the unknown algebraically before substituting numerical values when
    required.
  \item Your answer should have 2 or 3 significant figures.
  \item For questions where you are asked to ``justify your answer'', use two
    or three complete sentences. Be clear and concise.
  \item \textbf{Do not use the text box function on Classkick to write
    equations. They will not be counted towards your marks.}
  \item In an \emph{actual} AP Physics 1 exam, you will have 90 minutes to
    complete 7 questions of various lengths and difficulties. The total mark
    for each questions is printed on the test itself.
  \item Please write clearly. Your AP Physics teacher has many superpowers,
    but mind-reading is not one of them.
  \end{itemize}
\item The questions will \emph{not} be provided to you on a PDF file. Read
  all the questions through your web browser.
\item Unlike the actual AP Physics 1 exam, you are allowed to use all teaching
  material from Meritus Academy, including slides, handouts, and any notes you
  took during class. But pay attention: what you need to look up is likely to
  be where your weaknesses are. It will be advantageous to you to have the
  AP exam equation sheet handy.
\item Any questions for your teacher should be asked over on the Zoom chat
  window, and they will get back to you, also via the Zoom chat, at the
  earliest possible time.
\item During the multiple-choice questions, your teacher will monitor your
  progress in real time on Classkick.
\item Please inform your teacher if you have any difficulties accessing the
  test through Classkick.
\end{itemize}
\end{document}
