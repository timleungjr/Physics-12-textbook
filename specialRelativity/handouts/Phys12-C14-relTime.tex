\documentclass{../../oss-handout}
\usepackage{newtxtext,newtxmath}
\usepackage{enumitem}
\usepackage{siunitx}

\setlength{\parindent}{0pt}
\setlength{\parskip}{8pt}
\setlength{\headheight}{26pt}

\newcommand{\bigsqrt}{\ensuremath\sqrt{1-\left(\dfrac vc\right)^2}}

% Set the page style for the document
\pagestyle{plain}

% Course & handout information
\renewcommand{\institution}{Meritus Academy}
\renewcommand{\coursetitle}{Grade 12 Physics}
\renewcommand{\term}{Updated Fall 2023}
\title{In-Class Example: Relativity of Time}
\author{Dr.\ Timothy Leung}
\date{\today}

\begin{document}
\thispagestyle{title}
\gentitle

\fbox{
  \begin{minipage}{.97\linewidth}
    \textbf{Example 1a:} A rocket speeds past an asteroid at $0.800c$. If Kim,
    an observer on the rocket, sees \SI{10.}{\second} pass on her clock, how
    long would that time interval be as see by Jim, an observer on the asteroid?
  \end{minipage}
}

This is a standard problem in relativistic mechanics. Here, there are two
distinct events:
\begin{enumerate}[nosep]
\item Kim's clock on the rocket starts measuring time, and
\item Kim's clock on the rocket stops measuring time.
\end{enumerate}
This problem is \emph{how} the time interval between is perceived by different
observers. The clock is stationary relative to Kim. According to her, it
measures the \emph{proper} time $t$. As far as Kim is concerned, she, the
rocket, and her clock are \emph{all} stationary; everything else is moving.
However, that clock is moving relative to Jim, so he measures the
\emph{dilated} time $t'$, which is related to proper time by the motion of the
rocket:
\begin{displaymath}
  t' =\frac t\bigsqrt = \frac{10.0}{\sqrt{1-(0.800)^2}}=\SI{16.7}\second
\end{displaymath}
To Jim, \SI{16.7}{\second} has passed on his clock, but only \SI{10.}{\second}
has passed on Kim's clock on the rocket. Therefore, Jim concludes that Kim's
clock is running slow. \textbf{A moving clock runs slow.} Jim and Kim disagree
on how much time has passed.

\vspace{.2in}\fbox{
  \begin{minipage}{.97\linewidth}
    \textbf{Example 1b:} A rocket speeds past an asteroid at $0.800c$. If Jim,
    an observer in the \emph{asteroid} sees \SI{10.}{\second} pass on his
    clock, how long would that time interval be as see by Kim, an observer on
    the \emph{rocket}?
  \end{minipage}
}

This next example is a more a interesting problem when paired with the first.
This time, there are also two distinct events to consider:
\begin{enumerate}[nosep]
\item Jim's clock on the \emph{asteroid} starts measuring time, and
\item Jim's clock on the \emph{asteroid} stops measuring time.
\end{enumerate}
The difference, of course, is the \emph{location} of the clock, which is now on
the asteroid, and therefore stationary relative to Jim. The \SI{10.}{\second}
that Jim measures is the proper time. As far as Jim is concerned, he, the
clock, and the asteroid, are all stationary; everything else is moving. On the
other hand, Kim sees the clock moving, and therefore she measures the dilated
time, which is related to the motion of the rocket by:
\begin{displaymath}
  t' =\frac t\bigsqrt = \frac{10.0}{\sqrt{1-(0.800)^2}}=\SI{16.7}\second
\end{displaymath}
There are no additional calculations to do; you have already done it in the last
problem! In fact, \textbf{both observers see the other's clock run slow!}

But how is this even possible? How can Kim's clock be slower than Jim's, while
Jim's clock is also slower than Kim's? The answer lies with the concept of the
relativity of simultaneity.

When Kim measures \SI{10.0}{\second} on her clock, in order to compare time,
Jim must \emph{synchronize} his own clock to start and stop with Kim's. (We
will assume that he is successful.) Jim concludes that Kim's clock runs slow,
because clearly, he can see that \SI{16.7}{\second} has passed on his clock.
Kim disputes Jim's conclusion, because \textbf{events that are simultaneous
  in one reference frame is not simultaneous in another}. From Kim's reference
frame, Jim's clock was \emph{never} synchronized; he \emph{didn't} the start
his clock at the same time as hers (he was late) and \emph{didn't} stop his
clock at the same time either (he was late again). Kim concludes that Jim is
just terrible at doing experiments.

However, when we reverse the example, Kim is the one who has to synchronize her
clock to Jim's, and to her surprise, she finds that Jim's clock runs slow! Jim,
however, observes that Kim makes the same mistake that she accuses him of: she
starts her clock late and stops even later. He chuckles at Kim being the bad
experimenter herself.

How Jim and Kim resolve their differences is left to your imagination. In the
meantime, students in this class know why the disagreement started, and what
role relativity plays.

\vspace{.3in}A Dr.\ Seuss-style tongue twister is inevitable:
\begin{center}
  \begin{minipage}{.4\linewidth}
    \emph{Now Kim sees Jim's clock run slow.\\
      Now Jim sees Kim's clock run slow.\\
      Whose clock is slower?\\
      Both clocks are slower!
    }
  \end{minipage}
\end{center}
  

%\fbox{
%  \begin{minipage}{.97\linewidth}
%    \textbf{Example 2:} A spacecraft passes Earth at a speed of
%    \SI{2.00e8}{m/s}. If observers on Earth measure the length of the
%    spacecraft to be \SI{554}{\metre}, how long would it be according to its
%    passengers?
%  \end{minipage}
%}
%
%Since the observers on Earth are in a different frame of reference from the
%spacecraft, the length that they observe is the contracted length. The
%passengers on the spacecraft itself will observe the proper length:
%\begin{displaymath}
%  L=L_0\bigsqrt\;\;\;\longrightarrow\;\;\;
%  L_0
%  =\frac{L}{\bigsqrt}
%  =\frac{554}{\sqrt{1-\left(\frac{\num{2.00e8}}{\num{3.00e8}}\right)^2}}
%  =\SI{743}{\metre}
%\end{displaymath}
%Likewise, the passengers on the spacecraft will also see that the earth
%contracted along the direction of motion, and has turned into an oval instead.
%
%%\fbox{
%%  \begin{minipage}{.97\linewidth}
%%    \textbf{Example 3:} An electron has a rest mass of
%%    \SI{9.11e-31}{\kilo\gram}. In a detector, it behaves as if it has a mass
%%    of \SI{12.55e-31}{\kilo\gram}. How fast is that electron moving relative to
%%    the detector?
%%  \end{minipage}
%%}
%%
%%    Since the observers on Earth are in a different frame of reference from the
%%    spacecraft, the length that they observe is the contracted length. The
%%    passengers on the spacecraft itself will observe the proper length:
%%    \begin{displaymath}
%%      L=L_0\bigsqrt\;\;\;\longrightarrow\;\;\;
%%      L_0
%%      =\frac{L}{\bigsqrt}
%%      =\frac{554}{\sqrt{1-\left(\frac{2.00\e{8}}{3.00\e{8}}\right)^2}}
%%      =
%%    \end{displaymath}
%%    Similar to Problems 1a and 1b, the passengers on the spacecraft will also
%%    see that the earth contracted along the direction of motion, and has turne
%%    into an oval instead.
%
%\fbox{
%  \begin{minipage}{.97\linewidth}
%    \textbf{Example 3:} A rocket car with a mass of \SI{2.00e3}{\kilo\gram} is
%    accelerated to \SI{1.00e8}{\metre/\second}. Calculate its kinetic energy.
%    \begin{enumerate}[noitemsep,topsep=0pt]
%    \item Using the classical or general equation for kinetic energy
%    \item Using the relativistic equation for kinetic energy
%    \end{enumerate}
%  \end{minipage}
%}
%
%%    \begin{enumerate}
%Using the classical equation, kinetic energy is given by:
%\begin{displaymath}
%  K
%  =\frac{1}{2}mv^2=\frac{1}{2}\times\num{2.00e3}\times(\num{1.00e8})^2
%  =\SI{1.00e19}{\joule}
%\end{displaymath}
%Using relativistic equation for mass:
%\begin{align*}
%  K &= mc^2-m_0c^2=\gamma m_0c^2-m_0c^2=m_0c^2\left(\gamma-1\right)\\
%  &=\num{2.00e3}\times(\num{3.00e8})^2\times\left(\frac{1}{\sqrt{1-\left(\frac{\num{1.00e8}}{\num{3.00e8}}\right)^2}}-1\right)
%=\SI{1.09e19}{\joule}
%\end{align*}
\end{document}
