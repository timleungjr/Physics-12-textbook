\documentclass{../oss-handout}
\usepackage{enumitem}
\usepackage{titlesec}
\usepackage{newtxtext}

\setlength{\parindent}{0pt}
\setlength{\parskip}{2pt}
\setlength{\headheight}{26pt}

\titleformat*{\section}{\large\bfseries}
\titlespacing\section{0pt}{10pt plus 4pt minus 4pt}{4pt plus 12pt minus 20pt}

% Set the page style for the document
\pagestyle{plain}

% Course & handout information
\renewcommand{\institution}{Meritus Academy}
\renewcommand{\coursetitle}{Grade 11 Physics}
\renewcommand{\term}{Winter/Spring 2024}%{Fall 2023}
\title{GRADE 11 PHYSICS COURSE SYLLABUS}
\author{}
\date{\today}

\begin{document}
\thispagestyle{title}
\gentitle


\section*{Course Objectives}
The goal of the Grade 11 Physics course at Meritus Academy is to:
\begin{itemize}[nosep]
\item Develop analytical skills, strategies, and habits of mind required for
  scientific inquiry, including critical thinking and inferring
\item Develop communicative skills, strategies, and habits required for
  scientific inquiry
\item Learn fundamental concepts of introductory high-school physics
\item Extend fundamental concepts beyond the mandate of the Ontario curriculum
\item Gain exposure to both mainstream and unconventional applications of
  scientific concepts
\end{itemize}


\section*{Class Times}
  % FOR FALL 2023 SESSION
  %\item Tuesdays 6:30pm to 9:00pm (Mr.\ Neell Young, Online)
  %\item Saturdays 10:30am to 1:00pm (Dr.\ Timothy Leung, In-Person)
  %\item Sundays 1:20pm to 3:50pm (Mr.\ Hardev Lad, online)
  %\item Sundays 4:10pm to 6:30pm (Mr.\ Ryan Lin, in-person)
  %\item Tuesdays 6:30pm to 9:00pm (Mr.\ Hardev Lad, online)
  %\end{enumerate}

For the Winter/Spring 2024 session, there are four sections of Grade 11 Physics
offered at Meritus Academy.
\begin{center}
  \begin{tabular}{|c|c|c|c|}
    \hline
    \rowcolor{lightgray}
    \textbf{Section} & \textbf{Day} & \textbf{Time} & \textbf{Format}\\
    \hline\hline
    Physics G11-1 & Tuesdays  & 6:30 -- 9:00 pm & Online \\
    \hline
    \hspace{.3in}Physics G11-2\hspace{.3in} &
    \hspace{.3in}Saturdays\hspace{.3in} &
    \hspace{.3in}7:00 -- 9:30 pm\hspace{.3in} &
    \hspace{.3in}In-person\hspace{.3in} \\
    \hline
    Physics G11-3 & Sundays   & 4:10 -- 6:40 pm & In-person \\
    \hline
    Physics G11-4 & Sundays   & 7:00 -- 9:30 pm & Online \\
    \hline
  \end{tabular}
\end{center}
Please contact \texttt{info@olympiads.ca} to request a Zoom link for make-up
class(es).
% FOR SUMMER 2023 SESSION
%There are 10 sections of the course during the Summer 2023 session:
%\begin{enumerate}[nosep]
%\item Monday \& Thursday, 10:00am--12:30pm (Dr.\ M.\ Horbatsch, online)
%\item Monday, Tuesday, Thursday \& Friday, 10:00am--12:30pm (Mr.\ A.\
%  Hodaei, Jul. 3 to 28, online)
%\item Monday, Tuesday, Thursday \& Friday, 10:00am--12:30pm (Mr.\ H.\ Lad,
%  Jul.\ 31 to Aug.\ 25, online)
%\item Monday \& Thursday, 1:00pm--3:30pm (Dr.\ M.\ Horbatsch, online)
%\item Monday \& Thursday, 7:00pm--9:30pm (Mr.\ J.\ You, online)
%\item Tuesday \& Friday, 1:00pm--3:30pm (Mr.\ J.\ You, online)
%\item Tuesday \& Friday, 4:00pm--6:30pm (Mr.\ R.\ Lin, online)
%\item Saturday \& Sunday, 10:00am--12:30pm (Mr.\ B.\ Ghadyanloo, in-person)
%\item Saturday \& Sunday, 4:00pm--6:30pm (Mr.\ B.\ Ghadyanloo, in-person)
%\item Saturday \& Sunday, 1:00pm--3:30pm (Mr.\ B.\ Ghadyanloo, in-person)

\section*{Online Tutorial}
There is one drop-in tutorial session for Grade 11 Physics students. (The
tutorial session is shared with AP Physics 1 students.) You are encouraged ask
the teacher any questions related to the course material, especially if you
have found some of the concepts difficult to understand.

\hspace{.5in}Day: \textbf{Thursdays}

\hspace{.5in}Time: \textbf{7:00 to 8:00pm}

\hspace{.5in}Dates: \textbf{From February 8 to May 23}

Zoom link can be found in the student account. Please contact
\texttt{info@olympiads.ca} if you have any questions or concernts.




\section*{Course Material}
No textbook is required. Presentation slides, handouts and homework sets are
downloadable from the school website. Students are expected to download course
material from the school website prior to class. When you attend your class,
please bring
\begin{itemize}[nosep]
\item a pen/pencil for note-taking
\item A scientific calculator for working in-class example problems
\end{itemize}




\section*{Homework}
Homework sets are assigned after every class based on the topics covered in
class. There are usually about 15 questions, consisting of multiple-choice,
short-answer and problem-solving questions.
\begin{itemize}[nosep]
\item Homework questions are posted on the school website each week.
  \begin{itemize}
  \item\textbf{In-person classes:} Printed copies of the homework sets are
    distributed at the beginning of each class.
  \item\textbf{Online classes:} Homework sets are posted on Classkick.com.
    Please log into your account access them. If you do not have an account on
    Classkick, your eacher will help you on Class 1.
  \end{itemize}
%\item Most of the homework questions will be reviewed in class before they are
%  due. However, this does \emph{not} mean you don't need to do your homework at
%  home. Always do your best.
\item Homework questions are due at the beginning of the following class. (The
  exceptions are the one-month and weekend classes during the summer session.)
\item Late homework is accepted. However, the usefulness of late-submissions
  are often diminished, and may incur extra work for your teacher.
\item The homework is marked with ``P'' for pass and ``I'' for incomplete. If
  you do get an ``I'', your teacher will let you know how to fix it so you can
  re-submit to pass the homework.
\item Not regularly completing the homework may prompt phone calls to
  communicate with the parents in order to help you better.
\end{itemize}


\section*{Tests}
There are two tests for assessment:
\begin{itemize}[nosep]
\item A \textbf{take-home midterm test} due at the end of Class 8. The test is
  assigned instead of a homework for Class 7. It will cover course material
  from the first two units (classes 1 to 6).
\item An \textbf{in-class final test} on the last day (Class 16). The test will
  cover material from the \emph{entire} course, but will focus on the last
  three units (Classes 8 to 15).
\end{itemize}





\section*{Course Outline}
\begin{enumerate}[itemsep=.05ex,label={\textbf{\arabic*.}}]
\item\textbf{Motion (Classes 1 to 4):} We introduce the concepts in
  \emph{kinematics}---mathematical description of objects in motion
  \begin{itemize}[nosep]
  \item Importance of significant figures and error analysis in science
  \item Vectors vs.\ scalar quantities
  \item Kinematic quantities: position, displacement, distance, velocity, speed
    and acceleration
  \item One-dimensional kinematic equations
  \item Graphical representation of motion in one-dimension
  \item Acceleration due to gravity
  \item Multi-dimensional vector: decomposition and  arithmetic
  \item Relative motion
  \item Projectile motion
  \end{itemize}

\item\textbf{Forces (Classes 5 and 6):} In this unit, Newton's laws of motion
  are introduced to answer the question of \emph{what makes objects change
  their motion?}
  \begin{itemize}[nosep]
  \item Laws of motion
  \item Common forces: gravity, normal force, static \& kinetic friction,
    air resistance, tension
  \item Free-body diagram
  \item Solving dynamics problems
  \item Motion of connected bodies
  \end{itemize}
  
\item\textbf{Energy (Classes 7 to 10):} The central theme in Physics 11 is to
  relate \emph{mechanical work} and \emph{energy} to all other branches of
  physics. We will also learn how kinetic and potential energies are defined,
  and how energy is conserved.
  \begin{itemize}[nosep]
  \item Mechanical work
  \item Work-energy theorem and the definition of kinetic energy
  \item Definitions of potential energies
  \item Conservative vs.\ non-conservative forces
  \item Law of conservation of energy
  \item Power and efficiency
  \item Thermal/internal energy of gases and solids
  \item Laws of thermodynamics %\& heat transfer
  \item Specific heat capacities
  \item Energies of phase change
  \item Nuclear physics
  \end{itemize}

\item\textbf{Waves (Classes 11 and 12):} In this unit, we are introduced to the
  concept of vibrations (harmonic motion), their properties, and their
  relationship to waves. The students will be introduced to the properties of
  mechanical waves, how waves are transmitted, and the concept of standing
  waves. We will study one specific application of waves: sound. We will study
  the nature of sound waves, how it is transmitted, and how it is related to
  music.
  \begin{itemize}[nosep]
  \item Vibrations (harmonic motion)
  \item Physical properties of mechanical waves
  \item Transmission and reflection of waves at boundaries
  \item Transverse and longitudinal waves
  \item Standing waves
  \item Speed of sound, Mach number
  \item Doppler effect
  \item Beat frequency
  \item Resonance frequencies, harmonics and overtones
  \item Musical instruments
  \end{itemize}
  
\item\textbf{Electricity and Magnetism (Classes 13 to 15):} The final unit will
  be on two fundamental topics that are related. Students will review the
  behaviour of charged particles, and relate them to the concept of work and
  energy. You will then extend their knowledge of parallel and series resistors
  in mixed circuit analysis. The relationship between electricity and magnetism
  is introduced, and students learn about motors, generators and transformers.
  \begin{itemize}[nosep]
  \item Electrical potential and electrical potential energy
  \item Parallel, series and mixed circuits
  \item Direct and alternating currents
  \item Electricity inducing magnetic field
  \item Magnetic field inducing electricity
  \item Motors, AC and DC generators and transformers
  \end{itemize}
\end{enumerate}




\section*{Classroom Expectations (Online)}
Students attending the online classes are expected to:
\begin{itemize}[nosep]
\item Log into the Zoom meeting a few minutes before the start of the class.
\item Have your full (first and last) name appear on the screen
\item Have your camera turned on showing your face
\item Be ready to learn and participate during class
\item Type in all your in-class questions and comments into the Zoom chat
  window. Don't wait too long before you ask a question; the longer you wait,
  the less effective it will be to answer your questions
\item Be \emph{specific} with your questions. As skilled as the teachers are,
  vague statements and questions like ``I don't understand, can you explain it
  again'' are difficult, if not impossible to answer
\item Please inform your teacher if you have to leave the class early for
  whatever reason
\item Be respectful to yourself, your teacher, and your fellow students. You
  are expected to act maturely and responsibly
\end{itemize}


\section*{Classroom Expectations (In-Person)}
Students attending the in-person classes are expected to:
\begin{itemize}[nosep]
\item Be in your seat and ready to learn and participate during class.
\item Stay on task without disturbing or distracting others.
\item Raise your hand if you have any questions or comments and wait to be
  called. Don't wait too long before you ask a question.
  \item Be \emph{specific} with your questions. As skilled as the teachers are,
  vague statements and questions like ``I don't understand, can you explain it
  again'' are difficult, if not impossible to answer
\item If you need to leave the class early, your parent needs to pick you up at
  the classroom door, or be brought to the front desk by a secretary.
\item Be respectful for yourself, others, and the facilities; always act in
  a responsible manner.
\end{itemize}



\section*{Homework Standard}
\begin{itemize}
\item You do not need to show work for multiple-choice questions. For online
  classes, please answer them directly on selection box on Classkick. For
  in-person classes, simply circle the correct letter.
\item For problem-solving questions, you must show \emph{all} work by providing
  complete and organized steps.
  \begin{itemize}[nosep]
  \item Always write out the relevant equations \emph{before} substituting
    numerical values
  \item If you introduce any new variables that are different from those used
    in class, briefly explain what they are
  \item Draw diagrams whenever necessary, or whenever it helps to explain your
    work
  \item Only use standard variable names in you calculations (e.g.\ when
    calculating speed, always use $v$ instead of ``$X$'')
  \item Show all algebraic steps and numerical calculations
  \item Circle or box all your final answers
  \item Proper math format must be used, including the proper use of ``='' sign
    and units
  \end{itemize}
  In short, answer the questions as if the reader is learning the concept from
  you, not as if they already understands it.
\item If a question requires you to \emph{explain}, please do so using short
  complete sentences with supporting detail. There is no need to write full
  paragraphs.
\end{itemize}

\vspace{\stretch1}
\section*{Teacher Information}
Teacher name \& contact information: \underline{\hspace{4.5in}}
\end{document}
