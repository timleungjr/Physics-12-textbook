

\section{Multi-Loop Circuits}

%{Circuits Aren't Always Simple}
Some of these problems require you to solve a system of linear equations.
Fig.~\ref{fig:multiloop1} is a simple example with two voltage sources that
cannot be simplified.
\begin{figure}[ht]
  \centering
  \begin{tikzpicture}[american voltages,scale=1.3,thick]
    \draw (0,0) to[battery,l=$\mathcal E_1$] (0,2) to[R,l=$R_1$] (2,2)
      to[R,l=$R_3$] (2,0)--(0,0);
      \draw (2,0)--(4,0) to[battery,l_=$\mathcal E_2$] (4,2)
      to[R,l_=$R_2$] (2,2);
      %\uncover<2->{
      %  \draw[vectors,red] (2,.4)--(2,0) node[midway,right]{$I_1+I_2$};
      %  \draw[vectors,red] (.6,.4) ..controls (0,2) and (2,2).. (1.6,.4)
      %  node[midway,below]{$I_1$};
      %  \draw[vectors,red] (3.4,.4) ..controls (4,2) and (2,2).. (2.4,.4)
      %  node[midway,below]{$I_2$};
      %}
  \end{tikzpicture}
  \caption{A simple multi-loop circuit}
  \label{fig:multiloop1}
\end{figure}
%  \uncover<2->{
%    In this case, we have to draw two loops of current.
%  }
%
%
%
%
%{A More Difficult Example}
%\begin{figure}[ht]
%  \centering
%  \begin{tikzpicture}[american voltages,thick]
%    \draw (0,0) to[battery, l=$\mathcal E_1$] (0,2) to[R=$R_1$] (2,2)
%    to[R=$R_3$] (2,0)--(0,0);
%    \draw (2,0)--(4,0) to[battery,l_=$\mathcal E_2$] (4,2) to[R,l_=$R_2$]
%    (2,2);
%    \begin{scope}[vectors,red]
%      \draw (2,.4)--(2,0) node[midway,right]{$I_1+I_2$};
%      \draw (.6,.4) ..controls (0,2) and (2,2).. (1.6,.4)
%      node[midway,below]{$I_1$};
%      \draw (3.4,0.4) ..controls (4,2) and (2,2).. (2.4,.4)
%      node[midway,below]{$I_2$};
%    \end{scope}
%  \end{tikzpicture}
%\end{figure}

%  We split the circuit into two loops, and apply Kirchkoff's voltage in both:
%  
%  \vspace{-.3in}{\large
%    \begin{align*}
%      \mathcal E_1-I_1R_1-(I_1+I_2)R_3&=0\\
%      \mathcal E_2-I_2R_2-(I_1+I_2)R_3&=0
%    \end{align*}
%  }
%
%  There are two equations and two unknowns ($I_1$ and $I_2$) which we can either
%  solve algebraically or numerically. %We can subtract
%%  (2) from (1), then solve for $I_1$ and $I_2$:
%%
%%  \eq{-.1in}{
%%    I_1=\frac{V_1-I_2R_3}{R_1+R_3}
%%    \quad\quad
%%    I_2=
%%    \frac{\left[V_2-\frac{(V_1-V_2)R_3}{R_1}\right]}
%%         {\left[R_2+\frac{(R_1+R_2)R_3}{R_1}\right]}
%%  }
%  (Try this at home as an exercise.)
%
%
%
%
%%    \begin{tikzpicture}[american voltages,scale=1.1,thick]
%%      \draw (0,0) to[short] (0,2) to[R=\SI3\ohm] (0,4)
%%      to[battery1,l=\SI{42}\volt] (2,4)
%%      to[R=\SI3\ohm] (4,4) to[short] (4,2)
%%      to[R=\SI3\ohm] (4,0) to[short] (2,0)
%%      to[battery1,l=\SI6\volt](2,2) to[R,l_=\SI4\ohm] (0,2);
%%      \draw (0,0) to[R,l_=\SI6\ohm](2,0);
%%      \draw (2,2) to[R,l_=\SI6\ohm](4,2);
%%      \uncover<2->{
%%        \begin{scope}[very thick,red,->]
%%          \draw (.5,3.5)..controls(4.75,4)and(4.75,2)..(.5,2.5)
%%          node[midway,left]{$I_1$};
%%          \draw (0.6,0.4)..controls (0,2) and (2,2)..(1.4,.4)
%%          node[midway,below]{$I_3$};
%%          \draw (2.6,0.4)..controls (2,2) and (4,2)..(3.4,.4)
%%          node[midway,below]{$I_2$};
%%        \end{scope}
%%      }
%%    \end{tikzpicture}
%%   
%%    \column{.57\textwidth}
%%    \begin{itemize}
%%    \item To solve this problem, we define a few `` current loops'' around the
%%      circuit: one on top, one on bottom left, and one on bottom right.
%%    \item<2-> Apply the voltage law in the loops. For example, in the
%%      lower left:
%%
%%      \eq{-.1in}{
%%        4(I_1-I_3)-6-6I_3=0
%%      }
%%    \item<2->Solve the system of equations to find the current. If the current
%%      that you worked out is negative, it means that you have the direction
%%      wrong.
%%    \end{itemize}
%%  \end{columns}
%%
%
%
%
\section{Capacitors in Circuits}

In Chapter~\ref{chapter:capacitor}, we learned that capacitor is a device that
stores energy in the electric field. Two parallel plates with equal and opposite
charges will generate a voltage. Whereas Chapter~\ref{chapter:capacitor} is
concerned primarily with defining the capacitance, here, we focus on the
capacitor's application in cicuits.

In a circuit, a capacitor acts as an energy storage device. When both sides of
a capacitor has a charge of $\pm Q$, the capacitor voltage is given by
\begin{equation*}
  V_c=\frac QC
\end{equation*}
where $C$ is the capacitance. The voltage $V_c$ can be used to drive a current
in the circuit. However, unlike in a battery, where a nearly-constant voltage is
generated \emph{chemically}, in a capacitor, voltage can vary depending on
whether the capacitor is charging or discharging.

\subsection{Capacitors in Parallel}

\begin{figure}[ht]
  \centering
  \begin{tikzpicture}[scale=1.2,thick]
    \draw (0,1) to[short,o-](1,1) to[C=$C_1$] (1,0)to[short,-o](0,0);
    \draw (1,1)--(3,1) to[C=$C_2$]       (3,0)--(1,0);
    \draw (3,1)--(5,1) to[C=$C_3\cdots$] (5,0)--(3,0);
  \end{tikzpicture}
  \caption{Capacitors connected in parallel}
  \label{fig:parallel-capacitors}
\end{figure}
From the voltage law, we know that the voltage across all the capacitors are
the same, i.e.\ $V_1=V_2=V_3=\cdots=V$. We can express the total charge
$Q_\text{tot}$ stored across all the capacitors in terms of capacitance and this
common voltage $V$: 
\begin{equation}
  Q_\text{tot}=Q_1+Q_2+Q_3+\cdots=C_1V+C_2V+C_3V+\cdots
\end{equation}  
Factoring out $V$ from each term gives us the equivalent capacitance, we find
the equivalent capacitance in parallel:
\begin{important-equation}
  C_p=\sum_i C_i
\end{important-equation}
Connecting capacitors effectively increases the cross-sectional area of the
plates, therefore increasing capacitance.

\subsection{Capacitors in Series}
Likewise, we can do a similar analysis to capacitors connected in series.
\begin{figure}[ht]
  \centering
  \begin{tikzpicture}[scale=1.2,thick]
    \draw (0,0) to[C=$C_1$,o-] (1.25,0) to[C=$C_2$] (2.5,0)
    to[C=$C_3$,-o] (3.75,0);
  \end{tikzpicture}
  \caption{Capacitors connected in series}
  \label{fig:series-capacitors}
\end{figure}
The total voltage across these capacitors are the sum of the voltages across
each of them, i.e.\
\begin{equation}
  V_\text{tot}=V_1+V_2+V_3+\cdots
\end{equation}
The charge stored on all the capacitors must be the same. We can then write the
total voltage in terms of capacitance and charge:
\begin{equation}
  V_\text{tot}=\frac Q{C_1}+\frac Q{C_2}+\frac Q{C_3}+\cdots
\end{equation}
The inverse of the equivalent capacitance for $N$ capacitors connected in
series is the sum of the inverses of the individual capacitance.
\begin{important-equation}
  \frac1{C_s}=\sum_i\frac1{C_i}
\end{important-equation}
Increasing the number of capacitors in series effectively increases the plate
separation, therefore decreasing the capacitance.
  
%  \vspace{.1in}\textbf{Make sure we don't confuse ourselves with resistors.}


\textbf{How Do We Know That Charges Are The Same?} It's simple to show that the
charges across all the capacitors are the same
\begin{figure}[ht]
  \centering
  \begin{tikzpicture}[scale=1.5]
    \draw[thick](0,0) to[C=$C_1$,o-] (1.25,0) to[C=$C_2$,-o] (2.5,0);
    \draw[dashed](0.625,-.75) rectangle (1.875,1.2);
  \end{tikzpicture}
  \caption{The plates of two capacitor is a single conductor}
\end{figure}
The capacitor plates and the wire connecting them are really one piece of
conductor; there is nowhere for the charges to leave the conductor, therefore
when charges are accumulating on $C_1$, $C_2$ must also have the same charge
because of conservation of charges.



\section{$RC$ Circuits}

Now that we have seen how resistors and capacitors behave in a circuit, we can
look into combining them in to an ``$RC$ circuit''.
%  \begin{center}
%    \begin{tikzpicture}[american voltages,scale=.8]
%      \draw[thick] (0,0) to[C=$C$] (0,3)--(3,3) to[R=$R$] (3,0)--(0,0);
%    \end{tikzpicture}
%  \end{center}

%  \begin{itemize}
%  \item The voltage on the capacitor drives a current in the circuit.
%%  and
%%  then connect to a voltage source. Because of the nature of capacitors, the
%  \item The current through the circuit will not be steady
%    %as were the case with only  resistors.
%  \end{itemize}
%  \vspace{.2in}
%
%
%
\subsection{A Discharging Capacitor}

The simplest example of an $RC$ circuit is the discharging capacitor case,
where a capacitor and a resistor are connected in series. with the capacitor
initially charged, with the capacitor initially charged to a voltage $V_C$,
as shown in Fig.~\ref{fig:discharing-capacitor}. (The initial charge stored in
the capacitor would have been $+Q_0=CV$ on one side, and $-Q_0$ on another.)

\begin{figure}[ht]
  \centering
  \begin{tikzpicture}[american voltages,scale=.8]
    \draw[thick] (0,0) to[C=$C$] (0,3)--(3,3) to[R=$R$] (3,0)--(0,0);
  \end{tikzpicture}
  \caption{A simple discharging capacitor circuit}
  \label{fig:discharing-capacitor}
\end{figure}
Initially, the switch is open, and no current flows in the circuit. At this
time, the capacitor has an initial charge $Q_o$ across the plates, and therefore
a voltage of $V_0=Q_0/C$. At $t=0$, the switch is closed, and current---driven
by the voltage in the capacitor plates---begin to flow. The initial current
$I_0$ through the circuit is based on Ohm's law:
\begin{displaymath}
  I_0=\frac{V_0}R
\end{displaymath}
As current flows, charge $Q(t)$ across the capacitor decreases, therefore
the capacitor voltage $V_c(t)$ across the capacitor also decreases. As the
capacitor discharges, the currrent $I(t)=V_c(t)/R$ out of the capacitor (and
through the entire circuit) also decreases with time. But since current
decreases, $Q(t)$ also decreases more slowly. Using calculus, we find that the
expression of charge $Q(t)$ across the capacitor is an exponential decay
function:
\begin{important-equation}
  Q(t)=Q_0e^{-t/\tau}
\end{important-equation}  
where $Q_0$ is the initial charge on the capacitor, and $\tau=RC$ is called the
\textbf{time constant}. As $Q$ decreases, voltage $V_c$ across the capacitor
also decreases, and the current $I(t)$ through the circuit also decreases as an
exponential decay function:
\begin{important-equation}
  I(t)=I_0e^{-t/\tau}
\end{important-equation}
The initially current can be expressed using $Q_0$, $V_0$, $C$, $R$ or $\tau$:
\begin{displaymath}
  I_0=\frac{V_0}R=\frac{Q_0}{RC}=\frac{Q_0}\tau
\end{displaymath}



\subsection{Charging a Capacitor}

Before a capacitor can be discharged, it must first be charged. The simplest
form of such a circuit is shown in Fig.~\ref{fig:charging-capacitor}.
\begin{figure}[ht]
  \centering
  \begin{tikzpicture}[american voltages,scale=1.2]
    \draw[thick] (0,0) to[battery,l=$\mathcal E$] (0,2)
    to[R=$R$] (2,2) to[C=$C$] (2,0)--(0,0);
  \end{tikzpicture}
  \caption{A simple capacitor charged by a battery}
  \label{fig:charging-capacitor}
\end{figure}
The capacitor is initially uncharged $Q=0$, therefore the voltage across the
capacitor is also zero: $V_c=0$. At $t=0$, current begins to flow. There is no
voltage across the capacitor, so (by Kirchhoff's voltage law) the capacitor
acts like a short circuit (as if the capacitor is not there at all). The
initial current through the circuit is:
\begin{displaymath}
  I_0=\dfrac{\mathcal E}R
\end{displaymath}
All the energy in the circuit is dissipated by resistor. As current flows,
charges $Q(t)$ begin to accumulate on the capacitor. The voltage
$V_c(t)=Q(t)/C$ across the capacitor begins to increase. By the voltage law,
voltage drop across the resistor decreaes. Since voltage across the resistor
decreases, the current $I=V_R/R$ also decreases.

As $t\longrightarrow\infty$, the voltage across the capacitor:
$V_c\longrightarrow\mathcal E$, and the charge across the capacitor:
$Q(t)\longrightarrow C\mathcal E$. The current through the circuit:
$I\longrightarrow 0$ begin to accumulate on the capacitor.
%%\begin{itemize}

When charging the capacitor, the charge across the capacitor is given by:
\begin{important-equation}
  Q(t)=Q_\text{tot}(1-e^{-t/\tau})
\end{important-equation}
the time constant $\tau=RC$ is the same as the discharging case

%    \centering
%    \begin{tikzpicture}[american voltages]
%      \draw[thick] (0,0)
%      to[battery,l=$\mathcal E$] (0,2)
%      to[R=$R$] (2,2)--(2,0)--(0,0);
%    \end{tikzpicture}
%

%  \begin{center}
%    \begin{tikzpicture}[american voltages,scale=1.1]
%      \draw[thick] (0,0) to[battery,l=$\mathcal E$] (0,2)
%        to[R=$R$] (2,2) to[C=$C$] (2,0)--(0,0);
%    \end{tikzpicture}
%  \end{center}
The current $I$ through the circuit decreases as an exponential decay function,
same as the discharging case:
\begin{important-equation}
  I_c(t)=I_0e^{-t/\tau}
\end{important-equation}


%
%
%
%
%
%{Charging a Capacitor}

%%\item The voltage $V_c(t)=Q(t)/C$ across the capacitor begins to increase
%%\item By Kirchhoff's voltage law, voltage drop across the resistor decreaes
%%\item Since voltage across the resistor decreases, the current $I=V_R/R$
%%  decreases
%  \end{itemize}
%  \begin{center}
%    \vspace{-.1in}
%    \begin{tikzpicture}[american voltages,scale=1.1]
%      \draw[thick] (0,0) to[battery,l=$\mathcal E$] (0,2)
%      to[R=$R$] (2,2) to[C=$C$] (2,0)--(0,0);
%    \end{tikzpicture}
%  \end{center}
%  At this time, the capacitor acts like an open circuit.
%
%
%%    
%%
%%    The initial current $I_0=Q_\text{tot}/\tau=V/R$. This makes sense because
%%    $V_C=0$ at $t=0$, and all of the energy must be dissipated through the
%%    resistor. At $t=\infty$, current through the capacitor is $I_c=0$
%
%
%
%
From the charging capacitor case, we make two small but very important
observations:
\begin{enumerate}
\item When a capacitor is uncharged, there is no voltage across the plate,
  it acts like a short circuit.
\item When a capacitor is charged, there is a voltage across it, but no
  current flows \emph{through} it. Effectively it acts like an open circuit.
\end{enumerate}
We will use this information to solve more difficult $RC$ circuit problems



\subsection{General $RC$ Circuits}

In general, $RC$ circuits would involve multiple resistors and capacitor that
cannot be simplified into the configurations shown in previous sections.

\begin{figure}[ht]
  \centering
  \begin{tikzpicture}[scale=1.2,american voltages]
    \draw[thick] (0,0)
    to[battery,l=\SI{12}\volt] (0,2)
    to[R=\SI4\ohm](2,2)
    to[short,-*](2.46,2.3);
    \draw[thick](2.5,2) to[short,*-] (3,2) to[short] (4,2)
    to[C=\SI6{\micro\farad}] (4,0)--(0,0);
    \draw[thick] (3,0) to[R=\SI8\ohm] (3,2);
  \end{tikzpicture}
  \caption{Example of a general $RC$ circuit}
\end{figure}




%  \textbf{Example:} The capacitor in the circuit is initially uncharged. Find
%  the current through the battery
%  \begin{enumerate}
%  \item Immediately after the switch is closed
%  \item A long time after the switch is closed
%  \end{enumerate}


\begin{figure}[ht]
  \centering
  \begin{subfigure}{.49\textwidth}
    \centering
    \begin{tikzpicture}[scale=1.2,american voltages]
      \draw[thick] (0,0)
      to[battery,l=\SI{12}\volt] (0,2)
      to[R=\SI4\ohm](2,2)--(3,2) to[short] (4,2)--(4,0)--(0,0);
      \draw[thick] (3,0) to[R=\SI8\ohm] (3,2);
      \draw[line width=12,opacity=.2,yellow] (0,0)--(0,2)--(4,2)--(4,0)--(0,0);
    \end{tikzpicture}
    \caption{Initial circuit at $t=0$}
  \end{subfigure}
  \begin{subfigure}{.49\textwidth}
    \centering
    \begin{tikzpicture}[scale=1.2,american voltages]
      \draw[thick] (0,0)
      to[battery,l=\SI{12}\volt] (0,2)
      to[R=\SI4\ohm](2,2)--(3,2) to[short] (4,2) to[short,-o] (4,1.2);
      \draw[thick] (4,.8) to[short,o-] (4,0)--(0,0);
      \draw[thick] (3,0) to[R=\SI8\ohm] (3,2);
      \draw[line width=12,opacity=.2,yellow] (0,0)--(0,2)--(3,2)--(3,0)--(0,0);
    \end{tikzpicture}
    \caption{Final circuit as $t\longrightarrow0$}
  \end{subfigure}
  \caption{The initial and steady-state circuit diagram for a general RC
    circuit}
\end{figure}
