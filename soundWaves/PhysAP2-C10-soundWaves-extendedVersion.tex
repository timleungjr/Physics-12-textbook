%\documentclass[12pt,compress,aspectratio=169]{beamer}
%\input{../mybeamer}
%
%\newcolumntype{g}{>{\columncolor{gray!15}}c}

\chapter{Physics of Sound and Music}
\label{chapter:sound}
%\input{../me}
%\input{../mycommands}
%
%%\colorset{wave}{red!80!black}
%

%{This is the Extended Version}
%  This is an ``extended version''\footnote{The movie analogy of this is the
%    ``Director's Cut'' that somehow runs twice as long as the theatrical
%    release.} of the slides for Class 10, and it contains a lot of information
%  that is not covered in the course. Many of you, especially those with a
%  strong musical background, will find the information useful beyond sciences.
%\end{frame}
%
%
%
\section{Properties of Sound Waves}
%
%\subsection{Compression/Rarefaction}
%
%{Sound Wave}
%  \textbf{Sound wave}\footnote{Also known as an \textbf{acoustic wave}} is a:
%  \begin{itemize}
%  \item\textbf{mechanical wave}
%    \begin{itemize}
%    \item Requires a medium
%    \item Can propagate in a gas (e.g.\ air), a liquid, or a solid
%    \item Does not propagate in a vacuum
%    \end{itemize}
%  \item\textbf{longitudinal wave}
%    \begin{itemize}
%    \item The vibrations of the medium is \emph{parallel} to the direction of
%      travel of the wave
%    \end{itemize}
%  \item\textbf{pressure wave}
%    \begin{itemize}
%    \item The crests and troughs of the wave corresponds to regions in the
%      medium where the pressure is at the highest or lowest
%    \end{itemize}
%  \end{itemize}
%
%
%
%
%{Example: Speaker}
%  \begin{itemize}
%  \item Air molecules near the speaker is disturbed by the vibration of the
%    speaker cone
%  \item The vibrations cause the air molecules to compress and
%    expand, creating a wave that travels through air
%  \end{itemize}
%  \begin{center}
%    \pic{.7}{images/speaker}
%  \end{center}
%
%
%
%
%{Example: tuning Fork}
%  A tuning fork works in a similar principle as the speaker
%  \begin{itemize}
%  \item The vibrations of the tuning fork disturbs the air molecules, creating
%    compression and rarefaction regions that travels through air
%  \end{itemize}
%  \begin{center}
%    \pic{.7}{images/tuningfork}
%  \end{center}
%
%
%
%
%{Longitudinal Wave Simulation}
%  \begin{center}
%    \pic{.7}{images/Lwave-v8-0}\\
%    \href{https://ossfiles.ca/index.php/s/ec938CYeAo5LgtL}{[Link to Animation]}
%  \end{center}
%  This simulation was created by Professor Daniel A.\ Russell at the
%  Pensylvannia State University.
%  (\href{http://acs.psu.edu/drussell/demos.html}
%  {acs.psu.edu/drussell/demos.html})
%
%
%
\section{Speed of Sound}
%
%{Sound Wave in Dry Air}
%  We express the amplitude of the sound wave by plotting the change in air
%  pressure. The ``crest'' of the wave is where the pressure is at the highest;
%  the ``trough'' is where the pressure is the lowest; the ``rest position'' is
%  the atmospheric pressure
%  \begin{center}
%    \pic{.75}{images/schematic-vs-graph}
%  \end{center}
%
%
%
%
%{Speed of Sound in Air}
%  The speed of sound $v_s$ in dry air can be calculated by this equation:
%
%  \eq{-.1in}{
%    \boxed{
%      v_s=331.3\sqrt{1+\frac{T_C}{273.15}}=331.3\sqrt{\frac T{273.15}}
%    }
%  }
%
%  where $T_c$ is the temperature in \emph{degree Celsius}, and $T$ is the
%  absolute temperature in \emph{kelvin}. For temperatures near \SI0\celsius,
%  the speed of sound can be further simplified by linearization of the above
%  equation:
%
%  \eq{-.1in}{
%    \boxed{v_s=331.3+0.606T_C}
%  }  
%
%
%
%
%{Speed of Sound in a Gas}
%  The \emph{actual} equation for the speed of sound in an ideal gas is a
%  thermodynamic property that depends on its temperature and its
%  composition
%  
%  \eq{-.1in}{
%    v_s=\sqrt{\frac{\gamma RT}M}
%  }
%  \begin{center}
%    \begin{tabular}{l|c|c}
%      \rowcolor{pink}
%      \textbf{Quantity} & \textbf{Symbol} & \textbf{SI Unit} \\ \hline
%      Speed of sound            & $v_s$ & \si{\metre\per\second}\\
%      Thermodynamic temperature & $T$ & \si\kelvin \\
%      Universal gas constant    & $R$ & \si{\joule\per\mol.\kelvin}\\
%      Molar mass                & $M$ & \si{\kilo\gram\per\mol}\\
%      Adiabatic constant ($c_p/c_v$) & $\gamma$ & (no units)
%    \end{tabular}
%  \end{center}
%  For air, $\gamma=1.4$ and $M=\SI{29e-3}{\kilo\gram\per\mol}$
%
%
%
%
%{Speed of Sound in Liquids and Solids}
%  
%    Speed of sound in a liquid depends on the ``bulk modulus'' $K$ of the
%    liquid, and density $\rho$:
%      
%    \eq{-.13in}{
%      v_s = \sqrt{\frac K\rho}
%    }
%   
%    Speed of sound in a solid depends on the ``Young's modulus'' $E$ of the
%    solid and density $\rho$:
%
%    \eq{-.13in}{
%      v_s = \sqrt{\frac E\rho}
%    }
%
%    \centering
%    {\small
%      \begin{tabular}{l|c}
%        \rowcolor{cyan!30}
%        \textbf{Material} & \textbf{Speed} (\si{\metre\per\second}) \\
%        \rowcolor{pink!70}
%        \multicolumn{2}{c}{Gases (\SI0\celsius, \SI{101}{\kilo\pascal})} \\
%        Carbon dioxide & 259 \\
%        Oxygen         & 316 \\
%        Air            & 331 \\
%        Helium         & 965 \\
%        \rowcolor{pink!70}
%        \multicolumn{2}{c}{Liquids (\SI{20}\celsius)} \\
%        Ethanol        & 1162 \\
%        Fresh water    & 1482 \\
%        Seawater       & 1440-1500 \\
%        \rowcolor{pink!70}
%        \multicolumn{2}{c}{Solids} \\
%        Copper         & 5010 \\
%        Glass          & 5640 \\
%        Steel          & 5960
%      \end{tabular}
%    }

%
%
%
%%{Example Problem}
%%  \textbf{Example}: Suppose the room temperature of a classroom is
%%  \SI{21}\celsius. Calculate the speed of sound in the classroom.
%%
%%
%%
%%
%%{Example Problem}
%%  \textbf{Example}: The temperature was \SI{4.}{\celsius} one morning as
%%  Martita hiked through a canyon. She shouted at the canyon wall, and
%%  \SI{2.8}{\second} later heard an echo. How far away was the canyon wall?
%%
%
%
%
%{Refraction of Sound Wave}
%  When a sound wave enters region where the speed of sound changes, the wave
%  will also refract. This phenomen is similar to the refraction of light
%  waves\footnote{This was studied in the physics unit in Grade 10 Science!}
%  when it transmits from one medium to another.\footnote{The refraction of
%    sound wave follows a modified version of Snell's law.}
%  \begin{center}
%    \pic{.5}{images/TST_03_Figure3}
%  \end{center}
%
%
%
%\subsection{Loudness}
%
%{Intensity of Sound}
%  The average power transmitted $P_\text{avg}$ by a sound wave is proportional
%  to the square of the ``pressure amplitude'' $\Delta p$, which is the pressure
%  difference from atmospheric pressure\footnote{Although not explicitly
%    mentioned in the previous class, there is a homework question that shows
%    that power transferred by a wave on a string is also proportional to the
%    square of the amplitude}:
%
%  \eq{-.15in}{
%    P\propto(\Delta p)^2
%  }
%
%  \vspace{-.2in}The \textbf{loudness} of sound, which is the sound wave's
%  \textbf{intensity}, is the power of the sound wave divided by the area ($S$)
%  that the sound wave passes through:
%
%  \eq{-.1in}{
%    \boxed{I=\frac PS}%\propto\frac1{r^2}}
%  }
%  
%  \vspace{-.05in}When sound is emitted from a point source, the area that the
%  wavefront passes through is $S=4\pi r^2$ (surface area of a sphere), where
%  $r$ is the distance from the source.
%
%
%
%
%{Intensity of Sound}
%  As the sound wave travels outwards from a source, the surface area where the
%  wavefront pass through increases with $r^2$. Therefore, intensity decreases:
%
%  \eq{-.1in}{
%    I\propto\frac1{r^2}
%  }
%  \begin{center}
%    \pic{.38}{images/isqb}
%  \end{center}
%  As the intensity drops further away from the source, the amplitude $\Delta p$
%  of the sound wave also decreases
%
%
%
%
%{Threshold of Hearing}
%  The lowest detectable sound intensity is called the \textbf{threshold of
%    hearing} $I_0$, defined as:
%
%  \eq{-.1in}{
%    I_0=\SI{e-12}{\watt\per\metre\squared}
%  }
%  
%  While the sound intensity at the \textbf{threshold of pain} can damage a human
%  ear:
%
%  \eq{-.1in}{
%    I_p=\SI1{\watt\per\metre\squared}
%  }
%
%
%
%
%{Threshold of Hearing}
%  \begin{center}
%    \pic{.5}{images/earcrv2}
%  \end{center}
%  \begin{itemize}
%  \item\vspace{-.15in}The actual threshold of hearing is actually experimentally
%    determined to be about \SI{4e-12}{\watt\per\metre\squared} at
%    \SI{1000}\hertz
%  \item Maximum sensitivity to sound is \num{3500} to \SI{4000}\hertz,
%    corresponding to the resonance of the auditory canal
%  \end{itemize}
%
%
%
%
%{The Decibel}
%  The \textbf{decibel} is defined as the logarithm of the ratio intensity of
%  sound $I$ compared to the threshold of hearing $I_0$:
%  
%  \eq{-.1in}{
%    \beta=10\log_{10}\left[\frac I{I_0}\right]
%  }
%  \begin{itemize}
%  \item The threshold of hearing is \SI0{dB} while the threshold of pain is
%    \SI{120}{dB}
%  \item Humans perceive a doubling of loudness when intensity is increased by a
%    factor of 10 (i.e.\ \SI{10}{dB})
%  \end{itemize}
%
%
%
%
%{Sound Intensity in Decibels}
%  \centering
%  \begin{tabular}{l|c|c}
%    \rowcolor{pink}
%    \textbf{Source} & $I$ (\si{\watt\per\metre\squared}) & dB \\ \hline
%    Threshold of Hearing      & \num{e-12} & 0   \\
%    Rustling Leaves           & \num{e-11} & 10  \\
%    Whisper                   & \num{e-10} & 20  \\
%    Normal conversation       & \num{e-6}  & 60  \\
%    Busy street traffic       & \num{e-5}  & 70  \\
%    Vacuum cleaner            & \num{e-4}  & 80  \\
%    Large orchestra           & \num{6e-3} & 98  \\
%    MP3 player maximum Level  & \num{e-2}  & 100 \\
%    Front row of rock concert & \num{e-1}  & 110 \\
%    Military jet during take-off   & \num{e2} & 140 \\
%    Instant perforation of eardrum & \num{e4} & 160
%  \end{tabular}
%
%
%
%
%{Loudness}
%    \pic1{images/800px-Lindos4}
%    
%    Even though loudness is associated with the decibel reading, ``equal''
%    loudness depends on the frequency as well.
%    \begin{itemize}
%    \item The blue lines are known as the Fletcher-Munson curves; they
%      represents the original experimental results relating decibels to the
%      perception of loudness
%    \item The red line comes from more recent studies, and are considered to be
%      more accurate
%    \end{itemize}
%  
%
%
%
%\subsection{Frequency/Pitch}
%
%{Frequency and Pitch}
%  
%    \begin{itemize}
%    \item Frequencies of sound are referred to as its \textbf{pitch}
%    \item Audible range for an adult human is approximately \SI{20}{\hertz} to
%      \SI{20000}\hertz
%    \item\textbf{Infrasound}: frequencies below audible range
%    \item\textbf{Ultrasound}: frequencies above audible range, e.g.
%      \begin{itemize}
%      \item Dog whistles
%      \item Medical ultrasound devices
%      \end{itemize}
%    \end{itemize}
%
%    \pic1{images/ultrasound}\\
%    {\footnotesize Medical ultrasound devices usually use frequencies between
%    \SI1{\mega\hertz} to \SI{20}{\mega\hertz},depending on its application\par}
%  
%
%
%
%
\section{Doppler Effect}
%
%\subsection{Mach Number}
%  Speeds close to the speed of sound are often expressed in terms of its ratio
%  to the speed of sound. This is called the \textbf{Mach number} ($M$):
%  
%  \eq{-.1in}{
%    \boxed{M=\frac v{v_s}}
%  }
%  \begin{center}
%    \begin{tabular}{l|c|c}
%      \rowcolor{pink}
%      \textbf{Quantity} & \textbf{Symbol} & \textbf{SI Unit} \\ \hline
%      Mach number          & $M$   & (no units) \\
%      Speed of the object  & $v$   & \si{\metre\per\second} \\
%      Local speed of sound & $v_s$ & \si{\metre\per\second}
%    \end{tabular}
%  \end{center}
%  \begin{itemize}
%  \item Subsonic: $M<1$
%  \item Supersonic: $1<M<5$
%  \item Hypersonic: $M\geq5$
%  \end{itemize}
%
%
%
%
%{Sound from a Stationary Source}
%  
%    \begin{tikzpicture}[scale=.55]
%      \begin{scope}[very thick]
%        %\foreach \x in {1,...,4} \draw[red!{(5-\x)*20}] circle(\x);
%        \draw[red!80] circle(1);
%        \draw[red!60] circle(2);
%        \draw[red!40] circle(3);
%        \draw[red!20] circle(4);
%      \end{scope}
%      \fill circle (.075) node[right]{source};
%      \draw[<->] (-1,0)--(-2,0) node[midway,below]{$\lambda$};
%      \foreach\theta in {0,20,...,340}
%      \draw[axes,rotate=\theta] (3.7,0)--(4.7,0);
%    \end{tikzpicture}
%
%    When a sound is emitted from a stationary point source, the sound wave moves
%    radially outward from the origin:
%    \begin{itemize}
%    \item Loudness (intensity) drops farther away from the source, proportional
%      to $\dfrac1{r^2}$
%    \item All points hear the same wavelength and frequency (pitch) of sound
%    \end{itemize}
%  
%
%
%
%
%{Sound from a Moving Source}

%
%    \begin{tikzpicture}[scale=.6]
%      \begin{scope}[very thick]
%        \draw[red!20] circle(4);
%        \draw[red!40] (0.5,0) circle (3);
%        \draw[red!60] (1.0,0) circle (2);
%        \draw[red!80] (1.5,0) circle (1);
%      \end{scope}
%      \draw[vectors] (0,0)--(2.7,0) node[above]{$v$};
%      \fill circle (.075) node[below]{1};
%      \fill (0.5,0) circle (.075) node[below]{2};
%      \fill (1.0,0) circle (.075) node[below]{3};
%      \fill (1.5,0) circle (.075) node[below]{4};
%      \draw[<->] (-4,0)--(-2.5,0) node[midway,above]{$\lambda_1$};
%      \draw[<->] (4,0)--(3.5,0);
%      \node (a) at (5,-1) {$\lambda_2$};
%      \draw (3.75,-.1) to[out=270,in=180] (a);
%    \end{tikzpicture}
%      
%    When sound is emitted from a subsonic \emph{moving} source, the diagram
%    looks different. In this case, the sound source is moving to the right,
%    from 1 to 4
%    \begin{itemize}
%    \item When the source is moving \emph{towards the observer}, the
%      wavelength $\lambda_2$ decreases, and the apparent frequency increases.
%    \item When the source is moving \emph{away from the observer}, wavelength
%      $\lambda_1$ increases, and the apparent frequency decreases.
%    \end{itemize}
%    This is called the \textbf{Doppler effect}.
%  
%
%
%
%
%{Doppler Effect}
%  We experience Doppler effect every time an ambulance speeds by us with its
%  sirens on.
%  \begin{center}
%    \pic{.6}{images/toronto-ambulance}
%  \end{center}
%  When it is moving towards us, the pitch of the siren is high, but
%  the moment it passes us, the pitch decreases.
%
%
%
%
%{Doppler Effect}
%  When a wave source is moving at a speed $v_\text{src}$ and an observer is
%  moving at $v_\text{ob}$, the perceived frequency $f'$ is shifted:
%
%  \eq{-.1in}{
%    \boxed{f'=\frac{v_s+v_\text{ob}}{v_s-v_\text{src}}f}
%  }
%  \begin{center}
%    \begin{tabular}{l|c|c}
%      \rowcolor{pink}
%      \textbf{Quantity} & \textbf{Symbol} & \textbf{SI Unit}\\
%      Perceived frequency & $f'$  & \si\hertz \\
%      Source frequency    & $f$   & \si\hertz \\
%      Speed of sound      & $v_s$ & \si{\metre\per\second}\\
%      Speed of source & $v_\text{src}$ & \si{\metre\per\second}\\
%      Speed of observer & $v_\text{ob}$ & \si{\metre\per\second}
%    \end{tabular}
%  \end{center}
%  $v_\text{src}$ and $v_\text{ob}$ are positive when the move towards each
%  other, and negative when they move away.
%
%
%
%
%\subsection{Sonic Boom}
%
%{Sound Source at Sonic Speed}
%  
%    \begin{tikzpicture}[scale=.55]
%      \begin{scope}[very thick]
%        \draw[red!20] circle (4);
%        \draw[red!40] (1,0) circle (3);
%        \draw[red!60] (2,0) circle (2);
%        \draw[red!80] (3,0) circle (1);
%      \end{scope}
%      \draw[vectors] (0,0)--(4.5,0) node[right]{$v=v_s$};
%      \fill circle (.075) node[below]{1};
%      \fill (1,0) circle (.075) node[below]{2};
%      \fill (2,0) circle (.075) node[below]{3};
%      \fill (3,0) circle (.075) node[below]{4};
%      \fill (4,0) circle (.075) node[below]{5};
%      \draw[thick,dashed] (4,7)--(4,-4)
%      node[pos=.05,right]{\scriptsize No disturbance in front of the shock}
%      node[pos=.05,left] {\scriptsize Disturbed air flow behind the shock};
%    \end{tikzpicture}
%      
%    When sound source is moving at the speed of sound ($M=1$):
%    \begin{itemize}
%    \item Wavefronts are bunched up just in front of the source
%    \item Since sound wave is a pressure wave, right in front of the sound
%      source, there is a large change in pressure (called a \textbf{shock
%        wave})
%    \item When the shock passes an observer, a loud bang can be heard (aka
%      \textbf{sonic boom})
%    \end{itemize}
%  
%
%
%
%
%{Sound from a Supersonic Source}
%  
%    \begin{tikzpicture}[scale=.55]
%      \fill (3.5,-3) circle (.1) node[right]{observer};
%      \begin{scope}[very thick]
%        \draw[red!20] circle (4);
%        \draw[red!40] (1.8,0) circle (3);
%        \draw[red!60] (3.6,0) circle (2);
%        \draw[red!80] (5.4,0) circle (1);
%      \end{scope}
%      \fill circle (.075) node[below]{1};
%      \fill (1.8,0) circle (.075) node[below]{2};
%      \fill (3.6,0) circle (.075) node[below]{3};
%      \fill (5.4,0) circle (.075) node[below]{4};
%      \fill (7.2,0) circle (.075) node[below]{5};
%
%      \draw[->] (-3.5,0)--(9,0) node[above left]{$v>v_s$};
%      \draw[dashed,thick,rotate around={213.8:(7.2,0)}] (7.2,0)--(15.5,0);
%      \draw[dashed,thick,rotate around={146.3:(7.2,0)}] (7.2,0)--(17.5,0);
%      \draw[axes] (-2.8,0) arc (180:146.3:10) node[pos=.4,right]{$\gamma$};
%    \end{tikzpicture}
%      
%    When sound source is moving at $M>1$, it out runs the sound that it makes:
%    An \emph{oblique shock} is formed at an angle (called the
%    \textbf{Mach angle}) given by:
%    
%    \eq{-.1in}{
%      \gamma=\sin^{-1}\left(\frac 1M\right)
%    }
%
%    An observer does not hear the sound source until it has gone past!
%  
%
%
%
%
%{Bullet in Supersonic Flight}
%  Generating a shock does not require an actual sound source. Any object
%  moving through air creates a pressure disturbance. For example, a bullet in
%  supersonic flight generates shock waves as it moves through air.
%  
%    \begin{itemize}
%    \item The flow around this bullet is taken inside a \emph{shock tube} that
%      generates a short burst of supersonic flow. A high-speed camera is used to
%      take the photo.
%    \item The Mach number changes as air flows around the bullet
%    \end{itemize}
%
%    \begin{tikzpicture}
%      \node at (0,0) {\pic{.9}{images/bullet2}};
%      \node[thick,magenta,draw=magenta,fill=white] (A) at (2.6,1.4){
%        \scriptsize Shock};
%      \draw[very thick,magenta,->] (A)--(.9,1.4);
%    \end{tikzpicture}
%  
%
%
%
%
%{Duck in Water}
%  A similar shock behaviour is observed when the duck swims in water, because
%  the duck swims faster than the speed of the water wave, it also creates a
%  cone shape.
%  \begin{center}
%    \begin{tikzpicture}
%      \node at (0,0) {\pic{.6}{images/duck}};
%      \node[thick,draw=blue,fill=white]
%      at (2.3,.8) {\scriptsize Undisturbed Water};
%      \node[thick,draw=blue,fill=white,above]
%      (A) at (-2,1.5) {\scriptsize Shock Wave};
%      \draw[very thick,blue,->] (A)--(-2,.3);
%    \end{tikzpicture}
%  \end{center}
%
%
%
%
%\section{Beats}
%
%{Visualizing Beat Frequency}
%  Two waves ({\color{magenta}$\Psi_1$} and {\color{cyan}$\Psi_2$}) moving in the
%  same medium\footnote{The two waves would be travelling at the same speed.}
%  but with different frequencies\footnote{Think of two musical instruments
%    playing out of tune from each other} go through regions of constructive and
%  destructive interference:
%  \begin{center}
%    \begin{tikzpicture}[yscale=.66]
%      \begin{scope}[->]
%        \draw (0,0) --(11,0) node[right]{$x$};
%        \draw (0,-5)-- (0,2) node[right]{$y$};
%        \draw (0,-3) --(11,-3) node[right]{$x$};
%      \end{scope}
%      \draw[vectors] (4,1.5)--(5,1.5) node[right]{$v$};
%      \begin{scope}[smooth,samples=200,domain=0:10,thick]
%        \draw[cyan] plot(\x,{sin(700*\x)});
%        \draw[magenta] plot(\x,{sin(770*\x)});
%        \draw[violet] plot(\x,{.8*(sin(700*\x)+sin(770*\x))-3});
%      \end{scope}
%      \begin{scope}[violet,thick,dash dot,smooth,samples=30,domain=0:10]
%        \draw plot(\x,{ 1.6*cos(35*\x)-3});
%        \draw plot(\x,{-1.6*cos(35*\x)-3});
%      \end{scope}
%      \node[right] at (10,1){\color{magenta}$\Psi_1$};
%      \node[right] at (10,.4){\color{cyan}$\Psi_2$}; 
%      \node[right] at (10,-2){\color{violet}$\Psi=\Psi_1+\Psi_2$}; 
%      \foreach \x in {90,180,270} \draw[gray] (\x/35,-5)--(\x/35,1.5);
%    \end{tikzpicture}
%  \end{center}
%
%
%
%
%{Beat Frequency}
%  \textbf{Beat frequency} is the absolute value of the difference of the
%  frequencies of the two component waves. At low frequencies, they sound like a
%  pulsating ``whoomf''\footnote{If the beat frequency is within the audible
%    range, it will be interpreted as an actual sound}. Musicians often use beat
%  frequencies to determine if someone is playing in tune.
%
%  \eq{-.1in}{
%    \boxed{f_b=|f_2-f_1|}
%  }
%
%  \vspace{-.1in}
%  \begin{center}
%    \begin{tabular}{l|c|c}
%      \rowcolor{pink}
%      \textbf{Quantity} & \textbf{Symbol} & \textbf{SI Unit} \\ \hline
%      Beat frequency              & $f_b$ & \si\hertz \\
%      Frequencies component waves & $f_1$, $f_2$ & \si\hertz
%    \end{tabular}
%  \end{center}
%  For those who are interested, the full derivation of the beat frequency can
%  be found on the accompanied handout for this class, but it is not a
%  requirement for Physics 11.
%
%
%
%
%\section{Harmonics}
%
%{Music vs.\ Noise}
%  Difference between \emph{noise} and \emph{music} is often difficult to
%  distinguish\footnote{Just ask your parents!}. Generally, the concept of music
%  is based on:
%  \begin{itemize}
%  \item Organized combinations of different frequencies
%  \item Harmonics of dominated frequencies, which are
%  \item Whole-number multiples of the lowest (fundamental) frequency
%  \end{itemize}
%
%
%
%
\subsection{Harmonic Waves \& Fourier Series}
Not all harmonic waves have a sinusoidal shape, because \emph{all} periodic
functions $U(x)$ with a period of $P$ \footnote{You can think of $P$ as the
``spatial period'', which is related to the wavelength of a wave.} are
actually infinite sums of sine and/or cosine functions, called the
\textbf{Fourier series}:
%
%  \eq{-.1in}{
%    U(x)= \frac{a_0}2+
%    \sum_{n=1}^\infty a_n\sin\left(\frac{2\pi n}P x\right)+
%    \sum_{n=1}^\infty b_n\cos\left(\frac{2\pi n}P x\right)
%  }
%%    \begin{align*}
%%      U(x)=&a_1\sin(x)+a_2\sin(2x)+a_3\sin(3x)+\cdots+\\
%%      &b_1\cos(x)+b_2\cos(2x)+b_3\cos(3x)+\cdots\\
%%      =&\sum_{n=1}^\infty a_n\sin(nx)+\sum_{n=1}^\infty b_n\cos(nx)
%%    \end{align*}
%%  }

Depending on the shape of the harmonic wave, some coefficients $a_n$ and $b_n$
are zeros. Note that $n$ is an integer. For example, this harmonic wave is a
composite of 5 sine waves all travelling at the same speed. (The coefficients
are arbitrarily chosen.) The wavelength is based on the component wave with the
lowest frequency.

%    \begin{displaymath}
%      u(x)=\sin(50x)+\frac25\sin(100x) +\frac1{10}\sin(150x)
%      +\frac15\sin(200x) +\frac3{10}\sin(250x)
%    \end{displaymath}

%  \begin{center}
%    \begin{tikzpicture}[xscale=.4]
%      \draw[thick] (-1,0)--(21,0);
%      \draw[smooth,samples=500,domain=-1:21,very thick,red]
%      plot(\x, {sin(50*\x)+.4*sin(100*\x)+.1*sin(150*\x)
%        +.2*sin(200*\x)+.3*sin(250*\x)});
%      \draw[|<->|] (7.75,1.5)--(14.9,1.5) node[midway,fill=white]{$\lambda$};
%      \draw[->,very thick,red] (22,.6)--(25,.6)
%      node[midway,above,black]{Direction of wave travel};
%    \end{tikzpicture}
%  \end{center}
%
%
%
%
%{Musical Instruments}
%  When a musical instrument produces a sound at a certain frequency, it also
%  produces many higher frequency sounds\footnote{the reason for this will
%    become apparent later in the slides}
%  \begin{itemize}
%  \item The higher frequency sounds are \emph{whole-number multiples} of the
%    lowest frequency
%  \item e.g.\ a violin playing at \SI{440}{\hertz} produces sound waves at
%    \begin{center}
%      {\Large
%        \vspace{.1in}
%        \SI{440}\hertz, \SI{880}\hertz, \SI{1320}\hertz, \SI{1760}\hertz,
%        \SI{2200}\hertz, \SI{2640}\hertz\ldots
%        \vspace{.1in}
%      }
%    \end{center}
%  \item Generally, the higher the frequency, the smaller the amplitude
%  \item The overall quality of the sound comes from the sum of all the waves
%    (principle of superposition). This is why a violin and a trumpet playing
%    the same note will always sound different
%  \end{itemize}
%
%
%
%
%[t]{Harmonic Frequencies}
%  The sound wave with the longest wavelength (and therefore lowest frequency)
%  is called the \textbf{fundamental frequency}, the \textbf{first partial}, or
%  the \textbf{first harmonic}. It is the $n=1$ term in the Fourier series shown
%  a few slides ago:
%  \begin{center}
%    \begin{tikzpicture}
%      \draw[->] (1,1.3)--(1.5,1.3) node[right]{$v$};
%      \draw[thick,blue,smooth,samples=30,domain=0:2.8] plot(\x,{sin(200*\x)});
%    \end{tikzpicture}
%    \hspace{.15in}
%    \begin{tikzpicture}
%      \draw[->] (1,1.3)--(1.5,1.3) node[right]{$v$};
%      \draw[thick,red!20,smooth,samples=50,domain=0:2.8] plot(\x,{sin(400*\x)});
%    \end{tikzpicture}
%    \hspace{.15in}
%    \begin{tikzpicture}
%      \draw[->] (1,1.3)--(1.5,1.3) node[right]{$v$};
%      \draw[thick,green!20,smooth,samples=70,domain=0:2.8]
%      plot(\x,{sin(600*\x)});
%    \end{tikzpicture}
%  \end{center}
%  Generally, when a musical instrument produces a sound, the fundamental
%  frequency is the one that is ``heard''
%
%
%
%
%[t]{Harmonic Frequencies}
%  A second wave is also created with half the wavelength and twice the
%  frequency. It's called the \textbf{second harmonic}, \textbf{second partial},
%  or the \textbf{first overtone}.
%  \begin{center}
%    \begin{tikzpicture}
%      \draw[->] (1,1.3)--(1.5,1.3) node[right]{$v$};
%      \draw[thick,blue!20,smooth,samples=20,domain=0:2.8]
%      plot(\x,{sin(200*\x)});
%    \end{tikzpicture}
%    \hspace{.15in}
%    \begin{tikzpicture}
%      \draw[->] (1,1.3)--(1.5,1.3) node[right]{$v$};
%      \draw[thick,red,smooth,samples=50,domain=0:2.8] plot(\x,{sin(400*\x)});
%    \end{tikzpicture}
%    \hspace{.15in}
%    \begin{tikzpicture}
%      \draw[->] (1,1.3)--(1.5,1.3) node[right]{$v$};
%      \draw[thick,green,smooth,samples=70,domain=0:2.8]
%      plot(\x,{sin(600*\x)});
%    \end{tikzpicture}
%  \end{center}
%  Beyond the 2nd harmonic, there are also the 3rd harmonic (2nd overtone),
%  4th harmonic (3rd overtone)\ldots etc. Whole-number multiples of the
%  fundamental frequency $f_1$ are its \textbf{harmonic frequencies}, and the
%  $n$-th harmonic is:
%
%  \eq{-.15in}{
%    \boxed{f_n=nf_1}\quad\quad n=1,2,3,\ldots
%  }
%
%
%
%
%{Different Musical Instruments}
%  Wave forms from different musical instruments at \SI{440}\hertz:
%  \begin{center}
%    \pic{.6}{images/F_InstrumentWaves}
%  \end{center}
%  The graph on the right correspond to the amplitudes at different frequencies.
%  Note the peaks at regular intervals. Those are the harmonic frequencies.
%
%
%
%
%{Fourier Analysis and Synthesis}
%  \begin{center}
%    \pic{.55}{images/Fourier-xform}
%  \end{center}
%
%
%
%
\section{Resonance Modes}
%
%\subsection{Strings}
%
%{Standing Waves on Strings}
%  \textbf{Resonance modes} are frequencies where a standing wave can be
%  established. For a string of length $L$, the condition of a standing wave is
%  that both ends of the string must be nodes. The fundamental mode (lowest
%  frequency) occurs at $\lambda=2L$:
%  \begin{center}
%    \begin{tikzpicture}[scale=1.5]
%      \draw[thick] (0,-.1)--(0,.1);
%      \draw[thick] (pi,-.1)--(pi,.1);
%      \draw[thick] (0,0)--(pi,0);
%      \begin{scope}[smooth,samples=20]
%        \draw[domain=0:pi,functions] plot(\x,{.4*sin(180/pi*\x)});
%        \draw[domain=pi:2*pi,dashed] plot(\x,{.4*sin(180/pi*\x)});
%        \draw[domain=0 :2*pi,dashed] plot(\x,{-.4*sin(180/pi*\x)});
%      \end{scope}
%      \draw[|<->|] (0,-.6)--(2*pi,-.6) node[midway,fill=white]{$\lambda=2L$};
%      \fill circle (.04) node[below]{N};
%      \fill (pi/2,0) circle (.04) node[below]{A};
%      \fill (pi,0) circle (.04) node[below]{N};
%    \end{tikzpicture}
%  \end{center}
%  
%  \vspace{-.1in}The fundamental frequency can be calculated based on the
%  relationship $v_\text{str}=f\lambda$:
%
%  \eq{-.1in}{
%    \boxed{
%      f_1=\frac{v_\text{str}}\lambda=\frac{v_\text{str}}{2L}
%      \quad\text{\normalsize where}\quad v_\text{str}=\sqrt{\frac{F_TL}m}
%    }
%  }
%
%
%
%
%{Standing Waves On a String of Length $L$}
%  The second resonance mode occurs when $\lambda=L$:
%
%    \centering
%    \begin{tikzpicture}[scale=1.5]
%      \draw[thick] (0,0)--(pi,0);
%      \draw[thick] (0,-.1)--(0,.1);
%      \draw[thick] (pi,-.1)--(pi,.1);
%      \begin{scope}[smooth,samples=20,domain=0:pi]
%        \draw[functions] plot(\x,{.4*sin(360/pi*\x)});
%        \draw[dashed] plot(\x,{-.4*sin(360/pi*\x)});
%      \end{scope}
%      \fill (0,0) circle (.04) node[below]{N};
%      \fill (pi,0) circle (.04) node[below]{N};
%      \fill (pi/2,0) circle (.04) node[below]{N};
%      \fill (pi/4,0) circle (.04) node[below]{A};
%      \fill (3*pi/4,0) circle (.04) node[below]{A};
%    \end{tikzpicture}
%    
%    \eq{-.01in}{
%      f_2 = \frac v\lambda=\frac vL=2f_1
%    }
%  
%
%  The third resonance mode occurs at $\lambda=\dfrac23L$:
%
%  
%    \centering
%    \begin{tikzpicture}[scale=1.5]
%      \draw[thick] (0,0)--(pi,0);
%      \draw[thick] (0,-.1)--(0,.1);
%      \draw[thick] (pi,-.1)--(pi,.1);
%      \begin{scope}[smooth,domain=0:pi]
%        \draw[samples=20,functions] plot(\x,{.4*sin(540/pi*\x)});
%        \draw[samples=30,dashed] plot(\x,{-.4*sin(540/pi*\x)});
%      \end{scope}
%      \foreach \x in {0,pi/3,pi,2*pi/3}
%      \fill (\x,0) circle (.04) node[below]{N};
%      %\fill (0,0) circle (.04) node[below]{N};
%      %\fill (pi,0) circle (.04) node[below]{N};
%      %\fill (pi/3,0) circle (.04) node[below]{N};
%      %\fill (2*pi/3,0) circle (.04) node[below]{N};
%      \fill (pi/6,0) circle (.04) node[below]{A};
%      \fill (pi/2,0) circle (.04) node[below]{A};
%      \fill (5*pi/6,0) circle (.04) node[below]{A};
%    \end{tikzpicture}
%    

%    \eq{-.01in}{
%      f_3=\frac v\lambda=\frac{3v}{2L}=3f_1
%    }
%  
%
%
%
%
%{Standing Waves On a String of Length $L$}
%  The $n$-th resonance mode of a wave on string is given by:
%
%  \eq{-.1in}{
%    \boxed{f_n=nf_1}\quad
%    \text{\normalsize where}\quad
%    \boxed{f_1=\frac v{2L}}
%  }
%  \begin{itemize}
%  \item $n=1,2,3,\ldots$ is a whole-number multiple
%  \item This equation is \emph{identical} to the equation for harmonic
%    frequencies, meaning that on a string, every harmonic is a resonance
%    frequency
%  \item A vibrating string is said to have a ``full set of harmonics''
%  \end{itemize}
%
%
%
%
%
\subsection{Closed Pipes}

%{Standing Waves in a Closed Pipe of Length $L$}
Similar standing-wave patterns can be found on pipes with both ends closed.
Like strings, the boundary condition is that the closed ends of the pipes
must be nodes.
%
%  \vspace{.2in}
%  
%    \centering
%    \begin{tikzpicture}[scale=1.3,yscale=.45]
%      \draw[thick] (0,-1) rectangle (pi,1);
%      \foreach \x in {0,pi} \node[below,fill=yellow!40] at (\x,-1.1){N};
%      \node[below,fill=pink!40] at (pi/2,-1.1){A};
%      \begin{scope}[smooth,samples=20,domain=0:pi]
%        \draw[functions] plot(\x,{sin(180/pi*\x)});
%        \draw[dashed] plot(\x,{-1*sin(180/pi*\x)});
%      \end{scope}
%      \node[above] at (pi/2,1){Fundamental Mode};
%    \end{tikzpicture}
%
%    \eq{-.25in}{
%        f_1=\frac{v_s}\lambda=\frac{v_s}{2L}
%    }
%  
%    \centering
%    \begin{tikzpicture}[scale=1.3,yscale=.45]
%      \draw[thick] (0,-1) rectangle (pi,1);
%      \foreach \x in {0,pi/2,pi} \node[below,fill=yellow!40] at(\x,-1.1){N};
%      \foreach \x in {pi/4,3*pi/4} \node[below,fill=pink!40] at(\x,-1.1){A};
%      \begin{scope}[smooth,samples=20,domain=0:pi]
%        \draw[functions] plot(\x,{sin(360/pi*\x)});
%        \draw[dashed] plot(\x,{-1*sin(360/pi*\x)});
%      \end{scope}
%      \node[above] at (pi/2,1){2nd Resonance Mode};
%    \end{tikzpicture}
%
%    \eq{-.25in}{
%      f_2=
%      \frac{v_s}\lambda=\frac{v_s}L=2f_1
%    }
%    

%    \centering
%    \begin{tikzpicture}[scale=1.3,yscale=.45]
%      \draw[thick] (0,-1) rectangle (pi,1);
%      \foreach \x in {0,pi/3,2*pi/3,pi}
%      \node[below,fill=yellow!40] at (\x,-1.1){N};
%      \foreach \x in {pi/6,pi/2,5*pi/6}
%      \node[below,fill=pink!40] at(\x,-1.1){A};
%      \begin{scope}[smooth,samples=20,domain=0:pi]
%        \draw[functions] plot(\x,{sin(540/pi*\x)});
%        \draw[dashed] plot(\x,{-1*sin(540/pi*\x)});
%      \end{scope}
%      \node[above] at (pi/2,1){3rd Resonance Mode};
%    \end{tikzpicture}
%
%    \eq{-.25in}{
%      f_3=\frac{3v_s}{2L}=3f_1
%    }
%    
%
%
%
%
%{Standing Waves in a Closed Pipe of Length $L$}
For pipes that are \emph{closed at both ends}, the $n$-th resonance mode is
given by:
%  
%  \eq{-.1in}{
%    \boxed{f_n=nf_1}\quad\text{\normalsize where}\quad
%    \boxed{f_1=\frac{v_s}{2L}}
%    \quad n=1,2,3,\ldots
%  }
%  \begin{itemize}
%  \item The above equation is \emph{identical} to the equation for harmonic
%    frequencies, i.e.\ every harmonic is a resonance mode
%  \item The difference between a closed pipe and a string is that the wave
%    speed is the speed of sound $v_s$ inside the pipe
%  \end{itemize}
%  No musical ``instruments'' are built this way, but you can use this to model
%  standing wave patterns inside a concert hall.
%
%
%
\subsection{Open Pipes}

%{Standing Waves in an Open Pipe of Length $L$}
Some organ pipes and flute have open ends on both sides. In this case, the
open ends are anti-nodes. The standing-wave pattern is similar to the closed
pipes, but the locations of the nodes and anti-nodes are reversed.
%  \begin{center}
%    \begin{tikzpicture}[scale=1.4,yscale=.8]
%      \draw[thick] (0,-.5)--(pi,-.5);
%      \draw[thick] (0,0.5)--(pi,0.5);
%      \foreach \x in {0,pi} \node[below,fill=pink!50] at (\x,-.55){A};
%      \node[below,fill=yellow!50] at (pi/2,-.55){N};
%      \begin{scope}[smooth,domain=0:pi,samples=15]
%        \draw[functions] plot(\x,{.5*sin(180/pi*\x+90)});
%        \draw[dashed] plot(\x,{-.5*sin(180/pi*\x+90)});
%      \end{scope}
%      \node[above] at (pi/2,.5){Fundamental Mode};
%    \end{tikzpicture}
%    \begin{tikzpicture}[scale=1.4,yscale=.8]
%      \draw[thick] (0,-.5)--(pi,-.5);
%      \draw[thick] (0,0.5)--(pi,0.5);
%      \foreach \x in {0,pi/2,pi} \node[below,fill=pink!50] at (\x,-.55){A};
%      \foreach \x in {pi/4,3*pi/4} \node[below,fill=yellow!50] at (\x,-.55){N};
%      \begin{scope}[smooth,samples=20,domain=0:pi]
%        \draw[functions] plot(\x,{.5*sin(360/pi*\x+90)});
%        \draw[dashed] plot(\x,{-.5*sin(360/pi*\x+90)});
%      \end{scope}
%      \node[above] at (pi/2,.5){2nd Resonance Mode};
%    \end{tikzpicture}
%    \begin{tikzpicture}[scale=1.4,yscale=.8]
%      \draw[thick](0,-.5)--(pi,-.5);
%      \draw[thick](0,0.5)--(pi,0.5);
%      \foreach\x in {0,pi/3,2*pi/3,pi}\node[below,fill=pink!50]at(\x,-.55){A};
%      \foreach\x in {pi/6,pi/2,5*pi/6}\node[below,fill=yellow!50]at(\x,-.55){N};
%      \begin{scope}[smooth,samples=30,domain=0:pi]
%        \draw[functions] plot(\x,{.5*sin(540/pi*\x+90)});
%        \draw[dashed] plot(\x,{-.5*sin(540/pi*\x+90)});
%      \end{scope}
%      \node[above] at (pi/2,.5){3rd Resonance Mode};
%    \end{tikzpicture}
%  \end{center}
%  Like strings and closed pipes, open pipes also have a ``full set of
%  harmonics''. The $n$-th resonance mode is given by:
%  
%  \eq{-.15in}{
%    \boxed{f_n=nf_1}
%    \quad n=1,2,3,\ldots
%    \quad\text{\normalsize where}
%    \quad\boxed{f_1=\frac{v_s}{2L}}
%  }
%
%
%
%
%%{Example Problem}
%%  \textbf{Example}: An air column, open at both ends, has a fundamental
%%  resonance mode at \SI{330}\hertz.
%%  \begin{enumerate}
%%  \item What are the frequencies of the second and third resonance modes?
%%  \item If the speed of sound in air is \SI{344}{\metre\per\second}, what is the
%%    length of the air column?
%%  \end{enumerate}



\subsection{Semi-Open Pipes}
%{Standing Waves in a Semi-Open Pipe of Length $L$}
However, not all configurations have a full set of harmonics. Most organ pipes,
as well as most woodwind (e.g.\ clarinet, oboe, bassoon) and brass instruments
(e.g.\ French horn, trumpet, trombone) have one closed end and one open end.
\begin{itemize}
\item The closed end is a node, while
\item The open end is an antinode
\end{itemize}
\begin{figure}[ht]
  \centering
  \begin{tikzpicture}[xscale=1.4]
    \draw[thick](pi,-.5)--(0,-.5)node[pos=0,below=1,fill=pink!50]{A}
    node[below=1,fill=yellow!50]{N}--(0,.5)--(pi,.5);
    \begin{scope}[smooth,samples=10,domain=0:pi]
      \draw[functions] plot(\x,{.5*sin(90/pi*\x)});
      \draw[dashed] plot(\x,{-.5*sin(90/pi*\x)});
    \end{scope}
    \node[above] at (pi/2,.5){Fundamental Mode};
  \end{tikzpicture}
  \begin{tikzpicture}[xscale=1.4]
    \draw[thick] (pi,-.5)--(0,-.5)--(0,.5)--(pi,.5);
    \foreach \x in {0,2*pi/3}\node[below,fill=yellow!50] at (\x,-.55){N};
    \foreach \x in {pi/3,pi} \node[below,fill=pink!50] at (\x,-.55){A};
    \begin{scope}[smooth,samples=20,domain=0:pi]
      \draw[functions] plot(\x,{.5*sin(270/pi*\x)});
      \draw[dashed] plot(\x,{-.5*sin(270/pi*\x)});
    \end{scope}
    \node[above] at (pi/2,.5){2nd Resonance Mode};
  \end{tikzpicture}
  \begin{tikzpicture}[xscale=1.4]
    \draw[thick] (pi,-.5)--(0,-.5)--(0,.5)--(pi,.5);
    \foreach \x in {0,2*pi/5,4*pi/5} \node[below,fill=yellow!50] at(\x,-.55){N};
    \foreach \x in {pi/5,3*pi/5,pi} \node[below,fill=pink!50] at (\x,-.55){A};
    \begin{scope}[smooth,samples=30,domain=0:pi]
      \draw[functions] plot(\x,{.5*sin(450/pi*\x)});
      \draw[dashed] plot(\x,{-.5*sin(450/pi*\x)});
    \end{scope}
    \node[above] at (pi/2,.5){3rd Resonance Mode};
  \end{tikzpicture}
\end{figure}

%
%
%
%{Standing Waves in a Semi-Open Pipe of Length $L$}
In this case, the fundamental mode occurs at $\lambda=4L$:
\begin{figure}[ht]
  \centering
  \begin{tikzpicture}[xscale=1.1,yscale=.5]
    \draw[thick] (pi,-1)--(0,-1)--(0,1)--(pi,1);
    \begin{scope}[smooth,samples=25]
      \draw[domain=0:pi,functions] plot(\x,{sin(90/pi*\x)});
      \draw[domain=pi:4*pi,dashed] plot(\x,{sin(90/pi*\x)});
      \draw[domain=0:4*pi,dashed] plot(\x,{-1*sin(90/pi*\x)});
    \end{scope}
    \draw[|<->|] (0,1.55)--(pi,1.55) node[midway,fill=white]{$L$};
    \draw[|<->|] (0,-1.3)--(4*pi,-1.3) node[midway,fill=white]{$\lambda$};
  \end{tikzpicture}
\end{figure}
Fundamental frequency $f_1$ differs from open-pipe and closed-pipe by a factor
of 2:
%
%  \eq{-.1in}{
%    \boxed{f_1=\frac{v_s}\lambda=\frac{v_s}{4L}}
%  }

Fundamental mode $f_1$ is lower than open-ended and closed-ended pipes for
the same length ; it has advantages when designing organ pipes that produces
low frequencies.




%{Organ Pipes}
%  The length of organ pipes are generally measured in \emph{feet} rather than
%  using SI units
%
%    \pic1{images/Organpipescloseup-scaled}
%    
%    The longest organ pipes are 16 ft (\SI{4.9}\metre) long. At room
%    temperature, where $v_s\approx\SI{345}{\metre\per\second}$, this
%    corresponds to a fundamental frequency of
%    \begin{displaymath}
%      f_1=\frac{v_s}{4L}=\frac{345}{4\cdot4.9}=\SI{18}\hertz
%    \end{displaymath}
%    which is just below the audible range. This is why organ pipes are almost
%    never longer than this
%  
%
%
%
%{Standing Waves in a Semi-Open Pipe of Length $L$}
A second resonance mode occurs at $\lambda=\dfrac43L$:
\begin{figure}[ht]
  \centering
  \begin{tikzpicture}[xscale=1.5,yscale=.5]
    \draw[thick] (pi,-1)--(0,-1)--(0,1)--(pi,1);
    \begin{scope}[smooth,samples=20,domain=0:pi]
      \draw[functions] plot(\x,{sin(270/pi*\x)});
      \draw[dashed] plot(\x,{-1*sin(270/pi*\x)});
    \end{scope}
    \begin{scope}[smooth,dashed,samples=10,domain=pi:1.5*pi]
      \draw plot(\x,{-1*sin(270/pi*\x)});
      \draw plot(\x,{1*sin(270/pi*\x)});
    \end{scope}
    \draw[|<->|] (0,-1.6)--(4/3*pi,-1.6) node[midway,fill=white]{$\lambda$};
  \end{tikzpicture}
\end{figure}
%    \eq{-.2in}{
%      f_2=\frac{v_s}\lambda=\frac{3v_s}{4L}=3f_1
%    }
%
%  \vspace{.1in}And a third resonance mode occurs at $\lambda=\dfrac45L$:
%  
%    \column{.45\textwidth}
%    \begin{tikzpicture}[xscale=1.5,yscale=.5]
%      \draw[thick] (pi,-1)--(0,-1)--(0,1)--(pi,1);
%      \begin{scope}[smooth,samples=20,domain=0:pi]
%        \draw[functions] plot(\x,{sin(450/pi*\x)});
%        \draw[dashed] plot(\x,{-1*sin(450/pi*\x)});
%      \end{scope}
%      \draw[<->](0,-1.6)--(.8*pi,-1.6) node[midway,fill=white]{$\lambda$};
%    \end{tikzpicture}
%
%    \column{.55\textwidth}
%    \eq{0in}{
%      f_3=\frac{v_s}\lambda=\frac{5v_s}{4L}=5f_1
%    }
%  
%
%
%
%
%{Standing Waves in a Semi-Open Pipe of Length $L$}
Semi-open pipes have an \emph{odd set of harmonics} because only
\emph{odd-number} multiples of the fundamental can be resonance modes.

%  \eq{-.1in}{
%    \boxed{f_n=(2n-1)f_1}
%    \quad n=1,2,3,\ldots
%    \quad\text{\normalsize where}\quad
%    \boxed{f_1=\frac{v_s}{4L}}
%  }
%
%  Remember: harmonic frequencies are multiples of the fundamental frequency.
%  This means that:
%  \begin{itemize}
%  \item 2nd resonance frequency = 3rd harmonic frequency
%  \item 3rd resonance frequency = 5th harmonic frequency
%  \item 4th resonance frequency = 7th harmonic frequency
%  \end{itemize}
%
%
%
%
%{Resonance Frequencies in Semi-Open Pipes}
%  
%    \column{.35\textwidth}
%    \pic1{images/clarf}
%    
%    \column{.65\textwidth}
%    Clarinet is an instrument that is modelled by a semi-open pipe. A fourier
%    analysis shows that the even-numbered harmonics are missing.
%  
%
%
%
%
%\subsection[Res.\ Length]{Resonance Length}
%
%{Resonance Length in a Semi-Open Pipe}
%  \begin{itemize}
%  \item Now that we have looked at resonance \emph{frequencies}, we'll look
%    at resonance \emph{lengths}
%  \item We produce a \emph{single frequency} in the pipe, and vary the length
%    of the pipe until we have resonance
%  \end{itemize}
%
%
%
%
%{Resonance Length in a Semi-Open Pipe}
%  Let's submerge a part of this pipe in water. We can change the effective
%  length of the pipe by lowering/raising it.
%  \begin{center}
%    \pic{.5}{images/res-length-closed}
%  \end{center}
%
%  \vspace{-.1in}If we place a sound source at the mouth of the pipe (e.g.\
%  tuning fork), at certain lengths, we hear a loud sound coming from the pipe
%
%
%
%
%{Resonance Length in a Semi-Open Pipe}
%  The resonance lengths are \emph{odd-number multiples} of the first
%  resonance length $L_1$:
%  
%  \eq{-.1in}{
%    \boxed{L_n = (2n-1)L_1}
%    \quad\text{\normalsize where}\quad
%    \boxed{L_1 = \frac\lambda4}
%  }
%
%
%
%
%{Resonance Lengths in Open Pipes}
%  We can also repeat this with pipes that are open on both ends.

%    \begin{tikzpicture}[scale=.9]
%      \draw[thick] (0,-.5)--(pi,-.5);
%      \draw[thick] (0,0.5)--(pi,0.5);
%      \begin{scope}[smooth,samples=10,domain=0:pi,functions]
%        \draw plot(\x,{0.5*sin(180/pi*\x+90)});
%        \draw plot(\x,{-.5*sin(180/pi*\x+90)});
%      \end{scope}
%      \draw[<->] (0,-.6)--(pi,-.6) node[midway,below]{$\frac12\lambda$};
%      \node at (5,0) {1st resonance length};
%    \end{tikzpicture}\\
%    \begin{tikzpicture}[scale=.9]
%      \draw[thick] (0,-.5)--(2*pi,-.5);
%      \draw[thick] (0,.5)--(2*pi,.5);
%      \begin{scope}[smooth,samples=20,domain=0:2*pi,functions]
%        \draw plot(\x,{0.5*sin(180/pi*\x+90)});
%        \draw plot(\x,{-.5*sin(180/pi*\x+90)});
%      \end{scope}
%      \draw[<->] (0,-.6)--(2*pi,-.6) node[midway,below]{$\lambda$};
%      \node at (8.2,0) {2nd resonance length};
%    \end{tikzpicture}\\
%    \begin{tikzpicture}[scale=.9]
%      \draw[thick] (0,-.5)--(3*pi,-.5);
%      \draw[thick] (0,0.5)--(3*pi,0.5);
%      \begin{scope}[smooth,samples=30,domain=0:3*pi,functions]
%        \draw plot(\x,{0.5*sin(180/pi*\x+90)});
%        \draw plot(\x,{-.5*sin(180/pi*\x+90)});
%      \end{scope}
%      \draw[<->] (0,-.6)--(3*pi,-.6) node[midway,below]{$1\frac12\lambda$};
%      \node at (11.3,0) {3rd resonance length};
%    \end{tikzpicture}
%  
%
%
%
%
%
%{Resonance Lengths in Open Pipes}
%  Resonance lengths of an open pipe are \emph{whole-number multiples} of the
%  first resonance length $L_1$:
%
%  \eq{-.1in}{
%    \boxed{L_n=nL_1}
%    \quad\text{\normalsize where}\quad
%    \boxed{L_1=\frac\lambda2}
%  }
%  
%  This equation looks a lot like the resonance frequency equation. When you
%  read your homework/test questions to make sure you know what the question
%  is asking for.
%
%
%
%
%%{Example Problem}
%%  \textbf{Example:} A vibrating tuning fork is held near the mouth of a narrow
%%  plastic pipe partially submerged in water. The pipe is raised, and the first
%%  loud sound is heard when the air column is \SI{9.}{\centi\metre} long. The
%%  temperature in the room is \SI{20}\celsius.
%%  \begin{itemize}
%%  \item Calculate the wavelength of the sound produced by the tuning fork.
%%  \item Calculate the length of the air column for the second and third
%%    resonances.
%%  \item Estimate the frequency of the tune
%%  \end{itemize}
%%
%
%
%\section{Temperament}
%
%{Your Piano is Always Out of Tune}
%  The concept of harmonic frequencies is the reason why some musical
%  instruments can never play well in harmony (e.g.\ pianos, modern organs).
%  Below is a table of frequencies of each musical note in the western (European)
%  scales.
%
%  {\scriptsize
%    \begin{center}
%      \begin{tabular}{g|c|g|c|g|c|g|c|g|c|g|c|g}
%        \rowcolor{pink!60}
%        & C & C\# & D & D\# & E & F & F\# & G & G\# & A & A\# & B\\
%        \hline
%        \textbf{0} & 16.35 & 17.32 & 18.35 & 19.45 & 20.60 & 21.83 & 23.12 &
%        24.50 & 25.96 & 27.50 & 29.14 & 30.87 \\ \hline
%        \textbf{1} & 32.70 & 34.56 & 36.71 & 38.89 & 41.20 & 43.65 & 46.25 &
%        49.00 & 51.91 & 55.00 & 58.27 & 61.74 \\ \hline
%        \textbf{2} & 65.41 & 69.30 & 73.42 & 77.78 & 82.41 & 87.31 & 92.50 &
%        98.00 & 103.8 & 110.0 & 116.5 & 123.5 \\ \hline
%        \textbf{3} & 130.8 & 138.6 & 146.8 & 155.6 & 164.8 & 174.6 & 185.0 &
%        196.0 & 207.7 & 220.0 & 233.1 & 246.9 \\ \hline
%        \textbf{4} & 261.6 & 277.2 & 293.7 & 311.1 & 329.6 & 349.2 & 370.0 &
%        392.0 & 415.3 & 440.0 & 466.2 & 493.9 \\ \hline
%        \textbf{5} & 523.3 & 554.4 & 587.3 & 622.3 & 659.3 & 698.5 & 740.0 &
%        784.0 & 830.6 & 880.0 & 932.3 & 987.8 \\ \hline
%        \textbf{6} & 1047 & 1109 & 1175 & 1245 & 1319 & 1397 & 1480 & 1568 &
%        1661 & 1760 & 1865 & 1976 \\ \hline
%        \textbf{7} & 2093 & 2217 & 2349 & 2489 & 2637 & 2794 & 2960 & 3136 &
%        3322 & 3520 & 3729 & 3951
%      \end{tabular}
%    \end{center}
%  }
%
%
%
%{Equal Temperament}
%  In \textbf{equal temperament} (which is how the frequency table was
%  generated), the frequencies of each successive semi-tone have the same ratio:
%  
%  \eq{-.1in}{
%    \boxed{f(n+1)=\sqrt[12]2f(n)}
%  }
%  
%  \vspace{-.15in}Using A4 (\SI{440}\hertz, the 49th key on a piano) as the
%  reference tone, frequency of the $n$-th key on a piano is then:
%
%  \eq{-.1in}{
%    \boxed{f(n)=440(\sqrt[12]2)^{n-49}}
%  }
%  
%  Equal temperament is an elegant solution to make musical
%  instruments---especially keyboard instruments---sound uniform across all
%  major and minor keys. However, the irrational nature $\sqrt[12]2$ is very
%  different from the integral nature of harmonic frequencies.
%
%
%
%
%{Harmonic Frequencies of A2}
%  Consider a cello playing an A on the G-string. Because the string is fixed on
%  both ends, the instrument produces sound in all of its harmonics (which are
%  also its resonance modes). The first 8 harmonics have the frequencies:
%
%  \eq{-.15in}{
%    110, 220, 330, 440, 550, 660, 770, \SI{880}\hertz\ldots
%  }
%
%  \vspace{-.1in}The piano, however, does \emph{not} generate these frequencies.
%  Instead, it generates frequencies that are \emph{close} but different:
%
%  \eq{-.15in}{
%    110, 220, {\color{orange}329.6}, 440, {\color{orange}554.4},
%    {\color{orange}659.3}, {\color{red} 784}, \SI{880}\hertz\ldots
%  }  
%
%
%
%
%{Difference Between Just and Even Temperament}
%  By comparing the corresponding notes to the tables, we can find out the
%  musical interval based on the frequencies:
%  {\footnotesize
%    \begin{center}
%      \begin{tabular}{g|c|g|c|g|c}
%        \hline
%        \rowcolor{pink!60}
%        Harmonic & Cello $f_n$ & Keyboard & Musical & Error &
%        Musical\\
%        \rowcolor{pink!60}
%        $n$ & (\si\hertz) & (\si\hertz) & Note & (\si\hertz) & Interval \\
%        \hline
%        1 & 110 & 110   & A   & 0   & \\
%        2 & 220 & 220   & A   & 0   & Octave ($1:2$) \\
%        3 & 330 & 329.6 & E   & 0.4 & Perfect 5th ($2:3$) \\
%        4 & 440 & 440   & A   & 0   & Perfect 4th ($3:4$) \\
%        5 & 550 & 554.4 & C\# & 4.4 & Major 3rd ($4:5$) \\
%        6 & 660 & 659.3 & E   & 0.7 & Minor 3rd ($5:6$) \\
%        7 & 770 & 784   & G   & 14  & Minor 3rd? ($6:7$) \\
%        8 & 880 & 880   & A   & 0   & Major 2nd  ($7:8$) \\
%        \hline
%      \end{tabular}
%    \end{center}
%  }
%
%
%
%
%{Tuning Musical Intervals Using Physics}
%  The ratio of frequencies gives us insight into how musicians\footnote{Even
%    though most musicians are not physicists!} tune their instruments. For
%  violins\footnote{This also applies to violas and cellos}, the interval between
%  strings is a perfect fifth, which has a ratio of $2:3$.
%  \begin{itemize}
%  \item A violinist will first tune their A-string to \SI{440}\hertz
%  \item The violinist will play both the A and E-strings simultaneously
%  \item The E-string is considered to be ``tuned'' when it is at \SI{660}\hertz.
%    At this time:
%    \begin{itemize}
%    \item The beat frequency between the two strings
%      \begin{displaymath}
%        f_b=|660-440|=\SI{220}\hertz
%      \end{displaymath}
%    \item $f_b$ is an audible note itself, one octave below the A that the
%      violinist has been tuning against, called a ``ghost note''
%    \end{itemize}
%  \item Note that the ``violin E'' is \emph{slightly} different from the
%    ``piano E''
%  \end{itemize}
%
%
%
%
%{Tuning Musical Intervals Using Physics}
%  The musician will then use the same technique to tune the D-string
%  (perfect 5th below) and then the G-string.
%  \begin{itemize}
%  \item When tuning the D-string, the frequency will be \SI{293.3}\hertz.
%    There will be a ``hum'' at $440-293.3=\SI{146.7}\hertz$ which is 1 octave
%    below the D.
%  \item Likewise, the G-string will have a frequency of \SI{195.6}\hertz
%  \item These frequencies are \emph{slightly} different from the piano (equal
%    temperament) frequencies of D4=\SI{293.7}{\hertz} and
%    G3=\SI{196.0}\hertz, but the difference is small and most casual listeners
%    will not notice the difference
%  \end{itemize}

