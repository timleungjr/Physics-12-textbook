\chapter{Electric Field}

%\section{The Charges Are: Let's Review Some Basics}
%  We already know quite a bit about charge particles:
%  \begin{itemize}
%  \item The \emph{net charge} of an object means an excess of protons or
%    electrons.
%  \item The SI unit for electric charge is a \emph{coulomb} (\si\coulomb),
%    which is the amount of charges that \SI1{\ampere} of current passes through
%    a circuit in \SI1{\second} (i.e.\ $\SI1\coulomb=\SI1{\ampere\second}$)
%  \item Elementary charge $e=\SI{1.602e-19}\coulomb$
%  \end{itemize}
%\end{frame}

From the onset, we know that there are two types of charge, which we have
assigned a positive or negative sign to facility mathematical calculations.

With the development of atomic theory in the beginning of the 20th Century,
we understand that the positive charge is carried by the proton, while the
negative the charge is carried by the electron. The ``charge'' used in this
chapter is the net difference between 


We will begin with \textbf{electrostatics}, where charges that are not moving
(or moving slowly) relative to one another



\section{Coulomb's Law for Electrostatic Force}

Electric charges exert an \textbf{electrostatic force}\footnote{Also known as
the \textbf{electric force}, or the \textbf{coulomb force}. It is a
fundamental force; and also a conservative force.} $\vec F_q$ on each other.
The direction of the force is based on the \emph{sign} of the charges:
\begin{itemize}
\item Like charges \emph{repel} (positive \& positive; negative \& negative)
\item Opposite charges \emph{attract} (positive \& negative)
\end{itemize}
\begin{figure}[ht]
  \centering
  \begin{subfigure}{.5\linewidth}
    \begin{tikzpicture}[scale=.5]
      \draw[vectors] (0,0)--(2.8,0) node[right]{$F_q$};
      \draw[vectors] (8,0)--(5.2,0) node[left]{$F_q$};
      \draw[poscharge] circle (.4) node[white]{$+$};
      \draw[negcharge] (8,0) circle (.4) node[white]{$-$};
    \end{tikzpicture}
  \end{subfigure}

  \begin{subfigure}{.5\linewidth}
    \begin{tikzpicture}[scale=.5]
      \draw[vectors] (0,0)--(-3,0) node[left]{$F_q$};
      \draw[vectors] (8,0)--(11,0) node[right]{$F_q$};
      \draw[poscharge] circle (.4) node[white]{$+$};
      \draw[poscharge] (8,0) circle (.4) node[white]{$+$};
    \end{tikzpicture}
  \end{subfigure}
    
  \begin{subfigure}{.5\linewidth}
    \begin{tikzpicture}[scale=.5]
      \draw[vectors] (0,0)--(-3,0) node[left]{$F_q$};
      \draw[vectors] (8,0)--(11,0) node[right]{$F_q$};
      \draw[negcharge] circle (.4) node[white]{$-$};
      \draw[negcharge] (8,0) circle (.4) node[white]{$-$};
    \end{tikzpicture}
  \end{subfigure}
\end{figure}


\begin{center}
  \begin{tikzpicture}[scale=.5]
    \draw[vectors] (0,0)--(3,0) node[right]{$F_q$};
    \draw[vectors] (10,0)--(7,0) node[left]{$F_q$};
    \draw[poscharge] circle (.3) node[white]{$+$} node[above=3]{$q_1$};
    \draw[negcharge] (10,0) circle (.3) node[white]{$-$} node[above=3]{$q_2$};
  \end{tikzpicture}
\end{center}
The magnitude of the electrostatic force between two \emph{point charges} is
given by \textbf{Coulomb's law}:
\begin{equation}
  \boxed{
    F_q=\frac{k\left|q_1q_2\right|}{r^2}
  }
\end{equation}
%\begin{center}
%  \begin{tabular}{l|c|c}
%    \rowcolor{pink}
%    \textbf{Quantity} & \textbf{Symbol} & \textbf{SI Unit} \\ \hline
%    Magnitude of electrostatic force & $F_q$ &  \si{\newton} \\
%    Coulomb's constant (electrostatic constant) & $k$ & \si{N.m^2/C^2} \\
%    Point charges 1 and 2 & $q_1$, $q_2$ & \si\coulomb \\
%    Distance between point charges & $r$ & \si\metre
%  \end{tabular}
%\end{center}
where $k=\SI{8.99e9}{N.m^2/C^2}$ is called \textbf{Coulomb's constant}.
Coulomb's law is very similar in form to the law of universal gravitation
(Eq.~\ref{eq:law-of-gravity}). The major difference is that whereas masses are
\emph{always} positive, and the gravitational force is \emph{always} attractive,
charges can be either positive or negative, and therefore the electrostatic
force may be attractive or repulsive. Nevetheless, much of the analysis used




%\section{Comparing $\bm F_g$ to $\bm F_q$}

%    \centering
%    Electrostatic force (point charge):
%
%    \eq{-.2in}{
%      \boxed{F_q=\frac{k|q_1q_2|}{r^2}}
%    }
%    
%    \centering
%    Gravitational force (point mass):
%
%    \eq{-.2in}{
%      \boxed{F_g=\frac{Gm_1m_2}{r^2}}
%    }
%  
%  \vspace{.15in}Similarities:
%  \begin{itemize}
%  \item Both inversely proportional to $r^2$ (inverse square law)
%  \item Both are scaled by a constant
%  \end{itemize}
%  Differences:
%  \begin{itemize}
%  \item For gravity there are only positive masses (only attraction)
%  \item Electric charge can be either positive or negative (charges can
%    attract or repel)
%  \end{itemize}



\section{Electric Field}
We can define the \textbf{electric field} by repeating the same procedure as
with gravitational field. Grouping the variables in Coulomb's law:
\begin{equation}
  F_q=\underbracket[1pt]{\left[\frac{k|q_1|}{r^2}\right]}_{E}q_2
\end{equation}
A source charge $q_s$ creates an ``electric field'' ($E$) with a magnitude of
%\footnote{i.e.\ also known as the \emph{strength} or \emph{intensity}
%of the field}
\begin{equation}
  E(q_s,r)=\frac{k|q_s|}{r^2}
  \label{eq:point-charge-electric-field}
\end{equation}
where $r$ is the distance from the point charge.
%\begin{center}
%  \begin{tabular}{l|c|c}
%      \rowcolor{pink}
%      \textbf{Quantity} & \textbf{Symbol} & \textbf{SI Unit} \\ \hline
%      Electric field strength & $E$ & \si{\newton\per\coulomb} \\
%      Coulomb's constant & $k$ & \si{N.m^2/C^2} \\
%      Source charge & $q_s$ & \si\coulomb \\
%      Distance from source charge & $r$ & \si\metre
%  \end{tabular}
%\end{center}
Similar to gravitational field, electric field $\bm E$ created by $q_s$ is
a function that shows how it influences other charged particles around it. The
direction of the field is radially outward from a positive point charge and
radially inward towards a negative charge.
\begin{remark}
  For those who have a strong background in vectors, we can directly express
  the electric field from a point charge in vector form, just as we did with
  gravity:
  \begin{equation*}
    \boxed{
      \bm E(q_s,\bm r)=\frac{kq_s}{|\bm r|^2}\hat{\bm r}
    }
  \end{equation*}
  where $\hat{\bm r}$ is the \emph{outward radial direction}. If the
  source charge $q_s$ is positive, the direction of the electric field points
  away from the charge, (in the $\hat{\bm r}$ direction with a magnitude
  speicified in Eq.~\ref{eq:point-charge-electric-field}). Conversely, if $q_s$
  is negative, the direction of the electric field is towards the charge.
\end{remark}

%The magnitude of the electric field
%charge is a function of the source charge $q_s$ and
%%inversely proportional for the square of the
%the distance $r$ from the charge:
%
%  \eq{-.1in}{
%    \boxed{E(q_s,r)=\frac{k|q_s|}{r^2}}
%  }


%  \centering
%  \begin{tikzpicture}
%    \shade[poscharge] circle (.25) node[midway,white]{$Q$};
%    \begin{scope}[vectors,blue]
%      \foreach \theta in {0,45,...,359} \draw[rotate=\theta] (.5,0)--+(1.2,0);
%      \foreach \theta in {0,30,...,359} \draw[rotate=\theta] (2.3,0)--+(.5,0);
%    \end{scope}
%    \node[left,blue] at (2.3,0) {$\bm E$};
%    \node[text width=153, draw=blue, fill=cyan!5, text=blue] at (0,3.5) {
%      \scriptsize Positive charge $Q$ generates an electric field with a
%      magnitude of
%
%      \vspace{-.07in}
%      \begin{displaymath}
%        E=\frac{kQ}{r^2}
%      \end{displaymath}
%      The direction of $\bm E$ is \emph{away from} the charge.\par
%    };
%
%    \uncover<3->{
%      \begin{scope}[violet,rotate=30]
%        \draw[vectors] (-1.2,0)--+(-1.5,0) node[above=0]{$\bm F_q$};
%        \draw[thick,fill=orange!5] (-1.2,0) circle (.18) node{$q$};
%      \end{scope}
%      \node[
%        text width=190,
%        draw=violet,
%        fill=violet!5,
%        text=violet] at (0,-2.7){\scriptsize Positive charge $q$ is inside the
%        electric field generated by {\color{blue}$Q$}, therefore it
%        experiences an electric force:
%        
%        \vspace{-.08in}
%        \begin{displaymath}
%          \bm F_q=q{\color{blue} \bm E}
%        \end{displaymath}
%        in the same direction as {\color{blue}$\bm E$}. The force is
%        repulsive.\par
%      };
%    }
%  \end{tikzpicture}
%  \hspace{.3in}
%  \begin{tikzpicture}
%    \uncover<2->{
%      \shade[negcharge] circle (.3) node[midway,white]{$-Q$};
%      \begin{scope}[vectors,magenta]
%        \foreach \x in {0,45,...,359} \draw[rotate=\x] (1.5,0)--+(-1,0);
%        \foreach \x in {0,30,...,359} \draw[rotate=\x] (2.8,0)--+(-.4,0);
%      \end{scope}
%      \node[right,magenta] at (1.5,0) {$\bm E$};
%      \node[text width=144, draw=magenta, fill=magenta!5, text=magenta]
%      at (0,3.5) {\scriptsize Negative charge $-Q$ generates an electric field
%        with a magnitude of
%
%        \vspace{-.07in}
%        \begin{displaymath}
%          E=\frac{k|-Q|}{r^2}=\frac{kQ}{r^2}
%        \end{displaymath}
%        The direction of $\bm E$ is \emph{towards} the charge.\par
%      };
%    }
%    \uncover<4->{
%      \begin{scope}[orange,rotate=15]
%        \draw[vectors] (-2.5,0)--+(1.2,0) node[right=0]{$\bm F_q$};
%        \draw[thick,fill=orange!5] (-2.5,0) circle (.18) node{$q$};
%      \end{scope}
%      \node[
%        text width=190,
%        draw=orange,
%        fill=orange!5,
%        text=orange] at (0,-2.7){\scriptsize Positive charge $q$ is inside the
%        electric field generated by {\color{magenta}$-Q$}, therefore it
%        experiences an electric force:
%        
%        \vspace{-.08in}
%        \begin{displaymath}
%          \bm F_q=q{\color{magenta}\bm E}
%        \end{displaymath}
%        in the same direction as {\color{magenta}$\bm E$}. The force is
%        attractive.\par
%      };
%    }
%  \end{tikzpicture}

When another charge $q$ enters the electric field $\bm E$, the field exerts
an electrostatic force $\bm F_q$ on the charge. The equation is similar to
the law of gravitation:
%  $\bm E$ \emph{doesn't do anything} until another charge interacts with it.
%  And when there is a charge $q$, the electrostatic force $\bm F_q$ that it
%  experiences in the presence of $\bm E$ is:
\begin{equation}
  \boxed{
    \bm F_q=q\bm E
  }
\end{equation}
If $q>0$ (a positive charge), $\bm F_q$ is in the \emph{same} direction as
$\bm E$; if $q<0$ (a negative charge), $\bm F_q$ is in the
\emph{opposite} direction as $\bm E$.


%\section{Electric Field from Multiple Charges}
When there are multiple charges, then the total electric field at any
location is the vector sum of all the electric field from each source
charge, i.e.:
\begin{equation}
  \bm E=\bm E_1+\bm E_2+\bm E_3+\bm E_4+\cdots=\sum_i \bm E_i
\end{equation}
Once the total electric field is known, we can use the equation
\begin{equation*}
  \bm F_q=q\bm E
\end{equation*}
without knowing \emph{what} created $\bm E$



%\section{Electric Field Lines}
%  If you place a positive charge in an electric field, the force on the charge
%  will be in the direction of the electric field.
%  
%    \pic{.95}{graphics/pos_charge}\\
%    \pic{.95}{graphics/neg_charge}
%
%    \pic{.95}{graphics/2charges}

%\begin{example}
%  What is the electric field strength at a point \SI{30}{\centi\metre} from the
%  centre of a small sphere that has a positive charge of
%  \SI{2.0}{\nano\coulomb}?
%\end{example}

\begin{example}
  Three point charges, $q_A=\SI{6.0}{\micro\coulomb}$,
  $q_B=\SI{-5.0}{\micro\coulomb}$, and $q_C=\SI{6.0}{\micro\coulomb}$, are
  located at the corners of a square with sides $L=\SI{5.0}{\centi\metre}$ long.
  What is the electric field at $D$?%Remember that electric field is a vector.
  \begin{center}
    \vspace{-.2in}
    \begin{tikzpicture}%[scale=1.4]
      \draw[axes] (2,2)--(3,2) node[right]{$x$};
      \draw[axes] (2,2)--(2,3) node[above]{$y$};
      \draw rectangle (2,2);
      \shade[poscharge] circle (.1) node[below left] {$B$};
      \shade[poscharge] (2,0) circle (.1) node[below right]{$C$};
      \shade[poscharge] (0,2) circle (.1) node[above left] {$A$};
      \fill (2,2) circle (.045) node[above right,black]{$D$};
      \node[above] at (1,2) {$L$};
    \end{tikzpicture}
  \end{center}
  \textbf{Solution:}
\end{example}
