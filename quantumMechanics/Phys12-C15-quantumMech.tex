\chapter{Introduction to Quantum Mechanics}

\begin{definition}
  \textbf{Anyone who is not shocked by quantum theory has not understood it.}

  \vspace{.1in}\flushright{- Niels Bohr}
\end{definition}



%\section[Intro]{Introduction}

Light must be a wave. After all, it has all the properties of waves:
\begin{itemize}
\item Refraction
\item Interference
\item Diffraction\ldots
\end{itemize}
We even know what \emph{kind} of a wave it is
\begin{itemize}
\item Electromagnetic (``EM'') wave
\item A transverse wave
\item Same as: radio waves, microwave, infrared, ultraviolet, x-ray\ldots
\item Travels in vacuum with a speed of \SI{2.998e8}{\metre\per\second},
  independent of the velocity of the object emitting the light
\end{itemize}
But now we're going to find out that things aren't as simple as it seems.
%
%
%
%
\section{Blackbody Radiation}
\begin{figure}[ht]
  \centering
  \pic{.4}{quantumMechanics/graphics/Black-body_realization}
\end{figure}

\begin{itemize}
\item The concept was coined by Gustav Kirchhoff in 1860
\item An idealized object that absorbs all incident EM radiation
\item Model of a blackbody:
  \begin{itemize}
  \item A box (``cavity'') with a mirror inside, and a hole where
    light (EM radiation) is allowed in
  \item Some of the light reflects inside the cavity; some gets absorbed
    by the cavity
  \item Eventually all the light inside the cavity is absorbed
  \end{itemize}

%    \pic{1.2}{graphics/Black-body_realization}
%    
%    \column{.8\textwidth}

\item A blackbody is in thermal equilibrium
  \begin{itemize}
  \item All of the absorbed energy is then immediately radiated back as EM
    radiation
  \item The spectral distribution depends on temperature
  \end{itemize}
\item A blackbody at room temperature appears black, because most of the
  radiative energy is infrared and invisible to the human eye
\item Thermal radiation spontaneously emitted by many ordinary objects can
  be approximated as blackbody radiation 
\end{itemize}
%  
%
%
%
%
%\begin{frame}{Raleigh-Jeans Law and the Ultraviolet Catastrophe}
%  
%    \column{.55\textwidth}
\begin{itemize}
\item Based on classical thermodynamics
\item Agrees with experimental results for long wavelengths, but the
  equation predicts that short wavelengths (e.g.\ ultraviolet radiation,
  x-ray) will have infinite intensity
\item Known as \textbf{``ultraviolet catastrophe''}
\end{itemize}
%
%    \column{.45\textwidth}
%    \pic1{graphics/1280px-Black_body}
  



\section{Quantization of Energy}
%  
%    \column{.2\textwidth}
%    \centering
%    \pic1{graphics/20973-050-F6EEBFF1}\\
%    {\footnotesize Max Planck}
%    
%    \column{.8\textwidth}
\begin{itemize}
\item Made a strange modification in the classical calculations
\item Derived a function of $P(\lambda,T)$ that agreed with experimental
  data for all wavelengths
\item First found an empirical function to fit the data
\item Then searched for a way to modify the usual calculations
\end{itemize}

Theorized that the walls of a blackbody are composed of subatomic electric
oscillators (``resonators'')
\begin{itemize}
\item The nature of these resonators were unknown
\item Billions of resonators vibrating at different frequencies, and
\item Emitting radiation at those frequencies (remember: for a wave, the
  frequency of source disturbance is the frequency of the wave)
\item In classical physics, resonators can have any energy, and change
  its amplitude continuously
\end{itemize}
To agree with experiments, Planck theorized that energy emitted by a
resonator must be \emph{discrete}
\begin{itemize}
\item When energy is emitted from a resonator, it drops to the next
  lower energy level
\end{itemize}

The total energy $E_\text{res}$ of a harmonic oscillator is proportional
to the integral multiples of its natural frequency $f$:
\begin{equation}
  \boxed{E_\text{res}=nhf}
\end{equation}
%    \begin{center}
%      \begin{tabular}{l|c|c}
%        \rowcolor{pink}
%        \textbf{Quantity} & \textbf{Symbol} & \textbf{SI Unit} \\ \hline
%        Energy of the resonator & $E_\text{res}$ & \si\joule\\
%        Energy level            & $n$ & (no unit)\\
%        Planck's constant       & $h$ & \si{\joule\second}\\
%        Frequency of resonator  & $f$ & \si\hertz
%      \end{tabular}
%    \end{center}

\begin{figure}[ht]
  \centering
  \begin{tikzpicture}[xscale=.55,yscale=.75]
    \foreach \y in {0,1,...,4} {
      \draw[very thick] (0,\y)--(4,\y)
      node[pos=0,left]{$\y$} node[right]{$\y hf$};
    }
    \foreach \y in {0,1,...,3} {
      \draw[very thick,red!70!black,<->] (\y+.2,\y)--(\y+.2,\y+1);
    }
    \draw[vectors] (4.7,4.5)--(4.7,5.9) node[above]{Energy $E$};
    \draw[vectors] (-.4,4.5)--(-.4,6) node[above]{to $n=\infty$};
  \end{tikzpicture}
\end{figure}
\textbf{Planck's constant} is experimentally determined to be
$h=\SI{6.626e-34}{\joule\second}$
%
%
%
\subsection{Energy Quantization in Ordinary Harmonic Oscillators}
%\begin{frame}{Blackbody Radiation}
%According to Planck, \emph{all} harmonic oscillators have discrete energies,
%not just vibrations at the subatomic level.
%\begin{itemize}
%\item Applies also to the swing of a pendulum, or mass-spring systems (Unit
%  2)
%\item The frequency $f$ is the natural frequencies of those systems
%\item Because $h$ is small, measuring the difference between energy levels
%  is difficult at macroscopic scale
%\end{itemize}


Planck envisioned that the quantization of energy applies to all of harmonic
oscillators, rather than just for the blackbody radiation. To understand how
energy quantization applies to standard problems in harmonic motion, let us
consider the frictionless horizontal spring-mass system that we studied in
Chapters 4 (energy) and 7 (harmonic motion). In Section 7.4, we derived the
equation for the natural frequency ($f_0$) of the system, which depends on the
spring constant k of the spring, and the mass m connected to it:
\begin{equation*}
  f_0 = \frac1{2\pi}\sqrt{\frac km}
\end{equation*}
When the mass oscillates, its total mechanical energy is the mass's kinetic
energy, plus the elastic potential energy stored in the spring, as we have
already shown in Eq. 4.15.
\begin{equation*}
  E = K + U_s
\end{equation*}
At maximum displacement (i.e. at amplitude $A$), all of the system's energy is
in elastic potential energy:
\begin{equation*}
  E = \frac12kA^2
\end{equation*}
We can then relate the total energy to the amplitude of the harmonic motion.
Solving for the amplitude, we have:
\begin{equation}
  A=\sqrt{\frac{2E}k}
  \label{eq:amplitude-from-energy}
\end{equation}
In classical physics, the pendulum can have \emph{any} energy levels: both the
mass's kinetic and elastic potential energies are continuous functions, and
therefore any amplitudes are plausible. However, in Planck's view, the energies
of this system must also be quantized. For a hypothetical system with
$k=\SI{987}{\newton\per\metre}$, and $m=\SI1{\kilo\gram}$, we will have a
natural frequency of $f_0=\SI{5.00}\hertz$. As it is oscillating at its natural
frequency, the allowable energies are:
\begin{align*}
  E_0 &= 0\\
  E_1 &= hf_0  = \SI{3.32e-33}{\joule}\\
  E_2 &= 2hf_0 = \SI{6.63e-33}{\joule}\\
  E_3 &= 3hf_0 = \SI{9.94e-33}{\joule}
\end{align*}
The gap between different energy levels is $\Delta E=hf=\SI{3.32e-33}\joule$.
Energies between these discrete levels(i.e. between $E_1$ and $E_2$, or between
$E_2$ and $E_3$) are not allowed. This means that the possible amplitudes of
the mass can be calculated by substituting the energy levels into
Eq.~\ref{eq:amplitude-from-energy}:
\begin{align*}
  A_0 &= 0\\
  A_1 &= \cdots\\
  A_2 &= \cdots\\
  A_3 &= \cdots
\end{align*}
For a spring-mass system with a low natural frequency, the energy gap---and
therefore the gaps between amplitudes---is laughably small, which is why energy
quantization cannot be observed in a macroscopic scale. But for a charge
particle emitting light, it will have to vibrate at a frequency between
\num{4e14} to \SI{7e14}{\hertz}. All of a sudden, the gap $\Delta E$ is no longer so
small.

When this mass is oscillating, frictional forces removes energy from the
system. N Chapter 7, we studied the use of a "damping force" $\bm F_D$ to
consolidate the effects of kinetic friction, drag and viscous damping, using
the equation
\begin{equation*}
  F_D= -bv
\end{equation*}
When this damping force is present, the total energy $E$ of the system decreases
with time $t$ as an exponential decay function:
\begin{equation}
  E(t)=E_0e^{-\frac bmt} %=E_0e^{-\frac{t}{\tau}}
\end{equation}
where $E_0$ is the initial energy of the system, and $b$ is the damping
constant. For simplicity, we are not showing the derivation of this equation,
except to point out that $E(t)$ is a continuous function in time, as shown in
Fig.\ 15.4a. However, if the energy levels of this harmonic oscillator is
indeed quantized, then $E(t)$ would, in fact, decrease as a step function, as
shown in Fig.\ 15.4b.


\section{Maxwell's Equations: Classical Laws of Electrodynamics}
%%  
%%    \column{.25\textwidth}
%%    \centering
%%    \pic{.97}{graphics/PORTRAIT-James-Clerk-Maxwell}\\  
%%    {\footnotesize James Clerk Maxwell\par}
%%    
%%    \column{.75\textwidth}
Maxwell's equations show that an oscillating charge generates a fluctuating
electric field and magnetic field, which travels through space as an
``electromagnetic wave'' with speed:
\begin{equation*}
  c=\frac1{\sqrt{\varepsilon_0\mu_0}}=\SI{299792458}{\meter\per\second}
\end{equation*}
By 1862, the speed of light was already measured to within about
\SI{.6}{\percent} of this value\footnote{By L\'{e}on Foucault using a
rotating mirror experiment. His experimental value was
\num{298000}$\pm 500$ \si{\metre\per\second}}. Does that mean that light is
an EM wave?




%\begin{frame}{Proving Light as an Electromagnetic Wave}
Proving that light is an EM wave requires an alternating current with a
frequency of \SI{e14}{\per\second}.
\begin{itemize}
\item Physicists already knew the wavelengths of visible light (e.g.\ through
  double-slit interference patterns)
\item The frequency can be determined simply applying $c=f\lambda$
\end{itemize}

Technology of Maxwell's time could only generate frequencies
around \SI{e8}{\per\second}
\begin{itemize}
\item Much higher than the \SI{60}{\per\second} that our electrical outlet
  uses
\item But still one million (\num{e6}) times too low
\end{itemize}




\section{Photoelectric Effect}

%%\begin{frame}{Heinrich Hertz}
%%  
%%    \column{.65\textwidth}
%%    \begin{itemize}
%%    \item German physicist (1857-1894)
%%    \item Devised the ``spark gap experiment'' to generate high frequencies
%%    \item The unit for frequency is named after him in his honour
%%    \end{itemize}
%%
%%    \column{.35\textwidth}
%%    \pic{1}{graphics/Heinrich_Rudolf_Hertz}
%%  
%%
%
%
%
%
%\begin{frame}{The Spark Gap Experiment}
German physicist Heinrich Hertz (1857-1894) devised a ``spark gap experiment''
to generate high frequencies required
\begin{figure}[ht]
  \centering
  \pic{.5}{quantumMechanics/graphics/Hertz_exp_2}
\end{figure}
\begin{itemize}
\item Produced EM waves with frequency \num{e14} oscillations per second
\item Showed that light waves have the same wavelengths as predicted by
  Maxwell's equations
\end{itemize}

%\begin{frame}{Discovery of the Photoelectric Effect}

Hertz left a terse remark in his results:
\begin{center}
  \fbox{
    \begin{minipage}{.75\textwidth}
      \emph{It is essential that the pole surfaces of the spark gap should be
      frequently re-polished to ensure reliable operation of the spark.}
      \end{minipage}
    }
  \end{center}
\begin{itemize}
\item Electrons were leaving the metal surface (caused by the UV radiation)
\item This is now known as the \textbf{photoelectric effect}
\item Hertz and other physicists who repeated his experiments did not have a
  satisfactory explanation using classical physics
\end{itemize}
This is, in fact, the first evidence that light is \emph{not} a wave after all.





%\begin{frame}{Photoelectric Effect}
When EM waves hits the (metallic) emitter plate, electrons are knocked
off the surface with some kinetic energy, generating a current
\begin{center}
  \begin{tikzpicture}[scale=.7,american voltages]
    \draw[thick,green!70!black,smooth,samples=30,domain=0:3,rotate=40,<-]
    plot({\x},{.15*sin(240*\x)}) node[above]{EM};
    \draw[thick] (.5,1)--(1.5,2);
    \draw[thick] (2,1)--(3,2);
    \draw[thick] (.5,1)--(0,1) arc (90:270:1)--(6,-1) arc (-90:90:1)--(2,1);
    \draw[ultra thick,blue!80!black] (0,.8)--(0,-.8)
    node[below,black]{Emitter};
    \draw[ultra thick,red!80!black] (6,.8)--(6,-.8)
    node[below,black]{Collector};
    \draw[thick] (0,0)--(-1.5,0)--(-1.5,-2.5)
    to[V,l_={Variable voltage}](3,-2.5)
    to[rmeter,t=A,l_={Ammeter}] (7.5,-2.5) --(7.5,0)--(6,0);
    \begin{scope}[blue!40]
      \fill (3,-.5) circle (.1);  \draw[vectors] (3,.-.5)--(4,-.5);
      \fill (1,0) circle (.1);    \draw[vectors] (1,0)--(2,0);
      \fill (2.2,.3) circle (.1); \draw[vectors] (2.2,.3)--(3.2,.3);
      \fill (4.5,.4) circle (.1); \draw[vectors] (4.5,.4)--(5.5,.4);
    \end{scope}
  \end{tikzpicture}
\end{center}
The current can be stopped by adjusting the voltage, called the
\textbf{stopping voltage}, allowing us to measure the kinetic energy of the
\textbf{photo electrons}
%
%
%
%
%
%\begin{frame}{Photoelectric Effect: Cannot be Explained by Classical View}
From the classical electrodynamics:
\begin{itemize}
\item Energy is transferred from light wave to the electrons
\item Kinetic energy of the emitted electrons should be proportional to both
  \emph{intensity} (related to the amplitude of the EM wave; brightness) and
  \emph{frequency}
\item Increasing intensity:
  \begin{itemize}
  \item Increases the number of electron emitted
  \item Increases the electrons' kinetic energy
  \end{itemize}
\item Even a dim light would \emph{eventually} transfer enough energy to an
  electron be emitted
\end{itemize}
\textbf{But this isn't what is happening!}




%\begin{frame}{Photoelectric Effect: What Actually Happens}
\begin{center}
  \pic{.9}{quantumMechanics/graphics/73bacc9f2bf571752483a89ef6c61a94f07470f7}
\end{center}
\begin{itemize}
\item Increasing intensity of light knocked off more electrons, but does not
  change their maximum kinetic energy $K_\text{max}$, but
\item Changing the \emph{frequency} of the light did change $K_\text{max}$
  though, although
\item Below a certain frequency, \emph{no} electrons were emitted
\end{itemize}




%\section{Einstein}
%
%%\begin{frame}{1905: \emph{Annus Mirabilis} (The Miraculous Year)}
%%  {The Year That Einstein Became Very Famous}
%%  \begin{itemize}
%%  \item<alert@2>\textbf{Photoelectric effect:} ``On a Heuristic Viewpoint
%%    Concerning the Production and Transformation of Light''
%%  \item\textbf{Brownian motion:} ``On the Motion of Small Particles Suspended
%%    in a Stationary Liquid, as Required by the Molecular Kinetic Theory of
%%    Heat''
%%  \item<alert@1>\textbf{Special relativity:} ``On the Electrodynamics of Moving
%%    Bodies''
%%  \item\textbf{Mass-energy equivalence:} ``Does the inertia of a body depend
%%    upon its energy content?''
%%  \end{itemize}




\subsection{The Photon: Packets of Energy}
In a paper published in 1905, Albert Einstein\footnote{1879--1955. Nobel Prize
winner 1921} offered a solution to the photoelectric effect: it can only be
explained if light behaves \emph{not} as a continuous wave, but rather, \emph{a
collection of discrete energy particles} called \textbf{photons}, each with
energy $E=hf$. In Einstein's model, the photons collide with the electrons, and
are then all of their energy are absorbed into the electron without delay.

The energy of the photon is determined by its \emph{frequency}, in agreement
with Planck, therefore, increasing the frequency of the light increases the
kinetic energy $K$ of the ejected photo electrons, in agreement with
experimental results. Additionally, the intensity (brightness) of light is
related to the number of photons. Therefore, increasing the brightness of
the light results in a higher number of electrons being knocked off the metal
surface. This is also in agreement with experimental results.

The relationship between kinetic energy of electrons and the photons can be
summarized in a simple linear function:
\begin{equation}
  \boxed{
    K_\text{max}=
    \begin{cases}
      hf-\varphi & \text{if}\;\;hf>\varphi\\
      0          & \text{otherwise}
    \end{cases}
  }
\end{equation}
where $K_\text{max}$ is the maximum kinetic energy of the photo-electrons; $h$
is Planck's constant; $f$ is the frequency of the EM wave; and $\varphi$ is
property of the material, called the \text{work function}.


%The ``work function'' is a property of the metal that determines how much
%energy is absorbed until an electron is knocked off

Bottom line: Classical electrodynamics concept cannot
be applied in photoelectric effect, and must be replaced with quantum
mechanics.




\subsection{Work Function $\varphi$}

\textbf{Work function} $\varphi$ is the minimum energy required to remove
an electron from the metal to a point just outside the metal surface. The
lowest frequency at which electrons are ejected is called the
\textbf{threshold frequency} $f_0$. The slope of the graph is $h$ independent
of the metal.

An example for zinc is shown as an example in Fig.~\ref{fig:zinc}. Zinc has a
work function of 4.3 eV, and therefore a threshold frequency of
\SI{1.04e15}{\hertz}. The frequency is in the ultra-violet range, and therefore
visible light will not cause electrons to be ejected.
\begin{figure}[ht]
  \centering
  \begin{tikzpicture}[xscale=.5,yscale=.65]
    \foreach \wav in {420, 422,...,740}{
      \definecolor{tmpcolor}{wave}{\wav}
      \colorlet{mycolor}[rgb]{tmpcolor}
      \fill[fill=mycolor,opacity=.2] ({3000/\wav},-5) rectangle +(.1,7);
    }
    \draw[axes] (0,0)--(15.7,0) node[above]{$f$ ($\times\SI{e14}{\hertz}$)};
    \draw[axes] (0,-5.2)--(0,2.7) node[right]{$\varphi$ ($\si{\electronvolt})$};
    \foreach \x in {2,4,...,14} \draw[thick](\x,0)--(\x,-.2) node[below]{$\x$};
    \foreach \y in {-5,...,2} \draw[thick](0,\y)--(-.2,\y) node[left]{$\y$};

    \draw[ultra thick,red] (0,0)--(10.4,0) node[above]{$f_0$};
    \draw[ultra thick,red,rotate around={22.46:(10.4,0)}]
    (10.4,0)--+(5,0) node[right]{Zn};
    \draw[ultra thick,dashed] (0,-4.3)--(10.4,0);
  \end{tikzpicture}
  \caption{Kinetic energy of photo-electrons for zinc.}
  \label{fig:zinc}
\end{figure}

\begin{table}[ht]
  \centering
  \begin{tabular}{l|c}
    \rowcolor{pink}
    \textbf{Metal} & \textbf{Work function} (\si\electronvolt) \\ \hline
    Aluminum & 4.28 \\ \hline
    Calcium  & 2.87 \\ \hline
    Cesium   & 2.14 \\ \hline
    Copper   & 4.65 \\ \hline
    Iron     & 4.50 \\ \hline
    Lead     & 4.25 \\ \hline
    Lithium  & 2.90 \\ \hline
    Nickel   & 5.15 \\ \hline
    Platinum & 5.65 \\ \hline
    Potassium & 2.30 \\ \hline
    Tin      & 4.42 \\ \hline
    Tungsten & 4.55 \\ \hline
    Zinc     & 4.33 \\
  \end{tabular}
  \caption{Work function for various metals}
  \label{tabl:work-function}
\end{table}
%  
%
%
%
%
\section{Compton Scattering}

%%    \pic{1.2}{graphics/Arthur_Compton_1927}
%%
%%    {\footnotesize Arthur H.\ Compton\par}

American physicist Arthur Compton\footnote{1892--1962; Nobel Prize winner in
1927}, was studying x-ray scattering by free electrons
\begin{itemize}
\item Classical theory cannot account for the scattering behaviour
\item Frequency shift only depends on scattering angle
\item Prediction possible if treating the x-ray as photons with
  momentum, just like a particle
\item In the \textbf{momentum-energy relation} by Einstein\footnote{From
the mass-energy equivalence $E=m'c^2$ which can be also expressed as
$E^2=p^2c^2+m^2c^4$}, a massless particle ($m=0$) has an energy related
  to its momentum by $E=pc$.
\end{itemize}

\begin{equation}
  \boxed{
    p=\frac Ec=\frac{hf}c=\frac h\lambda
  }
\end{equation}




%\begin{frame}{Compton Scattering}
By treating x-ray as photons (i.e.\ a particle) with momentum $p=h/\lambda$,
Compton showed that the behaviour of the scattered x-ray and the recoil
electron is consistent with an \emph{elastic collision}\footnote{Recall from
Unit 2 that in an elastic collision, both momentum and kinetic energy are
conserved} between the x-ray photon and the electron.
\begin{center}
  \pic{.45}{quantumMechanics/graphics/compton2}
\end{center}
%
%
%
%
%\begin{frame}{Momentum of a Photon}
The momentum of a photon is proportional to Planck's constant and inversely
proportional to its wavelength:
%
%  \eq{-.1in}{
%    \boxed{p=\frac h\lambda}
%  }
%  \begin{center}
%    \begin{tabular}{l|c|c}
%      \rowcolor{pink}
%      \textbf{Quantity} & \textbf{Symbol} & \textbf{SI Unit} \\ \hline
%      Momentum          & $p$       & \si{\kilo\gram\metre\per\second} \\
%      Planck's constant & $h$       & \si{\joule\second} \\
%      Wavelength        & $\lambda$ & \si\metre
%    \end{tabular}
%  \end{center}



\begin{example}
  Calculate the momentum of a photon of light that has
  frequency of \SI{5.09e14}\hertz.
\end{example}



\section{Matter Waves}

%\begin{frame}{Matter Waves: The De Broglie Hypothesis}
%%  
%%    \column{.25\textwidth}
%%    \centering
%%    \pic{1}{graphics/76562-004-66881FD5}\\
%%    {\footnotesize Louis De Broglie}
%%
%%    \column{.75\textwidth}
If electromagnetic waves are really particles of energy, then are particles
(e.g.\ electrons) waves?
\begin{itemize}
\item Louis Broglie\footnote{1892--1987; Nobel Prize winner in 1929},
  while completing his Ph.D.\ in 1924, proposed a hypothesis: a particle can
  also have a wavelength
\item Confirmed, accidentally, by the Davisson-Germer Experiment in 1927
  (beam of electron scattering on nickel crystal surface)
\end{itemize}




%\begin{frame}{Electron Interference}
%  
%    \column{.84\textwidth}
%    If I perform a double-slit experiment with a beam of electrons, will I get
%    an interference pattern?
%    \begin{center}
%    \pic{.6}{graphics/CNX_Chem_06_03_Electrnin}
%    \end{center}
%    
%    \column{.16\textwidth}
%\begin{figure}[ht]
%  \pic{.1}{quantumMechanics/graphics/206px-Double-slit_experiment_results_Tanamura_2}
%\end{figure}

If matter, like an electron, is also a wave, then it should have a wavelength
too. We can solve momentum equation to find $\lambda$:
\begin{equation}
    p=\frac h\lambda\;\;\rightarrow\;\;
    \lambda=\frac hp\;\;\rightarrow\;\;\boxed{\lambda=\frac h{mv}}
\end{equation}
%  \begin{center}
%    \begin{tabular}{l|c|c}
%      \rowcolor{pink}
%      \textbf{Quantity} & \textbf{Symbol} & \textbf{SI Unit} \\ \hline
%      Wavelength of a particle & $\lambda$ & \si\metre \\
%      Planck's constant & $h$       & \si{\joule\second} \\
%      Mass              & $m$       & \si{\kilo\gram} \\
%      Velocity          & $v$       & \si{\metre\per\second}
%    \end{tabular}
%  \end{center}
This applies if momentum is non-relativistic.





\begin{example}
  Calculate the wavelength of an electron moving with a velocity of
  \SI{6.39e6}{\metre\per\second}.
\end{example}


%\begin{frame}{Particle-Wave Duality}
%  \textbf{The Copenhagen interpretation:}
%  \begin{itemize}
%  \item Accepted view: wave-particle duality
%  \item An experiment can either show:
%    \begin{itemize}
%    \item The wave nature: diffraction, refraction (e.g.\ light)
%    \item The particle nature: scattering (Compton effect), photoelectric
%      effect
%    \item But not both.
%    \end{itemize}
%  \end{itemize}



\section{Uncertainty Principle}
If a particle is a wave, how can you tell where it is? Consider the
one-dimensional wave shown in Fig.~\ref{fig:1d-wave}.
\begin{figure}[ht]
  \centering
  $\Psi(x,t)=\cos(kx-\omega t)$\\
  \begin{tikzpicture}[scale=1.3]
    \draw[xscale=.3,function,smooth,samples=300,domain=-18:18]
    plot(\x,{sin(180*\x)});
  \end{tikzpicture}
  \caption{A one-dimensional wave}
  \label{fig:1d-wave}
\end{figure}
At any time $t$, the wave/particle can be described by the wave function
$\Psi(x,t)$. The function has a single wavelength of $\lambda$ (therefore a
single value of momentum $p=\frac h\lambda$), but it has no distinguishing
features that can tell you the particle's position $x$. In fact, the wave
functoin seems to suggest that the particle is \emph{everywhere}. Therefore we
conclude that, \textbf{when we have precise knowledge of a particle wave's
  \emph{momentum}, we have no knowledge of its \emph{position}.}

On the other extreme, consider a particle/wave defined as a
\emph{delta function}, as shown in Fig.~\ref{fig:1d-delta-wave}.
\begin{figure}[ht]
  \centering
  \vspace{-.3in}
  \begin{tikzpicture}[scale=1.2]
    \draw[function] (-5,0)--(-.04,0)--(0,2.3)--(.04,0)--(5,0);
    %\draw[function,smooth,samples=300,domain=-5:5]
    %plot(\x,{2*exp(-\x*\x/0.005)});
    \draw[axes] (3,2) to[out=180,in=90] (0,2.4);
    %\draw[axes] (3,2) to[out=270,in=270] (0,0);
    \node[right] at (3,2){The particle is here!};
  \end{tikzpicture}
  \caption{A one-dimensional delta wave}
  \label{fig:1d-delta-wave}
\end{figure}
The particle's position $x$ is well-defined but its wavelength $\lambda$ (and
therefore momentum $p$) is undefined. (Recall the definition of a wavelength:
it is the shortest distance between two points in a wave that is in phase. In
the case of a pulse wave, no two points are in phase.) The wave function
provides no information about the particle's velocity. Therefore, we conclude
that \textbf{when we have precise knowledge of a particle wave's
  \emph{position},  we have no knowledge of \emph{where} it is going.}

However, if a moving particle has small variations (uncertainties) in its
momentum $p$ (wavelength $\lambda$), as shown Fig.~\ref{fig:packet}.
\begin{figure}[ht]
  \centering
  $\cos(280x)+\cos(281x)+\cdots+\cos(300x)+\cdots+\cos(319x)+\cos(320x)$\\
  \begin{tikzpicture}[scale=1.2]
    \draw[xscale=.3,yscale=.07,function,smooth,samples=360,domain=-15:15]
    plot(\x,{.5*(
      sin(300*\x)+
      sin(301*\x)+ sin(302*\x)+ sin(303*\x)+sin(304*\x)+sin(305*\x)+
      sin(306*\x)+ sin(307*\x)+ sin(308*\x)+sin(309*\x)+sin(310*\x)+
      sin(311*\x)+ sin(312*\x)+ sin(313*\x)+sin(314*\x)+sin(315*\x)+
      sin(316*\x)+ sin(317*\x)+ sin(318*\x)+sin(319*\x)+sin(320*\x)+
      sin(280*\x)+ sin(281*\x)+ sin(282*\x)+sin(283*\x)+sin(284*\x)+
      sin(285*\x)+ sin(286*\x)+ sin(287*\x)+sin(288*\x)+sin(289*\x)+
      sin(290*\x)+ sin(291*\x)+ sin(292*\x)+sin(293*\x)+sin(294*\x)+
      sin(295*\x)+ sin(296*\x)+ sin(297*\x)+sin(298*\x)+sin(299*\x)
      )});
  \end{tikzpicture}
  \caption{A one-dimensional wave packet}
  \label{fig:packet}
\end{figure}
When we add up the different waves together, we begin to see a \textbf{packet}
forming.
\textbf{To gain knowledge about the \emph{position} of a particle, we must
  give up knowledge about its \emph{momentum}, and vice versa.}




%\begin{frame}{Uncertainty Principle}
Werner Heisenberg\footnote{1901--1976, Nobel Prize winner in 1932} showed that
because of wave properties of particles, an observer can never be completely
certain of \emph{both} an object's momentum $p$ and position $x$. The limit
to any measurement is given by the \textbf{uncertainty principle}:
\begin{equation}
  \boxed{\sigma_p\sigma_x\geq \frac h{4\pi}}
\end{equation}
\begin{center}
  \begin{tabular}{l|c|c}
    \rowcolor{pink}
    \textbf{Quantity} & \textbf{Symbol} & \textbf{SI Unit} \\ \hline
    Uncertainty in momentum & $\sigma_p$ & \si{\kilo\gram\metre\per\second}\\
    Uncertainty in position & $\sigma_x$ & \si\metre \\
    Planck's constant       & $h$        & \si{\joule\second}
  \end{tabular}
\end{center}
The more you know about an object's position, the less you know about its
momentum, and vice versa.




\section{Bohr Atomic Model}
%%
%%\begin{frame}{Atomic Model}
%%  J.\ J.\ Thomson: plum-pudding model (1897)
%%  \begin{itemize}
%%  \item Developed along with William Croakes
%%  \item Negatively-charged electrons are like raisins in a positively-charged
%%    ``dough''
%%  \end{itemize}
%%  Ernest Rutherford: planetary model (1911)
%%  \begin{itemize}
%%  \item The atom is mostly empty space
%%  \item Negatively-charged electrons orbiting a fixed, positively-charged
%%    nucleus in set, predictable paths (orbits)
%%  \end{itemize}
%%  The Rutherford model explains a lot more than the Thomson model, but misses
%%  out on a very important feature of an accelerating electron:
%%  \emph{it radiates energy as electromagnetic waves!}
%%
%%
%%
%%
%%\begin{frame}{Why the Planetary Model Doesn't Work}
%%  \begin{center}
%%    \pic{.23}{graphics/spiral}
%%  \end{center}
%%  According to classical electrodynamics:
%%  \begin{itemize}
%%  \item An accelerating electron has an oscillating electric field and an
%%    oscillating magnetic field
%%  \item Therefore it emits electromagnetic radiation
%%  \item The electron will lose energy and the orbit decays
%%  \item Eventually it'll collapse into the nucleus
%%  \end{itemize}
%%
%%
%%
%%
%%\begin{frame}{Bohr Atomic Model}
%%  Danish physicist Niels Bohr\footnote{1885--1962; Nobel Prize winner in
%%    1922} postulated that electrons can move in certain ``non-radiating''
%%  orbits, corresponding to energy levels:
\begin{equation}
  E_n=-\frac{k^2e^4m}{2\hslash^2}\frac{Z^2}{n^2}
\end{equation}
%%  \begin{center}
%%    \begin{tabular}{l|c|c}
%%      \rowcolor{pink}
%%      \textbf{Quantity} & \textbf{Symbol} & \textbf{SI Unit} \\ \hline
%%      Energy at level $n$ & $E_n$ & \si{\joule}    \\
%%      Coulomb's constant  & $k$   & \si{N.m^2/C^2}  \\
%%      Elementary charge   & $e$   & \si{\coulomb}   \\
%%      Atomic mass         & $m$   & \si{\kilo\gram} \\
%%      Reduced Planck's constant & $\hbar=h/2\pi$ & \si{\joule\second}\\
%%      Atomic number       & $Z$   & (integer)       \\
%%      Energy level        & $n$   & (integer)
%%    \end{tabular}
%%  \end{center}
%%
%%
%%
%%
%%\begin{frame}{Bohr Atomic  Model}
%%  Successful in describing the behaviour of the hydrogen atom---but fails for
%%  heavier atoms---although it still relies on 
%%  \begin{itemize}
%%  \item Coulomb forces between electrons and protons (classical)
%%  \item Centripetal forces (classical)
%%  \item Quantization of energy (new physics!)
%%  \end{itemize}
%%  De Broglie's hypothesis gives us a glimpse of what Bohr is missing
%%  \begin{itemize}
%%  \item The ``orbits'' correspond to a standing wave around the nucleus
%%  \item A standing wave does not lose energy
%%  \end{itemize}
%%
%%
%%
%%
%%\begin{frame}{Standing Wave on a String}
%%  
%%    \column{.3\textwidth}
%%    \pic{1}{graphics/strhar}
%%
%%    \column{.7\textwidth}
%%    \begin{itemize}
%%    \item We have studied standing waves in Grade 11
%%    \item If electron is to be in a ``stable orbit'' around a nucleus, it has
%%      to be in a standing wave pattern
%%    \item Otherwise, it will interfere with itself
%%    \end{itemize}
%%  
%%
%%
%%
%%\begin{frame}{Circular Standing Wave}
%%  \begin{center}
%%    \pic{.7}{graphics/oo1wp}
%%  \end{center}
%%  Electron resonance states $n=3,4,5,6$
%%
%
%
%
\section{Particle in a Box}
%  
\begin{figure}[ht]
  \centering
  \begin{tikzpicture}
    \foreach \y in {1,...,5}{
      \draw[dashed](0,\y-.5)--(3,\y-.5) node[pos=0,left]{$n=\y$};
      \draw[very thick,violet,smooth,samples=80,domain=0.01:2.99]
      plot({\x},{.4*sin(60*\y*\x)+\y-.5});
    }
    \fill[gray!40] (-.1,5) rectangle (0,0);
    \fill[gray!40] (-.1,0) rectangle (3.1,-.1);
    \fill[gray!40] (3,-.1) rectangle (3.1,5);
    \draw[thick] (0,5)
    --(0,0) node[pos=0,above]{$\psi(x)$} node[below]{0}
    --(3,0) node[below]{$L$}--(3,5);
  \end{tikzpicture}
\end{figure}

A particle confined to a 1D box behaves like a standing wave. The resonance
modes (frequencies where a stable standing wave exists) correspond to the
wavelengths:
\begin{equation}
  \lambda=\frac{2L}n\quad n=1,2,3\ldots
\end{equation}
and the momentum of the particle at the $n$-th mode is:
\begin{equation}
  p_n=\frac h\lambda=\frac{nh}{2L}
\end{equation}
%  
%
%
%
%
%\begin{figure}[ht]
%  \centering
%  \begin{tikzpicture}
%    \foreach \y in {1,...,5}{
%      \draw[dashed](0,\y-.5)--(3,\y-.5) node[pos=0,left]{$n=\y$};
%      \draw[very thick,violet,smooth,samples=80,domain=0.01:2.99]
%      plot({\x},{.4*sin(60*\y*\x)+\y-.5});
%    }
%    \fill[gray!40](-.1,5)rectangle(0,0);
%    \fill[gray!40](-.1,0)rectangle(3.1,-.1);
%    \fill[gray!40](3,-.1)rectangle(3.1,5);
%    \draw[thick](0,5)--(0,0) node[pos=0,above]{$\psi(x)$}
%    node[below]{0}--(3,0)node[below]{$L$}--(3,5);
%  \end{tikzpicture}
%\end{figure}   

Kinetic energy of the particle can be expressed in terms of momentum:
\begin{equation}
  E_n=\frac12mv^2=\frac{p_n^2}{2m}=\frac{n^2h^2}{8mL^2}
\end{equation}
If a particle is a standing wave, then kinetic energy of the particle can
never be zero, therefore
\begin{itemize}
\item The particle cannot have zero velocity
\item The lowest energy level ($n=1$) is called the
  \textbf{zero-point energy}
    \end{itemize}

\begin{example}
  A \SI{.150}{\kilo\gram} billiard ball is confined to the pool table
  \SI{1.42}{\metre} wide. How long (in seconds) will it take to travel from one
  side of the table to the other side at fundamental mode?
\end{example}
%
%
%
\section{Schr\"{o}dinger}
%%
%%\begin{frame}{What Else Can You Learn from Quantum Mechanics}
%%  {Schr\"{o}dinger's Equation}
%%  \begin{itemize}
%%  \item The differential equations that governs the quantum state of a
%%    quantum system changes in time
%%  \item It's like what Newton's second law of motion and conservation of
%%    energy to classical mechanics
%%  \item Gives full details of the behaviour of electrons in atoms
%%  \item ``Schr\"{o}dinger's Cat'' thought experiment
%%  \end{itemize}
