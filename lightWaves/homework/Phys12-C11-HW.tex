\section{Problems}

\begin{enumerate}
\item Of the following properties of a wave, the one that is
  \emph{independent} of the others is its \underline{\hspace{1in}}
  \begin{enumerate}
  \item amplitude
  \item speed
  \item wavelength
  \item frequency
  \item period
  \end{enumerate}
  
\item Light travelling in one material enters another material in which it
  travels faster. The light wave will \underline{\hspace{.5in}}
  \begin{enumerate}
  \item increase in frequency
  \item increase in wavelength
  \item decrease in frequency
  \item decrease in wavelength
  \item travel through the new material inverted
  \end{enumerate}

\item A laser beam passes through a prism and produces a bright dot of light a
  distance of $x$ from the prism, as shown in the figure below. Which of the
  following correctly explains the change in distance as the angle $\theta$ of
  the prism is decreased?
  \begin{center}
    \begin{tikzpicture}[scale=.8]
      \draw[thick,fill=gray!60] (0,0)--(0,5)--(2,0)--cycle;
      \draw[thick] (0,3.5) arc (270:270+atan(2/5):1.5)
      node[midway,below]{$\theta$};
      \draw[vectors] (-1,2.5)--(1,2.5)--(4.5,0);
      \draw[thick,|<->|] (2,-.2)--(4.5,-.2) node[midway,fill=white]{$x$};
    \end{tikzpicture}
  \end{center}
  \begin{enumerate}
  \item The distance increases because the angle of incidence increases.
  \item The distance increases because the angle of incidence decreases.
  \item The distance decreases because the angle of incidence increases.
  \item The distance decreases because the angle of incidence decreases.
  \end{enumerate}
  
\item Which of the following could be the path of a light ray passing through a
  glass prism with an index of refraction of $n=1.5$? The glass is an
  equilateral triangle. %\emph{Select two answers.}
  
  \begin{tikzpicture}[scale=2]
    \node at (-.3,.433){A.};
    \draw (0,0)--(1,0)--(.5,.866)--cycle;
    \draw[vectors,red] (0,.87)--(.25,.433)--(.75,.433)--(1,.87);
  \end{tikzpicture}
  \hspace{.1in}
  \begin{tikzpicture}[scale=2]
    \node at (-.3,.433){B.};
    \draw (0,.866)--(1,.866)--(.5,0)--cycle;
    \draw[vectors,red] (1.2,.7)--(.75,.433)--(.25,.433)--(-.2,.7);
  \end{tikzpicture}
  \hspace{.1in}  
  \begin{tikzpicture}[scale=2]
    \node at (-.3,.433){C.};
    \draw (0,.866)--(1,.866)--(.5,0)--cycle;
    \draw[vectors,red] (1.15,.7)--(.75,.433)--(.3,.35)--(0,0);
  \end{tikzpicture}
  \hspace{.1in}
  \begin{tikzpicture}[scale=2]
    \node at (-.3,.433){D.};
    \draw (0,0)--(1,0)--(.5,.866)--cycle;
    \draw[vectors,red] (.25,-.25)--(.25,.433)--(.75,.433)--(.75,-.25);
  \end{tikzpicture}
  
  \item Two waves travel towards each other, as shown in the figure. Sketch
  at least three unique interference patterns that will be seen as the waves
  pass each other.
  \begin{center}
    \begin{tikzpicture}[scale=1.1]
      \begin{scope}[very thick]
        \draw (-1,0)--(0,0);
        \draw[domain=0:2] plot(\x,{.7*sin(180*\x)});
        \draw (2,0)--(4,0);
        \draw[domain=4:6] plot(\x,{-.7*sin(180*\x)});
        \draw (6,0)--(7,0);
        \draw[vectors] (.5,1)--(1.5,1) node[midway,above]{$v$};
        \draw[vectors] (5.5,1)--(4.5,1) node[midway,above]{$v$};
      \end{scope}
    \end{tikzpicture}
  \end{center}
  
\item A beam of monochromatic red light with a wavelength of
  \SI{650}{\nano\metre} in air travels in water.
  \begin{enumerate}
    \item What is the wavelength in water?
    
    \item Does a swimmer underwater observe the same or different colour for
    this light? Explain.
  \end{enumerate}
  
%  \item An incident ray of light strikes the surface of a triangular glass prism,
%  as shown in the figure. Light exits all three sides of the prism.
%  \begin{center}
%    \vspace{-.2in}\pic{.25}{prism}
%  \end{center}
%  \begin{enumerate}
%  \item Sketch the path of the light ray as it travels through the prism to
%    show how light can exit the prism at three locations. Mark all normal
%    lines, making sure the angles you draw are proportionally correct.
%  \item Complete your drawing by sketching any reflected rays that you have not
%    already drawn on the figure. Make sure the angles you draw are
%    proportionally correct.
%  \end{enumerate}
  
  \item Explain Huygens' principle and point source model, and use the model
  to sketch how a plane wave and a circular wave front propagate forward.
  
  % Tipler p.1009 Question 22  
  \item A silver coin sits on the bottom of a swimming pool that is
  \SI{4.0}{\metre} deep. A beam of light reflected from the coin emerges from
  the pool making a \ang{20} angle with respect to the water's surface and
  enters the
  eye of the observer. Draw a ray from the coin to the eye of the observer.
  Extend this ray, which goes from the water-air vertical line drawn through
  the coin, straight back until it intersects with the vertical line drawn
  through the coin. What is the apparent depth of the swimming pool to this
  observer?
  
\item A point source of light is located at the bottom of a steel tank, and an
  opaque circular card of radius \SI{6.0}{\centi\metre} is placed over it. A
  transparent fluid is gently added to the tank such that the card floats on
  the surface with its centre directly above the light source. No light is
  seen by an observer until the fluid is \SI{5.0}{\centi\metre} deep. What is
  the index of refraction of the fluid?

%  \item Given that the refractive index of red light in water is $1.3318$
%  and that the refractive index of blue light in water is $1.3435$, find the
%  angular separation $\delta$ of these colours in the primary rainbow.
  
\item A light ray in dense flint glass with refractive index $n_g=1.655$ is
  incident on the glass surface. An unknown liquid condenses on the surface of
  the glass. Total internal reflection on the glass-liquid surface occurs at an
  angle of incidence on the surface of the glass-liquid surface at \ang{53.7}.
  \begin{enumerate}
  \item What is the refractive index of the unknown liquid?
  \item If the liquid is removed, what is the angle of incidence for total
    internal reflection?
  \item For the angle of incidence found in part (b), what is the angle of
    refraction of the ray into the liquid film? Does a ray emerge from the
    liquid film into the air above?
  \end{enumerate}
  Assume the glass and liquid have perfectly planar surfaces.
  
%  \item Light passes symmetrically through a prism having an apex angle of
%  $\alpha$ as shown in the figure.
%  \begin{enumerate}[noitemsep,topsep=0pt,leftmargin=18pt]
%  \item Show that the angle of deviation $\delta$ is related to the apex angle
%    $\alpha$ by:
%    \begin{displaymath}
%      \sin\frac{\alpha+\gamma}2=n\sin\frac{\alpha}2
%    \end{displaymath}
%  \item If the index of refraction for red light is $1.48$ and for violet is
%    $1.52$, what is the angle of separation of visible light for a prism with
%    an apex angle of \ang{60}?
%  \end{enumerate}
%  \begin{tikzpicture}[scale=2.3]
%    \draw[fill=green!50!gray!40] (0,0)--(2,0)--(1,2)--cycle;
%    \draw[thick](.85,1.7) arc (240:308:.26) node[midway,below]{$\alpha$};
%    \draw[vectors,red] (.6,1.2)--(1,1.2);
%    \draw[very thick, red] (1,1.2)--(1.4,1.2);
%    \draw[dashed,thick] (1.4,1.2)--(2.6,1.6);
%    \draw[thick] (2,.9) arc (330:393:.48) node[midway,right]{$\gamma$};
%    \draw[vectors,red] (1.4,1.2)--(2.3,.75);
%    \draw[vectors,red] (-.3,.9)--(.6,1.2);
%    \draw[dashed,thick] (.6,1.2)--(2.6,1.87);
%  \end{tikzpicture}
  
  \item A student is given a semicircular glass prism and a laser. The
  student directs the laser perpendicular to the curved surface, as shown in the
  figure below.
  
  \begin{minipage}{.3\textwidth}
    \begin{tikzpicture}[scale=2.5]
      \draw[mass] (-1,0)--(1,0) arc (0:180:1);
      \draw[vectors] (-1,1.5)--(0,0);
      \node at (-.5,.1) {Glass};
      \node at (-.5,-.1) {Air};
    \end{tikzpicture}
  \end{minipage}
  \begin{minipage}{.65\textwidth}
    \begin{enumerate}
      \item Sketch the paths of the ray exiting the prism.
      \item Explain the paths of the light exiting the prism, making reference
      to the speed of light in the glass and air.
      \item The index of refraction of the glass can be found by graphing a
      straight line. Indicate what quantities should be graphed to produce a
      straight line graph and how the graph could be used to determine the index
      of refraction of the glass.
    \end{enumerate}
  \end{minipage}

\item A laser is directed at a rectangular block of glass, as shown in the
  figure below. When the laser is directed along path A, the light exits both
  the top and left side of the glass block. When directed along path B, the
  light only exits the left side of the glass block.
  \begin{center}
    \begin{tikzpicture}[scale=1.2]
      \draw[mass] rectangle (5,3.5);
      \draw[vectors] (5.45,.2)--(5,1.75) node[pos=0,below]{A};
      \draw[vectors,dashed] (5.9,2)--(5,2.6) node[pos=0,right]{B};
    \end{tikzpicture}
  \end{center}
  \begin{enumerate}
  \item Using a solid line, sketch the light along path A. Be sure to show
    enough detail so it is apparent how the light exits both the top and the
    left side of the block.
  \item Using a dashed line, sketch the light along path B. Be sure to show
    enough detail so it is apparent how the light exits the left side of the
    block.
  \item Explain why the light behaves differently depending on the path.
  \item Which path will produce the brightest beam exiting the left side of
    the block? Justify your reasoning.
  \end{enumerate}

\item A ray of light falls on a rectangular glass block ($n_g=1.50$) that is
  almost completely submerged in water ($n_w=1.33$) as shown below.
  \begin{enumerate}
  \item Find the angle $\theta$ for which total internal reflection just
    occurs at point $P$.
  \item Would the total internal reflection at point $P$ occur for the value
    of $\theta$ in part (a) if the water were removed? Explain.
  \end{enumerate}
  \begin{tikzpicture}
    \draw[mass] rectangle (6,3);
    \draw (0,3)--(6,3) node[pos=.1,below]{water};
    \draw[thick,fill=green!50!gray!40] (1.2,.3) rectangle (5,3.3)
    node[midway,below]{glass};
    \draw[dashed] (3,2)--(3,5);
    \draw[dashed] (.3,1)--(2.5,1);
    \draw[vectors,red] (6,4.8)--(3,3.3)--(1.2,1)--(1.52,.6);
    \fill (1.2,1) circle (.07) node[above left]{$P$};
    \draw[axes] (3,4) arc (90:27:.7) node[midway,above]{$\theta$};
  \end{tikzpicture}
  
%  % Tipler p.1010 Question 43
%  \item In the figure below, light is initially in a medium (such as air) of
%  refractive index $n_1$. It is incident at angle $\theta_1$ on the surface of
%  a liquid (such as water) of refractive index $n_2$. The light passes through
%  the layer of liquid and enters glass of refractive index $n_3$. If $\theta_3$
%  is the angle of  refraction in the glass, show that
%  $n_1\sin\theta_1=n_3\sin\theta_3$. That is, show that the second medium can be
%  neglected when finding the angle of refraction in the third medium.
%  
%  \begin{tikzpicture}[scale=.8]
%    \fill[blue!50!gray!40] (0,0) rectangle (8,3);
%    \fill[green!50!gray!40] (0,3) rectangle (8,4.5);
%    \draw (0,3)--(8,3);
%    \draw (0,4.5)--(8,4.5);
%    \draw[dashed] (4,6)--(4,3);
%    \draw[dashed] (5,4.5)--(5,0);
%    \draw[very thick,red] (2,6)--(4,4.5)--(5,3)--(5.8,0);
%    \node at (9.2,5.25) {$n_1$};
%    \node at (9.2,3.75) {$n_2>n_1$};
%    \node at (9.2,1.5)  {$n_3>n_2$};
%    \draw[<->] (4,5.5) arc (90:144:1)  node[pos=0,right]{$\theta_1$};
%    \draw[<->] (4,3.5) arc (270:305:1) node[pos=0,left]{$\theta_2$};
%    \draw[<->] (5,1) arc (270:285:2) node[pos=0,left]{$\theta_3$};
%  \end{tikzpicture}
\end{enumerate}
