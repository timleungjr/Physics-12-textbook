\newpage
\section*{Problems}
For questions involving relativistic speeds, please answer your questions
relative to the speed of light, e.g.\ ``$v=0.56c$''.

\begin{enumerate}[itemsep=6pt]
\item Which one of the following systems would constitute an inertial
  reference frame?
  \begin{enumerate}
  \item A weather balloon descending at constant velocity
  \item A rocket undergoing uniform acceleration
  \item An orbiting space station
  \item A rotating merry-go-round
  \item A roller coaster travelling around a corkscrew turn at constant speed
  \end{enumerate}
  
%%  \item Which one of the following is a consequence of the postulates of
%%  special relativity?
%%  \begin{enumerate}
%%  \item There is no such thing as an inertial reference frame.
%%  \item Newton's laws of motion apply in every reference frame.
%%  \item Coulomb's law of electrostatics applies in any reference
%%    frame.
%%  \item The question of whether an object is at rest in the universe is
%%    meaningless.
%%  \item The value of every physical quantity depends on the reference frame in
%%    which it is measured.
%%  \end{enumerate}
%%
%%  \item Which one of the following statements is a consequence of special
%%  relativity?
%%  \begin{enumerate}
%%  \item Clocks that are moving run slower than when they are at rest.
%%  \item The length of a moving object is larger than it was at rest.
%%  \item Events occur at the same coordinates for observers in all inertial
%%    reference frames.
%%  \item Events occur at the same time for observers in all inertial reference
%%    frames.
%%  \item The speed of light has the same value for observers in all reference
%%    frames.
%%  \end{enumerate}

\item Two events that are simultaneous in one frame of reference will be
  \begin{enumerate}
  \item simultaneous in all frames of reference
  \item simultaneous in another frame that is moving in the opposite
    direction
  \item simultaneous in another frame of reference that is moving in the
    same direction 
  \item simultaneous in the same frame of reference
  \item simultaneous in another non-inertial frame of reference
  \end{enumerate}
  
\item An astronaut in a rocket is passing by a space station at a velocity of
  $0.33c$. Looking out the window, the astronaut sees a scientist on the space
  station fire a laser at a target. The laser is pointed in the same direction
  that the astronaut is travelling. On which of the following observations will
  the astronaut and scientist agree?
  \begin{enumerate}
  \item The length of the rocket.
  \item The time it takes the laser to hit the target.
  \item The speed of the laser beam.
  \item The astronaut and scientist will not agree on any of these measurements.
  \end{enumerate}
  
\item The headlights are shining on a truck travelling at
  \SI{100}{\kilo\metre\per\hour}. The speed of the light from the headlights
  relative to the road will be
  \begin{enumerate}
  \item $c+\SI{100}{\kilo\metre\per\hour}$
  \item $c-\SI{100}{\kilo\metre\per\hour}$
  \item $c$
  \item depends on the temperature, but faster than the speed if the truck
    was not moving.
  \item Impossible to calculate with the given information 
  \end{enumerate}
%
%  \uplevel{
%    \centering
%    \begin{tikzpicture}
%      \draw[thick] rectangle(3,.2);
%      \draw[vectors] (3.2,.1)--+(1,0) node[right]{$v=0.60c$};
%    \end{tikzpicture}
%  }
\item You see a horizontal metre stick moving at $0.60c$ towards you, as
  shown above. Its length appears to be
%  \begin{enumerate}
%    \item\SI{.60}\metre
%    \item\SI{.64}\metre 
%    \item\SI{.80}\metre 
%    \item\SI{1.00}\metre 
%    \item\SI{1.25}\metre
%  \end{enumerate}
%  \newpage
%
%  \uplevel{
%    \centering
%    \begin{tikzpicture}
%      \draw[thick] rectangle(.2,2.5);
%      \draw[vectors] (.4,1.25)--+(1,0) node[right]{$v=0.60c$};
%    \end{tikzpicture}
%  }
\item You see a vertical metre stick moving at $0.60c$ towards you, as
  shown above. Its length appears to be
  \begin{enumerate}
  \item\SI{.60}\metre 
  \item\SI{.64}\metre 
  \item\SI{.80}\metre 
  \item\SI{1.00}\metre 
  \item\SI{1.25}\metre
  \end{enumerate}
  
\item An alarm clock is set to ring for a total of 1.00 min. It is placed
  on a spaceship moving at $0.60c$. According to Mission Control on Earth, the
  alarm clock rings for
  \begin{enumerate}
  \item 0.60 min
  \item 0.80 min
  \item 1.00 min
  \item 1.25 min
  \item 2.00 min
  \end{enumerate}

\item Two spaceships are flying towards one another at speeds of $c/2$. Each
  ship will see the light from the other ship travelling at
  \begin{enumerate}
  \item $2c$
  \item $c/2$
  \item $c$    
  \item No light will be visible 
  \item None of the above
  \end{enumerate}
%
%  %\item The rest energy of an electron ($m=\SI{9.11e-31}{\kilo\gram}$) in
%  %joules is: 
%  %\begin{enumerate}
%  %  \item\SI{1.37e-22}\joule 
%  %  \item\SI{2.73e-22}\joule
%  %  \item\SI{2.79e-14}\joule 
%  %  \item\SI{4.1e-14}\joule
%  %  \item\SI{8.20e-14}\joule
%  %\end{enumerate}
%
%  %\item The total energy of a 0.050 kg object travelling at $0.70c$ is
%  %\begin{enumerate}
%  %  \item\SI{2.10e7}\joule
%  %  \item\SI{3.06e7}\joule
%  %  \item\SI{2.46e15}\joule
%  %  \item\SI{6.30e15}\joule
%  %  \item\SI{8.82e15}\joule
%  %\end{enumerate}
%
%  %\item The rest energy of an electron ($m_e=\SI{9.11e-31}{\kilo\gram}$) in
%  %\si{\electronvolt} is ($\SI1\electronvolt=\SI{1.602e-19}\joule$)
%  %\begin{enumerate}
%  %  \item\SI{0.856}{\milli\electronvolt}
%  %  \item\SI{1.700}{\milli\electronvolt}
%  %  \item\SI{0.174}{\mega\electronvolt}
%  %  \item\SI{0.256}{\mega\electronvolt}
%  %  \item\SI{0.511}{\mega\electronvolt}
%  %\end{enumerate}
  
\item What is proper length?
  \begin{enumerate}
  \item The exact length of an object in all frames of reference.
  \item The length of a stationary object measured from a moving frame of
    reference.
  \item A length unit that is suitable for a given measurement.
  \item The length of an object measured by an observer who is stationary
    relative to the object.
  \end{enumerate}
  
\item An object that is \SI{50}{\centi\metre} long passes an observer who
  measures it as \SI{51}{\centi\metre}. Which statement is most likely to be
  correct?
  \begin{enumerate}
  \item The object is moving very quickly.
  \item The object is travelling backwards.
  \item The object's length is dilated by relativity.
  \item The observer's time is slowed.
  \item The observer made an error in measurement.
  \end{enumerate}

\item According to the principle of conservation of mass-energy, rest
  energy is equivalent to which of the following?
  \begin{enumerate}
  \item Rest mass
  \item Relativistic mass
  \item Relativistic mass plus rest mass
  \item Relativistic mass minus rest mass
  \item Relativistic momentum
  \end{enumerate}
  
\item Which of these effects would be felt by an astronaut travelling at a
  constant speed of $0.50c$?
  \begin{enumerate}
  \item Her weight would increase.
  \item Her height would decrease.
  \item She will move more slowly.
  \item All of the above
  \item None of the above
  \end{enumerate}
  
\item A spacecraft passes a spherical space station. Observers in the
  spacecraft measure the station's minor axis as \SI{145}{\kilo\metre} and the
  major axis as \SI{190}{\kilo\metre}.
  \begin{enumerate}[itemsep=3pt]
  \item How fast is the spacecraft travelling relative to the space station?
  \item Why does the station not look like a sphere to the observers in the
    spacecraft?
  \end{enumerate}
  \pic{.25}{specialRelativity/graphics/Death_Star_Render_01}
    
\item A rocket passes by Earth at a speed of $0.60c$. If a person on the rocket
  takes \SI{245}{\second} to drink a cup of coffee, according to his watch, how
  long would that same event take according to an observer on Earth?
  
\item A kaon particle has a lifetime at rest in a laboratory of
  \SI{1.2e-8}\second. At what speed must it travel to have its lifetime
  measured as \SI{3.6e-8}\second?
  
\item An electron is moving at $0.95c$ parallel to a meter stick. How long will
  the meter stick be in the electron's frame of reference?
 
\item A speck of dust in space has a rest mass of \SI{463}{\micro\gram}. If it
  is approaching Earth with a relative speed of $0.100c$, what will be its mass
  as measured in the Earth frame of reference? (In questions like this, it does
  not matter what the actual mass units are, as long as they are the same for
  both the rest mass and relativistic mass.)
  
\item How fast should a particle be travelling relative to an experimenter in
  order to have a measured mass that is 20.0 times its rest mass?

\item A nuclear power reactor generates \SI{3.}{\giga\watt} of power. In
  one year, what is the change in the mass of the nuclear fuel to generate the
  energy?
  
\item A certain chemical reaction requires 13.8 J of thermal energy.
  \begin{enumerate}[itemsep=3pt]
  \item What mass gain does this represent?
  \item Why would the chemist still believe in the law of conservation of
    mass? 
  \end{enumerate}

\item A physicist measures the mass of a speeding proton as being
  \SI{2.20e-27}{\kilo\gram}. If its rest mass is \SI{1.68e-27}{\kilo\gram}, how
  much kinetic energy does the proton possess?
 
\item How fast must a neutron be travelling relative to a detector in order to
  have a measured kinetic energy that is equal to its rest energy?

\item If the mass loss during a nuclear reaction is \SI{14}{\micro\gram},
  how much energy is released? (Be careful with unit conversion.)
 
%  %\item Calculate the mass increase for a completely inelastic head-on
%  %collision of two \SI{.50}{\kilo\gram} piece of Play-Doh each moving towards
%  %the other at \SI{8.5e7}{\metre\per\second} (as measured in your rest frame)
%  %in opposite direction.
  
\item Two spaceships are each \SI{25}{\metre} long, as measured in their own
  rest frame. Ship A is approaching Earth at $0.65c$, while Ship B is
  approaching Earth from the opposite direction at $0.50c$. Find the length of
  B as measured in
  \begin{enumerate}[itemsep=3pt]
  \item Earth's frame of reference
  \item Ship A's frame of reference
  \end{enumerate}

\item Alpha Centauri is the nearest star system from our solar system, at
  about 4.2 light years from Earth, measured in the common rest frame of Earth
  and Alpha Centauri. Suppose you took a fast spaceship to Alpha Centauri, so
  it got you there in 7.0 years as measured on the ship. If you sent a radio
  message home as soon as you reached Alpha Centauri, how long after you left
  Earth would it arrive, according to the timekeepers on Earth? (This is a
  more difficult question. To simplify the problem, instead of using SI units,
  use $c=1$, time $t$ in \emph{years}, distances $L$ in \emph{light years}, and
  express all speeds $v$ in terms of $c$, e.g.\ use $v=0.50$ for
  \SI{50}{\percent} of the speed of light.)
\end{enumerate}
