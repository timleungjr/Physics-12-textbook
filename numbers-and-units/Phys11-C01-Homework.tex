\documentclass[12pt]{../ossphysics}

\begin{document}
\setheader{Physics 11 Class 1 Homework}
\hwtitle{11}{1}{Motion Quantities}
\begin{center}
  \fcolorbox{black}{yellow!10}{
    \begin{minipage}{.98\linewidth}
      For \emph{vector} quantities (position, displacement, velocity,
      acceleration), you must answer with both \emph{magnitude} and
      \emph{direction}. For 1D problems, you may answer using $+$/$-$
      directions, but you must indicate the positive direction. Scalar
      quantities (speed, distance, time) do not have a direction. Write neatly.
      Put a box around your answers. Answer with the appropriate number of
      significant figures.
    \end{minipage}
  }
\end{center}
\begin{questions}
  \question State the number of significant figures. If there are infinite
  number of sig.\ figs., write ``$\infty$''; if there is not enough
  information to tell, write an ``X''. %\textbf{Read the additional slides
  %on significant figures first.}
  The numbers in \textbf{BOLD FACE} %highlighted in
  %\textcolor{red}{red}
  are to be treated as integers. Pay attention to units; they may give you
  clues about how to treat the numbers.
  
  \vspace{.15in}
  \begin{minipage}{.32\textwidth}
    \begin{enumerate}[
        itemsep=20pt,
        label={(\alph*) \underline{\hspace{.6in}}},
        leftmargin=70pt]
    \item\SI{4206}{\kilo\metre}
    \item\SI{2.998e8}{\metre\per\second}
    \item$\dfrac\pi{\mathbf 2}$
    %\item\SI{5.00}{\second}
    \end{enumerate}
  \end{minipage}
  \begin{minipage}{.32\textwidth}
    \begin{enumerate}[itemsep=20pt,label={(\alph*) \underline{\hspace{.6in}}},
        leftmargin=70pt,start=5]
    \item$\sqrt{\mathbf 2}$
    %\item\SI{1000}{lumens}
    \item$e^{\mathbf 4}$
    \item\SI{21}{cars}
    \end{enumerate}
  \end{minipage}
  \begin{minipage}{.32\textwidth}
    \begin{enumerate}[itemsep=20pt,label={(\alph*) \underline{\hspace{.6in}}},
        leftmargin=70pt,start=9]
    %\item\SI{9.30e-6}\joule
    \item\SI{.0050010}\tesla
    \item\SI{-273.15}\celsius
    \item\SI{3600}\newton
    \end{enumerate}
  \end{minipage}
  \vspace{.15in}

  \question Express the following arithmetic operations with the correct number
  of significant figures and units:
  
  \vspace{.15in}\begin{minipage}{.45\textwidth}
    \begin{enumerate}[itemsep=20pt,label={(\alph*)}]
    \item $\SI{4.13}{\metre\per\second}\times\SI{6.2}\second=$
    \item $\SI{8.9921}\metre-\SI{2.63}\metre=$
    \item $\SI{26.195}{\metre\per\second\squared}\times\SI{89}{\second}=$
    \end{enumerate}
  \end{minipage}
  \begin{minipage}{.45\textwidth}
    \begin{enumerate}[itemsep=20pt,start=4,label={(\alph*)}]
    \item $\SI{6.4}\metre\times\SI{3.62}\metre=$
    \item $\SI{1.419934}\metre\div\SI{62}{\metre\per\second}=$
    \item $\SI{1.897}\ohm+\SI{.2119}\ohm=$
    \end{enumerate}
  \end{minipage}
  \vspace{.15in}
  
  \question Convert the following quantities into different common units.
  Answer with the appropriate number of significant figures. (i.e.\ the number
  of sig.\ figs.\ after the conversion is the same as what is given to
  you.) Conversion factors from imperial to SI units are \emph{exact}.
  \begin{enumerate}[itemsep=20pt,label={(\alph*)}]
  \item $\SI{90}{km/h}=\underline{\hspace{1in}}\;\si{\metre\per\second}$
  \item $\SI{30.0}{\metre\per\second}=\underline{\hspace{1in}}\;\si{km/h}$
  \item $\SI{24.0}\hour=\underline{\hspace{1in}}\;\si\second$
  \item $\SI{16}{ft}=\underline{\hspace{1in}}\;\si\metre$
    ($\SI1{ft}=\SI{12}{in}$, $\SI1{in}=\SI{0.0254}\metre$)
  \item $\SI{7.3e9}\joule=\underline{\hspace{1in}}\;\si{\kilo\watt\hour}$
    ($\SI1{\kilo\watt\hour}=\SI{3.6e6}\joule$)
  %\item $\SI{6.5}{\kilo\watt\hour}=\underline{\hspace{1in}}\;\si{J}$
  \end{enumerate}
  \newpage

  \question The Earth is approximately a sphere of radius \SI{6.37e6}\metre.
  \begin{parts}
    \part What is its circumference in metres?
    \vspace{\stretch1}
    
    \part What is its surface area in square metres?
    \vspace{\stretch1}

    \part What is its volume in cubic metres?
    \vspace{\stretch1}
  \end{parts}
  
  \question One US gallon of paint ($\SI1{gal}=\SI{3.78e-3}{\metre\cubed}$)
  covers an area of \SI{25.0}{\metre\squared}. What is the thickness of the
  paint on the wall?
  \vspace{\stretch1}
  
  \question Calculate the number of kilometres in \SI{20.0}{mi} using
  \emph{only} the following conversion factors:\\ $\SI1{mi}=\SI{5280}{ft}$,
  $\SI1{ft}=\SI{12}{in}$, $\SI1{in}=\SI{2.54}{\centi\metre}$,
  $\SI1\metre=\SI{100}{\centi\metre}$, and $\SI1{\kilo\metre}=\SI{1000}\metre$.
  \vspace{\stretch{1.5}}

  \question How many grams of copper are required to make a \emph{hollow}
  spherical shell with an inner radius of $r_i=\SI{5.70}{\centi\metre}$ and an
  outer radius of $r_o=\SI{5.75}{\centi\metre}$? The density of copper is
  $\rho=\SI{8.93}{\gram\per\centi\metre\cubed}$. (Mass and volume are related
  by $m=\rho V$.)
  \vspace{\stretch3}
  \newpage
  
  \question A rock is thrown straight upward from the edge of a 30 m cliff,
  rising 10 m then falling all the way down to the base of the cliff.
  \begin{parts}
    \part What is the \emph{distance} that the rock travelled?
    \vspace{\stretch1}
    
    \part What is it \emph{displacement} after it hits the base of the cliff?
    \vspace{\stretch1}
  \end{parts}
  
  \question A student athlete runs around a \SI{400}{\metre} track and
  completes it in \SI{53}\second. Find her average speed and velocity:
  \vspace{\stretch{1.5}}
  
  \question A stalled car starts to roll backwards down a hill. At the instant
  that it has a velocity of \SI{4.0}{\metre\per\second} down the hill, the
  driver is able to start the car and start accelerating backup. After
  accelerating for \SI{3.0}\second, the car is travelling uphill at
  \SI{3.5}{\metre\per\second}. Determine the car's average acceleration once
  the driver got it started.
  \vspace{\stretch1}

  %\question Is it possible to have an average velocity of zero for some motion
  %but an average speed of 120 km/h for that same motion? Provide a quantitative
  %example.
  
  \question A basketball player gains the ball at centre court. He then
  dribbles down to the opponents' basket and scores \SI{6.0}{\second} later.
  After scoring, he runs back to guard his own team's basket, taking
  \SI{9.0}{\second} to run down the court. A basketball court is \SI{30}{\metre}
  long. Find the average velocity of the basketball player
  %Using centre court as his reference position, calculate his average
  %velocity
  \begin{parts}
    \part while he is dribbling up the opponents' basket
    \vspace{\stretch1}

    \part while he is running down from the opponents' basket to his own team's
    basket
    \vspace{\stretch1}

    \part for the entire sequence of motion
    \vspace{\stretch1}
  \end{parts}
  \newpage
  
%  \question A field hockey player runs \SI{45}{\metre} forward and then straight
%  back \SI{65}{\metre} during a game. If we consider forward to be the positive
%  direction,
%  \begin{parts}
%    \part What is her displacement?
%    \vspace{\stretch1}
%    
%    \part What is the distance travelled?
%    \vspace{\stretch1}
%  \end{parts}
%  
%  \question If a motorcycle with an initial velocity of
%  \SI{25}{\metre\per\second} forward changes its velocity to
%  \SI{55}{\metre\per\second} in \SI{4.5}\second, What is the average
%  acceleration of the motorcycle?
%  \vspace{\stretch{1.5}}

  %\question At the end of the school day, at exactly 2:30 pm, a group of
  %students run out of the school building and reach the edge of the school
  %property at 2:30:45 pm. Which of the following correctly describes the motion
  %in terms of time?
  %\begin{choices}
  %  \choice $\Delta t=2$:$30$
  %  \choice $t_1=2$:$30$, $t_2=\SI{45}{\second}$
  %  \choice $t_2=2$:$30$:$45$ pm, $\Delta t=\SI{45}{\second}$
  %  \choice $t_1=0$, $t_2=2$:$30$:$45$ pm, $\Delta t=\SI{45}{\second}$
  %  \choice None of the above
  %\end{choices}

  \question A car is travelling west and approaching a stop sign. As it is
  slowing to a stop, the directions associated with the object's velocity and
  acceleration, respectively, are
  \begin{choices}
    \choice West, East
    \choice West, West
    \choice East, East
    \choice East, West
    \choice There is not enough information to tell.
  \end{choices}
  
  %\question An athlete runs around a 400 m standard oval track 4 times. Her
  %distance and displacement are, respectively,
  %\begin{choices}
  %  \choice 0, 0
  %  \choice 1600 m, 0
  %  \choice 0, 1600 m
  %  \choice 1600 m, 1600 m [forward]
  %  \choice 100 m, 0
  %\end{choices}
  
  \question If a car travelling at \SI{60}{\kilo\metre\per\hour} [S] stops in
  a time of \SI{3.5}\second, its acceleration is:
  \begin{choices}
    \choice\SI{4.8}{\metre\per\second\squared} [S]
    \choice\SI{4.8}{\metre\per\second\squared} [N]
    \choice\SI{17}{\metre\per\second\squared} [S]
    \choice\SI{17}{\metre\per\second\squared} [N]
    %\choice\SI{17.1}{\metre\per\second\squared} [S]
  \end{choices}
  
  %\question A car travels 35 km [N] in 30 minutes and then hits a traffic jam
  %and spends 90 minutes travelling \SI{16.7}{\kilo\metre\per\hour} [N]. The
  %average velocity of the car is:
  %\begin{choices}
  %  \choice \SI{43.4}{\kilo\metre\per\hour} [N]
  %  \choice \SI{51.7}{\kilo\metre\per\hour} [N]
  %  \choice \SI{16.7}{\metre\per\second} [N]
  %  \choice \SI{8.34}{\metre\per\second} [N]
  %  \choice 0
  %\end{choices}

  %\question A boy throws a ball straight up off a second floor balcony and it
  %then lands on the ground. Neglecting air resistance, the magnitude of
  %velocity is greatest:
  %\begin{choices}
  %  \choice just after it leaves the boy's hand
  %  \choice at the peak of the ball's trajectory
  %  \choice just before it hits the ground
  %  \choice it remains the same throughout the motion
  %  \choice impossible to tell without knowing the angle of projection
  %\end{choices}

  \question The distance $d$ travelled by an accelerating object is described
  by the equation
  \begin{equation*}
    d=\dfrac12a\Delta t^2
    \end{equation*}
  where $a$ is the acceleration, and $\Delta t$ is the time interval for the
  acceleration. How would the distance change if $\Delta t$ is doubled?
  \begin{choices}
    \choice Increase by a factor of 2
    \choice Increase by a factor of 4
    \choice Remains the same
    \choice Decrease by a factor of $1/2$
    \choice Decrease by a factor of $1/4$
  \end{choices}
  
  \question The gravitational force between two masses $m_1$, $m_2$ at a
  distance $r$ apart is described by the law of universal gravitation:
  \begin{displaymath}
    F=\frac{Gm_1m_2}{r^2}
  \end{displaymath}
  What is the force $F'$ if $r$ is tripled and $m_1$ is decreased by half?
  \begin{choices}
    \choice $F'=F/6$
    \choice $F'=F/18$
    \choice $F'=9F/2$
    \choice $F'=3F/2$
    \choice $F'=6F$
  \end{choices}
  
  \question For any motion, the average speed is always
  \underline{\hspace{.8in}} the magnitude of average velocity.
  \begin{choices}
    \choice greater than
    \choice greater than or equal to
    \choice equal to
    \choice less than or equal to
    \choice less than
  \end{choices}
\end{questions}
\end{document}
