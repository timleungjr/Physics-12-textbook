\section*{Problems}

\begin{enumerate}[itemsep=6pt]
\item As a glowing black body gets \emph{cooler}, what happens to its
  colour and what happens to the brightness of the light it emits?
%  \begin{choices}
%    \choice colour gets more blue; doesn't change brightness.
%    \choice colour gets more blue; emits less light.
%    \choice colour gets more blue; emits more light.
%    \choice colour gets more red; emits more light.
%    \choice colour gets more red; emits less light.
%  \end{choices}
  
\item A uniform ultraviolet light source shines on two metal plates, causing
  electrons to be emitted from both plates. The two plates are made of
  different materials but have the same surface area. Plate A emits more
  electrons than plate B. However, the electrons emitted from plate B have a
  higher kinetic energy. Which of the following describe plausible explanations
  for the differences in electron emissions?
%  \begin{choices}
%    \choice Plate B has a larger work function than plate A.
%    \choice Higher energy electrons from plate B would be produced by placing
%    the plate closer to the light source, where it would receive more
%    ultraviolet photons from the source.
%    \choice More electrons would be produced from plate A by placing the plate
%    closer to the light source, where it would receive more ultraviolet photons
%    from the source.
%    \choice Plate A emits more electrons of lesser energy, while plate B emit
%    fewer electrons of higher energy, but the total combined energy of the
%    emitted electrons is the same.
%  \end{choices}
  
\item Light shines on a metallic surface causing electrons to be ejected.
  Increasing its \emph{intensity}
%  \begin{choices}
%    \item causes ejected electrons to have more kinetic energy
%    \item causes ejected electrons to have lower velocity
%    \item causes more electrons to be ejected
%    \item causes no change to the ejected electrons or to the number of
%    ejected electrons
%    \item none of the above
%  \end{choices}
  
\item Light shines on a metallic surface causing electrons to be ejected.
  Increasing its \emph{frequency}
%  \begin{choices}
%    \item causes ejected electrons to have more kinetic energy
%    \item causes ejected electrons to have lower velocity
%    \item causes more electrons to be ejected
%    \item causes no change to the ejected electrons or to the number of
%    ejected electrons
%    \item none of the above
%  \end{choices}
  
\item The greater the work function for a metal,
%  \begin{choices}
%    \item the greater the speed of the ejected electron
%    \item the slower an ejected electron for a given incident light
%    \item the more electrons are ejected per unit time
%    \item the lower the threshold frequency
%    \item none of the above
%  \end{choices}
  
\item Blue light has a wavelength which is half that of red light. Therefore,
  photons of blue light each carry \underline{\hspace{.8in}} energy as is
  carried by photons of red light.
%  \begin{choices}
%    \item half as much 
%    \item four times as much
%    \item one fourth as much
%    \item twice as much
%    \item the same
%  \end{choices}

\item Which of the following statements about photons is \emph{false}?
%  \begin{choices}
%    \item Higher energy photons have a higher frequency.
%    \item In a vacuum, photons always travel at the speed of light.
%    \item Photons behave like particles.
%    \item Low energy photons move more slowly than high-energy photons.
%    \item A gamma-ray photon is more energetic than a visible light photon.
%  \end{choices}
%  
%%  \item The energy of light with a frequency of \SI{6.5e14}{\hertz} is
%%  \begin{choices}
%%    \item\SI{4.3e-19}\joule
%%    \item\SI{1.0e-48}\joule
%%    \item\SI{8.6e-19}\joule
%%    \item\SI{2.15e-19}\joule
%%    \item\SI{2.0e-34}\joule
%%  \end{choices}
%
%%  \item A photon of energy \SI{2.00}{\electronvolt} has a wavelength of
%%  \begin{choices}
%%    \item\SI{621}{\nano\metre}
%%    \item\SI{545}{\nano\metre}
%%    \item\SI{400}{\nano\metre}
%%    \item\SI{726}{\nano\metre}
%%    \item\SI{647}{\nano\metre}
%%  \end{choices}
%
%%  \item The momentum of a photon with a wavelength of \SI{725}{\nano\metre}
%%  is
%%  \begin{choices}
%%    \item\SI{9.14e-28}{\kilo\gram.\metre\per\second}
%%    \item\SI{9.14e-25}{\kilo\gram.\metre\per\second}
%%    \item\SI{9.14e-31}{\kilo\gram.\metre\per\second}
%%    \item\SI{9.14e-21}{\kilo\gram.\metre\per\second}
%%    \item\SI{9.14e-45}{\kilo\gram.\metre\per\second}
%%  \end{choices}
%
%%  \item A \SI{65.}{\kilo\gram} person moving at
%%  \SI{12.5}{\metre\per\second} has a wavelength of
%%  \begin{choices}
%%    \item\SI{1.5e-34}\metre
%%    \item\SI{2.9e-33}\metre
%%    \item\SI{7.4e-35}\metre
%%    \item\SI{1.4e-34}\metre
%%    \item\SI{2.9e-34}\metre
%%  \end{choices}
%
%%  \item A 65.0 kg person moving at 12.5 m/s has a wavelength of:
%%  \begin{enumerate}[itemsep=3pt]
%%    \item What is her total energy?
%%    \item What is her wavelength?
%%  \end{parts}
%  
%  \item The momentum of a radio-wave photon with a wavelength of 1.55 m is:
%  \vspace{\stretch1}
%  
%  \item What would be the frequency of a photon with a momentum of
%  \SI{2.45e-32}{\kilo\gram\metre\per\second}?
%  \vspace{\stretch1}
%  
%  \item An electron that has a de Broglie wavelength of
%  \SI{3.32e-10}{\metre} is travelling at a speed of:

%%  \item The wavelength of a proton that is moving at \SI{3.79e6}{m/s} is:
  
\item A microwave oven uses electromagnetic radiation at \SI{2.40}{\giga\hertz}.
  \begin{enumerate}[itemsep=3pt]
  \item What is the energy, in joules, of each microwave photon?
  \item At what rate does a \SI{900}{\watt} oven produce photons?
  \end{enumerate}

%  %\item How slowly must an electron be moving for its de Broglie wavelength
%  %to be equal to \SI1{\milli\metre}?
    
\item The maximum electron energy in a photoelectric experiment is
  \SI{2.8}\electronvolt. When the wavelength of the illuminating radiation is
  increased by \SI{50}\percent, the maximum electron energy drops to
  \SI{1.1}\electronvolt. Find
  \begin{enumerate}[itemsep=3pt]
  \item The work function of the emitting surface, and
  \item The wavelength of the \emph{original} radiation.
  \end{enumerate}

\item Is it possible to measure an electron's velocity to an accuracy of
  $\pm$\SI1{\metre\per\second} while simultaneously finding its position to
  an accuracy of $\pm$\SI1{\micro\metre}? Explain. What about a proton?
  
\item An electron is trapped in an infinitely deep ``quantum well''
  \SI{20}{\nano\metre} wide. What is the minimum speed that it could have?
  
%  \item A positron is an anti-matter particle with the same mass as the
%  electron but the opposite electric charge. When an electron and positron
%  meet, they annihilate and produce a pair of identical gamma ray photons. Find
%  the energy and wavelength of the gamma ray. %(Hint: Use the mass-energy
%  %equivalence from last class. Assume negligible kinetic energy $K$.)
%  %The resulting gamma ray is used in PET (positron emission tomography) scan to
%  %image processes inside the body.
\end{enumerate}
